\section{Degrowth and end to growth in the literature}
\begin{itemize}
	\item \cite{Dasgupta2021}\ar impossibility to grow indefinitely \ar need to reduce to not surpass safe operating space
	\item \cite{Schor2005SustainableReduction}
	\item \cite{VanVuuren2018AlternativeTechnologies}: limit to carbon capture and storage technologies; if output growth requires fossil energy, than infinite growth would need infinite storage; to reduce dependence on this technology beneficial to reduce demand 
	\item \cite{Bertram2018TargetedScenarios}: reduction in demand to simultaneously meet emission targets and sustainability goals (global acceptability)\ar reduction of energy demand alleviates competition between reaching emission limits and sustainability goals (zero hunger, affordability of energy); to lower sustainability risk;
	\\   change demand as a parameter in  model; motivation: taking global inequality into account alternative measures (mitigation policies) to carbon taxes  become optimal. These include lifestyle changes \textbf{in addition to sector-specific carbon taxes!} (25\% lower energy demand and -20\% lower demand for agricultural products )
\end{itemize}

In his review, \cite{Dasgupta2021} attempts to construct an economics of biodiversity. By taking nature's maintenance services, that is, services without which human activity and live would not be possible, as a constituent of total factor productivity, economic growth is no longer disconnected from planetary boundaries (p.137).\footnote{\ The term \textit{planetary boundaries} has been coined by \cite{Rockstrom2009AHumanity} who use it to refer to a state of nature in which humans can safely exist.} 
Papers on environmental economics acknowledge that natural conditions are important to production by assuming tipping points \citep{Acemoglu2012TheChange} or rising temperatures affecting output as in \textit{cite Nordhaus1994, Stern 2006} \cite{Barrage2019OptimalPolicy}.
In addition, \cite{Dasgupta2021} accounts for the waste resulting from production and consumption which again requires nature for 

\cite{Dasgupta2021} argues that infinite GDP growth is impossible given planetary boundaries (p. 47), i.e., the safe space for humans to exist, and that waste which degrades the environment is always positively related to output. 

My project relates by looking at one particular boundary: the one on carbon emissions through a limited carbon budget postulated in the natural sciences \citep{IPCC2022, Rockstrom2009AHumanity}. The model is designed to allow for a stop to technological and consumption growth. 

 