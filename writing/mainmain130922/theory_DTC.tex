\section{Theoretic Results}
\subsection{An emission target calls for a reduction policy under likely parameter values}

\paragraph{Effect of tax progressivity on energy output ratios}


\subsection{Tax progressivity affects the composition of total output}
In the model, tax progressivity affects the innovation decision due to heterogeneous effects on skill supply. 
The optimal ratio of skills supplied by the household is
\begin{align}
\frac{h_{ht}}{h_{lt}}=\left(\frac{w_{ht}}{w_{lt}}\right)^\frac{1-\tau_{lt}}{\tau_{lt}+\sigma}.
\end{align}
The semi-elasticity of the ratio of aggregate skill supply, defined as $\frac{H_h}{H_l}:=\frac{z_hh_h}{z_lh_l}$, in response to a change in income tax progressivity is then given by
\begin{align}
\frac{d\log\left(\frac{H_h}{H_l}\right)}{d\tau_l}=-\frac{1+\sigma}{(\tau_l+\sigma)^2}\log\left(\frac{w_{h}}{w_l}\right). \end{align}
The direct effect, with fixed prices is negative given a positive wage premium for high skill labour. Hence, a higher tax progressivity implies a decline in the relative supply of high skill labour. 

\paragraph{Effect on the externality }
\begin{prop}Assume that (1) the income share of high-skilled labour exceeds that of low-skilled labour, $\frac{H_lw_l}{H_hw_h}<1$, (2) high-skill labour earns a premium, $\frac{w_h}{w_l}>1$, and (3) energy inputs are substitutes, $\varepsilon_e>1$, then
a rise in income tax progressivity increases the share of fossil energy in the economy when the sum of labour shares in fossil and the green sector is below unity, $\theta_f+\theta_{g}<1$, (intuitively, high skill has a lower income share than low skill). When the sum of income shares of high-skill labour is sufificiently high, and necessarily above unity, a rise in tax progressivity lowers the share of fossil to green energy, $\frac{dlog\left(\frac{F_t}{G_t}\right)}{d\tau_l}<0$.
\end{prop}
\footnote{\textit{Proof}\\ 
	The equation follows from iteratively applying skill demand by labour input producers and the skill market clearing conditions and substituting low-skill supply by the households optimality condition. This gives the following relation of high-skill hours employed in the green sector to total high-skill supply:  
	\begin{align*}
	\frac{h_{hg}}{H_h}=\frac{1-\left(\frac{w_l}{w_h}\right)^\frac{1+\sigma}{\sigma+\tau_{lt}}\frac{z_l}{z_h}\frac{\theta_f}{1-\theta_f}}{1-\frac{\theta_f}{\theta_g}\frac{1-\theta_g}{1-\theta_f}}.
	\end{align*}}
For an intuition consider the output ratio as a function of taxes and the wage ratio in equilibrium:
\begin{align*}
\frac{F_t}{G_t}=\left(\frac{(1-\tau_f)(1-\alpha_f)/(1-\alpha_g)}{\left(\frac{w_l}{w_h}\right)^{\frac{1+\sigma}{\sigma+\tau_l}}\frac{z_l}{z_h}\frac{\theta_f}{1-\theta_f}-\frac{1-\theta_g}{1-\theta_f}}\right)^\frac{1}{\varepsilon_e-1}
\end{align*}

As tax progressivity increases, high-skill supply reduces relative to low-skill supply. 
When high skill has a relatively smaller share in the fossil sector than low-skill in the green sector, the reduction in high skill supply makes 

This translates to an increase in low-skill labour employed in the fossil sector and high skill in the fossil sector rises proportionately leading to a fall in high skill in the green sector by market clearing. When 



\subsection{Growth in the dirty sector has to stop}
Growth in the dirty sector eventually has to stop given the net-zero emission target. Assuming no possibility to increase capture and storage technologies, this would be the case by 2050. Otherwise, it suffices to assume a limit to C02 capture-storage technology. This implies a condition on taxes to counter growth in the green sector. I summarise that result in the following proposition.

\begin{prop}Assume that energy inputs are substitutes. Then, growth in the green sector has to be offset by a rise in the fossil tax. Alternatively, %when high skill is in sufficiently high demand, $\theta_f+\theta_g>>1$, then a rise in low skill supply counteracts a rise in fossil growth, i.e. a more progressive tax. If high skill is in lower demand,  $\theta_f+\theta_g<1$, then a drop in the low-to-high skill ratio, a more regressive tax, can offset green growth. 
	the ratio of low-to-high skill income has to rise, that is, at a positive wage premium for high skill labour a more progressive tax is required. 
\end{prop}

\begin{corollary}
	As a subsidy to green innovation boosts growth in the green sector, it must be counteracted by a stronger corrective tax or a respective change in income tax progressivity. In other words, a green subsidy contributes to growth pressure in fossil energy. The more so the less substitutable goods are.
\end{corollary}

\begin{corollary}
	A higher progressivity of the income tax contributes to keeping fossil production low. A double dividend of redistribution: in addition to lowering inequality it lowers emissions.
\end{corollary}

\tr{\textbf{Next: find an expression for wage ratio as a function of growth rates \ar can discuss exogenous growth case!}}