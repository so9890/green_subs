\paragraph{Literature}




\hrule



\textbf{Overview relation to the literature}
	The finding has important consequences for, first, the discussion of the optimal usage of carbon tax revenues, since abstaining from redistributing environmental tax revenues results in an inefficient allocation. Second, the optimal environmental policy increases equity either through lump-sum transfers or income taxation. Third, the result complements the \textit{weak} double-dividend hypothesis: there exists a lower bound on distortionary income taxes after which a further reduction decreases efficiency.

\textbf{To read}
\begin{itemize}
	\item Fullerton 1997
\end{itemize}
\paragraph{Optimal environmental policy}
\textbf{Main claim: focus on environmental taxes and recomposition}
\begin{itemize}
	\item with exogenous growth
	\item with endogenous growth
\end{itemize}


\paragraph{The Pigou principle }


\paragraph{Recycling of environmental tax revenues}
\textbf{Main claim: they overlook that when lump-sum transfers are not available, then labor supply is inefficiently high \ar a lower bound on the optimal distortionary labor tax: even if env. tax revenues would suffice to cover exogenous funding constraint, the labor tax should be progressive, and that env. tax exceeds (? as in table \ref{tab:lin_nolst}) scc in optimal policy}
\begin{itemize}
	\item in general: \cite{Fried2018TheGenerations}
	\item double dividend literature
	\begin{itemize}
		\item \cite{LansBovenberg1994EnvironmentalTaxation}: government spending enters utility of the household \ar when public spending rises, the household feels richer \ar reduces labor supply
		\item there are no lump-sum transfers
		\item defined "first best" as: no need to generate public funds through distortionary labor taxes or $\tau_l=0$ no other pre tax distortion
		\item note that dirty and clean consumption are not necessarily perfect substitutes, as this depends on the utility function; they make no assumption; \ar labor does have a negative externality if production is never perfectly clean
		\item " Changes in employment would not affect welfare; in the absence of distortionary labor taxation, the social opportunity costs of additional employment exactly offset the social benefits."\ar this finding follows from implicitly assuming lump-sum transfers to redistribute environmental tax revenues! So that the resource constraint holds without government consumption! 
		\item when $\tau_D=SCC$ then a marginal rise in $\tau_D$ lowers welfare if labor supply decreases due to the reduction in labor tax revenues
		\item I can show that a fall in $\tau_D$ (erosion of the tax base) increases utility as it decreases government spending through the tax base if environmental tax revenues are not fully redistributed lump-sum 
		\item question: they discuss effect of a change in dirt tax but in a model derived
	\end{itemize} \citep{Barrage2019OptimalPolicy}
\end{itemize}
These findings have important consequences for the literature on the so-called double dividend of environmental policy. 
\cite{LansBovenberg1994EnvironmentalTaxation} have shown that environmental tax revenues do not constitute a double dividend in that its revenues could be used to lower distortions of fund-raising taxes. 
They argue, instead, that environmental tax revenues exert efficiency costs, too, by lowering the gains from work. When the substitution effect dominates, labor supply reduces and the tax base for the income tax diminishes. This mechanism makes it more difficult for the government to raise funds. They show further that environmental taxes exacerbate instead of alleviate pre-existing distortions: in comparison to labor income taxes, environmental taxes distort commodity demand in addition to labor supply. This makes them less costly to generate government revenues than income taxes.  However, \cite{LansBovenberg1994EnvironmentalTaxation} discuss that it is optimal to recycle environmental tax revenues to lower distortionary labor income taxes as opposed to higher lump-sum transfers. The latter in contrast to the former usage decreases labor supply even further through an income effect. This is the so-called \textit{weak} double dividend of environmental taxes. \tr{Check this understanding in \cite{Jacobs2019RedistributionCurves}.} 


On another note, the same literature focuses on the gap between the social cost of pollution and the environmental tax, when an exogenous funding requirement exists but no lump-sum transfers are available \cite{LansBovenberg1994EnvironmentalTaxation, LansBovenberg1996OptimalAnalyses, Barrage2019OptimalPolicy}. 
Due to the labor-reducing effect of environmental taxes, the optimal environmental tax   falls short to capture the social cost of the externality. 
The government forgoes an efficient reduction of the externality in order to prevent a erosion of the tax base of fiscal policies. 
In a similar setting without transferring environmental tax revenues to households, yet absent a requirement on government funds, I find that (i) the optimal environmental tax exceeds the social cost of pollution and that (ii) the optimal labor tax is progressive. 
%This result has gained a lot of attention in the literature \citep{LansBovenberg1996OptimalAnalyses, Barrage2019OptimalPolicy} \tr{What is the difference between Bov/Goulder 96 and BOv/deMooji 94?}

 While this literature argues for the recycling of environmental tax revenues to lower pre-existing tax distortions, my paper constitutes an argument for a lower bound on distortionary income taxes: some reduction of labor supply is in fact efficient. 
Furthermore, when revenues are not redistributed lump-sum, the Pigouvian tax does not implement the efficient allocation. 

\tr{\cite{LansBovenberg1994EnvironmentalTaxation}: they compare two cases 1) no gov spending requirement, and lump-sum transfers; 2) gov spending requirement/ no lump sum transfers; What is missing and interesting: no gov spend requirements and no lump-sum transfers: BUT they only verbally comment on the weak double dividend}
\paragraph{Comment: Understanding \cite{LansBovenberg1994EnvironmentalTaxation}}

The analysis in \cite{LansBovenberg1994EnvironmentalTaxation} rests on the premise that government revenues cannot be raised through lump-sum transfers. However, when there is no prerequisite to raise government funds, then environmental tax revenues are redistributed through lump-sum transfers. The second claim becomes obvious from the resource constraint, equation (1) in their paper. 

As a result, the costs of labor effort and the benefits cancel when the environmental tax is set to the social cost of pollution. However, when no lump-sum transfers are available for redistribution, then the social gains and the social costs of labor supply are not equal when the environmental tax is set to the social costs of pollution. Then, labor supply is inefficiently high: the costs of labor are not compensated for by enough through more consumption. In what way the optimal environmental tax deviates from the social cost of the externality depends on (i) how government spending affects household utility and (ii) if dirt taxes exacerbate the inefficiency of labor supply. 

To see this take the total differential of the resource constraint in \cite{LansBovenberg1994EnvironmentalTaxation} equation (1) under the assumption that all environmental tax revenues are consumed by the government: $G=\tau_D D$.

\begin{align}
hNdL = dC+dD+\tau_D dD +D d\tau_d.
\end{align}
 Then the total effect of a revenue neutral 


\paragraph{Environmental protection and inequality}
\ar 1) Inequality and environment as competing goods.

In the literature discussing environmental policies in an unequal framework, a competition between equity and environmental good provision have been discussed. 
This trade-off can be separated into (i) the competition for public funds \citep{LansBovenberg1996OptimalAnalyses, Jacobs2019RedistributionCurves} and (ii) effects of either environmental policies on equity or equalizing policies on environmental quality \citep{Jacobs2019RedistributionCurves, Sager2019IncomeCurves, Dobkowitz2022}. 

Since hours are inefficiently high, equalizing policies become part of the optimal environmental policy. As a byproduct, the distribution of income becomes more equal.


\ar 2) Inequality to shape effects of environmental policies and effect of fiscal policy on environment due to heterogeneity

 Furthermore, the differentiation of skills and the skill-bias of the green sector in the paper give rise to a new channel through which labor taxation affects environmental protection. The literature has primarily focused on a demand channel arising from non-linear Engel curves through which inequality and redistribution shape the degree of dirty production in the economy.   

%\textbf{Non-linear Engel Curves: redistribution}
\cite{Jacobs2019RedistributionCurves} the motive to redistribute and to provide an environmental good compete for government resources due to a negative effect of environmental taxes on the wage rate. Even with lump-sum transfers, the optimal environmental tax does not follow the Pigou principle when the government seeks to enhance equity.

\cite{Sager2019IncomeCurves} argues empirically, that redistribution to poorer households may result in a higher demand for polluting goods. 
\paragraph{Pubic Finance literature}
\textbf{I add: a new perspective on labor income taxes as a tool to lower inefficiently high hours worked. }

 \cite{Heathcote2017OptimalFramework}, \cite{Loebbing2019NationalChange}

\paragraph{Reductive policies in the literature}



\tr{This observation relates to the literature in several ways: first, the literature which discusses the optimal recycling of carbon tax revenues. Because when revenues are not recycled as lump-sum transfers, then labor supply is inefficiently high and additional policy measures are necessary to implement the efficient allocation today. 
	In other words: because lump-sum transfers are not available, the literature argues, the government should use corrective tax revenues to lower pre-existing tax distortions. But by how much? I argue, that there is an optimal size of positive tax distortions when lump-sum transfers are not available. Hence, under the premise of non-lump sum transfers, distortionary labor income taxes arise as an optimal policy tool even absent an exogenous financing condition or inequality. }

The paper relates broadly to the literature discussing optimal environmental policies. I separate them into two strands: one with inelastic and one with elastic labor supply. 
\paragraph{Optimal environmental policy: exogenous labor supply}

\paragraph{Lit: environmental policy and distortionary fiscal setting}

\begin{itemize}
	\item Williams 2013 Double dividend 
	\item talk to Mireille Chiroleu Assouline: paper on double dividend
	\item Mireille with Aubert or Fodha (PSE)
\end{itemize}
%Inequality-environment nexus: normally motivated by a demand-side perspective; in this project I focus on a supply side explanation
labor supply becomes elastically in the literature studying the interaction of environmental taxes and distortionary taxes.  This strand of the literature generally focuses on the gap between the social cost of carbon and the optimal environmental tax arising from pre-existing distortionary labor income taxes or an exogenous requirement on government funds \citep{Bovenberg1997EnvironmentalGrowth,  Kaplow2012OPTIMALTAXATION, Jacobs2019RedistributionCurves, Barrage2019OptimalPolicy}. labor income taxes form a passive component of these analyses. 
The general findings of this literature is that the optimal environmental tax falls below the social cost of carbon to mitigate efficiency costs and enable the government to raise revenues. 
Furthermore, the literature argues for a recycling of environmental tax revenues to be used to lower income taxes. A recycling through transfers would intensify reductions in labor supply. These arguments rely on the premise that no lump-sum transfers are available. I add to this literature the perspective that a reduction in labor supply is part of the efficient policy. If lump-sum transfers are not available - as is to be assumed in this literature to motivate the existence of distortionary taxes - then labor income taxes should be positive to cope with distortions in the labor supply. Hence, there is a lower bound up to which environmental tax revenues are optimally used to lower distortionary taxes. This is not recognised by the literature. \tr{How do \cite{LansBovenberg1994EnvironmentalTaxation} argue for the use of env. tax revenues to lower distortionary taxes? Verbally or analytically?\ar Verbally! } \ar look at \cite{LansBovenberg1996OptimalAnalyses} do they focus on weak dividend,. No they focus on the deviation of env tax from pigou?



The inefficiency of environmental taxes arises absent pre-existing income tax distortions or the motive to redistribute.
In difference to this literature, where the presence of a distortionary income tax shapes the optimal level of the environmental tax, the existence of the environmental tax rationalises a progressive income tax in the present paper.
Importantly, the equity and the environmental targets of government intervention are perceived as competing goals as both tax instruments exert efficiency costs through a reduction in labor supply. 
I argue in this paper that what has commonly been perceived as an efficiency cost -  the reduction in labor supply in response to environmental and income taxation - is part of the optimal environmental policy. Hence, income taxation has a double dividend: an environmental and an equity one.  

In this literature, there are either no transfers to households at all \citep{Bovenberg2002EnvironmentalRegulation, LansBovenberg1994EnvironmentalTaxation} or an exogenously given requirement for transfers \citep{Barrage2019OptimalPolicy}. Hence, there is no lump-sum transfer instrument.

\citep{Fullerton1997EnvironmentalComment} writes in its introduction 
\begin{quote}
	With no revenue requirement, or where government can use lump-sum taxes, Arthur C. Pigou (1947) shows that the first-best tax on pollution is equal to the marginal environmental damage.
\end{quote}
\ar What is the optimal environmental tax when labor supply is elastic and there are no lump-sum funds?
\paragraph{Environment and elastic labor supply}

\paragraph{Recycling of environmental tax revenues}
\cite{Fried2018TheGenerations}
\paragraph{Environment and (endogenous) growth}
\begin{itemize}
	\item limits to growth
	\item general literature on end growth and the environment
\end{itemize}
