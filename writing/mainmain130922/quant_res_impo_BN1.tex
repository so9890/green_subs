\section{Quantitative Results}

In this section, I, first, discuss the quantitative results.
Subsection AA presents the optimal allocation and policy given the emission target. Subsection BB discusses the results. In particular, I focus on understanding the role and importance of tax progressivity. 

\subsection{Main results}
\begin{figure}[h!!]
	\centering
	\caption{Optimal Policy }\label{fig:optPol}
	\begin{minipage}[]{0.4\textwidth}
		\centering{\footnotesize{(a) Income tax progressivity, $\tau_{lt}$}}
		%	\captionsetup{width=.45\linewidth}
		\includegraphics[width=1\textwidth]{../../codding_model/own_basedOnFried/optimalPol_elastS_DisuSci/figures/all_1705/Single_OPT_T_NoTaus_taul_spillover0_sep1_BN0_ineq0_etaa0.79.png}
	\end{minipage}
\begin{minipage}[]{0.1\textwidth}
\
\end{minipage}
	\begin{minipage}[]{0.4\textwidth}
		\centering{\footnotesize{(b) Fossil tax, $\tau_{ft}$ }}
		%	\captionsetup{width=.45\linewidth}
		\includegraphics[width=1\textwidth]{../../codding_model/own_basedOnFried/optimalPol_elastS_DisuSci/figures/all_1705/Single_OPT_T_NoTaus_tauf_spillover0_sep1_BN0_ineq0_etaa0.79.png}
	\end{minipage}
\end{figure} 
%\begin{figure}[h!!]
%	\centering
%	\caption{Optimal Policy }\label{fig:optPol}
%	\begin{minipage}[]{0.4\textwidth}
%		\centering{\footnotesize{(a) Income tax progressivity, $\tau_{lt}$}}
%		%	\captionsetup{width=.45\linewidth}
%		\includegraphics[width=1\textwidth]{../../codding_model/own_basedOnFried/optimalPol_elastS_DisuSci/figures/all_1705/Single_OPT_T_NoTaus_taul_spillover0_sep1_BN1_ineq0_etaa0.79.png}
%	\end{minipage}
%	\begin{minipage}[]{0.1\textwidth}
%		\
%	\end{minipage}
%	\begin{minipage}[]{0.4\textwidth}
%		\centering{\footnotesize{(b) Fossil tax, $\tau_{ft}$ }}
%		%	\captionsetup{width=.45\linewidth}
%		\includegraphics[width=1\textwidth]{../../codding_model/own_basedOnFried/optimalPol_elastS_DisuSci/figures/all_1705/Single_OPT_T_NoTaus_tauf_spillover0_sep1_BN1_ineq0_etaa0.79.png}
%	\end{minipage}
%\end{figure}
To optimally meet the IPCCs suggested emission target, the optimal income tax is progressive. As the emission target is less strict, between 2030 to 2045, optimal income tax progressivity is around $\tau_{lt}=0.02$. As the emission target jumps to net-zero emissions in 2050, optimal tax progressivity accelerates to above 0.08 and gradually increases in the subsequent years to around 0.09. This is approximately half the size found for the US in \cite{Heathcote2017OptimalFramework}: $\tau_{l}=0.181$. 
In the period without emission target from 2020 to 2030, the optimal income tax is regressive.

The optimal fossil tax displays a similar step pattern as income tax progressivity. From 2020 to the beginning of 2030, it is negative. It jumps to around 50\% as the emission target is to reduce emissions by 50\% relative to 2019 emissions. As the emission target rises  to net-zero emissions in 2050, the optimal tax on fossil sales is close to 95\%. 

\begin{figure}[h!!]
	\centering
	\caption{Optimal Policy }\label{fig:optAll}
	\begin{minipage}[]{0.32\textwidth}
		\centering{\footnotesize{(a) Growth fossil sector}}
		%	\captionsetup{width=.45\linewidth}
		\includegraphics[width=1\textwidth]{../../codding_model/own_basedOnFried/optimalPol_elastS_DisuSci/figures/all_1705/Single_OPT_T_NoTaus_Af_spillover0_sep1_BN0_ineq0_etaa0.79.png}
	\end{minipage}
	\begin{minipage}[]{0.32\textwidth}
		\centering{\footnotesize{(b) Growth green sector }}
		%	\captionsetup{width=.45\linewidth}
		\includegraphics[width=1\textwidth]{../../codding_model/own_basedOnFried/optimalPol_elastS_DisuSci/figures/all_1705/Single_OPT_T_NoTaus_Ag_spillover0_sep1_BN0_ineq0_etaa0.79.png}
	\end{minipage}
\begin{minipage}[]{0.32\textwidth}
	\centering{\footnotesize{(c) Growth neutral sector}}
	%	\captionsetup{width=.45\linewidth}
	\includegraphics[width=1\textwidth]{../../codding_model/own_basedOnFried/optimalPol_elastS_DisuSci/figures/all_1705/Single_OPT_T_NoTaus_An_spillover0_sep1_BN0_ineq0_etaa0.79.png}
\end{minipage}
	\begin{minipage}[]{0.32\textwidth}
	\centering{\footnotesize{(d) Labour fossil sector}}
	%	\captionsetup{width=.45\linewidth}
	\includegraphics[width=1\textwidth]{../../codding_model/own_basedOnFried/optimalPol_elastS_DisuSci/figures/all_1705/Single_OPT_T_NoTaus_Lf_spillover0_sep1_BN0_ineq0_etaa0.79.png}
\end{minipage}
\begin{minipage}[]{0.32\textwidth}
	\centering{\footnotesize{(e) Low-skilled labour }}
	%	\captionsetup{width=.45\linewidth}
	\includegraphics[width=1\textwidth]{../../codding_model/own_basedOnFried/optimalPol_elastS_DisuSci/figures/all_1705/Single_OPT_T_NoTaus_hl_spillover0_sep1_BN0_ineq0_etaa0.79.png}
\end{minipage}
\begin{minipage}[]{0.32\textwidth}
	\centering{\footnotesize{(f) High-skilled labour}}
	%	\captionsetup{width=.45\linewidth}
	\includegraphics[width=1\textwidth]{../../codding_model/own_basedOnFried/optimalPol_elastS_DisuSci/figures/all_1705/Single_OPT_T_NoTaus_hh_spillover0_sep1_BN0_ineq0_etaa0.79.png}
\end{minipage}
	\begin{minipage}[]{0.32\textwidth}
	\centering{\footnotesize{(d) Consumption}}
	%	\captionsetup{width=.45\linewidth}
	\includegraphics[width=1\textwidth]{../../codding_model/own_basedOnFried/optimalPol_elastS_DisuSci/figures/all_1705/Single_OPT_T_NoTaus_C_spillover0_sep1_BN0_ineq0_etaa0.79.png}
\end{minipage}
\begin{minipage}[]{0.32\textwidth}
	\centering{\footnotesize{(e) Social Welfare}}
	%	\captionsetup{width=.45\linewidth}
	\includegraphics[width=1\textwidth]{../../codding_model/own_basedOnFried/optimalPol_elastS_DisuSci/figures/all_1705/Single_OPT_T_NoTaus_SWF_spillover0_sep1_BN0_ineq0_etaa0.79.png}
\end{minipage}
\begin{minipage}[]{0.32\textwidth}
	\centering{\footnotesize{(f) Emissions}}
	%	\captionsetup{width=.45\linewidth}
	\includegraphics[width=1\textwidth]{../../codding_model/own_basedOnFried/optimalPol_elastS_DisuSci/figures/all_1705/Single_OPT_T_NoTaus_Emnet_spillover0_sep1_BN0_ineq0_etaa0.79.png}
\end{minipage}
\end{figure} 

\subsection{Discussion}
To study the role of income tax progressivity, I compare the optimal policy and allocation in the full model to a  model where no labour income tax is available.

The first thing to note is that a fossil tax suffices to meet the emission target, panel (a) in figure \ref{fig:Compno_taul}. The fossil tax is slightly higher in periods with emission target, compare panel (b).
The advantage from relying on labour income taxes to meet the emission target stems from a higher utility from leisure especially from high-skilled workers which outweighs lower consumption levels, compare panels (c) to (e). In fact, the rise in social welfare arises from the periods with net-zero emission target as shown by panel (f) which compares social welfare levels across policy regimes. 

Income tax progressivity increases social welfare in the net-zero emission world, as households work inefficiently high hours absent an income tax. Higher hours worked result in too high labour effort and research in the green and non-energy sector. The allocation in the fossil sector remains unchanged to meet the emission target; compare panels (g) to (j). 

Although consumption rises due to the higher work effort when there is no income tax, the gains from labour effort are diminished due to the cap on fossil energy. Since green and fossil energy are no perfect substitutes, the economy cannot profit as much from the rise in green energy. HYPOTHESIS: WITH ENERGY SOURCES BEING BETTER COMPLEMENTS, WORK EFFORTS WOULD BE MORE FRUITEFUL. The muted effect of green energy on total energy output is intensified when considering total output where input goods are complements. 

\begin{figure}[h!!]
	\centering
	\caption{Optimal Policy }\label{fig:Compno_taul}
			\begin{minipage}[]{0.32\textwidth}
		\centering{\footnotesize{(a) Emissions}}
		%	\captionsetup{width=.45\linewidth}
		\includegraphics[width=1\textwidth]{../../codding_model/own_basedOnFried/optimalPol_elastS_DisuSci/figures/all_1705/comp_notaul_OPT_T_NoTaus_Emnet_spillover0_sep1_BN0_ineq0_etaa0.79_lgd1.png}
	\end{minipage}
		\begin{minipage}[]{0.32\textwidth}
		\centering{\footnotesize{(b) Fossil tax}}
		%	\captionsetup{width=.45\linewidth}
		\includegraphics[width=1\textwidth]{../../codding_model/own_basedOnFried/optimalPol_elastS_DisuSci/figures/all_1705/comp_notaul_OPT_T_NoTaus_tauf_spillover0_sep1_BN0_ineq0_etaa0.79.png}
	\end{minipage}
	\begin{minipage}[]{0.32\textwidth}
		\centering{\footnotesize{(c) Consumption}}
		%	\captionsetup{width=.45\linewidth}
		\includegraphics[width=1\textwidth]{../../codding_model/own_basedOnFried/optimalPol_elastS_DisuSci/figures/all_1705/comp_notaul_OPT_T_NoTaus_C_spillover0_sep1_BN0_ineq0_etaa0.79.png}
	\end{minipage}
	\begin{minipage}[]{0.32\textwidth}
		\centering{\footnotesize{\ \\(d) High skill }}
		%	\captionsetup{width=.45\linewidth}
		\includegraphics[width=1\textwidth]{../../codding_model/own_basedOnFried/optimalPol_elastS_DisuSci/figures/all_1705/comp_notaul_OPT_T_NoTaus_hh_spillover0_sep1_BN0_ineq0_etaa0.79.png}
	\end{minipage}
	\begin{minipage}[]{0.32\textwidth}
		\centering{\footnotesize{\ \\(e) Low skill}}
		%	\captionsetup{width=.45\linewidth}
		\includegraphics[width=1\textwidth]{../../codding_model/own_basedOnFried/optimalPol_elastS_DisuSci/figures/all_1705/comp_notaul_OPT_T_NoTaus_hl_spillover0_sep1_BN0_ineq0_etaa0.79.png}
	\end{minipage}
	\begin{minipage}[]{0.32\textwidth}
	\centering{\footnotesize{\ \\(f) Social welfare}}
	%	\captionsetup{width=.45\linewidth}
	\includegraphics[width=1\textwidth]{../../codding_model/own_basedOnFried/optimalPol_elastS_DisuSci/figures/all_1705/comp_notaul_OPT_T_NoTaus_SWF_spillover0_sep1_BN0_ineq0_etaa0.79.png}
\end{minipage}
	\begin{minipage}[]{0.32\textwidth}
		\centering{\footnotesize{\ \\(g) Labour fossil}}
		%	\captionsetup{width=.45\linewidth}
		\includegraphics[width=1\textwidth]{../../codding_model/own_basedOnFried/optimalPol_elastS_DisuSci/figures/all_1705/comp_notaul_OPT_T_NoTaus_Lf_spillover0_sep1_BN0_ineq0_etaa0.79.png}
	\end{minipage}
	\begin{minipage}[]{0.32\textwidth}
		\centering{\footnotesize{\ \\(h) Labour green}}
		%	\captionsetup{width=.45\linewidth}
		\includegraphics[width=1\textwidth]{../../codding_model/own_basedOnFried/optimalPol_elastS_DisuSci/figures/all_1705/comp_notaul_OPT_T_NoTaus_Lg_spillover0_sep1_BN0_ineq0_etaa0.79.png}
	\end{minipage}
\begin{minipage}[]{0.32\textwidth}
	\centering{\footnotesize{\ \\(i) Labour non-energy}}
	%	\captionsetup{width=.45\linewidth}
	\includegraphics[width=1\textwidth]{../../codding_model/own_basedOnFried/optimalPol_elastS_DisuSci/figures/all_1705/comp_notaul_OPT_T_NoTaus_Ln_spillover0_sep1_BN0_ineq0_etaa0.79.png}
\end{minipage}
\begin{minipage}[]{0.32\textwidth}
	\centering{\footnotesize{\ \\(j) Research fossil}}
	%	\captionsetup{width=.45\linewidth}
	\includegraphics[width=1\textwidth]{../../codding_model/own_basedOnFried/optimalPol_elastS_DisuSci/figures/all_1705/comp_notaul_OPT_T_NoTaus_sff_spillover0_sep1_BN0_ineq0_etaa0.79.png}
\end{minipage}
	\begin{minipage}[]{0.32\textwidth}
		\centering{\footnotesize{\ \\(k) Green research}}
		%	\captionsetup{width=.45\linewidth}
		\includegraphics[width=1\textwidth]{../../codding_model/own_basedOnFried/optimalPol_elastS_DisuSci/figures/all_1705/comp_notaul_OPT_T_NoTaus_sg_spillover0_sep1_BN0_ineq0_etaa0.79.png}
	\end{minipage}
\begin{minipage}[]{0.32\textwidth}
	\centering{\footnotesize{\ \\(l) Non-energy research }}
	%	\captionsetup{width=.45\linewidth}
	\includegraphics[width=1\textwidth]{../../codding_model/own_basedOnFried/optimalPol_elastS_DisuSci/figures/all_1705/comp_notaul_OPT_T_NoTaus_sn_spillover0_sep1_BN0_ineq0_etaa0.79.png}
\end{minipage}

\begin{minipage}[]{0.32\textwidth}
	\centering{\footnotesize{\ \\(m) Fossil output}}
	%	\captionsetup{width=.45\linewidth}
	\includegraphics[width=1\textwidth]{../../codding_model/own_basedOnFried/optimalPol_elastS_DisuSci/figures/all_1705/comp_notaul_OPT_T_NoTaus_F_spillover0_sep1_BN0_ineq0_etaa0.79.png}
\end{minipage}
\begin{minipage}[]{0.32\textwidth}
	\centering{\footnotesize{\ \\(n) Green output}}
	%	\captionsetup{width=.45\linewidth}
	\includegraphics[width=1\textwidth]{../../codding_model/own_basedOnFried/optimalPol_elastS_DisuSci/figures/all_1705/comp_notaul_OPT_T_NoTaus_G_spillover0_sep1_BN0_ineq0_etaa0.79.png}
\end{minipage}
\begin{minipage}[]{0.32\textwidth}
	\centering{\footnotesize{\ \\(o) Non-energy output}}
	%	\captionsetup{width=.45\linewidth}
	\includegraphics[width=1\textwidth]{../../codding_model/own_basedOnFried/optimalPol_elastS_DisuSci/figures/all_1705/comp_notaul_OPT_T_NoTaus_N_spillover0_sep1_BN0_ineq0_etaa0.79.png}
\end{minipage}
\end{figure} 
\begin{figure}

\begin{minipage}[]{0.32\textwidth}
	\centering{\footnotesize{\ \\(p) Energy output}}
	%	\captionsetup{width=.45\linewidth}
	\includegraphics[width=1\textwidth]{../../codding_model/own_basedOnFried/optimalPol_elastS_DisuSci/figures/all_1705/comp_notaul_OPT_T_NoTaus_E_spillover0_sep1_BN0_ineq0_etaa0.79.png}
\end{minipage}
\begin{minipage}[]{0.32\textwidth}
	\centering{\footnotesize{\ \\(q) Final output}}
	%	\captionsetup{width=.45\linewidth}
	\includegraphics[width=1\textwidth]{../../codding_model/own_basedOnFried/optimalPol_elastS_DisuSci/figures/all_1705/comp_notaul_OPT_T_NoTaus_Y_spillover0_sep1_BN0_ineq0_etaa0.79.png}
\end{minipage}
\end{figure}

Another central aspect of the paper is the importance of inequality for the optimal environmental policy. How does household heterogeneity in labour supply shape the optimal environmental policy? First, I hypothesise that the skill bias of the green sector makes a less progressive income tax optimal. 
\subsection{Sensitivity}
In this subsection, I discuss results under counterfactual parameter values to elicit the robustness of the main result: the preference of progressive labour taxation above higher fossil taxes. 
First, the productivity gap between sectors might be driving the results. Second, how do results change as within sector spillovers of research are positive? Third, I study the results of a more general specification of the utility function where income affects the choice of hours worked \citep{Boppart2019LaborPerspectiveb, Bick2018HowImplications}. 