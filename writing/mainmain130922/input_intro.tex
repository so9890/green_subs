\section{Introduction}

\paragraph{Motivation}% Real life setting}
Due to climate change and vital threats to biodiversity we need to reduce the consumption of resources. % why this?/ Resource consumption= to emissions??? 
Macroeconomic research has largely focused on a green \textit{recomposition} of consumption. However, doubt has been raised if a recomposition alone suffices to fight climate change. An alternative and most likely complementary approach that has been proposed is a \textit{reduction} of economic output \citep[e.g.][]{Dasgupta2021, GoughCANGREEN, Naqvi2017FiftyPollutants}. %consumption levels. 
In light of the uncertainty about the success of a pure recomposition policy and the urgency to act, it is time to ask: what are the effects of a reduction in economic output? 


More specifically, I am interested in two aspects. First, the paper focuses on the political conflicts which may arise from a reduction in economic output. 
For example, what are the effects on distinct sources of income: the wage rate paid in green and non-green sectors, firm profits and dividends?  Due to a heterogeneous distribution of skills across sectors \cite{Bowen2018CharacterisingComposition, Consoli2016DoCapital}, low and high income households might be affected differently by the reduction in output. 
Second, the analysis draws attention to general equilibrium effects which might matter for the total effect an initial reduction in output has on the externality. 
A reduction in demand or labour, for instance, could well affect the incentives for R\&D. To capture the effect I integrate  endogenous, directed technolgical innovation following  \cite{Acemoglu2012TheChange} into the model. Furthermore, in an extension, I allow for income-dependent marginal propensities to consume so that aggregate demand for green products varies with the distribution of income. When income of the poor falls in reaction to a drop in consumption by the rich, for instance, they might revert to consume a higher share of an unsustainable yet cheaper good. This again mitigates the reduction in resource consumption. 

% \paragraph{Hypotheses/ why a macro model?}
%In order to satisfy basic needs, these households may increase their sustainable demand at the expense of sustainable consumption. This channel mitigates the effect of a reduction in demand on the externality. Second, endogenous and directed innovation might be negatively affected by the reduction in demand as it becomes less profitable. Again, this might slow down the recomposition of output. However, this channel might lower returns to investment which again only affects income of the rich. It is hence a priori unclear what effect a reduction in demand has on inequality and the externality calling for an analysis in a general equilibrium model. 

%\paragraph{Qualitative versus quantitative model?}
%A quantitative model would be needed if I was interested in numbers which I would want to compare to the data and take more seriously. 
%I think I am more interested in understanding the conditions (parameter values etc.) under which a reduction is beneficial to the poor or to the rich. And also under what circumstances a reduction in output might be impeded by other mechanisms in lowering resource usage (endogenous innovation, general equilibrium effects through prices (and \textit{subjective} basic needs)). Finally, accompanying policies to support a reduction in economic output should be studied. 


\paragraph{Main Results} xxx

\paragraph{Data}
Two crucial aspects of the mechanisms scrutinised in this paper are consumption and inequality. 
While the consumption of emissions along the income distribution has already been studied \citep[e.g.][]{Sager2019IncomeCurves}, the data used for these exercises do in general not differentiate between the quality of goods, that is, whether they are green or non-green versions of a given product. 
However, this matters for the emissions associated with consumption.  For example, sustainable versus non-sustainable energy sources vary in ... REFER TO STUDY ON EMISSIONS OR WHATEVER MATTERS FOR RESOURCE USAGES
%Most empirical studies on resource consumption only capture differences in the type of products consumed due to data availability. \citep{Sager2019IncomeCurves}

%(1) First, the distribution of labour across the green and non-green sector. How are income and employment related? \textit{What data is available here? Is it sufficient to draw from the existing literature here, e.g. }
Secondly, I investigate whether there is evidence on a voluntary reduction of consumption by households. (Studies which have already looked at such a possibility seem not to look at actual consumption but experiments...need to do a more in-depth literature review here )


\paragraph{Model}
I use a general equilibrium framework in which the level of economic productivity is determined by demand. There are two household types: low and high-skilled labour. Furthermore, high-skill labour also invest into firms (PROVIDE REFERENCE OR DATA). Production separates into two sectors which produce a green or a non-green final consumption good. Sectors differ with respect to the labour input they use for production so that the share of high-skilled worker is higher in the green sector \citep[as found by][for the US]{Consoli2016DoCapital}. The intermediate goods used for final good production is a labour input good and machines. As in \cite{Acemoglu2012TheChange}, machines are sold in a sector-specific monpolistically competitive market, and research determines the productivity of machines. Scientists decide in what sector to conduct research so that the average productivity in a sector is determined endogenously.

\paragraph{Exercise}
In this setting, I study the effects of two distinct policy interventions: first, a reduction in working time, and second, a reduction in demand. The latter could be motivated intrinsically  or a policy intervention. \\
Note: \\
  at the moment, I am analysing in a simple model how these two policies might differ/ or rather what features the model needs to see a difference in these policies; this should also help to explain counteracting mechanisms more clearly\\

%\begin{comment}
%\paragraph{Potential consequences of the paper and how it could lead to learn about the ability of the market to comply with climate goals}
%
%\textit{I might find that, under the assumption that recomposition alone is not sufficient and a reduction in consumption becomes necessary, a reduction in output might impede technological change so much that the market solution is not feasible \ar motivates to think about accompanying policies.}
%
%content...
%\end{comment}
%

\paragraph{Literature}
The paper relates to several strands of literature. First, to papers discussing policies in a climate-externality context. 
Second, macro models with demand-determined output level \citep{Auerbach2021InequalityEconomy}\cite{Michaillat2015AggregateUnemployment}.
Third, political economy. 
Fourth, degrowth. 
Fifth, endogeneous directed innovation. 