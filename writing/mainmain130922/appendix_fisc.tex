\clearpage
\appendix
\section{Natural Scientific Background}
\begin{comment}
\subsection{Why output (growth) reduction might be optimal}
It is a vibrant debate whether technological process will result in a production technology that is perfectly clean in that it does not exert any environmental externality. 
\begin{itemize}
	%\item \underline{Extensions to technology in \cite{Acemoglu2012TheChange} }
	%\begin{itemize}
	\item \underline{externality of ``clean'' sector} \citep[see also][]{Dasgupta2021, Brock2005ChapterEmpirics}
	\begin{itemize}
		\item[-] renewable/ non-fossil fuels \ar externalities in production process are present e.g. production of solar panels uses toxic inputs \citep{Yue2014DomesticAnalysis}; non-fossil fuel nitrogen generation (e.g., biomass burning to clear land) important ($\approx$ 50\%) \citep{Song2021ImportantEmissions}; low but chronical levels of nitrogen cause species extinctions \citep{Clark2008LossGrasslands}
		\item[-] waste (after use) \ar depends on recycling technology %\ar recycling system for solar panels not profitable enough today
		%	\item[-] substitutability of nature in production (input sources eg. fossil vs. non-fossil fuels)
		%\end{itemize}
		%\item Irreversibilities already before thresholds are hit (e.g. species extinction)
		
	\end{itemize}
	%\item greenhouse gases: Carbon dioxide $CO_2$ (vast majority), Nitrous oxide $N_2O$, methane $CH_4$
	%\item stock of nature globally determined
	\item \underline{parallel positive trend in demand} (population growth, rebound effect) that outperforms increase in clean technology growth \small{(no long-run issue if perfectly clean technology exists)}
	\item \normalsize{\underline{objective function}:} \cite{Arrow2004AreMuch}(Journal of Economic Perspectives) \ar using a sustainability measure they provide evidence that consumption is too high (= not leaving enough natural resources for future generations)
	\item \underline{risk, ambiguity}
	\item if have to meet climate target in short run, might need to lower production to do so; or it might be better in terms of inequality?
\end{itemize}

content...
\end{comment}
\subsection{Greenhouse-gas emissions and the Paris Agreement}

Two alternatives exist to specifiy the relation between the environment and production: (i) a broad approach considering natures use as a sink and as a resource, and all relevant pollutants. 
In order to determine \textit{relevant}, I refer to the planetary boundaries discussed in \cite{Rockstrom2009AHumanity}. (ii) a more specific approach that focuses on greenhouse gas emissions in particular which allows to draw on emission goals specified by country. Paris agreement goal: `'\textit{Climate neutral world by the mid-century}'' (source \url{https://unfccc.int/process-and-meetings/the-paris-agreement/the-paris-agreement}). In 2020 countries had to submit plansfor a \textit{long-term low ghg emissions} (LT-LEDS) where long term means mid-century (I assume). In the EU member states have submitted \textit{integrated national-energy and climate plans} (NECPS) (source \url{https://ec.europa.eu/info/energy-climate-change-environment/implementation-eu-countries/energy-and-climate-governance-and-reporting/national-long-term-strategies_en}). According to this source, emissions occur in the following fields: 
\textit{emission reductions and enhancements of removals in individual sectors, including \textbf{electricity, industry, transport, the heating and cooling and buildings sector (residential and tertiary), agriculture, waste and land use, land-use change and forestry (LULUCF)}}; the website also contains documents on country specific plans and actions

\subsection{Modelling choice: external emission target}\label{app:emission_climate_targets}
There is a multitude of uncertainties shaping the relation of production, on the one hand, and nature and climate warming, on the other hand. These uncertainties relate to (i) the technological possibilities to reduce emissions in the future and (ii) the relation of emissions and the climate. 

In the Paris Agreement clear political goals have been formulated in 2015. Under this treaty, states have agreed on a legally binding maximum increase in temperature to well below 2°C, preferably 1.5° over pre-industrial levels, and the global community seeks to be climate-neutral in 2050  (compare:\\ \url{https://unfccc.int/process-and-meetings/the-paris-agreement/the-paris-agreement}). 

\paragraph{Uncertainty 1): Emissions $\rightarrow$ temperature}
Carbon dioxide has been the focus of the literature integrating climate change and economic models \citep[such as,][]{Golosov2014OptimalEquilibrium,Barrage2019OptimalPolicy}. 
The (geo-)physical mechanisms which determine the interrelation between carbon emissions and temperature changes are highly uncertain and complex. For example, (1) there is no good understanding of the relation of CO2 and the climate as the temperature rises to certain limits, (2)  feedback of the Earth system, such as permafrost thawing, has to be taken into account, as well as (3) interactions of carbon with non-CO2 emissions, (\citep[][p.96, 2nd paragraph]{Rogelj2018MitigationDevelopment.}).  Uncertainty also surrounds the regeneration rate of the environment \citep{Acemoglu2012TheChange} and irreversibilities (might CITE Hassler handbook chapter here). In a quantitative study on optimal environmental policies, hence, a lot of assumptions and simplifications have to be made. 

In chapter 2 of the
\textit{IPCC Special Report} \citep{Rogelj2018MitigationDevelopment.}, scientists quantify emission pathways to meet the 1.5°C goal of the Paris Agreement by carefully taking uncertainties and the complex geophysical processes into account. I use these limits on emissions as constraints to the government's objective function. This approach is clearly policy relevant, while at the same time reduces the need to make (geophysical) assumptions. Furthermore, it allows me to take other important non-CO2 emissions into account, too. (\textit{Look at the discussion of integrated assessment models in \cite{Hassler2016EnvironmentalMacroeconomics} for the advantages of integrating a simplified carbon cycle into macro models (\ar dynamics) })
%\tr{\ar In a nutshell, I take from these reports the emission reduction pathways. I do have to make assumptions on the possibilities of technological innovations. No carbon cycle needed but less assumptions have to be made.}

\paragraph{Uncertainty 2: Technological progress $\rightarrow$ emissions}
An important modelling uncertainty remains: what degree of emission reduction can be achieved by technological progress in the specified time frame? I use different specifications of technological possibilities: (i) a scenario where technological progress is sufficient to reduce emissions to zero until 2050 at current consumption levels per capita, (ii) and one where innovation steps are insufficient.
Also look at different modelling approaches to technological change: (i) sector-specific innovations, (ii) porgress on the substitutability of clean and dirty input goods.

\paragraph{Uncertainty 2: regeneration rate of nature}

\paragraph{Uncertainty 2: Substitutability of natural capital in welfare}
Related to technological possibilities is the substitutability of clean and dirty production. 

A key reference is \cite{Cohen2019AnnualSubstitutable}. They argue for a limit of substitutability of \textit{natural capital} (defined as the stock of renewable and nonrenewable resources including minerals, soils, plants, animals, water, air, and energy). The role of natural capital for welfare: resources for production, absorption of waste, basic-life support, direct conrtibution to human welfare \ar No perfect substitutability: Humans cannot live without natural resources. 

BUT: my model so far is about greenhouse-gas emissions only.

Definition \textit{sustainability} common in economics: maintaining a non-decreasing level of welfare across generations. (Sonja: this shouls include provision of natural services: human-friendly climate; how to know the utility function of future generations? Eg if habits are relevant, than they might be happy with less consumption). As regards renewables, it must be ensured that renewability is maintained. It is about the substitutability of natural inputs to the welfare function. \textit{Sonja: (1) How should we know how substitutable biodiversity is for humanity if it is about pleasure derived from living in a diverse world. (2) If it is about other factors of biodiversity, such as, maintaining more basal services for human live, such as safety, there might be less disagreement; more certainty on the importance also for future generations; a utility approach but based on objectively defined values: basic needs.}

\cite{Cohen2019AnnualSubstitutable} make the following distinction: \textit{within-input substitution}: =resources used in welfare production: within-input substitution allows to reduce environmental \textbf{impacts} if the same type of input can be obtained from different sources (e.g.: energy: from emission-low sources instead of emission high sources \tr{\ar This is substitution of dirty with clean goods}); \textit{between-input substitution}: = from energy to manufactured capital;\tr{ includes more efficient energy use (\textbf{with the same amount of energy, more can be produced.})} , recycling, reforestation.

\textbf{Sonja:} within-input substitution: \ar same input but different source (progress= less emissions for energy); between-input substitution: 
\ar sector specific production functions are cobb-douglas \ar unit elasticity of input goods.


\textit{Brundtland Comission} definition of sustainable development: \textit{development that meets the \textbf{needs} of the present without compromising the ability of future generations to meet their own needs}.
If various forms of capital are substitutable (i.e. can use manufactured capital, human capital to replace nature) then production only depends on the total capital stock. Then, economic growth is said to be \textit{weakly sustainable} when the total capital stock is non-decreasing; i..e aggregate savings rate is above depreciation rate on all forms of capital. 

\textit{Strong sustainability view}:  a minimum of natural capital must be sustained as it provides non-substitutable inputs to utility. Then, long-run growth must be able to maintain a natural capital stock.  
\ar Define $Y$ broadly to incorporate other necessary consumption goods: stable climate, breathable air, food, water; but these are partially not traded in markets, rather public goods. Then, better to model as public goods; \textbf{Or as another input in final good production}.

\begin{align*}
Y= \left[\left(\underbrace{\left(Y_c^{\frac{\varepsilon_c-1}{\varepsilon_c}}+Y_d^{\frac{\varepsilon_c-1}{\varepsilon_c}}\right)^{\frac{\varepsilon_c}{\varepsilon_c-1}}}_{\text{consumption  good}}\right)^{\frac{\varepsilon_o-1}{\varepsilon_o}}+(\underbrace{N-\bar{N}}_{\text{natural capital}})^{\frac{\varepsilon_o-1}{\varepsilon_o}}\right]^{\frac{\varepsilon_o}{\varepsilon_o-1}}
\end{align*} 
where $\varepsilon_o$  governs the substitutability of natural capital and consumption in the final good; it translates into the substitutability discussed in \cite{Cohen2019AnnualSubstitutable}. For the basic needs level ins terms of natural capital holds: $\bar{N}>0$ so that nature is a necessary good. The variable $N$ determines the consumption of natural capital, such as breathing air, stable climate.
Model technological growth on substitutability between natural capital (including nature as a waste), and other production inputs.  

 
%%%%%%%%%%%%%%%%%%%%%%%%%%%%%%%%%%%%%%%
\subsection{Greenhouse gas emissions Data}
To calibrate the relation of economic production and emissions, I use data from the EPA \url{https://www.epa.gov/newsreleases/latest-inventory-us-greenhouse-gas-emissions-and-sinks-shows-long-term-reductions-0}.
In 2019, the US greenhouse gas emissions amounted to 6,558 million metric tons of carbon dioxide equivalents; that is, 6.558Gt. 

Global greenhouse gas emissions amounted to 34.2 Gt in C02 equivalents in 2019. There was a decline in 2020 (presumably due to the pandemic, and a rebound in 2021 by 5\%)Found here: \url{https://www.iea.org/reports/greenhouse-gas-emissions-from-energy-overview/global-ghg-emissions}; the Global Energy review of the iea is to  be found here: \url{https://www.iea.org/reports/global-energy-review-2021/co2-emissions}.

Natural sinks are, for example, forests, vegetation, soils. 
The epa report (\url{https://www.epa.gov/ghgemissions/inventory-us-greenhouse-gas-emissions-and-sinks-1990-2019}) includes information on sinks. 

Net emissions after taking sinks into account are estimated by the epa to 5,769.1 in 
\paragraph{Translation metric ton to gigatonne}
1.000.000.000 metric tons are 1 gigatonne 

\paragraph{Demand and Production approach}
The OECD, \url{https://www.oecd.org/sti/ind/carbondioxideemissionsembodiedininternationaltrade.htm} differentaites between a demand and a production-side approach to determine emissions on country level! I am now using the production approach. 

For now, I assume that the global emission target is given in gros emissions. I match the US gros emission target so that contribution in 2019 equals contribution to the global gros reduction. 

\subsection{Radiative Forcing}
\ar  leads to uncertainty in temperature response to emissions
(from \url{https://climate.mit.edu/explainers/radiative-forcing})

Radiative forcing is what happens when the amount of energy that enters the Earth’s atmosphere is different from the amount of energy that leaves it. Energy travels in the form of radiation: solar radiation entering the atmosphere from the sun, and infrared radiation exiting as heat. \textbf{If more radiation is entering Earth than leaving—as is happening today—then the atmosphere will warm up.} This is called radiative forcing because the difference in energy can force changes in the Earth’s climate.
Heat in, Heat out

Sunlight is always shining on half of the Earth’s surface. Some of this sunlight (about 30 percent) is reflected back to space. The rest is absorbed by the planet. But as with any warm object sitting in cold surroundings—and space is a very cold place—some energy from Earth is always radiating back out into space as heat.

Radiative forcing measures how much energy is coming in from the sun, compared to how much is leaving. The analysis needed to pin down this exact number is very complicated. \textbf{Many factors, including clouds, polar ice, and the physical properties of gases in the atmosphere, have an effect on this balancing act}, and each has its own level of uncertainty and its own difficulties in being precisely measured. However, we do know that today, more heat is coming in than going out.


Before the industrial era, radiative forcing was in very close balance, and the Earth’s average temperature was more or less stable. For this reason, researchers calculate radiative forcing based on a “baseline” year sometime before the beginning of world industrialization. For example, the Intergovernmental Panel on Climate Change uses 1750 as a baseline year.

Compared to this baseline, radiative forcing can directly measure the ways recent human activities have changed the planet’s climate. \textbf{The biggest change has been the greenhouse gases we have added to the atmosphere, which keep heat from escaping the Earth.} But there have been other changes too. For example, by \textbf{cutting down forests, we have exposed more of the Earth’s surface to sunlight. If that surface is darker than the forest cover, the Earth absorbs more solar radiation;} where it’s lighter, like in the arctic, more sunlight is reflected back into space.

Humans are also adding small particles called\textbf{ aerosols} to the air, from smokestacks, airplanes, and the tailpipes of cars. Aerosols make radiative forcing especially hard to measure, because their effects are highly complex and can work both ways. For example, bright aerosols (like sulfates from coal-burning) can help cool the atmosphere by reflecting light, while dark aerosols (like black carbon from diesel exhausts) absorb heat and lead to warming.

Finally, measures of radiative forcing also include any natural effects that have changed since the baseline year, such as changes in the sun’s output (which has caused a little more warming) and aerosols released into the atmosphere by volcanoes (which cause temporary cooling).

\subsection{Methane emissions: second important ghg}
(source \url{https://climate.mit.edu/ask-mit/why-do-we-compare-methane-carbon-dioxide-over-100-year-timeframe-are-we-underrating})
Methane is a colorless, odorless gas that’s produced both by nature (such as in wetlands when plants decompose underwater) and in industry (for example, natural gas is mostly made of methane). 
Methane is a quick fading (on average it tstays 10 years in the atmoshpere) ghg. But it is far more damaging than CO2, which sticks longer in the atmosphere; for centuries. 
Methane traps 100times more heat than CO2; Co2 closes the gap over time when  methane has broken down. Old standard: look at warming effect of methane over 100 years; but that is too long a horizon as climate change advances. 

Sources of methane: natural gas production, decompostation of plants under water. 
Methane. CH4

\textbf{(from NASA \url{https://svs.gsfc.nasa.gov/4799})}

Methane is a powerful greenhouse gas that traps heat 28 times more effectively than carbon dioxide over a 100-year timescale. Concentrations of methane have increased by more than 150\% since industrial activities and intensive agriculture began. After carbon dioxide, methane is responsible for about 23\% of climate change in the twentieth century. Methane is produced under conditions where little to no oxygen is available. About 30\% of methane emissions are produced by wetlands, including ponds, lakes and rivers. Another 20\% is produced by agriculture, due to a combination of livestock, waste management and rice cultivation. Activities related to oil, gas, and coal extraction release an additional 30\%. The remainder of methane emissions come from minor sources such as wildfire, biomass burning, permafrost, termites, dams, and the ocean. Scientists around the world are working to better understand the budget of methane with the ultimate goals of reducing greenhouse gas emissions and improving prediction of environmental change. For additional information, see the Global Methane Budget.
\subsection{Natural gas}
(source \url{https://www.eia.gov/energyexplained/natural-gas/})
fossil energy source! remains of plants and animals

natural gas can be produced from shale and other types of sedimentary rock formations by forcing water, chemicals, and sand down a well under high pressure (fracking); tis breaks up the formation, and releases the natural gas from the rock. 
\subsection{Energy sources}
(source: eia, \url{https://www.eia.gov/energyexplained/what-is-energy/sources-of-energy.php})

\subsection{Bioenergy with Carbon Capture and Storage (BECCS)}
source. \url{https://en.wikipedia.org/wiki/Bioenergy_with_carbon_capture_and_storage}

 is the process of extracting bioenergy from biomass and capturing and storing the carbon, thereby removing it from the atmosphere. Energy is extracted in useful forms (electricity, heat, biofuels, etc.) as the biomass is utilized through combustion, fermentation, pyrolysis or other conversion methods.
Wide deployment of BECCS is constrained by cost and availability of biomass.

\subsection{Biosequestration} capture and storage of the atmospheric co2 by continual or enhanced biological processes.; reforestation

\subsection{IPCC report 2018 \citep{Rogelj2018MitigationDevelopment.}}
\begin{itemize}
\item literature focused on demand side: 
\begin{itemize}
	\item \cite{Arrow2004AreMuch}: raise the question if consumption is too high
	\item 
	\cite{Bertram2018TargetedScenarios}:   change demand as a parameter in  model; motivation: taking global inequality into account alternative meausres (mitigation policies) to carbon taxes  become optimal. These include lifestyle changes \textbf{in addition to sector-specific carbon taxes!} (25\% lower energy demand and -20\% lower demand for agricultural products )
	\item \cite{Grubler2018ATechnologies} argue for a low increase in demand (projections)
	\item \cite{Liu2018SocioeconomicC}: lifestyle changes become more and more important under more stringent 1.5°C goal 
	\item \cite{VanVuuren2018AlternativeTechnologies}: analyse lifestyle changes motivated by uncertainty and risks surrounding CDRs (carbon dioxide removal) and their competition for land, yet many mitigation pathways to meet Paris Agreement rely on these technologies.
	
	\textit{This paper  explores  a  set  of  what-if  scenarios  that  explore  these  alter-native  assumptions,  and  analyses  the  extent  to  which  they  reduce  the  need  for  CDR.  The  evaluated  measures  (Table  1)  have  been  mentioned in scientific literature and could possibly limit CDR use. } (p. 2) \ar hence: assuming a lower demand what does this imply?
	
	\textit{The  lifestyle  change  scenario  (LiStCh)  assumes  a  radi-cal  value  shift  towards  more  environmentally  friendly  behaviour,  including a healthy, low-meat diet, changes in transport habits and a reduction of heating and cooling levels at homes. Such a shift could be motivated by both environmental and health con} (p.2)
	
	Their models also take other forcers into account 
	
	\textit{ Given  the  possible  disadvantages  of  BECCS,  it  is  important  to  seriously  discuss  and  appraise such alternative pathways. This could focus on issues such as feasibility, social acceptance, associated costs and benefits, requir-ing input from other scientific disciplines to complement the model-based  scenarios. } p. vorletzte
	
	How they introduce a reduction in demand: 
	\begin{itemize}
\item reduced meat consumption; higher consumption of pulses and oilcrops
\item reduction in food waste
(by households and during production process)
\item transport changes: (1) reduced volume of transport, (2) reduction of energy intense modes of transport, \ar (3)  less private vehicle use and increased car sharing; less motorised options; also implies \ar (4) lower air travel demand
\item less residential energy use
\item reduced appliance ownership (a maximum of two per household) and use (less standby, smarter use)!
\item lower demand for plastic and chemicals 
	\end{itemize}
\end{itemize}
\end{itemize}

\section{Labour}
\subsection{Elasticity of labour}
\cite{Bick2018HowImplications}
\begin{itemize}
	\item \ar informative on the behaviour of individuals in the cross-section
	\item from the abstract: \textit{Within
	countries, hours worked per worker are also decreasing in the individual wage for most countries, though in the richest countries,
	hours worked are flat or increasing in the wage.}
\item there is heterogeneity in the wage elasticity of labour across countries! \ar could test theory in data? \ar those countries identified by \cite{Bick2018HowImplications} as having a negative relation of wage and hours worked \ar different effect of fiscal policy on green labour supply 
\item will imply heterogeneity in policy recommendation!:
\begin{itemize}
	\item poorer countries: negative relationship of individual wage and hours worked \ar higher income tax \ar lower wage \ar work more hours; 
	\ar income effect dominates! 
	\item richer countries:  positive relationship; higher income tax \ar lower wage \ar lower hours worked!
	\ar substitution effect dominates!!
\end{itemize}
\item over time hours worked per adult in the US have been falling. 
\item similar patterns when looking at the cross-section of countries
\item extensive margin: employment shares are falling with GDP for poor to medium countries, modest increase when country is rich
\end{itemize}
\cite{Boppart2019LaborPerspectiveb}
\begin{itemize}
	\item a long run perspective, but in the end I am also looking at a long run model!
	\item only look at the intensive margin: hours worked per worker! 
	\item \textbf{they argue for a higher income effect in the long run} BUT ON AGGREGATE 
	\item They look at a time dimension, this is what I am looking at too.
\end{itemize}
\ar GOAL: on the one hand match the cross-sectional wage elasticity, and on the other hand match the elasticity over time
\cite{LansBovenberg1994EnvironmentalTaxation}
\begin{itemize}
	\item they assume an upward sloping labour-supply curve \ar as the wage rate rises, households work more \ar as the wage rate falls, households work less
	\item an increase in the pollution tax reduces employment if the \textbf{uncompensated wage elasticity of labor supply} is positive \ar i.e. the substitution effect exceeds the income effect
\end{itemize}

\subsection{Skill and environment}
\tr{\textbf{Question} How is skill accumulated here? Retraining to higher skill level or long run decision based on education?\ar pre-job entry; How calibrated?\ar not a quantitative model; Inequality?\ar choice variable skill differences due to OLG structure...but only live for one period... not sure where both skill levels result from when households are all the same...}
\begin{itemize}
	\item \cite{Vona2018EnvironmentalExploration}
	\begin{itemize}
		\item differentiate also occupations\ar engineering skills and managering skills; this  paper less on skill but more on jobs?
		\item ``\textit{we identify two core sets of green skills for which green jobs differ from non-green jobs: engineering skills, and managerial skills }'' (p.2) \ar these skills (=jobs) are especially used in green occupations; \ar jobs which are used heavily in the green sector (p.3)
		\item environmental regulatory policies increase demand for some green skills (engineering, managing); no impact on employment in general
		\item ``\textit{The adjustment costs from job losses can be exacerbated when the skill profile of expanding jobs does not match the skill profile of contracting jobs}'' (p.3)
		\item ``\textit{while the skill gap between green and brown jobs within the same occupational group is generally small }'' (p.4) \ar model as neutral jobs, ``\textit{... exceptions emerge: largest skill gaps occur in construction and extraction occupations [...] important for climate change}''
		\item transitioning to greener production requires a transition in skills
	\end{itemize}
	\item \cite{Borissov2019CarbonDevelopment}
	\begin{itemize}
		\item skills are crucial a determinant of green growth as the labour required for green production is special: higher skills/ higher human capital accumulation
		\item on this they cite policy recommendation papers and \cite{Vona2018EnvironmentalExploration} which they cite as "green skills are closely related to the design, production, management and monitoring of technology and conclude that education emerges as a critical ingredient in the policy mix to promote sustainable economic growth"
		\item their focus seems not to lie on inequality!
		\item their idea: requiring non-developed countries to reduce emissions \ar higher carbon taxes \ar increase in human capital investment since the green sector uses high-skilled labour \ar economic growth! (\textit{how measured?}) \ar a win-win situation
		\item North-South knowledge spillover: if the carbon tax leads to knowledge growth in the green sector this might spread to the South even if the South itself does not levy a pollution tax
		\item human capital accumulation is the driver of growth in this model (not innovations! no directed technical change)
		\item no a-priori inequality in their model! Skill is a choice variable! 
		\item "clean sectors  tend to be skill intensive"
		\item positive intergenerational spill over in skills! (longer-run effects)
		\item positive effect of human capital accumulation on TFP see Lucas 1988, Glomm and Ravikumar 1992
		\item \textbf{incentives to human capital accumulation through carbon taxes!}
		\item \textbf{Model}
		\begin{itemize}
			\item model of successive generations: OLG
			\item no preference heterogeneity, acquiring education costs labour income
			\item skilled, unskilled is a discrete choice
			\item spill over of skills within country
			\item production only requires labour skilled and unskilled,
			\item sector specific goods are perfect substitutes! they produce the same stuff
			\item general production function, but also Cobb-Douglas, then MPhigh skill relative to MP lowskill in clean sector is higher in the clean than in the dirty sector if income share of high skilled in clean is higher than in dirty. \ar calculate relative marginal product of skilled labour in clean and dirty sector
			\item section 5: endogenous growth version: higher share of high-skilled \ar higher growth\ar carbon tax implies higher growth!
			\item carbon tax leads to growth! in their model due to intensified skill accumulation
		\end{itemize}
	\end{itemize}
	
	\item \cite{Consoli2016DoCapital}
	\begin{itemize}
		\item findings: labour force characteristics of green and non-green jobs
		\begin{itemize}
			\item green jobs: high-level cognitive and interpersonal, higher level of formal education, more work experience and on-the-job training compared to non-green; 
			\item use O*NET (Occupational Information Network) comprising 905 occupations
			\item in new occupations which emerge due to a higher demand of green skills on-the-job trainging is a distinctive feature but not in already existing occupations
			\item review policy effects on employment in literature; comment that distinction between job characteristics in these papers are missing
			\item green occupations: (estimates are within SOC3 digit occupations, that is, they are conditional differences in expectations; not unconditional, in the paper I would want to take macro occupational differences into account too, its not only about being green but also that these green jobs are within some specific group: driven by heterogeneity in the average skill content in the macro group ); and green occupations are more within high skilled occupational groups, on the other hand I also only need to focus on occupations where a distinction between green and non-green can be made, and another sector that is neutral (quantitative analysis)\ar the conditional estimates fit well
			\item findings:
			\begin{itemize}
				\item 
				significantly more non-routine tasks and significantly less routine tasks than in the non-green counterparts, p.1052
				\item 19 percent more years of eductaion ($\approx$ 13 weeks), 43\% more years of experience ( $\approx$ 10 months at the mean), 41\% more years of trainnig,($\approx$ 15 weeks); p.1053
				\item green enhanced jobs are more exposed to all measures of technology (p.1053) 
			\end{itemize}
		\end{itemize}
	\end{itemize}
\end{itemize}

\section{Model}
\subsection{Model as in \cite{Fried2018ClimateAnalysis}}
\begin{align*}
\text{Household}\\
& C_t=w_{lft}L_{lft}+w_{lgt}L_{lgt}+w_{lnt}L_{lnt}+w_{sft}S_{sft}+w_{sgt}S_{sgt}+w_{snt}S_{snt}+\\ &\int_{0}^{1}\pi_{fit}+\pi_{git}+\pi_{nit}di+T_t\\
\text{Final good producer optimality}&\\
\text{Optimality}\ \vspace{4mm}& \delta_yY_t^\frac{1}{\varepsilon_y}E_t^{-\frac{1}{\varepsilon_e}}=p_{Et}\\
& (1-\delta_y)Y_t^\frac{1}{\varepsilon_y}N_t^{-\frac{1}{\varepsilon_e}}=p_{Nt}\\
&
p_{Et}E_t^\frac{1}{\varepsilon_e}\tilde{F}_t^{-\frac{1}{\varepsilon_e}}=p_{\tilde{F}t}\\
& p_{Et}E_t^\frac{1}{\varepsilon_e}G_t^{-\frac{1}{\varepsilon_e}}=p_{Gt}\\
& p_{\tilde{F}t}\tilde{F}_t^\frac{1}{\varepsilon_f}\delta_fF_t^{-\frac{1}{\varepsilon_f}}=p_{Ft}+\tau_{f}\\
& p_{\tilde{F}t}\tilde{F}_t^\frac{1}{\varepsilon_f}(1-\delta_f)O_t^{-\frac{1}{\varepsilon_f}}=p_{Ot}+\tau_{o}\\
\text{Definitions prices}\ \vspace{4mm}&
p_{yt}= \left[\delta_y^{\varepsilon_y}p_{Et}^{1-\varepsilon_y}+(1-\delta_y)^{\varepsilon_y}p_{Nt}^{1-\varepsilon_y}\right]^\frac{1}{1-\varepsilon_y}\\
& p_{Et}= \left[p_{\tilde{F}t}^{1-\varepsilon_e}+p_{Gt}^{1-\varepsilon_y}\right]^\frac{1}{1-\varepsilon_e}\\
& p_{\tilde{F}t}= \left[\delta_f^{\varepsilon_f}(p_{Ft}+\tau_{f})^{1-\varepsilon_f}+(1-\delta_f)^{\varepsilon_f}(p_{nt}+\tau_{o})^{1-\varepsilon_f}\right]^\frac{1}{1-\varepsilon_f}\\
\text{Production}\ \vspace{4mm}& 
Y_t=\left(\delta_yE_t^\frac{\varepsilon_y-1}{\varepsilon_y}+(1-\delta_y)N_t^\frac{\varepsilon_y-1}{\varepsilon_y}\right)^\frac{\varepsilon_y}{\varepsilon_y-1}\\
&E_t=\left(\tilde{F}_t^\frac{\varepsilon_e-1}{\varepsilon_e}+G_t^\frac{\varepsilon_e-1}{\varepsilon_e}\right)^\frac{\varepsilon_e}{\varepsilon_e-1}\\
&\tilde{F}_t=\left(\delta_fF_t^\frac{\varepsilon_f-1}{\varepsilon_f}+(1-\delta_f)O_t^\frac{\varepsilon_f-1}{\varepsilon_f}\right)^\frac{\varepsilon_f}{\varepsilon_f-1}\\
\text{Intermediate good producers}\\
\text{Production}\ \vspace{4mm}& F_t= (\alpha_f^2p_{Ft})^\frac{ \alpha_f}{1-\alpha_f}A_{ft}L_{ft}\\
&N_t= (\alpha_n^2p_{Nt})^\frac{ \alpha_n}{1-\alpha_n}A_{nt}L_{nt}\\
&G_t= (\alpha_g^2p_{Gt})^\frac{ \alpha_g}{1-\alpha_g}A_{gt}L_{gt}\\
\text{ Definition agg. Technology}\ \vspace{4mm}&
A_t= \frac{\rho_gA_{gt}+\rho_nA_{nt}+\rho_fA_{ft}}{\rho_n+\rho_g+\rho_f}
\end{align*}

\begin{align*}
\text{Labour demand}\\
& w_{lft}=p_{Ft}(1-\alpha_f)(\alpha_f^2p_{Ft})^\frac{\alpha_f}{1-\alpha_f}A_{ft}\\
& w_{lnt}=p_{Nt}(1-\alpha_n)(\alpha_n^2p_{Nt})^\frac{\alpha_n}{1-\alpha_n}A_{nt}\\
& w_{lgt}=p_{Gt}(1-\alpha_g)(\alpha_g^2p_{Gt})^\frac{\alpha_g}{1-\alpha_g}A_{gt}\\
\text{Mashine demand}\ \vspace{4mm}\\
&x_{fit}= \left(\alpha_f^2 p_{Ft}\right)^\frac{1}{1-\alpha_f}L_{ft}A_{fit}\\
&x_{nit}= \left(\alpha_n^2 p_{Nt}\right)^\frac{1}{1-\alpha_n}L_{nt}A_{nit}\\
&x_{git}= \left(\alpha_g^2 p_{Gt}\right)^\frac{1}{1-\alpha_g}L_{gt}A_{git}\\
\text{Mashine producers}\\
\text{Price setting}\ \vspace{4mm}&p_{fit}^x=\frac{1}{\alpha_f}\\
&p_{nit}^x=\frac{1}{\alpha_n}\\
&p_{git}^x=\frac{1}{\alpha_g}\\ 
\text{Demand Scientists}\ \vspace{4mm}&
w_{sft}=\frac{\eta \gamma \alpha_f A_{ft-1}^{1-\phi}A_{t-1}^{\phi}\left(\frac{S_{ft}}{\rho_f}\right)^{\eta}p_{Ft}F_t}{\frac{1}{1-\alpha_f}S_{ft}A_{ft}}\\
&
w_{snt}=\frac{\eta \gamma \alpha_n A_{nt-1}^{1-\phi}A_{t-1}^{\phi}\left(\frac{S_{nt}}{\rho_n}\right)^{\eta}p_{Nt}N_t}{\frac{1}{1-\alpha_n}S_{nt}A_{nt}}\\
&
w_{sgt}=\frac{\eta \gamma \alpha_g A_{gt-1}^{1-\phi}A_{t-1}^{\phi}\left(\frac{S_{gt}}{\rho_g}\right)^{\eta}p_{Gt}G_t}{\frac{1}{1-\alpha_g}S_{gt}A_{gt}}\\
\text{Innovation}\ \vspace{4mm}&
A_{fit}=A_{ft-1}\left(1+\gamma\left(\frac{S_{fit}}{\rho_f}\right)^{\eta}\left(\frac{A_{t-1}}{A_{ft-1}}\right)^{\phi}\right) \\
&
A_{nit}=A_{nt-1}\left(1+\gamma\left(\frac{S_{nit}}{\rho_n}\right)^{\eta}\left(\frac{A_{t-1}}{A_{nt-1}}\right)^{\phi}\right) \\
&
A_{git}=A_{gt-1}\left(1+\gamma\left(\frac{S_{git}}{\rho_g}\right)^{\eta}\left(\frac{A_{t-1}}{A_{gt-1}}\right)^{\phi}\right) \\
\text{Markets}&\\
&S_{ft}+S_{nt}+S_{gt}=S\\
&L_{ft}+L_{nt}+L_{gt}=L\\
&C_{t}+\int_{0}^{1} x_{fit}+x_{nit}+x_{git}d_i+P_{Ot}O_t=Y_t\\
&P_{Ot} \text{taken as given, imports}
\end{align*}

\subsection{Balanced growth path in \cite{Fried2018ClimateAnalysis}}
She assumes that the ratio of prices is constant and energy prices themselves are constant (p.103) (\textit{that sounds wrong}). Assuming that the spillover effect is sufficiently strong (that is, $\phi$ is large) then a balanced growth path may exist on which the ratio of green and fossil technology, $A_g/A_f$, is constant. This ratio being constant follows from constant price ratios! 

In equilibrium it has to hold that 
\begin{align*}
P_{Gt}G_t= P_{\tilde{F}t}\tilde{F}\left(\frac{P_{\tilde{F}t}}{P_{Gt}}\right)^{\varepsilon_e-1}.
\end{align*} 

\subsection{Model}
\begin{align}
\text{\textbf{Household}}& \max \frac{C_t^{1-\theta}}{1-\theta}-\chi\frac{z_hh_{ht}^{1+\sigma}+z_lh_{lt}^{1+\sigma}}{1+\sigma}-\chi_s\frac{S^{1+\sigma}}{{1+\sigma}} %when z is also to the power of 1+sigma than, the higher zh the lower hours supplied! Not reasonable
\\
\text{Budget}\ \vspace{4mm}& C_t=z_h \lambda_t (w_{ht}h_{ht})^{1-\tau_{lt}}+z_l \lambda_t (w_{lt}h_{lt})^{1-\tau_{lt}}+T^{Gov}_t\\
\text{Optimality}\ \vspace{4mm}
& C_t^{-\theta}= \mu_tp_{yt}\\
& \chi h_{ht}^{\sigma}=\mu_t \lambda_t(1-\tau_{lt})w_{ht}^{1-\tau_{lt}}h_{ht}^{-\tau_{lt}}-\gamma_{ht}/z_h\\
& \chi h_{lt}^{\sigma}=\mu_t \lambda_t(1-\tau_{lt})w_{lt}^{1-\tau_{lt}}h_{lt}^{-\tau_{lt}}-\gamma_{lt}/(1-z_h)\\
%&( h_{st})^{\sigma}=\mu_t \lambda_t(1-\tau_{lt})w_{st}^{1-\tau_{lt}}h_{st}^{-\tau_{lt}}\\
\Rightarrow\ \ & \frac{h_{ht}}{h_{lt}}=\left(\frac{w_{ht}}{w_{lt}}\right)^{\frac{1-\tau_{lt}}{{\sigma+\tau_{lt}}}}\ \text{(Interior solution)}
\\
& \chi_s S^\sigma =\mu w_s-\gamma_{st}\ \text{(scientist income confiscated by gov.)}\\
\text{\textbf{Final good and Energy producers}}&\\
\text{Optimality}\ \vspace{4mm}&
\frac{E_t}{N_t}=\frac{\delta_y}{(1-\delta_y)}\left(\frac{p_{Nt}}{p_{Et}}\right)^{\varepsilon_y} p_{yt}\delta_yY_t^\frac{1}{\varepsilon_y}E_t^{-\frac{1}{\varepsilon_e}}=p_{Et}\\
& p_{yt}(1-\delta_y)Y_t^\frac{1}{\varepsilon_y}N_t^{-\frac{1}{\varepsilon_e}}=p_{Nt}\\
&
p_{Et}E_t^\frac{1}{\varepsilon_e}{F}_t^{-\frac{1}{\varepsilon_e}}=p_{{F}t}\\
& p_{Et}E_t^\frac{1}{\varepsilon_e}G_t^{-\frac{1}{\varepsilon_e}}=p_{Gt}\\
%& p_{\tilde{F}t}\tilde{F}_t^\frac{1}{\varepsilon_f}\delta_fF_t^{-\frac{1}{\varepsilon_f}}=p_{Ft}+\tau_{f}\\
%& p_{\tilde{F}t}\tilde{F}_t^\frac{1}{\varepsilon_f}(1-\delta_f)O_t^{-\frac{1}{\varepsilon_f}}=p_{Ot}+\tau_{o}
%\\
\text{Definitions prices}\ \vspace{4mm}&
p_{yt}= \left[\delta_y^{\varepsilon_y}p_{Et}^{1-\varepsilon_y}+(1-\delta_y)^{\varepsilon_y}p_{Nt}^{1-\varepsilon_y}\right]^\frac{1}{1-\varepsilon_y}\\
& p_{Et}= \left[p_{{F}t}^{1-\varepsilon_e}+p_{Gt}^{1-\varepsilon_y}\right]^\frac{1}{1-\varepsilon_e}\\
%& p_{\tilde{F}t}= \left[\delta_f^{\varepsilon_f}(p_{Ft}+\tau_{f})^{1-\varepsilon_f}+(1-\delta_f)^{\varepsilon_f}(p_{nt}+\tau_{o})^{1-\varepsilon_f}\right]^\frac{1}{1-\varepsilon_f}
%\\
\text{Production}\ \vspace{4mm}& 
Y_t=\left(\delta_yE_t^\frac{\varepsilon_y-1}{\varepsilon_y}+(1-\delta_y)N_t^\frac{\varepsilon_y-1}{\varepsilon_y}\right)^\frac{\varepsilon_y}{\varepsilon_y-1}\\
&E_t=\left({F}_t^\frac{\varepsilon_e-1}{\varepsilon_e}+G_t^\frac{\varepsilon_e-1}{\varepsilon_e}\right)^\frac{\varepsilon_e}{\varepsilon_e-1}\\
%&\tilde{F}_t=\left(\delta_fF_t^\frac{\varepsilon_f-1}{\varepsilon_f}+(1-\delta_f)O_t^\frac{\varepsilon_f-1}{\varepsilon_f}\right)^\frac{\varepsilon_f}{\varepsilon_f-1}\\
\text{\textbf{Intermediate good producers}}&\\
\text{Production}\ \vspace{4mm}& F_t= (\alpha_f(p_{Ft}(1-\tau_{ft})))^\frac{ \alpha_f}{1-\alpha_f}A_{ft}L_{ft}\\
&N_t= (\alpha_np_{Nt})^\frac{ \alpha_n}{1-\alpha_n}A_{nt}L_{nt}\\
&G_t= (\alpha_gp_{Gt})^\frac{ \alpha_g}{1-\alpha_g}A_{gt}L_{gt}\\
%\end{align}
%
%\begin{align}
\text{Labour demand}\label{eq:lab_demand}\\
& w_{lft}=(p_{Ft}(1-\tau_{ft}))^\frac{1}{1-\alpha_f}(1-\alpha_f)(\alpha_f)^\frac{\alpha_f}{1-\alpha_f}A_{ft}\\
& w_{lnt}=p_{Nt}^\frac{1}{1-\alpha_n}(1-\alpha_n)(\alpha_n)^\frac{\alpha_n}{1-\alpha_n}A_{nt}\\
& w_{lgt}=p_{Gt}^\frac{1}{1-\alpha_g}(1-\alpha_g)(\alpha_g)^\frac{\alpha_g}{1-\alpha_g}A_{gt}
\\
\text{Machine demand}&
\\
&x_{fit}= \left(\alpha_f p_{Ft}(1-\tau_{ft})\right)^\frac{1}{1-\alpha_f}L_{ft}A_{fit}\\
&x_{nit}= \left(\alpha_n p_{Nt}\right)^\frac{1}{1-\alpha_n}L_{nt}A_{nit}\\
&x_{git}= \left(\alpha_g p_{Gt}\right)^\frac{1}{1-\alpha_g}L_{gt}A_{git}
\\
\text{\textbf{Labour producers}}&
\\
\text{Production}\ \vspace{4mm}& L_{ft}=h_{hft}^{\theta_{f}}h_{lft}^{1-\theta_{f}}\\
& L_{nt}=h_{hnt}^{\theta_{n}}h_{lnt}^{1-\theta_{n}}\\
& L_{gt}=h_{hgt}^{\theta_{g}}h_{lgt}^{1-\theta_{g}}\\
\ \\
\text{Optimality}\ \vspace{4mm}& h_{hft}= \theta_{f}L_{ft}\frac{w_{lft}}{w_{ht}}\label{eq:opt_lab_pro}\\
& h_{hnt}= \theta_{n}L_{nt}\frac{w_{lnt}}{w_{ht}}\\
& h_{hgt}= \theta_{g}L_{gt}\frac{w_{lgt}}{w_{ht}}\\
& h_{lft}= (1-\theta_{f})L_{ft}\frac{w_{lft}}{w_{lt}}\label{eq:opt_lab_pro_low}\\
& h_{lnt}= (1-\theta_{n}) L_{nt}\frac{w_{lnt}}{w_{lt}}\\
& h_{lgt}= (1-\theta_{g}) L_{gt}\frac{w_{lgt}}{w_{lt}}\\
%\end{align}
%
%\begin{align}
\text{\textbf{Machine producers}}\\
\text{Price setting}\ \vspace{4mm}&p_{fit}^x=\frac{1}{\alpha_f(1+\zeta_f)}\\
&p_{nit}^x=\frac{1}{\alpha_n(1+\zeta_n)}\\
&p_{git}^x=\frac{1}{\alpha_g(1+\zeta_g)}
\\ 
\text{Demand Scientists}\ \vspace{4mm}&
w_{sft}=\frac{\eta \gamma A_{ft-1}^{1-\phi}A_{t-1}^{\phi}\left(\frac{S_{ft}}{\rho_f}\right)^{\eta}p_{Ft}(1-\tau_{ft})F_t}{\frac{1}{1-\alpha_f}S_{ft}A_{ft}}\\
&
w_{snt}=\frac{\eta \gamma  A_{nt-1}^{1-\phi}A_{t-1}^{\phi}\left(\frac{S_{nt}}{\rho_n}\right)^{\eta}p_{Nt}N_t}{\frac{1}{1-\alpha_n}S_{nt}A_{nt}}\\
&
w_{sgt}=\frac{\eta \gamma  A_{gt-1}^{1-\phi}A_{t-1}^{\phi}\left(\frac{S_{gt}}{\rho_g}\right)^{\eta}p_{Gt}G_t}{\frac{1}{1-\alpha_g}S_{gt}A_{gt}(1-\tau_{st})}\\
\text{Innovation}\ \vspace{4mm}&
A_{fit}=A_{ft-1}\left(1+\gamma\left(\frac{S_{fit}}{\rho_f}\right)^{\eta}\left(\frac{A_{t-1}}{A_{ft-1}}\right)^{\phi}\right) \\
&
A_{nit}=A_{nt-1}\left(1+\gamma\left(\frac{S_{nit}}{\rho_n}\right)^{\eta}\left(\frac{A_{t-1}}{A_{nt-1}}\right)^{\phi}\right) \\
&
A_{git}=A_{gt-1}\left(1+\gamma\left(\frac{S_{git}}{\rho_g}\right)^{\eta}\left(\frac{A_{t-1}}{A_{gt-1}}\right)^{\phi}\right) \\
%\text{Demand Scientists}\ \vspace{4mm}&
%w_{sft}=\frac{\eta \gamma \alpha_f A_{ft-1}\left(\frac{S_{ft}}{\rho_f}\right)^{\eta}p_{Ft}F_t}{\frac{1}{1-\alpha_f}S_{ft}A_{ft}}\label{eq:demand_sc}\\
%&
%w_{snt}=\frac{\eta \gamma \alpha_n A_{nt-1}\left(\frac{S_{nt}}{\rho_n}\right)^{\eta}p_{Nt}N_t}{\frac{1}{1-\alpha_n}S_{nt}A_{nt}}\\
%&
%w_{sgt}=\frac{\eta \gamma \alpha_g A_{gt-1}\left(\frac{S_{gt}}{\rho_g}\right)^{\eta}p_{Gt}G_t}{\frac{1}{1-\alpha_g}S_{gt}A_{gt}}\\
%\text{Innovation}\ \vspace{4mm}&
%A_{fit}=A_{ft-1}\left(1+\gamma\left(\frac{S_{fit}}{\rho_f}\right)^{\eta}\right) \\
%&
%A_{nit}=A_{nt-1}\left(1+\gamma\left(\frac{S_{nit}}{\rho_n}\right)^{\eta}\right) \\
%&
%A_{git}=A_{gt-1}\left(1+\gamma\left(\frac{S_{git}}{\rho_g}\right)^{\eta}\right) \\
%&A_t=\max\{A_{nt}, A_{ft}, A_{gt}\}\\`
%DROP THE FOLLOWING AS IT DOES NOT GROW AT A CONSTANT RATE IF SHARES ARE CHANGING! &A_t= \frac{\rho_fA_{ft}+\rho_gA_{gt}+\rho_nA_{nt}}{\rho_f+\rho_n+\rho_g}\\
\text{\textbf{Government}}&\\
&T_t=\int_{0}^{1}\pi_{fit}+\pi_{git}+\pi_{nit}di+z_h(w_{ht}h_{ht}-\lambda_t(w_{ht}h_{ht})^{(1-\tau_{lt})})\\&+z_l(w_{lt}h_{lt}-\lambda_t(w_{lt}h_{lt})^{(1-\tau_{lt})})+w_{st}s_{ft}+w_{st}s_{gt}+w_{st}s_{nt}\\ &+\tau_{ct}p_{ft}(\omega_FF_t)-w_{sgt}\tau_{st}s_{gt} \\ &+\int_{0}^{1} p_{fit}\zeta_{ft}x_{fit}+p_{git}\zeta_{gt}x_{git}+ p_{hit}\zeta_{ht}x_{hit}di\\
\text{with}\ \vspace{4mm}&\zeta_{jt}=\frac{1-\alpha_j}{\alpha_j} \\
\text{simplified}\ \vspace{4mm} & T_t= z_h(w_{ht}h_{ht}-\lambda_t(w_{ht}h_{ht})^{(1-\tau_{lt})})\\&+z_l(w_{lt}h_{lt}-\lambda_t(w_{lt}h_{lt})^{(1-\tau_{lt})})+\tau_{ct}p_{ft}(\omega_FF_t) \\
\text{\textbf{Markets}}&\\
& S_{ft}+ S_{nt}+ S_{gt}=S_t\\
&h_{hft}+h_{hnt}+h_{hgt}=z_{h} h_{ht}\\
&h_{lft}+h_{lnt}+h_{lgt}=z_{l} h_{lt}\\
&C_{t}+\int_{0}^{1} x_{fit}+x_{nit}+x_{git}d_i=Y_t
\end{align}

\subsection{BGP: which can accomodate a trend in hours and diverging productivity shares}
\textbf{\tr{But this version assumes constant labour shares \ar this implies constant mashine growth and labour growth in each sector. Then again, on the BGP labour supply reduces, I only assume this affects sector labour input proportionately. There may also not be any growth in machines. }}
From the optimality condition in skill demand by labour producing firms, equations \ref{eq:opt_lab_pro} and \ref{eq:opt_lab_pro_low}, and the price paid by intermediate good producers in sector $j\in\{F, G, N\}$, $w_{ljt}$, equations \ref{eq:lab_demand}, yields the sector specific demand for high and low skill as a function of output in this sector. Substituting these equations in the intermediate good production function determines technology as a function of prices:
\begin{align}
A_{jt} = \left[\alpha_j^{2\frac{\alpha_j}{1-\alpha_j}}p_{jt}^\frac{1}{1-\alpha_j}(1-\alpha_j)\left(\frac{1-\theta_j}{w_{lt}}\right)^{1-\theta_j}\left(\frac{\theta_j}{w_{ht}}\right)^{\theta_j}\right] ^{-1}
\end{align}
Under the assumption of a stable wage premium, one can detrend technological progress as:
\begin{align}
\hat{A_{jt}}:=\frac{A_{jt}p_{jt}^\frac{1}{1-\alpha_j}}{w_{ht}}= \left[\alpha_j^{2\frac{\alpha_j}{1-\alpha_j}}(1-\alpha_j)\left(1-\theta_j\right)^{1-\theta_j}\theta_j^{\theta_j}\left(\frac{w_{ht}}{w_{lt}}\right)^{1-\theta_j}\right] ^{-1}\label{eq:A_det}
\end{align}
Both wage rates for high and low skill labour hence grow at the rate
\begin{align}
\gamma_{w}=\left(\frac{p_{jt+1}}{p_{jt}}\right)^\frac{1}{1-\alpha_j}\frac{A_{jt+1}}{A_{jt}}-1  \ \forall \ j. 
\end{align}
Hence, free skill movement across labour input firms, implies that on a BGP
\begin{align}
\left(\frac{p_{gt+1}}{p_{gt}}\right)^\frac{1}{1-\alpha_g}\frac{A_{gt+1}}{A_{gt}}=\left(\frac{p_{ft+1}}{p_{ft}}\right)^\frac{1}{1-\alpha_f}\frac{A_{ft+1}}{A_{ft}}=\left(\frac{p_{nt+1}}{p_{nt}}\right)^\frac{1}{1-\alpha_n}\frac{A_{nt+1}}{A_{nt}}. \label{eq:const_prA} 
\end{align}

\tr{Drop assumption that input shares are constant}
These conditions ensure that relative expenditures on intermediate goods are constant \textbf{whenever the labour input ratios are constant}. Substituting intermeidate good production into the expenditure ratio $p_{ft}F_t/(p_{gt}G_t)$  yields
\begin{align}
&\frac{p_{ft+1}^\frac{1}{1-\alpha_f}A_{ft+1}L_{ft+1}}{p_{gt+1}^\frac{1}{1-\alpha_g}A_{gt+1}L_{gt+1}}= \frac{p_{ft}^\frac{1}{1-\alpha_f}A_{ft}L_{ft}}{p_{gt}^\frac{1}{1-\alpha_g}A_{gt}L_{gt}}\\
&\Leftrightarrow \frac{L_{ft+1}}{L_{gt+1}}=\frac{L_{ft}}{L_{gt}},
\end{align}
where the second line follows from \ref{eq:const_prA}. 

Note that the assumption that the ratio of technological progress is constant over time is necessary to have aggregate technology, $A_t$, as defined in Fried constant. I don't want to make this assumption to allow for zero growth in the fossil sector. Therefore, I define the leading technology as 
\begin{align}
A_t= \max\{A_{nt}, A_{gt}, A_{ft}\}
\end{align}
Since on the BGP each technology growths at a constant rate, the leading technology growths at a constant rate  whenever the fastest growing technology is also the leading one in levels.

The assumption of a stable wage premium together with a constant progressivity parameter ensures that relative skill supply on the BGP is stable, too: 
\begin{align}
\frac{H_{ht}}{H_{lt}}=\left(\frac{w_{ht}}{w_{lt}}\right)^\frac{1-\tau_{lt}}{1+\sigma}\frac{z_h}{z_l}.
\end{align}

A constant wage ratio also implies that relative skill employment in each sector is constant, $\frac{h_{hj}}{h_{lf}}$, which follows from skill demand by labour producers. 


Observe that skill shares employed in each sector, $\frac{h_{pjt}}{H_{pt}} \forall \ j\ \text{and} \ p\in\{h,l\}$, are constant given constant expenditure ratios. 
To see this, substitute demand for skill inputs by labour producing firms in the market clearing for high skill labour. Substituting labour demand by intermediate good producers yields
\begin{align}
w_{ht}H_{ht}= \left(\underbrace{\theta_f (1-\alpha_f)\frac{p_{ft}F_t}{p_{gt}G_t}+\theta_{g}(1-\alpha_g)+\theta_n(1-\alpha_n)\frac{p_{nt}N_t}{p_{gt}G_t}}_{:=\vartheta_t}\right)p_{gt}G_t
\end{align}
Replacing $G_t$ by its production function and substituting $L_{gt}$ using the production function and optimal skill ratios implies
\begin{align}
\frac{h_{lgt}}{H_{ht}}=\left(\frac{1}{\vartheta_t\alpha_g^{2\frac{\alpha_g}{1-\alpha_g}}\left(\frac{\theta_g}{1-\theta_g}\frac{w_{lt}}{w_{ht}}\right)^{\theta_{g}}}\right)\frac{w_{ht}}{A_{gt}p_{gt}^\frac{1}{1-\alpha_g}}.
\end{align}
Since expenditure shares are constant, so is the first multiplier. The second multiplier is constant following equation \ref{eq:A_det}. Since $\frac{H_h}{H_l}$ is constant by households optimality condition, above equation implies that $\frac{h_{lgt}}{H_{lt}}$ is time invariant. Note that I assume that skill shares (relative to total skill supplied) are stable on the BGP. 
\paragraph{Scientists and the marginal gains from innovation}
Scientists are in fixed supply. They receive the competitive wage rate and consume their income. Their income is not taxed and they do not receive transfers. This is to avoid redistribution from scientists to households and vice versa.  (In an extension could assume preferential tax schemes for scientists.) Could also add income from scientists to representative household but let it not be taxed. 
%The amount of scientists on the BGP is flexible in order to keep sector-specific technological growth constant. This follows from the law of motion from technologym let $\gamma_{Aj}$ denote technology growth in sector j:
%\begin{align}
%S_{jt}=\left(\frac{\gamma_{Aj}}{\gamma}\right)^\frac{1}{\eta}\rho_j{A_{t-1}}^{\frac{-\phi}{\eta}}. \label{eq:scien}
%\end{align}
%The amount of scientists varies with overall productivity. For $\eta>0$ and $\phi>0$ the higher output the lower the amount of scientists employed. 
%The marginal profit of innovation positively depends on aggregate technology. 


In equilibrium, wages for scientists, i.e. the marginal product of innovations, are determined by equations \ref{eq:demand_sc}. 
Free movement of scientists requires that
\begin{align}
\left(\frac{S_{gt}}{S_{ft}}\right)^{\eta-1}= \frac{(1-\alpha_f)\alpha_f}{(1-\alpha_g)\alpha_g}\frac{p_{ft}F_t}{p_{gt}G_t}\left(\frac{\rho_g}{\rho_f}\right)^\eta.
\end{align}
Note, that in equilibrium, the equation states that the higher the ratio of scientists in sector g relative to sector f, the higher the technology gap in favour of sector g, under the assumption of increasing returns to scale, $\eta>1$. Increasing returns to scale in innovation seem reasonable, for example, synergy effects from teamwork. Creativity benefits from communication. Increasing returns can be perceived as positive spill-over effects within a sector. 


Substituting equation \ref{eq:scien}, yields a condition on the sector-specific spillover of innovation $\eta$ so that the ratio of workers across sectors is constant on the BGP.
It has to hold that
\begin{align}
\frac{L_{gt}}{L_{ft}}= \left(\frac{\gamma_{AF}}{\gamma_{AG}}\right)^{\eta-1}\left(\frac{1+\gamma_{AG}}{1+\gamma_{AF}}\right)^\phi \left(\frac{A_{ft-1}}{A_{gt-1}}\right)^{\phi(\eta-2)}\frac{\rho_g}{\rho_f}\frac{(1-\alpha_f)\alpha_f^{2\frac{\alpha_f}{1-\alpha_f}}}{(1-\alpha_g)\alpha_g^{2\frac{\alpha_g}{1-\alpha_g}}}\frac{p_{ft}^\frac{1}{1-\alpha_f}A_{ft}}{p_{gt}^\frac{1}{1-\alpha_g}A_{gt}}.
\end{align}

All terms are stable except for $\left(\frac{A_{ft-1}}{A_{gt-1}}\right)^{\phi(\eta-2)}$. Hence, for a balanced growth to have constant worker ratios across sectors, there have to be increasing returns to scientist within sectors and $\eta=2$.
 
Alternatively, one might want to abstain from stable employment ratios. However, then expenditure shares would not be constant, which again is consistent with structural change. When expenditure shares are not constant, then skill moves across sectors. 
However, my take on the BGP in this model is far in the future, after all transitions across sectors have taken place. Note, that I do not need to assume a BGP from 2050 onwards. 
 

\begin{comment}
%content...
\subsection{Equilibrium conditions: own model}

\begin{align*}
\text{\textbf{Household solved:}} \hspace{50mm}& \\
\text{FOCs labour supply}\hspace{4mm}&  %\log(H_t)=\frac{1}{1+\sigma}\log(1-\tau_{lt})\\
H_t=(1-\tau_{lt})^\frac{1}{1+\sigma}\\
\ \hspace{4mm} & %\log(w_{ht})=\log(w_{lt})+\log(\zeta)\\
w_{ht}=\zeta w_{lt}\\
\text{Budget}\hspace{4mm}&  %\log(c_t)= \log(\lambda_t)+ (1-\tau_{lt})\left[\frac{1}{1+\sigma}\log(1-\tau_{lt})+\log(w_{lt})\right]\\
c_t= \lambda_t (H_tw_{lt})^{(1-\tau_{lt})}\\
\text{definition}\  H_t\hspace{4mm} & %\log(H_t)=\log(h_{lt}+\zeta h_{ht})\\
H_t=\zeta h_{ht}+h_{lt}
\\
\text{\textbf{General Household Problem:}} \hspace{50mm}& \\
\text{FOC consumption}\hspace{4mm}& Mu_{ct}=p_t\mu_t\\
\text{FOC low skill}\hspace{4mm} & -Mu_{h_lt}=\mu_t \frac{\partial I_t}{\partial h_{lt}}\\
\text{FOC high skill}\hspace{4mm} & -Mu_{h_ht}=\mu_t \frac{\partial I_t}{\partial h_{ht}}\\
\text{Budget}\hspace{4mm}& c_tp_t= I_t\\
\text{definition}\  H_t\hspace{4mm} & H_t=\zeta h_{ht}+h_{lt}\\
\text{\textbf{Labour sectors:}}\hspace{50mm}&\\
\text{Production clean labour input} \hspace{4mm}& L_{ct}=l_{hct}^{\theta_c}l_{lct}^{1-\theta_c}\\ 
\text{Production dirty labour input} \hspace{4mm}& L_{dt}=l_{hdt}^{\theta_d}l_{ldt}^{1-\theta_d}\\
%
\text{Demand high skill clean sector}\hspace{4mm}&l_{hct}= \left(\frac{p_{cLt}}{w_{ht}}\right)^{\frac{1}{1-\theta_c}}\theta_c^{\frac{1}{1-\theta_c}}l_{lct}\\
%
\text{Demand low skill clean sector } \hspace{4mm}&l_{lct}= \left(\frac{p_{cLt}}{w_{lt}}\right)^{\frac{1}{\theta_c}}(1-\theta_c)^{\frac{1}{\theta_c}}l_{hct}\\
%
\text{Demand high skill dirty sector} \hspace{4mm}&l_{hdt}= \left(\frac{p_{dLt}}{w_{ht}}\right)^{\frac{1}{1-\theta_d}}\theta_d^{\frac{1}{1-\theta_d}}l_{ldt}\\
%
\text{Demand low skill dirty sector } \hspace{4mm}&l_{ldt}= \left(\frac{p_{dLt}}{w_{lt}}\right)^{\frac{1}{\theta_d}}(1-\theta_d)^{\frac{1}{\theta_d}}l_{hdt}\\
\text{\textbf{Government}}\hspace{50mm}& \nonumber\\
\text{Budget}\hspace{4mm}& G_t=H_tw_{lt}-\lambda_t(H_t w_{lt})^{(1-\tau_{lt})}
\\
\text{\textbf{Technology:}}\hspace{50mm}&\\
\text{Clean sector}\hspace{4mm}& A_{ict+1}=(1+\upsilon_{ct})A_{ict}\\
\text{Dirty sector}\hspace{4mm}& A_{idt+1}=(1+\upsilon_{dt})A_{idt}\\
%\text{Progress bound}\hspace{4mm}& \upsilon_{ct}+\upsilon_{dt}=\Upsilon\\
\text{Definition average clean technology}\hspace{4mm}& A_{ct}=\int_0^1A_{ict}di\\
\text{Definition average dirty technology}\hspace{4mm}& A_{dt}=\int_0^1A_{idt}di
\end{align*}

\begin{align}
\text{\textbf{Production:}} \hspace{4mm}
\text{\textbf{Final Good Producer}}&\\
\text{Profit maximisation}\hspace{4mm} & Y_{nt}=\left(\frac{p_{ct}}{p_{dt}}\right)^\varepsilon Y_{ct}\\
\text{Production}\hspace{4mm} & Y_t=\left[Y_{ct}^{\frac{\varepsilon-1}{\varepsilon}}+Y_{dt}^{\frac{\varepsilon-1}{\varepsilon}}\right]^{\frac{\varepsilon}{\varepsilon-1}}\\
\text{Price}\hspace{4mm}& p_t:=\left[p_{ct}^{1-\varepsilon}+p_{dt}^{1-\varepsilon}\right]^{\frac{1}{1-\varepsilon}}\\
\text{\textbf{Clean Sector}}\\
\text{Production}\hspace{4mm}& Y_{ct}=L^{1-\alpha}_{ct}\int_{0}^{1}A^{1-\alpha}_{ict}x_{ict}^{\alpha}di=  \left(\alpha\frac{p_{ct}}{\psi}\right)^{\frac{\alpha}{1-\alpha}}A_{ct} L_{ct} \label{eqbm:outputc}
\\ & =x_{ct}^{\alpha}\left(A_{ct}L_{ct}\right)^{1-\alpha} \\ 
\text{labour demand}\hspace{4mm} & p_{cLt} =
(1-\alpha)\left(\frac{\alpha}{\psi}\right)^\frac{\alpha}{1- \alpha}p_{ct}^\frac{1}{1-\alpha}A_{ct} \label{eqbm:labc} \\
\text{machine demand}\hspace{4mm} & x_{ict} = \left(\alpha\frac{ p_{ct}}{p_{ict}}\right)^\frac{1}{1-\alpha}A_{ict}L_{ct}\\
& x_{ct}:=\int_{0}^{1}x_{ict} di= \left(\alpha\frac{p_{ct}}{\psi}\right)^\frac{1}{1-\alpha}A_{ct}L_{ct}\\
%
\text{Supply machines (price)}\hspace{4mm}& p_{ict}=\psi \\
%
\text{\textbf{Dirty Sector}}\\
\text{Production}\hspace{4mm} & Y_{dt}=L^{1-\alpha}_{dt}\int_{0}^{1}A^{1-\alpha}_{idt}x_{idt}^{\alpha}di=  \left(\alpha\frac{p_{dt}}{\psi}\right)^{\frac{\alpha}{1-\alpha}}A_{dt} L_{dt}\label{eqbm:outputd}\\ & =x_{dt}^{\alpha}\left(A_{dt}L_{dt}\right)^{1-\alpha} \\ 
\text{labour demand}\hspace{4mm} & p_{dLt} =
(1-\alpha)\left(\frac{\alpha}{\psi}\right)^\frac{\alpha}{1- \alpha}p_{dt}^\frac{1}{1-\alpha}A_{dt}\label{eqbm:labd}\\
\text{machine demand}\hspace{4mm} & x_{idt} = \left(\alpha\frac{ p_{dt}}{p_{idt}}\right)^\frac{1}{1-\alpha}A_{idt}L_{dt}\\
& x_{dt}:=\int_{0}^{1}x_{idt} di= \left(\alpha\frac{p_{dt}}{\psi}\right)^\frac{1}{1-\alpha}A_{dt}L_{dt}\\
\text{Supply machines (price)}\hspace{4mm}& p_{idt}=\psi\\
\text{\textbf{Market clearing:}}\hspace{50mm}& \nonumber\\
\text{Final Good}\hspace{4mm}& Y_{t}=c_t+\psi\left(\int_{0}^1x_{idt}di+\int_{0}^1x_{ict}di\right)+G_t%\psi \left(\int_0^1\left(\alpha\frac{p_{dt}}{\psi}\right)^\frac{1}{1-\alpha}A_{idt}L_{dt}+\int_0^1\left(\alpha\frac{p_{ct}}{\psi}\right)^\frac{1}{1-\alpha}A_{ict}L_{ct}\right)
\\
%& \ (\text{Numeraire}\ \  p_t=1)\\
\text{high skill}\hspace{4mm}& l_{hct}+l_{hdt}=h_{ht}\\
\text{low skill}\hspace{4mm}&l_{lct}+l_{ldt}=h_{lt}
\end{align}
\end{comment}


\begin{comment}
\subsection{Equilibrium conditions: Simplified model with Lc=hh, ld=hl}

\begin{align*}
\text{\textbf{Household solved:}} \hspace{50mm}& \\
\text{FOCs labour supply}\hspace{4mm}&  %\log(H_t)=\frac{1}{1+\sigma}\log(1-\tau_{lt})\\
H_t=(1-\tau_{lt})^\frac{1}{1+\sigma}\\
\ \hspace{4mm} & %\log(w_{ht})=\log(w_{lt})+\log(\zeta)\\
w_{ht}=\zeta w_{lt}\\
\text{Budget}\hspace{4mm}&  %\log(c_t)= \log(\lambda_t)+ (1-\tau_{lt})\left[\frac{1}{1+\sigma}\log(1-\tau_{lt})+\log(w_{lt})\right]\\
c_t= \lambda_t (H_tw_{lt})^{(1-\tau_{lt})}\\
\text{definition}\  H_t\hspace{4mm} & %\log(H_t)=\log(h_{lt}+\zeta h_{ht})\\
H_t=\zeta h_{ht}+h_{lt}
\\
%\text{\textbf{Labour sectors:}}\hspace{50mm}&\\
%\text{Production clean labour input} \hspace{4mm}& L_{ct}=h_{ht}\\ 
%\text{Production dirty labour input} \hspace{4mm}& L_{dt}=h_{lt}\\
%
\text{\textbf{Government}}\hspace{50mm}& \nonumber\\
\text{Budget}\hspace{4mm}& G_t=H_tw_{lt}-\lambda_t(H_t w_{lt})^{(1-\tau_{lt})}
\\
\text{\textbf{Technology:}}\hspace{50mm}&\\
\text{Clean sector}\hspace{4mm}& A_{ict+1}=(1+\upsilon_{ct})A_{ict}\\
\text{Dirty sector}\hspace{4mm}& A_{idt+1}=(1+\upsilon_{dt})A_{idt}\\
%\text{Progress bound}\hspace{4mm}& \upsilon_{ct}+\upsilon_{dt}=\Upsilon\\
\text{Definition average clean technology}\hspace{4mm}& A_{ct}=\int_0^1A_{ict}di\\
\text{Definition average dirty technology}\hspace{4mm}& A_{dt}=\int_0^1A_{idt}di
\end{align*}

\begin{align*}
\text{\textbf{Production:}} \hspace{4mm}
\text{\textbf{Final Good Producer}}&\\
\text{Profit maximisation}\hspace{4mm} & Y_{nt}=\left(\frac{p_{ct}}{p_{dt}}\right)^\varepsilon Y_{ct}\\
\text{Production}\hspace{4mm} & Y_t=\left[Y_{ct}^{\frac{\varepsilon-1}{\varepsilon}}+Y_{dt}^{\frac{\varepsilon-1}{\varepsilon}}\right]^{\frac{\varepsilon}{\varepsilon-1}}\\
\text{Price}\hspace{4mm}& p_t:=\left[p_{ct}^{1-\varepsilon}+p_{dt}^{1-\varepsilon}\right]^{\frac{1}{1-\varepsilon}}\\
\text{\textbf{Clean Sector}}\\
\text{Production}\hspace{4mm}& Y_{ct}=L^{1-\alpha}_{ct}\int_{0}^{1}A^{1-\alpha}_{ict}x_{ict}^{\alpha}di=  \left(\alpha\frac{p_{ct}}{\psi}\right)^{\frac{\alpha}{1-\alpha}}A_{ct} L_{ct}
\\ & =x_{ct}^{\alpha}\left(A_{ct}L_{ct}\right)^{1-\alpha} \\ 
\text{labour demand}\hspace{4mm} & w_{ht} =
(1-\alpha)\left(\frac{\alpha}{\psi}\right)^\frac{\alpha}{1- \alpha}p_{ct}^\frac{1}{1-\alpha}A_{ct}\\
\text{machine demand}\hspace{4mm} & x_{ict} = \left(\alpha\frac{ p_{ct}}{p_{ict}}\right)^\frac{1}{1-\alpha}A_{ict}L_{ct}\\
& x_{ct}:=\int_{0}^{1}x_{ict} di= \left(\alpha\frac{p_{ct}}{\psi}\right)^\frac{1}{1-\alpha}A_{ct}L_{ct}\\
%
\text{Supply machines (price)}\hspace{4mm}& p_{ict}=\psi \\
%
\text{\textbf{Dirty Sector}}\\
\text{Production}\hspace{4mm} & Y_{dt}=L^{1-\alpha}_{dt}\int_{0}^{1}A^{1-\alpha}_{idt}x_{idt}^{\alpha}di=  \left(\alpha\frac{p_{dt}}{\psi}\right)^{\frac{\alpha}{1-\alpha}}A_{dt} L_{dt}\\ & =x_{dt}^{\alpha}\left(A_{dt}L_{dt}\right)^{1-\alpha} \\ 
\text{labour demand}\hspace{4mm} & w_{lt} =
(1-\alpha)\left(\frac{\alpha}{\psi}\right)^\frac{\alpha}{1- \alpha}p_{dt}^\frac{1}{1-\alpha}A_{dt}\\
\text{machine demand}\hspace{4mm} & x_{idt} = \left(\alpha\frac{ p_{dt}}{p_{idt}}\right)^\frac{1}{1-\alpha}A_{idt}L_{dt}\\
& x_{dt}:=\int_{0}^{1}x_{idt} di= \left(\alpha\frac{p_{dt}}{\psi}\right)^\frac{1}{1-\alpha}A_{dt}L_{dt}\\
\text{Supply machines (price)}\hspace{4mm}& p_{idt}=\psi\\
\text{\textbf{Market clearing:}}\hspace{50mm}& \nonumber\\
\text{Final Good}\hspace{4mm}& Y_{t}=c_t+\psi\left(\int_{0}^1x_{idt}di+\int_{0}^1x_{ict}di\right)+G_t%\psi \left(\int_0^1\left(\alpha\frac{p_{dt}}{\psi}\right)^\frac{1}{1-\alpha}A_{idt}L_{dt}+\int_0^1\left(\alpha\frac{p_{ct}}{\psi}\right)^\frac{1}{1-\alpha}A_{ict}L_{ct}\right)
\\
%& \ (\text{Numeraire}\ \  p_t=1)\\
\text{high skill}\hspace{4mm}& L_{ct}=h_{ht}\\
\text{low skill}\hspace{4mm}&L_{dt}=h_{lt}
\end{align*}

\section{Solution of tractable model}\label{app:solu}
Define
\begin{align*}
	\tilde{\kappa}:=\ &\frac{(1-\theta_c)(1-\theta_d)\left[\left(\frac{A_c}{A_d}\right)^{(1-\alpha)(1-\varepsilon)}\zeta^{-(\theta_c-\theta_d)(1-\alpha)(1-\varepsilon)}\tilde{\chi}+1\right]}{(1-\theta_d)+(1-\theta_c)\left[\left(\frac{A_c}{A_d}\right)^{(1-\alpha)(1-\varepsilon)}\zeta^{-(\theta_c-\theta_d)(1-\alpha)(1-\varepsilon)}\tilde{\chi}\right]}\\
	\gamma_j:=\ & \left(\frac{\theta_j}{\zeta(1-\theta_j)}\right)^{\theta_j}\\
	z_j:=\ &\theta_j^{\theta_j}(1-\theta_j)^{1-\theta_j} \\
	\chi:=\ &% \frac{(1-\theta_d)(1-\theta_c)}{\theta_c(1-\theta_d)-\theta_d(1-\theta_c)}
	\frac{(1-\theta_d)(1-\theta_c)}{\theta_c-\theta_d}\\
	\tilde{\chi}: =\ &  (\theta_c^{\theta_c}\theta_d^{-\theta_d})^{(1-\alpha) (1-\varepsilon)}(1-\theta_c)^{-\theta_c-(1-\theta_c)(\alpha+\varepsilon(1-\alpha))}(1-\theta_d)^{\theta_d+(1-\theta_d)(\alpha+\varepsilon(1-\alpha))}
\end{align*}
From profit maximisation by labour input good producers follows that the price of the labour input good relative to the skill-specific wage rate is constant. Substituting demand for low skill in the clean sector into the demand for high skill yields

\begin{align*}
	w_{h}^{\frac{1}{1-\theta_c}}w_l^{\frac{1}{\theta_c}}= p_{cL}^\frac{1}{(1-\theta_c)\theta_c}\theta_c^\frac{1}{1-\theta_c}(1-\theta_c)^\frac{1}{\theta_c}.
\end{align*}
Multiplying the left-hand side with $(w_h/w_h)^\frac{1}{\theta_c}$ and
using the FOC governing skill supply $w_h/w_l=\zeta$, it holds that

\begin{align}\label{eq:constant}
%	& \zeta^\frac{-1}{\theta_c}w_h^\frac{1}{(1-\theta_c)\theta_c}= p_{cL}^\frac{1}{(1-\theta_c)\theta_c}\theta_c^\frac{1}{1-\theta_c}(1-\theta_c)^\frac{1}{\theta_c}\nonumber\\
%	\Leftrightarrow\ 
& \frac{p_{cL}}{w_h}= \frac{\zeta^{-(1-\theta_c)}}{z_c}.
\end{align}
%\noindent \tr{Note: this result does not rely on the claim that the labour input good is constant.}

Analogously to \ref{eq:constant}, it follows that
\begin{align}
	\frac{p_{cL}}{w_l}&=\frac{\zeta^{\theta_c}}{z_c}\label{eq:pcl_wl}\\
	\frac{p_{dL}}{w_l}&=\frac{\zeta^{\theta_d}}{z_d}%\ \Leftrightarrow\ w_l= p_{dL}\zeta^{-\theta_d}\theta_d^{\theta_d}(1-\theta_d)^{1-\theta_d}
	\label{eq:pdl_wl}\\
	\frac{p_{dL}}{w_h}&=\frac{\zeta^{-(1-\theta_d)}}{z_d}.
\end{align}
Therefore, the optimal skill input ratios in the labour good production are given by
\begin{align}\label{eq:inputr}
	\frac{l_{hc}}{l_{lc}}=\frac{\theta_c}{\zeta (1-\theta_c)} \hspace{2mm} \text{and}\hspace{3mm} \frac{l_{hd}}{l_{ld}}=\frac{\theta_d}{\zeta (1-\theta_d)}.
\end{align}
This is the common result that  factor shares 
% this refers to (wh lhc)/(wl llc)
are constant over time with a Cobb-Douglas production function. 
Imposing labour market clearing for both skills and optimal skill demand yields 
\begin{align}
	&l_{ld}=\chi\left(\frac{1}{1-\theta_c}h_l-H\right)\label{eq:lld}\\ %\frac{\theta_c}{1-\theta_c}\chi h_l-\chi \zeta h_h,\\
	& l_{lc}=\chi \left(H-\frac{1}{1-\theta_d}h_l\right)\label{eq:llc} %\\
%	with \ & \chi:= \frac{(1-\theta_d)(1-\theta_c)}{\theta_c(1-\theta_d)-\theta_d(1-\theta_c)}=\frac{(1-\theta_d)(1-\theta_c)}{\theta_c-\theta_d}.
	%& l_{hc}= \frac{\theta_c}{\zeta (1-\theta_c)}l_{lc}\\
	%& l_{hd}=\frac{\theta_d}{\zeta (1-\theta_d)}l_{ld}
\end{align}
Labour good supply follows from the labour input good's production function and optimal skill inputs, equations \ref{eq:inputr}, as
\begin{align}
	L_c&=\gamma_cl_{lc}\label{eq:lab_inputc} \\
	L_d&=\gamma_dl_{ld}.\label{eq:lab_inputd}
\end{align}
%\tr{Note that now policy can affect inflation/ relative prices through changes in labour supply---NOPE: cancels}
A relation of the relative price in equilibrium results from equating demand for the labour input goods, equations \ref{eqbm:labc} and \ref{eqbm:labd}, % (which relates the price for the labour input good and the price for the sector-specific final good), 
demand for low skill input by labour producers, equations \ref{eq:pcl_wl} and \ref{eq:pdl_wl}, and free movement of skills: 
\begin{align}\label{eq:price_ratio_labourinput}
	\frac{p_c}{p_d}= \left(\frac{A_d}{A_c}\frac{z_d}{z_c}\zeta^{\theta_c-\theta_d}\right)^{1-\alpha}& \text{(optimality labour input production)}
\end{align}
%where
%\begin{align*}
%	z_j=\theta_j^{\theta_j}(1-\theta_j)^{1-\theta_j}
%\end{align*}
Together with the definition of the aggregate price level and the choice of $Y$ as numeraire, equation \ref{eq:price_ratio_labourinput} determines sector-specific prices as a function of parameters and productivity:
%\begin{align*}
%	p_c= \left(1+\left(\frac{\gamma_c}{\gamma_d}\frac{A_c}{A_d}\frac{l_{lc}}{l_{ld}}\right)^{\frac{(1-\alpha)(1-\varepsilon)}{\alpha+\varepsilon(1-\alpha)}}\right)^{-\frac{1}{1-\varepsilon}}.
%\end{align*}
%Substituting equation \ref{eq:lldllc} gives the price of the clean good in equilibrium as
\begin{align}
	p_c%& = \frac{1}{\left(1+\left(\frac{A_c}{A_d}\right)^{(1-\alpha)(1-\varepsilon)}\left(\frac{z_c}{z_d}\right)^{(1-\alpha)(1-\varepsilon)}\zeta^{-(\theta_c-\theta_d)(1-\alpha)(1-\varepsilon)}\right)^{\frac{1}{1-\varepsilon}}}\\
	&= \left(\frac{\left(A_dz_d\zeta^{\theta_c}\right)^{(1-\alpha)(1-\varepsilon)}}{\left(A_dz_d\zeta^{\theta_c}\right)^{(1-\alpha)(1-\varepsilon)}+\left(A_cz_c\zeta^{\theta_d}\right)^{(1-\alpha)(1-\varepsilon)}}\right)^{\frac{1}{1-\varepsilon}}\label{eq:eq_pc}\\
%\end{align}
%%and using equation \ref{eq:price_ratio_labourinput} yields
%\begin{align}
	p_d%& =\frac{1}{\left(\left(\frac{A_d}{A_c}\right)^{(1-\alpha)(1-\varepsilon)}\left(\frac{z_d}{z_c}\right)^{(1-\alpha)(1-\varepsilon)}\zeta^{(\theta_c-\theta_d)(1-\alpha)(1-\varepsilon)}+1\right)^{\frac{1}{1-\varepsilon}}}\\
	&= \left(\frac{\left(A_cz_c\zeta^{\theta_d}\right)^{(1-\alpha)(1-\varepsilon)}}{\left(A_dz_d\zeta^{\theta_c}\right)^{(1-\alpha)(1-\varepsilon)}+\left(A_cz_c\zeta^{\theta_d}\right)^{(1-\alpha)(1-\varepsilon)}}\right)^{\frac{1}{1-\varepsilon}} \label{eq:eq_pd}
\end{align}

\paragraph{Skill allocation}
To solve for the equilibrium ratio of skill inputs in the clean and dirty sector, I substitute labour input, equations \ref{eq:lab_inputc} and \ref{eq:lab_inputd}, in the sector-specific production functions, \ref{eqbm:outputc} and \ref{eqbm:outputd}. Exploiting demand for sector goods, $Y_d=\left(\frac{p_c}{p_d}\right)^\varepsilon Y_c$, and the price ratio in equilibrium pinned down by equation \ref{eq:price_ratio_labourinput} yields
%\begin{align}\label{eq:price_ratio_output}

%\end{align}
%Substituting equation \ref{eq:price_ratio_output} into equation \ref{eq:price_ratio_labourinput} determines the equilibrium ratio of low-skill input in the dirty to the clean sector: 
\begin{align}
	\frac{p_c}{p_d} =&\left(\frac{\gamma_d}{\gamma_c}\frac{A_d}{A_c}\frac{l_{ld}}{l_{lc}}\right)^{\frac{1-\alpha}{\alpha+\varepsilon(1-\alpha)}}\\ %& \text{(Demand for sector-specific goods)}\\
\Leftrightarrow\ 	\frac{l_{ld}}{l_{lc}}=&%\left(\frac{A_c}{A_d}\right)^{(1-\alpha)(1-\varepsilon)}\frac{\gamma_c}{\gamma_d}\left(\frac{z_d}{z_c}\right)^{\alpha+\varepsilon(1-\alpha)}\zeta^{(\theta_c-\theta_d)(\alpha+\varepsilon(1-\alpha))}\nonumber\\	=&
	\left(\frac{A_c}{A_d}\right)^{(1-\alpha)(1-\varepsilon)}\left(\zeta^{\theta_c-\theta_d}\frac{\gamma_c}{\gamma_d}\frac{z_d}{z_c}\right)^{\alpha+\varepsilon(1-\alpha)}\label{eq:lldllc}%\\
	%	\text{where}&\\
	%	\tilde{\chi}= &\  (\theta_c^{\theta_c}\theta_d^{-\theta_d})^{(1-\alpha) (1-\varepsilon)}(1-\theta_c)^{-\theta_c-(1-\theta_c)(\alpha+\varepsilon(1-\alpha))}(1-\theta_d)^{\theta_d+(1-\theta_d)(\alpha+\varepsilon(1-\alpha))}\nonumber
\end{align}

\paragraph{Skill supply}
Using equations \ref{eq:lld}, \ref{eq:llc}, and \ref{eq:lldllc} one can solve for $h_l$ as a function of total skill supply in equilibrium
\begin{align}
	h_l= \underbrace{\frac{(1-\theta_c)(1-\theta_d)\left[\left(\frac{A_c}{A_d}\right)^{(1-\alpha)(1-\varepsilon)}\zeta^{-(\theta_c-\theta_d)(1-\alpha)(1-\varepsilon)}\tilde{\chi}+1\right]}{(1-\theta_d)+(1-\theta_c)\left[\left(\frac{A_c}{A_d}\right)^{(1-\alpha)(1-\varepsilon)}\zeta^{-(\theta_c-\theta_d)(1-\alpha)(1-\varepsilon)}\tilde{\chi}\right]}}_{:=\tilde{\kappa}}H
\end{align}
Now, one can solve for labour input and sector-specific output as a function of tax progessivity  in equilibrium. 
$L_c$ and $L_d$ are
\begin{align}
	L_c&= \gamma_c \chi \left(1-\frac{\tilde{\kappa}}{1-\theta_d}\right)H\\
	L_d&= \gamma_d \chi \left(\frac{\tilde{\kappa}}{1-\theta_c}-1\right)H=\zeta^{-\theta_d}z_dp_d^{1-\varepsilon}H
\end{align}
This solves the model, since, in equilibrium,  $H$ is a function of parameters and policy variables only. 
Replacing dirty labour input and machines in dirty production leads to the expression for dirty output growth used in the text. 
\begin{comment}
\paragraph{Summary of equilibrium equations}
\begin{align*}
H=\ & (1-\tau_l)^{\frac{1}{1+\sigma}}\\
h_l=\ & \tilde{\kappa}H\\
L_c=\ & \gamma_c \chi \left(1-\frac{\tilde{\kappa}}{1-\theta_d}\right)H
\\
L_d=\ & \gamma_d \chi \left(\frac{\tilde{\kappa}}{1-\theta_c}-1\right)H%= \zeta^{-\theta_d}z_d\frac{\left(A_cz_c\zeta^{\theta_d}\right)^{(1-\alpha)(1-\varepsilon)}}{\left(A_dz_d\zeta^{\theta_c}\right)^{(1-\alpha)(1-\varepsilon)}+\left(A_cz_c\zeta^{\theta_d}\right)^{(1-\alpha)(1-\varepsilon)}}H
=\zeta^{-\theta_d}z_dp_d^{1-\varepsilon}H\\
p_d=\ &\left(\frac{\left(A_cz_c\zeta^{\theta_d}\right)^{(1-\alpha)(1-\varepsilon)}}{\left(A_dz_d\zeta^{\theta_c}\right)^{(1-\alpha)(1-\varepsilon)}+\left(A_cz_c\zeta^{\theta_d}\right)^{(1-\alpha)(1-\varepsilon)}}\right)^{\frac{1}{1-\varepsilon}} \\
p_c=\ & \left(\frac{\left(A_dz_d\zeta^{\theta_c}\right)^{(1-\alpha)(1-\varepsilon)}}{\left(A_dz_d\zeta^{\theta_c}\right)^{(1-\alpha)(1-\varepsilon)}+\left(A_cz_c\zeta^{\theta_d}\right)^{(1-\alpha)(1-\varepsilon)}}\right)^{\frac{1}{1-\varepsilon}}
\end{align*}

content...
\end{comment}
\section{Balanced Growth Path}

The model features structural transformation stemming from price effects (\cite{Ngai2007StructuralGrowth}, Baumol (1967)), since heterogeneous growth rates result in relative price changes over time. %A shown by \cite{Ngai2007StructuralGrowth}, the model features a balanced-growth path with certain parameter values: 
For certain parameter values the model exhibits a generalised balanced growth path\footnote{\ 
In contrast to a balanced growth path, which is commonly defined by constant growth in all variables, a GBGP is less strict and certain variables are allowed to grow at non-constant rates. The literature on structural transformation commonly reverts to this concept as transitions across sectors are essential to this literature.}.
\cite{Ngai2007StructuralGrowth} show that with goods being complements, employment shares shift to sectors with lower TFP growth; eventually, all labour is in the sector with the lowest TFP. In the present model, this is the clean sector. 

\section{Model Isomorphic to model with investment and rented capital}
The model is isomorphic to a model with (instantaneously productive) investment and full depreciation: 
\begin{align*}
I_t&=\psi(x_{dt}+x_{ct})\\
(LOM capital) \ K_t&=I_t= I_{ct}+I_{dt}
\end{align*}
That is, the capital good is produced by the following technology
\begin{align*}
x_{ijt}=\frac{I_{ijt}}{\psi}
\end{align*}
Machine producing firms rent the investment good, $I_{jt}$ and pay the real rate. They maximise over the choice of investment, i.e. capital, to borrow:
\begin{align*}
\underset{I_{ijt}}{\max}\hspace{2mm}p_{ijt}x_{ijt}-r_tI_{ijt}
\end{align*}
Profit maximisation of machine producing firms yields
\begin{align*}
\frac{p_{ijt}}{\psi}=r_t
\end{align*}
Free movement of capital and homogeneity of production costs imply that machine prices are equal across firms and sectors. 

Imposing market clearing for investment, $I_t=\int_{0}^{1}I_{idt}di+\int_{0}^{1}I_{ict}di$, and market clearing for machines yield a condition for the real rate in equilibrium
\begin{align*}
r_t=\alpha \psi^{-\alpha}\left(\frac{p_{dt}^{\frac{1}{1-\alpha}}A_{dt}L_{dt}+p_{ct}^{\frac{1}{1-\alpha}}A_{ct}L_{ct}}{K_t}\right)^{1-\alpha}
\end{align*}

\section{Results}
\begin{figure}[h!!]
	\centering
	\caption{Business as usual versus laissez-faire, substitutes, additional variables }\label{fig:onlyBAU_add}
	
	\begin{minipage}[]{0.32\textwidth}
		\centering{\footnotesize{(a) Clean output, $y_c$ }}
		%	\captionsetup{width=.45\linewidth}
		\includegraphics[width=1\textwidth]{../../codding_model/Own/figures/Rep_agent/staticBAU_LF_separate_yc_periods59_eppsilon4.00_zeta1.40_Ad08_Ac04_thetac0.70_thetad0.56_HetGrowth1_tauul0.181_util0_withtarget0_lgd0.png}
	\end{minipage}
	\begin{minipage}[]{0.32\textwidth}
		\centering{\footnotesize{(b) Dirty output, $y_d$}}
		%	\captionsetup{width=.45\linewidth}
		\includegraphics[width=1\textwidth]{../../codding_model/Own/figures/Rep_agent/staticBAU_LF_separate_yd_periods59_eppsilon4.00_zeta1.40_Ad08_Ac04_thetac0.70_thetad0.56_HetGrowth1_tauul0.181_util0_withtarget0_lgd0.png}
	\end{minipage}
	\begin{minipage}[]{0.32\textwidth}
		\centering{\footnotesize{(c) Labour input clean, $L_c$ }}
		%	\captionsetup{width=.45\linewidth}
		\includegraphics[width=1\textwidth]{../../codding_model/Own/figures/Rep_agent/staticBAU_LF_separate_Lc_periods59_eppsilon4.00_zeta1.40_Ad08_Ac04_thetac0.70_thetad0.56_HetGrowth1_tauul0.181_util0_withtarget0_lgd0.png}
	\end{minipage}
	\begin{minipage}[]{0.32\textwidth}
		\centering{\footnotesize{(d) Labour input dirty, $L_d$ }}
		%	\captionsetup{width=.45\linewidth}
		\includegraphics[width=1\textwidth]{../../codding_model/Own/figures/Rep_agent/staticBAU_LF_separate_Ld_periods59_eppsilon4.00_zeta1.40_Ad08_Ac04_thetac0.70_thetad0.56_HetGrowth1_tauul0.181_util0_withtarget0_lgd0.png}
	\end{minipage}
\begin{minipage}[]{0.32\textwidth}
	\centering{\footnotesize{(e) Machines clean, $x_c$}}
	%	\captionsetup{width=.45\linewidth}
	\includegraphics[width=1\textwidth]{../../codding_model/Own/figures/Rep_agent/staticBAU_LF_separate_xc_periods59_eppsilon4.00_zeta1.40_Ad08_Ac04_thetac0.70_thetad0.56_HetGrowth1_tauul0.181_util0_withtarget0_lgd0.png}
\end{minipage}
	\begin{minipage}[]{0.32\textwidth}
		\centering{\footnotesize{(f) Machines dirty, $x_d$}}
		%	\captionsetup{width=.45\linewidth}
		\includegraphics[width=1\textwidth]{../../codding_model/Own/figures/Rep_agent/staticBAU_LF_separate_xd_periods59_eppsilon4.00_zeta1.40_Ad08_Ac04_thetac0.70_thetad0.56_HetGrowth1_tauul0.181_util0_withtarget0_lgd0.png}
	\end{minipage}
\end{figure}

\begin{figure}[h!!]
	\centering
	\caption{Business as usual versus laissez-faire, complements, additional variables }\label{fig:onlyBAU_comp_add}
		\begin{minipage}[]{0.32\textwidth}
		\centering{\footnotesize{(a) Clean output }}
		%	\captionsetup{width=.45\linewidth}
		\includegraphics[width=1\textwidth]{../../codding_model/Own/figures/Rep_agent/staticBAU_LF_separate_yc_periods59_eppsilon0.40_zeta1.40_Ad08_Ac04_thetac0.70_thetad0.56_HetGrowth1_tauul0.181_util0_withtarget0_lgd0.png}
	\end{minipage}
	\begin{minipage}[]{0.32\textwidth}
		\centering{\footnotesize{(b) Dirty output }}
		%	\captionsetup{width=.45\linewidth}
		\includegraphics[width=1\textwidth]{../../codding_model/Own/figures/Rep_agent/staticBAU_LF_separate_yd_periods59_eppsilon0.40_zeta1.40_Ad08_Ac04_thetac0.70_thetad0.56_HetGrowth1_tauul0.181_util0_withtarget0_lgd0.png}
	\end{minipage}
	\begin{minipage}[]{0.32\textwidth}
		\centering{\footnotesize{(c) Labour input clean, $L_c$ }}
		%	\captionsetup{width=.45\linewidth}
		\includegraphics[width=1\textwidth]{../../codding_model/Own/figures/Rep_agent/staticBAU_LF_separate_Lc_periods59_eppsilon0.40_zeta1.40_Ad08_Ac04_thetac0.70_thetad0.56_HetGrowth1_tauul0.181_util0_withtarget0_lgd0.png}
	\end{minipage}
	\begin{minipage}[]{0.32\textwidth}
		\centering{\footnotesize{(d) Labour input dirty, $L_d$ }}
		%	\captionsetup{width=.45\linewidth}
		\includegraphics[width=1\textwidth]{../../codding_model/Own/figures/Rep_agent/staticBAU_LF_separate_Ld_periods59_eppsilon0.40_zeta1.40_Ad08_Ac04_thetac0.70_thetad0.56_HetGrowth1_tauul0.181_util0_withtarget0_lgd0.png}
	\end{minipage}
	\begin{minipage}[]{0.32\textwidth}
		\centering{\footnotesize{(e) Machines clean, $x_c$}}
		%	\captionsetup{width=.45\linewidth}
		\includegraphics[width=1\textwidth]{../../codding_model/Own/figures/Rep_agent/staticBAU_LF_separate_xc_periods59_eppsilon0.40_zeta1.40_Ad08_Ac04_thetac0.70_thetad0.56_HetGrowth1_tauul0.181_util0_withtarget0_lgd0.png}
	\end{minipage}
	\begin{minipage}[]{0.32\textwidth}
		\centering{\footnotesize{(f) Machines dirty, $x_d$}}
		%	\captionsetup{width=.45\linewidth}
		\includegraphics[width=1\textwidth]{../../codding_model/Own/figures/Rep_agent/staticBAU_LF_separate_xd_periods59_eppsilon0.40_zeta1.40_Ad08_Ac04_thetac0.70_thetad0.56_HetGrowth1_tauul0.181_util0_withtarget0_lgd0.png}
	\end{minipage}
\end{figure}

\begin{figure}[h!!]
	\centering
	\caption{Optimal allocation with emission target, complements, additional variables }\label{fig:optallo_comp_onlyR_add}
	\begin{minipage}[]{0.32\textwidth}
	\centering{\footnotesize{(a) Labour input clean, $L_c$ }}
	%	\captionsetup{width=.45\linewidth}
	\includegraphics[width=1\textwidth]{../../codding_model/Own/figures/Rep_agent/staticonlyRam_separate_Lc_periods59_eppsilon0.40_zeta1.40_Ad08_Ac04_thetac0.70_thetad0.56_HetGrowth1_tauul0.181_util0_withtarget1_lgd0.png}
\end{minipage}
	\begin{minipage}[]{0.32\textwidth}
		\centering{\footnotesize{(b) Labour input dirty, $L_d$ }}
		%	\captionsetup{width=.45\linewidth}
		\includegraphics[width=1\textwidth]{../../codding_model/Own/figures/Rep_agent/staticonlyRam_separate_Ld_periods59_eppsilon0.40_zeta1.40_Ad08_Ac04_thetac0.70_thetad0.56_HetGrowth1_tauul0.181_util0_withtarget1_lgd0.png}
	\end{minipage}
	\begin{minipage}[]{0.32\textwidth}
		\centering{\footnotesize{(c) Machines clean, $x_c$}}
		%	\captionsetup{width=.45\linewidth}
		\includegraphics[width=1\textwidth]{../../codding_model/Own/figures/Rep_agent/staticonlyRam_separate_xc_periods59_eppsilon0.40_zeta1.40_Ad08_Ac04_thetac0.70_thetad0.56_HetGrowth1_tauul0.181_util0_withtarget1_lgd0.png}
	\end{minipage}
\begin{minipage}[]{0.32\textwidth}
\centering{\footnotesize{(d) Machines dirty, $x_d$}}
%	\captionsetup{width=.45\linewidth}
\includegraphics[width=1\textwidth]{../../codding_model/Own/figures/Rep_agent/staticonlyRam_separate_xd_periods59_eppsilon0.40_zeta1.40_Ad08_Ac04_thetac0.70_thetad0.56_HetGrowth1_tauul0.181_util0_withtarget1_lgd0.png}
\end{minipage}
\begin{minipage}[]{0.32\textwidth}
	\centering{\footnotesize{(e) Price clean good, $p_c$}}
	%	\captionsetup{width=.45\linewidth}
	\includegraphics[width=1\textwidth]{../../codding_model/Own/figures/Rep_agent/staticonlyRam_separate_pc_periods59_eppsilon0.40_zeta1.40_Ad08_Ac04_thetac0.70_thetad0.56_HetGrowth1_tauul0.181_util0_withtarget1_lgd0.png}
\end{minipage}
	\begin{minipage}[]{0.32\textwidth}
		\centering{\footnotesize{(f) Price dirty good, $p_d$}}
		%	\captionsetup{width=.45\linewidth}
		\includegraphics[width=1\textwidth]{../../codding_model/Own/figures/Rep_agent/staticonlyRam_separate_pd_periods59_eppsilon0.40_zeta1.40_Ad08_Ac04_thetac0.70_thetad0.56_HetGrowth1_tauul0.181_util0_withtarget1_lgd0.png}
	\end{minipage}
	\begin{minipage}[]{0.32\textwidth}
	\centering{\footnotesize{(g) $\lambda$}}
	%	\captionsetup{width=.45\linewidth}
	\includegraphics[width=1\textwidth]{../../codding_model/Own/figures/Rep_agent/staticonlyRam_separate_lambdaa_periods59_eppsilon0.40_zeta1.40_Ad08_Ac04_thetac0.70_thetad0.56_HetGrowth1_tauul0.181_util0_withtarget1_lgd0.png}
\end{minipage}
\end{figure}



\section{Skill supply}
\paragraph{Effect of $\tau_l$ on skill investment}
From the definition of $H$ it has to hold that 
\begin{align}
&1=\frac{dh_l}{dH}+\zeta \frac{dh_h}{dH}\label{eq:ident} \\
\Leftrightarrow\ & \frac{dh_h}{dH}=\frac{1-\frac{dh_l}{dH}}{\zeta}.\label{eq:resp}
\end{align}
Using this equation, one can show that high skill supply is relatively more responsive to changes in total effective hours worked, i.e.,  $\frac{dh_h}{dH}>\frac{dh_l}{dH}$, if one excludes the case that high skill supply reduces as effective hours increase.\footnote{\ Proof: Suppose   $\frac{dh_h}{dH}>0$. Now, assume by contradiction that low skill supply is relatively more responsive. Hence, $\frac{dh_h}{dH}<\frac{dh_l}{dH}$. Using equation \ref{eq:resp}, one gets that $\frac{dh_l}{dH}>1+\zeta$. Replacing this inequality in the identity \ref{eq:ident}, it follows that $0>\zeta[1+\frac{dh_h}{dH}]$. Since $\zeta>1$ by assumption, it has to hold that $\frac{dh_h}{dH}<-1$ which contradicts the premise that $\frac{dh_h}{dH}>0$. } Thus, as the household reduces total effective hours supplied, the reduction in high skilled hours is higher. \tr{This should be due to the marginal utility from less high skill is higher than from less low skill hours.} This should show up in general equilibrium effects... but relative wages are fixed. 


\section{Calculations, partially wrong}
\noindent\rule[1ex]{\textwidth}{1pt}

\paragraph{Progressivity and emission targets}\tr{This result also rests on wrong premisses, but needs to be replicated!}
The constraint on emissions in the government's objective function implies that $Y_{dt}=\frac{\delta}{\kappa}$, thus, $(1+g_{ydt})=1$, for all time periods starting from 2050, $t\geq 30$. 

From the dirty sector's production function and equation \ref{eq:inf_d} we have that
\begin{align}
&\frac{Y_{d}'}{Y_d}=(1+\pi_d)^{\frac{\alpha}{1-\alpha}}(1+\upsilon_{d})\label{eq:gyd}\\
\Leftrightarrow\ &(1+\upsilon_{d})^{\frac{(1-\alpha)(1-\tau_l-\varepsilon)}{(1-\tau_l)-(\varepsilon(1-\alpha)+\alpha)}}=1\label{eq:def_taul}
\end{align}
The inflation rate in equation \ref{eq:gyd} captures the role of machine demand by the dirty sector. When the price at which dirty firms can sell their output is high, they demand more machines. A positive inflation, therefore, implies a rise in dirty output.

At the same time, a rise in the dirty good's price reduces demand. This counteracting mechanism is accounted for in equation \ref{eq:def_taul}. 
As will be shown below, this mechanism ensures that the government can target dirty sector production through tax progressivity. 

First, I establish an optimal policy result. Assume that the government cannot set the growth rate in the dirty sector, then equation \ref{eq:def_taul} defines $\tau_l$ on a balanced growth path.

\begin{prop}[Optimal tax progressivity]
	Assume growth of the dirty technology, $\upsilon_{d}$, is exogenously determined. 
	Then, to comply with the Paris Agreement, the government has to set the tax progressivity parameter, $\tau_l$, to $\tau^*_l=1-\varepsilon$ for $\varepsilon\neq 1$ (as otherwise the exponent in \ref{eq:def_taul} is not defined under the optimal tax rate.).
	When goods are complements, the optimal tax system is progressive. If goods are substitutes, the optimal tax system is regressive.
\end{prop}

The intuition is, that by choosing tax progressivity, the government affects price inflation in the dirty sector; compare equation \ref{eq:inf_d}. 
The result implies that inflation in the dirty sector under the optimal policy is negative when there is positive growth in dirty technology. The demand for machines has to decline by the same rate as technology growths for dirty output to be constant.  

Can this be an equilibrium as the price for dirty products declines?

How does tax progressivity affect inflation? 
First note that at a flat tax, the inflation rate is independent of the sector-specific technological growth rate. This is due to offsetting mechanisms.\tr{Continue}


\tr{Start from effect on HH:} 
(1) A rise in $\tau_l$ reduces disposable income and aggregate demand falls. This is a mechanical result from a higher tax rate, and a reduction in aggregate hours supplied.  
(1) For the dirty sector to demand labour, the costs of the labour input good has to balance its marginal product which positively depends on technological progress and the sector specific price. 

\noindent\rule[1ex]{\textwidth}{1pt} 


\noindent\rule[1ex]{\textwidth}{1pt}

\textcolor{blue}{below is wrong since the aggregate price level is not constant when G is disposed off.}
Define sector-specific inflation as: $1+\pi_{j}=\frac{p'_j}{p_j}$.
Using the definition of the aggregate price level, final good production, and optimality conditions in the clean sector, one can show that 
\begin{align}\label{eq:agg_supply}
\frac{Y'}{Y}= (1+\pi_c)^{\frac{\varepsilon(1-\alpha)+\alpha}{1-\alpha}}(1+\upsilon_{c}), \hspace{3mm} \text{(Supply side)}
\end{align}
since $\frac{L_{c}'}{L_c}=1$.
Using goods market clearance, the budget condition, and the FOC for total skill supply, it follows that 
\begin{align}\label{eq:agg_demand}
\frac{Y'}{Y}= \left(\frac{w'_h}{w_h}\right)^{1-\tau_l}. \hspace{3mm} \text{(Demand side)}\tr{\text{wrong, misses gov expenditures... or let lambdaa adjust, and machine production}}
\end{align}
Demand for the labour input good implies that 
\begin{align}\label{eq:labour income}
\frac{p'_{cL}}{p_{cL}}= (1+\pi_c)^\frac{1}{1-\alpha}(1+\upsilon_{c})
\end{align}
(independent of growth in $L_c$).

Multiplying both sides with $\left(\frac{w_h'}{w_h}\right)^{-1}$, using equation \ref{eq:agg_growth}, and that $\frac{p_{cL}}{w_h}$ is constant, it follows that 

\begin{align}
\frac{\frac{p'_{cL}}{w'_h}}{\frac{p_{cL}}{w_h}}= (1+\pi_c)^\frac{1}{1-\alpha}(1+\upsilon_{c})\left(\frac{Y'}{Y}\right)^{-\frac{1}{1-\tau_l}}=1.
\end{align}

Above equation determines inflation in the clean sector:
\begin{align}\label{eq:inf_c}
1+\pi_c=(1+\upsilon_{c})^{\frac{\tau_l(1-\alpha)}{(1-\tau_l)-\varepsilon(1-\alpha)-\alpha}}.
\end{align}
By symmetry of (i) how goods enter the production fo the final good and of (ii) sectors, it also holds that 
\begin{align}\label{eq:inf_d}
1+\pi_d=(1+\upsilon_{d})^{\frac{\tau_l(1-\alpha)}{(1-\tau_l)-\varepsilon(1-\alpha)-\alpha}}.
\end{align}


\noindent \textbf{(Aggregate output result, less relevant for main story)}

Hence, 
\begin{align}\label{eq:agg_growth}
(1+g_y)=\frac{Y'}{Y}=(1+\upsilon_{c})^\frac{(1-\tau_l)[1-(\varepsilon(1-\alpha)+\alpha)]}{(1-\tau_l)-(\varepsilon(1-\alpha)+\alpha)}.
\end{align}
Equation \ref{eq:agg_growth} implies the following proposition:
\begin{prop}[aggregate growth]
	\textit{For a proportional tax system, $\tau_l=0$, aggregate growth equals growth in the clean sector. 
		When the tax system is progressive\footnote{\ In the sense defined in \cite{Heathcote2017OptimalFramework}.}, $\tau_l>0$, then aggregate growth exceeds technology growth in the clean sector. When the tax rate is regressive, $\tau_l<0$, aggregate growth is smaller than technology growth in the clean sector. }
\end{prop}
\tr{Have to understand why.} 
With a flat tax system there is no inflation in the clean sector; compare equation \ref{eq:inf_c}. When the tax system is progressive, ...

\paragraph{Proof: labour input good constant}
%\textit{Check that the labour input good is constant:} 

First note that $\frac{l_{hc}}{l_{lc}}$ is constant over time. 
From the FOC governing high skill demand in the clean sector and equation \ref{eq:constant} we have:

\begin{align*}
\frac{l_{hc}}{l_{lc}}=\left(\frac{p_{cL}}{w_h}\theta_c\right)^{\frac{1}{1-\theta_c}}= constant.
\end{align*}

Substitution into the production function of the clean labour input good yields

\begin{align*}
\frac{L'_c}{L_c}=\frac{l_{lc}'}{l_{lc}}.
\end{align*}

\tr{To be continued.}

\textbf{\tr{To be shown next:  How $\tau_l$ affects (1) skill supply (level) and (2) externality. }}
\\

\paragraph{Overview BGP compatible growth rates}
\begin{align}
\frac{Y'}{Y}\\
\frac{w_h'}{w_h}=\frac{w_l'}{w_l}\\
tbc
\end{align}
\textbf{Conditions for BGP to exist}
Next to assuming no transition of labour input goods across sectors, a joint condition on tax progressivity and substitutability of sector goods ensures that output growth in both sectors is positive which has to be the case as otherwise production of one good tends to zero which cannot be an equilibrium when goods are no perfect substitutes.
Hence, on a BGP it has to hold that 
\begin{align*}
\frac{(1-\alpha)(1-\tau_l-\varepsilon)}{(1-\tau_l)-(\varepsilon(1-\alpha)+\alpha)}>0.
\end{align*}
There are two possible ranges of parameter values reads
\begin{align*}
\text{either}\\	&(1-\tau_l)>\varepsilon\hspace{4mm}&\text{if goods are substitutes} ;\\ \text{and}\ \ & (1-\tau_l)>\varepsilon(1-\alpha)+\alpha\hspace{4mm}&\text{if goods are complements}\\
\text{or}\\	&(1-\tau_l)<\varepsilon\hspace{4mm}&\text{if goods are complements} ;\\ \text{and}\ \ & (1-\tau_l)<\varepsilon(1-\alpha)+\alpha\hspace{4mm}&\text{if goods are substitutes}
\end{align*}
Focus on the case that goods are substitutes. Then the condition in the \textit{either}-statement prevents the government from choosing a progressive tax system, since $\tau_l<1-\varepsilon<0$.
Analogously, when goods are complements, the \textit{or}-statement excludes regressive tax systems.
I, therefore, ensure that when goods are complements, it holds that $\tau_l<(1-\alpha)(1-\varepsilon)$. When they are substitutes, it holds that $\tau_l>(1-\alpha)(1-\varepsilon)$.


\tr{Remaining problem: prices are not constant on BGP with fixed growth rates...Look at literature on positive trend inflation...}

\begin{comment}
\textbf{Below wrong Because wrong hl used}
From here,  equilibrium conditions determine prices $p_{dL}, p_{cL}$. Using \ref{eq:constant} skill wages follow. Together with the FOC on hours supply, wages determine aggregate demand. Imposing goods market clearing and using equations \ref{eq:lab_inputc} and \ref{eq:lab_inputd}, determines low skill hour demand in equilibrium.

\begin{align*}
h_l=\left( \frac{1}{\left(\frac{\alpha}{\psi}\right)^{\frac{\alpha}{1-\alpha}}\left[\left(p_c^\frac{\alpha}{1-\alpha}\chi_c A_c\right)^\frac{\varepsilon-1}{\varepsilon}+\left(p_d^\frac{\alpha}{1-\alpha}\chi_d A_d\right)^\frac{\varepsilon-1}{\varepsilon}\right]^\frac{\varepsilon}{\varepsilon-1}}\right)\ \lambda \left(H w_l\right)^{1-\tau_l}.
\end{align*}

Knowing $h_l$, the variables $L_c, \ L_d, \ h_h, \ l_{lc}, l_{ld}, l_{hc}, l_{hd}$ follow. 

Output of the clean and dirty sector read
\begin{align}
%Y_d& =  \frac{\chi_d A_d}{\left[\left(\left(\chi_d A_d\right)^\frac{\alpha}{\alpha+\varepsilon(1-\alpha)}\left(\chi_c A_c\right)^\frac{\varepsilon(1-\alpha)}{\alpha+\varepsilon(1-\alpha)}\right)^\frac{\varepsilon-1}{\varepsilon}+\left(\chi_d A_d\right)^\frac{\varepsilon-1}{\varepsilon}\right]^\frac{\varepsilon}{\varepsilon-1}} \lambda (H w_l)^{1-\tau_l}\\
&Y_d = \left(\frac{1}{\left(\frac{\chi_c A_c}{\chi_d A_d}\right)^{\frac{(\varepsilon-1)(1-\alpha)}{\alpha+\varepsilon(1-\alpha)}}+1}\right)^\frac{\varepsilon}{\varepsilon-1}\lambda (H w_l)^{1-\tau_l}\\
& Y_c= \left(\frac{1}{1+\left(\frac{\chi_d A_d}{\chi_c A_c}\right)^{\frac{(\varepsilon-1)(1-\alpha)}{\alpha+\varepsilon(1-\alpha)}}}\right)^\frac{\varepsilon}{\varepsilon-1}\lambda (H w_l)^{1-\tau_l}.
\end{align}

The government can affect dirty production by lowering aggregate demand. Note that $\chi_c,\ \chi_d$ are functions of the disutility from high skill labour supply, $\zeta$. As a result, the elasticity of diryt and clean output to tax progressivity is asymmetric.

\end{comment}