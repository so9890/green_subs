\documentclass[12pt]{article}
\usepackage[utf8]{inputenc}
\usepackage{xcolor}
\usepackage{graphicx}
\usepackage{listings}
\usepackage{epstopdf}
\usepackage{etoc}
\usepackage{pdfpages}
\usepackage[capposition=top]{floatrow}
\usepackage{pdflscape} % landsacpe package
% set font to times
%\usepackage{mathptmx} % times!!! 
%\usepackage[T1]{fontenc}
\usepackage{amsmath}
\usepackage{soul}
\usepackage[left=2.5cm, right=2.5cm, top=2.5cm, bottom =2.5cm]{geometry}
\usepackage{natbib}
%\usepackage[natbibapa]{apacite}
%\usepackage{apacite}
%\bibliographystyle{apacite}
\bibliographystyle{apa}
%\renewcommand{\footnotesize}{\fontsize{10pt}{11pt}\selectfont}
\usepackage[onehalfspacing]{setspace}
\usepackage{listings}
\renewcommand{\figurename}{\textbf{Figure}}
\renewcommand{\hat}{\widehat}
\usepackage[bf]{caption}
\usepackage{tikz}
%\begin{comment}
%\usepackage[headsepline,footsepline]{scrlayer-scrpage} % has to come before package!!! otherwise option clash
%\usepackage{scrlayer-scrpage}
%\pagestyle{scrheadings} % kopfzeile/ fußzeile
%\clearpairofpagestyles
%\ohead{}
%\ihead{\textit{Redistribution, Demand and  Sustainable Production}}
%\cfoot{\thepage}
%\pagestyle{plain} % comment this one to have header
%\end{comment}
\allowdisplaybreaks
\usepackage{comment}
 \usepackage{siunitx}
  \usepackage{textcomp}
\definecolor{sonja}{cmyk}{0.9,0,0.3,0}
%\definecolor{purple}{model}{color-spec}
\usepackage{amssymb}
\newcommand{\ar}{$\Rightarrow$ \ }
\newcommand{\frp}[2]{\frac{\partial{#1}}{\partial{#2}}}
\newcommand{\tr}[1]{\textcolor{red}{#1}}
\newcommand{\vlt}[1]{\textcolor{violet}{#1}}
\newcommand{\bl}[1]{\textcolor{blue}{#1}}
\newcommand{\sn}[1]{\textcolor{sonja}{#1}}
%%% TIKZS
\usepackage{tikz}
\usetikzlibrary{mindmap,trees}
\usetikzlibrary{backgrounds}
\tikzstyle{every edge}=  [fill=orange]  
\usetikzlibrary{tikzmark}
\usetikzlibrary{decorations.markings}
\usepackage{tikz-cd}
\usetikzlibrary{arrows,calc,fit}
\tikzset{mainbox/.style={draw=sonja, text=black, fill=white, ellipse, rounded corners, thick, node distance=5em, text width=8em, text centered, minimum height=3.5em}}
\tikzset{mainboxbig/.style={draw=sonja, text=black, fill=white, ellipse, rounded corners, thick, node distance=5em, text width=13em, text centered, minimum height=3.5em}}
\tikzset{dummybox/.style={draw=none, text=black , rectangle, rounded corners, thick, node distance=4em, text width=20em, text centered, minimum height=3.5em}}
\tikzset{box/.style={draw , rectangle, rounded corners, thick, node distance=7em, text width=8em, text centered, minimum height=3.5em}}
\tikzset{container/.style={draw, rectangle, dashed, inner sep=2em}}
\tikzset{line/.style={draw, very thick, -latex'}}
\tikzset{    pil/.style={
		->,
		thick,
		shorten <=2pt,
		shorten >=2pt,}}
	
% other stuff
\newcommand{\innermid}{\nonscript\;\delimsize\vert\nonscript\;}
\newcommand{\activatebar}{%
	\begingroup\lccode`\~=`\|
	\lowercase{\endgroup\let~}\innermid 
	\mathcode`|=\string"8000
}
%\usepackage{biblatex}
%\addbibresource{bib_mt.bib}
\usepackage{ulem}
\title{The Role of Fiscal Policies in Meeting Climate Targets}
%\title{The Environment, Inequality, and Growth\\ \small{ optimal fiscal policy in an endogenous growth model with inequality and emission targets}}
\date{Sonja Dobkowitz\\ Bonn Graduate School of Economics\\ %University of Bonn\\
\vspace{1mm}
Preliminary and incomplete\\
%First version: January 9, 2022\\
This version: \today }
\usepackage{graphicx,caption}
\usepackage{hyperref}
\PassOptionsToPackage{hyphens}{url}\usepackage{hyperref}
\usepackage{minitoc}
\setcounter{secttocdepth}{5}
\usetikzlibrary{shapes.geometric}

% for tabular

%\usepackage{array}
\usepackage{makecell}
\usepackage{multirow}
\usepackage{bigdelim}

%propositions etc
\newtheorem{prop}{Proposition}
\newtheorem{corollary}{Corollary}[prop]
\newtheorem{lemma}[prop]{Lemma}

\renewenvironment{abstract}
{\small
	\list{}{
		\setlength{\leftmargin}{0.025\textwidth}%
		\setlength{\rightmargin}{\leftmargin}%
	}%
	\item\relax}
{\endlist}
\begin{document}
%	\includepdf[pages=-]{../titlepage.pdf}
	\maketitle
	\begin{abstract}
		\begin{singlespacing}
			\textbf{Abstract \ }
			
			To reach climate targets the International Panel on Climate Change has identified net-zero emissions by 2050 an essential element. I show that progressive income taxes are optimally used in concert with corrective taxes to meet emission targets.
			On the one hand, progressive income taxes reduce labour efforts as leisure becomes relatively cheaper. The overall reduction in production lowers emissions. On the other hand, tax progressivity recomposes the structure of the economy away from green energy production through a skill supply channel calling for a more regressive tax. For a reasonable calibration, I find that the reduction effect dominates and the optimal income tax schedule is progressive. The welfare advantage of income tax progressivity arises from a reduction of inefficiently high hours worked. The model suggests that including progressive income taxes as a tool to lower emissions accounts for a rise in social welfare by 0.1\% over the 60 years from 2020 to 2080.  % as relative supply of the through a skill-supply channel. As richer, high-skilled workers reduce their labour supply more in response to a more progressive tax, low-skill labour becomes more abundant. 
			
%			Natural scientists have identified the reduction of demand %for energy and land %thus, a change in lifestyle 
%			as an important contributor to meeting global climate targets. However, a general equilibrium analysis of reduction policies is missing.
%		%	I study the general equilibrium effects of reduction policies: such as income taxes, a restriction of hours worked, or consumption taxes. 
%		A higher labour income tax progressivity can achieve such a reduction as it lowers labour supply. 
%		Then again, tax progressivity alters the relative skill supply. As the green sector is relatively more skill-biased, the economy recomposes production towards the dirty sector. What is the optimal policy when the government has to meet an exogenous emission and demand target?
		%This additional benefit of income taxes changes the  equity-efficiency trade-off which classically determines optimal fiscal policies. 
		%To answer the question, I build a model of directed technical change and skill heterogeneity.
		
		
%In a set up with representative agent, the necessity to meet emission targets makes the optimal income tax highly progressive. The more goods are substitutes the higher the optimal tax progressivity. When goods are complements, the more slowly growing clean sector dampens production in the dirty sector making a lower progressivity sufficient. 
	%		\noindent \textit{JEL classification}: E71, H21, H23,  O11, O13, Q58
			
		\end{singlespacing}
		
		\end{abstract}
%\tableofcontents

\section{Introduction}

%\begin{quote}
%"Mitigation pathways limiting warming to 1.5°C [...]  reduce emissions further to reach net zero $CO_2$ emissions in the 2050s [...] \textit{(medium confidence)}."
%\end{quote}

The latest reports of the International Panel on Climate Change (IPCC) \citep{ IPCC2022, Rogelj2018MitigationDevelopment.} highlights the importance of an absolute emission limit to comply with the Paris Agreement on limiting temperature rise to 1.5°C or well below 2°C.\footnote{ \ The Paris Agreement of 2015 formulates clear political goals to mitigate climate change. Under this treaty, states have agreed on a legally binding maximum increase in temperature to well below 2°C, preferably 1.5° over pre-industrial levels, and the global community seeks to be climate-neutral in 2050  (compare:\\ \url{https://unfccc.int/process-and-meetings/the-paris-agreement/the-paris-agreement}). 
} 
The economics literature on environmental policy has by and large allowed for a trade-off between consumption and pollution \citep{Barrage2019OptimalPolicy, Golosov2014OptimalEquilibrium} or studied relative emission targets \citep{Fried2018ClimateAnalysis}. 
The presence of an absolute emission target may change the optimal environmental policy, as it poses a limit on growth in fossil energy usage.
%Depending on the substitutability of green and fossil energy and the velocity of the green sector to grow, the absolute emission target may, first, pose a limit to consumption growth and, second, make 
In particular, untraditional policy measures in addition to corrective taxes may become optimal. %\textit{ (WHY THIS DIFFERENCE? Also with externalities in consumption pollution cannot be compensated for by consumption as the marginal utility of consumption reduces. BUT STILL MORE IS BETTER! SO IT CAN BE COMPENSATED! )} 

%For a reasonably calibrated endogenous growth model,
 I find that the optimal labour income tax is progressive when accounting for an absolute emission target even though corrective taxes on fossil energy are available. This finding highlights the importance of policy measures targeted at a \textit{reduction} of production in tandem with \textit{recomposing} policies such as carbon taxes to mitigate climate change. % Then again, I present data indicating a voluntary reduction in household consumption. Given this behavioural change, the optimal income tax progressivity could become regressive in order to boost high-skill labour supply. 

%MODEL
To investigate the effect of an exogenous emission target on the optimal policy, I study an endogenous growth model building on \cite{Fried2018ClimateAnalysis}. 
%Calibration

% Quantitative Experiment and Results
The main finding is that progressive income taxes are optimally used in concert with carbon taxes to meet emission targets. Indeed, the emission target could be satisfied without income taxation by use of higher carbon taxes, yet, at lower welfare.  
Meeting emission targets with a fossil tax alone, hours supplied are inefficiently high. Due to the cap on fossil energy and green and fossil energy being no perfect substitutes, the additional work effort is not sufficiently compensated for by rising consumption.  
In fact, by use of a more regressive tax the government could subsidise research. Nevertheless, the planner optimally reduces labour supply thereby forfeiting higher growth rates in the green sector. %However, in presence of the emission target, higher work effort at higher productivity would violate the emission target. 

\paragraph{Literature}
The paper relates to two strands of literature. Firstly, it contributes to the literature on environmental policy and directed technical change \citep[e.g.][]{Acemoglu2012TheChange, Acemoglu2016TransitionTechnology} by focusing on unconventional policy measures which is justified given the urgent nature of climate change mitigation.  The paper is most closely related to \cite{Fried2018ClimateAnalysis} extending the analysis by (i) an optimal policy analysis, (ii) studying an absolute emission target, and (iii) providing the government with an additional policy tool: income taxes.
%\\
%{Limits to and endogenous growth}
%The paper also relates to the endogenous growth literature by inclu

Secondly, the paper connects to the literature on public policy which focuses on an efficiency-equity trade-off \citep{Heathcote2017OptimalFramework, Loebbing2019NationalChange}. In this project, instead, the reduction in labour effort induced by distortionary labour taxes has an advantageous effect: it reduces emissions. On the other hand, there is a recomposing effect which counteracts the reduction of emissions. This reduction-recomposition trade-off shapes the optimal tax progressivity in the present paper. 

%Third, the finding that hours worked are inefficiently high relates the paper to the literature on inefficiently high work efforts. Generally, too high hours supplied arise from some negative externality of consumption such as  a keeping-up-with the Joneses motive or envy \cite{Alvarez-Cuadrado2007EnvyHours}. \cite{Arrow2004AreMuch} also discuss the question of too high consumption levels.
\tr{Is there literature on optimal income taxes and environmental taxes? chosen jointly}

Thirdly, the paper relates to the literature discussing optimal environmental taxation in a distortionary fiscal setting \citep{Bovenberg1997EnvironmentalGrowth, Barrage2019OptimalPolicy, LansBovenberg1994EnvironmentalTaxation}. Redistribution and environmental protection arise as competing targets in the optimal policy. The reason is that the environmental tax reduces labour efforts as it diminishes the returns to labour or makes consumption more expensive.  

However, in contrast to this literature, the present paper 


\tr{Which tax schedule is closer to reality? }
\paragraph{Outline} Section \ref{sec:model} lays out the model which is calibrated in section \ref{sec:calib}. I subsequently show and discuss results in section \ref{sec:res}. Section \ref{sec:con} concludes. 
%



\paragraph{Quantitative model}
Recent work has shown, that  higher tax progressivity is amplified in lowering inequality through a compression of the wage rate. (there is a second effect...). On the other hand, high skill labour is used in a higher share in green sectors \cite{Consoli2016DoCapital}. Therefore, progressivity implies a shift to dirty innovations and a higher externality. These channels constitute a new trade-off between inequality and climate change mitigation. 

% motivation from consumption reduction proponents
Furthermore, a higher tax progressivity reduces consumption at higher income levels (only if not smoothed by savings), production and externalities. Then again, high skill labour may be missing for green production. The reduction and recomposition mechanism counteract each other once accounting for skill heterogeneity. 

%\begin{itemize}
%	\item What is the optimal policy to achieve emission targets by 2050?
%	\item role for fiscal policy due to time frame and skill supply?
%	\item inefficient low supply of high skill labour \ar regressive tax optimal?
%	\item demand side: to counteract the negative effect of redistribution through lower skill supply? \ar demand side effect
%	\item voluntary reduction in consumption \ar even lower skill supply? \ar who are these households? (rich/ high low skill?)
%	\item non-monetary motive for scientists? 
%\end{itemize}

\subsection*{Comment: 31/03/22}
It seems difficult to solve the problem when no balanced growth path exist since fossil output has to be constant under the optimal policy. (But that is a constant growth rate, just that output ratios are not constant) 
Therefore,
\begin{enumerate}
	\item solve under assumption of constant growth rates \ar that would be the limit and determines the continuation value of the economy
	\item assume a planner who only cares about the transition to the net-zero emission economy \ar most important to satisfy voters today
	\begin{itemize}
		\item the objective function is the sum of transition periods (2020 to 2080), with the length of a period =5 years \ar 30 periods.
		\item using numeric method as in Barrage should be solvable; no continuation value; (later adding continuation value not a big deal)
		\item no need to make it stationary
		\item how does the economy evolve afterwards?
	\end{itemize}

\end{enumerate}
\subsection*{Comment: 25/03/2022}
In the models on endogenous growth, emissions positively depend on innovations in the dirty sector. Technology in this sector has to be perceived as more goods being produced each of which exerts the same level of the externality. This seems sensible, when these goods need the same input of emissions generating factors but can be produced with the same number of machines. Also sensible when thinking about waste which is on product level. But then again could think of progress as needing less input goods to achieve the same number of outputs, then should measure emissions by input factors and not output. This is captured by growth in the green sector.

Now, assuming emissions are proportional to output, and introducing the exogenous limit on emissions s.t. net-emissions have to equal zero from mid-century onward, then the assumption that on the BGP all technology ratios are constant implies that all growth has to stop. (also assuming here that carbon capture cannot grow without end). This seems very restrictive. Could divert from assumption of constant technology ratios on BGP: instead assuming a generalised BGP which allows for transitions across sectors.  On this GBGP fossil output has to remain at the same level (as an upper bound assume the one prescribed by the IPCC). Then growth in the fossil sector is zero (compatible with BGP, but technology ratios are not constant.) In fact, a BGP. 

Due to spill overs there might be room for ever growing corrective taxes to counter market forces. 
As the green sector growth, green energy becomes relatively cheaper and cheaper compared to fossil energy. This market effect could redirect production and innovation to green energy. 
The labour income tax could support by changing relative skill supply. 

\subsection*{Comment: 19/03/2022}
\begin{itemize}

	\item \textbf{Question 1 a)}: \textbf{how does the presence of a progressive tax (as calibrated) (or a demand target) change the optimal environmental policy?}\\ \ \\
%	\\ Steps
%	\begin{enumerate}
%		\item max swf st emission constraints; save optimal policy
%		\item max swf st emission constraint and demand constraint
%		\item compare optimal policies
%	\end{enumerate}
\textbf{Motivation:} How tax progressivity affects the externality: lowering demand on the one hand reduces emissions as output decreases; on the other hand, labour supply incentives change, potentially more so for high skilled than for low skilled workers.\\
In the literature on optimal environmental policies, positive labour income taxes lower the optimal environmental tax below the Pigouvian rate, due to efficiency costs. In this setting, however, the optimal environmental tax might be higher to counteract the positive effect of income tax progressivity on the externality. \\
Starting from \cite{Fried2018ClimateAnalysis}; set up:  model with rep agent, max social welfare function plus target;  the planner can choose corrective taxes the income tax is given exogenously
\begin{enumerate}
	\item amend model to incorporate income taxes and skill heterogeneity
\item set up her model to find the optimal environmental tax as done in \cite{Barrage2019OptimalPolicy}
\item 
\end{enumerate}
\item[\ar] check model implications in the data: is there heterogeneity in the effect of tax progressvity on the externality across countries which differ in their wage-hour elasticity. 

	\item Question 1 b): (Optimal policy) Is there the potential for the income tax code to be targeted at the externality even if a corrective tax is present and there is no demand target?
Why? Rather, a further reduction of tax progressivity might be optimal to increase high-skill labour supply. 
\ar \textbf{another trade-off between inequality and the externality}. Not only via the efficiency channel but due to redistribution, innovations will be directed towards the polluting sector. \\
Add exogenous demand target as suggested by natural scientists. What is the optimal policy in this case? Could also find that the environemntal tax is used to lower aggregate output when goods are complements! 

\item \textbf{Adding inequality}
On the other hand, the environmental tax could be lower than absent inequality as it increases wage dispersion. 

	\item Question 2 (This refers to \cite{Loebbing2019NationalChange}): reducing consumption bears the potential of social unrest. \ar add inequality. How would a social planner choose to meet the targets if he searches to minimise social tension/ impact on the poor/ utilitarian swf? 
	\\
	set up: two household types, max swf and targets; additional motive to avoid inequality
	\\
	Steps
	\begin{enumerate}
	\item what are the distributional effects of the policy found for question 1?
	\item How would the optimal planner set the optimal policy now? 
	\end{enumerate}
\item why not look at good specific taxes? \ar because (1) hit the poor (regressivity), (2) corresponds to corrective tax! \ar but then no need for income tax
\end{itemize}

New motivation: \\
Natural scientists have identified a reduction in demand for energy and land-intense products as key to meeting climate targets and sustainability goals jointly, or to diminish the reliance on risky carbon dioxide reduction technologies. More broadly, \cite{Arrow2004AreMuch} argue for the efficiency gains of lowering aggregate consumption through the use of public policy instruments. 
However, a dynamic general equilibrium analysis is missing. 

\paragraph{How to get to this?}
\ar Amend \cite{Fried2018ClimateAnalysis}. 
\begin{enumerate}
\item add income tax and elastic labour supply
\end{enumerate}
\subsection*{Comment: 16/03/2022}
\begin{itemize}
	\item this version: the planner maximises social welfare but is constrained by an emission target; the planner only has labour income taxes at its disposal; the economy can be represented by a representative agent
	\item way forward (quantitatively):
	\begin{enumerate}
		\item model rep agent as only supplying one skill \ar $\zeta=1$
			\item depart from log utility of consumption; use preferences with a slightly higher income effect than substitution effect as suggested by \cite{Boppart2019labourPerspectiveb}.
		\item  add a target on demand to the Ramsey problem\ar the planner not only has to meet emission targets but also a target on demand which is motivated by the natural sciences debate on climate change: 
		\begin{quote}Reduction policies alleviate the pressure to meet other sustainability goals \citep{Bertram2018TargetedScenarios}, they reduce the necessity to rely on \textit{carbon dioxide removal} technologies which are not without risk as they rely on  underground CO2 storage and compete with land needed for food production and biodiversity protection \citep{VanVuuren2018AlternativeTechnologies}.
		\end{quote}
	then, the planner might choose income taxes to meet the emission target even if corrective taxes are available.

		\item  introduce heterogeneity  (skills) 
		\item  introduce directed technical change 
		\item what if households want to lower consumption absent any policy intervention? How does this change the analysis?
	\end{enumerate}
	
\end{itemize}

\section{Introduction}
% this intro refers to the following setup:
% The government can only use labour taxes and corrective taxes are not available
% e.g what can the government achieve with common 

% Structure Intro
% 1. Motivation: (2) setting real world, (3) Why is the question interesting? (Tradeoff)

% 2. What I do: Contribution and main finding

% 3. Model (several layers)

% 4. Calibration

% 5. Main quantitative experiment and results
% 
% To do: 
%\tr{ (i) Connect paragraphs,
% (ii) guide reader, 
% (iii) make smooth }

\begin{comment}
\textcolor{violet}{Still to do:
\begin{itemize}
	\item possibilities to model technical change: substitutability of goods, growth in sector, innovation on substitutability versus consumption growth
\end{itemize}
}

content...
\end{comment}

%\paragraph{Classical use of fiscal instruments}
An equity-efficiency trade-off is central to the discussion of optimal labour income taxation and tax progressivity in the public finance literature.  The benefits of labour taxes and progressivity arise, inter alia, from redistribution. %and from generating government revenues. 
With concave utility specifications full redistribution is efficient. However, the optimal tax system does not feature full redistribution when labour supply is endogenous. Instead, redistribution is traded off against aggregate output as individuals reduce their labour supply and skill investment in response to labour income taxation. 

%\paragraph{Environmental Externality}
Adding environmental externalities to the classical public finance framework changes the perception of efficiency costs. Instead of merely reducing welfare, direct benefits through a reduction of the externality arise by lowering output. 
In theory, corrective, environmental taxes can establish the efficient allocation in a representative agent economy. Absent inequality, such a tax instrument is optimally set to the social cost of an externality. Originators then internalise these costs in addition to their private ones. However, governments face political difficulties in implementing such policy instruments.\footnote{\ Compare, for instance, the Yellow Vest movement in France in 2018.} On the other hand, scientific research has emphasised the urgency to act and highlighted the advantages of lowering demand for land and energy.
For example, reduction policies alleviate the pressure to meet other sustainability goals \citep{Bertram2018TargetedScenarios}, they reduce the necessity to rely on \textit{carbon dioxide removal} technologies which are not without risk as they rely on  underground CO2 storage and compete with land needed for food production and biodiversity protection \citep{VanVuuren2018AlternativeTechnologies}.
 Therefore, this paper shifts the focus of optimal environmental policies  to fiscal tax instruments as tools to lower demand and meet emission targets. What can be achieved in terms of climate targets and what are the costs?

%\paragraph{Trade-off/ Mechanisms}
 Consumption reduction in affluent countries has been promoted as an environmental policy \citep{Schor2005SustainableReduction, Pullinger2014WorkingDesign, Arrow2004AreMuch}. But, the general equilibrium effects are less well understood.
While having an advantageous direct effect on the externality, counteracting indirect effects may exist in a general equilibrium framework. Proponents of a reduction policy especially focus on consumption by the rich which consume a higher amount of natural resources.\footnote{\ There is a bunch of research on the consumption of resources by income groupd; see for instance \cite{Sager2019IncomeCurves}.} %\footnote{\ Note Sonja: Abstracting from inequality, would it still be best to reduce consumption by the rich when the poor have a higher marginal propensity to consume dirty? \textit{Could be an important aspect in the model}. Not in the baseline, look at it in an extension...}
This concern could add to the benefits of tax progressivity.
In contrast, targeting rich households in particular for environmental reasons will lower the supply of high skilled labour.\footnote{\ The relation of labour income tax progressivity and skill investment has been studied by \cite{Heathcote2017OptimalFramework}.} Yet, these skills are essentially important in greener sectors of the economy \citep{Consoli2016DoCapital}. As a result, dirty production becomes relatively cheaper and the dirty share of production rises. 
% I want to add endogenous innovation later

%\paragraph{Model}
% this version: With Rep agent
I build a tractable model which incorporates the key aspects sketched above. There are two sectors one of which emits pollutants: the dirty sector. Both clean and dirty goods are necessary inputs to the final consumption good. Sectors produce with a sector-specific labour input good. The labour input good in the clean sector contains a higher share of high-skilled labour. 
The economy behaves as if there was a representative household which provides high and low-skilled labour. The former exerts a higher utility cost for the household generating a wage premium for high-skilled labour. 

The government maximises social welfare from a Ramsey planner's perspective. However, it is constrained by an exogenous limit on emissions. The advantage of this approach is that it suffers less from  model misspesifications due to  uncertainties about how emissions affect the environment. Furthermore, it is closely related to the current political debate.\footnote{\ Compare appendix section \ref{app:emission_climate_targets} for a more in depth discussion of this aspect. } 

%\paragraph{Calibration}
I inform the exogenous emission limit by the  targets proposed in the 2018 report of the Intergovernmental Panel on Climate Change (IPCC)\footnote{\ A body of the United Nations established to assess the science related to climate change.},  \cite{Rogelj2018MitigationDevelopment.}. These targets are designed for states to comply with the Paris Agreement: global net greenhouse-gas emissions in 2030 shall equal 25-30 GtCO2e per year and zero in 2050 (p.95 in \cite{Rogelj2018MitigationDevelopment.}).%Indeed,  the agreement  foresees a tight time frame for emission reductions: climate neutrality should be achieved by mid-century.
\footnote{\ Under this treaty, states have agreed on limiting temperature rise to well below 2°C, preferably to 1.5°C, and to achieve climate neutrality by mid-century \url{https://unfccc.int/process-and-meetings/the-paris-agreement/the-paris-agreement}. }
%Compared to integrated climate assessment models, (CHECK DEFINITION) this approach requires less assumptions concerning the relation of emissions and the climate. What

Another important calibration choice is the substitutability of clean and dirty production in the final consumption good. I make the cautious assumption that goods are no perfect substitutes. In other words, there is always at least a small amount of dirty production necessary to produce the final consumption good. \tr{\cite{Cohen2019AnnualSubstitutable} discuss and estimate the substitutability of natural capital in production with a focus on energy. }

%\paragraph{Quantitative Exercise and Results}
The paper is divided into two parts: an analytical part where I derive propositions concerning the role of fiscal policy and a quantitative part which discusses the optimal policy and transitions. 

The main theoretical result is that in the laissez-faire economy, emissions grow without bound. Irrespective of whether the clean and dirty good are substitutes or compliments.

In the quantitative exercise I let a planner choose the optimal policy by maximizing a utilitarian social welfare function but it faces an constraint on emissions. I solve explicitly for the optimal policy in each period making the optimal tax progressivity time dependent. 


\paragraph{Literature}

The paper is related to 3 strands of literature. 

First, to the public finance literature.  \cite{Heathcote2017OptimalFramework} study optimal labour tax progressivity on 



Second, to the literature on optimal environmental policy. 

Third, to the literature on directed technical change. 
\textbf{HEMOUS and Olsen} discuss an endogenous growth model with heterogeneous labour input:
\begin{itemize}
	\item the wage premium is not constant on a BGP which they specify as stable if innovation occurs in both sectors
	\item hence: a non stable BGP is one where innovation does not occur in both sectors at some point
	\item need to allow for this option< when solving the model
	\item quality ladder model: each scientist after having chosen a sector of production, there is no congestion (each scientist works on one machine) legitimate due to within-sector spillovers
	\item on a BGP with equal growth the wage premium may grow (Result in \cite{Acemoglu2002DirectedChange}) 
	\item in \cite{Acemoglu2012TheChange} 
	\begin{itemize}
		\item as emissions are proportional to dirty output implicit assumption of a Leontief production function if there was energy (and also this as only source of emissions); 
		 \item endogenous labour (no sector-specific labour supply) \ar the more productive sector attracts more labour (the MPL is higher at an equal ratio so that more labour ends up in the more productive sector to have equal wages)
		 \item for substitutes innovation might be stuck in the more advanced market as the price effect (which directs innovation to the less productive market) is muted
		 \item[\ar] in \cite{Fried2018ClimateAnalysis} fossil and green energy are substitutes \ar stuck in fossil innovation; but non-energy goods and energy are complements \ar price effect strong; equalising effect
		 \item in LF stuck with dirty innovation, government can redirect innovation towards the clean sector until it catches up \ar policy intervention needer for a " sufficient amount of time" \ar might be missing in today's world
		 \item with DTC need postponing intervention problematic!
		 \item subsidies and corrective taxes needed to implement first best 
	\end{itemize}
\item \cite{Acemoglu2016TransitionTechnology} incremental (sector-specific) and radical (building on the leading technology irrespective of sector) innovation \ar cross-sectoral spillovers which were absent in \cite{Acemoglu2012TheChange}
\end{itemize}

%Finally, the paper is meant to add to the discussion on reduction versus recomposition policies as tools to reduce human impact on the environment. 

\paragraph{Outline}
The paper is structured as follows. In the next section, I define a simple model. % to analyse the role of fiscal taxes on the environment. 
Section \ref{sec:theory} discusses theoretical optimal policy results. Section \ref{sec:calib} argues for the plausibility of chosen parameter values. In section \ref{sec:simul}, I show dynamics of the economy under the laissez-faire and the optimal policy regimes. 


%\section{Literature}


\paragraph{Notes \cite{Consoli2016DoCapital}}
\begin{itemize}
	\item in contrast to previous studies, they focus on occupation-level task desciptions
	\item previous work more on the industry level to prox greenness of occupation 
	\item skill and human capital dimension and green jobs; previous work on the effect of environmental regulation abstracted from these quality dimensions
	\item main findings:
	\begin{enumerate}
		\item green occupations exhibit a stronger intensity of high-level cognitive skills (\ar same occupation green version higher cognitive skills)
		\item changing occupations (becoming greener) more formal education, more work experience, more on the job training
	\end{enumerate}
\item method: within SOC 3digit groups compare skill level of occupations identified as emerging due to enviornmental needs, and those jobs transitioning to green tasks
\item SOC3, e.g engineers
\item otherwise, findings could be driven by green jobs beeing differen broader categories. (which I would want to include)
\item \textbf{skill heterogeneity is driven by two dimensions: }
\begin{enumerate}
	\item green occupations cluster in \textbf{high-skill} (\textit{intensive in abstract skills"}) \textbf{macro-occupations} p.1051 (\ar differences in type of jobs); important: non-routine analytical and interactive skills; remaining in mid-skill occupational skills
	\item differences within job types: green version of \textbf{the same job is more skill intense} (this is what they study in this paper! )
\end{enumerate}
\item SOC2 not included occupations: agriculture, public sector
\item sort SOC2 occupations according to average tasks


\begin{itemize}
	\item non-routine analytical: \\
	4.A.2.a.4 (IM) Analyzing data or information
	4.A.2.b.2 (IM) Thinking creatively
	4.A.4.a.1 (IM) Interpreting the meaning of information for others
	\item Non-routine interactive (NRI)
\\
	4.A.4.a.4 (IM) Establishing and maintaining interpersonal relationships
	4.A.4.b.4 (IM) Guiding, directing, and motivating subordinates
	4.A.4.b.5 (IM) Coaching and developing others
	\item Routine cognitive (RC)
\\
	4.C.3.b.4 (CX) Importance of being exact or accurate
	4.C.3.b.7 (CX) Importance of repeating same tasks
	4.C.3.b.8 (CX, reverse) Structured versus unstructured work
	\item Routine manual (RM)
\\
	4.A.3.a.3 (IM) Controlling machines and processes
	4.C.2.d.1.i (CX) Spend time making repetitive motions
	4.C.3.d.3 (CX) Pace determined by speed of equipment
	\item Non-routine manual (NRM)\\ 
	4.A.3.a.4 (IM) Operating vehicles, mechanised devices, or equipment
	4.C.2.d.1.g (CX) Spend time using hands to handle, control or feel objects, tools or controls
	1.A.2.a.2 (IM) Manual dexterity
	1.A1.f.1 (IM) Spatial orientation 
\end{itemize}
\end{itemize}

Green jobs are defined as either recomposing the energy mix towards green energy or as increasing energy efficiency in general: "\textit{reducing the use of fossil fuels, decreasing pollution and greenhouse-gas emissions, increasing the efficiency of energy usage, recycling materials, and developing and adopting renewable sources of energy}"

Idea to include technology on energy efficiency, which contributes to lowering emissions. 
Model so far does not allow for an increase in energy efficiency
\begin{align*}
Y=(E^{\frac{\varepsilon_e-1}{\varepsilon_e}}+N^{\frac{\varepsilon_e-1}{\varepsilon_e}})^\frac{\varepsilon_e}{\varepsilon_e-1}
\end{align*}
what research does in the baseline model is increasing output for the same amounts of input. To allow for an increase in energy efficiency could add
\begin{align*}
Y=((A_eE)^{\frac{\varepsilon_e-1}{\varepsilon_e}}+N^{\frac{\varepsilon_e-1}{\varepsilon_e}})^\frac{\varepsilon_e}{\varepsilon_e-1}
\end{align*}
But, in fact most of the jobs counted as green refer to increasing the use of renewable energy resources. Could also argue that there is an upper bound on the efficiency of energy
\subsection{Models}
\begin{itemize}
\item \cite{Bilbiie2012EndogenousCycles}
\begin{itemize}
\item a model with rep agent
\item investment in the form of stock 
\item innovation as a form of new products
\item one final good sector
\item monopolistic competition
\item homothetic preferences
\end{itemize}
\item \cite{Ravn2006DeepHabits}
\begin{itemize}
 \item habits over average previous consumption of specific good! not over total consumption
 \item rep agent 
 \item habits: marginal utility rises as habits rise \ar could look at what happens as habits are reduced! \ar marginal utility at given consumption level reduces!
 \item more is always better! Plus increases habits \ar I want: that more might not be better after some point
\end{itemize}
\item \cite{McKay2021LumpyPolicy}
\begin{itemize}
\item New Keynesian model with durable and non-durable consumption 
\end{itemize}
\item \cite{Acemoglu2012TheChange}
\begin{itemize}
\item endogenous growth
\item rep agent
\item single labour market
\item no resource use in clean sector! ; abstracts from waste
\item disaster risk!: There is a lower bound on the quality of the environment 
\item environmental externality only affects Utility! So no chance for \textbf{environmental quality} to drive production to zero!
BUT there is a natural resource which is used in production; \textit{How do the two relate?} \ar when environmental quality affects regeneration of exhaustible resource, then there would be some connection, but there is no regeneration of the resource, I think
\item there is degradation of the environment through unsustainable production (only!) and 
\end{itemize}
Functional forms
\begin{align*}
S\in[0,\bar{S}],\ & \text{where}\ \bar{S}\ \text{is the quality of the environment without pollution;}\\
S_v=0 \Rightarrow S_t=0 \forall t\geq v,\ &  0 \ \text{is the point of no return.}\\
\underset{S\rightarrow0}{lim} U(C,S)=-\infty\ & \text{S=0 is a disaster!}\\
\underset{S\rightarrow0}{lim}\frac{\partial U(C,S)}{\partial S}=\infty\ &\\
S_{t+1}= -\xi Y_{dt}+(1+\delta)S_t& \\ 
\text{evolution of environmental quality:} & \text{ falls in dirty production; regeneration rate }\\
 \text{both are exponential relationships}\Rightarrow&\text{ smaller env. quality slower regeneration}\\ 
 &\text{ higher pollution, stronger degradation}
\end{align*}
The dirty sector uses an exploitable resource in the production process
\begin{align*}
Y_{dt}= R_t^{\alpha_2}L_{dt}^{1-\alpha}\int_{0}^{1}A_{dit}^{1-\alpha_1}x_{dit}^{\alpha_1}di
\end{align*}
$R_t$ is the exhaustible resource
\begin{align*}
Q_{t+1}=Q_t-R_t
\end{align*}
they look at a version where the resource is common property (water, air) or owned (Hotelling rule)
\item \cite{Heikkinen2015DegrowthConsumers}: macro model with voluntary reduction in consumption
\item \cite{Borissov2019CarbonDevelopment}: model labour sector in more detail: skill, sectors, and transition
\item \cite{Michaillat2015AggregateUnemployment, Auerbach2021InequalityEconomy} examples of models with economic slack. But both do not feature a satiation point of consumption. 
\item 
\cite{Loebbing2019NationalChange}
\begin{itemize}
	\item studies the welfare effects of a progressive tax reform in a model where skill supply drives the innovation decision of firms
	\item when the rich reduce their labour supply more \ar innovation is directed towards low skills\ar the wage distribution compresses
	\item the overall effect on welfare is mixed since not only do low skill wages catch up but also does the tax base reduce as the high skilled reduce their labour supply
	\item in sum, he finds that, taking directed technical change into account increases the set of welfare improving tax reforms 
	\item first he focuses on the mechanism, second on the optimal tax scheme
	\item optimal tax reform discussed in a comparison to the optimal tax a planenr would choose who perceives technology as fixed
	\item then quantification \ar he finds small responses of labour supply to progressive tax reforms
\end{itemize}
\end{itemize}


\subsection{Motivation}
\begin{itemize}
\item \cite{Schor2005SustainableReduction}
\begin{itemize}
	\item arguments against unlimited growth
	\begin{itemize}
\item hhh
	\end{itemize}
\end{itemize}
\item \cite{Dasgupta2021}
\begin{itemize}
\item emphasises the use of nature as a sink (stock) and as an input to production \ar can the two be combined?
\end{itemize}
\end{itemize}
 %<- contains summaries of potentially relevant papers
%\section{Model}\label{sec:model}

The previous section has argued that the optimal environmental policy consists not only of a recomposing but also a reductive policy measure. When lump-sum transfers are not available, the reduction in labor supply can be established through progressive income taxes. 
 However, it is unclear, if a progressive income tax scheme is still optimal in a more realistic model with endogenous growth and skill-bias of the green sector. Then, two mechanisms may render a regressive tax scheme preferable. First, a reduction in labor supply lowers research effort thereby reducing growth. Second, when the green sector is skill-biased - an observation \cite{Consoli2016DoCapital} provide empirical evidence for - and high-skilled labor is more responsive to a higher tax progressivity, then a more progressive tax redirects production towards the fossil sector. The reason is that the fossil-specific input good is in relative higher supply. As a result, emissions increase. This effect is intensified by endogenous growth as research shifts towards the sector with a higher market share.

To investigate these considerations, this section extends the core model of section \ref{sec:mod_an} to a quantitative framework mainly building on \cite{Fried2018ClimateAnalysis}.
In the quantitative model, there is a third neutral sector which is combined with an energy good to form final output. The differentiation between clean and dirty production is allocated to the energy sector which produces using fossil (dirty) and green (clean) intermediate goods.
The representative household provides two skills: high and low, which are used in different shares in the neutral, fossil, and green sector. 
Endogenous growth is modeled in form of directed technical change resulting from research. The government seeks to maximise utility of the representative household under the constraint of meeting an exogenous emission target. Emissions arise from the usage of fossil energy.
I study the model for a fixed amount of periods as I do not want to make any assumption on steady growth due to the absolute constraint on fossil production. 

\paragraph{Households}
% the rep agent
Modeling the economy as a representative family allows to abstract from inequality as a motive for government intervention while at the same time being able to study skill heterogeneity.
 The household chooses hours of high- and low-skilled workers and average consumption taking prices as given. The share of worker types is fixed with a lower share of high-skilled workers, $z_h$, resulting in a skill premium. The household's problem reads

\begin{align}
%U=
\underset{\{C_{t}\}_{t=0}^{\infty}, \{h_{lt}\}_{t=0}^{\infty}, \{h_{ht}\}_{t=0}^{\infty}}{max}&
\sum_{t=0}^{\infty}\beta^t u(C_{t}, h_{lt}, h_{ht})\\
%U_{s}=\underset{\{c_{st}\}_{t=0}^{\infty}, \{h_{st}\}_{t=0}^{\infty}}{max}&\sum_{t=0}^{\infty}\beta^t u_s(c_{st}, h_{st}; S_t)\\
s.t.& \ \ p_{t}C_{t}\leq% (1-\tau_{lt})(h_{ht}w_{ht}+h_{lt}w_{lt})+T_t\\ 
z_h\lambda_t \left(h_{ht}w_{ht}\right)^{1-\tau_{lt}}+(1-z_h)\lambda_t\left(h_{lt}w_{lt}\right)^{1-\tau_{lt}}+T_t\\
\ & h_{ht}\leq \bar{H}_t\\
\ & h_{lt}\leq \bar{H}_t.
\end{align}
The government levies a tax on skill-level income. 
%The tax schedule is characterised by  a scaling factor, $\lambda$, and a parameter determining the progressivity of the tax schedule, $\tau_{lt}$, \citep[compare, e.g.,][]{Heathcote2017OptimalFramework}. The household receives lump-sum transfers from the government, $T_t$.  As $\tau_{lt}$ increases, the elasticity of disposable income with respect to hours worked decreases. 
The effect of a change in tax progressivity through lower income is similar across skill types; however, the substitution effect is higher for high-skilled workers. As this type works more hours prior to a tax change, additional leisure is more valuable to them. Therefore, as the tax schedule becomes more progressive, high-skilled workers decrease their time spent working relatively more.

%The choice to focus on a representative family enables to abstract from inequality as a motive for government intervention. 
\paragraph{Production}
Production separates into final good production, energy production, intermediate good production, the production of machines and the intermediate labour input good. 

The final good producing sector is perfectly competitive combining the non-energy and energy goods according to:
\begin{align}
Y_t=\left[\delta_y^\frac{1}{\varepsilon_y}E_{t}^{\frac{\varepsilon_y-1}{\varepsilon_Y}}+(1-\delta_y)^\frac{1}{\varepsilon_y}N_{t}^{\frac{\varepsilon_y-1}{\varepsilon_y}}\right]^\frac{\varepsilon_y}{\varepsilon_y-1}
\end{align} 
I take the composite good as the numeraire and define its price as $p_t=\left[\delta_yp_{Et}^{1-\varepsilon_y}+(1-\delta_y)p_{Nt}^{1-\varepsilon}\right]^{\frac{1}{1-\varepsilon}}$.

Energy producers perfectly competitively combine fossil and green energy to a composite energy good:
\begin{align}
E_t=\left[F_t^\frac{\varepsilon_e-1}{\varepsilon_e}+G_t^\frac{\varepsilon_e-1}{\varepsilon_e}\right]^\frac{\varepsilon_e}{\varepsilon_e-1}.
\end{align}
The price of Energy is determined as  $p_{Et}= \left[p_{Ft}^{1-\varepsilon_e}+p_{Gt}^{1-\varepsilon_e}\right]^\frac{1}{{1-\varepsilon_e}}$.

Intermediate goods, fossil,$F_t$, green, $G_t$, and non-energy, $N_t$, are again produced by competitive sectors using a sector-specific labour input good and machines. The production function in sector $J\in \{F,G,N\}$ reads
\begin{align}
&J_{t}= L_{jt}^{1-\alpha_j}\int_{0}^{1}A_{jit}^{1-\alpha_j}x_{jit}^{\alpha_j} di.
\end{align}
$A_{jit}$ indicates the productivity of machine $i$ in sector $j$ at time $t$: $x_{jit}$. In contrast to the two other sectors, the government may levy a sales tax on fossil producers. Their profits read
\begin{align}
\pi_{ft}=p_{ft}(1-\tau_{ft})F_t-w_{lft}L_{ft}-\int_{0}^{1}p_{xfit}x_{fit}di.
\end{align}

The labour input good, $L_{jt}$, is produced by a perfectly competitive and sector-specific labour industry according to: 
\begin{align}
L_{jt}=h_{hjt}^{\theta_j}h_{ljt}^{1-\theta_j}.
\end{align}
This additional intermediate sector allows to capture differences in skills by sector, especially in the fossil and green sector.

Machine producers are imperfect monopolists and produce sector specific machines. Machine producers maximise their profits by choosing the price at which to sell their machines to intermediate good producers. They also decide on the amount of scientists to employ. As the productivity of machine $ij$ increases, demand for this machine increases too; this provides the incentive for machine producers to invest in research. Irrespective of the sector, the costs of producing one machine is set to one unit of the final output good \citep[similar to][]{Fried2018ClimateAnalysis, Acemoglu2012TheChange}.
Each period, machine producers solve
\begin{align}
\underset{p_{xjit}, s_{jit}}{\max}p_{xjit}(1-\zeta_{jt})x_{jit}-x_{jit}-w_{sjt}s_{jit},
\end{align}
internalising demand for machines as a function of technology and the price of machines. 
The government subsidises machine production by $\zeta_{jt}$ which is chosen to correct for the monopolistic structure of machine markets. Profits are confiscated by the government to simplify notation.

\paragraph{Research and technology}
Technological growth is driven by research and spillover effects. The marginal product of research determines the amount of researchers employed and, thus, growth. 
The law of motion of technology in sector $J$ is modeled as
\begin{align}
A_{jit}=A_{jit-1}\left(1+\gamma\left(\frac{s_{jit}}{\rho_j}\right)^\eta\left(\frac{A_{t-1}}{A_{jt-1}}\right)^\phi\right).
\end{align}
Where I define
\begin{align}
A_{jt}=\int_{0}^{1}A_{jit}di,\\
A_{t}=\frac{\rho_fA_{ft}+\rho_gA_{gt}+\rho_n A_{nt}}{\rho_f+\rho_g+\rho_n}.
\end{align}
The parameters $\rho_j$ capture the number of research processes by sector. This ensures that returns to scale refer to the ratio of scientists to research processes \citep{Fried2018ClimateAnalysis}. In the baseline calibration $\eta$ is smaller unity implying diminishing returns to research within sector following \cite{Fried2018ClimateAnalysis}. 
The private benefits of research for machine producers diverges from the social benefits as they do neither observe the effect of today's research on tomorrow's productivity nor the positive spillovers for all research sectors captured by the term $\left(\frac{A_{t-1}}{A_{jt-1}}\right)^\phi$ with $\phi>0$. 

The marginal return to research in equilibrium is determined as
\begin{align}
w_{sjt}= \frac{\eta \gamma \left(\frac{A_{t-1}}{A_{jt-1}}\right)^\phi (1-\alpha_j)s_{jt}^{\eta-1}\Pi_{jt}}{\gamma_{jt}\rho_j^\eta},
\end{align}
where $\Pi_{jt}$ denotes returns in sector $J$. Abstracting from price effects, revenues are increasing in labour and hence skill supply which again is affected by income tax progressivity. The term $\gamma_{jt}$ refers to the growth rate in sector $j$.

The supply of scientists is endogenous in my model. With this choice, I depart from the standard assumption of a fixed supply of scientists in the literature on directed technical change \cite{Acemoglu2012TheChange, Fried2018ClimateAnalysis}.  Modeling the supply of researchers flexibly gives more freedom for the planner to choose lower growth levels: no a-priori fixed amount of research has to be employed. Furthermore, I do not assume free movement of scientists which simplifies the numeric calculation of equilibria in a model without balanced growth. 
Scientists are risk  neutral and behave according to 
\begin{align}
\underset{s_{jt}}{\max}\ \ & w_{jst}s_{jt}-\chi_s \frac{s_{jt}^{1+\sigma_s}}{1+\sigma_s}\\
s.t. \ \ & s_{jt}\leq \bar{H}.
\end{align}

I assume that all income from science is confiscated by the government to again facilitate notation. The assumption that scientists are risk neutral, introduces an additional externality as scientists do not internalise the social value of their research on society which is shaped by the shadow value of income. The advantage of this specification is that it prevents income tax parameters to affect the supply of scientists allowing to focus on the supply of hours by workers and consumption as the channels through which income taxes affect emissions. 
\begin{comment}
\paragraph{Impossibility of reaching target in laissez-faire with exogenous growth}
\tr{Note that this is wrong! There is an option for the gov to affect inflation which then redirects demand.}
Note that with exogenous growth in each sector there is no possibility for the government to stop emissions from growing, since production of the dirty good is essential for the consumption good (no perfect substitution: $\varepsilon<\infty$). To meet the emission target, the government either needs to affect the growth rate in the economy; i.e., $\upsilon_j$ is a choice variable, or work and consumption need to be set to zero; or the emission target has to be defined in relative terms. The latter possibility contradicts the Paris Agreement which is concerned with absolute emissions.  
I therefore assume, that the government can change the growth rate.

The government chooses the growth rate in each sector, taking into account that research is constrained by an exogenous  amount of scientists
\begin{align}
\upsilon_{ct}+\upsilon_{dt}\leq\Upsilon
\end{align}
\end{comment} 
  
\paragraph{Markets}
In equilibrium I require markets to clear. I explicitly model markets for skill and the final consumption good:
\begin{align}
z_h h_{ht}&=h_{hft}+h_{hgt}+h_{hnt},\\
(1-z_h) h_{lt}&=h_{lft}+h_{lgt}+h_{lnt},\\
C_t&=Y_t-\int_{0}^{1}x_{fit}+x_{git}+x_{nit}di.
\end{align}
\paragraph{Government}

The government maximises social welfare defined as the sum of utilities in the economy, but it is constrained by meeting emission targets in line with the Paris Agreement. The government can use income taxes and corrective taxes levied on fossil sales to maximise social welfare. The planner solves:

\begin{align*}
\underset{\{\tau_{ft}\}_{t=0}^{T},\{\tau_{lt}\}_{t=0}^{T}}{max}&\sum_{t=0}^{T}\beta^t\left( u(C_{t}, h_{ht}, h_{lt})-\chi_s\frac{s_{ft}^{1-\sigma_s}}{{1-\sigma_s}}-\chi_s\frac{s_{gt}^{1-\sigma_s}}{{1-\sigma_s}}-\chi_s\frac{s_{nt}^{1-\sigma_s}}{{1-\sigma_s}}\right) \\
s.t.\ %& (1)\  \tau_{lt}(h_{ht}w_{ht}+h_{lt}w_{lt})=T_t\  \forall \ t\geq 0\\
& (1)\ \omega F_{t} -\delta \leq \Omega_t \ \hspace{3mm} \forall t \in\{0,T\}, \\ %\hspace{3mm} \text{(emission target)}\\
& (2)\ z_h\left(w_{ht}h_{ht}-\lambda_t \left(w_{ht}h_{ht}\right)^{1-\tau_{lt}}\right)+(1-z_h)\left(w_{lt}h_{lt}-\lambda_t\left(w_{lt}h_{lt}\right)^{1-\tau_{lt}}\right) +\tau_{ft}p_{ft}F_{t}  \\
& \hspace{5mm} +\sum_{j\in\{f,g,n\}}\left(\int_{0}^{I}\pi_{jit}di-\zeta_{jt}\int_{0}^{I}p_{jixt}x_{jit}di+w_{sjt}s_{jt}\right)= T_t \ \hspace{3mm} \forall \ t\in\{0,T\}, \\
%& (3)\ \upsilon_{ct}+\upsilon_{dt}\leq\Upsilon\  \forall \ t\geq 0\\
& (3)\ \text{behaviour of firms, scientists and households},\\
& (4)\ \text{feasibility}
\end{align*}

The first constraint is the emission target. I denote flow emissions in period $t$ by $\Omega_t$.  The parameter $\delta$ captures the capacity of the environment to reduce emitted $CO_2$ through sinks, such as forests and moors.  I assume that the sink capacity is constant.  This is a simplifying assumption. What is crucial qualitatively is the assumption that sinks are finite. Indeed, natural sinks and carbon capture and storage (CCS) technologies rely on the use of land \citep{VanVuuren2018AlternativeTechnologies} which is in limited supply. In addition, the need of land for food production makes land even scarcer especially in light of a growing population. The parameter $\omega$ determines greenhouse-gas emission in $CO_2$ equivalents caused by the fossil sector which is assumed to be the sole source of emissions in the model. % \tr{Read up in \cite{Hassler2016EnvironmentalMacroeconomics} what possibilities there are in the literature}
%Hence, under the emission target it has to hold that $Y_{nt}=\frac{\delta+E_t}{\kappa}$ assuming that the analysis starts in 2020.
I require the government budget, constraint (2), to balance and set government transfers to zero, $T_t=0\ \forall t\in\{0,T\}$. The scale parameter on income taxes, $\lambda_t$, adjusts to balance the budget.
The government generates revenues from taxing labour income and fossil sales and from confiscating profits from machine producers and wages of scientists. It has to finance the subsidy on machine production. In equilibrium, profits from machine producers, scientists' incomes and the subsidy cancel. 



%
\section{Tractable model}
I present a simple model to provide intuition how income tax progressivity affects emissions. In this simplified version of the model, there is only one skill and the neutral sector is passive, so that $L_g+L_f=h$. 

Labour demand by the green and fossil sector read
\begin{align*}
w=(p_f(1-\tau_f))^{\frac{1}{1-\alpha_f}}(1-\alpha_f)\alpha_f^{\frac{\alpha_f}{1-\alpha_f}}A_f
\end{align*}


\section{Model}
This section presents a simple representative household model. The representative household provides two skills: high and low. 
The model builds on \cite{Acemoglu2012TheChange} and \cite{Heathcote2017OptimalFramework} and adds a sector specific labour input good. I abstract from directed technical change in this version of the model.

\paragraph{Households}
% the rep agent
The economy behaves as if there was a representative household. The household chooses between high and low-skilled labour.  In equilibrium, the high-skilled labour receives a higher wage rate so that the household is indifferent between which skill type to provide.

\begin{align}
U=\underset{\{\{c_{t}\}_{t=0}^{\infty}, \{h_{lt}\}_{t=0}^{\infty}, \{h_{ht}\}_{t=0}^{\infty}\}}{max}&
\sum_{t=0}^{\infty}\beta^t u(c_{t}, h_{lt}, h_{ht})\\
%U_{s}=\underset{\{c_{st}\}_{t=0}^{\infty}, \{h_{st}\}_{t=0}^{\infty}}{max}&\sum_{t=0}^{\infty}\beta^t u_s(c_{st}, h_{st}; S_t)\\
s.t.& \ \ c_{t}p_{t}=% (1-\tau_{lt})(h_{ht}w_{ht}+h_{lt}w_{lt})+T_t\\ 
\lambda \left(h_{ht}w_{ht}+h_{lt}w_{lt}\right)^{1-\tau}\\
\ & h_{ht}+h_{lt}\leq \bar{H}_t\\
\ & h_{st}\geq 0 \ \forall s\in \{l,h\}
\end{align}
The household experiences utility costs from  providing the high skill.
\begin{align}
	u(c_{t}, h_{lt}, h_{ht})&= %\frac{c_t^{1-\gamma}}{1-\gamma}
	\log(c_t)-\frac{(h_{lt}+\zeta h_{ht})^{1+\sigma}}{1+\sigma}%-v(h_{ht+1})%,\\
	%\text{where}\  & v(h_{ht+1})=\left\{\begin{array}{lll}\zeta& \hspace{2mm} \text{if} \hspace*{2mm}  h_{ht+1}> 0, &\\
%0  &\hspace{2mm}\text{if}\hspace{2mm}  h_{ht+1}= 0.
%	\end{array}
%	\right. 		
\end{align}
The positive parameter $\zeta>1$ implies a higher marginal disutility for high-skilled than low-skilled labour. As a result, in equilibirum,  
high-skilled labour earns a premium to compensate the representative household for the higher disutility. %When labour income gets taxed, the returns to learning reduce and skilled labour becomes scarcer on impact. 
%The log-utility from consumption ensures balanced-growth-path compatibility of hours worked. However, this makes the reduction in hours supplied independent of the wage rate. Still, the extensive margin through learning should remain.\footnote{\ \textcolor{sonja}{With log-utility the income and substitution effects of the wage rate on hour supply cancel. With GHH preferences, in contrast, the wealth effect cancels.} }
I define the variable $H_t:=\zeta h_{ht}+h_{lt}$ which facilitates the derivation of results. 

\paragraph{Production}
There are two production sectors: a clean and a dirty one, indexed by $c$ and $d$. Sector-specific goods are imperfect substitutes in the production of the final consumption good.\footnote{\ As a result, production is never perfectly clean.}  
The final good producing sector is perfectly competitive:
$Y_t=\left(Y_{ct}^{\frac{\varepsilon-1}{\varepsilon}}+Y_{dt}^{\frac{\varepsilon-1}{\varepsilon}}\right)^\frac{\varepsilon}{\varepsilon-1}$. 
I take the composite good as the numeraire so that $\left[p_{dt}^{1-\varepsilon}+p_{ct}^{1-\varepsilon}\right]^{\frac{1}{1-\varepsilon}}=1$.

In both sectors, a unit mass of competitive firms, indexed by $i$, produces a sector-specific intermediate good. All firms use machines, $x_{jit}$ and a labour input good, $L_{jt}$ to produce according to: %\footnote{\ For now I abstract from a natural resource.} 
\begin{align*}
&Y_{dt}= L_{dt}^{1-\alpha}\int_{0}^{1}A_{dit}^{1-\alpha}x_{dit}^{\alpha} di,\ \hspace{2mm} Y_{ct}= L_{ct}^{1-\alpha}\int_{0}^{1}A_{cit}^{1-\alpha}x_{cit}^{\alpha} di.
\end{align*}

The labour input good is produced by a perfectly competitive and sector-specific labour industry according to: 
\begin{align}
L_{jt}=l_{jht}^{\theta_j}l_{jlt}^{1-\theta_j}, \ for \ j \in\{c,d\},
\end{align}
where $\theta_c>\theta_d$ so that the clean sector's labour input has a higher share of high-skilled labour. 

A perfectly competitive sector produces machines, $x_{ijt}$, and sells them to final good firms in the respective sector at price $p_{ijt}$. I assume that the costs to produce one machine, $\psi$, are homogeneous across firms. % It follows that $p_{ijt}=\psi$.


\paragraph{Technological progress}
Technological progress is exogenous:
\begin{align}
A_{ijt+1}=(1+\upsilon_{jt}) A_{ijt}\ for \ j \in\{c,d\}. 
\end{align}

\begin{comment}
\paragraph{Impossibility of reaching target in laissez-faire with exogenous growth}
\tr{Note that this is wrong! There is an option for the gov to affect inflation which then redirects demand.}
Note that with exogenous growth in each sector there is no possibility for the government to stop emissions from growing, since production of the dirty good is essential for the consumption good (no perfect substitution: $\varepsilon<\infty$). To meet the emission target, the government either needs to affect the growth rate in the economy; i.e., $\upsilon_j$ is a choice variable, or work and consumption need to be set to zero; or the emission target has to be defined in relative terms. The latter possibility contradicts the Paris Agreement which is concerned with absolute emissions.  
I therefore assume, that the government can change the growth rate.

The government chooses the growth rate in each sector, taking into account that research is constrained by an exogenous  amount of scientists
\begin{align}
\upsilon_{ct}+\upsilon_{dt}\leq\Upsilon
\end{align}
\end{comment} 
  
\paragraph{Government}

The government maximises social welfare but is constrained by meeting emission targets in line with the Paris Agreement. Furthermore, the government does not have corrective taxes at its disposal. Instead, only already established distortionary labour taxes are available. The planner solves:

\begin{align*}
\underset{\{\tau_{t}\}_{t=0}^{\infty}}{max}&\sum_{t=0}^{\infty}\beta^t u(c_{t}, h_{ht}, h_{lt})\\
s.t.\ %& (1)\  \tau_{lt}(h_{ht}w_{ht}+h_{lt}w_{lt})=T_t\  \forall \ t\geq 0\\
& (1)\ \kappa Y_{nt} -\delta \leq E_t \  \forall \ t\geq 0\hspace{3mm} \text{(emission target)}\\
& (2)\ (w_{ht}h_{ht}+w_{lt}h_{lt})-\lambda_t (w_{ht}h_{ht}+w_{lt}h_{lt})^{1-\tau_{t}} = G_t\hspace{3mm} \text{(gov. budget)}\\
%& (3)\ \upsilon_{ct}+\upsilon_{dt}\leq\Upsilon\  \forall \ t\geq 0\\
& (2)\ \text{behaviour of firms and households}\\
& (3)\ \text{feasibility}
\end{align*}

$E_t$ are flow emissions per year.  The parameter $\delta$ captures the capacity of the environment to reduce emitted $CO2$ through sinks, such as forests and moors.  For simplicity, I assume that the regeneration rate is constant. $\kappa$ determines greenhouse-gas emission in CO2 equivalents caused by production. % \tr{Read up in \cite{Hassler2016EnvironmentalMacroeconomics} what possibilities there are in the literature}
%Hence, under the emission target it has to hold that $Y_{nt}=\frac{\delta+E_t}{\kappa}$ assuming that the analysis starts in 2020.
The government generates revenues from taxing labour income and redistributes to run a balanced budget. 


\paragraph{Markets}
Three markets are modelled explicitly: a market for final goods, and one for each type of skill.
\begin{align*}
\text{final good}\hspace{4mm}& Y_{t}=c_t+\psi\left(\int_{0}^1x_{idt}di+\int_{0}^1x_{ict}di\right)+ G_t\\
%\end{align*}
% %I study two cases one with full disposal, i.e., $\iota=0$, and one without $\iota=1$. In the first scenario, the price of the final good is determined by the market clearing condition as Walras' law does not hold. 
%\begin{align*}
\text{high skill:}\hspace{4mm}& l_{hct}+l_{hdt}=h_{ht}\\
\text{low skill:}\hspace{4mm}&l_{lct}+l_{ldt}=h_{lt}.
\end{align*}



%\section{Results}
%\subsection{Balanced Growth path}


%It follows that the labour input good does not grow, since hours worked are constant and transitional dynamics are ruled out by definition.\footnote{\ In a subsection below, I prove this claim.}

%I write the evolution of the model as a function of growth rates and initial conditions $A_{c0}, A_{d0}$. I also impose that policy variables, $\upsilon_{c}, \upsilon_{d}, \tau_l, \lambda$, are constant on the balanced growth path. 
%\begin{align}\label{eq:price_ratio_labourinput}
%	\frac{p_c}{p_d}= \left(\frac{A_d}{A_c}\frac{z_d}{z_c}\zeta^{\theta_c-\theta_d}\right)^{1-\alpha}& \text{(optimality labour input production)}
%\end{align}

%\begin{prop}[Skill scarcity and prices, assuming $\theta_c>\theta_d$] 
%	
%	The effect of skill scarcity on relative prices depends on the substitutability of goods. When goods are complements, $\varepsilon<1$, the price of the more skill-intense clean good rises with the disutility of high-skill labour, while the price of the dirty good falls.
%	Production of the clean good becomes more expensive. 
%	
%	When goods are substitutes, $\varepsilon>1$, then the clean good becomes cheaper the scarcer high skills and the price of the dirty good rises. Still, production of the clean good becomes more expensive, but it can be substituted by the dirty good. As the clean good becomes more expensive, demand shifts from the clean to the dirty good and market clearing implies a drop in the clean goods price. In total the general equilibrium effect outweighs the rise in production costs. 
%\end{prop}

%\paragraph{Intution}
%Consider equation
%The ratio of low labour in the dirty versus the clean sector negatively depends on the distutility of high-skill labour, when goods are substitutes.
%\tr{Continue later}


%\section{Theoretical Results}\label{sec:theory}

In this section, I show, first, the mechanisms which drive changes in emissions, and, second, that under the assumption of a representative agent, labour taxes can only affect emissions through the level of production; the output ratio is independent of fiscal policies. 
%In what follows, I present two  I am less interested in the long-run growth aspects of the economy, but rather on the transitional mechanisms across sectors. 


\subsection{Emission growth}
Emissions are a constant fraction of dirty output; they, thus, grow at the same rate.
Dirty output growth is given by
\begin{align*}
	\frac{Y_d'}{Y_d}=\left(\frac{p_d'}{p_d}\right)^{\frac{\alpha+(1-\alpha)(1-\varepsilon)}{1-\alpha}}\frac{A_d'}{A_d}\frac{H'}{H}.
\end{align*}
For a derivation and a full solution to the model see appendix section \ref{app:solu}. The dash indicates next period variables.
The effect of sector-specific inflation on output captures, on the one hand, the positive effect of a higher demand for machines when the price for the respective good grows. This raises the marginal product of labour so that the dirty sector demands more labour; reflected by the term $\frac{\alpha}{1-\alpha}$ in the exponent. In addition,  a rise in the dirty good's price increases the marginal profit the dirty sector generates from increasing labour input one-for-one, this is captured by the exponent of 1. On the other hand,  the rise in the dirty good's price lowers demand for dirty output by final goods producers, even more so the more goods are substitutes, captured by the exponent $\varepsilon$. The elasticity of dirty output with respect to dirty sector inflation is positive, when the reduction in demand by final good producers, $\varepsilon$,  is smaller than the rise in the marginal profit of labour, $\frac{1}{1-\alpha}$. 
All else equal, a rise in dirty productivity, $A_d$, increases dirty output, as does a rise in aggregate (disutility-weighted) labour supply, $H$.

Since prices in this simple model are a function of total factor productivity only, the government can only affect dirty output growth through total labour supply by households. As the ratio of dirty to clean goods demanded by final good producers only depends on prices, the output ratio is irresponsive to the progressivity of the tax schedule.%The percentage change in labour input goods by sectors are equivalent. This together with prices being independent of skill supply implies that the output ratio of sectors is unaffected by tax progressivity.
%\footnote{\ This is directly obvious from the ratio demand by final good producers, which is only a function of the price ratio. IN OTHER MODELS PRICES MAY DEPEND ON LABOUR SUPPLY1}


	\begin{comment}
	To see this write:
\begin{align}
	\frac{d\left(\frac{Y_d}{Y_c}\right)}{d \tau_l}=\frac{Y_d}{Y_c}\left(\frac{\frac{dY_d}{Y_d}}{d \tau_l}-\frac{\frac{dY_c}{Y_c}}{d \tau_l}\right)=0
\end{align}
and observe that the percentage change in sector output is homogeneous. 
\begin{align}
	\frac{1}{Y_d}\frac{dY_d}{d \tau_l}= \frac{1}{L_d}\frac{d L_d}{d \tau_l}=\frac{1}{H}\frac{d H}{d \tau_l}\ \text{and} \ \frac{1}{Y_c}\frac{dY_c}{d \tau_l}= \frac{1}{L_c}\frac{d L_c}{d \tau_l}=\frac{1}{H}\frac{d H}{d \tau_l}.
\end{align}
\textbf{}
content...
\end{comment}

\begin{prop}[Effect of $\tau_l$ on dirty output]
	In the representative agent model with log utility, tax progressivity does not affect the equilibrium ratio of sector production. Fiscal policy can only lower dirty output by reducing aggregate labour supply.
\end{prop}

\subsection{Excursus: Model with Inequality and MaCurdy preferences}
Following \cite{Boppart2019LaborPerspectiveb}, I look at a MaCurdy preference specification which allows for a slightly higher income than substitution effect in response to a change in productivity. In this section, I focus on a heterogeneous agent version of the model. Households differ in the exogenous level of skill they can provide.\footnote{\ \tr{In the representative agent model, the household is indifferent what kind of skill to supply, the optimal ratio of labour supply is independent of fiscal policy. Also when considering a representative family, equalising consumption across family members'  makes the income effect marginally homogeneous.} } As a result, the response of hours worked to a change in tax progressivity is income dependent. 
The utility of a household with skill level $s$ reads
\begin{align}
	u_s(c_{st},h_{st})=
	\frac{c_{st}^{1-\gamma}}{1-\gamma}-
	\frac{h_{st}^{1+\sigma}}{1+\sigma}.%-(\bar{S}-S_t).
\end{align}
The household's optimality conditions imply the following policy functions for consumption and hours worked:
\begin{align}
	\log(c_{st})=& \frac{1}{\sigma +\gamma +\tau(1-\gamma)}\left[(1-\tau)\log(1-\tau)+(1+\sigma)\log\left(\frac{\lambda}{p_{t}}\right)+(1+\sigma)(1-\tau)\log(w_{st})\right]\\
		\log(h_{st})=& \frac{1}{\sigma+\gamma+\tau(1-\gamma)}\left[\log(1-\tau)+(1-\gamma)\log\left(\frac{\lambda}{p_{t}}\right)+(1-\gamma)(1-\tau)\log(w_{st})\right]\label{eq:labour_supp}.
\end{align}
Note that the multiplier is positive for plausible parameters values.\footnote{\ HSV 2014 estimate $\gamma= 1.7$ and $\sigma=2$.} %Hence, a rise in the elasticity of post- to pre-tax income (equivalent to a smaller $\tau$) raises hours worked irrespective of income (the first summand in brackets). There is a second term including 
Focus on the last summand in equation \ref{eq:labour_supp}.
By allowing for a coefficient of relative risk aversion, $\gamma$, different from 1, the hourly wage rate, $w_{st}$, shapes the impact of the tax progressivity $\tau$ on labour supply and consumption. The income and the substitution effect do not cancel as in the log-utility version studied in \cite{Heathcote2017OptimalFramework}. 
 At the parameter values found in the literature, e.g. $\gamma= 1.7$ and $\sigma=2$, the uncompensated wage elasticity is negative since the income effect dominates under a progressive tax system.  To see this note that  $\frac{d log(h)}{d log(w)}= \frac{(1-\gamma)(1-\tau)}{\sigma+\tau(1-\gamma)+\gamma}<0$ for all levels of admissible tax progressivity $\tau<1$.  
%Whenever the coefficient of relative risk aversion is smaller than one, $\gamma<1$, under the assumption of a concave utility in consumption, $\gamma>0$, and a progressive tax system, i.e. $\tau>0$ ( and note that $\tau<1$, since otherwise the post tax income depends negatively on pre-tax income )  the substitution effect dominates and the household 
%	increases labour supply as its hourly wage rate rises. Proof: The tax adjusted uncompensated wage elasticity reads:. Assuming a progressive tax system, the denominator is positive whenever utility is concave in consumption, i.e., $\gamma>0$. Hence, in sum, $\gamma\in\left(0,1\right)$ implies an increase in labour supply as the hourly wage rate rises under a progressive labour income tax. 	
%	In contrast,

The marginal impact of a change in tax progressivity on hours worked depends on the wage rate. 	
Intuitively, $1-\tau$ is the elasticity of post to pre-tax income. As this elasticity increases, that is, a decline in $\tau$ and the system becomes less progressive, households with a higher wage rate reduce their labour supply more. A marginal increase in progressivity raises labour supply by richer households more.  

Households which receive a higher income per hour worked, $w_s$, are more responsive to an increase in tax progressivity. As a consequence, a change in tax progressivity alters supply of distinct skills heterogeneously on impact. 

%\section{Theoretic Results}
\subsection{An emission target calls for a reduction policy under likely parameter values}

\paragraph{Effect of tax progressivity on energy output ratios}


\subsection{Tax progressivity affects the composition of total output}
In the model, tax progressivity affects the innovation decision due to heterogeneous effects on skill supply. 
The optimal ratio of skills supplied by the household is
\begin{align}
\frac{h_{ht}}{h_{lt}}=\left(\frac{w_{ht}}{w_{lt}}\right)^\frac{1-\tau_{lt}}{\tau_{lt}+\sigma}.
\end{align}
The semi-elasticity of the ratio of aggregate skill supply, defined as $\frac{H_h}{H_l}:=\frac{z_hh_h}{z_lh_l}$, in response to a change in income tax progressivity is then given by
\begin{align}
\frac{d\log\left(\frac{H_h}{H_l}\right)}{d\tau_l}=-\frac{1+\sigma}{(\tau_l+\sigma)^2}\log\left(\frac{w_{h}}{w_l}\right). \end{align}
The direct effect, with fixed prices is negative given a positive wage premium for high skill labour. Hence, a higher tax progressivity implies a decline in the relative supply of high skill labour. 

\paragraph{Effect on the externality }
\begin{prop}Assume that (1) the income share of high-skilled labour exceeds that of low-skilled labour, $\frac{H_lw_l}{H_hw_h}<1$, (2) high-skill labour earns a premium, $\frac{w_h}{w_l}>1$, and (3) energy inputs are substitutes, $\varepsilon_e>1$, then
a rise in income tax progressivity increases the share of fossil energy in the economy when the sum of labour shares in fossil and the green sector is below unity, $\theta_f+\theta_{g}<1$, (intuitively, high skill has a lower income share than low skill). When the sum of income shares of high-skill labour is sufificiently high, and necessarily above unity, a rise in tax progressivity lowers the share of fossil to green energy, $\frac{dlog\left(\frac{F_t}{G_t}\right)}{d\tau_l}<0$.
\end{prop}
\footnote{\textit{Proof}\\ 
	The equation follows from iteratively applying skill demand by labour input producers and the skill market clearing conditions and substituting low-skill supply by the households optimality condition. This gives the following relation of high-skill hours employed in the green sector to total high-skill supply:  
	\begin{align*}
	\frac{h_{hg}}{H_h}=\frac{1-\left(\frac{w_l}{w_h}\right)^\frac{1+\sigma}{\sigma+\tau_{lt}}\frac{z_l}{z_h}\frac{\theta_f}{1-\theta_f}}{1-\frac{\theta_f}{\theta_g}\frac{1-\theta_g}{1-\theta_f}}.
	\end{align*}}
For an intuition consider the output ratio as a function of taxes and the wage ratio in equilibrium:
\begin{align*}
\frac{F_t}{G_t}=\left(\frac{(1-\tau_f)(1-\alpha_f)/(1-\alpha_g)}{\left(\frac{w_l}{w_h}\right)^{\frac{1+\sigma}{\sigma+\tau_l}}\frac{z_l}{z_h}\frac{\theta_f}{1-\theta_f}-\frac{1-\theta_g}{1-\theta_f}}\right)^\frac{1}{\varepsilon_e-1}
\end{align*}

As tax progressivity increases, high-skill supply reduces relative to low-skill supply. 
When high skill has a relatively smaller share in the fossil sector than low-skill in the green sector, the reduction in high skill supply makes 

This translates to an increase in low-skill labour employed in the fossil sector and high skill in the fossil sector rises proportionately leading to a fall in high skill in the green sector by market clearing. When 



\subsection{Growth in the dirty sector has to stop}
Growth in the dirty sector eventually has to stop given the net-zero emission target. Assuming no possibility to increase capture and storage technologies, this would be the case by 2050. Otherwise, it suffices to assume a limit to C02 capture-storage technology. This implies a condition on taxes to counter growth in the green sector. I summarise that result in the following proposition.

\begin{prop}Assume that energy inputs are substitutes. Then, growth in the green sector has to be offset by a rise in the fossil tax. Alternatively, %when high skill is in sufficiently high demand, $\theta_f+\theta_g>>1$, then a rise in low skill supply counteracts a rise in fossil growth, i.e. a more progressive tax. If high skill is in lower demand,  $\theta_f+\theta_g<1$, then a drop in the low-to-high skill ratio, a more regressive tax, can offset green growth. 
	the ratio of low-to-high skill income has to rise, that is, at a positive wage premium for high skill labour a more progressive tax is required. 
\end{prop}

\begin{corollary}
	As a subsidy to green innovation boosts growth in the green sector, it must be counteracted by a stronger corrective tax or a respective change in income tax progressivity. In other words, a green subsidy contributes to growth pressure in fossil energy. The more so the less substitutable goods are.
\end{corollary}

\begin{corollary}
	A higher progressivity of the income tax contributes to keeping fossil production low. A double dividend of redistribution: in addition to lowering inequality it lowers emissions.
\end{corollary}

\tr{\textbf{Next: find an expression for wage ratio as a function of growth rates \ar can discuss exogenous growth case!}}<- potentially relevant stuff
\subsection{Calibration}\label{subsec:calib}

%\tr{Inflation data \url{/home/sonja/Documents/projects/subjective_BN/writing/mainmain}}

Section \ref{sec:ems} derives and discusses the emission target. 
Secton \ref{sec:modpar} calibrates the remaining model parameters.

\subsubsection{Emission target}\label{sec:ems}
To calibrate the emission target, I consider CO$_2$ emissions only and abstract from other greenhouse gasses since carbon is the most important pollutant with the highest mitigation potential \citep[p.29]{IPCC2022}.
%	 WG3 IPCC report (p.37) \textbf{\textit{The trajectory of future CO$_2$ emissions plays a critical role in mitigation, given CO$_2$ long-term impact and dominance in total greenhouse gas forcing}}. Furthermore, \textbf{The main reason is that scenarios reduce non-CO$_2$ greenhouse gas emissions less than CO$_2$ due to a limited mitigation potential (see 3.3.2.2)} p.34 in foxit, 3-26 in chapter 3}.  
The most recent IPCC report \citep{IPCC2022} formulates a reduction of global CO$_2$ emissions in the 2030s by 50\% relative to 2019 and net-zero emissions in the 2050s  as essential to meeting the 1.5°C climate target.\footnote{ ``\textit{Mitigation pathways limiting warming to 1.5°C [...] reach 50\% reductions of CO$_2$ in the 2030s, relative to 2019, then reduce emissions further to reach net zero CO$_2$ emissions in the 2050s [...] (\textnormal{medium confidence}).}" \citep[p.5, Chapter 3]{IPCC2022} }  Furthermore, the report stipulates a remaining global net CO$_2$ budget of 510 GtCO$_2$ %($\approx$ 510,000 million metric tons of CO$_2$) 
from 2020 to the net-zero phase starting from 2050 \citep[p.5, Chapter 3]{IPCC2022}. 
To deduce an emission target for the US, further assumptions on the distribution of mitigation burdens have to be made. I follow \cite{RobiouDuPont2017EquitableGoals} who consider 5 distinct principles of distributive burden sharing. I use an \textit{equal-per-capita} approach according to which emissions per capita shall be equalized across countries. 
 Appendix \ref{app:calib} details the calculation of the emission target. 
Figure \ref{fig:emlimit}  visualizes the resulting emission limit for the US starting from 2020. The value for 2015-2019 refers to observed emissions.

% data
%, 2022: Mitigation pathways compatible with long-term goals. In IPCC, 2022: Climate
% Change 2022: Mitigation of Climate Change. Contribution of Working Group III to the Sixth
% Assessment Report of the Intergovernmental Panel on Climate Change [P.R. Shukla, J. Skea, R.
% Slade, A. Al Khourdajie, R. van Diemen, D. McCollum, M. Pathak, S. Some, P. Vyas, R. Fradera, M.
% Belkacemi, A. Hasija, G. Lisboa, S. Luz, J. Malley, (eds.)]. Cambridge University Press, Cambridge,
% UK and New York, NY, USA. doi: 10.1017/9781009157926.005
% 


%\begin{table}[hh!!!!!]
%	\begin{center}
%		\captionsetup{width=0.9\textwidth}
%		\caption{Net CO$_2$ emission limit for the US by model period}
%		\label{tab:emlimit}
%		\begin{tabular}{l|rrrrrrrr}
	%			\hline 
	%			\hline
	%			Periods&20-24&25-29&30-34&35-39&40-44&45-49&50-80\\
	%			Limits in GtCO$_2$&3.6079&3.5396&3.4798&3.4245&3.3697&3.3164&0\\
	%			\hline \hline
	%			
	%		\end{tabular}
%	\end{center}
%\end{table}	

\begin{figure}
\caption{Net CO$_2$ emission limit in gigatons  (Gt)}\label{fig:emlimit}
%	\graphicspath{{../../codding_model/own_basedOnFried/optimalPol_010922_revision/figures/all_13Sept22_Tplus30/}{../../codding_model/own_basedOnFried/optimalPol_010922_revision/figures/all_13Sept22/}}
\includegraphics[width=0.4\textwidth]{../../../subjective_BN/codding_model/own_basedOnFried/optimalPol_010922_revision/figures/all_13Sept22_Tplus30/Emnet.png}
\end{figure}
%  In summary, I calibrate the net-emission target vector for the period from 2030 to 2080 as 
% $\omega_{2030-2050}$= 2.4899Gt and $\omega_{2050-2080}$= 0Gt.
%\footnote{Another alternative 
%} 
% I assume here that each country contributes to the global reduction by the same percentage of 50\% of its own emissions.\footnote{ Alternatively, one could assume that the global reduction is allocated in the same share as countries contributed to global emissions in 2019. This would result in an even stricter target for the US which contributed almost 20\% to global greenhouse gas emissions in 2019 (based on own calculations where total emissions come from the EIA global greenhouse gas information, to be found here \url{https://www.iea.org/reports/global-energy-review-2021/CO$_2$-emissions}).}
% Starting from 2050, the net-emission target is zero. 
% sinks and emission from fossil sector

\paragraph{Discussion}
The reduction in net CO$_2$ emissions necessary to meet the emission limit relative to 2019 emissions in the US  is substantial. It amounts to around 85\%. The result is not only explained by the global emission limit but also by the US emitting beyond its population share in 2019. In 2019, US emissions accounted for 10.44\% of global net emissions while the population share of the US was 4.3\%. Hence, even without an emission limit, the US would have to reduce emissions according to the \textit{equal-per-capita} principle.

The necessary reduction in net CO$_2$ emissions found in this calibration exceeds political goals. On April 22, 2021, President Biden announced a 50-52\% reduction in net greenhouse gas emissions relative to 2005 levels in 2030 % \footnote{ If pollutants were to be reduced by an equal share, this means a 50-52\% reduction in net CO$_2$ emissions.} 
and net-zero emissions no later than 2050.\footnote{ Source: \href{https://www.whitehouse.gov/briefing-room/statements-releases/2021/04/22/fact-sheet-president-biden-sets-2030-greenhouse-gas-pollution-reduction-target-aimed-at-creating-good-paying-union-jobs-and-securing-u-s-leadership-on-clean-energy-technologies/}{https://www.whitehouse.gov/briefing-room/statements-releases/2021/04/22/}, retrieved 14 September 2022.} 
However, relative to 2019, the planned reduction for 2030 corresponds to a 38\% decline only.
The resulting net emissions in the US would then amount to 103.21 Gt.\footnote{ This calculation assumes emissions where left at 2019-levels until 2030 and then lowered to the Biden target from 2030 to 2050 and net-zero afterwards.} This is roughly 5 times the budget acceptable for the US,  if the global remaining carbon budget was allocated on a \textit{equal-per-capita} basis.\footnote{ The remaining net carbon budget for the US based on its population share is 20.738Gt for the period from 2020 to 2050.} % This amounts to 27\% of emissions which the US would emit if annual emissions equaled 2019 net emissions.}  

\subsubsection{Model parameters}\label{sec:modpar}

\paragraph{Functional forms}
I assume the following functional form of period utility:
\begin{align*}
u(C_t,H_{t}, )= \log(C_t)-\chi\frac{H_{t}^{1+\sigma}}{{1+\sigma}}.
\end{align*}
The log-utility implies constant hours worked over time in a laissez-faire allocation since income and substitution effects cancel. This simplifies the analysis. %\footnote{  On the other hand, recent research has shown that substitution and income effects of the wage rate most likely do not cancel. \cite{Boppart2019LaborPerspectiveb} argue for a slightly higher income effect so that hours fall over time as productivity increases. I plan to conduct a sensitivity analysis by assuming the utility specification suggested in their paper.}

%Most likely, the continuous rise in carbon taxation over time lowers the wage rate and labor supply increases. A lack of lump-sum rebates would most likely aggravate the inefficiency of hours worked. %Nevertheless, as shown in the analytical part for a general utility function, some reductive policy is required to implement the efficient allocation. But, compared to the laissez-faire scenario, hours will rise. 

\paragraph{Parameter values}
To calibrate the model, I proceed in three steps. First, I set certain parameters to values found in the literature. Second, I calibrate the remaining variables requiring that equilibrium conditions and target equations hold. Third, parameters relating production and emissions are chosen. \autoref{tab:calib2} summarizes the calibrated parameter values.

I calibrate the model to the US in the baseline period from 2015 to 2019. Using this calibration approach, it is not ensured that the economy is on a balanced growth path. However, the goal of this paper is to study necessary interventions to meet an absolute emission limit. Therefore, %in contrast to a relative reduction objective, 
it is important to capture whether the economy is transitioning, for example, to %a balanced growth path with
a higher fossil share. The optimal dynamic policy has to counteract these forces. %These transitions are relevant for the dynamic policy. 
%To differentiate model dynamics from policy effects, I take care to interpret results as deviations from the economy without policy intervention. 


In the first step, I mainly rely on \cite{Fried2018ClimateAnalysis} to calibrate the parameters governing research processes, $\eta, \rho_F,\rho_N, \rho_G, \phi, \gamma, S $, and production, $\varepsilon_e, \varepsilon_y, \alpha_F, \alpha_G, \alpha_N$. The labor share in the green sector is remarkably low with $\alpha_G=0.91$. This diminishes the significance of labor supply for green innovation and production. Furthermore, fossil and green energy are no close substitutes with $\varepsilon_e=1.5$ so that the cap on fossil energy cannot be fully substituted for by green energy.
Returns to research are decreasing with $\eta=0.79<1$. This makes extreme distributions of researchers across sectors unproductive. The non-energy sector is the biggest research sector with $\rho_N=1$ and $\rho_F=\rho_G=0.01$. 
The utility parameters, $\beta, \sigma$, are set to $0.984^5$ and $0.75^{-1}$ following \cite{Barrage2019OptimalPolicy} and \cite{Chetty2011AreMargins}, respectively. The business-as-usual policy is set to $\tau_\iota=0.24, \tau_F=0$, where I borrow the tax rate from \cite{Barrage2019OptimalPolicy}. 
%The period over which the government maximizes, T, is chosen to focus on the population living during the transition to the net-zero emission limit. 
%One can think of the T as the periods under the regency of the government. I set T to 11 so that the planner 
%explicitly derives  allocations and polices for 55 years. In their overlapping-generations model, \cite{Kotlikoff2021MakingWin} use the same number to calibrate the working life of a household as it captures the years a household is typically active in economic markets. 
%\tr{ Regency: T=11=55 years, and explicit optimization over T+1 periods. 55 is a suggests to be a sensible number for the explicit optimization interval.  }

In the second step, I calibrate the weight on energy in final good production by matching the average expenditure share on energy relative to GDP over the period from 2015 to 2019 taken from the US Energy Information Administration \citep[][Table 1.7]{EIAEnergy}. The expenditure share equals 6\%. The resulting weight on energy is $\delta_y=0.30$. %\footnote{ Note that in difference to \cite{Fried2018ClimateAnalysis} I raise the weight on intermediate inputs in final production to the power $\frac{1}{\varepsilon_y}$, so that in the limit the function approaches the Leontief specification as $\varepsilon_y\rightarrow 0$ \citep{Herrendorf2014GrowthTransformation}.}
 The disutility of labor, $\chi$, is set to match equilibrium average hours worked to average hours over the period from 2015-2019 drawing from OECD data \citep{OECDHoursworked}, $\chi=9.66$. I normalize total economic time endowment for workers and scientists per day, which I set to 14.5 as found in \cite{Jones1993OptimalGrowth}, to 1. 

 Initial productivity levels follow from normalizing output in the base period to $Y=1$ and matching the ratio of fossil-to-green energy utilization over the years 2015-2019 which equals 7.33 according to \cite[][Table 1.3]{EIAEnergy}. I find that total factor productivities in the baseline period are $A_{N0}^{1-\alpha_N}=1.90$, $A_{F0}^{1-\alpha_F}=4.52$, and $A_{G0}^{1-\alpha_G}=1.17$. %Since the green and fossil energy good are no close substitutes with $\varepsilon_e=1.5$, the fossil sector has to be technologically more advanced to 

Finally, I calibrate the sink capacity to match the average difference between gross and net CO$_2$ emissions over the baseline period from 2015 to 2019. Information on emissions comes from the US Environmental Protection Agency \citep{EPAems}. Since sinks are relevant for all greenhouse gasses, I only use the proportion of total sink capacity which reflects contribution of carbon dioxide to gross greenhouse gas emissions. The resulting sink capacity per model period is $\delta=3.19$GtCO$_2$.\footnote{ I consider this capacity to be constant. This is a simplifying assumption. What is crucial qualitatively is the assumption that sinks are finite. Indeed, natural sinks and carbon capture and storage (CCS) technologies rely on the use of land \citep{VanVuuren2018AlternativeTechnologies} which is in limited supply. In addition, the importance of land for food production makes land even scarcer especially in light of a growing world population.}
The parameter relating CO$_2$ emissions and fossil energy in the base period equals $\omega=345.33$.\footnote{  I perceive the fossil sector in the model as source of all CO$_2$ emissions including, for instance, non-energy use of fuels and incineration of waste.}  

\begin{table}[h!]
\begin{center}
\captionsetup{width=0.9\textwidth}
\caption{ Calibration}
\label{tab:calib2}
\resizebox{5in}{!}{
	\begin{tabular}{c|ll}
		%			\hline \hline
		%			\multicolumn{7}{c}{Calibration based on basic needs}\\
		\hline \hline
		Parameter& Target/Source& \makecell[l]{Value}\\ 
		\hline
		Household&\multicolumn{2}{c}{}\\
		\hline 
		$\sigma$ &  \makecell[l]{\cite{Chetty2011AreMargins}}& $1.33$  \\
		$\chi$ &  \makecell[l]{average hours worked per\\ economic time endowment\\ by worker: 0.34 \citep{OECDHoursworked}}& 9.66 \\
		$\beta$ &  \makecell[l]{\cite{Barrage2019OptimalPolicy}}& 0.93 \\
		$\bar{H}$& \makecell[l]{14.5 hours per day\\ \cite{Jones1993OptimalGrowth}}&1.00 \\
		\hline
		Research&\multicolumn{2}{c}{}
		\\
		\hline 
		$\eta$ & & 0.79 \\
		($\rho_F$, $\rho_G$, $\rho_N$) & & (0.01, 0.01, 1.00) \\
		$\phi$ &\makecell[l]{\cite{Fried2018ClimateAnalysis}} & 0.50 \\
		$S$ && 0.01\\
		$\gamma$ && 3.96\\
		\hline
		Production&\multicolumn{2}{c}{}\\
		\hline
		($\varepsilon_y$, $\varepsilon_e$)&\cite{Fried2018ClimateAnalysis}&(0.05, 1.50)\\			
		$\delta_y$&\makecell[l]{expenditure share \\ on energy \citep{EIAEnergy}}&0.30\\	
		($\alpha_F$, $\alpha_G$, $\alpha_N$)&\cite{Fried2018ClimateAnalysis} &(0.72, 0.91, 0.36)\\
		%\hline
		%$\beta$&\makecell{ annual nominal rate 3\%\\ and annual inflation rate of 2\%}& 0.9903& discount factor\\ 
		\hline
		Initial total factor productivity&\multicolumn{2}{c}{}\\
		\hline
		($A_{F0}^{1-\alpha_F}$, $A_{G0}^{1-\alpha_G}$, $A_{N0}^{1-\alpha_N}$)& energy shares \citep{EIAEnergy} &(4.12, 1.17, 1.90)  \\
		\hline 
		Government&\multicolumn{2}{c}{}\\
		\hline
		$\tau_F$&- &0.00\\
		$\tau_{\iota}$&\cite{Barrage2019OptimalPolicy} &0.24\\
		\hline
		Emissions&\multicolumn{2}{c}{}\\
		\hline
		$\delta$& \makecell[l]{\cite{EPAems}}&3.19\\
		$\omega$& \cite{EPAems}&345.33\\
		\hline \hline
\end{tabular}	}
\end{center}
\end{table}

%According to the IEA, global greenhouse gas emissions from fuel combustion amounted to 34.2 Gt in CO$_2$ equivalents in 2019.\footnote{ Retrieved from \url{https://www.iea.org/reports/global-energy-review-2021/CO$_2$-emissions} on February 2, 2022.} I use the share the US contributed to global emissions in 2019, 19.18\%, to proxy the share in reductions I require the US to contribute to total reductions from 2019 to 2030. 

% procedure


% \textit{Convergence towards equal annual emissions per person} as a fair allocation of reductions. Then US emissions per capita should equal world emissions per capita. 
% I use the UN projected population measure to proxy for future population size.
%  The calibration is done with respect to CO$_2$ emissions. 



%Hence, the smallest adjustment follows from equal budgets per period. 
%I reduce each limit in the same proportion in the 2035-2050 period so that the remaining budget for the US for the period 2020 to 2035 

%This result leads to the following emission limits
%From 2020 to 2035 there is a total budget of net-CO$_2$ emissions of 10.627Gt for the US. From 2035 to 2050 model-period emissions may amount to [2.900, 2.854, 2.809].\footnote{ I use here that in earlier test runs the emission limits have been fully exploited. }

% \clearpage

%\thispagestyle{plain}
% \clearpage
%
%\paragraph{Sources data}
%%\url{https://www.eia.gov/totalenergy/data/monthly/#prices}
%
%Total energy data: 
%For data on skill and premium see references in 
%paper saved in data \citep{Slavik2020WagePremium}
%
%The model is calibrated to parameter values common in the literature. I bestow more care on  calibrating the emission target. 
%I match emissions in the model to emission targets suggested in the IPCC report \citep{Rogelj2018MitigationDevelopment.}. 
%%How to determine the economy in 2050? Should the economy have reached a steady state? or should it be in a transitional path? Maybe no need to specify this...it will be a outcome. All I have to use is that for all years after 2050 net-emissions have to be zero. Whether the economy is on the transitional path or in a steady state is an outcome. 
%The IPCC prescribes net-zero emissions starting from 2050. In 2030 emissions should be between 25 and 30 GtCO$_2$e per year.
%

%

%\clearpage
%\section{Model}
\begin{itemize}
	\item do a broad model set up and simplify to do analytics, eg a binary distribution of learning ability
\end{itemize}
\tr{Preliminary}
In this section, I spell out the tractable model. Skills are exogenous so that there is only an intensive margin, hours worked, which determines skill supply in equilibrium. In an extension, I introduce  skills as an endogenous variable the distribution of which is driven by learning ability as in \cite{Heathcote2017OptimalFramework}. \tr{I think I will change this set up to have both an intensive margin and an extensive margin through skill investment; the latter is the only margin in \cite{Heathcote2017OptimalFramework}. }

\subsection{Households}
There is a unit mass of households in the economy which \textit{exogenously} differ in skills, indicated by s, and effective labour productivity, $e_{s}$. There is neither a life-cycle structure in the model nor idiosyncratic risk.

Each household chooses a sequence of labour supply and consumption to maximise its lifetime utility. 
\begin{align}
U_{s}=\underset{\{c_{st}\}_{t=0}^{\infty}, \{h_{st}\}_{t=0}^{\infty}}{max}&\sum_{t=0}^{\infty}\beta^t u_s(c_{st}, h_{st})\\
%U_{s}=\underset{\{c_{st}\}_{t=0}^{\infty}, \{h_{st}\}_{t=0}^{\infty}}{max}&\sum_{t=0}^{\infty}\beta^t u_s(c_{st}, h_{st}; S_t)\\
s.t.& \ \ c_{st}p_{t}=\lambda \left(h_{st}e_{s}w_{st}\right)^{1-\tau}
\end{align}
%$S_t$ indicates the state of the environment and is taken as given by the household.
A household's income is subject to a nonlinear tax schedule, $T(y_{st})=y_{st}-\lambda y_{st}^{1-\tau}$, similar to \cite{Heathcote2017OptimalFramework}. Pre-tax income is determined by hours worked, effective labour productivity, and the skill-specific wage rate $w_{st}$. Disposable income is given by $\tilde{y}_{st}=\lambda \left(h_{st}e_{s}w_{st}\right)^{1-\tau}$.
Departure from log-utility introduces an intensive margin in the labour supply decision which instantaneously reacts to changes in taxation. 

\textit{Note: effective labour productivity can be used to match average income by type and supply, otherwise supply would determine income, too. A skill premium in HSV arises by assuming disutility from skill accumulation and an exponentially distributed learning ability; the higher this ability the lower the disutility from skill investment (compare p. 1705); with endogenous growth income then interacts with availability of machines. }

\subsection{Production}
The production side of the tractable model follows \cite{Acemoglu2012TheChange}. 
\paragraph{Composite consumption good}
The consumption good is a composite of the final output of a dirty and a cleaner sector. The final good producing sector is perfectly competitive:
$Y_t=\left(Y_{ct}^{\frac{\varepsilon-1}{\varepsilon}}+Y_{dt}^{\frac{\varepsilon-1}{\varepsilon}}\right)^\frac{\varepsilon}{\varepsilon-1}$. 
I take the composite good as the numeraire: $\left[p_{dt}^{1-\varepsilon}+p_{ct}^{1-\varepsilon}\right]^{\frac{1}{1-\varepsilon}}=1$.
\paragraph{Final good sectors}
There are two sectors a cleaner, $c$, and a dirty sector, $d$,  indexed by $j\in\{c,d\}$. In both a unit mass of competitive firms $i$ produces an individual consumption good. All firms use machines, $x_{jit}$, an intermediate labour good, $L_{jt}$, and an exhaustible resource, $R_{jt}$, as input. 
\begin{align*}
&Y_{dt}=R_{dt}^{\alpha_2} L_{dt}^{1-\alpha}\int_{0}^{1}A_{dit}^{1-\alpha}x_{dit}^{\alpha} di,\ \hspace{2mm} Y_{ct}=\pmb{R_{ct}^{\alpha_3}} L_{ct}^{1-\alpha}\int_{0}^{1}A_{cit}^{1-\alpha}x_{cit}^{\alpha} di.
\end{align*}

\paragraph{Labour input good}
\tr{Under construction: might make sense to look at two households in the analytical part; more complicated when skill is endogenous}
For simplicity, I assume there exist two skill types high and low: $s \in {h,l}$. % in the exogenous skill setting. 
A fraction $\zeta$ of the population is of type $s=h$; the remainder provides low-skilled labour. 
The labour input good used by final good producers is sector specific. The competitive labour input good producing firm in sector $j$ combines high and low-skilled labour according to:
\begin{align}
L_{jt}=l_{jht}^{\theta_j}l_{jlt}^{1-\theta_j},
\end{align}
where $\theta_c>\theta_d$, that is, the labour input good of the clean sector has a higher share of high-skilled labour.

\paragraph{Machine producing firms}
A monopolistically competitive sector produces machines, $x_{ijt}$, and sells them to final good firms in the respective sector at price $p_{ijt}$. It is assumed that the costs to produce one machine, $\psi$, are homogeneous across firms.
%\begin{align*}
%\underset{p_{ijt}}{max}\  p_{ijt}x_{ijt}(p_{ijt})-\psi x_{ijt}(p_{ijt}) \overset{!}{=}0.
%\end{align*}
 \paragraph{Research}
 Scientists perform research and decide on the sector where to conduct their research, $\{c,d\}$. 
 Researchers maximise their expected profits by allocating their research either to the cleaner or dirty sector. The amount of scientists is fixed and exogenously given (for tractability, assumption also made in \cite{Acemoglu2002DirectedChange}.)
 
 \textbf{The innovation decision}
 \begin{enumerate}
 	\item scientists compare expected profits to decide in which sector to innovate
 	\item they are successful with probability $\eta_j$ (sector dependent)
 	\item if successful: scientists receive a \textbf{one-period patent} and become the entrepreneur of the respective machine she has been randomly assigned to; the machines are more productive than in the previous period: $A_{ijt}=(1+\xi)A_{ijt-1}$
 	\item if not successful: no revenue as scientist is no entrepreneur; the machine will produce with last period's  technology: $A_{ijt}=A_{ijt-1}$
 	\item profits of the scientist are thus 0 if unsuccessful and $\pi_{ijt}$ otherwise; the expectation is about (1) the firm to which the scientist is allocated with equal probability, $f(i)=\frac{1}{|I|}=1$, and (2) the probability of success. By the law of iterated expectations we have
 	\begin{align*}
 	E_{i,\text{success}}[\pi^s_{ijt}| \Omega_t]=& \int_{0}^{1}E_\text{success}[\pi^s_{ijt}|i, \Omega_t]f(i)di,\\ \text{where}&\\
 	E_\text{success}[\pi_{ijt}|i, \Omega_t]=& \eta_j\  \pi_{ijt}+(1-\eta_j)\ 0=\eta_j (1-\alpha)\alpha p_{jt}^\frac{1}{1-\alpha}L_{jt}(1+\xi)A_{ijt},\\
 	\text{and hence}&\\
 	E_{i,\text{success}}[\pi^s_{ijt}| \Omega_t]=&\eta_j (1-\alpha)\alpha p_{jt}^\frac{1}{1-\alpha}L_{jt}(1+\xi)A_{jt}=: \Pi_{jt}.
 	\end{align*}
 	where $\Omega_t$ is the information set available at time t before innovation success has materialised, that is, equilibrium aspects also not known. $\pi^s_{ijt}$ is the profit of the scientist in period t. 
 	
 \end{enumerate}
 (Note: Monopolistic competition happens \textbf{across} sectors since it is the price elasticity of substitution between sectors that matters for the demand monopolistic producers face.
 The decision by scientists where to invent also depends on this elasticity, and potentially on the satiation level. 
 )
 
\paragraph{Market clearance}
Markets for skills and the composite consumption good clear each period:
\begin{align}
l_{dht}+l_{cht}=\zeta e_h h_{ht},\\
l_{dlt}+l_{clt}=(1-\zeta) e_l h_{lt},\\
Y_t=\zeta c_{ht}+(1-\zeta)c_{lt}.
\end{align}

\subsection{Environment}
In this section I model how sectors and emission targets relate.
There are Co2 and non-CO2 emissions which are relevant for climate warming. 
CO2 sinks are relevant for the mode, too. 


\begin{comment}
The environment in the literature

In \cite{Acemoglu2012TheChange}

\begin{itemize}
	\item quality of nature, $S_t$, and irreversibility
	\begin{align*}
	S_{t}= -\xi Y_{nt}\pmb{{-\kappa \xi Y_{st}}}+(1+\delta)S_{t-1} & \hspace{3mm} \text{if}\  S_{t}\in[0,\bar{S}]\\
	S_{t+s}=0 \ \forall s>0& \hspace{3mm}  \text{if} \ S_{t}<0
	\end{align*}
	\item environmental disaster: $S_t<0$
	\begin{align*}
	\underset{S\rightarrow0}{lim} u(C,h;S)=-\infty; 
	\hspace{5mm} 
	\underset{S\rightarrow0}{lim}\frac{\partial u(C,h;S)}{\partial S}=\infty
	\end{align*}
	\item stock of natural resources
	\begin{align*}
	Q_{t+1}=Q_t-R_{nt}\pmb{{-R_{st}}}
	\end{align*}
\end{itemize}

content...
\end{comment}

\subsection{Government}
The government is modelled as a Ramsey planner who maximises a Utilitarian social welfare function subject to a time path of emission targets.

\noindent
The exogenous emission targets from IPCC report \citep{Rogelj2018MitigationDevelopment.} are:
\begin{itemize}
\item reduction of emissions in 2030 to 25-30GtCO2 per year 
\item net-zero emissions by 2050 (model economy has to be climate-neutral from 2050 onwards)

\end{itemize}
\section{Closed-form solutions}
{Goals:
\begin{enumerate}
\item show how labour supply depends on productivity and wage (which again is determined by skill scarcity and machine supply) \checkmark
\item how do prices depend on tax progressivity (with and without endogenous innovation)
\item How does the social welfare function depend on taxes?
\item Later: how does the direction of innovation depend on tax progressivity?
%\item[\ar] all these points do not depend on the relation of production and environment (if nature does not affect productivity)
\end{enumerate}}

\subsection{Labour supply}
To derive a closed-form solution, I assume the following functional form\footnote{\ The utility function is not balanced-growth path consistent. In the quantitative analysis, I will adjust the functional form.}
\begin{align}
u_s(c_{st},h_{st})=
\frac{c_{st}^{1-\gamma}}{1-\gamma}-
\frac{h_{st}^{1+\sigma}}{1+\sigma}.%-(\bar{S}-S_t).
\end{align}
The household's optimality conditions imply the following policy functions for consumption and hours worked:
\begin{align}
\log(h_{st})= \frac{1}{\sigma+\gamma+\tau(1-\gamma)}\left[\log(1-\tau)+(1-\gamma)\log\left(\frac{\lambda}{p_{t}}\right)+(1-\gamma)(1-\tau)\log(e_sw_{st})\right],\label{eq:labour_sup}\\
\log(c_{st})= \frac{1}{\sigma +\gamma +\tau(1-\gamma)}\left[(1-\tau)\log(1-\tau)+(1+\sigma)\log\left(\frac{\lambda}{p_{t}}\right)+(1+\sigma)(1-\tau)\log(e_sw_{st})\right].
\end{align}

Focus on the last summand in equation \ref{eq:labour_sup}.
By allowing for a coefficient of relative risk aversion, $\gamma$, different from 1, the hourly wage rate, $e_sw_{st}$, shapes the impact of the tax progressivity $\tau$ on labour supply and consumption. The income and the substitution effect do not cancel as in the log-utility version studied in \cite{Heathcote2017OptimalFramework}.\footnote{\ \tr{Preliminary} Whenever the coefficient of relative risk aversion is smaller than one, $\gamma<1$, under the assumption of a concave utility in consumption and a progressive tax system, i.e. $\tau>0$ and \tr{And $\tau<1$.},  the substitution effect dominates and the household 
	increases labour supply as its hourly wage rate rises. Proof: The tax adjusted uncompensated wage elasticity reads: $\frac{d log(h)}{d log(ew)}= \frac{(1-\gamma)(1-\tau)}{\sigma+\tau(1-\gamma)+\gamma}$. \tr{And $\tau<1$.} Assuming a progressive tax system, the denominator is positive whenever utility is concave in consumption, i.e., $\gamma>0$. Hence, in sum, $\gamma\in\left(0,1\right)$ implies an increase in labour supply as the hourly wage rate rises under a progressive labour income tax. } 
Households which receive a higher income per hour worked are more responsive to an increase in tax progressivity. As a consequence, a change in tax progressivity alters supply of distinct skills heterogeneously on impact. 

%\section{Quantitative experiment and results}\label{sec:simul}

To solve for the optimal policy, I solve the Ramsey problem in each period for 60 periods which corresponds to the time frame from 2020 to 2080. The static structure of the model allows for this approach which is numerically equivalent to the results in a dynamic setting.\footnote{\ In the dynamic approach I follow \cite{Jones1993OptimalGrowth}: the planner takes the time span up to some finite period into account when maximising the social welfare function plus a continuation value. Note that there is no continuation value in this model since machines depreciate fully in each period; in other words, there is no savings technology. However, the problem simplifies substantially when solving the model separately for each period. }

In the following, I discuss the evolution of the economy, first, under the business as usual calibration, where tax progressivity equals $\tau =0.181$, and a laissez-faire scenario. Second, I study the optimal policy and allocations.



\subsection{Business as usual and Laissez-faire}
In the representative agent model, the laissez-faire economy coincides with the economy under the optimal policy without an emission target, since there are neither inefficiencies nor inequality which would motivate government action. 

Figures \ref{fig:onlyBAU_subs} and \ref{fig:onlyBAU_comp} show the evolution of the economy under business as usual and the laissez-faire economy for substitutes and complements, respectively.
Independent of the degree of substitutability of goods, the positive progressivity parameter inefficiently reduces labour supply, production, and output; compare the dashed to the solid graphs in  figures \ref{fig:onlyBAU_subs} and \ref{fig:onlyBAU_comp}.

When goods are substitutes, production inputs transition to the more productive, dirty sector. As the dirty good becomes relatively cheaper at a relatively higher productivity, demand for this good increases. Since the clean good can be substituted for by the dirty one, the percentage change in the output ratio relative to the change in prices exceeds unity. Input goods all transition to the dirty sector; compare figure \ref{fig:onlyBAU_add}. 

When goods are complements, instead, the economy, too, features a rising output ratio of dirty to clean goods. However, input goods transition to the less productive sector; see figure \ref{fig:onlyBAU_comp_add}. As final good producer's demand is less responsive to the relative price, demand for the clean good remains relatively high.  In fact, it increases as the dirty good becomes cheaper. To satisfy the rise in demand for clean goods, input factors are allocated to the less productive, clean sector. As a result, the transition to the dirty sector is muted, nevertheless, the economy  both under the business-as-usual calibration and in the laissez-faire scenario violates the emission target. This finding is in line with \cite{Ngai2007StructuralGrowth} who study the effect of heterogeneous productivities on structural transformation.

\begin{figure}[h!!]
	\centering
	\caption{Business as usual (BAU) versus laissez-faire, substitutes }\label{fig:onlyBAU_subs}
	\begin{minipage}[]{0.32\textwidth}
		\centering{\footnotesize{(a) Consumption, $c$}}
		%	\captionsetup{width=.45\linewidth}
		\includegraphics[width=1\textwidth]{../../codding_model/Own/figures/Rep_agent/staticBAU_LF_separate_c_periods59_eppsilon4.00_zeta1.40_Ad08_Ac04_thetac0.70_thetad0.56_HetGrowth1_tauul0.181_util0_withtarget0_lgd1.png}
	\end{minipage}
	\begin{minipage}[]{0.32\textwidth}
		\centering{\footnotesize{(b) High skill, $h_h$ }}
		%	\captionsetup{width=.45\linewidth}
		\includegraphics[width=1\textwidth]{../../codding_model/Own/figures/Rep_agent/staticBAU_LF_separate_hh_periods59_eppsilon4.00_zeta1.40_Ad08_Ac04_thetac0.70_thetad0.56_HetGrowth1_tauul0.181_util0_withtarget0_lgd0.png}
	\end{minipage}
	\begin{minipage}[]{0.32\textwidth}
		\centering{\footnotesize{(c) Low skill, $h_l$}}
		%	\captionsetup{width=.45\linewidth}
		\includegraphics[width=1\textwidth]{../../codding_model/Own/figures/Rep_agent/staticBAU_LF_separate_hl_periods59_eppsilon4.00_zeta1.40_Ad08_Ac04_thetac0.70_thetad0.56_HetGrowth1_tauul0.181_util0_withtarget0_lgd0.png}
	\end{minipage}
\begin{minipage}[]{0.32\textwidth}
\centering{\footnotesize{(d) Price dirty good, $p_d$}}
%	\captionsetup{width=.45\linewidth}
\includegraphics[width=1\textwidth]{../../codding_model/Own/figures/Rep_agent/staticBAU_LF_separate_pd_periods59_eppsilon4.00_zeta1.40_Ad08_Ac04_thetac0.70_thetad0.56_HetGrowth1_tauul0.181_util0_withtarget0_lgd0.png}
\end{minipage}
\begin{minipage}[]{0.32\textwidth}
\centering{\footnotesize{(e) Price clean good, $p_c$}}
%	\captionsetup{width=.45\linewidth}
\includegraphics[width=1\textwidth]{../../codding_model/Own/figures/Rep_agent/staticBAU_LF_separate_pc_periods59_eppsilon4.00_zeta1.40_Ad08_Ac04_thetac0.70_thetad0.56_HetGrowth1_tauul0.181_util0_withtarget0_lgd0.png}
\end{minipage}
	\begin{minipage}[]{0.32\textwidth}
		\centering{\footnotesize{(f) Output ratio, $y_d/y_c$}}
		%	\captionsetup{width=.45\linewidth}
		\includegraphics[width=1\textwidth]{../../codding_model/Own/figures/Rep_agent/staticBAU_LF_separate_ydyc_periods59_eppsilon4.00_zeta1.40_Ad08_Ac04_thetac0.70_thetad0.56_HetGrowth1_tauul0.181_util0_withtarget0_lgd0.png}
	\end{minipage}
\end{figure}

\begin{figure}[h!!!]
	\centering
	\caption{Business as usual (BAU) versus laissez-faire, complements }\label{fig:onlyBAU_comp}
	\begin{minipage}[]{0.32\textwidth}
		\centering{\footnotesize{(a) Consumption, $c$}}
		%	\captionsetup{width=.45\linewidth}
		\includegraphics[width=1\textwidth]{../../codding_model/Own/figures/Rep_agent/staticBAU_LF_separate_c_periods59_eppsilon0.40_zeta1.40_Ad08_Ac04_thetac0.70_thetad0.56_HetGrowth1_tauul0.181_util0_withtarget0_lgd1.png}
	\end{minipage}
	\begin{minipage}[]{0.32\textwidth}
		\centering{\footnotesize{(b) High skill, $h_h$ }}
		%	\captionsetup{width=.45\linewidth}
		\includegraphics[width=1\textwidth]{../../codding_model/Own/figures/Rep_agent/staticBAU_LF_separate_hh_periods59_eppsilon0.40_zeta1.40_Ad08_Ac04_thetac0.70_thetad0.56_HetGrowth1_tauul0.181_util0_withtarget0_lgd0.png}
	\end{minipage}
	\begin{minipage}[]{0.32\textwidth}
		\centering{\footnotesize{(c) Low skill, $h_l$}}
		%	\captionsetup{width=.45\linewidth}
		\includegraphics[width=1\textwidth]{../../codding_model/Own/figures/Rep_agent/staticBAU_LF_separate_hl_periods59_eppsilon0.40_zeta1.40_Ad08_Ac04_thetac0.70_thetad0.56_HetGrowth1_tauul0.181_util0_withtarget0_lgd0.png}
	\end{minipage}
	\begin{minipage}[]{0.32\textwidth}
	\centering{\footnotesize{(d) Price dirty good, $p_d$ }}
	%	\captionsetup{width=.45\linewidth}
	\includegraphics[width=1\textwidth]{../../codding_model/Own/figures/Rep_agent/staticBAU_LF_separate_pd_periods59_eppsilon0.40_zeta1.40_Ad08_Ac04_thetac0.70_thetad0.56_HetGrowth1_tauul0.181_util0_withtarget0_lgd0.png}
\end{minipage}
\begin{minipage}[]{0.32\textwidth}
	\centering{\footnotesize{(e) Price clean good, $p_c$}}
	%	\captionsetup{width=.45\linewidth}
	\includegraphics[width=1\textwidth]{../../codding_model/Own/figures/Rep_agent/staticBAU_LF_separate_pc_periods59_eppsilon0.40_zeta1.40_Ad08_Ac04_thetac0.70_thetad0.56_HetGrowth1_tauul0.181_util0_withtarget0_lgd0.png}
\end{minipage}
	\begin{minipage}[]{0.32\textwidth}
		\centering{\footnotesize{(f) Output ratio, $y_d/y_c$}}
		%	\captionsetup{width=.45\linewidth}
		\includegraphics[width=1\textwidth]{../../codding_model/Own/figures/Rep_agent/staticBAU_LF_separate_ydyc_periods59_eppsilon0.40_zeta1.40_Ad08_Ac04_thetac0.70_thetad0.56_HetGrowth1_tauul0.181_util0_withtarget0_lgd0.png}
	\end{minipage}
\end{figure}

\subsection{Optimal policies and allocations}
As argued earlier, without emission target, the optimal policy is to set a flat tax rate, hence, the elasticity of post- to pre-tax income is 1; compare panel (a) in figure \ref{fig:optpol}. This does not interfere with households' labour supply decision.

The remaining panels in the same figure depict the optimal progressivity parameter when the government has to satisfy an emission target for when the clean and dirty goods are substitutes, panel (b), or complements, panel (c).
When the government is concerned with satisfying an emission target, the optimal tax system is progressive in all periods: $\tau >0$. Progressivity increases over time and jumps close to but below the highest  possible level, $\tau=1$,\footnote{\ So that post- and pre-tax income are still positively correlated.} when net emissions have to be zero. This result is independent of whether goods are substitutes or complements.   

The dynamic pattern of the optimal tax progressivity follows from the positive growth rate in dirty output. To counteract the rise in output, labour supply has to decrease more to satisfy the emission target. Tax progressivity is chosen higher when goods are substitutes. As argued above, when goods are substitutes a higher demand for the more productive, dirty good increases emissions more. Consequently, to counter this market force, the government has to set an even higher tax progressivity. In contrast, when goods are complements, market mechanisms make less government intervention necessary. 

%This finding is in contrast to the results in \cite{Acemogxxx}. In their model with directed technical change, a higher substitutability of goods renders less government intervention sufficient. In this model with exogeneous growth, complementability of goods retards the output growth in the dirty sector due to the slower growth in the complementary clean sector. With exogeneous growth, complementarity of goods directs too much resources to the dirty sector absent government intervention. \tr{HOW IS IT IN LAISSEZ FAIRE?}


\begin{figure}[h!!]
	\centering
	\caption{Optimal Policy }\label{fig:optpol}
	\begin{minipage}[]{0.32\textwidth}
		\centering{\footnotesize{(a) Without target }}
		%	\captionsetup{width=.45\linewidth}
		\includegraphics[width=1\textwidth]{../../codding_model/Own/figures/Rep_agent/staticRam_LF_separate_tauul_periods59_eppsilon4.00_zeta1.40_Ad08_Ac04_thetac0.70_thetad0.56_HetGrowth1_tauul0.181_util0_withtarget0_lgd0.png}
	\end{minipage}
	\begin{minipage}[]{0.32\textwidth}
	\centering{\footnotesize{(b) With target, $\varepsilon>1$ }}
	%	\captionsetup{width=.45\linewidth}
	\includegraphics[width=1\textwidth]{../../codding_model/Own/figures/Rep_agent/staticRam_LF_separate_tauul_periods59_eppsilon4.00_zeta1.40_Ad08_Ac04_thetac0.70_thetad0.56_HetGrowth1_tauul0.181_util0_withtarget1_lgd0.png}
\end{minipage}
\begin{minipage}[]{0.32\textwidth}
	\centering{\footnotesize{(c) With target, $\varepsilon<1$ }}
	%	\captionsetup{width=.45\linewidth}
	\includegraphics[width=1\textwidth]{../../codding_model/Own/figures/Rep_agent/staticRam_LF_separate_tauul_periods59_eppsilon0.40_zeta1.40_Ad08_Ac04_thetac0.70_thetad0.56_HetGrowth1_tauul0.181_util0_withtarget1_lgd0.png}
\end{minipage}
\end{figure}

Figures \ref{fig:optallo_subs_onlyR} and \ref{fig:optallo_comp_onlyR} depict the allocation resulting from the optimal policy.
Irrespective of the degree of substitutability between goods, the optimal policy enforces a decreasing consumption and work pattern.  
Since the return to labour reduces as tax progressivity increases, the representative household diminishes its labour supply over time, panels (b) and (c).\footnote{\ In the optimal labour supply, $H=(1-\tau)^{\frac{1}{1+\sigma}}$, the exponent of $(1/(1+\sigma))$ constitutes a multiplier effect: as the return to labour in terms of income reduces, the household, on impact, diminishes its labour supply, which again raises the shadow value of income; this effect mitigates the total reduction in hours worked.} 
As a result, consumption is  decreasing and relatively lower than in the laissez-faire allocation. 

When goods are complements,  the smaller tax progressivity suffices to reduce dirty output more than in the world where goods are substitutes. Compare panel (e) across figures. Furthermore, the smaller reduction in labour supply under the less aggressive policy keeps clean production comparably high; compare panels (d) across figures. Nevertheless, consumption falls more due to the necessity of dirty output in final good production. 

\newpage
% only ramsey Subs
\begin{figure}[h!!]
	\centering
	\caption{Optimal allocation with emission target, substitutes }\label{fig:optallo_subs_onlyR}
	\begin{minipage}[]{0.32\textwidth}
		\centering{\footnotesize{(a) Consumption, $c$ }}
		%	\captionsetup{width=.45\linewidth}
		\includegraphics[width=1\textwidth]{../../codding_model/Own/figures/Rep_agent/staticonlyRam_separate_c_periods59_eppsilon4.00_zeta1.40_Ad08_Ac04_thetac0.70_thetad0.56_HetGrowth1_tauul0.181_util0_withtarget1_lgd0.png}
	\end{minipage}
	\begin{minipage}[]{0.32\textwidth}
		\centering{\footnotesize{(b) High skill, $h_h$ }}
		%	\captionsetup{width=.45\linewidth}
		\includegraphics[width=1\textwidth]{../../codding_model/Own/figures/Rep_agent/staticonlyRam_separate_hh_periods59_eppsilon4.00_zeta1.40_Ad08_Ac04_thetac0.70_thetad0.56_HetGrowth1_tauul0.181_util0_withtarget1_lgd0.png}
	\end{minipage}
	\begin{minipage}[]{0.32\textwidth}
		\centering{\footnotesize{(c) Low skill, $h_l$}}
		%	\captionsetup{width=.45\linewidth}
		\includegraphics[width=1\textwidth]{../../codding_model/Own/figures/Rep_agent/staticonlyRam_separate_hl_periods59_eppsilon4.00_zeta1.40_Ad08_Ac04_thetac0.70_thetad0.56_HetGrowth1_tauul0.181_util0_withtarget1_lgd0.png}
	\end{minipage}
	\begin{minipage}[]{0.32\textwidth}
		\centering{\footnotesize{(d) Clean output, $y_c$ }}
		%	\captionsetup{width=.45\linewidth}
		\includegraphics[width=1\textwidth]{../../codding_model/Own/figures/Rep_agent/staticonlyRam_separate_yc_periods59_eppsilon4.00_zeta1.40_Ad08_Ac04_thetac0.70_thetad0.56_HetGrowth1_tauul0.181_util0_withtarget1_lgd0.png}
	\end{minipage}
	\begin{minipage}[]{0.32\textwidth}
		\centering{\footnotesize{(e) Dirty output, $y_d$ }}
		%	\captionsetup{width=.45\linewidth}
		\includegraphics[width=1\textwidth]{../../codding_model/Own/figures/Rep_agent/staticonlyRam_separate_yd_periods59_eppsilon4.00_zeta1.40_Ad08_Ac04_thetac0.70_thetad0.56_HetGrowth1_tauul0.181_util0_withtarget1_lgd0.png}
	\end{minipage}
	\begin{minipage}[]{0.32\textwidth}
		\centering{\footnotesize{(g) Output ratio, $y_d/y_c$}}
		%	\captionsetup{width=.45\linewidth}
		\includegraphics[width=1\textwidth]{../../codding_model/Own/figures/Rep_agent/staticonlyRam_separate_ydyc_periods59_eppsilon4.00_zeta1.40_Ad08_Ac04_thetac0.70_thetad0.56_HetGrowth1_tauul0.181_util0_withtarget1_lgd0.png}
	\end{minipage}
\end{figure}

% only ramsey Comp
\begin{figure}[h!!]
	\centering
	\caption{Optimal allocation with emission target, complements }\label{fig:optallo_comp_onlyR}
	\begin{minipage}[]{0.32\textwidth}
		\centering{\footnotesize{(a) Consumption, $c$ }}
		%	\captionsetup{width=.45\linewidth}
		\includegraphics[width=1\textwidth]{../../codding_model/Own/figures/Rep_agent/staticonlyRam_separate_c_periods59_eppsilon0.40_zeta1.40_Ad08_Ac04_thetac0.70_thetad0.56_HetGrowth1_tauul0.181_util0_withtarget1_lgd0.png}
	\end{minipage}
	\begin{minipage}[]{0.32\textwidth}
		\centering{\footnotesize{(b) High skill, $h_h$ }}
		%	\captionsetup{width=.45\linewidth}
		\includegraphics[width=1\textwidth]{../../codding_model/Own/figures/Rep_agent/staticonlyRam_separate_hh_periods59_eppsilon0.40_zeta1.40_Ad08_Ac04_thetac0.70_thetad0.56_HetGrowth1_tauul0.181_util0_withtarget1_lgd0.png}
	\end{minipage}
	\begin{minipage}[]{0.32\textwidth}
		\centering{\footnotesize{(c) Low skill, $h_l$}}
		%	\captionsetup{width=.45\linewidth}
		\includegraphics[width=1\textwidth]{../../codding_model/Own/figures/Rep_agent/staticonlyRam_separate_hl_periods59_eppsilon0.40_zeta1.40_Ad08_Ac04_thetac0.70_thetad0.56_HetGrowth1_tauul0.181_util0_withtarget1_lgd0.png}
	\end{minipage}
	\begin{minipage}[]{0.32\textwidth}
		\centering{\footnotesize{(d) Clean output, $y_c$ }}
		%	\captionsetup{width=.45\linewidth}
		\includegraphics[width=1\textwidth]{../../codding_model/Own/figures/Rep_agent/staticonlyRam_separate_yc_periods59_eppsilon0.40_zeta1.40_Ad08_Ac04_thetac0.70_thetad0.56_HetGrowth1_tauul0.181_util0_withtarget1_lgd0.png}
	\end{minipage}
	\begin{minipage}[]{0.32\textwidth}
		\centering{\footnotesize{(e) Dirty output, $y_d$ }}
		%	\captionsetup{width=.45\linewidth}
		\includegraphics[width=1\textwidth]{../../codding_model/Own/figures/Rep_agent/staticonlyRam_separate_yd_periods59_eppsilon0.40_zeta1.40_Ad08_Ac04_thetac0.70_thetad0.56_HetGrowth1_tauul0.181_util0_withtarget1_lgd0.png}
	\end{minipage}
	\begin{minipage}[]{0.32\textwidth}
		\centering{\footnotesize{(g) Output ratio, $y_d/y_c$}}
		%	\captionsetup{width=.45\linewidth}
		\includegraphics[width=1\textwidth]{../../codding_model/Own/figures/Rep_agent/staticonlyRam_separate_ydyc_periods59_eppsilon0.40_zeta1.40_Ad08_Ac04_thetac0.70_thetad0.56_HetGrowth1_tauul0.181_util0_withtarget1_lgd0.png}
	\end{minipage}
\end{figure}

\begin{comment}

\begin{figure}[h!!]
	\centering
	\caption{Optimal allocation with and without emission target, complements }\label{fig:optallo_comp_target}
	\begin{minipage}[]{0.32\textwidth}
		\centering{\footnotesize{(a) Consumption }}
		%	\captionsetup{width=.45\linewidth}
		\includegraphics[width=1\textwidth]{../../codding_model/Own/figures/Rep_agent/static_CompTarget_c_periods59_eppsilon0.40_zeta1.40_Ad08_Ac04_thetac0.70_thetad0.56_HetGrowth1_util0_lgd1.png}
	\end{minipage}
	\begin{minipage}[]{0.32\textwidth}
		\centering{\footnotesize{(b) High skill supply }}
		%	\captionsetup{width=.45\linewidth}
		\includegraphics[width=1\textwidth]{../../codding_model/Own/figures/Rep_agent/static_CompTarget_hh_periods59_eppsilon0.40_zeta1.40_Ad08_Ac04_thetac0.70_thetad0.56_HetGrowth1_util0_lgd0.png}
	\end{minipage}
	\begin{minipage}[]{0.32\textwidth}
		\centering{\footnotesize{(c) Low skill supply}}
		%	\captionsetup{width=.45\linewidth}
		\includegraphics[width=1\textwidth]{../../codding_model/Own/figures/Rep_agent/staticRam_LF_separate_hl_periods59_eppsilon0.40_zeta1.40_Ad08_Ac04_thetac0.70_thetad0.56_HetGrowth1_tauul0.181_util0_withtarget1_lgd0.png}
	\end{minipage}
	\begin{minipage}[]{0.32\textwidth}
		\centering{\footnotesize{(d) clean output }}
		%	\captionsetup{width=.45\linewidth}
		\includegraphics[width=1\textwidth]{../../codding_model/Own/figures/Rep_agent/staticRam_LF_separate_yc_periods59_eppsilon0.40_zeta1.40_Ad08_Ac04_thetac0.70_thetad0.56_HetGrowth1_tauul0.181_util0_withtarget1_lgd0.png}
	\end{minipage}
	\begin{minipage}[]{0.32\textwidth}
		\centering{\footnotesize{(e) dirty output }}
		%	\captionsetup{width=.45\linewidth}
		\includegraphics[width=1\textwidth]{../../codding_model/Own/figures/Rep_agent/staticRam_LF_separate_yd_periods59_eppsilon0.40_zeta1.40_Ad08_Ac04_thetac0.70_thetad0.56_HetGrowth1_tauul0.181_util0_withtarget1_lgd0.png}
	\end{minipage}
	\begin{minipage}[]{0.32\textwidth}
		\centering{\footnotesize{(f) machines dirty}}
		%	\captionsetup{width=.45\linewidth}
		\includegraphics[width=1\textwidth]{../../codding_model/Own/figures/Rep_agent/staticRam_LF_separate_xd_periods59_eppsilon0.40_zeta1.40_Ad08_Ac04_thetac0.70_thetad0.56_HetGrowth1_tauul0.181_util0_withtarget1_lgd0.png}
	\end{minipage}
	\begin{minipage}[]{0.32\textwidth}
		\centering{\footnotesize{(f) machines clean}}
		%	\captionsetup{width=.45\linewidth}
		\includegraphics[width=1\textwidth]{../../codding_model/Own/figures/Rep_agent/staticRam_LF_separate_xc_periods59_eppsilon0.40_zeta1.40_Ad08_Ac04_thetac0.70_thetad0.56_HetGrowth1_tauul0.181_util0_withtarget1_lgd0.png}
	\end{minipage}
	\begin{minipage}[]{0.32\textwidth}
		\centering{\footnotesize{(g) Output ratio $y_d/y_c$}}
		%	\captionsetup{width=.45\linewidth}
		\includegraphics[width=1\textwidth]{../../codding_model/Own/figures/Rep_agent/staticRam_LF_separate_ydyc_periods59_eppsilon0.40_zeta1.40_Ad08_Ac04_thetac0.70_thetad0.56_HetGrowth1_tauul0.181_util0_withtarget1_lgd0.png}
	\end{minipage}
\end{figure}

	content...
\end{comment}




%\section{Quantitative results}\label{sec:res}

In this section, I present and discuss the quantitative results.
Subsection \ref{subsec:mr} depicts the optimal policy given the emission target. Subsection \ref{subsec:dis} discusses the results. In particular, I focus on understanding the role of income tax progressivity. %Finally, section \ref{sec:sens} will soon present a sensitivity analysis.


\subsection{Main results}\label{subsec:mr}
\begin{figure}[h!!]
	\centering
	\caption{Optimal Policy }\label{fig:optPol}
	\begin{minipage}[]{0.4\textwidth}
		\centering{\footnotesize{(a) Income tax progressivity, $\tau_{lt}$}}
		%	\captionsetup{width=.45\linewidth}
		\includegraphics[width=1\textwidth]{../../codding_model/own_basedOnFried/optimalPol_elastS_DisuSci/figures/all_1705/Single_OPT_T_NoTaus_taul_spillover0_sep1_BN0_ineq0_red0_etaa0.79.png}
	\end{minipage}
\begin{minipage}[]{0.1\textwidth}
\
\end{minipage}
	\begin{minipage}[]{0.4\textwidth}
		\centering{\footnotesize{(b) Fossil tax, $\tau_{ft}$ }}
		%	\captionsetup{width=.45\linewidth}
		\includegraphics[width=1\textwidth]{../../codding_model/own_basedOnFried/optimalPol_elastS_DisuSci/figures/all_1705/Single_OPT_T_NoTaus_tauf_spillover0_sep1_BN0_ineq0_red0_etaa0.79.png}
	\end{minipage}
\end{figure} 
%\begin{figure}[h!!]
%	\centering
%	\caption{Optimal Policy }\label{fig:optPol}
%	\begin{minipage}[]{0.4\textwidth}
%		\centering{\footnotesize{(a) Income tax progressivity, $\tau_{lt}$}}
%		%	\captionsetup{width=.45\linewidth}
%		\includegraphics[width=1\textwidth]{../../codding_model/own_basedOnFried/optimalPol_elastS_DisuSci/figures/all_1705/Single_OPT_T_NoTaus_taul_spillover0_sep1_BN1_ineq0_etaa0.79.png}
%	\end{minipage}
%	\begin{minipage}[]{0.1\textwidth}
%		\
%	\end{minipage}
%	\begin{minipage}[]{0.4\textwidth}
%		\centering{\footnotesize{(b) Fossil tax, $\tau_{ft}$ }}
%		%	\captionsetup{width=.45\linewidth}
%		\includegraphics[width=1\textwidth]{../../codding_model/own_basedOnFried/optimalPol_elastS_DisuSci/figures/all_1705/Single_OPT_T_NoTaus_tauf_spillover0_sep1_BN1_ineq0_etaa0.79.png}
%	\end{minipage}
%\end{figure}

To meet the IPCCs suggested emission target, the optimal income tax is progressive for all periods between 2030 and 2080; see panel (a) in figure \ref{fig:optPol}. As the emission target is less strict, between 2030 to 2045, optimal income tax progressivity is around $\tau_{lt}=0.04$. As the emission target jumps to net-zero emissions in 2050, optimal tax progressivity accelerates to above 0.08 and gradually increases in the subsequent years to around 0.09. This is approximately half the size found for the US in \cite{Heathcote2017OptimalFramework}: $\tau_{l}=0.181$. 
In the period without emission target from 2020 to 2030, the optimal income tax is slightly regressive.

Consider panel (b). The optimal fossil tax displays a similar step pattern as the income tax progressivity. From 2020 to the beginning of 2030, it is negative. It jumps to around 70\% as the emission target is to reduce emissions by 50\% relative to 2019 emissions. As the emission target rises  to net-zero emissions in 2050, the optimal tax on fossil sales is close to 90\%. 

Figure \ref{fig:optAll} depicts the optimal allocation while meeting emission targets. Limiting emissions in line with the Paris Agreement is concomitant with both a reduction and recomposition of consumption and production over time. 

Panel (a) shows consumption which reduces significantly when new emission limits become active, in 2030 and in 2050, but starting from the new low levels continues to grow. labour effort of both skill types also reduces visibly as stricter emission targets are enforced; panel (b). In contrast to consumption, hours worked for both types of labour decrease over time. In comparison to hours supplied by low-skilled workers, high-skilled workers reduce hours more; compare panel (c) which shows the ratio of hours worked by high to low skill workers. 

The rise in consumption after each reduction is driven by technological progress in all sectors; compare panel (d) which shows growth rates by sector and as aggregate in per cent. 
The green sector sees a rise in technological progress, the dashed black line, while growth in the fossil and the non-energy sector is positive, yet diminishing over time. Overall, aggregate growth is positive but decreasing; compare the grey dashed graph. 
Summing up the last two paragraphs, the emission target is best achieved with more leisure at higher technology levels in all sectors. 

Nevertheless, there would be potential for more growth which is forfeited to meet emission targets. This becomes apparent when looking at the allocation of scientists in panel (e). Again, there is a recomposition towards the green sector: while research in the non-energy and the fossil sector decrease over time, green research effort rises. Yet, overall, the amount of scientists reduces; compare the grey graph which depicts the sum of researchers across sectors. 
Finally, labour input goods are also redirected towards the green sector; see panel (f). 

%The recomposing aspect of the optimal policy is best underlined by looking at labour inputs and research. Panels (g) to (i) show the labour composite used in the distinct sectors. While labour input in the fossil sector reduces, it increases in the green sector. The economy recomposes its energy consumption towards green energy. The reductive aspect of the optimal policy, is highlighted by the reduction of non-energy labour input; panel (i). 
%\begin{comment}
\begin{figure}[h!!]
	\centering
	\caption{Optimal Allocation }\label{fig:optAll}
	
	
	\begin{minipage}[]{0.32\textwidth}
		\centering{\footnotesize{(a) Consumption}}
		%	\captionsetup{width=.45\linewidth}
		\includegraphics[width=1\textwidth]{../../codding_model/own_basedOnFried/optimalPol_elastS_DisuSci/figures/all_1705/Single_OPT_T_NoTaus_C_spillover0_sep1_BN0_ineq0_red0_etaa0.79.png}
	\end{minipage}
	\begin{minipage}[]{0.32\textwidth}
		\centering{\footnotesize{(b) Hours worked }}
		%	\captionsetup{width=.45\linewidth}
		\includegraphics[width=1\textwidth]{../../codding_model/own_basedOnFried/optimalPol_elastS_DisuSci/figures/all_1705/SingleJointTOT_OPT_T_NoTaus_labour_spillover0_sep1_BN0_ineq0_red0_etaa0.79_lgd1.png}
	\end{minipage}
	\begin{minipage}[]{0.32\textwidth}
		\centering{\footnotesize{(c) High-to-low-skill ratio hours}}
		%	\captionsetup{width=.45\linewidth}
		\includegraphics[width=1\textwidth]{../../codding_model/own_basedOnFried/optimalPol_elastS_DisuSci/figures/all_1705/Single_OPT_T_NoTaus_hhhl_spillover0_sep1_BN0_ineq0_red0_etaa0.79.png}
	\end{minipage}
	\begin{minipage}[]{0.32\textwidth}
		\centering{\footnotesize{\ \\ (d) Technology growth}}
		%	\captionsetup{width=.45\linewidth}
		\includegraphics[width=1\textwidth]{../../codding_model/own_basedOnFried/optimalPol_elastS_DisuSci/figures/all_1705/SingleJointTOT_OPT_T_NoTaus_Growth_spillover0_sep1_BN0_ineq0_red0_etaa0.79_lgd1.png}
	\end{minipage}
	\begin{minipage}[]{0.32\textwidth}
		\centering{\footnotesize{\ \\(e) Scientists }}
		%	\captionsetup{width=.45\linewidth}
		\includegraphics[width=1\textwidth]{../../codding_model/own_basedOnFried/optimalPol_elastS_DisuSci/figures/all_1705/SingleJointTOT_OPT_T_NoTaus_Science_spillover0_sep1_BN0_ineq0_red0_etaa0.79_lgd1.png}
	\end{minipage}
\begin{minipage}[]{0.32\textwidth}
	\centering{\footnotesize{\ \\(f) labour input}}
	%	\captionsetup{width=.45\linewidth}
	\includegraphics[width=1\textwidth]{../../codding_model/own_basedOnFried/optimalPol_elastS_DisuSci/figures/all_1705/SingleJointTOT_OPT_T_NoTaus_labourInp_spillover0_sep1_BN0_ineq0_red0_etaa0.79_lgd1.png}
\end{minipage}
%	\begin{minipage}[]{0.32\textwidth}
%	\centering{\footnotesize{(d) labour fossil sector}}
%	%	\captionsetup{width=.45\linewidth}
%	\includegraphics[width=1\textwidth]{../../codding_model/own_basedOnFried/optimalPol_elastS_DisuSci/figures/all_1705/Single_OPT_T_NoTaus_Lf_spillover0_sep1_BN0_ineq0_etaa0.79.png}
%\end{minipage}
%\begin{minipage}[]{0.32\textwidth}
%	\centering{\footnotesize{(e) labour green}}
%	%	\captionsetup{width=.45\linewidth}
%	\includegraphics[width=1\textwidth]{../../codding_model/own_basedOnFried/optimalPol_elastS_DisuSci/figures/all_1705/Single_OPT_T_NoTaus_Lg_spillover0_sep1_BN0_ineq0_etaa0.79.png}
%\end{minipage}
%\begin{minipage}[]{0.32\textwidth}
%	\centering{\footnotesize{(f) labour neutral}}
%	%	\captionsetup{width=.45\linewidth}
%	\includegraphics[width=1\textwidth]{../../codding_model/own_basedOnFried/optimalPol_elastS_DisuSci/figures/all_1705/Single_OPT_T_NoTaus_Ln_spillover0_sep1_BN0_ineq0_etaa0.79.png}
%\end{minipage}
\end{figure} 

%\end{comment}


\subsection{Discussion}\label{subsec:dis}
To study the role of income tax progressivity, I compare the optimal policy and allocation in the full model to a  model where no labour income tax is available in subsection \ref{subsub:withwithout}. In section \ref{subsub:compeff}, I compare these figures to the allocation a social planner would choose.
But first, I highlight the effect of government intervention relative to the business as usual policy in \ref{subsub:bau}. 

\subsubsection{Business as usual}\label{subsub:bau}
This subsection serves to underline that government intervention to satisfy emission targets is necessary. Compare figure \ref{fig:BAU}.  Without any change in government policy, emissions grow almost twice as big than the emission limit in the period 2030-2050 amounting to around 6 Gt. Net emissions increase gradually under business as usual reaching more than 6Gt in 2080. 
As the economy grows, consumption and the high-to-low skill ratio of hours worked increase; compare the dashed orange graphs in panels (b) and (c), respectively. While the government implements a decreasing consumption pattern to meet emission targets, consumption is relatively higher for the first 30 years considered. The optimal policy becomes reductive relative to the BAU economy only once the net emission target falls to zero. In the BAU calibration income tax progressivity is fixed at $\tau_{lt}=0.181$ which is higher than the optimal policy to meet the emission target. This explains the increase in consumption and high-to-low skill supply under the constrained optimal policy. 

\begin{figure}[h!!]
	\centering
	\caption{Emissions under Business as usual }\label{fig:BAU}
	\begin{minipage}[]{0.32\textwidth}
		\centering{\footnotesize{(a) Net emissions}}
		%	\captionsetup{width=.45\linewidth}
		\includegraphics[width=1\textwidth]{../../codding_model/own_basedOnFried/optimalPol_elastS_DisuSci/figures/all_1705/Single_BAU_Emnet_spillover0_sep1_BN0_ineq0_red0_etaa0.79.png}
	\end{minipage}
	\begin{minipage}[]{0.32\textwidth}
		\centering{\footnotesize{(b) Consumption}}
		%	\captionsetup{width=.45\linewidth}
		\includegraphics[width=1\textwidth]{../../codding_model/own_basedOnFried/optimalPol_elastS_DisuSci/figures/all_1705/C_BAUCompOPT_T_NoTaus_spillover0_sep1_BN0_ineq0_red0_etaa0.79_lgd1.png}
	\end{minipage}
	\begin{minipage}[]{0.32\textwidth}
		\centering{\footnotesize{(c) High-to-low skill ratio hours}}
		%	\captionsetup{width=.45\linewidth}
		\includegraphics[width=1\textwidth]{../../codding_model/own_basedOnFried/optimalPol_elastS_DisuSci/figures/all_1705/hhhl_BAUCompOPT_T_NoTaus_spillover0_sep1_BN0_ineq0_red0_etaa0.79_lgd0.png}
	\end{minipage}
\end{figure} 


\subsubsection{Comparison with and without income taxes}\label{subsub:withwithout}
\begin{figure}[h!!]
	\centering
	\caption{Comparison to regime without income tax }\label{fig:Compno_taul_BN0}
	\begin{minipage}[]{0.32\textwidth}
		\centering{\footnotesize{(a) Emissions}}
		%	\captionsetup{width=.45\linewidth}
		\includegraphics[width=1\textwidth]{../../codding_model/own_basedOnFried/optimalPol_elastS_DisuSci/figures/all_1705/comp_notaul_OPT_T_NoTaus_Emnet_spillover0_sep1_BN0_ineq0_red0_etaa0.79_lgd1.png}
	\end{minipage}
	\begin{minipage}[]{0.32\textwidth}
		\centering{\footnotesize{(b) Fossil tax}}
		%	\captionsetup{width=.45\linewidth}
		\includegraphics[width=1\textwidth]{../../codding_model/own_basedOnFried/optimalPol_elastS_DisuSci/figures/all_1705/comp_notaul_OPT_T_NoTaus_tauf_spillover0_sep1_BN0_ineq0_red0_etaa0.79_lgd0.png}
	\end{minipage}
	\begin{minipage}[]{0.32\textwidth}
		\centering{\footnotesize{(c) Consumption}}
		%	\captionsetup{width=.45\linewidth}
		\includegraphics[width=1\textwidth]{../../codding_model/own_basedOnFried/optimalPol_elastS_DisuSci/figures/all_1705/comp_notaul_OPT_T_NoTaus_C_spillover0_sep1_BN0_ineq0_red0_etaa0.79_lgd0.png}
	\end{minipage}
	\begin{minipage}[]{0.32\textwidth}
		\centering{\footnotesize{\ \\(d) High skill hours worked }}
		%	\captionsetup{width=.45\linewidth}
		\includegraphics[width=1\textwidth]{../../codding_model/own_basedOnFried/optimalPol_elastS_DisuSci/figures/all_1705/comp_notaul_OPT_T_NoTaus_hh_spillover0_sep1_BN0_ineq0_etaa0.79.png}
	\end{minipage}
	\begin{minipage}[]{0.32\textwidth}
		\centering{\footnotesize{\ \\(e) Low skill hours worked}}
		%	\captionsetup{width=.45\linewidth}
		\includegraphics[width=1\textwidth]{../../codding_model/own_basedOnFried/optimalPol_elastS_DisuSci/figures/all_1705/comp_notaul_OPT_T_NoTaus_hl_spillover0_sep1_BN0_ineq0_etaa0.79.png}
	\end{minipage}
	\begin{minipage}[]{0.32\textwidth}
		\centering{\footnotesize{\ \\(f) Social welfare}}
		%	\captionsetup{width=.45\linewidth}
		\includegraphics[width=1\textwidth]{../../codding_model/own_basedOnFried/optimalPol_elastS_DisuSci/figures/all_1705/comp_notaul_OPT_T_NoTaus_SWF_spillover0_sep1_BN0_ineq0_red0_etaa0.79_lgd0.png}
	\end{minipage}
	\begin{minipage}[]{0.32\textwidth}
		\centering{\footnotesize{\ \\(g) labour fossil}}
		%	\captionsetup{width=.45\linewidth}
		\includegraphics[width=1\textwidth]{../../codding_model/own_basedOnFried/optimalPol_elastS_DisuSci/figures/all_1705/comp_notaul_OPT_T_NoTaus_Lf_spillover0_sep1_BN0_ineq0_red0_etaa0.79_lgd0.png}
	\end{minipage}
	\begin{minipage}[]{0.32\textwidth}
		\centering{\footnotesize{\ \\(h) labour green}}
		%	\captionsetup{width=.45\linewidth}
		\includegraphics[width=1\textwidth]{../../codding_model/own_basedOnFried/optimalPol_elastS_DisuSci/figures/all_1705/comp_notaul_OPT_T_NoTaus_Lg_spillover0_sep1_BN0_ineq0_red0_etaa0.79_lgd0.png}
	\end{minipage}
	\begin{minipage}[]{0.32\textwidth}
		\centering{\footnotesize{\ \\(i) labour non-energy}}
		%	\captionsetup{width=.45\linewidth}
		\includegraphics[width=1\textwidth]{../../codding_model/own_basedOnFried/optimalPol_elastS_DisuSci/figures/all_1705/comp_notaul_OPT_T_NoTaus_Ln_spillover0_sep1_BN0_ineq0_red0_etaa0.79_lgd0.png}
	\end{minipage}
	\begin{minipage}[]{0.32\textwidth}
		\centering{\footnotesize{\ \\(j) Research fossil}}
		%	\captionsetup{width=.45\linewidth}
		\includegraphics[width=1\textwidth]{../../codding_model/own_basedOnFried/optimalPol_elastS_DisuSci/figures/all_1705/comp_notaul_OPT_T_NoTaus_sff_spillover0_sep1_BN0_ineq0_red0_etaa0.79_lgd0.png}
	\end{minipage}
	\begin{minipage}[]{0.32\textwidth}
		\centering{\footnotesize{\ \\(k) Green research}}
		%	\captionsetup{width=.45\linewidth}
		\includegraphics[width=1\textwidth]{../../codding_model/own_basedOnFried/optimalPol_elastS_DisuSci/figures/all_1705/comp_notaul_OPT_T_NoTaus_sg_spillover0_sep1_BN0_ineq0_red0_etaa0.79_lgd0.png}
	\end{minipage}
	\begin{minipage}[]{0.32\textwidth}
		\centering{\footnotesize{\ \\(l) Non-energy research }}
		%	\captionsetup{width=.45\linewidth}
		\includegraphics[width=1\textwidth]{../../codding_model/own_basedOnFried/optimalPol_elastS_DisuSci/figures/all_1705/comp_notaul_OPT_T_NoTaus_sn_spillover0_sep1_BN0_ineq0_red0_etaa0.79_lgd0.png}
	\end{minipage}
	
%	\begin{minipage}[]{0.32\textwidth}
%		\centering{\footnotesize{\ \\(m) Fossil output}}
%		%	\captionsetup{width=.45\linewidth}
%		\includegraphics[width=1\textwidth]{../../codding_model/own_basedOnFried/optimalPol_elastS_DisuSci/figures/all_1705/comp_notaul_OPT_T_NoTaus_F_spillover0_sep1_BN0_ineq0_etaa0.79.png}
%	\end{minipage}
%	\begin{minipage}[]{0.32\textwidth}
%		\centering{\footnotesize{\ \\(n) Green output}}
%		%	\captionsetup{width=.45\linewidth}
%		\includegraphics[width=1\textwidth]{../../codding_model/own_basedOnFried/optimalPol_elastS_DisuSci/figures/all_1705/comp_notaul_OPT_T_NoTaus_G_spillover0_sep1_BN0_ineq0_etaa0.79.png}
%	\end{minipage}
%	\begin{minipage}[]{0.32\textwidth}
%		\centering{\footnotesize{\ \\(o) Non-energy output}}
%		%	\captionsetup{width=.45\linewidth}
%		\includegraphics[width=1\textwidth]{../../codding_model/own_basedOnFried/optimalPol_elastS_DisuSci/figures/all_1705/comp_notaul_OPT_T_NoTaus_N_spillover0_sep1_BN0_ineq0_etaa0.79.png}
%	\end{minipage}
\end{figure} 

\begin{figure}[h!!]
	\centering
	\caption{Comparison to regime without income tax }\label{fig:Compno_taul_prices}
	\begin{minipage}[]{0.24\textwidth}
		\centering{\footnotesize{(a) high skill wage}}
		%	\captionsetup{width=.45\linewidth}
		\includegraphics[width=1\textwidth]{../../codding_model/own_basedOnFried/optimalPol_elastS_DisuSci/figures/all_1705/wh_CompEffOPT_T_NoTaus_spillover0_sep1_BN0_ineq0_red0_etaa0.79_lgd1.png}
	\end{minipage}
	\begin{minipage}[]{0.24\textwidth}
		\centering{\footnotesize{(b) Low skill wage}}
		%	\captionsetup{width=.45\linewidth}
		\includegraphics[width=1\textwidth]{../../codding_model/own_basedOnFried/optimalPol_elastS_DisuSci/figures/all_1705/wl_CompEffOPT_T_NoTaus_spillover0_sep1_BN0_ineq0_red0_etaa0.79_lgd0.png}
	\end{minipage}
\begin{minipage}[]{0.24\textwidth}
	\centering{\footnotesize{(c) High skill hours}}
	%	\captionsetup{width=.45\linewidth}
	\includegraphics[width=1\textwidth]{../../codding_model/own_basedOnFried/optimalPol_elastS_DisuSci/figures/all_1705/hh_CompEffOPT_T_NoTaus_spillover0_sep1_BN0_ineq0_red0_etaa0.79_lgd0.png}
\end{minipage}
\begin{minipage}[]{0.24\textwidth}
	\centering{\footnotesize{(d) Low skill hours}}
	%	\captionsetup{width=.45\linewidth}
	\includegraphics[width=1\textwidth]{../../codding_model/own_basedOnFried/optimalPol_elastS_DisuSci/figures/all_1705/hl_CompEffOPT_T_NoTaus_spillover0_sep1_BN0_ineq0_red0_etaa0.79_lgd0.png}
\end{minipage}
\end{figure} 

Figure  \ref{fig:Compno_taul_BN0} compares the allocation under the policy regime with income tax, the black solid lines, to the counterfactual regime without income tax, the orange dashed graphs. 
The first thing to note is that a fossil tax suffices to meet the emission target: As shown by panel (a), net emissions are similar under both policy regimes. To achieve this level of emissions, the optimal fossil tax is slightly higher absent an income tax from 2030 onwards; compare panel (b).

% how labour income tax contributes to meeting the target
The advantage from relying on labour income taxes to meet the emission target stems from a higher utility from leisure. Less time spent working, especially for high-skilled workers,  outweighs lower consumption levels; compare panels (c) to (e). In fact, the rise in social welfare arises from the periods with  emission targets, especially under the net-zero emission limit, as shown by panel (f) which compares social welfare levels across policy regimes. A comparison of welfare measured in present value  shows that using income taxes amounts to a welfare gain of 0.1\% over the period from 2020 to 2080.

Hence, households work too high hours absent an income tax. % The higher hours worked do not translate into a sufficient rise in consumption which would outweigh the disutility from labour since fossil output is constrained. 
As green energy is not a perfect substitute for fossil energy and, in addition, energy and non-energy goods are complements, the cap on fossil output prevents a rise in final production which would compensate the additional hours worked. 
Panels (g) to (l) compare labour and research effort for the three different sectors. labour and research input in the fossil sector remain unchanged to meet emission targets irrespective of the availability of an income tax, compare panels (g) and (j). Higher hours worked result in too high labour effort and research in the green and non-energy sector. In other words, there are utility gains from reducing productivity growth in exchange for more leisure time.

Income taxes can be used to boost research through a labour supply channel: By subsidising labour supply production increases rendering research more profitable. When the government does not care about emissions, it optimally implements a regressive tax to boost research; compare panel (a) in figure \ref{fig:Compno_eff_BN0_notarget} in the appendix. %This market size effect might make more growth infeasible as it is intensified by higher labour supply.\footnote{\ It might be the case that due to the positive correlation of growth and labour supply, the emission target requires a stronger reduction in an endogenous growth model than in an exogenous one. % investigate this in the literature This finding would be in difference to the result in \cite{Fried2018ClimateAnalysis}, who finds that a lower corrective tax suffices to meet en exogenous emission target as market mechanisms boost green innovation.} 


%Although consumption rises due to the higher work effort when there is no income tax, the gains from labour effort are diminished due to the cap on fossil energy. Since green and fossil energy are no perfect substitutes, the economy cannot profit as much from the rise in green energy. HYPOTHESIS: WITH ENERGY SOURCES BEING BETTER COMPLEMENTS, WORK EFFORTS WOULD BE MORE FRUITEFUL. The muted effect of green energy on total energy output is intensified when considering total output where input goods are complements. 
%\begin{figure}
%
%\begin{minipage}[]{0.32\textwidth}
%	\centering{\footnotesize{\ \\(p) Energy output}}
%	%	\captionsetup{width=.45\linewidth}
%	\includegraphics[width=1\textwidth]{../../codding_model/own_basedOnFried/optimalPol_elastS_DisuSci/figures/all_1705/comp_notaul_OPT_T_NoTaus_E_spillover0_sep1_BN0_ineq0_etaa0.79.png}
%\end{minipage}
%\begin{minipage}[]{0.32\textwidth}
%	\centering{\footnotesize{\ \\(q) Final output}}
%	%	\captionsetup{width=.45\linewidth}
%	\includegraphics[width=1\textwidth]{../../codding_model/own_basedOnFried/optimalPol_elastS_DisuSci/figures/all_1705/comp_notaul_OPT_T_NoTaus_Y_spillover0_sep1_BN0_ineq0_etaa0.79.png}
%\end{minipage}
%\end{figure}
 
\subsubsection{Comparison to social planner allocation}\label{subsub:compeff}


In figure \ref{fig:Compno_eff_BN0}, I contrast the optimal allocation under the regime with income tax, the blue dashed lines, and without income tax, the orange dotted graph, to the efficient allocation  a social planner would choose, the black solid graph. 
Without income tax, the Ramsey planner matches the low and high skill ratio a social planner would choose in all periods; compare panel (a). Yet, both skills are supplied in too high amounts in periods with emission target; as shown by panels (b) and (c). The gap widens, the stricter the target. 

Reverting to an income tax allows the Ramsey planner to close this gap by reducing hours supplied. This, however, comes at the cost of inefficiently low relative supply of high skill labour; panel (a). When the emission target becomes active in 2030, the optimal supply of high-skilled labour falls below the efficient level while the supply of low-skill labour is inefficiently high. The reason is that the response of high-skilled labour to the income tax is more pronounced than that of low-skilled labour due to the skill premium. While income tax progressivity affects the supply of both skills equivalently through an income effect, the substitution effect is more pronounced for high-skill workers.

Interestingly, with an income tax, the supply of both skills is inefficiently high when there is no emission target although levels closer to the efficient allocation would be attainable at zero tax progressivity. Yet, as a result of higher work effort, consumption is closer to the efficient level in the world with income taxes; see panel (d).

Despite the lower work effort until 2050, the efficient allocation implies higher consumption levels for all years considered, panel (c). More consumption is caused by a higher technology levels under the social planner; see panels (e) to (g) in all sectors.\footnote{\ The abrupt reduction in skill effort under the social planner could be driven by the missing continuation value in the social and Ramsey planner problem. }

Looking at labour input, panels (j) to (l), and research, panels (h) to (i), highlights that in the optimal allocation with income taxation the planner forfeits an efficient rise in growth in order to align leisure time closer to the efficient level. As noted earlier, the muted effect on growth due to the cap on fossil energy makes additional work efforts too costly. 
%Panel (h) depicts the ratio of scientists employed in the green relative to the fossil sector. In all planner versions considered, the ratio increases as the emission target becomes stricter. It is remarkably that the social planner chooses a lower ratio of green to fossil scientists compared to the Ramsey planner. This picture underlines that work and research effort in the green and neutral sector under the optimal policy are too high despite an arguably emission-reducing compostition as shown by panels (i) and (k). 

In sum, the social planner meets the emission target at higher technology levels and generally lower hours worked. This allocation is not attainable for the Ramsey planner as a reduction in hours worked comes at a recomposition of skill supply. Furthermore, the government is not able to achieve efficient growth rates as both machine producers and scientists do not internalise the social value of research.  One way to increase research would be through higher labour effort. Yet, the overall return to more hours worked is too low given the cap on fossil energy. 
%If it would, hours worked would increase as the marginal product of labour rises. However, higher work effort at higher productivity would violate the emission target. 
%(WITH ONE SKILL THE RAMSEY PLANNER SHOULD BE ABLE TO MEET SAME WORK ALLOCATION... OR IS THE TRADE OFF (A) HOUSEHOLDS WORK LESS \ar (B) LOWER RESEARCH INPUT \ar Compare to version without endogenous growth (say gov can choose technology level within a range)!)

% IDEA: study how the Ramsey planner achieves emission target when having to accept efficient growth levels!
 
 %IDEQ2: Interpretation: we cannot grow more because households would work too much. WRONG as in this specification work effort is independent of technology growth. But instead: More work effort would be needed to foster more research but this would violate the emission target!
 
%\subsubsection{Role of skill heterogeneity}
\begin{figure}[h!!]
	\centering
	\caption{Comparison to efficient allocation }\label{fig:Compno_eff_BN0}
		\begin{minipage}[]{0.32\textwidth}
		\centering{\footnotesize{(a) High to low skill hours worked}}
		%	\captionsetup{width=.45\linewidth}
		\includegraphics[width=1\textwidth]{../../codding_model/own_basedOnFried/optimalPol_elastS_DisuSci/figures/all_1705/hhhl_CompEffOPT_T_NoTaus_spillover0_sep1_BN0_ineq0_red0_etaa0.79_lgd1.png}
	\end{minipage}
	\begin{minipage}[]{0.32\textwidth}
		\centering{\footnotesize{(b) High skill hours worked}}
		%	\captionsetup{width=.45\linewidth}
		\includegraphics[width=1\textwidth]{../../codding_model/own_basedOnFried/optimalPol_elastS_DisuSci/figures/all_1705/hh_CompEffOPT_T_NoTaus_spillover0_sep1_BN0_ineq0_red0_etaa0.79_lgd0.png}
	\end{minipage}
	\begin{minipage}[]{0.32\textwidth}
		\centering{\footnotesize{(c) Low skill hours worked}}
		%	\captionsetup{width=.45\linewidth}
		\includegraphics[width=1\textwidth]{../../codding_model/own_basedOnFried/optimalPol_elastS_DisuSci/figures/all_1705/hl_CompEffOPT_T_NoTaus_spillover0_sep1_BN0_ineq0_red0_etaa0.79_lgd0.png}
	\end{minipage}
	\begin{minipage}[]{0.32\textwidth}
		\centering{\footnotesize{(d) Consumption}}
		%	\captionsetup{width=.45\linewidth}
		\includegraphics[width=1\textwidth]{../../codding_model/own_basedOnFried/optimalPol_elastS_DisuSci/figures/all_1705/C_CompEffOPT_T_NoTaus_spillover0_sep1_BN0_ineq0_red0_etaa0.79_lgd0.png}
	\end{minipage}
	\begin{minipage}[]{0.32\textwidth}
		\centering{\footnotesize{\ \\(e)  Technology fossil}}
		%	\captionsetup{width=.45\linewidth}
		\includegraphics[width=1\textwidth]{../../codding_model/own_basedOnFried/optimalPol_elastS_DisuSci/figures/all_1705/Af_CompEffOPT_T_NoTaus_spillover0_sep1_BN0_ineq0_red0_etaa0.79_lgd0.png}
	\end{minipage}
\begin{minipage}[]{0.32\textwidth}
\centering{\footnotesize{\ \\(f) Technology green}}
%	\captionsetup{width=.45\linewidth}
\includegraphics[width=1\textwidth]{../../codding_model/own_basedOnFried/optimalPol_elastS_DisuSci/figures/all_1705/Ag_CompEffOPT_T_NoTaus_spillover0_sep1_BN0_ineq0_red0_etaa0.79_lgd0.png}
\end{minipage}
	\begin{minipage}[]{0.32\textwidth}
		\centering{\footnotesize{\ \\(g) Technology neutral}}
		%	\captionsetup{width=.45\linewidth}
		\includegraphics[width=1\textwidth]{../../codding_model/own_basedOnFried/optimalPol_elastS_DisuSci/figures/all_1705/An_CompEffOPT_T_NoTaus_spillover0_sep1_BN0_ineq0_red0_etaa0.79_lgd0.png}
	\end{minipage}
	\begin{minipage}[]{0.32\textwidth}
		\centering{\footnotesize{\ \\(h) Green scientists}}
		%	\captionsetup{width=.45\linewidth}
		\includegraphics[width=1\textwidth]{../../codding_model/own_basedOnFried/optimalPol_elastS_DisuSci/figures/all_1705/sg_CompEffOPT_T_NoTaus_spillover0_sep1_BN0_ineq0_red0_etaa0.79_lgd0.png}
	\end{minipage}
	\begin{minipage}[]{0.32\textwidth}
	\centering{\footnotesize{\ \\(i) Fossil scientists}}
	%	\captionsetup{width=.45\linewidth}
	\includegraphics[width=1\textwidth]{../../codding_model/own_basedOnFried/optimalPol_elastS_DisuSci/figures/all_1705/sff_CompEffOPT_T_NoTaus_spillover0_sep1_BN0_ineq0_red0_etaa0.79_lgd0.png}
\end{minipage}
%	\begin{minipage}[]{0.32\textwidth}
%		\centering{\footnotesize{\ \\(h) Scientists fossil}}
%		%	\captionsetup{width=.45\linewidth}
%		\includegraphics[width=1\textwidth]{../../codding_model/own_basedOnFried/optimalPol_elastS_DisuSci/figures/all_1705/sff_CompEffOPT_T_NoTaus_spillover0_sep1_BN0_ineq0_red0_etaa0.79_lgd0.png}
%	\end{minipage}
%	\begin{minipage}[]{0.32\textwidth}
%		\centering{\footnotesize{\ \\(i) Scientists neutral}}
%		%	\captionsetup{width=.45\linewidth}
%		\includegraphics[width=1\textwidth]{../../codding_model/own_basedOnFried/optimalPol_elastS_DisuSci/figures/all_1705/sn_CompEffOPT_T_NoTaus_spillover0_sep1_BN0_ineq0_red0_etaa0.79_lgd0.png}
%	\end{minipage}
	\begin{minipage}[]{0.32\textwidth}
		\centering{\footnotesize{\ \\(j) labour green}}
		%	\captionsetup{width=.45\linewidth}
		\includegraphics[width=1\textwidth]{../../codding_model/own_basedOnFried/optimalPol_elastS_DisuSci/figures/all_1705/Lg_CompEffOPT_T_NoTaus_spillover0_sep1_BN0_ineq0_red0_etaa0.79_lgd0.png}
	\end{minipage}
	\begin{minipage}[]{0.32\textwidth}
		\centering{\footnotesize{\ \\(k) labour fossil}}
		%	\captionsetup{width=.45\linewidth}
		\includegraphics[width=1\textwidth]{../../codding_model/own_basedOnFried/optimalPol_elastS_DisuSci/figures/all_1705/Lf_CompEffOPT_T_NoTaus_spillover0_sep1_BN0_ineq0_red0_etaa0.79_lgd0.png}
	\end{minipage}
	\begin{minipage}[]{0.32\textwidth}
		\centering{\footnotesize{\ \\(l) labour neutral}}
		%	\captionsetup{width=.45\linewidth}
		\includegraphics[width=1\textwidth]{../../codding_model/own_basedOnFried/optimalPol_elastS_DisuSci/figures/all_1705/Ln_CompEffOPT_T_NoTaus_spillover0_sep1_BN0_ineq0_red0_etaa0.79_lgd0.png}
	\end{minipage}
\end{figure}
%\subsection{Sensitivity}\label{sec:sens}

\section{Results with inequality}

\begin{figure}[h!!]
	\centering
	\caption{Optimal Policy }\label{fig:optPol_inq}
	\begin{minipage}[]{0.4\textwidth}
		\centering{\footnotesize{(a) Income tax progressivity, $\tau_{lt}$}}
		%	\captionsetup{width=.45\linewidth}
		\includegraphics[width=1\textwidth]{../../codding_model/own_basedOnFried/optimalPol_elastS_DisuSci/figures/all_1705/Single_OPT_T_NoTaus_taul_spillover0_sep1_BN0_ineq0_etaa0.79.png}
	\end{minipage}
	\begin{minipage}[]{0.1\textwidth}
		\
	\end{minipage}
	\begin{minipage}[]{0.4\textwidth}
		\centering{\footnotesize{(b) Fossil tax, $\tau_{ft}$ }}
		%	\captionsetup{width=.45\linewidth}
		\includegraphics[width=1\textwidth]{../../codding_model/own_basedOnFried/optimalPol_elastS_DisuSci/figures/all_1705/taul_TargetCompOPT_T_NoTaus_spillover0_sep1_BN0_ineq1_red0_etaa0.79_lgd1.png}
	\end{minipage}
\end{figure} 

%\section{Quantitative Results}

In this section, I, first, discuss the quantitative results.
Subsection AA presents the optimal allocation and policy given the emission target. Subsection BB discusses the results. In particular, I focus on understanding the role and importance of tax progressivity. 

\subsection{Main results}
\begin{figure}[h!!]
	\centering
	\caption{Optimal Policy }\label{fig:optPol}
	\begin{minipage}[]{0.4\textwidth}
		\centering{\footnotesize{(a) Income tax progressivity, $\tau_{lt}$}}
		%	\captionsetup{width=.45\linewidth}
		\includegraphics[width=1\textwidth]{../../codding_model/own_basedOnFried/optimalPol_elastS_DisuSci/figures/all_1705/Single_OPT_T_NoTaus_taul_spillover0_sep1_BN0_ineq0_etaa0.79.png}
	\end{minipage}
\begin{minipage}[]{0.1\textwidth}
\
\end{minipage}
	\begin{minipage}[]{0.4\textwidth}
		\centering{\footnotesize{(b) Fossil tax, $\tau_{ft}$ }}
		%	\captionsetup{width=.45\linewidth}
		\includegraphics[width=1\textwidth]{../../codding_model/own_basedOnFried/optimalPol_elastS_DisuSci/figures/all_1705/Single_OPT_T_NoTaus_tauf_spillover0_sep1_BN0_ineq0_etaa0.79.png}
	\end{minipage}
\end{figure} 
%\begin{figure}[h!!]
%	\centering
%	\caption{Optimal Policy }\label{fig:optPol}
%	\begin{minipage}[]{0.4\textwidth}
%		\centering{\footnotesize{(a) Income tax progressivity, $\tau_{lt}$}}
%		%	\captionsetup{width=.45\linewidth}
%		\includegraphics[width=1\textwidth]{../../codding_model/own_basedOnFried/optimalPol_elastS_DisuSci/figures/all_1705/Single_OPT_T_NoTaus_taul_spillover0_sep1_BN1_ineq0_etaa0.79.png}
%	\end{minipage}
%	\begin{minipage}[]{0.1\textwidth}
%		\
%	\end{minipage}
%	\begin{minipage}[]{0.4\textwidth}
%		\centering{\footnotesize{(b) Fossil tax, $\tau_{ft}$ }}
%		%	\captionsetup{width=.45\linewidth}
%		\includegraphics[width=1\textwidth]{../../codding_model/own_basedOnFried/optimalPol_elastS_DisuSci/figures/all_1705/Single_OPT_T_NoTaus_tauf_spillover0_sep1_BN1_ineq0_etaa0.79.png}
%	\end{minipage}
%\end{figure}
To optimally meet the IPCCs suggested emission target, the optimal income tax is progressive. As the emission target is less strict, between 2030 to 2045, optimal income tax progressivity is around $\tau_{lt}=0.02$. As the emission target jumps to net-zero emissions in 2050, optimal tax progressivity accelerates to above 0.08 and gradually increases in the subsequent years to around 0.09. This is approximately half the size found for the US in \cite{Heathcote2017OptimalFramework}: $\tau_{l}=0.181$. 
In the period without emission target from 2020 to 2030, the optimal income tax is regressive.

The optimal fossil tax displays a similar step pattern as income tax progressivity. From 2020 to the beginning of 2030, it is negative. It jumps to around 50\% as the emission target is to reduce emissions by 50\% relative to 2019 emissions. As the emission target rises  to net-zero emissions in 2050, the optimal tax on fossil sales is close to 95\%. 

\begin{figure}[h!!]
	\centering
	\caption{Optimal Policy }\label{fig:optAll}
	\begin{minipage}[]{0.32\textwidth}
		\centering{\footnotesize{(a) Growth fossil sector}}
		%	\captionsetup{width=.45\linewidth}
		\includegraphics[width=1\textwidth]{../../codding_model/own_basedOnFried/optimalPol_elastS_DisuSci/figures/all_1705/Single_OPT_T_NoTaus_Af_spillover0_sep1_BN0_ineq0_etaa0.79.png}
	\end{minipage}
	\begin{minipage}[]{0.32\textwidth}
		\centering{\footnotesize{(b) Growth green sector }}
		%	\captionsetup{width=.45\linewidth}
		\includegraphics[width=1\textwidth]{../../codding_model/own_basedOnFried/optimalPol_elastS_DisuSci/figures/all_1705/Single_OPT_T_NoTaus_Ag_spillover0_sep1_BN0_ineq0_etaa0.79.png}
	\end{minipage}
\begin{minipage}[]{0.32\textwidth}
	\centering{\footnotesize{(c) Growth neutral sector}}
	%	\captionsetup{width=.45\linewidth}
	\includegraphics[width=1\textwidth]{../../codding_model/own_basedOnFried/optimalPol_elastS_DisuSci/figures/all_1705/Single_OPT_T_NoTaus_An_spillover0_sep1_BN0_ineq0_etaa0.79.png}
\end{minipage}
	\begin{minipage}[]{0.32\textwidth}
	\centering{\footnotesize{(d) Labour fossil sector}}
	%	\captionsetup{width=.45\linewidth}
	\includegraphics[width=1\textwidth]{../../codding_model/own_basedOnFried/optimalPol_elastS_DisuSci/figures/all_1705/Single_OPT_T_NoTaus_Lf_spillover0_sep1_BN0_ineq0_etaa0.79.png}
\end{minipage}
\begin{minipage}[]{0.32\textwidth}
	\centering{\footnotesize{(e) Low-skilled labour }}
	%	\captionsetup{width=.45\linewidth}
	\includegraphics[width=1\textwidth]{../../codding_model/own_basedOnFried/optimalPol_elastS_DisuSci/figures/all_1705/Single_OPT_T_NoTaus_hl_spillover0_sep1_BN0_ineq0_etaa0.79.png}
\end{minipage}
\begin{minipage}[]{0.32\textwidth}
	\centering{\footnotesize{(f) High-skilled labour}}
	%	\captionsetup{width=.45\linewidth}
	\includegraphics[width=1\textwidth]{../../codding_model/own_basedOnFried/optimalPol_elastS_DisuSci/figures/all_1705/Single_OPT_T_NoTaus_hh_spillover0_sep1_BN0_ineq0_etaa0.79.png}
\end{minipage}
	\begin{minipage}[]{0.32\textwidth}
	\centering{\footnotesize{(d) Consumption}}
	%	\captionsetup{width=.45\linewidth}
	\includegraphics[width=1\textwidth]{../../codding_model/own_basedOnFried/optimalPol_elastS_DisuSci/figures/all_1705/Single_OPT_T_NoTaus_C_spillover0_sep1_BN0_ineq0_etaa0.79.png}
\end{minipage}
\begin{minipage}[]{0.32\textwidth}
	\centering{\footnotesize{(e) Social Welfare}}
	%	\captionsetup{width=.45\linewidth}
	\includegraphics[width=1\textwidth]{../../codding_model/own_basedOnFried/optimalPol_elastS_DisuSci/figures/all_1705/Single_OPT_T_NoTaus_SWF_spillover0_sep1_BN0_ineq0_etaa0.79.png}
\end{minipage}
\begin{minipage}[]{0.32\textwidth}
	\centering{\footnotesize{(f) Emissions}}
	%	\captionsetup{width=.45\linewidth}
	\includegraphics[width=1\textwidth]{../../codding_model/own_basedOnFried/optimalPol_elastS_DisuSci/figures/all_1705/Single_OPT_T_NoTaus_Emnet_spillover0_sep1_BN0_ineq0_etaa0.79.png}
\end{minipage}
\end{figure} 

\subsection{Discussion}
To study the role of income tax progressivity, I compare the optimal policy and allocation in the full model to a  model where no labour income tax is available.

The first thing to note is that a fossil tax suffices to meet the emission target, panel (a) in figure \ref{fig:Compno_taul}. The fossil tax is slightly higher in periods with emission target, compare panel (b).
The advantage from relying on labour income taxes to meet the emission target stems from a higher utility from leisure especially from high-skilled workers which outweighs lower consumption levels, compare panels (c) to (e). In fact, the rise in social welfare arises from the periods with net-zero emission target as shown by panel (f) which compares social welfare levels across policy regimes. 

Income tax progressivity increases social welfare in the net-zero emission world, as households work inefficiently high hours absent an income tax. Higher hours worked result in too high labour effort and research in the green and non-energy sector. The allocation in the fossil sector remains unchanged to meet the emission target; compare panels (g) to (j). 

Although consumption rises due to the higher work effort when there is no income tax, the gains from labour effort are diminished due to the cap on fossil energy. Since green and fossil energy are no perfect substitutes, the economy cannot profit as much from the rise in green energy. HYPOTHESIS: WITH ENERGY SOURCES BEING BETTER COMPLEMENTS, WORK EFFORTS WOULD BE MORE FRUITEFUL. The muted effect of green energy on total energy output is intensified when considering total output where input goods are complements. 

\begin{figure}[h!!]
	\centering
	\caption{Optimal Policy }\label{fig:Compno_taul}
			\begin{minipage}[]{0.32\textwidth}
		\centering{\footnotesize{(a) Emissions}}
		%	\captionsetup{width=.45\linewidth}
		\includegraphics[width=1\textwidth]{../../codding_model/own_basedOnFried/optimalPol_elastS_DisuSci/figures/all_1705/comp_notaul_OPT_T_NoTaus_Emnet_spillover0_sep1_BN0_ineq0_etaa0.79_lgd1.png}
	\end{minipage}
		\begin{minipage}[]{0.32\textwidth}
		\centering{\footnotesize{(b) Fossil tax}}
		%	\captionsetup{width=.45\linewidth}
		\includegraphics[width=1\textwidth]{../../codding_model/own_basedOnFried/optimalPol_elastS_DisuSci/figures/all_1705/comp_notaul_OPT_T_NoTaus_tauf_spillover0_sep1_BN0_ineq0_etaa0.79.png}
	\end{minipage}
	\begin{minipage}[]{0.32\textwidth}
		\centering{\footnotesize{(c) Consumption}}
		%	\captionsetup{width=.45\linewidth}
		\includegraphics[width=1\textwidth]{../../codding_model/own_basedOnFried/optimalPol_elastS_DisuSci/figures/all_1705/comp_notaul_OPT_T_NoTaus_C_spillover0_sep1_BN0_ineq0_etaa0.79.png}
	\end{minipage}
	\begin{minipage}[]{0.32\textwidth}
		\centering{\footnotesize{\ \\(d) High skill }}
		%	\captionsetup{width=.45\linewidth}
		\includegraphics[width=1\textwidth]{../../codding_model/own_basedOnFried/optimalPol_elastS_DisuSci/figures/all_1705/comp_notaul_OPT_T_NoTaus_hh_spillover0_sep1_BN0_ineq0_etaa0.79.png}
	\end{minipage}
	\begin{minipage}[]{0.32\textwidth}
		\centering{\footnotesize{\ \\(e) Low skill}}
		%	\captionsetup{width=.45\linewidth}
		\includegraphics[width=1\textwidth]{../../codding_model/own_basedOnFried/optimalPol_elastS_DisuSci/figures/all_1705/comp_notaul_OPT_T_NoTaus_hl_spillover0_sep1_BN0_ineq0_etaa0.79.png}
	\end{minipage}
	\begin{minipage}[]{0.32\textwidth}
	\centering{\footnotesize{\ \\(f) Social welfare}}
	%	\captionsetup{width=.45\linewidth}
	\includegraphics[width=1\textwidth]{../../codding_model/own_basedOnFried/optimalPol_elastS_DisuSci/figures/all_1705/comp_notaul_OPT_T_NoTaus_SWF_spillover0_sep1_BN0_ineq0_etaa0.79.png}
\end{minipage}
	\begin{minipage}[]{0.32\textwidth}
		\centering{\footnotesize{\ \\(g) Labour fossil}}
		%	\captionsetup{width=.45\linewidth}
		\includegraphics[width=1\textwidth]{../../codding_model/own_basedOnFried/optimalPol_elastS_DisuSci/figures/all_1705/comp_notaul_OPT_T_NoTaus_Lf_spillover0_sep1_BN0_ineq0_etaa0.79.png}
	\end{minipage}
	\begin{minipage}[]{0.32\textwidth}
		\centering{\footnotesize{\ \\(h) Labour green}}
		%	\captionsetup{width=.45\linewidth}
		\includegraphics[width=1\textwidth]{../../codding_model/own_basedOnFried/optimalPol_elastS_DisuSci/figures/all_1705/comp_notaul_OPT_T_NoTaus_Lg_spillover0_sep1_BN0_ineq0_etaa0.79.png}
	\end{minipage}
\begin{minipage}[]{0.32\textwidth}
	\centering{\footnotesize{\ \\(i) Labour non-energy}}
	%	\captionsetup{width=.45\linewidth}
	\includegraphics[width=1\textwidth]{../../codding_model/own_basedOnFried/optimalPol_elastS_DisuSci/figures/all_1705/comp_notaul_OPT_T_NoTaus_Ln_spillover0_sep1_BN0_ineq0_etaa0.79.png}
\end{minipage}
\begin{minipage}[]{0.32\textwidth}
	\centering{\footnotesize{\ \\(j) Research fossil}}
	%	\captionsetup{width=.45\linewidth}
	\includegraphics[width=1\textwidth]{../../codding_model/own_basedOnFried/optimalPol_elastS_DisuSci/figures/all_1705/comp_notaul_OPT_T_NoTaus_sff_spillover0_sep1_BN0_ineq0_etaa0.79.png}
\end{minipage}
	\begin{minipage}[]{0.32\textwidth}
		\centering{\footnotesize{\ \\(k) Green research}}
		%	\captionsetup{width=.45\linewidth}
		\includegraphics[width=1\textwidth]{../../codding_model/own_basedOnFried/optimalPol_elastS_DisuSci/figures/all_1705/comp_notaul_OPT_T_NoTaus_sg_spillover0_sep1_BN0_ineq0_etaa0.79.png}
	\end{minipage}
\begin{minipage}[]{0.32\textwidth}
	\centering{\footnotesize{\ \\(l) Non-energy research }}
	%	\captionsetup{width=.45\linewidth}
	\includegraphics[width=1\textwidth]{../../codding_model/own_basedOnFried/optimalPol_elastS_DisuSci/figures/all_1705/comp_notaul_OPT_T_NoTaus_sn_spillover0_sep1_BN0_ineq0_etaa0.79.png}
\end{minipage}

\begin{minipage}[]{0.32\textwidth}
	\centering{\footnotesize{\ \\(m) Fossil output}}
	%	\captionsetup{width=.45\linewidth}
	\includegraphics[width=1\textwidth]{../../codding_model/own_basedOnFried/optimalPol_elastS_DisuSci/figures/all_1705/comp_notaul_OPT_T_NoTaus_F_spillover0_sep1_BN0_ineq0_etaa0.79.png}
\end{minipage}
\begin{minipage}[]{0.32\textwidth}
	\centering{\footnotesize{\ \\(n) Green output}}
	%	\captionsetup{width=.45\linewidth}
	\includegraphics[width=1\textwidth]{../../codding_model/own_basedOnFried/optimalPol_elastS_DisuSci/figures/all_1705/comp_notaul_OPT_T_NoTaus_G_spillover0_sep1_BN0_ineq0_etaa0.79.png}
\end{minipage}
\begin{minipage}[]{0.32\textwidth}
	\centering{\footnotesize{\ \\(o) Non-energy output}}
	%	\captionsetup{width=.45\linewidth}
	\includegraphics[width=1\textwidth]{../../codding_model/own_basedOnFried/optimalPol_elastS_DisuSci/figures/all_1705/comp_notaul_OPT_T_NoTaus_N_spillover0_sep1_BN0_ineq0_etaa0.79.png}
\end{minipage}
\end{figure} 
\begin{figure}

\begin{minipage}[]{0.32\textwidth}
	\centering{\footnotesize{\ \\(p) Energy output}}
	%	\captionsetup{width=.45\linewidth}
	\includegraphics[width=1\textwidth]{../../codding_model/own_basedOnFried/optimalPol_elastS_DisuSci/figures/all_1705/comp_notaul_OPT_T_NoTaus_E_spillover0_sep1_BN0_ineq0_etaa0.79.png}
\end{minipage}
\begin{minipage}[]{0.32\textwidth}
	\centering{\footnotesize{\ \\(q) Final output}}
	%	\captionsetup{width=.45\linewidth}
	\includegraphics[width=1\textwidth]{../../codding_model/own_basedOnFried/optimalPol_elastS_DisuSci/figures/all_1705/comp_notaul_OPT_T_NoTaus_Y_spillover0_sep1_BN0_ineq0_etaa0.79.png}
\end{minipage}
\end{figure}

Another central aspect of the paper is the importance of inequality for the optimal environmental policy. How does household heterogeneity in labour supply shape the optimal environmental policy? First, I hypothesise that the skill bias of the green sector makes a less progressive income tax optimal. 
\subsection{Sensitivity}
In this subsection, I discuss results under counterfactual parameter values to elicit the robustness of the main result: the preference of progressive labour taxation above higher fossil taxes. 
First, the productivity gap between sectors might be driving the results. Second, how do results change as within sector spillovers of research are positive? Third, I study the results of a more general specification of the utility function where income affects the choice of hours worked \citep{Boppart2019LaborPerspectiveb, Bick2018HowImplications}. 
\section{Conclusion}\label{sec:con}
Some scholars argue that  reductive policies are necessary to handle environmental limits \citep{Schor2005SustainableReductionb, VanVuuren2018AlternativeTechnologies, Bertram2018TargetedScenarios}, and the question has been raised whether consumption is too high \citep{Arrow2004AreMuch}. On the other hand, the focus of environmental policy discussions in economics rests on corrective environmental taxation. In the light of tightening environmental limits \citep{Rockstrom2009AHumanity, IPCC2022}, I study whether labor income taxes - as a reductive policy tool - can help mitigate externalities. 

In the analytical part of the paper, I show in a simple model that labor income taxes are  progressive as part of the optimal environmental policy. %The model does not feature inequality.
% Quantitative results
% baseline model
When environmental tax revenues are not redistributed lump sum, labor supply is inefficiently high. Then, income taxes serve to diminish hours worked closer to the efficient level. The result prevails absent income inequality.


% quantitative
In the second part of the paper, I analyze in a quantitative model with skill heterogeneity and endogenous growth whether the optimal labor income tax remains progressive. Again, there are no equity concerns, but workers are perfectly ensured against income differences. 
The optimal income tax is progressive to reduce inefficiently high hours worked. The quantitative model reveals that income taxes also serve as a substitute for corrective taxes. Knowledge spillovers from the non-energy sector render environmental taxes especially costly. 
Fossil taxes make energy relatively more expensive which directs research from non-energy to energy sectors. As the non-energy sector features the most research processes it is especially important for aggregate technology and knowledge spillovers. Using income taxes instead of fossil taxes to lower emissions allows to direct more research to the non-energy sector and to profit from knowledge spillovers.
In sum, however, the reduction in labor supply outweighs the positive effect on growth and consumption decreases compared to a scenario where no income tax is used. 

In the quantitative setting, the income tax affects the economic structure through two channels. First, because the fossil sector is comparably labor intense, a reduction in labor supply favors the green sector. This mechanism makes a higher tax progressivity optimal. However, the effect vanishes in equilibrium due to endogenous growth.
Second, a skill-recomposition channel makes green energy production more costly compared to fossil production. This effect arises from a skill bias in the green sector and high-skill labor being more responsive to income taxation. 
The second channel dominates the recomposing effect of  income tax progressivity in equilibrium. A market size effect amplifies the skill-recomposition channel directing research to the fossil sector. 

%Initially, the intention not to harm growth too much makes a lower progressivity optimal. As growth in the fossil sector accelerates due to the dynamic structure of endogenous growth too low progressive income taxes conflict with meeting the emission limit. As a result, optimal progressivity increases over time.
%The optimal path of income tax progressivity is decreasing, a feature mainly driven by endogenous growth. As a result, the optimal income tax progressivity and the optimal fossil tax seem to behave like substitutes in the quantitative model. 

%Skill heterogeneity depresses optimal tax progressivity due to the adverse recomposing effect of a lower high-to-low skill labor supply on the green-to-fossil energy ratio. A higher corrective tax is required to meet emission limits when there is only one skill type: with only one skill the supply of fossil-specific inputs increases thereby violating the emission limit.

%% lump-sum transfers
%When environmental tax revenues are redistributed lump-sum, the motive to use labor income taxes to deal with inefficiently high labor supply vanishes. Instead, income taxes serve to boost growth as long as this does not conflict with meeting emission limits. Therefore, they are regressive. 
%\tr{not true! it is rather that the more in research is not worth it given the dynamics! and decreasing utility gains}
%However, the regressivity decreases since more labor supply causes more emissions especially the more progressed the technology. With only a labor income tax as a tool to raise growth, accelerating technology growth is not feasible as it is concomitant with more production and emissions. 

% extensions
In an extension, I am planning to give the Ramsey planner the opportunity to limit working hours directly. The literature advocating a reduction in consumption levels \citep[e.g.,][]{Schor2005SustainableReductionb} proposes a restriction of hours worked as policy instrument to lower the consumption of resources.
Even though advocated in the literature, there is evidence for political difficulties in reducing working hours. In 2020, the French Citizens' Convention on Climate voted against reducing working hours as a measure to handle climate change. Potentially, ignorance about economic consequences is an explanation. The extension would serve to better understand economic consequences. 



\begin{comment}
\paragraph{Extension: What if the low skilled get a higher share \ar they reduce even less \ar more fossil input supply}

Redistribution to households with a higher marginal propensity to consume emissions counteracts the externality. This effect is amplified by a market size effect  of dirty goods. 

content...
\end{comment}

% I plan to discuss results under counterfactual parameter values to elicit the robustness of the main result: the preference of progressive labour taxation above higher fossil taxes. 
%First, the productivity gap between sectors might be driving the results. Second, I will abstract from endogenous growth to learn about the labour-supply-innovation channel as a driver of the optimal policy. Finally, I plan to study how results change as returns to research are increasing within sector. 
%Due to the endogeneity of technological growth in the model, the reduction in work effort fosters less research especially in the non-energy sector.  %However, more hours worked in the Ramsey model fostering research would violate the emission target. As a result, growth in technology and in consumption is inefficiently low in order to meet the emission target. 

\begin{comment}
To shed more light on the main findings, I plan conduct several additional quantitative experiments. First, I want to reduce the size of the emission target, second, I allow for a longer time frame until net-zero emissions have to be reached. The IPCC report states that for a temperature target of 2°C net-zero emissions have to be reached by 2070 only. How does this laxer target affect the importance of labour income taxes. Given the wider time frame, the green sector might be able to catch up and growth could continue. Finally, how does a change in spillovers shape the result? % \textit{(Question: I guess that substitutability is key here! Growth in green implies growths in fossil when goods are no perfect substitutes! )}
content...

%Another central aspect of the paper is the importance of inequality for the optimal environmental policy. How does household heterogeneity in labour supply shape the optimal environmental policy? First, I hypothesise that the skill bias of the green sector makes a less progressive income tax optimal. 
One main result of the paper is reduction of consumption and work effort as an optimal policy. So far, I have assumed that households are passive and preferences are fixed; there is no trade-off between environmental quality and consumption from a household perspective.
In an extension to the baseline model, I plan to depart from the representative agent assumption and explicitly model household heterogeneity. This setting allows to capture a change in household behaviour: A share of households is willing to voluntarily reduce consumption. I provide evidence for such behaviour using a representative Dutch dataset. More than 50\% of households are willing to reduce consumption in order to help the economy. Importantly, these households have a higher likelihood to work in the green sector. How does such a change in behaviour affect the optimal policy? Given the additional reduction in green-specific labour supply, the planner might find it optimal to set a more regressive tax to booster green production and research.    

\end{comment}

%However, data suggests, that households do care, and they express a willingness to reduce consumption.\footnote{\ The data I have studied comes from the Liss Panel, a representative sample of Dutch households, more than 50\% of participants indicate a readiness to change their behaviour to help the environment.} I want to study the effect of such behavioural  change on the optimal policy. Interestingly, households in high-skill jobs are more likely to declare their willingness to reduce. This linkage may intensify the trade-off between reduction and green labour supply. 


%1) BN and inequality
%2) preferences for labour
\begin{comment}
Preferences and the trade-off between leisure and consumption determining household behaviour seem to be key to the results. As argued by \cite{Boppart2019labourPerspectiveb}, the intensive margin of hours worked have been falling steadily over the last 130 years. They argue for the consistency of preferences which feature a slightly higher income effect than substitution effect. In the current model with log-utility and representative family framework,  the substitution effects offset each other. With the preferences suggested in \cite{Boppart2019labourPerspectiveb}, growth would affect hours worked, assumably changing the optimal policy. It could, for instance, be the case, that growth has to be slowed down even more, to prevent too high work efforts and consumption levels. % high-income, high-skill households might increase their labour supply with growth. 

content...
\end{comment}



%Finally, endogenising growth constitutes another interesting trade-off when the impact of fiscal policy is skill specific. 
%As regards growth, it seems reasonable to consider growth as a change in the substitutability of dirty and clean goods in the final consumption good. As it stands now, growth in the dirty sector results in emission growth, ceteris paribus. Growth might instead be associated with a more efficient use of dirty energy sources, so that more output can be generated at lower emissions.
%
%Think about effects of government using revenues for other consumption. Then reducing demand will diminish demand for the final good. 
%Broadly speaking, there are two channels through which distortionary labour taxation affects emissions. First, by affecting households' labour supply decision (efficiency channel) and second in a mechanical way by changing households disposable income. The latter effect cancels out when tax revenues are used by the government to consume the final output good. Allowing the government to recycle revenues in a different way than for final good consumption uncloses another instrument to reduce emissions. 

%Further ideas for extensions: include behavioural aspects: a voluntary reduction in demand, and a lower disutility from working in the green sector.
\begin{comment}
\paragraph{Ways forward}
How to introduce compositional effects:
\begin{enumerate}
	\item 	Utility function: With substitution and income effect not canceling (u(c)=$\frac{c^{1-\gamma}}{1-\gamma},\ \gamma\neq 1$), the wage rate might play a role, depends on GE effects.
	\item endogenising skill supply (rep agent chooses how much skill to supply, but this he already does... / might need to introduce structure as in HSV)
	\item government revenues are not used for final good consumption. Instead,  disposed of/ used for sth useful (this could be an extension and contribute to benefits of progressivity) THINK THIS ONLY CHANGES THE LEVEL TOO!
\end{enumerate}
\paragraph{Point 1 above}
change the utility function in the code to see what happens, if $\frac{Y_d}{Y_c}$ is constant in particular 
\paragraph{Point 3 above}
\textcolor{blue}{2) Government consumption wasted}
Letting the government not consume the final output good may alter the result. 
Now, the aggregate price level is determined endogenously as the goods market does not clear by Walras' law. 

In the equilibrium equations, I drop $p_t=1$ and use goods market clearing instead\\ $Y=c+\psi (x_c+x_d)$.

Blödsinn, only changes level

content...
\end{comment}
%\clearpage
\appendix
\section{Model}\label{app:model}
\subsection{Competitive equilibrium in simple model}

\begin{align}
\text{Utility}\hspace{5mm}& \frac{C_t^{1-\theta}-1}{1-\theta}-\chi \frac{h_t^{1+\sigma}}{1+\sigma}-\varphi(\omega F)^\eta\\
\text{Budget}\hspace{5mm}& C_t = \lambda_t(w_th_t)^{1-\tau_{\iota t}}\\
\text{optimality HH}\hspace{5mm}& h^{\sigma+\tau_{\iota t}+\theta(1-\tau_{\iota t})}=\lambda_t^{1-\theta}(1-\tau_{\iota t})w_t^{(1-\tau_{\iota t})(1-\theta)}\\
\text{Final Production}\hspace{5mm}&Y=F^{\varepsilon_y}G^{1-
	\varepsilon_y}\\ %\left[F^\frac{\varepsilon_y-1}{\varepsilon_y}+G^\frac{\varepsilon_y-1}{\varepsilon_y}\right]^\frac{\varepsilon_y}{\varepsilon_y-1}\\
\text{price}\hspace{5mm}&1=p_y= \left(\frac{p_f}{\varepsilon_y}\right)^{\varepsilon_y}\left(\frac{p_g}{1-\varepsilon_y}\right)^{1-\varepsilon_y}\\
\text{Demand final prod}\hspace{5mm}&\frac{F}{G}=\left(\frac{\varepsilon_y}{1-\varepsilon_y}\right)\left(\frac{p_g}{p_f}\right)\\
\text{Production F and G}\hspace{5mm}&F=A_fL_f\\\
& G=A_gL_g\\
\text{labour demand}\hspace{5mm}& w=p_f(1-\tau_{ft})A_f\\
& w=p_gA_g\\
\text{technology}\hspace{5mm}&A_{ft+1}=(1+\nu_f)A_{ft}\\
&A_{gt+1}=(1+\nu_g)A_{gt}\\
\text{Government}\hspace{5mm}&T=\tau_{f}p_fF
\\
\text{Balanced income tax revenues}\hspace{5mm}&\lambda_t=\frac{w_t h_t}{(w_t h_t)^{1-\tau_{\iota t}}}\\
&E_{net}=\omega F-\delta
\end{align}

\section{Derivations and proofs}\label{app:derivations}
\subsection{Derivation expression for $h^{FB}$}
Rewriting equation \ref{eq:fbh}, the efficient amount of hours worked can be indirectly expressed as:
\begin{align}
h^{FB}=\frac{1}{\chi^\frac{1}{\sigma}}\left(w_{eff}^{1-\theta}-\frac{dE}{dF}A_f s^{FB}\left(h^{FB}\right)^\theta \right)^\frac{1}{\sigma+\theta}.\label{eq:heff_1}
\end{align}

Note that an explicit expression for $h^{FB}$ follows from equation \ref{eq:fbs} when there is an externality and $\frac{dE}{dF}\neq 0$. Then 

\begin{align}
h^{FB}= \left(\frac{\varepsilon(1-s)-s(1-\varepsilon)}{s(1-s)}\frac{w_{eff}^{1-\theta}}{\frac{dE}{dF}A_f }\right)
\end{align}
and the result follows from substituting the last expression in expression  \ref{eq:heff_1}.

\subsection{Proof: Hours worked with only the efficient share of dirty labor are inefficiently high}

\begin{proof}
The proof proceeds in two steps. First, I show that the share of labor allocated to the dirty sector is smaller than its efficient level absent externality which is $s=\varepsilon$.
In the second step, I show that even if the environmental tax is set to the tax which replicates the efficient share of dirty labor, denoted by $h_{CE, s^{eff}}$\footnote{\ In the competitive equilibrium, the share of dirty labor is solely determined by the environmental tax. The relation follows from labor market clearing and intermediate goods market clearing: $\tau_f=1-\frac{1-\varepsilon}{\varepsilon}\frac{s}{1-s}$.}, exceeds the efficient level of hours worked, $h_{FB}$, when neither lump-sum transfers no labour income taxes are available.

\textbf{Step 1:} $\frac{dE}{dF}>0$ \ar $\varepsilon>s$\\
Rewriting equation \ref{eq:fbs} yields
\begin{align}
\frac{\varepsilon(1-s)-s(1-\varepsilon)}{s(1-s)}=\frac{dE}{dF}A_fh^\theta w_{FB}^{1-\theta}.
\end{align}
When the externality is negative, i.e., $\frac{dE}{dF}>0$, then the right-hand side is positive.
Since $s\in(0,1)$ - due to both intermediate goods being necessary to produce the final good and zero consumption is not a solution - the left-hand side is positive when
\begin{align}
\varepsilon(1-s)-s(1-\varepsilon)>0,
\end{align}
which holds true if and only if $\varepsilon>s$.

\textbf{Step 2:} $\varepsilon>s$ \ar $h_{CE, s^{eff}}>h_{FB}$\\
I prove the claim by evoking a contradiction to the assumption that $h_{CE, s^{eff}}\leq h_{FB}$. Using equation \ref{eq:hopt} and \ref{eq:heff} the expression becomes

\begin{align}
&\left(\frac{w^{1-\theta}}{\chi}\right)^{\frac{1}{\sigma+\theta}}\leq \left(\frac{w_{FB}^{1-\theta}}{\chi}\frac{1-\varepsilon}{1-s}\right)^\frac{1}{\sigma+\theta}
\\
\Leftrightarrow&\left(\frac{w}{w_{FB}}\right)^{\frac{1-\theta}{\sigma+\theta}}\leq \left(\frac{1-\varepsilon}{1-s}\right)^\frac{1}{\sigma+\theta}
\end{align}

Note that the ratio of the wage in the competitive economy to the marginal product of labor is $\frac{w}{w_{FB}}=\frac{1-\varepsilon}{1-s}$, which follows from equation \ref{eq:compw} and the definition of $w_{FB}$ under the assumption that the dirty labor share is set to the first best equivalent. Substituting this in the previous equation and rearranging terms yields
\begin{align}
\left(\frac{1}{1-\varepsilon}\right)^\frac{\theta}{\sigma+\theta}\leq \left(\frac{1}{1-s}\right)^\frac{\theta}{\sigma+\theta}
\end{align}
which holds true whenever
\begin{align}
\varepsilon<s.
\end{align}
This contradicts $s<\varepsilon$ which has been shown to hold in presence of a negative externality in the dirty sector in step 1. 
\end{proof}

\subsection{Proof: lump-sum transfers restore the efficient allocation}
\begin{proof}
To establish that lump-sum transfers of environmental tax revenues restore the efficient allocation, I first derive the size of lump-sum transfers which implement the efficient level of hours worked given that the efficient dirty labor share is established by choice of the environmental tax. In a second step, I show that this level of transfers coincides with the revenues from the environmental tax when this is set to implement the efficient dirty labor share. 
These two steps prove that the efficient amount of hours results from lump-sum transferring environmental tax revenues. Finally, I show that the resulting level of consumption is efficient. This completes the proof.
\textbf{Step 1:}
I equalize equation \ref{eq:heff} and \ref{eq:hopt} setting the income tax progressivity, $\tau_{\iota}$, to zero.
\begin{align}
\left(\frac{w^{1-\theta}\left(1+\frac{T}{wh}\right)^{-\theta}}{\chi}\right)^\frac{1}{\sigma+\theta}=\left(\frac{w_{FB}^{1-\theta}}{\chi}\frac{1-\varepsilon}{1-s}\right)^\frac{1}{\sigma+\theta}.
\end{align}
Using the relation of $w_{FB}$ and $w$ established in the previous proof, I can rewrite the right-hand side
\begin{align}
&\left(\frac{w^{1-\theta}\left(1+\frac{T}{wh}\right)^{-\theta}}{\chi}\right)^\frac{1}{\sigma+\theta}=\left(\frac{w_{}^{1-\theta}}{\chi}\left(\frac{1-s}{1-\varepsilon}\right)^{1-\theta}\frac{1-\varepsilon}{1-s}\right)^\frac{1}{\sigma+\theta}.
\Leftrightarrow\ & 
\end{align}
This step is instructive in showing that transfers will correct for the two inefficiencies in the competitive economy: (i) the too low wage rate captured by the term $\left(\frac{1-s}{1-\varepsilon}\right)^{1-\theta}$, and (ii) the neglect of the effect of hours worked on the externality, captures by $\frac{1-\varepsilon}{1-s}<1$. 
\end{proof}

\section{Results}
\begin{figure}[h!!]
	\centering
	\caption{Comparison to efficient allocation absent emission target }\label{fig:Compno_eff_BN0_notarget}
		\begin{minipage}[]{0.32\textwidth}
		\centering{\footnotesize{(a) Income tax progressivity, $\tau_{lt}$}}
		%	\captionsetup{width=.45\linewidth}
		\includegraphics[width=1\textwidth]{../../codding_model/own_basedOnFried/optimalPol_elastS_DisuSci/figures/all_1705/taul_CompEffOPT_NOT_NoTaus_spillover0_sep1_BN0_ineq0_red0_etaa0.79_lgd1.png}
	\end{minipage}
\begin{minipage}[]{0.32\textwidth}
\centering{\footnotesize{(a) Income tax progressivity, $\tau_{lt}$; No skill}}
%	\captionsetup{width=.45\linewidth}
\includegraphics[width=1\textwidth]{../../codding_model/own_basedOnFried/optimalPol_elastS_DisuSci/figures/all_1705/taul_CompEffOPT_NOT_NoTaus_spillover0_noskill1_sep1_BN0_ineq0_red0_etaa0.79_lgd1.png}
\end{minipage}
	\begin{minipage}[]{0.32\textwidth}
	\centering{\footnotesize{(b) Corrective tax, $\tau_{ft}$}}
	%	\captionsetup{width=.45\linewidth}
	\includegraphics[width=1\textwidth]{../../codding_model/own_basedOnFried/optimalPol_elastS_DisuSci/figures/all_1705/tauf_CompEffOPT_NOT_NoTaus_spillover0_sep1_BN0_ineq0_red0_etaa0.79_lgd0.png}
\end{minipage}
	\begin{minipage}[]{0.32\textwidth}
		\centering{\footnotesize{(c) High skill labour}}
		%	\captionsetup{width=.45\linewidth}
		\includegraphics[width=1\textwidth]{../../codding_model/own_basedOnFried/optimalPol_elastS_DisuSci/figures/all_1705/hh_CompEffOPT_NOT_NoTaus_spillover0_sep1_BN0_ineq0_red0_etaa0.79_lgd0.png}
	\end{minipage}
	\begin{minipage}[]{0.32\textwidth}
		\centering{\footnotesize{(d) Low skill labour}}
		%	\captionsetup{width=.45\linewidth}
		\includegraphics[width=1\textwidth]{../../codding_model/own_basedOnFried/optimalPol_elastS_DisuSci/figures/all_1705/hl_CompEffOPT_NOT_NoTaus_spillover0_sep1_BN0_ineq0_red0_etaa0.79_lgd0.png}
	\end{minipage}
	\begin{minipage}[]{0.32\textwidth}
		\centering{\footnotesize{(e) Consumption}}
		%	\captionsetup{width=.45\linewidth}
		\includegraphics[width=1\textwidth]{../../codding_model/own_basedOnFried/optimalPol_elastS_DisuSci/figures/all_1705/C_CompEffOPT_NOT_NoTaus_spillover0_sep1_BN0_ineq0_red0_etaa0.79_lgd0.png}
	\end{minipage}
%	\begin{minipage}[]{0.32\textwidth}
%		\centering{\footnotesize{(f) Technology green}}
%		%	\captionsetup{width=.45\linewidth}
%		\includegraphics[width=1\textwidth]{../../codding_model/own_basedOnFried/optimalPol_elastS_DisuSci/figures/all_1705/Ag_CompEffOPT_NOT_NoTaus_spillover0_sep1_BN0_ineq0_etaa0.79_lgd0.png}
%	\end{minipage}
%	\begin{minipage}[]{0.32\textwidth}
%		\centering{\footnotesize{(g)  Technology fossil}}
%		%	\captionsetup{width=.45\linewidth}
%		\includegraphics[width=1\textwidth]{../../codding_model/own_basedOnFried/optimalPol_elastS_DisuSci/figures/all_1705/Af_CompEffOPT_NOT_NoTaus_spillover0_sep1_BN0_ineq0_red0_etaa0.79_lgd0.png}
%	\end{minipage}
%	\begin{minipage}[]{0.32\textwidth}
%		\centering{\footnotesize{(h) Technology neutral}}
%		%	\captionsetup{width=.45\linewidth}
%		\includegraphics[width=1\textwidth]{../../codding_model/own_basedOnFried/optimalPol_elastS_DisuSci/figures/all_1705/An_CompEffOPT_NOT_NoTaus_spillover0_sep1_BN0_ineq0_red0_etaa0.79_lgd0.png}
%	\end{minipage}
	\begin{minipage}[]{0.32\textwidth}
		\centering{\footnotesize{(i) Scientists green}}
		%	\captionsetup{width=.45\linewidth}
		\includegraphics[width=1\textwidth]{../../codding_model/own_basedOnFried/optimalPol_elastS_DisuSci/figures/all_1705/sg_CompEffOPT_NOT_NoTaus_spillover0_sep1_BN0_ineq0_red0_etaa0.79_lgd0.png}
	\end{minipage}
	\begin{minipage}[]{0.32\textwidth}
		\centering{\footnotesize{(j) Scientists fossil}}
		%	\captionsetup{width=.45\linewidth}
		\includegraphics[width=1\textwidth]{../../codding_model/own_basedOnFried/optimalPol_elastS_DisuSci/figures/all_1705/sff_CompEffOPT_NOT_NoTaus_spillover0_sep1_BN0_ineq0_red0_etaa0.79_lgd0.png}
	\end{minipage}
	\begin{minipage}[]{0.32\textwidth}
		\centering{\footnotesize{(k) Scientists neutral}}
		%	\captionsetup{width=.45\linewidth}
		\includegraphics[width=1\textwidth]{../../codding_model/own_basedOnFried/optimalPol_elastS_DisuSci/figures/all_1705/sn_CompEffOPT_NOT_NoTaus_spillover0_sep1_BN0_ineq0_red0_etaa0.79_lgd0.png}
	\end{minipage}

\begin{minipage}[]{0.32\textwidth}
	\centering{\footnotesize{(i) Technology green}}
	%	\captionsetup{width=.45\linewidth}
	\includegraphics[width=1\textwidth]{../../codding_model/own_basedOnFried/optimalPol_elastS_DisuSci/figures/all_1705/Ag_CompEffOPT_NOT_NoTaus_spillover0_sep1_BN0_ineq0_red0_etaa0.79_lgd0.png}
\end{minipage}
\begin{minipage}[]{0.32\textwidth}
	\centering{\footnotesize{(j) Technology fossil}}
	%	\captionsetup{width=.45\linewidth}
	\includegraphics[width=1\textwidth]{../../codding_model/own_basedOnFried/optimalPol_elastS_DisuSci/figures/all_1705/Af_CompEffOPT_NOT_NoTaus_spillover0_sep1_BN0_ineq0_red0_etaa0.79_lgd0.png}
\end{minipage}
\begin{minipage}[]{0.32\textwidth}
	\centering{\footnotesize{(k) Technology neutral}}
	%	\captionsetup{width=.45\linewidth}
	\includegraphics[width=1\textwidth]{../../codding_model/own_basedOnFried/optimalPol_elastS_DisuSci/figures/all_1705/An_CompEffOPT_NOT_NoTaus_spillover0_sep1_BN0_ineq0_red0_etaa0.79_lgd0.png}
\end{minipage}
	\begin{minipage}[]{0.32\textwidth}
		\centering{\footnotesize{(l) labour green}}
		%	\captionsetup{width=.45\linewidth}
		\includegraphics[width=1\textwidth]{../../codding_model/own_basedOnFried/optimalPol_elastS_DisuSci/figures/all_1705/Lg_CompEffOPT_NOT_NoTaus_spillover0_sep1_BN0_ineq0_red0_etaa0.79_lgd0.png}
	\end{minipage}
	\begin{minipage}[]{0.32\textwidth}
		\centering{\footnotesize{(m) labour fossil}}
		%	\captionsetup{width=.45\linewidth}
		\includegraphics[width=1\textwidth]{../../codding_model/own_basedOnFried/optimalPol_elastS_DisuSci/figures/all_1705/Lf_CompEffOPT_NOT_NoTaus_spillover0_sep1_BN0_ineq0_red0_etaa0.79_lgd0.png}
	\end{minipage}
	\begin{minipage}[]{0.32\textwidth}
		\centering{\footnotesize{\ \\(l) labour neutral}}
		%	\captionsetup{width=.45\linewidth}
		\includegraphics[width=1\textwidth]{../../codding_model/own_basedOnFried/optimalPol_elastS_DisuSci/figures/all_1705/Ln_CompEffOPT_NOT_NoTaus_spillover0_sep1_BN0_ineq0_red0_etaa0.79_lgd0.png}
	\end{minipage}
	\begin{minipage}[]{0.32\textwidth}
		\centering{\footnotesize{(g) Final output}}
		%	\captionsetup{width=.45\linewidth}
		\includegraphics[width=1\textwidth]{../../codding_model/own_basedOnFried/optimalPol_elastS_DisuSci/figures/all_1705/Y_CompEffOPT_NOT_NoTaus_spillover0_sep1_BN0_ineq0_red0_etaa0.79_lgd0.png}
	\end{minipage}
	\begin{minipage}[]{0.32\textwidth}
		\centering{\footnotesize{(h) Energy output}}
		%	\captionsetup{width=.45\linewidth}
		\includegraphics[width=1\textwidth]{../../codding_model/own_basedOnFried/optimalPol_elastS_DisuSci/figures/all_1705/E_CompEffOPT_NOT_NoTaus_spillover0_sep1_BN0_ineq0_red0_etaa0.79_lgd0.png}
	\end{minipage}
	\begin{minipage}[]{0.32\textwidth}
		\centering{\footnotesize{(i) Neutral output}}
		%	\captionsetup{width=.45\linewidth}
		\includegraphics[width=1\textwidth]{../../codding_model/own_basedOnFried/optimalPol_elastS_DisuSci/figures/all_1705/N_CompEffOPT_NOT_NoTaus_spillover0_sep1_BN0_ineq0_red0_etaa0.79_lgd0.png}
	\end{minipage}
\end{figure}

\begin{figure}[h!!]
	\centering
	\caption{Comparison to efficient allocation absent emission target without skill }\label{fig:Compno_eff_BN0_notarget_noskill}
	\begin{minipage}[]{0.32\textwidth}
		\centering{\footnotesize{(a) Income tax progressivity, $\tau_{lt}$; No skill}}
		%	\captionsetup{width=.45\linewidth}
		\includegraphics[width=1\textwidth]{../../codding_model/own_basedOnFried/optimalPol_elastS_DisuSci/figures/all_1705/taul_CompEffOPT_NOT_NoTaus_spillover0_noskill1_sep1_BN0_ineq0_red0_etaa0.79_lgd1.png}
	\end{minipage}
	\begin{minipage}[]{0.32\textwidth}
		\centering{\footnotesize{(b) Corrective tax, $\tau_{ft}$}}
		%	\captionsetup{width=.45\linewidth}
		\includegraphics[width=1\textwidth]{../../codding_model/own_basedOnFried/optimalPol_elastS_DisuSci/figures/all_1705/tauf_CompEffOPT_NOT_NoTaus_spillover0_noskill1_sep1_BN0_ineq0_red0_etaa0.79_lgd0.png}
	\end{minipage}
	\begin{minipage}[]{0.32\textwidth}
		\centering{\footnotesize{(c) High skill labour}}
		%	\captionsetup{width=.45\linewidth}
		\includegraphics[width=1\textwidth]{../../codding_model/own_basedOnFried/optimalPol_elastS_DisuSci/figures/all_1705/hh_CompEffOPT_NOT_NoTaus_spillover0_noskill1_sep1_BN0_ineq0_red0_etaa0.79_lgd0.png}
	\end{minipage}
	\begin{minipage}[]{0.32\textwidth}
		\centering{\footnotesize{(d) Low skill labour}}
		%	\captionsetup{width=.45\linewidth}
		\includegraphics[width=1\textwidth]{../../codding_model/own_basedOnFried/optimalPol_elastS_DisuSci/figures/all_1705/hl_CompEffOPT_NOT_NoTaus_spillover0_noskill1_sep1_BN0_ineq0_red0_etaa0.79_lgd0.png}
	\end{minipage}
	\begin{minipage}[]{0.32\textwidth}
		\centering{\footnotesize{(e) Consumption}}
		%	\captionsetup{width=.45\linewidth}
		\includegraphics[width=1\textwidth]{../../codding_model/own_basedOnFried/optimalPol_elastS_DisuSci/figures/all_1705/C_CompEffOPT_NOT_NoTaus_spillover0_noskill1_sep1_BN0_ineq0_red0_etaa0.79_lgd0.png}
	\end{minipage}
	\begin{minipage}[]{0.32\textwidth}
		\centering{\footnotesize{(i) Scientists green}}
		%	\captionsetup{width=.45\linewidth}
		\includegraphics[width=1\textwidth]{../../codding_model/own_basedOnFried/optimalPol_elastS_DisuSci/figures/all_1705/sg_CompEffOPT_NOT_NoTaus_spillover0_noskill1_sep1_BN0_ineq0_red0_etaa0.79_lgd0.png}
	\end{minipage}
	\begin{minipage}[]{0.32\textwidth}
		\centering{\footnotesize{(j) Scientists fossil}}
		%	\captionsetup{width=.45\linewidth}
		\includegraphics[width=1\textwidth]{../../codding_model/own_basedOnFried/optimalPol_elastS_DisuSci/figures/all_1705/sff_CompEffOPT_NOT_NoTaus_spillover0_noskill1_sep1_BN0_ineq0_red0_etaa0.79_lgd0.png}
	\end{minipage}
	\begin{minipage}[]{0.32\textwidth}
		\centering{\footnotesize{(k) Scientists neutral}}
		%	\captionsetup{width=.45\linewidth}
		\includegraphics[width=1\textwidth]{../../codding_model/own_basedOnFried/optimalPol_elastS_DisuSci/figures/all_1705/sn_CompEffOPT_NOT_NoTaus_spillover0_noskill1_sep1_BN0_ineq0_red0_etaa0.79_lgd0.png}
	\end{minipage}
	\begin{minipage}[]{0.32\textwidth}
		\centering{\footnotesize{(i) Technology green}}
		%	\captionsetup{width=.45\linewidth}
		\includegraphics[width=1\textwidth]{../../codding_model/own_basedOnFried/optimalPol_elastS_DisuSci/figures/all_1705/Ag_CompEffOPT_NOT_NoTaus_spillover0_noskill1_sep1_BN0_ineq0_red0_etaa0.79_lgd0.png}
	\end{minipage}
	\begin{minipage}[]{0.32\textwidth}
		\centering{\footnotesize{(j) Technology fossil}}
		%	\captionsetup{width=.45\linewidth}
		\includegraphics[width=1\textwidth]{../../codding_model/own_basedOnFried/optimalPol_elastS_DisuSci/figures/all_1705/Af_CompEffOPT_NOT_NoTaus_spillover0_noskill1_sep1_BN0_ineq0_red0_etaa0.79_lgd0.png}
	\end{minipage}
	\begin{minipage}[]{0.32\textwidth}
		\centering{\footnotesize{(k) Technology neutral}}
		%	\captionsetup{width=.45\linewidth}
		\includegraphics[width=1\textwidth]{../../codding_model/own_basedOnFried/optimalPol_elastS_DisuSci/figures/all_1705/An_CompEffOPT_NOT_NoTaus_spillover0_noskill1_sep1_BN0_ineq0_red0_etaa0.79_lgd0.png}
	\end{minipage}
	\begin{minipage}[]{0.32\textwidth}
		\centering{\footnotesize{(l) labour green}}
		%	\captionsetup{width=.45\linewidth}
		\includegraphics[width=1\textwidth]{../../codding_model/own_basedOnFried/optimalPol_elastS_DisuSci/figures/all_1705/Lg_CompEffOPT_NOT_NoTaus_spillover0_noskill1_sep1_BN0_ineq0_red0_etaa0.79_lgd0.png}
	\end{minipage}
	\begin{minipage}[]{0.32\textwidth}
		\centering{\footnotesize{(m) labour fossil}}
		%	\captionsetup{width=.45\linewidth}
		\includegraphics[width=1\textwidth]{../../codding_model/own_basedOnFried/optimalPol_elastS_DisuSci/figures/all_1705/Lf_CompEffOPT_NOT_NoTaus_spillover0_noskill1_sep1_BN0_ineq0_red0_etaa0.79_lgd0.png}
	\end{minipage}
	\begin{minipage}[]{0.32\textwidth}
		\centering{\footnotesize{\ \\(l) labour neutral}}
		%	\captionsetup{width=.45\linewidth}
		\includegraphics[width=1\textwidth]{../../codding_model/own_basedOnFried/optimalPol_elastS_DisuSci/figures/all_1705/Ln_CompEffOPT_NOT_NoTaus_spillover0_noskill1_sep1_BN0_ineq0_red0_etaa0.79_lgd0.png}
	\end{minipage}
	\begin{minipage}[]{0.32\textwidth}
		\centering{\footnotesize{(g) Final output}}
		%	\captionsetup{width=.45\linewidth}
		\includegraphics[width=1\textwidth]{../../codding_model/own_basedOnFried/optimalPol_elastS_DisuSci/figures/all_1705/Y_CompEffOPT_NOT_NoTaus_spillover0_noskill1_sep1_BN0_ineq0_red0_etaa0.79_lgd0.png}
	\end{minipage}
	\begin{minipage}[]{0.32\textwidth}
		\centering{\footnotesize{(h) Energy output}}
		%	\captionsetup{width=.45\linewidth}
		\includegraphics[width=1\textwidth]{../../codding_model/own_basedOnFried/optimalPol_elastS_DisuSci/figures/all_1705/E_CompEffOPT_NOT_NoTaus_spillover0_noskill1_sep1_BN0_ineq0_red0_etaa0.79_lgd0.png}
	\end{minipage}
	\begin{minipage}[]{0.32\textwidth}
		\centering{\footnotesize{(i) Neutral output}}
		%	\captionsetup{width=.45\linewidth}
		\includegraphics[width=1\textwidth]{../../codding_model/own_basedOnFried/optimalPol_elastS_DisuSci/figures/all_1705/N_CompEffOPT_NOT_NoTaus_spillover0_sep1_BN0_ineq0_red0_etaa0.79_lgd0.png}
	\end{minipage}
\end{figure}


%---- no skill heterogeneity with target
\begin{figure}[h!!]
	\centering
	\caption{Comparison to efficient allocation without skill heterogeneity }\label{fig:Compno_eff_noskill}
		\begin{minipage}[]{0.32\textwidth}
		\centering{\footnotesize{(b) Income tax progressivity}}
		%	\captionsetup{width=.45\linewidth}
		\includegraphics[width=1\textwidth]{../../codding_model/own_basedOnFried/optimalPol_elastS_DisuSci/figures/all_1705/taul_CompEffOPT_T_NoTaus_spillover0_noskill1_sep1_BN0_ineq0_red0_etaa0.79_lgd0.png}
	\end{minipage}
	\begin{minipage}[]{0.32\textwidth}
	\centering{\footnotesize{(b) Fossil tax}}
	%	\captionsetup{width=.45\linewidth}
	\includegraphics[width=1\textwidth]{../../codding_model/own_basedOnFried/optimalPol_elastS_DisuSci/figures/all_1705/tauf_CompEffOPT_T_NoTaus_spillover0_noskill1_sep1_BN0_ineq0_red0_etaa0.79_lgd0.png}
\end{minipage}
\begin{minipage}[]{0.32\textwidth}
	\centering{\footnotesize{(b) fossil price}}
	%	\captionsetup{width=.45\linewidth}
	\includegraphics[width=1\textwidth]{../../codding_model/own_basedOnFried/optimalPol_elastS_DisuSci/figures/all_1705/pf_CompEffOPT_T_NoTaus_spillover0_noskill1_sep1_BN0_ineq0_red0_etaa0.79_lgd0.png}
\end{minipage}
	\begin{minipage}[]{0.32\textwidth}
	\centering{\footnotesize{(b) Utility labour}}
	%	\captionsetup{width=.45\linewidth}
	\includegraphics[width=1\textwidth]{../../codding_model/own_basedOnFried/optimalPol_elastS_DisuSci/figures/all_1705/Utillab_CompEffOPT_T_NoTaus_spillover0_noskill1_sep1_BN0_ineq0_red0_etaa0.79_lgd0.png}
\end{minipage}
\begin{minipage}[]{0.32\textwidth}
	\centering{\footnotesize{(b) Utility Consumption}}
	%	\captionsetup{width=.45\linewidth}
	\includegraphics[width=1\textwidth]{../../codding_model/own_basedOnFried/optimalPol_elastS_DisuSci/figures/all_1705/Utilcon_CompEffOPT_T_NoTaus_spillover0_noskill1_sep1_BN0_ineq0_red0_etaa0.79_lgd0.png}
\end{minipage}
\begin{minipage}[]{0.32\textwidth}
	\centering{\footnotesize{(b) Utility scientists}}
	%	\captionsetup{width=.45\linewidth}
	\includegraphics[width=1\textwidth]{../../codding_model/own_basedOnFried/optimalPol_elastS_DisuSci/figures/all_1705/UtilSci_CompEffOPT_T_NoTaus_spillover0_noskill1_sep1_BN0_ineq0_red0_etaa0.79_lgd0.png}
\end{minipage}
\begin{minipage}[]{0.32\textwidth}
\centering{\footnotesize{(b) green price}}
%	\captionsetup{width=.45\linewidth}
\includegraphics[width=1\textwidth]{../../codding_model/own_basedOnFried/optimalPol_elastS_DisuSci/figures/all_1705/pg_CompEffOPT_T_NoTaus_spillover0_noskill1_sep1_BN0_ineq0_red0_etaa0.79_lgd0.png}
\end{minipage}
\begin{minipage}[]{0.32\textwidth}
	\centering{\footnotesize{(b) fossil labour}}
	%	\captionsetup{width=.45\linewidth}
	\includegraphics[width=1\textwidth]{../../codding_model/own_basedOnFried/optimalPol_elastS_DisuSci/figures/all_1705/Lf_CompEffOPT_T_NoTaus_spillover0_noskill1_sep1_BN0_ineq0_red0_etaa0.79_lgd0.png}
\end{minipage}
\begin{minipage}[]{0.32\textwidth}
	\centering{\footnotesize{(b) green labour}}
	%	\captionsetup{width=.45\linewidth}
	\includegraphics[width=1\textwidth]{../../codding_model/own_basedOnFried/optimalPol_elastS_DisuSci/figures/all_1705/Lg_CompEffOPT_T_NoTaus_spillover0_noskill1_sep1_BN0_ineq0_red0_etaa0.79_lgd0.png}
\end{minipage}
\begin{minipage}[]{0.32\textwidth}
	\centering{\footnotesize{(b) wage rate}}
	%	\captionsetup{width=.45\linewidth}
	\includegraphics[width=1\textwidth]{../../codding_model/own_basedOnFried/optimalPol_elastS_DisuSci/figures/all_1705/wh_CompEffOPT_T_NoTaus_spillover0_noskill1_sep1_BN0_ineq0_red0_etaa0.79_lgd0.png}
\end{minipage}
	\begin{minipage}[]{0.32\textwidth}
		\centering{\footnotesize{(b) Hours worked}}
		%	\captionsetup{width=.45\linewidth}
		\includegraphics[width=1\textwidth]{../../codding_model/own_basedOnFried/optimalPol_elastS_DisuSci/figures/all_1705/hh_CompEffOPT_T_NoTaus_spillover0_noskill1_sep1_BN0_ineq0_red0_etaa0.79_lgd0.png}
	\end{minipage}
	\begin{minipage}[]{0.32\textwidth}
		\centering{\footnotesize{(d) Consumption}}
		%	\captionsetup{width=.45\linewidth}
		\includegraphics[width=1\textwidth]{../../codding_model/own_basedOnFried/optimalPol_elastS_DisuSci/figures/all_1705/C_CompEffOPT_T_NoTaus_spillover0_noskill1_sep1_BN0_ineq0_red0_etaa0.79_lgd0.png}
	\end{minipage}
	\begin{minipage}[]{0.32\textwidth}
		\centering{\footnotesize{\ \\(e)  Technology fossil}}
		%	\captionsetup{width=.45\linewidth}
		\includegraphics[width=1\textwidth]{../../codding_model/own_basedOnFried/optimalPol_elastS_DisuSci/figures/all_1705/Af_CompEffOPT_T_NoTaus_spillover0_noskill1_sep1_BN0_ineq0_red0_etaa0.79_lgd0.png}
	\end{minipage}
	\begin{minipage}[]{0.32\textwidth}
		\centering{\footnotesize{\ \\(f) Technology green}}
		%	\captionsetup{width=.45\linewidth}
		\includegraphics[width=1\textwidth]{../../codding_model/own_basedOnFried/optimalPol_elastS_DisuSci/figures/all_1705/Ag_CompEffOPT_T_NoTaus_spillover0_noskill1_sep1_BN0_ineq0_red0_etaa0.79_lgd0.png}
	\end{minipage}
	\begin{minipage}[]{0.32\textwidth}
		\centering{\footnotesize{\ \\(g) Technology neutral}}
		%	\captionsetup{width=.45\linewidth}
		\includegraphics[width=1\textwidth]{../../codding_model/own_basedOnFried/optimalPol_elastS_DisuSci/figures/all_1705/An_CompEffOPT_T_NoTaus_spillover0_noskill1_sep1_BN0_ineq0_red0_etaa0.79_lgd0.png}
	\end{minipage}
	\begin{minipage}[]{0.32\textwidth}
		\centering{\footnotesize{\ \\(h) Green scientists}}
		%	\captionsetup{width=.45\linewidth}
		\includegraphics[width=1\textwidth]{../../codding_model/own_basedOnFried/optimalPol_elastS_DisuSci/figures/all_1705/sg_CompEffOPT_T_NoTaus_spillover0_noskill1_sep1_BN0_ineq0_red0_etaa0.79_lgd0.png}
	\end{minipage}
	\begin{minipage}[]{0.32\textwidth}
		\centering{\footnotesize{\ \\(i) Fossil scientists}}
		%	\captionsetup{width=.45\linewidth}
		\includegraphics[width=1\textwidth]{../../codding_model/own_basedOnFried/optimalPol_elastS_DisuSci/figures/all_1705/sff_CompEffOPT_T_NoTaus_spillover0_noskill1_sep1_BN0_ineq0_red0_etaa0.79_lgd0.png}
	\end{minipage}
	%	\begin{minipage}[]{0.32\textwidth}
	%		\centering{\footnotesize{\ \\(h) Scientists fossil}}
	%		%	\captionsetup{width=.45\linewidth}
	%		\includegraphics[width=1\textwidth]{../../codding_model/own_basedOnFried/optimalPol_elastS_DisuSci/figures/all_1705/sff_CompEffOPT_T_NoTaus_spillover0_sep1_BN0_ineq0_red0_etaa0.79_lgd0.png}
	%	\end{minipage}
	%	\begin{minipage}[]{0.32\textwidth}
	%		\centering{\footnotesize{\ \\(i) Scientists neutral}}
	%		%	\captionsetup{width=.45\linewidth}
	%		\includegraphics[width=1\textwidth]{../../codding_model/own_basedOnFried/optimalPol_elastS_DisuSci/figures/all_1705/sn_CompEffOPT_T_NoTaus_spillover0_sep1_BN0_ineq0_red0_etaa0.79_lgd0.png}
	%	\end{minipage}
	\begin{minipage}[]{0.32\textwidth}
		\centering{\footnotesize{\ \\(j) labour green}}
		%	\captionsetup{width=.45\linewidth}
		\includegraphics[width=1\textwidth]{../../codding_model/own_basedOnFried/optimalPol_elastS_DisuSci/figures/all_1705/Lg_CompEffOPT_T_NoTaus_spillover0_noskill1_sep1_BN0_ineq0_red0_etaa0.79_lgd0.png}
	\end{minipage}
	\begin{minipage}[]{0.32\textwidth}
		\centering{\footnotesize{\ \\(k) labour fossil}}
		%	\captionsetup{width=.45\linewidth}
		\includegraphics[width=1\textwidth]{../../codding_model/own_basedOnFried/optimalPol_elastS_DisuSci/figures/all_1705/Lf_CompEffOPT_T_NoTaus_spillover0_noskill1_sep1_BN0_ineq0_red0_etaa0.79_lgd0.png}
	\end{minipage}
	\begin{minipage}[]{0.32\textwidth}
		\centering{\footnotesize{\ \\(l) labour neutral}}
		%	\captionsetup{width=.45\linewidth}
		\includegraphics[width=1\textwidth]{../../codding_model/own_basedOnFried/optimalPol_elastS_DisuSci/figures/all_1705/Ln_CompEffOPT_T_NoTaus_spillover0_noskill1_sep1_BN0_ineq0_red0_etaa0.79_lgd0.png}
	\end{minipage}
\end{figure}



\begin{figure}[h!!]
	\centering
	\caption{Comparison to efficient allocation without target but externality }\label{fig:Compno_eff_extern}
	\begin{minipage}[]{0.32\textwidth}
		\centering{\footnotesize{(b) Income tax progressivity}}
		%	\captionsetup{width=.45\linewidth}
		\includegraphics[width=1\textwidth]{../../codding_model/own_basedOnFried/optimalPol_elastS_DisuSci/figures/all_1705/Extern_CompEff_taul_spillover0_noskill0_sep1_BN0_ineq0_red0_etaa0.79_lgd1.png}
	\end{minipage}
	\begin{minipage}[]{0.32\textwidth}
		\centering{\footnotesize{(b) Fossil tax}}
		%	\captionsetup{width=.45\linewidth}
		\includegraphics[width=1\textwidth]{../../codding_model/own_basedOnFried/optimalPol_elastS_DisuSci/figures/all_1705/Extern_CompEff_tauf_spillover0_noskill0_sep1_BN0_ineq0_red0_etaa0.79_lgd0.png}
	\end{minipage}
	\begin{minipage}[]{0.32\textwidth}
		\centering{\footnotesize{(b) fossil price}}
		%	\captionsetup{width=.45\linewidth}
		\includegraphics[width=1\textwidth]{../../codding_model/own_basedOnFried/optimalPol_elastS_DisuSci/figures/all_1705/Extern_CompEff_pf_spillover0_noskill0_sep1_BN0_ineq0_red0_etaa0.79_lgd0.png}
	\end{minipage}
	\begin{minipage}[]{0.32\textwidth}
		\centering{\footnotesize{(b) Utility labour}}
		%	\captionsetup{width=.45\linewidth}
		\includegraphics[width=1\textwidth]{../../codding_model/own_basedOnFried/optimalPol_elastS_DisuSci/figures/all_1705/Extern_CompEff_Utillab_spillover0_noskill0_sep1_BN0_ineq0_red0_etaa0.79_lgd0.png}
	\end{minipage}
	\begin{minipage}[]{0.32\textwidth}
		\centering{\footnotesize{(b) Utility Consumption}}
		%	\captionsetup{width=.45\linewidth}
		\includegraphics[width=1\textwidth]{../../codding_model/own_basedOnFried/optimalPol_elastS_DisuSci/figures/all_1705/Extern_CompEff_UtilCon_spillover0_noskill0_sep1_BN0_ineq0_red0_etaa0.79_lgd0.png}
	\end{minipage}
\end{figure}
\begin{figure}[h!!]
	\centering
	\caption{Comparison to efficient allocation without target but externality }\label{fig:Compno_eff_extern22}
	\begin{minipage}[]{0.32\textwidth}
		\centering{\footnotesize{(b) Utility scientists}}
		%	\captionsetup{width=.45\linewidth}
		\includegraphics[width=1\textwidth]{../../codding_model/own_basedOnFried/optimalPol_elastS_DisuSci/figures/all_1705/Extern_CompEff_UtilSci_spillover0_noskill0_sep1_BN0_ineq0_red0_etaa0.79_lgd0.png}
	\end{minipage}
	\begin{minipage}[]{0.32\textwidth}
		\centering{\footnotesize{(b) green price}}
		%	\captionsetup{width=.45\linewidth}
		\includegraphics[width=1\textwidth]{../../codding_model/own_basedOnFried/optimalPol_elastS_DisuSci/figures/all_1705/Extern_CompEff_pg_spillover0_noskill0_sep1_BN0_ineq0_red0_etaa0.79_lgd0.png}
	\end{minipage}
	\begin{minipage}[]{0.32\textwidth}
		\centering{\footnotesize{(b) fossil labour}}
		%	\captionsetup{width=.45\linewidth}
		\includegraphics[width=1\textwidth]{../../codding_model/own_basedOnFried/optimalPol_elastS_DisuSci/figures/all_1705/Extern_CompEff_Lf_spillover0_noskill0_sep1_BN0_ineq0_red0_etaa0.79_lgd0.png}
	\end{minipage}
	\begin{minipage}[]{0.32\textwidth}
		\centering{\footnotesize{(b) green labour}}
		%	\captionsetup{width=.45\linewidth}
		\includegraphics[width=1\textwidth]{../../codding_model/own_basedOnFried/optimalPol_elastS_DisuSci/figures/all_1705/Extern_CompEff_Lg_spillover0_noskill0_sep1_BN0_ineq0_red0_etaa0.79_lgd0.png}
	\end{minipage}
\begin{minipage}[]{0.32\textwidth}
\centering{\footnotesize{\ \\(l) labour neutral}}
%	\captionsetup{width=.45\linewidth}
\includegraphics[width=1\textwidth]{../../codding_model/own_basedOnFried/optimalPol_elastS_DisuSci/figures/all_1705/Extern_CompEff_Ln_spillover0_noskill0_sep1_BN0_ineq0_red0_etaa0.79_lgd0.png}
\end{minipage}
	\begin{minipage}[]{0.32\textwidth}
		\centering{\footnotesize{(b) wage rate high skill}}
		%	\captionsetup{width=.45\linewidth}
		\includegraphics[width=1\textwidth]{../../codding_model/own_basedOnFried/optimalPol_elastS_DisuSci/figures/all_1705/Extern_CompEff_wh_spillover0_noskill0_sep1_BN0_ineq0_red0_etaa0.79_lgd0.png}
	\end{minipage}
\begin{minipage}[]{0.32\textwidth}
	\centering{\footnotesize{(b) wage rate low skill}}
	%	\captionsetup{width=.45\linewidth}
	\includegraphics[width=1\textwidth]{../../codding_model/own_basedOnFried/optimalPol_elastS_DisuSci/figures/all_1705/Extern_CompEff_wl_spillover0_noskill0_sep1_BN0_ineq0_red0_etaa0.79_lgd0.png}
\end{minipage}
\begin{minipage}[]{0.32\textwidth}
	\centering{\footnotesize{(b) Hours high skill}}
	%	\captionsetup{width=.45\linewidth}
	\includegraphics[width=1\textwidth]{../../codding_model/own_basedOnFried/optimalPol_elastS_DisuSci/figures/all_1705/Extern_CompEff_hh_spillover0_noskill0_sep1_BN0_ineq0_red0_etaa0.79_lgd0.png}
\end{minipage}
	\begin{minipage}[]{0.32\textwidth}
		\centering{\footnotesize{(b) Hours low skill}}
		%	\captionsetup{width=.45\linewidth}
		\includegraphics[width=1\textwidth]{../../codding_model/own_basedOnFried/optimalPol_elastS_DisuSci/figures/all_1705/Extern_CompEff_hl_spillover0_noskill0_sep1_BN0_ineq0_red0_etaa0.79_lgd0.png}
	\end{minipage}
	\begin{minipage}[]{0.32\textwidth}
		\centering{\footnotesize{(d) Consumption}}
		%	\captionsetup{width=.45\linewidth}
		\includegraphics[width=1\textwidth]{../../codding_model/own_basedOnFried/optimalPol_elastS_DisuSci/figures/all_1705/Extern_CompEff_C_spillover0_noskill0_sep1_BN0_ineq0_red0_etaa0.79_lgd0.png}
	\end{minipage}
\end{figure}

\begin{figure}[h!!]
	\centering
	\caption{Comparison to efficient allocation without target but externality 2}\label{fig:Compno_eff_extern2}
	\begin{minipage}[]{0.32\textwidth}
	\centering{\footnotesize{\ \\(e)  Technology fossil}}
	%	\captionsetup{width=.45\linewidth}
	\includegraphics[width=1\textwidth]{../../codding_model/own_basedOnFried/optimalPol_elastS_DisuSci/figures/all_1705/Extern_CompEff_Af_spillover0_noskill0_sep1_BN0_ineq0_red0_etaa0.79_lgd0.png}
\end{minipage}
\begin{minipage}[]{0.32\textwidth}
	\centering{\footnotesize{\ \\(f) Technology green}}
	%	\captionsetup{width=.45\linewidth}
	\includegraphics[width=1\textwidth]{../../codding_model/own_basedOnFried/optimalPol_elastS_DisuSci/figures/all_1705/Extern_CompEff_Ag_spillover0_noskill0_sep1_BN0_ineq0_red0_etaa0.79_lgd0.png}
\end{minipage}
\begin{minipage}[]{0.32\textwidth}
	\centering{\footnotesize{\ \\(g) Technology neutral}}
	%	\captionsetup{width=.45\linewidth}
	\includegraphics[width=1\textwidth]{../../codding_model/own_basedOnFried/optimalPol_elastS_DisuSci/figures/all_1705/Extern_CompEff_An_spillover0_noskill0_sep1_BN0_ineq0_red0_etaa0.79_lgd0.png}
\end{minipage}
\begin{minipage}[]{0.32\textwidth}
	\centering{\footnotesize{\ \\(h) Green scientists}}
	%	\captionsetup{width=.45\linewidth}
	\includegraphics[width=1\textwidth]{../../codding_model/own_basedOnFried/optimalPol_elastS_DisuSci/figures/all_1705/Extern_CompEff_sg_spillover0_noskill0_sep1_BN0_ineq0_red0_etaa0.79_lgd0.png}
\end{minipage}
\begin{minipage}[]{0.32\textwidth}
	\centering{\footnotesize{\ \\(i) Fossil scientists}}
	%	\captionsetup{width=.45\linewidth}
	\includegraphics[width=1\textwidth]{../../codding_model/own_basedOnFried/optimalPol_elastS_DisuSci/figures/all_1705/Extern_CompEff_sff_spillover0_noskill0_sep1_BN0_ineq0_red0_etaa0.79_lgd0.png}
\end{minipage}

\end{figure}

\begin{figure}[h!!]
	\centering
	\caption{Comparison to efficient allocation without target but externality 2}\label{fig:Compno_eff_Growth}
	\begin{minipage}[]{0.32\textwidth}
		\centering{\footnotesize{\ \\(e)  Technology fossil}}
		%	\captionsetup{width=.45\linewidth}
		\includegraphics[width=1\textwidth]{../../codding_model/own_basedOnFried/optimalPol_elastS_DisuSci/figures/all_1705/Extern_CompEff_Af_spillover0_noskill0_sep1_BN0_ineq0_red0_etaa0.79_lgd0.png}
	\end{minipage}
	\begin{minipage}[]{0.32\textwidth}
		\centering{\footnotesize{\ \\(f) Technology green}}
		%	\captionsetup{width=.45\linewidth}
		\includegraphics[width=1\textwidth]{../../codding_model/own_basedOnFried/optimalPol_elastS_DisuSci/figures/all_1705/Extern_CompEff_Ag_spillover0_noskill0_sep1_BN0_ineq0_red0_etaa0.79_lgd0.png}
	\end{minipage}
	\begin{minipage}[]{0.32\textwidth}
		\centering{\footnotesize{\ \\(g) Technology neutral}}
		%	\captionsetup{width=.45\linewidth}
		\includegraphics[width=1\textwidth]{../../codding_model/own_basedOnFried/optimalPol_elastS_DisuSci/figures/all_1705/Extern_CompEff_An_spillover0_noskill0_sep1_BN0_ineq0_red0_etaa0.79_lgd0.png}
	\end{minipage}
	\begin{minipage}[]{0.32\textwidth}
		\centering{\footnotesize{\ \\(h) Green scientists}}
		%	\captionsetup{width=.45\linewidth}
		\includegraphics[width=1\textwidth]{../../codding_model/own_basedOnFried/optimalPol_elastS_DisuSci/figures/all_1705/Extern_CompEff_sg_spillover0_noskill0_sep1_BN0_ineq0_red0_etaa0.79_lgd0.png}
	\end{minipage}
	\begin{minipage}[]{0.32\textwidth}
		\centering{\footnotesize{\ \\(i) Fossil scientists}}
		%	\captionsetup{width=.45\linewidth}
		\includegraphics[width=1\textwidth]{../../codding_model/own_basedOnFried/optimalPol_elastS_DisuSci/figures/all_1705/Extern_CompEff_sff_spillover0_noskill0_sep1_BN0_ineq0_red0_etaa0.79_lgd0.png}
	\end{minipage}
	
\end{figure}
%\clearpage
\appendix
\section{Natural Scientific Background}
\begin{comment}
\subsection{Why output (growth) reduction might be optimal}
It is a vibrant debate whether technological process will result in a production technology that is perfectly clean in that it does not exert any environmental externality. 
\begin{itemize}
	%\item \underline{Extensions to technology in \cite{Acemoglu2012TheChange} }
	%\begin{itemize}
	\item \underline{externality of ``clean'' sector} \citep[see also][]{Dasgupta2021, Brock2005ChapterEmpirics}
	\begin{itemize}
		\item[-] renewable/ non-fossil fuels \ar externalities in production process are present e.g. production of solar panels uses toxic inputs \citep{Yue2014DomesticAnalysis}; non-fossil fuel nitrogen generation (e.g., biomass burning to clear land) important ($\approx$ 50\%) \citep{Song2021ImportantEmissions}; low but chronical levels of nitrogen cause species extinctions \citep{Clark2008LossGrasslands}
		\item[-] waste (after use) \ar depends on recycling technology %\ar recycling system for solar panels not profitable enough today
		%	\item[-] substitutability of nature in production (input sources eg. fossil vs. non-fossil fuels)
		%\end{itemize}
		%\item Irreversibilities already before thresholds are hit (e.g. species extinction)
		
	\end{itemize}
	%\item greenhouse gases: Carbon dioxide $CO_2$ (vast majority), Nitrous oxide $N_2O$, methane $CH_4$
	%\item stock of nature globally determined
	\item \underline{parallel positive trend in demand} (population growth, rebound effect) that outperforms increase in clean technology growth \small{(no long-run issue if perfectly clean technology exists)}
	\item \normalsize{\underline{objective function}:} \cite{Arrow2004AreMuch}(Journal of Economic Perspectives) \ar using a sustainability measure they provide evidence that consumption is too high (= not leaving enough natural resources for future generations)
	\item \underline{risk, ambiguity}
	\item if have to meet climate target in short run, might need to lower production to do so; or it might be better in terms of inequality?
\end{itemize}

content...
\end{comment}
\subsection{Greenhouse-gas emissions and the Paris Agreement}

Two alternatives exist to specifiy the relation between the environment and production: (i) a broad approach considering natures use as a sink and as a resource, and all relevant pollutants. 
In order to determine \textit{relevant}, I refer to the planetary boundaries discussed in \cite{Rockstrom2009AHumanity}. (ii) a more specific approach that focuses on greenhouse gas emissions in particular which allows to draw on emission goals specified by country. Paris agreement goal: `'\textit{Climate neutral world by the mid-century}'' (source \url{https://unfccc.int/process-and-meetings/the-paris-agreement/the-paris-agreement}). In 2020 countries had to submit plansfor a \textit{long-term low ghg emissions} (LT-LEDS) where long term means mid-century (I assume). In the EU member states have submitted \textit{integrated national-energy and climate plans} (NECPS) (source \url{https://ec.europa.eu/info/energy-climate-change-environment/implementation-eu-countries/energy-and-climate-governance-and-reporting/national-long-term-strategies_en}). According to this source, emissions occur in the following fields: 
\textit{emission reductions and enhancements of removals in individual sectors, including \textbf{electricity, industry, transport, the heating and cooling and buildings sector (residential and tertiary), agriculture, waste and land use, land-use change and forestry (LULUCF)}}; the website also contains documents on country specific plans and actions

\subsection{Modelling choice: external emission target}\label{app:emission_climate_targets}
There is a multitude of uncertainties shaping the relation of production, on the one hand, and nature and climate warming, on the other hand. These uncertainties relate to (i) the technological possibilities to reduce emissions in the future and (ii) the relation of emissions and the climate. 

In the Paris Agreement clear political goals have been formulated in 2015. Under this treaty, states have agreed on a legally binding maximum increase in temperature to well below 2°C, preferably 1.5° over pre-industrial levels, and the global community seeks to be climate-neutral in 2050  (compare:\\ \url{https://unfccc.int/process-and-meetings/the-paris-agreement/the-paris-agreement}). 

\paragraph{Uncertainty 1): Emissions $\rightarrow$ temperature}
Carbon dioxide has been the focus of the literature integrating climate change and economic models \citep[such as,][]{Golosov2014OptimalEquilibrium,Barrage2019OptimalPolicy}. 
The (geo-)physical mechanisms which determine the interrelation between carbon emissions and temperature changes are highly uncertain and complex. For example, (1) there is no good understanding of the relation of CO2 and the climate as the temperature rises to certain limits, (2)  feedback of the Earth system, such as permafrost thawing, has to be taken into account, as well as (3) interactions of carbon with non-CO2 emissions, (\citep[][p.96, 2nd paragraph]{Rogelj2018MitigationDevelopment.}).  Uncertainty also surrounds the regeneration rate of the environment \citep{Acemoglu2012TheChange} and irreversibilities (might CITE Hassler handbook chapter here). In a quantitative study on optimal environmental policies, hence, a lot of assumptions and simplifications have to be made. 

In chapter 2 of the
\textit{IPCC Special Report} \citep{Rogelj2018MitigationDevelopment.}, scientists quantify emission pathways to meet the 1.5°C goal of the Paris Agreement by carefully taking uncertainties and the complex geophysical processes into account. I use these limits on emissions as constraints to the government's objective function. This approach is clearly policy relevant, while at the same time reduces the need to make (geophysical) assumptions. Furthermore, it allows me to take other important non-CO2 emissions into account, too. (\textit{Look at the discussion of integrated assessment models in \cite{Hassler2016EnvironmentalMacroeconomics} for the advantages of integrating a simplified carbon cycle into macro models (\ar dynamics) })
%\tr{\ar In a nutshell, I take from these reports the emission reduction pathways. I do have to make assumptions on the possibilities of technological innovations. No carbon cycle needed but less assumptions have to be made.}

\paragraph{Uncertainty 2: Technological progress $\rightarrow$ emissions}
An important modelling uncertainty remains: what degree of emission reduction can be achieved by technological progress in the specified time frame? I use different specifications of technological possibilities: (i) a scenario where technological progress is sufficient to reduce emissions to zero until 2050 at current consumption levels per capita, (ii) and one where innovation steps are insufficient.
Also look at different modelling approaches to technological change: (i) sector-specific innovations, (ii) porgress on the substitutability of clean and dirty input goods.

\paragraph{Uncertainty 2: regeneration rate of nature}

\paragraph{Uncertainty 2: Substitutability of natural capital in welfare}
Related to technological possibilities is the substitutability of clean and dirty production. 

A key reference is \cite{Cohen2019AnnualSubstitutable}. They argue for a limit of substitutability of \textit{natural capital} (defined as the stock of renewable and nonrenewable resources including minerals, soils, plants, animals, water, air, and energy). The role of natural capital for welfare: resources for production, absorption of waste, basic-life support, direct conrtibution to human welfare \ar No perfect substitutability: Humans cannot live without natural resources. 

BUT: my model so far is about greenhouse-gas emissions only.

Definition \textit{sustainability} common in economics: maintaining a non-decreasing level of welfare across generations. (Sonja: this shouls include provision of natural services: human-friendly climate; how to know the utility function of future generations? Eg if habits are relevant, than they might be happy with less consumption). As regards renewables, it must be ensured that renewability is maintained. It is about the substitutability of natural inputs to the welfare function. \textit{Sonja: (1) How should we know how substitutable biodiversity is for humanity if it is about pleasure derived from living in a diverse world. (2) If it is about other factors of biodiversity, such as, maintaining more basal services for human live, such as safety, there might be less disagreement; more certainty on the importance also for future generations; a utility approach but based on objectively defined values: basic needs.}

\cite{Cohen2019AnnualSubstitutable} make the following distinction: \textit{within-input substitution}: =resources used in welfare production: within-input substitution allows to reduce environmental \textbf{impacts} if the same type of input can be obtained from different sources (e.g.: energy: from emission-low sources instead of emission high sources \tr{\ar This is substitution of dirty with clean goods}); \textit{between-input substitution}: = from energy to manufactured capital;\tr{ includes more efficient energy use (\textbf{with the same amount of energy, more can be produced.})} , recycling, reforestation.

\textbf{Sonja:} within-input substitution: \ar same input but different source (progress= less emissions for energy); between-input substitution: 
\ar sector specific production functions are cobb-douglas \ar unit elasticity of input goods.


\textit{Brundtland Comission} definition of sustainable development: \textit{development that meets the \textbf{needs} of the present without compromising the ability of future generations to meet their own needs}.
If various forms of capital are substitutable (i.e. can use manufactured capital, human capital to replace nature) then production only depends on the total capital stock. Then, economic growth is said to be \textit{weakly sustainable} when the total capital stock is non-decreasing; i..e aggregate savings rate is above depreciation rate on all forms of capital. 

\textit{Strong sustainability view}:  a minimum of natural capital must be sustained as it provides non-substitutable inputs to utility. Then, long-run growth must be able to maintain a natural capital stock.  
\ar Define $Y$ broadly to incorporate other necessary consumption goods: stable climate, breathable air, food, water; but these are partially not traded in markets, rather public goods. Then, better to model as public goods; \textbf{Or as another input in final good production}.

\begin{align*}
Y= \left[\left(\underbrace{\left(Y_c^{\frac{\varepsilon_c-1}{\varepsilon_c}}+Y_d^{\frac{\varepsilon_c-1}{\varepsilon_c}}\right)^{\frac{\varepsilon_c}{\varepsilon_c-1}}}_{\text{consumption  good}}\right)^{\frac{\varepsilon_o-1}{\varepsilon_o}}+(\underbrace{N-\bar{N}}_{\text{natural capital}})^{\frac{\varepsilon_o-1}{\varepsilon_o}}\right]^{\frac{\varepsilon_o}{\varepsilon_o-1}}
\end{align*} 
where $\varepsilon_o$  governs the substitutability of natural capital and consumption in the final good; it translates into the substitutability discussed in \cite{Cohen2019AnnualSubstitutable}. For the basic needs level ins terms of natural capital holds: $\bar{N}>0$ so that nature is a necessary good. The variable $N$ determines the consumption of natural capital, such as breathing air, stable climate.
Model technological growth on substitutability between natural capital (including nature as a waste), and other production inputs.  

 
%%%%%%%%%%%%%%%%%%%%%%%%%%%%%%%%%%%%%%%
\subsection{Greenhouse gas emissions Data}
To calibrate the relation of economic production and emissions, I use data from the EPA \url{https://www.epa.gov/newsreleases/latest-inventory-us-greenhouse-gas-emissions-and-sinks-shows-long-term-reductions-0}.
In 2019, the US greenhouse gas emissions amounted to 6,558 million metric tons of carbon dioxide equivalents; that is, 6.558Gt. 

Global greenhouse gas emissions amounted to 34.2 Gt in C02 equivalents in 2019. There was a decline in 2020 (presumably due to the pandemic, and a rebound in 2021 by 5\%)Found here: \url{https://www.iea.org/reports/greenhouse-gas-emissions-from-energy-overview/global-ghg-emissions}; the Global Energy review of the iea is to  be found here: \url{https://www.iea.org/reports/global-energy-review-2021/co2-emissions}.

Natural sinks are, for example, forests, vegetation, soils. 
The epa report (\url{https://www.epa.gov/ghgemissions/inventory-us-greenhouse-gas-emissions-and-sinks-1990-2019}) includes information on sinks. 

Net emissions after taking sinks into account are estimated by the epa to 5,769.1 in 
\paragraph{Translation metric ton to gigatonne}
1.000.000.000 metric tons are 1 gigatonne 

\paragraph{Demand and Production approach}
The OECD, \url{https://www.oecd.org/sti/ind/carbondioxideemissionsembodiedininternationaltrade.htm} differentaites between a demand and a production-side approach to determine emissions on country level! I am now using the production approach. 

For now, I assume that the global emission target is given in gros emissions. I match the US gros emission target so that contribution in 2019 equals contribution to the global gros reduction. 

\subsection{Radiative Forcing}
\ar  leads to uncertainty in temperature response to emissions
(from \url{https://climate.mit.edu/explainers/radiative-forcing})

Radiative forcing is what happens when the amount of energy that enters the Earth’s atmosphere is different from the amount of energy that leaves it. Energy travels in the form of radiation: solar radiation entering the atmosphere from the sun, and infrared radiation exiting as heat. \textbf{If more radiation is entering Earth than leaving—as is happening today—then the atmosphere will warm up.} This is called radiative forcing because the difference in energy can force changes in the Earth’s climate.
Heat in, Heat out

Sunlight is always shining on half of the Earth’s surface. Some of this sunlight (about 30 percent) is reflected back to space. The rest is absorbed by the planet. But as with any warm object sitting in cold surroundings—and space is a very cold place—some energy from Earth is always radiating back out into space as heat.

Radiative forcing measures how much energy is coming in from the sun, compared to how much is leaving. The analysis needed to pin down this exact number is very complicated. \textbf{Many factors, including clouds, polar ice, and the physical properties of gases in the atmosphere, have an effect on this balancing act}, and each has its own level of uncertainty and its own difficulties in being precisely measured. However, we do know that today, more heat is coming in than going out.


Before the industrial era, radiative forcing was in very close balance, and the Earth’s average temperature was more or less stable. For this reason, researchers calculate radiative forcing based on a “baseline” year sometime before the beginning of world industrialization. For example, the Intergovernmental Panel on Climate Change uses 1750 as a baseline year.

Compared to this baseline, radiative forcing can directly measure the ways recent human activities have changed the planet’s climate. \textbf{The biggest change has been the greenhouse gases we have added to the atmosphere, which keep heat from escaping the Earth.} But there have been other changes too. For example, by \textbf{cutting down forests, we have exposed more of the Earth’s surface to sunlight. If that surface is darker than the forest cover, the Earth absorbs more solar radiation;} where it’s lighter, like in the arctic, more sunlight is reflected back into space.

Humans are also adding small particles called\textbf{ aerosols} to the air, from smokestacks, airplanes, and the tailpipes of cars. Aerosols make radiative forcing especially hard to measure, because their effects are highly complex and can work both ways. For example, bright aerosols (like sulfates from coal-burning) can help cool the atmosphere by reflecting light, while dark aerosols (like black carbon from diesel exhausts) absorb heat and lead to warming.

Finally, measures of radiative forcing also include any natural effects that have changed since the baseline year, such as changes in the sun’s output (which has caused a little more warming) and aerosols released into the atmosphere by volcanoes (which cause temporary cooling).

\subsection{Methane emissions: second important ghg}
(source \url{https://climate.mit.edu/ask-mit/why-do-we-compare-methane-carbon-dioxide-over-100-year-timeframe-are-we-underrating})
Methane is a colorless, odorless gas that’s produced both by nature (such as in wetlands when plants decompose underwater) and in industry (for example, natural gas is mostly made of methane). 
Methane is a quick fading (on average it tstays 10 years in the atmoshpere) ghg. But it is far more damaging than CO2, which sticks longer in the atmosphere; for centuries. 
Methane traps 100times more heat than CO2; Co2 closes the gap over time when  methane has broken down. Old standard: look at warming effect of methane over 100 years; but that is too long a horizon as climate change advances. 

Sources of methane: natural gas production, decompostation of plants under water. 
Methane. CH4

\textbf{(from NASA \url{https://svs.gsfc.nasa.gov/4799})}

Methane is a powerful greenhouse gas that traps heat 28 times more effectively than carbon dioxide over a 100-year timescale. Concentrations of methane have increased by more than 150\% since industrial activities and intensive agriculture began. After carbon dioxide, methane is responsible for about 23\% of climate change in the twentieth century. Methane is produced under conditions where little to no oxygen is available. About 30\% of methane emissions are produced by wetlands, including ponds, lakes and rivers. Another 20\% is produced by agriculture, due to a combination of livestock, waste management and rice cultivation. Activities related to oil, gas, and coal extraction release an additional 30\%. The remainder of methane emissions come from minor sources such as wildfire, biomass burning, permafrost, termites, dams, and the ocean. Scientists around the world are working to better understand the budget of methane with the ultimate goals of reducing greenhouse gas emissions and improving prediction of environmental change. For additional information, see the Global Methane Budget.
\subsection{Natural gas}
(source \url{https://www.eia.gov/energyexplained/natural-gas/})
fossil energy source! remains of plants and animals

natural gas can be produced from shale and other types of sedimentary rock formations by forcing water, chemicals, and sand down a well under high pressure (fracking); tis breaks up the formation, and releases the natural gas from the rock. 
\subsection{Energy sources}
(source: eia, \url{https://www.eia.gov/energyexplained/what-is-energy/sources-of-energy.php})

\subsection{Bioenergy with Carbon Capture and Storage (BECCS)}
source. \url{https://en.wikipedia.org/wiki/Bioenergy_with_carbon_capture_and_storage}

 is the process of extracting bioenergy from biomass and capturing and storing the carbon, thereby removing it from the atmosphere. Energy is extracted in useful forms (electricity, heat, biofuels, etc.) as the biomass is utilized through combustion, fermentation, pyrolysis or other conversion methods.
Wide deployment of BECCS is constrained by cost and availability of biomass.

\subsection{Biosequestration} capture and storage of the atmospheric co2 by continual or enhanced biological processes.; reforestation

\subsection{IPCC report 2018 \citep{Rogelj2018MitigationDevelopment.}}
\begin{itemize}
\item literature focused on demand side: 
\begin{itemize}
	\item \cite{Arrow2004AreMuch}: raise the question if consumption is too high
	\item 
	\cite{Bertram2018TargetedScenarios}:   change demand as a parameter in  model; motivation: taking global inequality into account alternative meausres (mitigation policies) to carbon taxes  become optimal. These include lifestyle changes \textbf{in addition to sector-specific carbon taxes!} (25\% lower energy demand and -20\% lower demand for agricultural products )
	\item \cite{Grubler2018ATechnologies} argue for a low increase in demand (projections)
	\item \cite{Liu2018SocioeconomicC}: lifestyle changes become more and more important under more stringent 1.5°C goal 
	\item \cite{VanVuuren2018AlternativeTechnologies}: analyse lifestyle changes motivated by uncertainty and risks surrounding CDRs (carbon dioxide removal) and their competition for land, yet many mitigation pathways to meet Paris Agreement rely on these technologies.
	
	\textit{This paper  explores  a  set  of  what-if  scenarios  that  explore  these  alter-native  assumptions,  and  analyses  the  extent  to  which  they  reduce  the  need  for  CDR.  The  evaluated  measures  (Table  1)  have  been  mentioned in scientific literature and could possibly limit CDR use. } (p. 2) \ar hence: assuming a lower demand what does this imply?
	
	\textit{The  lifestyle  change  scenario  (LiStCh)  assumes  a  radi-cal  value  shift  towards  more  environmentally  friendly  behaviour,  including a healthy, low-meat diet, changes in transport habits and a reduction of heating and cooling levels at homes. Such a shift could be motivated by both environmental and health con} (p.2)
	
	Their models also take other forcers into account 
	
	\textit{ Given  the  possible  disadvantages  of  BECCS,  it  is  important  to  seriously  discuss  and  appraise such alternative pathways. This could focus on issues such as feasibility, social acceptance, associated costs and benefits, requir-ing input from other scientific disciplines to complement the model-based  scenarios. } p. vorletzte
	
	How they introduce a reduction in demand: 
	\begin{itemize}
\item reduced meat consumption; higher consumption of pulses and oilcrops
\item reduction in food waste
(by households and during production process)
\item transport changes: (1) reduced volume of transport, (2) reduction of energy intense modes of transport, \ar (3)  less private vehicle use and increased car sharing; less motorised options; also implies \ar (4) lower air travel demand
\item less residential energy use
\item reduced appliance ownership (a maximum of two per household) and use (less standby, smarter use)!
\item lower demand for plastic and chemicals 
	\end{itemize}
\end{itemize}
\end{itemize}

\section{Labour}
\subsection{Elasticity of labour}
\cite{Bick2018HowImplications}
\begin{itemize}
	\item \ar informative on the behaviour of individuals in the cross-section
	\item from the abstract: \textit{Within
	countries, hours worked per worker are also decreasing in the individual wage for most countries, though in the richest countries,
	hours worked are flat or increasing in the wage.}
\item there is heterogeneity in the wage elasticity of labour across countries! \ar could test theory in data? \ar those countries identified by \cite{Bick2018HowImplications} as having a negative relation of wage and hours worked \ar different effect of fiscal policy on green labour supply 
\item will imply heterogeneity in policy recommendation!:
\begin{itemize}
	\item poorer countries: negative relationship of individual wage and hours worked \ar higher income tax \ar lower wage \ar work more hours; 
	\ar income effect dominates! 
	\item richer countries:  positive relationship; higher income tax \ar lower wage \ar lower hours worked!
	\ar substitution effect dominates!!
\end{itemize}
\item over time hours worked per adult in the US have been falling. 
\item similar patterns when looking at the cross-section of countries
\item extensive margin: employment shares are falling with GDP for poor to medium countries, modest increase when country is rich
\end{itemize}
\cite{Boppart2019LaborPerspectiveb}
\begin{itemize}
	\item a long run perspective, but in the end I am also looking at a long run model!
	\item only look at the intensive margin: hours worked per worker! 
	\item \textbf{they argue for a higher income effect in the long run} BUT ON AGGREGATE 
	\item They look at a time dimension, this is what I am looking at too.
\end{itemize}
\ar GOAL: on the one hand match the cross-sectional wage elasticity, and on the other hand match the elasticity over time
\cite{LansBovenberg1994EnvironmentalTaxation}
\begin{itemize}
	\item they assume an upward sloping labour-supply curve \ar as the wage rate rises, households work more \ar as the wage rate falls, households work less
	\item an increase in the pollution tax reduces employment if the \textbf{uncompensated wage elasticity of labor supply} is positive \ar i.e. the substitution effect exceeds the income effect
\end{itemize}

\subsection{Skill and environment}
\tr{\textbf{Question} How is skill accumulated here? Retraining to higher skill level or long run decision based on education?\ar pre-job entry; How calibrated?\ar not a quantitative model; Inequality?\ar choice variable skill differences due to OLG structure...but only live for one period... not sure where both skill levels result from when households are all the same...}
\begin{itemize}
	\item \cite{Vona2018EnvironmentalExploration}
	\begin{itemize}
		\item differentiate also occupations\ar engineering skills and managering skills; this  paper less on skill but more on jobs?
		\item ``\textit{we identify two core sets of green skills for which green jobs differ from non-green jobs: engineering skills, and managerial skills }'' (p.2) \ar these skills (=jobs) are especially used in green occupations; \ar jobs which are used heavily in the green sector (p.3)
		\item environmental regulatory policies increase demand for some green skills (engineering, managing); no impact on employment in general
		\item ``\textit{The adjustment costs from job losses can be exacerbated when the skill profile of expanding jobs does not match the skill profile of contracting jobs}'' (p.3)
		\item ``\textit{while the skill gap between green and brown jobs within the same occupational group is generally small }'' (p.4) \ar model as neutral jobs, ``\textit{... exceptions emerge: largest skill gaps occur in construction and extraction occupations [...] important for climate change}''
		\item transitioning to greener production requires a transition in skills
	\end{itemize}
	\item \cite{Borissov2019CarbonDevelopment}
	\begin{itemize}
		\item skills are crucial a determinant of green growth as the labour required for green production is special: higher skills/ higher human capital accumulation
		\item on this they cite policy recommendation papers and \cite{Vona2018EnvironmentalExploration} which they cite as "green skills are closely related to the design, production, management and monitoring of technology and conclude that education emerges as a critical ingredient in the policy mix to promote sustainable economic growth"
		\item their focus seems not to lie on inequality!
		\item their idea: requiring non-developed countries to reduce emissions \ar higher carbon taxes \ar increase in human capital investment since the green sector uses high-skilled labour \ar economic growth! (\textit{how measured?}) \ar a win-win situation
		\item North-South knowledge spillover: if the carbon tax leads to knowledge growth in the green sector this might spread to the South even if the South itself does not levy a pollution tax
		\item human capital accumulation is the driver of growth in this model (not innovations! no directed technical change)
		\item no a-priori inequality in their model! Skill is a choice variable! 
		\item "clean sectors  tend to be skill intensive"
		\item positive intergenerational spill over in skills! (longer-run effects)
		\item positive effect of human capital accumulation on TFP see Lucas 1988, Glomm and Ravikumar 1992
		\item \textbf{incentives to human capital accumulation through carbon taxes!}
		\item \textbf{Model}
		\begin{itemize}
			\item model of successive generations: OLG
			\item no preference heterogeneity, acquiring education costs labour income
			\item skilled, unskilled is a discrete choice
			\item spill over of skills within country
			\item production only requires labour skilled and unskilled,
			\item sector specific goods are perfect substitutes! they produce the same stuff
			\item general production function, but also Cobb-Douglas, then MPhigh skill relative to MP lowskill in clean sector is higher in the clean than in the dirty sector if income share of high skilled in clean is higher than in dirty. \ar calculate relative marginal product of skilled labour in clean and dirty sector
			\item section 5: endogenous growth version: higher share of high-skilled \ar higher growth\ar carbon tax implies higher growth!
			\item carbon tax leads to growth! in their model due to intensified skill accumulation
		\end{itemize}
	\end{itemize}
	
	\item \cite{Consoli2016DoCapital}
	\begin{itemize}
		\item findings: labour force characteristics of green and non-green jobs
		\begin{itemize}
			\item green jobs: high-level cognitive and interpersonal, higher level of formal education, more work experience and on-the-job training compared to non-green; 
			\item use O*NET (Occupational Information Network) comprising 905 occupations
			\item in new occupations which emerge due to a higher demand of green skills on-the-job trainging is a distinctive feature but not in already existing occupations
			\item review policy effects on employment in literature; comment that distinction between job characteristics in these papers are missing
			\item green occupations: (estimates are within SOC3 digit occupations, that is, they are conditional differences in expectations; not unconditional, in the paper I would want to take macro occupational differences into account too, its not only about being green but also that these green jobs are within some specific group: driven by heterogeneity in the average skill content in the macro group ); and green occupations are more within high skilled occupational groups, on the other hand I also only need to focus on occupations where a distinction between green and non-green can be made, and another sector that is neutral (quantitative analysis)\ar the conditional estimates fit well
			\item findings:
			\begin{itemize}
				\item 
				significantly more non-routine tasks and significantly less routine tasks than in the non-green counterparts, p.1052
				\item 19 percent more years of eductaion ($\approx$ 13 weeks), 43\% more years of experience ( $\approx$ 10 months at the mean), 41\% more years of trainnig,($\approx$ 15 weeks); p.1053
				\item green enhanced jobs are more exposed to all measures of technology (p.1053) 
			\end{itemize}
		\end{itemize}
	\end{itemize}
\end{itemize}

\section{Model}
\subsection{Model as in \cite{Fried2018ClimateAnalysis}}
\begin{align*}
\text{Household}\\
& C_t=w_{lft}L_{lft}+w_{lgt}L_{lgt}+w_{lnt}L_{lnt}+w_{sft}S_{sft}+w_{sgt}S_{sgt}+w_{snt}S_{snt}+\\ &\int_{0}^{1}\pi_{fit}+\pi_{git}+\pi_{nit}di+T_t\\
\text{Final good producer optimality}&\\
\text{Optimality}\ \vspace{4mm}& \delta_yY_t^\frac{1}{\varepsilon_y}E_t^{-\frac{1}{\varepsilon_e}}=p_{Et}\\
& (1-\delta_y)Y_t^\frac{1}{\varepsilon_y}N_t^{-\frac{1}{\varepsilon_e}}=p_{Nt}\\
&
p_{Et}E_t^\frac{1}{\varepsilon_e}\tilde{F}_t^{-\frac{1}{\varepsilon_e}}=p_{\tilde{F}t}\\
& p_{Et}E_t^\frac{1}{\varepsilon_e}G_t^{-\frac{1}{\varepsilon_e}}=p_{Gt}\\
& p_{\tilde{F}t}\tilde{F}_t^\frac{1}{\varepsilon_f}\delta_fF_t^{-\frac{1}{\varepsilon_f}}=p_{Ft}+\tau_{f}\\
& p_{\tilde{F}t}\tilde{F}_t^\frac{1}{\varepsilon_f}(1-\delta_f)O_t^{-\frac{1}{\varepsilon_f}}=p_{Ot}+\tau_{o}\\
\text{Definitions prices}\ \vspace{4mm}&
p_{yt}= \left[\delta_y^{\varepsilon_y}p_{Et}^{1-\varepsilon_y}+(1-\delta_y)^{\varepsilon_y}p_{Nt}^{1-\varepsilon_y}\right]^\frac{1}{1-\varepsilon_y}\\
& p_{Et}= \left[p_{\tilde{F}t}^{1-\varepsilon_e}+p_{Gt}^{1-\varepsilon_y}\right]^\frac{1}{1-\varepsilon_e}\\
& p_{\tilde{F}t}= \left[\delta_f^{\varepsilon_f}(p_{Ft}+\tau_{f})^{1-\varepsilon_f}+(1-\delta_f)^{\varepsilon_f}(p_{nt}+\tau_{o})^{1-\varepsilon_f}\right]^\frac{1}{1-\varepsilon_f}\\
\text{Production}\ \vspace{4mm}& 
Y_t=\left(\delta_yE_t^\frac{\varepsilon_y-1}{\varepsilon_y}+(1-\delta_y)N_t^\frac{\varepsilon_y-1}{\varepsilon_y}\right)^\frac{\varepsilon_y}{\varepsilon_y-1}\\
&E_t=\left(\tilde{F}_t^\frac{\varepsilon_e-1}{\varepsilon_e}+G_t^\frac{\varepsilon_e-1}{\varepsilon_e}\right)^\frac{\varepsilon_e}{\varepsilon_e-1}\\
&\tilde{F}_t=\left(\delta_fF_t^\frac{\varepsilon_f-1}{\varepsilon_f}+(1-\delta_f)O_t^\frac{\varepsilon_f-1}{\varepsilon_f}\right)^\frac{\varepsilon_f}{\varepsilon_f-1}\\
\text{Intermediate good producers}\\
\text{Production}\ \vspace{4mm}& F_t= (\alpha_f^2p_{Ft})^\frac{ \alpha_f}{1-\alpha_f}A_{ft}L_{ft}\\
&N_t= (\alpha_n^2p_{Nt})^\frac{ \alpha_n}{1-\alpha_n}A_{nt}L_{nt}\\
&G_t= (\alpha_g^2p_{Gt})^\frac{ \alpha_g}{1-\alpha_g}A_{gt}L_{gt}\\
\text{ Definition agg. Technology}\ \vspace{4mm}&
A_t= \frac{\rho_gA_{gt}+\rho_nA_{nt}+\rho_fA_{ft}}{\rho_n+\rho_g+\rho_f}
\end{align*}

\begin{align*}
\text{Labour demand}\\
& w_{lft}=p_{Ft}(1-\alpha_f)(\alpha_f^2p_{Ft})^\frac{\alpha_f}{1-\alpha_f}A_{ft}\\
& w_{lnt}=p_{Nt}(1-\alpha_n)(\alpha_n^2p_{Nt})^\frac{\alpha_n}{1-\alpha_n}A_{nt}\\
& w_{lgt}=p_{Gt}(1-\alpha_g)(\alpha_g^2p_{Gt})^\frac{\alpha_g}{1-\alpha_g}A_{gt}\\
\text{Mashine demand}\ \vspace{4mm}\\
&x_{fit}= \left(\alpha_f^2 p_{Ft}\right)^\frac{1}{1-\alpha_f}L_{ft}A_{fit}\\
&x_{nit}= \left(\alpha_n^2 p_{Nt}\right)^\frac{1}{1-\alpha_n}L_{nt}A_{nit}\\
&x_{git}= \left(\alpha_g^2 p_{Gt}\right)^\frac{1}{1-\alpha_g}L_{gt}A_{git}\\
\text{Mashine producers}\\
\text{Price setting}\ \vspace{4mm}&p_{fit}^x=\frac{1}{\alpha_f}\\
&p_{nit}^x=\frac{1}{\alpha_n}\\
&p_{git}^x=\frac{1}{\alpha_g}\\ 
\text{Demand Scientists}\ \vspace{4mm}&
w_{sft}=\frac{\eta \gamma \alpha_f A_{ft-1}^{1-\phi}A_{t-1}^{\phi}\left(\frac{S_{ft}}{\rho_f}\right)^{\eta}p_{Ft}F_t}{\frac{1}{1-\alpha_f}S_{ft}A_{ft}}\\
&
w_{snt}=\frac{\eta \gamma \alpha_n A_{nt-1}^{1-\phi}A_{t-1}^{\phi}\left(\frac{S_{nt}}{\rho_n}\right)^{\eta}p_{Nt}N_t}{\frac{1}{1-\alpha_n}S_{nt}A_{nt}}\\
&
w_{sgt}=\frac{\eta \gamma \alpha_g A_{gt-1}^{1-\phi}A_{t-1}^{\phi}\left(\frac{S_{gt}}{\rho_g}\right)^{\eta}p_{Gt}G_t}{\frac{1}{1-\alpha_g}S_{gt}A_{gt}}\\
\text{Innovation}\ \vspace{4mm}&
A_{fit}=A_{ft-1}\left(1+\gamma\left(\frac{S_{fit}}{\rho_f}\right)^{\eta}\left(\frac{A_{t-1}}{A_{ft-1}}\right)^{\phi}\right) \\
&
A_{nit}=A_{nt-1}\left(1+\gamma\left(\frac{S_{nit}}{\rho_n}\right)^{\eta}\left(\frac{A_{t-1}}{A_{nt-1}}\right)^{\phi}\right) \\
&
A_{git}=A_{gt-1}\left(1+\gamma\left(\frac{S_{git}}{\rho_g}\right)^{\eta}\left(\frac{A_{t-1}}{A_{gt-1}}\right)^{\phi}\right) \\
\text{Markets}&\\
&S_{ft}+S_{nt}+S_{gt}=S\\
&L_{ft}+L_{nt}+L_{gt}=L\\
&C_{t}+\int_{0}^{1} x_{fit}+x_{nit}+x_{git}d_i+P_{Ot}O_t=Y_t\\
&P_{Ot} \text{taken as given, imports}
\end{align*}

\subsection{Balanced growth path in \cite{Fried2018ClimateAnalysis}}
She assumes that the ratio of prices is constant and energy prices themselves are constant (p.103) (\textit{that sounds wrong}). Assuming that the spillover effect is sufficiently strong (that is, $\phi$ is large) then a balanced growth path may exist on which the ratio of green and fossil technology, $A_g/A_f$, is constant. This ratio being constant follows from constant price ratios! 

In equilibrium it has to hold that 
\begin{align*}
P_{Gt}G_t= P_{\tilde{F}t}\tilde{F}\left(\frac{P_{\tilde{F}t}}{P_{Gt}}\right)^{\varepsilon_e-1}.
\end{align*} 

\subsection{Model}
\begin{align}
\text{\textbf{Household}}& \max \frac{C_t^{1-\theta}}{1-\theta}-\chi\frac{z_hh_{ht}^{1+\sigma}+z_lh_{lt}^{1+\sigma}}{1+\sigma}-\chi_s\frac{S^{1+\sigma}}{{1+\sigma}} %when z is also to the power of 1+sigma than, the higher zh the lower hours supplied! Not reasonable
\\
\text{Budget}\ \vspace{4mm}& C_t=z_h \lambda_t (w_{ht}h_{ht})^{1-\tau_{lt}}+z_l \lambda_t (w_{lt}h_{lt})^{1-\tau_{lt}}+T^{Gov}_t\\
\text{Optimality}\ \vspace{4mm}
& C_t^{-\theta}= \mu_tp_{yt}\\
& \chi h_{ht}^{\sigma}=\mu_t \lambda_t(1-\tau_{lt})w_{ht}^{1-\tau_{lt}}h_{ht}^{-\tau_{lt}}-\gamma_{ht}/z_h\\
& \chi h_{lt}^{\sigma}=\mu_t \lambda_t(1-\tau_{lt})w_{lt}^{1-\tau_{lt}}h_{lt}^{-\tau_{lt}}-\gamma_{lt}/(1-z_h)\\
%&( h_{st})^{\sigma}=\mu_t \lambda_t(1-\tau_{lt})w_{st}^{1-\tau_{lt}}h_{st}^{-\tau_{lt}}\\
\Rightarrow\ \ & \frac{h_{ht}}{h_{lt}}=\left(\frac{w_{ht}}{w_{lt}}\right)^{\frac{1-\tau_{lt}}{{\sigma+\tau_{lt}}}}\ \text{(Interior solution)}
\\
& \chi_s S^\sigma =\mu w_s-\gamma_{st}\ \text{(scientist income confiscated by gov.)}\\
\text{\textbf{Final good and Energy producers}}&\\
\text{Optimality}\ \vspace{4mm}&
\frac{E_t}{N_t}=\frac{\delta_y}{(1-\delta_y)}\left(\frac{p_{Nt}}{p_{Et}}\right)^{\varepsilon_y} p_{yt}\delta_yY_t^\frac{1}{\varepsilon_y}E_t^{-\frac{1}{\varepsilon_e}}=p_{Et}\\
& p_{yt}(1-\delta_y)Y_t^\frac{1}{\varepsilon_y}N_t^{-\frac{1}{\varepsilon_e}}=p_{Nt}\\
&
p_{Et}E_t^\frac{1}{\varepsilon_e}{F}_t^{-\frac{1}{\varepsilon_e}}=p_{{F}t}\\
& p_{Et}E_t^\frac{1}{\varepsilon_e}G_t^{-\frac{1}{\varepsilon_e}}=p_{Gt}\\
%& p_{\tilde{F}t}\tilde{F}_t^\frac{1}{\varepsilon_f}\delta_fF_t^{-\frac{1}{\varepsilon_f}}=p_{Ft}+\tau_{f}\\
%& p_{\tilde{F}t}\tilde{F}_t^\frac{1}{\varepsilon_f}(1-\delta_f)O_t^{-\frac{1}{\varepsilon_f}}=p_{Ot}+\tau_{o}
%\\
\text{Definitions prices}\ \vspace{4mm}&
p_{yt}= \left[\delta_y^{\varepsilon_y}p_{Et}^{1-\varepsilon_y}+(1-\delta_y)^{\varepsilon_y}p_{Nt}^{1-\varepsilon_y}\right]^\frac{1}{1-\varepsilon_y}\\
& p_{Et}= \left[p_{{F}t}^{1-\varepsilon_e}+p_{Gt}^{1-\varepsilon_y}\right]^\frac{1}{1-\varepsilon_e}\\
%& p_{\tilde{F}t}= \left[\delta_f^{\varepsilon_f}(p_{Ft}+\tau_{f})^{1-\varepsilon_f}+(1-\delta_f)^{\varepsilon_f}(p_{nt}+\tau_{o})^{1-\varepsilon_f}\right]^\frac{1}{1-\varepsilon_f}
%\\
\text{Production}\ \vspace{4mm}& 
Y_t=\left(\delta_yE_t^\frac{\varepsilon_y-1}{\varepsilon_y}+(1-\delta_y)N_t^\frac{\varepsilon_y-1}{\varepsilon_y}\right)^\frac{\varepsilon_y}{\varepsilon_y-1}\\
&E_t=\left({F}_t^\frac{\varepsilon_e-1}{\varepsilon_e}+G_t^\frac{\varepsilon_e-1}{\varepsilon_e}\right)^\frac{\varepsilon_e}{\varepsilon_e-1}\\
%&\tilde{F}_t=\left(\delta_fF_t^\frac{\varepsilon_f-1}{\varepsilon_f}+(1-\delta_f)O_t^\frac{\varepsilon_f-1}{\varepsilon_f}\right)^\frac{\varepsilon_f}{\varepsilon_f-1}\\
\text{\textbf{Intermediate good producers}}&\\
\text{Production}\ \vspace{4mm}& F_t= (\alpha_f(p_{Ft}(1-\tau_{ft})))^\frac{ \alpha_f}{1-\alpha_f}A_{ft}L_{ft}\\
&N_t= (\alpha_np_{Nt})^\frac{ \alpha_n}{1-\alpha_n}A_{nt}L_{nt}\\
&G_t= (\alpha_gp_{Gt})^\frac{ \alpha_g}{1-\alpha_g}A_{gt}L_{gt}\\
%\end{align}
%
%\begin{align}
\text{Labour demand}\label{eq:lab_demand}\\
& w_{lft}=(p_{Ft}(1-\tau_{ft}))^\frac{1}{1-\alpha_f}(1-\alpha_f)(\alpha_f)^\frac{\alpha_f}{1-\alpha_f}A_{ft}\\
& w_{lnt}=p_{Nt}^\frac{1}{1-\alpha_n}(1-\alpha_n)(\alpha_n)^\frac{\alpha_n}{1-\alpha_n}A_{nt}\\
& w_{lgt}=p_{Gt}^\frac{1}{1-\alpha_g}(1-\alpha_g)(\alpha_g)^\frac{\alpha_g}{1-\alpha_g}A_{gt}
\\
\text{Machine demand}&
\\
&x_{fit}= \left(\alpha_f p_{Ft}(1-\tau_{ft})\right)^\frac{1}{1-\alpha_f}L_{ft}A_{fit}\\
&x_{nit}= \left(\alpha_n p_{Nt}\right)^\frac{1}{1-\alpha_n}L_{nt}A_{nit}\\
&x_{git}= \left(\alpha_g p_{Gt}\right)^\frac{1}{1-\alpha_g}L_{gt}A_{git}
\\
\text{\textbf{Labour producers}}&
\\
\text{Production}\ \vspace{4mm}& L_{ft}=h_{hft}^{\theta_{f}}h_{lft}^{1-\theta_{f}}\\
& L_{nt}=h_{hnt}^{\theta_{n}}h_{lnt}^{1-\theta_{n}}\\
& L_{gt}=h_{hgt}^{\theta_{g}}h_{lgt}^{1-\theta_{g}}\\
\ \\
\text{Optimality}\ \vspace{4mm}& h_{hft}= \theta_{f}L_{ft}\frac{w_{lft}}{w_{ht}}\label{eq:opt_lab_pro}\\
& h_{hnt}= \theta_{n}L_{nt}\frac{w_{lnt}}{w_{ht}}\\
& h_{hgt}= \theta_{g}L_{gt}\frac{w_{lgt}}{w_{ht}}\\
& h_{lft}= (1-\theta_{f})L_{ft}\frac{w_{lft}}{w_{lt}}\label{eq:opt_lab_pro_low}\\
& h_{lnt}= (1-\theta_{n}) L_{nt}\frac{w_{lnt}}{w_{lt}}\\
& h_{lgt}= (1-\theta_{g}) L_{gt}\frac{w_{lgt}}{w_{lt}}\\
%\end{align}
%
%\begin{align}
\text{\textbf{Machine producers}}\\
\text{Price setting}\ \vspace{4mm}&p_{fit}^x=\frac{1}{\alpha_f(1+\zeta_f)}\\
&p_{nit}^x=\frac{1}{\alpha_n(1+\zeta_n)}\\
&p_{git}^x=\frac{1}{\alpha_g(1+\zeta_g)}
\\ 
\text{Demand Scientists}\ \vspace{4mm}&
w_{sft}=\frac{\eta \gamma A_{ft-1}^{1-\phi}A_{t-1}^{\phi}\left(\frac{S_{ft}}{\rho_f}\right)^{\eta}p_{Ft}(1-\tau_{ft})F_t}{\frac{1}{1-\alpha_f}S_{ft}A_{ft}}\\
&
w_{snt}=\frac{\eta \gamma  A_{nt-1}^{1-\phi}A_{t-1}^{\phi}\left(\frac{S_{nt}}{\rho_n}\right)^{\eta}p_{Nt}N_t}{\frac{1}{1-\alpha_n}S_{nt}A_{nt}}\\
&
w_{sgt}=\frac{\eta \gamma  A_{gt-1}^{1-\phi}A_{t-1}^{\phi}\left(\frac{S_{gt}}{\rho_g}\right)^{\eta}p_{Gt}G_t}{\frac{1}{1-\alpha_g}S_{gt}A_{gt}(1-\tau_{st})}\\
\text{Innovation}\ \vspace{4mm}&
A_{fit}=A_{ft-1}\left(1+\gamma\left(\frac{S_{fit}}{\rho_f}\right)^{\eta}\left(\frac{A_{t-1}}{A_{ft-1}}\right)^{\phi}\right) \\
&
A_{nit}=A_{nt-1}\left(1+\gamma\left(\frac{S_{nit}}{\rho_n}\right)^{\eta}\left(\frac{A_{t-1}}{A_{nt-1}}\right)^{\phi}\right) \\
&
A_{git}=A_{gt-1}\left(1+\gamma\left(\frac{S_{git}}{\rho_g}\right)^{\eta}\left(\frac{A_{t-1}}{A_{gt-1}}\right)^{\phi}\right) \\
%\text{Demand Scientists}\ \vspace{4mm}&
%w_{sft}=\frac{\eta \gamma \alpha_f A_{ft-1}\left(\frac{S_{ft}}{\rho_f}\right)^{\eta}p_{Ft}F_t}{\frac{1}{1-\alpha_f}S_{ft}A_{ft}}\label{eq:demand_sc}\\
%&
%w_{snt}=\frac{\eta \gamma \alpha_n A_{nt-1}\left(\frac{S_{nt}}{\rho_n}\right)^{\eta}p_{Nt}N_t}{\frac{1}{1-\alpha_n}S_{nt}A_{nt}}\\
%&
%w_{sgt}=\frac{\eta \gamma \alpha_g A_{gt-1}\left(\frac{S_{gt}}{\rho_g}\right)^{\eta}p_{Gt}G_t}{\frac{1}{1-\alpha_g}S_{gt}A_{gt}}\\
%\text{Innovation}\ \vspace{4mm}&
%A_{fit}=A_{ft-1}\left(1+\gamma\left(\frac{S_{fit}}{\rho_f}\right)^{\eta}\right) \\
%&
%A_{nit}=A_{nt-1}\left(1+\gamma\left(\frac{S_{nit}}{\rho_n}\right)^{\eta}\right) \\
%&
%A_{git}=A_{gt-1}\left(1+\gamma\left(\frac{S_{git}}{\rho_g}\right)^{\eta}\right) \\
%&A_t=\max\{A_{nt}, A_{ft}, A_{gt}\}\\`
%DROP THE FOLLOWING AS IT DOES NOT GROW AT A CONSTANT RATE IF SHARES ARE CHANGING! &A_t= \frac{\rho_fA_{ft}+\rho_gA_{gt}+\rho_nA_{nt}}{\rho_f+\rho_n+\rho_g}\\
\text{\textbf{Government}}&\\
&T_t=\int_{0}^{1}\pi_{fit}+\pi_{git}+\pi_{nit}di+z_h(w_{ht}h_{ht}-\lambda_t(w_{ht}h_{ht})^{(1-\tau_{lt})})\\&+z_l(w_{lt}h_{lt}-\lambda_t(w_{lt}h_{lt})^{(1-\tau_{lt})})+w_{st}s_{ft}+w_{st}s_{gt}+w_{st}s_{nt}\\ &+\tau_{ct}p_{ft}(\omega_FF_t)-w_{sgt}\tau_{st}s_{gt} \\ &+\int_{0}^{1} p_{fit}\zeta_{ft}x_{fit}+p_{git}\zeta_{gt}x_{git}+ p_{hit}\zeta_{ht}x_{hit}di\\
\text{with}\ \vspace{4mm}&\zeta_{jt}=\frac{1-\alpha_j}{\alpha_j} \\
\text{simplified}\ \vspace{4mm} & T_t= z_h(w_{ht}h_{ht}-\lambda_t(w_{ht}h_{ht})^{(1-\tau_{lt})})\\&+z_l(w_{lt}h_{lt}-\lambda_t(w_{lt}h_{lt})^{(1-\tau_{lt})})+\tau_{ct}p_{ft}(\omega_FF_t) \\
\text{\textbf{Markets}}&\\
& S_{ft}+ S_{nt}+ S_{gt}=S_t\\
&h_{hft}+h_{hnt}+h_{hgt}=z_{h} h_{ht}\\
&h_{lft}+h_{lnt}+h_{lgt}=z_{l} h_{lt}\\
&C_{t}+\int_{0}^{1} x_{fit}+x_{nit}+x_{git}d_i=Y_t
\end{align}

\subsection{BGP: which can accomodate a trend in hours and diverging productivity shares}
\textbf{\tr{But this version assumes constant labour shares \ar this implies constant mashine growth and labour growth in each sector. Then again, on the BGP labour supply reduces, I only assume this affects sector labour input proportionately. There may also not be any growth in machines. }}
From the optimality condition in skill demand by labour producing firms, equations \ref{eq:opt_lab_pro} and \ref{eq:opt_lab_pro_low}, and the price paid by intermediate good producers in sector $j\in\{F, G, N\}$, $w_{ljt}$, equations \ref{eq:lab_demand}, yields the sector specific demand for high and low skill as a function of output in this sector. Substituting these equations in the intermediate good production function determines technology as a function of prices:
\begin{align}
A_{jt} = \left[\alpha_j^{2\frac{\alpha_j}{1-\alpha_j}}p_{jt}^\frac{1}{1-\alpha_j}(1-\alpha_j)\left(\frac{1-\theta_j}{w_{lt}}\right)^{1-\theta_j}\left(\frac{\theta_j}{w_{ht}}\right)^{\theta_j}\right] ^{-1}
\end{align}
Under the assumption of a stable wage premium, one can detrend technological progress as:
\begin{align}
\hat{A_{jt}}:=\frac{A_{jt}p_{jt}^\frac{1}{1-\alpha_j}}{w_{ht}}= \left[\alpha_j^{2\frac{\alpha_j}{1-\alpha_j}}(1-\alpha_j)\left(1-\theta_j\right)^{1-\theta_j}\theta_j^{\theta_j}\left(\frac{w_{ht}}{w_{lt}}\right)^{1-\theta_j}\right] ^{-1}\label{eq:A_det}
\end{align}
Both wage rates for high and low skill labour hence grow at the rate
\begin{align}
\gamma_{w}=\left(\frac{p_{jt+1}}{p_{jt}}\right)^\frac{1}{1-\alpha_j}\frac{A_{jt+1}}{A_{jt}}-1  \ \forall \ j. 
\end{align}
Hence, free skill movement across labour input firms, implies that on a BGP
\begin{align}
\left(\frac{p_{gt+1}}{p_{gt}}\right)^\frac{1}{1-\alpha_g}\frac{A_{gt+1}}{A_{gt}}=\left(\frac{p_{ft+1}}{p_{ft}}\right)^\frac{1}{1-\alpha_f}\frac{A_{ft+1}}{A_{ft}}=\left(\frac{p_{nt+1}}{p_{nt}}\right)^\frac{1}{1-\alpha_n}\frac{A_{nt+1}}{A_{nt}}. \label{eq:const_prA} 
\end{align}

\tr{Drop assumption that input shares are constant}
These conditions ensure that relative expenditures on intermediate goods are constant \textbf{whenever the labour input ratios are constant}. Substituting intermeidate good production into the expenditure ratio $p_{ft}F_t/(p_{gt}G_t)$  yields
\begin{align}
&\frac{p_{ft+1}^\frac{1}{1-\alpha_f}A_{ft+1}L_{ft+1}}{p_{gt+1}^\frac{1}{1-\alpha_g}A_{gt+1}L_{gt+1}}= \frac{p_{ft}^\frac{1}{1-\alpha_f}A_{ft}L_{ft}}{p_{gt}^\frac{1}{1-\alpha_g}A_{gt}L_{gt}}\\
&\Leftrightarrow \frac{L_{ft+1}}{L_{gt+1}}=\frac{L_{ft}}{L_{gt}},
\end{align}
where the second line follows from \ref{eq:const_prA}. 

Note that the assumption that the ratio of technological progress is constant over time is necessary to have aggregate technology, $A_t$, as defined in Fried constant. I don't want to make this assumption to allow for zero growth in the fossil sector. Therefore, I define the leading technology as 
\begin{align}
A_t= \max\{A_{nt}, A_{gt}, A_{ft}\}
\end{align}
Since on the BGP each technology growths at a constant rate, the leading technology growths at a constant rate  whenever the fastest growing technology is also the leading one in levels.

The assumption of a stable wage premium together with a constant progressivity parameter ensures that relative skill supply on the BGP is stable, too: 
\begin{align}
\frac{H_{ht}}{H_{lt}}=\left(\frac{w_{ht}}{w_{lt}}\right)^\frac{1-\tau_{lt}}{1+\sigma}\frac{z_h}{z_l}.
\end{align}

A constant wage ratio also implies that relative skill employment in each sector is constant, $\frac{h_{hj}}{h_{lf}}$, which follows from skill demand by labour producers. 


Observe that skill shares employed in each sector, $\frac{h_{pjt}}{H_{pt}} \forall \ j\ \text{and} \ p\in\{h,l\}$, are constant given constant expenditure ratios. 
To see this, substitute demand for skill inputs by labour producing firms in the market clearing for high skill labour. Substituting labour demand by intermediate good producers yields
\begin{align}
w_{ht}H_{ht}= \left(\underbrace{\theta_f (1-\alpha_f)\frac{p_{ft}F_t}{p_{gt}G_t}+\theta_{g}(1-\alpha_g)+\theta_n(1-\alpha_n)\frac{p_{nt}N_t}{p_{gt}G_t}}_{:=\vartheta_t}\right)p_{gt}G_t
\end{align}
Replacing $G_t$ by its production function and substituting $L_{gt}$ using the production function and optimal skill ratios implies
\begin{align}
\frac{h_{lgt}}{H_{ht}}=\left(\frac{1}{\vartheta_t\alpha_g^{2\frac{\alpha_g}{1-\alpha_g}}\left(\frac{\theta_g}{1-\theta_g}\frac{w_{lt}}{w_{ht}}\right)^{\theta_{g}}}\right)\frac{w_{ht}}{A_{gt}p_{gt}^\frac{1}{1-\alpha_g}}.
\end{align}
Since expenditure shares are constant, so is the first multiplier. The second multiplier is constant following equation \ref{eq:A_det}. Since $\frac{H_h}{H_l}$ is constant by households optimality condition, above equation implies that $\frac{h_{lgt}}{H_{lt}}$ is time invariant. Note that I assume that skill shares (relative to total skill supplied) are stable on the BGP. 
\paragraph{Scientists and the marginal gains from innovation}
Scientists are in fixed supply. They receive the competitive wage rate and consume their income. Their income is not taxed and they do not receive transfers. This is to avoid redistribution from scientists to households and vice versa.  (In an extension could assume preferential tax schemes for scientists.) Could also add income from scientists to representative household but let it not be taxed. 
%The amount of scientists on the BGP is flexible in order to keep sector-specific technological growth constant. This follows from the law of motion from technologym let $\gamma_{Aj}$ denote technology growth in sector j:
%\begin{align}
%S_{jt}=\left(\frac{\gamma_{Aj}}{\gamma}\right)^\frac{1}{\eta}\rho_j{A_{t-1}}^{\frac{-\phi}{\eta}}. \label{eq:scien}
%\end{align}
%The amount of scientists varies with overall productivity. For $\eta>0$ and $\phi>0$ the higher output the lower the amount of scientists employed. 
%The marginal profit of innovation positively depends on aggregate technology. 


In equilibrium, wages for scientists, i.e. the marginal product of innovations, are determined by equations \ref{eq:demand_sc}. 
Free movement of scientists requires that
\begin{align}
\left(\frac{S_{gt}}{S_{ft}}\right)^{\eta-1}= \frac{(1-\alpha_f)\alpha_f}{(1-\alpha_g)\alpha_g}\frac{p_{ft}F_t}{p_{gt}G_t}\left(\frac{\rho_g}{\rho_f}\right)^\eta.
\end{align}
Note, that in equilibrium, the equation states that the higher the ratio of scientists in sector g relative to sector f, the higher the technology gap in favour of sector g, under the assumption of increasing returns to scale, $\eta>1$. Increasing returns to scale in innovation seem reasonable, for example, synergy effects from teamwork. Creativity benefits from communication. Increasing returns can be perceived as positive spill-over effects within a sector. 


Substituting equation \ref{eq:scien}, yields a condition on the sector-specific spillover of innovation $\eta$ so that the ratio of workers across sectors is constant on the BGP.
It has to hold that
\begin{align}
\frac{L_{gt}}{L_{ft}}= \left(\frac{\gamma_{AF}}{\gamma_{AG}}\right)^{\eta-1}\left(\frac{1+\gamma_{AG}}{1+\gamma_{AF}}\right)^\phi \left(\frac{A_{ft-1}}{A_{gt-1}}\right)^{\phi(\eta-2)}\frac{\rho_g}{\rho_f}\frac{(1-\alpha_f)\alpha_f^{2\frac{\alpha_f}{1-\alpha_f}}}{(1-\alpha_g)\alpha_g^{2\frac{\alpha_g}{1-\alpha_g}}}\frac{p_{ft}^\frac{1}{1-\alpha_f}A_{ft}}{p_{gt}^\frac{1}{1-\alpha_g}A_{gt}}.
\end{align}

All terms are stable except for $\left(\frac{A_{ft-1}}{A_{gt-1}}\right)^{\phi(\eta-2)}$. Hence, for a balanced growth to have constant worker ratios across sectors, there have to be increasing returns to scientist within sectors and $\eta=2$.
 
Alternatively, one might want to abstain from stable employment ratios. However, then expenditure shares would not be constant, which again is consistent with structural change. When expenditure shares are not constant, then skill moves across sectors. 
However, my take on the BGP in this model is far in the future, after all transitions across sectors have taken place. Note, that I do not need to assume a BGP from 2050 onwards. 
 

\begin{comment}
%content...
\subsection{Equilibrium conditions: own model}

\begin{align*}
\text{\textbf{Household solved:}} \hspace{50mm}& \\
\text{FOCs labour supply}\hspace{4mm}&  %\log(H_t)=\frac{1}{1+\sigma}\log(1-\tau_{lt})\\
H_t=(1-\tau_{lt})^\frac{1}{1+\sigma}\\
\ \hspace{4mm} & %\log(w_{ht})=\log(w_{lt})+\log(\zeta)\\
w_{ht}=\zeta w_{lt}\\
\text{Budget}\hspace{4mm}&  %\log(c_t)= \log(\lambda_t)+ (1-\tau_{lt})\left[\frac{1}{1+\sigma}\log(1-\tau_{lt})+\log(w_{lt})\right]\\
c_t= \lambda_t (H_tw_{lt})^{(1-\tau_{lt})}\\
\text{definition}\  H_t\hspace{4mm} & %\log(H_t)=\log(h_{lt}+\zeta h_{ht})\\
H_t=\zeta h_{ht}+h_{lt}
\\
\text{\textbf{General Household Problem:}} \hspace{50mm}& \\
\text{FOC consumption}\hspace{4mm}& Mu_{ct}=p_t\mu_t\\
\text{FOC low skill}\hspace{4mm} & -Mu_{h_lt}=\mu_t \frac{\partial I_t}{\partial h_{lt}}\\
\text{FOC high skill}\hspace{4mm} & -Mu_{h_ht}=\mu_t \frac{\partial I_t}{\partial h_{ht}}\\
\text{Budget}\hspace{4mm}& c_tp_t= I_t\\
\text{definition}\  H_t\hspace{4mm} & H_t=\zeta h_{ht}+h_{lt}\\
\text{\textbf{Labour sectors:}}\hspace{50mm}&\\
\text{Production clean labour input} \hspace{4mm}& L_{ct}=l_{hct}^{\theta_c}l_{lct}^{1-\theta_c}\\ 
\text{Production dirty labour input} \hspace{4mm}& L_{dt}=l_{hdt}^{\theta_d}l_{ldt}^{1-\theta_d}\\
%
\text{Demand high skill clean sector}\hspace{4mm}&l_{hct}= \left(\frac{p_{cLt}}{w_{ht}}\right)^{\frac{1}{1-\theta_c}}\theta_c^{\frac{1}{1-\theta_c}}l_{lct}\\
%
\text{Demand low skill clean sector } \hspace{4mm}&l_{lct}= \left(\frac{p_{cLt}}{w_{lt}}\right)^{\frac{1}{\theta_c}}(1-\theta_c)^{\frac{1}{\theta_c}}l_{hct}\\
%
\text{Demand high skill dirty sector} \hspace{4mm}&l_{hdt}= \left(\frac{p_{dLt}}{w_{ht}}\right)^{\frac{1}{1-\theta_d}}\theta_d^{\frac{1}{1-\theta_d}}l_{ldt}\\
%
\text{Demand low skill dirty sector } \hspace{4mm}&l_{ldt}= \left(\frac{p_{dLt}}{w_{lt}}\right)^{\frac{1}{\theta_d}}(1-\theta_d)^{\frac{1}{\theta_d}}l_{hdt}\\
\text{\textbf{Government}}\hspace{50mm}& \nonumber\\
\text{Budget}\hspace{4mm}& G_t=H_tw_{lt}-\lambda_t(H_t w_{lt})^{(1-\tau_{lt})}
\\
\text{\textbf{Technology:}}\hspace{50mm}&\\
\text{Clean sector}\hspace{4mm}& A_{ict+1}=(1+\upsilon_{ct})A_{ict}\\
\text{Dirty sector}\hspace{4mm}& A_{idt+1}=(1+\upsilon_{dt})A_{idt}\\
%\text{Progress bound}\hspace{4mm}& \upsilon_{ct}+\upsilon_{dt}=\Upsilon\\
\text{Definition average clean technology}\hspace{4mm}& A_{ct}=\int_0^1A_{ict}di\\
\text{Definition average dirty technology}\hspace{4mm}& A_{dt}=\int_0^1A_{idt}di
\end{align*}

\begin{align}
\text{\textbf{Production:}} \hspace{4mm}
\text{\textbf{Final Good Producer}}&\\
\text{Profit maximisation}\hspace{4mm} & Y_{nt}=\left(\frac{p_{ct}}{p_{dt}}\right)^\varepsilon Y_{ct}\\
\text{Production}\hspace{4mm} & Y_t=\left[Y_{ct}^{\frac{\varepsilon-1}{\varepsilon}}+Y_{dt}^{\frac{\varepsilon-1}{\varepsilon}}\right]^{\frac{\varepsilon}{\varepsilon-1}}\\
\text{Price}\hspace{4mm}& p_t:=\left[p_{ct}^{1-\varepsilon}+p_{dt}^{1-\varepsilon}\right]^{\frac{1}{1-\varepsilon}}\\
\text{\textbf{Clean Sector}}\\
\text{Production}\hspace{4mm}& Y_{ct}=L^{1-\alpha}_{ct}\int_{0}^{1}A^{1-\alpha}_{ict}x_{ict}^{\alpha}di=  \left(\alpha\frac{p_{ct}}{\psi}\right)^{\frac{\alpha}{1-\alpha}}A_{ct} L_{ct} \label{eqbm:outputc}
\\ & =x_{ct}^{\alpha}\left(A_{ct}L_{ct}\right)^{1-\alpha} \\ 
\text{labour demand}\hspace{4mm} & p_{cLt} =
(1-\alpha)\left(\frac{\alpha}{\psi}\right)^\frac{\alpha}{1- \alpha}p_{ct}^\frac{1}{1-\alpha}A_{ct} \label{eqbm:labc} \\
\text{machine demand}\hspace{4mm} & x_{ict} = \left(\alpha\frac{ p_{ct}}{p_{ict}}\right)^\frac{1}{1-\alpha}A_{ict}L_{ct}\\
& x_{ct}:=\int_{0}^{1}x_{ict} di= \left(\alpha\frac{p_{ct}}{\psi}\right)^\frac{1}{1-\alpha}A_{ct}L_{ct}\\
%
\text{Supply machines (price)}\hspace{4mm}& p_{ict}=\psi \\
%
\text{\textbf{Dirty Sector}}\\
\text{Production}\hspace{4mm} & Y_{dt}=L^{1-\alpha}_{dt}\int_{0}^{1}A^{1-\alpha}_{idt}x_{idt}^{\alpha}di=  \left(\alpha\frac{p_{dt}}{\psi}\right)^{\frac{\alpha}{1-\alpha}}A_{dt} L_{dt}\label{eqbm:outputd}\\ & =x_{dt}^{\alpha}\left(A_{dt}L_{dt}\right)^{1-\alpha} \\ 
\text{labour demand}\hspace{4mm} & p_{dLt} =
(1-\alpha)\left(\frac{\alpha}{\psi}\right)^\frac{\alpha}{1- \alpha}p_{dt}^\frac{1}{1-\alpha}A_{dt}\label{eqbm:labd}\\
\text{machine demand}\hspace{4mm} & x_{idt} = \left(\alpha\frac{ p_{dt}}{p_{idt}}\right)^\frac{1}{1-\alpha}A_{idt}L_{dt}\\
& x_{dt}:=\int_{0}^{1}x_{idt} di= \left(\alpha\frac{p_{dt}}{\psi}\right)^\frac{1}{1-\alpha}A_{dt}L_{dt}\\
\text{Supply machines (price)}\hspace{4mm}& p_{idt}=\psi\\
\text{\textbf{Market clearing:}}\hspace{50mm}& \nonumber\\
\text{Final Good}\hspace{4mm}& Y_{t}=c_t+\psi\left(\int_{0}^1x_{idt}di+\int_{0}^1x_{ict}di\right)+G_t%\psi \left(\int_0^1\left(\alpha\frac{p_{dt}}{\psi}\right)^\frac{1}{1-\alpha}A_{idt}L_{dt}+\int_0^1\left(\alpha\frac{p_{ct}}{\psi}\right)^\frac{1}{1-\alpha}A_{ict}L_{ct}\right)
\\
%& \ (\text{Numeraire}\ \  p_t=1)\\
\text{high skill}\hspace{4mm}& l_{hct}+l_{hdt}=h_{ht}\\
\text{low skill}\hspace{4mm}&l_{lct}+l_{ldt}=h_{lt}
\end{align}
\end{comment}


\begin{comment}
\subsection{Equilibrium conditions: Simplified model with Lc=hh, ld=hl}

\begin{align*}
\text{\textbf{Household solved:}} \hspace{50mm}& \\
\text{FOCs labour supply}\hspace{4mm}&  %\log(H_t)=\frac{1}{1+\sigma}\log(1-\tau_{lt})\\
H_t=(1-\tau_{lt})^\frac{1}{1+\sigma}\\
\ \hspace{4mm} & %\log(w_{ht})=\log(w_{lt})+\log(\zeta)\\
w_{ht}=\zeta w_{lt}\\
\text{Budget}\hspace{4mm}&  %\log(c_t)= \log(\lambda_t)+ (1-\tau_{lt})\left[\frac{1}{1+\sigma}\log(1-\tau_{lt})+\log(w_{lt})\right]\\
c_t= \lambda_t (H_tw_{lt})^{(1-\tau_{lt})}\\
\text{definition}\  H_t\hspace{4mm} & %\log(H_t)=\log(h_{lt}+\zeta h_{ht})\\
H_t=\zeta h_{ht}+h_{lt}
\\
%\text{\textbf{Labour sectors:}}\hspace{50mm}&\\
%\text{Production clean labour input} \hspace{4mm}& L_{ct}=h_{ht}\\ 
%\text{Production dirty labour input} \hspace{4mm}& L_{dt}=h_{lt}\\
%
\text{\textbf{Government}}\hspace{50mm}& \nonumber\\
\text{Budget}\hspace{4mm}& G_t=H_tw_{lt}-\lambda_t(H_t w_{lt})^{(1-\tau_{lt})}
\\
\text{\textbf{Technology:}}\hspace{50mm}&\\
\text{Clean sector}\hspace{4mm}& A_{ict+1}=(1+\upsilon_{ct})A_{ict}\\
\text{Dirty sector}\hspace{4mm}& A_{idt+1}=(1+\upsilon_{dt})A_{idt}\\
%\text{Progress bound}\hspace{4mm}& \upsilon_{ct}+\upsilon_{dt}=\Upsilon\\
\text{Definition average clean technology}\hspace{4mm}& A_{ct}=\int_0^1A_{ict}di\\
\text{Definition average dirty technology}\hspace{4mm}& A_{dt}=\int_0^1A_{idt}di
\end{align*}

\begin{align*}
\text{\textbf{Production:}} \hspace{4mm}
\text{\textbf{Final Good Producer}}&\\
\text{Profit maximisation}\hspace{4mm} & Y_{nt}=\left(\frac{p_{ct}}{p_{dt}}\right)^\varepsilon Y_{ct}\\
\text{Production}\hspace{4mm} & Y_t=\left[Y_{ct}^{\frac{\varepsilon-1}{\varepsilon}}+Y_{dt}^{\frac{\varepsilon-1}{\varepsilon}}\right]^{\frac{\varepsilon}{\varepsilon-1}}\\
\text{Price}\hspace{4mm}& p_t:=\left[p_{ct}^{1-\varepsilon}+p_{dt}^{1-\varepsilon}\right]^{\frac{1}{1-\varepsilon}}\\
\text{\textbf{Clean Sector}}\\
\text{Production}\hspace{4mm}& Y_{ct}=L^{1-\alpha}_{ct}\int_{0}^{1}A^{1-\alpha}_{ict}x_{ict}^{\alpha}di=  \left(\alpha\frac{p_{ct}}{\psi}\right)^{\frac{\alpha}{1-\alpha}}A_{ct} L_{ct}
\\ & =x_{ct}^{\alpha}\left(A_{ct}L_{ct}\right)^{1-\alpha} \\ 
\text{labour demand}\hspace{4mm} & w_{ht} =
(1-\alpha)\left(\frac{\alpha}{\psi}\right)^\frac{\alpha}{1- \alpha}p_{ct}^\frac{1}{1-\alpha}A_{ct}\\
\text{machine demand}\hspace{4mm} & x_{ict} = \left(\alpha\frac{ p_{ct}}{p_{ict}}\right)^\frac{1}{1-\alpha}A_{ict}L_{ct}\\
& x_{ct}:=\int_{0}^{1}x_{ict} di= \left(\alpha\frac{p_{ct}}{\psi}\right)^\frac{1}{1-\alpha}A_{ct}L_{ct}\\
%
\text{Supply machines (price)}\hspace{4mm}& p_{ict}=\psi \\
%
\text{\textbf{Dirty Sector}}\\
\text{Production}\hspace{4mm} & Y_{dt}=L^{1-\alpha}_{dt}\int_{0}^{1}A^{1-\alpha}_{idt}x_{idt}^{\alpha}di=  \left(\alpha\frac{p_{dt}}{\psi}\right)^{\frac{\alpha}{1-\alpha}}A_{dt} L_{dt}\\ & =x_{dt}^{\alpha}\left(A_{dt}L_{dt}\right)^{1-\alpha} \\ 
\text{labour demand}\hspace{4mm} & w_{lt} =
(1-\alpha)\left(\frac{\alpha}{\psi}\right)^\frac{\alpha}{1- \alpha}p_{dt}^\frac{1}{1-\alpha}A_{dt}\\
\text{machine demand}\hspace{4mm} & x_{idt} = \left(\alpha\frac{ p_{dt}}{p_{idt}}\right)^\frac{1}{1-\alpha}A_{idt}L_{dt}\\
& x_{dt}:=\int_{0}^{1}x_{idt} di= \left(\alpha\frac{p_{dt}}{\psi}\right)^\frac{1}{1-\alpha}A_{dt}L_{dt}\\
\text{Supply machines (price)}\hspace{4mm}& p_{idt}=\psi\\
\text{\textbf{Market clearing:}}\hspace{50mm}& \nonumber\\
\text{Final Good}\hspace{4mm}& Y_{t}=c_t+\psi\left(\int_{0}^1x_{idt}di+\int_{0}^1x_{ict}di\right)+G_t%\psi \left(\int_0^1\left(\alpha\frac{p_{dt}}{\psi}\right)^\frac{1}{1-\alpha}A_{idt}L_{dt}+\int_0^1\left(\alpha\frac{p_{ct}}{\psi}\right)^\frac{1}{1-\alpha}A_{ict}L_{ct}\right)
\\
%& \ (\text{Numeraire}\ \  p_t=1)\\
\text{high skill}\hspace{4mm}& L_{ct}=h_{ht}\\
\text{low skill}\hspace{4mm}&L_{dt}=h_{lt}
\end{align*}

\section{Solution of tractable model}\label{app:solu}
Define
\begin{align*}
	\tilde{\kappa}:=\ &\frac{(1-\theta_c)(1-\theta_d)\left[\left(\frac{A_c}{A_d}\right)^{(1-\alpha)(1-\varepsilon)}\zeta^{-(\theta_c-\theta_d)(1-\alpha)(1-\varepsilon)}\tilde{\chi}+1\right]}{(1-\theta_d)+(1-\theta_c)\left[\left(\frac{A_c}{A_d}\right)^{(1-\alpha)(1-\varepsilon)}\zeta^{-(\theta_c-\theta_d)(1-\alpha)(1-\varepsilon)}\tilde{\chi}\right]}\\
	\gamma_j:=\ & \left(\frac{\theta_j}{\zeta(1-\theta_j)}\right)^{\theta_j}\\
	z_j:=\ &\theta_j^{\theta_j}(1-\theta_j)^{1-\theta_j} \\
	\chi:=\ &% \frac{(1-\theta_d)(1-\theta_c)}{\theta_c(1-\theta_d)-\theta_d(1-\theta_c)}
	\frac{(1-\theta_d)(1-\theta_c)}{\theta_c-\theta_d}\\
	\tilde{\chi}: =\ &  (\theta_c^{\theta_c}\theta_d^{-\theta_d})^{(1-\alpha) (1-\varepsilon)}(1-\theta_c)^{-\theta_c-(1-\theta_c)(\alpha+\varepsilon(1-\alpha))}(1-\theta_d)^{\theta_d+(1-\theta_d)(\alpha+\varepsilon(1-\alpha))}
\end{align*}
From profit maximisation by labour input good producers follows that the price of the labour input good relative to the skill-specific wage rate is constant. Substituting demand for low skill in the clean sector into the demand for high skill yields

\begin{align*}
	w_{h}^{\frac{1}{1-\theta_c}}w_l^{\frac{1}{\theta_c}}= p_{cL}^\frac{1}{(1-\theta_c)\theta_c}\theta_c^\frac{1}{1-\theta_c}(1-\theta_c)^\frac{1}{\theta_c}.
\end{align*}
Multiplying the left-hand side with $(w_h/w_h)^\frac{1}{\theta_c}$ and
using the FOC governing skill supply $w_h/w_l=\zeta$, it holds that

\begin{align}\label{eq:constant}
%	& \zeta^\frac{-1}{\theta_c}w_h^\frac{1}{(1-\theta_c)\theta_c}= p_{cL}^\frac{1}{(1-\theta_c)\theta_c}\theta_c^\frac{1}{1-\theta_c}(1-\theta_c)^\frac{1}{\theta_c}\nonumber\\
%	\Leftrightarrow\ 
& \frac{p_{cL}}{w_h}= \frac{\zeta^{-(1-\theta_c)}}{z_c}.
\end{align}
%\noindent \tr{Note: this result does not rely on the claim that the labour input good is constant.}

Analogously to \ref{eq:constant}, it follows that
\begin{align}
	\frac{p_{cL}}{w_l}&=\frac{\zeta^{\theta_c}}{z_c}\label{eq:pcl_wl}\\
	\frac{p_{dL}}{w_l}&=\frac{\zeta^{\theta_d}}{z_d}%\ \Leftrightarrow\ w_l= p_{dL}\zeta^{-\theta_d}\theta_d^{\theta_d}(1-\theta_d)^{1-\theta_d}
	\label{eq:pdl_wl}\\
	\frac{p_{dL}}{w_h}&=\frac{\zeta^{-(1-\theta_d)}}{z_d}.
\end{align}
Therefore, the optimal skill input ratios in the labour good production are given by
\begin{align}\label{eq:inputr}
	\frac{l_{hc}}{l_{lc}}=\frac{\theta_c}{\zeta (1-\theta_c)} \hspace{2mm} \text{and}\hspace{3mm} \frac{l_{hd}}{l_{ld}}=\frac{\theta_d}{\zeta (1-\theta_d)}.
\end{align}
This is the common result that  factor shares 
% this refers to (wh lhc)/(wl llc)
are constant over time with a Cobb-Douglas production function. 
Imposing labour market clearing for both skills and optimal skill demand yields 
\begin{align}
	&l_{ld}=\chi\left(\frac{1}{1-\theta_c}h_l-H\right)\label{eq:lld}\\ %\frac{\theta_c}{1-\theta_c}\chi h_l-\chi \zeta h_h,\\
	& l_{lc}=\chi \left(H-\frac{1}{1-\theta_d}h_l\right)\label{eq:llc} %\\
%	with \ & \chi:= \frac{(1-\theta_d)(1-\theta_c)}{\theta_c(1-\theta_d)-\theta_d(1-\theta_c)}=\frac{(1-\theta_d)(1-\theta_c)}{\theta_c-\theta_d}.
	%& l_{hc}= \frac{\theta_c}{\zeta (1-\theta_c)}l_{lc}\\
	%& l_{hd}=\frac{\theta_d}{\zeta (1-\theta_d)}l_{ld}
\end{align}
Labour good supply follows from the labour input good's production function and optimal skill inputs, equations \ref{eq:inputr}, as
\begin{align}
	L_c&=\gamma_cl_{lc}\label{eq:lab_inputc} \\
	L_d&=\gamma_dl_{ld}.\label{eq:lab_inputd}
\end{align}
%\tr{Note that now policy can affect inflation/ relative prices through changes in labour supply---NOPE: cancels}
A relation of the relative price in equilibrium results from equating demand for the labour input goods, equations \ref{eqbm:labc} and \ref{eqbm:labd}, % (which relates the price for the labour input good and the price for the sector-specific final good), 
demand for low skill input by labour producers, equations \ref{eq:pcl_wl} and \ref{eq:pdl_wl}, and free movement of skills: 
\begin{align}\label{eq:price_ratio_labourinput}
	\frac{p_c}{p_d}= \left(\frac{A_d}{A_c}\frac{z_d}{z_c}\zeta^{\theta_c-\theta_d}\right)^{1-\alpha}& \text{(optimality labour input production)}
\end{align}
%where
%\begin{align*}
%	z_j=\theta_j^{\theta_j}(1-\theta_j)^{1-\theta_j}
%\end{align*}
Together with the definition of the aggregate price level and the choice of $Y$ as numeraire, equation \ref{eq:price_ratio_labourinput} determines sector-specific prices as a function of parameters and productivity:
%\begin{align*}
%	p_c= \left(1+\left(\frac{\gamma_c}{\gamma_d}\frac{A_c}{A_d}\frac{l_{lc}}{l_{ld}}\right)^{\frac{(1-\alpha)(1-\varepsilon)}{\alpha+\varepsilon(1-\alpha)}}\right)^{-\frac{1}{1-\varepsilon}}.
%\end{align*}
%Substituting equation \ref{eq:lldllc} gives the price of the clean good in equilibrium as
\begin{align}
	p_c%& = \frac{1}{\left(1+\left(\frac{A_c}{A_d}\right)^{(1-\alpha)(1-\varepsilon)}\left(\frac{z_c}{z_d}\right)^{(1-\alpha)(1-\varepsilon)}\zeta^{-(\theta_c-\theta_d)(1-\alpha)(1-\varepsilon)}\right)^{\frac{1}{1-\varepsilon}}}\\
	&= \left(\frac{\left(A_dz_d\zeta^{\theta_c}\right)^{(1-\alpha)(1-\varepsilon)}}{\left(A_dz_d\zeta^{\theta_c}\right)^{(1-\alpha)(1-\varepsilon)}+\left(A_cz_c\zeta^{\theta_d}\right)^{(1-\alpha)(1-\varepsilon)}}\right)^{\frac{1}{1-\varepsilon}}\label{eq:eq_pc}\\
%\end{align}
%%and using equation \ref{eq:price_ratio_labourinput} yields
%\begin{align}
	p_d%& =\frac{1}{\left(\left(\frac{A_d}{A_c}\right)^{(1-\alpha)(1-\varepsilon)}\left(\frac{z_d}{z_c}\right)^{(1-\alpha)(1-\varepsilon)}\zeta^{(\theta_c-\theta_d)(1-\alpha)(1-\varepsilon)}+1\right)^{\frac{1}{1-\varepsilon}}}\\
	&= \left(\frac{\left(A_cz_c\zeta^{\theta_d}\right)^{(1-\alpha)(1-\varepsilon)}}{\left(A_dz_d\zeta^{\theta_c}\right)^{(1-\alpha)(1-\varepsilon)}+\left(A_cz_c\zeta^{\theta_d}\right)^{(1-\alpha)(1-\varepsilon)}}\right)^{\frac{1}{1-\varepsilon}} \label{eq:eq_pd}
\end{align}

\paragraph{Skill allocation}
To solve for the equilibrium ratio of skill inputs in the clean and dirty sector, I substitute labour input, equations \ref{eq:lab_inputc} and \ref{eq:lab_inputd}, in the sector-specific production functions, \ref{eqbm:outputc} and \ref{eqbm:outputd}. Exploiting demand for sector goods, $Y_d=\left(\frac{p_c}{p_d}\right)^\varepsilon Y_c$, and the price ratio in equilibrium pinned down by equation \ref{eq:price_ratio_labourinput} yields
%\begin{align}\label{eq:price_ratio_output}

%\end{align}
%Substituting equation \ref{eq:price_ratio_output} into equation \ref{eq:price_ratio_labourinput} determines the equilibrium ratio of low-skill input in the dirty to the clean sector: 
\begin{align}
	\frac{p_c}{p_d} =&\left(\frac{\gamma_d}{\gamma_c}\frac{A_d}{A_c}\frac{l_{ld}}{l_{lc}}\right)^{\frac{1-\alpha}{\alpha+\varepsilon(1-\alpha)}}\\ %& \text{(Demand for sector-specific goods)}\\
\Leftrightarrow\ 	\frac{l_{ld}}{l_{lc}}=&%\left(\frac{A_c}{A_d}\right)^{(1-\alpha)(1-\varepsilon)}\frac{\gamma_c}{\gamma_d}\left(\frac{z_d}{z_c}\right)^{\alpha+\varepsilon(1-\alpha)}\zeta^{(\theta_c-\theta_d)(\alpha+\varepsilon(1-\alpha))}\nonumber\\	=&
	\left(\frac{A_c}{A_d}\right)^{(1-\alpha)(1-\varepsilon)}\left(\zeta^{\theta_c-\theta_d}\frac{\gamma_c}{\gamma_d}\frac{z_d}{z_c}\right)^{\alpha+\varepsilon(1-\alpha)}\label{eq:lldllc}%\\
	%	\text{where}&\\
	%	\tilde{\chi}= &\  (\theta_c^{\theta_c}\theta_d^{-\theta_d})^{(1-\alpha) (1-\varepsilon)}(1-\theta_c)^{-\theta_c-(1-\theta_c)(\alpha+\varepsilon(1-\alpha))}(1-\theta_d)^{\theta_d+(1-\theta_d)(\alpha+\varepsilon(1-\alpha))}\nonumber
\end{align}

\paragraph{Skill supply}
Using equations \ref{eq:lld}, \ref{eq:llc}, and \ref{eq:lldllc} one can solve for $h_l$ as a function of total skill supply in equilibrium
\begin{align}
	h_l= \underbrace{\frac{(1-\theta_c)(1-\theta_d)\left[\left(\frac{A_c}{A_d}\right)^{(1-\alpha)(1-\varepsilon)}\zeta^{-(\theta_c-\theta_d)(1-\alpha)(1-\varepsilon)}\tilde{\chi}+1\right]}{(1-\theta_d)+(1-\theta_c)\left[\left(\frac{A_c}{A_d}\right)^{(1-\alpha)(1-\varepsilon)}\zeta^{-(\theta_c-\theta_d)(1-\alpha)(1-\varepsilon)}\tilde{\chi}\right]}}_{:=\tilde{\kappa}}H
\end{align}
Now, one can solve for labour input and sector-specific output as a function of tax progessivity  in equilibrium. 
$L_c$ and $L_d$ are
\begin{align}
	L_c&= \gamma_c \chi \left(1-\frac{\tilde{\kappa}}{1-\theta_d}\right)H\\
	L_d&= \gamma_d \chi \left(\frac{\tilde{\kappa}}{1-\theta_c}-1\right)H=\zeta^{-\theta_d}z_dp_d^{1-\varepsilon}H
\end{align}
This solves the model, since, in equilibrium,  $H$ is a function of parameters and policy variables only. 
Replacing dirty labour input and machines in dirty production leads to the expression for dirty output growth used in the text. 
\begin{comment}
\paragraph{Summary of equilibrium equations}
\begin{align*}
H=\ & (1-\tau_l)^{\frac{1}{1+\sigma}}\\
h_l=\ & \tilde{\kappa}H\\
L_c=\ & \gamma_c \chi \left(1-\frac{\tilde{\kappa}}{1-\theta_d}\right)H
\\
L_d=\ & \gamma_d \chi \left(\frac{\tilde{\kappa}}{1-\theta_c}-1\right)H%= \zeta^{-\theta_d}z_d\frac{\left(A_cz_c\zeta^{\theta_d}\right)^{(1-\alpha)(1-\varepsilon)}}{\left(A_dz_d\zeta^{\theta_c}\right)^{(1-\alpha)(1-\varepsilon)}+\left(A_cz_c\zeta^{\theta_d}\right)^{(1-\alpha)(1-\varepsilon)}}H
=\zeta^{-\theta_d}z_dp_d^{1-\varepsilon}H\\
p_d=\ &\left(\frac{\left(A_cz_c\zeta^{\theta_d}\right)^{(1-\alpha)(1-\varepsilon)}}{\left(A_dz_d\zeta^{\theta_c}\right)^{(1-\alpha)(1-\varepsilon)}+\left(A_cz_c\zeta^{\theta_d}\right)^{(1-\alpha)(1-\varepsilon)}}\right)^{\frac{1}{1-\varepsilon}} \\
p_c=\ & \left(\frac{\left(A_dz_d\zeta^{\theta_c}\right)^{(1-\alpha)(1-\varepsilon)}}{\left(A_dz_d\zeta^{\theta_c}\right)^{(1-\alpha)(1-\varepsilon)}+\left(A_cz_c\zeta^{\theta_d}\right)^{(1-\alpha)(1-\varepsilon)}}\right)^{\frac{1}{1-\varepsilon}}
\end{align*}

content...
\end{comment}
\section{Balanced Growth Path}

The model features structural transformation stemming from price effects (\cite{Ngai2007StructuralGrowth}, Baumol (1967)), since heterogeneous growth rates result in relative price changes over time. %A shown by \cite{Ngai2007StructuralGrowth}, the model features a balanced-growth path with certain parameter values: 
For certain parameter values the model exhibits a generalised balanced growth path\footnote{\ 
In contrast to a balanced growth path, which is commonly defined by constant growth in all variables, a GBGP is less strict and certain variables are allowed to grow at non-constant rates. The literature on structural transformation commonly reverts to this concept as transitions across sectors are essential to this literature.}.
\cite{Ngai2007StructuralGrowth} show that with goods being complements, employment shares shift to sectors with lower TFP growth; eventually, all labour is in the sector with the lowest TFP. In the present model, this is the clean sector. 

\section{Model Isomorphic to model with investment and rented capital}
The model is isomorphic to a model with (instantaneously productive) investment and full depreciation: 
\begin{align*}
I_t&=\psi(x_{dt}+x_{ct})\\
(LOM capital) \ K_t&=I_t= I_{ct}+I_{dt}
\end{align*}
That is, the capital good is produced by the following technology
\begin{align*}
x_{ijt}=\frac{I_{ijt}}{\psi}
\end{align*}
Machine producing firms rent the investment good, $I_{jt}$ and pay the real rate. They maximise over the choice of investment, i.e. capital, to borrow:
\begin{align*}
\underset{I_{ijt}}{\max}\hspace{2mm}p_{ijt}x_{ijt}-r_tI_{ijt}
\end{align*}
Profit maximisation of machine producing firms yields
\begin{align*}
\frac{p_{ijt}}{\psi}=r_t
\end{align*}
Free movement of capital and homogeneity of production costs imply that machine prices are equal across firms and sectors. 

Imposing market clearing for investment, $I_t=\int_{0}^{1}I_{idt}di+\int_{0}^{1}I_{ict}di$, and market clearing for machines yield a condition for the real rate in equilibrium
\begin{align*}
r_t=\alpha \psi^{-\alpha}\left(\frac{p_{dt}^{\frac{1}{1-\alpha}}A_{dt}L_{dt}+p_{ct}^{\frac{1}{1-\alpha}}A_{ct}L_{ct}}{K_t}\right)^{1-\alpha}
\end{align*}

\section{Results}
\begin{figure}[h!!]
	\centering
	\caption{Business as usual versus laissez-faire, substitutes, additional variables }\label{fig:onlyBAU_add}
	
	\begin{minipage}[]{0.32\textwidth}
		\centering{\footnotesize{(a) Clean output, $y_c$ }}
		%	\captionsetup{width=.45\linewidth}
		\includegraphics[width=1\textwidth]{../../codding_model/Own/figures/Rep_agent/staticBAU_LF_separate_yc_periods59_eppsilon4.00_zeta1.40_Ad08_Ac04_thetac0.70_thetad0.56_HetGrowth1_tauul0.181_util0_withtarget0_lgd0.png}
	\end{minipage}
	\begin{minipage}[]{0.32\textwidth}
		\centering{\footnotesize{(b) Dirty output, $y_d$}}
		%	\captionsetup{width=.45\linewidth}
		\includegraphics[width=1\textwidth]{../../codding_model/Own/figures/Rep_agent/staticBAU_LF_separate_yd_periods59_eppsilon4.00_zeta1.40_Ad08_Ac04_thetac0.70_thetad0.56_HetGrowth1_tauul0.181_util0_withtarget0_lgd0.png}
	\end{minipage}
	\begin{minipage}[]{0.32\textwidth}
		\centering{\footnotesize{(c) Labour input clean, $L_c$ }}
		%	\captionsetup{width=.45\linewidth}
		\includegraphics[width=1\textwidth]{../../codding_model/Own/figures/Rep_agent/staticBAU_LF_separate_Lc_periods59_eppsilon4.00_zeta1.40_Ad08_Ac04_thetac0.70_thetad0.56_HetGrowth1_tauul0.181_util0_withtarget0_lgd0.png}
	\end{minipage}
	\begin{minipage}[]{0.32\textwidth}
		\centering{\footnotesize{(d) Labour input dirty, $L_d$ }}
		%	\captionsetup{width=.45\linewidth}
		\includegraphics[width=1\textwidth]{../../codding_model/Own/figures/Rep_agent/staticBAU_LF_separate_Ld_periods59_eppsilon4.00_zeta1.40_Ad08_Ac04_thetac0.70_thetad0.56_HetGrowth1_tauul0.181_util0_withtarget0_lgd0.png}
	\end{minipage}
\begin{minipage}[]{0.32\textwidth}
	\centering{\footnotesize{(e) Machines clean, $x_c$}}
	%	\captionsetup{width=.45\linewidth}
	\includegraphics[width=1\textwidth]{../../codding_model/Own/figures/Rep_agent/staticBAU_LF_separate_xc_periods59_eppsilon4.00_zeta1.40_Ad08_Ac04_thetac0.70_thetad0.56_HetGrowth1_tauul0.181_util0_withtarget0_lgd0.png}
\end{minipage}
	\begin{minipage}[]{0.32\textwidth}
		\centering{\footnotesize{(f) Machines dirty, $x_d$}}
		%	\captionsetup{width=.45\linewidth}
		\includegraphics[width=1\textwidth]{../../codding_model/Own/figures/Rep_agent/staticBAU_LF_separate_xd_periods59_eppsilon4.00_zeta1.40_Ad08_Ac04_thetac0.70_thetad0.56_HetGrowth1_tauul0.181_util0_withtarget0_lgd0.png}
	\end{minipage}
\end{figure}

\begin{figure}[h!!]
	\centering
	\caption{Business as usual versus laissez-faire, complements, additional variables }\label{fig:onlyBAU_comp_add}
		\begin{minipage}[]{0.32\textwidth}
		\centering{\footnotesize{(a) Clean output }}
		%	\captionsetup{width=.45\linewidth}
		\includegraphics[width=1\textwidth]{../../codding_model/Own/figures/Rep_agent/staticBAU_LF_separate_yc_periods59_eppsilon0.40_zeta1.40_Ad08_Ac04_thetac0.70_thetad0.56_HetGrowth1_tauul0.181_util0_withtarget0_lgd0.png}
	\end{minipage}
	\begin{minipage}[]{0.32\textwidth}
		\centering{\footnotesize{(b) Dirty output }}
		%	\captionsetup{width=.45\linewidth}
		\includegraphics[width=1\textwidth]{../../codding_model/Own/figures/Rep_agent/staticBAU_LF_separate_yd_periods59_eppsilon0.40_zeta1.40_Ad08_Ac04_thetac0.70_thetad0.56_HetGrowth1_tauul0.181_util0_withtarget0_lgd0.png}
	\end{minipage}
	\begin{minipage}[]{0.32\textwidth}
		\centering{\footnotesize{(c) Labour input clean, $L_c$ }}
		%	\captionsetup{width=.45\linewidth}
		\includegraphics[width=1\textwidth]{../../codding_model/Own/figures/Rep_agent/staticBAU_LF_separate_Lc_periods59_eppsilon0.40_zeta1.40_Ad08_Ac04_thetac0.70_thetad0.56_HetGrowth1_tauul0.181_util0_withtarget0_lgd0.png}
	\end{minipage}
	\begin{minipage}[]{0.32\textwidth}
		\centering{\footnotesize{(d) Labour input dirty, $L_d$ }}
		%	\captionsetup{width=.45\linewidth}
		\includegraphics[width=1\textwidth]{../../codding_model/Own/figures/Rep_agent/staticBAU_LF_separate_Ld_periods59_eppsilon0.40_zeta1.40_Ad08_Ac04_thetac0.70_thetad0.56_HetGrowth1_tauul0.181_util0_withtarget0_lgd0.png}
	\end{minipage}
	\begin{minipage}[]{0.32\textwidth}
		\centering{\footnotesize{(e) Machines clean, $x_c$}}
		%	\captionsetup{width=.45\linewidth}
		\includegraphics[width=1\textwidth]{../../codding_model/Own/figures/Rep_agent/staticBAU_LF_separate_xc_periods59_eppsilon0.40_zeta1.40_Ad08_Ac04_thetac0.70_thetad0.56_HetGrowth1_tauul0.181_util0_withtarget0_lgd0.png}
	\end{minipage}
	\begin{minipage}[]{0.32\textwidth}
		\centering{\footnotesize{(f) Machines dirty, $x_d$}}
		%	\captionsetup{width=.45\linewidth}
		\includegraphics[width=1\textwidth]{../../codding_model/Own/figures/Rep_agent/staticBAU_LF_separate_xd_periods59_eppsilon0.40_zeta1.40_Ad08_Ac04_thetac0.70_thetad0.56_HetGrowth1_tauul0.181_util0_withtarget0_lgd0.png}
	\end{minipage}
\end{figure}

\begin{figure}[h!!]
	\centering
	\caption{Optimal allocation with emission target, complements, additional variables }\label{fig:optallo_comp_onlyR_add}
	\begin{minipage}[]{0.32\textwidth}
	\centering{\footnotesize{(a) Labour input clean, $L_c$ }}
	%	\captionsetup{width=.45\linewidth}
	\includegraphics[width=1\textwidth]{../../codding_model/Own/figures/Rep_agent/staticonlyRam_separate_Lc_periods59_eppsilon0.40_zeta1.40_Ad08_Ac04_thetac0.70_thetad0.56_HetGrowth1_tauul0.181_util0_withtarget1_lgd0.png}
\end{minipage}
	\begin{minipage}[]{0.32\textwidth}
		\centering{\footnotesize{(b) Labour input dirty, $L_d$ }}
		%	\captionsetup{width=.45\linewidth}
		\includegraphics[width=1\textwidth]{../../codding_model/Own/figures/Rep_agent/staticonlyRam_separate_Ld_periods59_eppsilon0.40_zeta1.40_Ad08_Ac04_thetac0.70_thetad0.56_HetGrowth1_tauul0.181_util0_withtarget1_lgd0.png}
	\end{minipage}
	\begin{minipage}[]{0.32\textwidth}
		\centering{\footnotesize{(c) Machines clean, $x_c$}}
		%	\captionsetup{width=.45\linewidth}
		\includegraphics[width=1\textwidth]{../../codding_model/Own/figures/Rep_agent/staticonlyRam_separate_xc_periods59_eppsilon0.40_zeta1.40_Ad08_Ac04_thetac0.70_thetad0.56_HetGrowth1_tauul0.181_util0_withtarget1_lgd0.png}
	\end{minipage}
\begin{minipage}[]{0.32\textwidth}
\centering{\footnotesize{(d) Machines dirty, $x_d$}}
%	\captionsetup{width=.45\linewidth}
\includegraphics[width=1\textwidth]{../../codding_model/Own/figures/Rep_agent/staticonlyRam_separate_xd_periods59_eppsilon0.40_zeta1.40_Ad08_Ac04_thetac0.70_thetad0.56_HetGrowth1_tauul0.181_util0_withtarget1_lgd0.png}
\end{minipage}
\begin{minipage}[]{0.32\textwidth}
	\centering{\footnotesize{(e) Price clean good, $p_c$}}
	%	\captionsetup{width=.45\linewidth}
	\includegraphics[width=1\textwidth]{../../codding_model/Own/figures/Rep_agent/staticonlyRam_separate_pc_periods59_eppsilon0.40_zeta1.40_Ad08_Ac04_thetac0.70_thetad0.56_HetGrowth1_tauul0.181_util0_withtarget1_lgd0.png}
\end{minipage}
	\begin{minipage}[]{0.32\textwidth}
		\centering{\footnotesize{(f) Price dirty good, $p_d$}}
		%	\captionsetup{width=.45\linewidth}
		\includegraphics[width=1\textwidth]{../../codding_model/Own/figures/Rep_agent/staticonlyRam_separate_pd_periods59_eppsilon0.40_zeta1.40_Ad08_Ac04_thetac0.70_thetad0.56_HetGrowth1_tauul0.181_util0_withtarget1_lgd0.png}
	\end{minipage}
	\begin{minipage}[]{0.32\textwidth}
	\centering{\footnotesize{(g) $\lambda$}}
	%	\captionsetup{width=.45\linewidth}
	\includegraphics[width=1\textwidth]{../../codding_model/Own/figures/Rep_agent/staticonlyRam_separate_lambdaa_periods59_eppsilon0.40_zeta1.40_Ad08_Ac04_thetac0.70_thetad0.56_HetGrowth1_tauul0.181_util0_withtarget1_lgd0.png}
\end{minipage}
\end{figure}



\section{Skill supply}
\paragraph{Effect of $\tau_l$ on skill investment}
From the definition of $H$ it has to hold that 
\begin{align}
&1=\frac{dh_l}{dH}+\zeta \frac{dh_h}{dH}\label{eq:ident} \\
\Leftrightarrow\ & \frac{dh_h}{dH}=\frac{1-\frac{dh_l}{dH}}{\zeta}.\label{eq:resp}
\end{align}
Using this equation, one can show that high skill supply is relatively more responsive to changes in total effective hours worked, i.e.,  $\frac{dh_h}{dH}>\frac{dh_l}{dH}$, if one excludes the case that high skill supply reduces as effective hours increase.\footnote{\ Proof: Suppose   $\frac{dh_h}{dH}>0$. Now, assume by contradiction that low skill supply is relatively more responsive. Hence, $\frac{dh_h}{dH}<\frac{dh_l}{dH}$. Using equation \ref{eq:resp}, one gets that $\frac{dh_l}{dH}>1+\zeta$. Replacing this inequality in the identity \ref{eq:ident}, it follows that $0>\zeta[1+\frac{dh_h}{dH}]$. Since $\zeta>1$ by assumption, it has to hold that $\frac{dh_h}{dH}<-1$ which contradicts the premise that $\frac{dh_h}{dH}>0$. } Thus, as the household reduces total effective hours supplied, the reduction in high skilled hours is higher. \tr{This should be due to the marginal utility from less high skill is higher than from less low skill hours.} This should show up in general equilibrium effects... but relative wages are fixed. 


\section{Calculations, partially wrong}
\noindent\rule[1ex]{\textwidth}{1pt}

\paragraph{Progressivity and emission targets}\tr{This result also rests on wrong premisses, but needs to be replicated!}
The constraint on emissions in the government's objective function implies that $Y_{dt}=\frac{\delta}{\kappa}$, thus, $(1+g_{ydt})=1$, for all time periods starting from 2050, $t\geq 30$. 

From the dirty sector's production function and equation \ref{eq:inf_d} we have that
\begin{align}
&\frac{Y_{d}'}{Y_d}=(1+\pi_d)^{\frac{\alpha}{1-\alpha}}(1+\upsilon_{d})\label{eq:gyd}\\
\Leftrightarrow\ &(1+\upsilon_{d})^{\frac{(1-\alpha)(1-\tau_l-\varepsilon)}{(1-\tau_l)-(\varepsilon(1-\alpha)+\alpha)}}=1\label{eq:def_taul}
\end{align}
The inflation rate in equation \ref{eq:gyd} captures the role of machine demand by the dirty sector. When the price at which dirty firms can sell their output is high, they demand more machines. A positive inflation, therefore, implies a rise in dirty output.

At the same time, a rise in the dirty good's price reduces demand. This counteracting mechanism is accounted for in equation \ref{eq:def_taul}. 
As will be shown below, this mechanism ensures that the government can target dirty sector production through tax progressivity. 

First, I establish an optimal policy result. Assume that the government cannot set the growth rate in the dirty sector, then equation \ref{eq:def_taul} defines $\tau_l$ on a balanced growth path.

\begin{prop}[Optimal tax progressivity]
	Assume growth of the dirty technology, $\upsilon_{d}$, is exogenously determined. 
	Then, to comply with the Paris Agreement, the government has to set the tax progressivity parameter, $\tau_l$, to $\tau^*_l=1-\varepsilon$ for $\varepsilon\neq 1$ (as otherwise the exponent in \ref{eq:def_taul} is not defined under the optimal tax rate.).
	When goods are complements, the optimal tax system is progressive. If goods are substitutes, the optimal tax system is regressive.
\end{prop}

The intuition is, that by choosing tax progressivity, the government affects price inflation in the dirty sector; compare equation \ref{eq:inf_d}. 
The result implies that inflation in the dirty sector under the optimal policy is negative when there is positive growth in dirty technology. The demand for machines has to decline by the same rate as technology growths for dirty output to be constant.  

Can this be an equilibrium as the price for dirty products declines?

How does tax progressivity affect inflation? 
First note that at a flat tax, the inflation rate is independent of the sector-specific technological growth rate. This is due to offsetting mechanisms.\tr{Continue}


\tr{Start from effect on HH:} 
(1) A rise in $\tau_l$ reduces disposable income and aggregate demand falls. This is a mechanical result from a higher tax rate, and a reduction in aggregate hours supplied.  
(1) For the dirty sector to demand labour, the costs of the labour input good has to balance its marginal product which positively depends on technological progress and the sector specific price. 

\noindent\rule[1ex]{\textwidth}{1pt} 


\noindent\rule[1ex]{\textwidth}{1pt}

\textcolor{blue}{below is wrong since the aggregate price level is not constant when G is disposed off.}
Define sector-specific inflation as: $1+\pi_{j}=\frac{p'_j}{p_j}$.
Using the definition of the aggregate price level, final good production, and optimality conditions in the clean sector, one can show that 
\begin{align}\label{eq:agg_supply}
\frac{Y'}{Y}= (1+\pi_c)^{\frac{\varepsilon(1-\alpha)+\alpha}{1-\alpha}}(1+\upsilon_{c}), \hspace{3mm} \text{(Supply side)}
\end{align}
since $\frac{L_{c}'}{L_c}=1$.
Using goods market clearance, the budget condition, and the FOC for total skill supply, it follows that 
\begin{align}\label{eq:agg_demand}
\frac{Y'}{Y}= \left(\frac{w'_h}{w_h}\right)^{1-\tau_l}. \hspace{3mm} \text{(Demand side)}\tr{\text{wrong, misses gov expenditures... or let lambdaa adjust, and machine production}}
\end{align}
Demand for the labour input good implies that 
\begin{align}\label{eq:labour income}
\frac{p'_{cL}}{p_{cL}}= (1+\pi_c)^\frac{1}{1-\alpha}(1+\upsilon_{c})
\end{align}
(independent of growth in $L_c$).

Multiplying both sides with $\left(\frac{w_h'}{w_h}\right)^{-1}$, using equation \ref{eq:agg_growth}, and that $\frac{p_{cL}}{w_h}$ is constant, it follows that 

\begin{align}
\frac{\frac{p'_{cL}}{w'_h}}{\frac{p_{cL}}{w_h}}= (1+\pi_c)^\frac{1}{1-\alpha}(1+\upsilon_{c})\left(\frac{Y'}{Y}\right)^{-\frac{1}{1-\tau_l}}=1.
\end{align}

Above equation determines inflation in the clean sector:
\begin{align}\label{eq:inf_c}
1+\pi_c=(1+\upsilon_{c})^{\frac{\tau_l(1-\alpha)}{(1-\tau_l)-\varepsilon(1-\alpha)-\alpha}}.
\end{align}
By symmetry of (i) how goods enter the production fo the final good and of (ii) sectors, it also holds that 
\begin{align}\label{eq:inf_d}
1+\pi_d=(1+\upsilon_{d})^{\frac{\tau_l(1-\alpha)}{(1-\tau_l)-\varepsilon(1-\alpha)-\alpha}}.
\end{align}


\noindent \textbf{(Aggregate output result, less relevant for main story)}

Hence, 
\begin{align}\label{eq:agg_growth}
(1+g_y)=\frac{Y'}{Y}=(1+\upsilon_{c})^\frac{(1-\tau_l)[1-(\varepsilon(1-\alpha)+\alpha)]}{(1-\tau_l)-(\varepsilon(1-\alpha)+\alpha)}.
\end{align}
Equation \ref{eq:agg_growth} implies the following proposition:
\begin{prop}[aggregate growth]
	\textit{For a proportional tax system, $\tau_l=0$, aggregate growth equals growth in the clean sector. 
		When the tax system is progressive\footnote{\ In the sense defined in \cite{Heathcote2017OptimalFramework}.}, $\tau_l>0$, then aggregate growth exceeds technology growth in the clean sector. When the tax rate is regressive, $\tau_l<0$, aggregate growth is smaller than technology growth in the clean sector. }
\end{prop}
\tr{Have to understand why.} 
With a flat tax system there is no inflation in the clean sector; compare equation \ref{eq:inf_c}. When the tax system is progressive, ...

\paragraph{Proof: labour input good constant}
%\textit{Check that the labour input good is constant:} 

First note that $\frac{l_{hc}}{l_{lc}}$ is constant over time. 
From the FOC governing high skill demand in the clean sector and equation \ref{eq:constant} we have:

\begin{align*}
\frac{l_{hc}}{l_{lc}}=\left(\frac{p_{cL}}{w_h}\theta_c\right)^{\frac{1}{1-\theta_c}}= constant.
\end{align*}

Substitution into the production function of the clean labour input good yields

\begin{align*}
\frac{L'_c}{L_c}=\frac{l_{lc}'}{l_{lc}}.
\end{align*}

\tr{To be continued.}

\textbf{\tr{To be shown next:  How $\tau_l$ affects (1) skill supply (level) and (2) externality. }}
\\

\paragraph{Overview BGP compatible growth rates}
\begin{align}
\frac{Y'}{Y}\\
\frac{w_h'}{w_h}=\frac{w_l'}{w_l}\\
tbc
\end{align}
\textbf{Conditions for BGP to exist}
Next to assuming no transition of labour input goods across sectors, a joint condition on tax progressivity and substitutability of sector goods ensures that output growth in both sectors is positive which has to be the case as otherwise production of one good tends to zero which cannot be an equilibrium when goods are no perfect substitutes.
Hence, on a BGP it has to hold that 
\begin{align*}
\frac{(1-\alpha)(1-\tau_l-\varepsilon)}{(1-\tau_l)-(\varepsilon(1-\alpha)+\alpha)}>0.
\end{align*}
There are two possible ranges of parameter values reads
\begin{align*}
\text{either}\\	&(1-\tau_l)>\varepsilon\hspace{4mm}&\text{if goods are substitutes} ;\\ \text{and}\ \ & (1-\tau_l)>\varepsilon(1-\alpha)+\alpha\hspace{4mm}&\text{if goods are complements}\\
\text{or}\\	&(1-\tau_l)<\varepsilon\hspace{4mm}&\text{if goods are complements} ;\\ \text{and}\ \ & (1-\tau_l)<\varepsilon(1-\alpha)+\alpha\hspace{4mm}&\text{if goods are substitutes}
\end{align*}
Focus on the case that goods are substitutes. Then the condition in the \textit{either}-statement prevents the government from choosing a progressive tax system, since $\tau_l<1-\varepsilon<0$.
Analogously, when goods are complements, the \textit{or}-statement excludes regressive tax systems.
I, therefore, ensure that when goods are complements, it holds that $\tau_l<(1-\alpha)(1-\varepsilon)$. When they are substitutes, it holds that $\tau_l>(1-\alpha)(1-\varepsilon)$.


\tr{Remaining problem: prices are not constant on BGP with fixed growth rates...Look at literature on positive trend inflation...}

\begin{comment}
\textbf{Below wrong Because wrong hl used}
From here,  equilibrium conditions determine prices $p_{dL}, p_{cL}$. Using \ref{eq:constant} skill wages follow. Together with the FOC on hours supply, wages determine aggregate demand. Imposing goods market clearing and using equations \ref{eq:lab_inputc} and \ref{eq:lab_inputd}, determines low skill hour demand in equilibrium.

\begin{align*}
h_l=\left( \frac{1}{\left(\frac{\alpha}{\psi}\right)^{\frac{\alpha}{1-\alpha}}\left[\left(p_c^\frac{\alpha}{1-\alpha}\chi_c A_c\right)^\frac{\varepsilon-1}{\varepsilon}+\left(p_d^\frac{\alpha}{1-\alpha}\chi_d A_d\right)^\frac{\varepsilon-1}{\varepsilon}\right]^\frac{\varepsilon}{\varepsilon-1}}\right)\ \lambda \left(H w_l\right)^{1-\tau_l}.
\end{align*}

Knowing $h_l$, the variables $L_c, \ L_d, \ h_h, \ l_{lc}, l_{ld}, l_{hc}, l_{hd}$ follow. 

Output of the clean and dirty sector read
\begin{align}
%Y_d& =  \frac{\chi_d A_d}{\left[\left(\left(\chi_d A_d\right)^\frac{\alpha}{\alpha+\varepsilon(1-\alpha)}\left(\chi_c A_c\right)^\frac{\varepsilon(1-\alpha)}{\alpha+\varepsilon(1-\alpha)}\right)^\frac{\varepsilon-1}{\varepsilon}+\left(\chi_d A_d\right)^\frac{\varepsilon-1}{\varepsilon}\right]^\frac{\varepsilon}{\varepsilon-1}} \lambda (H w_l)^{1-\tau_l}\\
&Y_d = \left(\frac{1}{\left(\frac{\chi_c A_c}{\chi_d A_d}\right)^{\frac{(\varepsilon-1)(1-\alpha)}{\alpha+\varepsilon(1-\alpha)}}+1}\right)^\frac{\varepsilon}{\varepsilon-1}\lambda (H w_l)^{1-\tau_l}\\
& Y_c= \left(\frac{1}{1+\left(\frac{\chi_d A_d}{\chi_c A_c}\right)^{\frac{(\varepsilon-1)(1-\alpha)}{\alpha+\varepsilon(1-\alpha)}}}\right)^\frac{\varepsilon}{\varepsilon-1}\lambda (H w_l)^{1-\tau_l}.
\end{align}

The government can affect dirty production by lowering aggregate demand. Note that $\chi_c,\ \chi_d$ are functions of the disutility from high skill labour supply, $\zeta$. As a result, the elasticity of diryt and clean output to tax progressivity is asymmetric.

\end{comment}<- Important notes on nature etc
%\section{Guideline Computations}
\begin{enumerate}
	\item calibrate initial situation to data, using observed tax rates (this would be a competitive equilibrium with taxes as given)
	\item find BGP; BGP exists when $c^*_s+c_n^*\geq \bar{c}$; that is optimal allocation without penalty satisfies basic needs; from this point onwards there are no reallocations across sectors and sectors grow at a constant rate, this is equivalent to the solution of the problem without penalty term \ar \textbf{Need to solve Ramsey problem for BGP absent penalty term}
	Could also solve for the BGP in Ramsey model numerically: get model equations and set growth corrected variables to constant values\\
	Follow \cite{Jones1993OptimalGrowth}:
	\begin{enumerate}
		\item fix assumed SS tax rates and transfers relative to output
		\item calculate ss values of consumption/output, other variables relative to output (constant in ss)
		\item make end corrections to Ramsey problem which is explicitly solved up to period T given the values from point 1 and 2 above
		\item iterate until guess in 1 matches with solution for ss value 
	\end{enumerate}
end corrections are derived analytically.
	\item for competitive equilibrium follow \cite{Acemoglu2008CapitalGrowth}: 
	\begin{enumerate}
		\item analytically or numerically calculate BGP values
		\item initial values and parameters match to data (including tax rates and transfers)
		\item use shooting (or relaxation) algorithm to find solution, i.e. sequence of allocations that solve two boundary value problem: shooting algorithm finds initial conditions that s.t. ss values are matched. 
		\item proof uniqueness of transition path? 
	\end{enumerate}
	
	they write \begin{quote}
		The previous subsection demonstrated that there exists a unique CGP with  nonbalanced  sectoral  growth;  that  is,  there  is  aggregate  output growth at a constant rate together with differential sectoral growth and reallocation of factors of production across sectors. We now investigate whether the competitive equilibrium will approach the CGP. 
	\end{quote} 
\ar From where my economy starts today with calibrated ws, and taxes (a constant growth path), does it converge to a new constant growth path where ws is higher? 
\item in my model growth rates of sectoral consumption are not constant over time! Households reallocate shares as they get richer
\end{enumerate}

Simple version without growth
\begin{enumerate}
	\item economy is in SS today, as households income does not change their consumption is fixed, period t=0
	\item then in period t=1 ws rises (1)\ar what is the optimal policy when ws rises starting from calibrated tax rates; (2)\ar what is the new ss and how does the economy converge?
	\item[\ar] I know initial and end conditions, the shock system is also known a priori, then use shooting or relaxation algorithm to calculate transition
\end{enumerate}
%-------------------------------------
\clearpage
\bibliography{../../../bib_2_0}
\addcontentsline{toc}{section}{References}
\end{document}