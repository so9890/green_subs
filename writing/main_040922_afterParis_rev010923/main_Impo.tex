\documentclass[12pt]{article}
\usepackage[utf8]{inputenc}
\usepackage{xcolor}
\usepackage{graphicx}
\usepackage{listings}
\usepackage{epstopdf}
\usepackage{etoc}
\usepackage{pdfpages}
\usepackage[capposition=top]{floatrow}
\usepackage{pdflscape} % landsacpe package
% set font to times
%\usepackage{mathptmx} % times!!! 
%\usepackage[T1]{fontenc}
\usepackage{amsmath}
\usepackage{amsthm}
\usepackage{soul}
\usepackage[left=2.5cm, right=2.5cm, top=2.5cm, bottom =2.5cm]{geometry}
\usepackage{natbib}
\usepackage{etoolbox}
\newif\ifabbreviation
\pretocmd{\thebibliography}{\abbreviationfalse}{}{}
\AtBeginDocument{\abbreviationtrue}
\DeclareRobustCommand\acroauthor[2]{%
	\ifabbreviation #2\else #1 (\mbox{#2})\fi}

%\usepackage[natbibapa]{apacite}
%\usepackage{apacite}
%\bibliographystyle{apacite}
\bibliographystyle{apa}
%\renewcommand{\footnotesize}{\fontsize{10pt}{11pt}\selectfont}
\usepackage[doublespacing]{setspace}
\usepackage{listings}
\renewcommand{\figurename}{\textbf{Figure}}
\renewcommand{\hat}{\widehat}
\usepackage[bf]{caption}
\usepackage{subcaption}
\usepackage{tikz}
%\begin{comment}
%\usepackage[headsepline,footsepline]{scrlayer-scrpage} % has to come before package!!! otherwise option clash
%\usepackage{scrlayer-scrpage}
%\pagestyle{scrheadings} % kopfzeile/ fußzeile
%\clearpairofpagestyles
%\ohead{}
%\ihead{\textit{Redistribution, Demand and  Sustainable Production}}
%\cfoot{\thepage}
%\pagestyle{plain} % comment this one to have header
%\end{comment}
\allowdisplaybreaks
\usepackage{comment}
 \usepackage{siunitx}
  \usepackage{textcomp}
\definecolor{sonja}{cmyk}{0.9,0,0.3,0}
%\definecolor{purple}{model}{color-spec}
\usepackage{amssymb}
\newcommand{\ar}{$\Rightarrow$ \ }
\newcommand{\frp}[2]{\frac{\partial{#1}}{\partial{#2}}}
\newcommand{\tr}[1]{\textcolor{red}{#1}}
\newcommand{\vlt}[1]{\textcolor{violet}{#1}}
\newcommand{\bl}[1]{\textcolor{blue}{#1}}
\newcommand{\sn}[1]{\textcolor{sonja}{#1}}
%%% TIKZS
\usepackage{tikz}
\usetikzlibrary{mindmap,trees}
\usetikzlibrary{backgrounds}
\tikzstyle{every edge}=  [fill=orange]  
\usetikzlibrary{tikzmark}
\usetikzlibrary{decorations.markings}
\usepackage{tikz-cd}
\usetikzlibrary{arrows,calc,fit}
\tikzset{mainbox/.style={draw=sonja, text=black, fill=white, ellipse, rounded corners, thick, node distance=5em, text width=8em, text centered, minimum height=3.5em}}
\tikzset{mainboxbig/.style={draw=sonja, text=black, fill=white, ellipse, rounded corners, thick, node distance=5em, text width=13em, text centered, minimum height=3.5em}}
\tikzset{dummybox/.style={draw=none, text=black , rectangle, rounded corners, thick, node distance=4em, text width=20em, text centered, minimum height=3.5em}}
\tikzset{box/.style={draw , rectangle, rounded corners, thick, node distance=7em, text width=8em, text centered, minimum height=3.5em}}
\tikzset{container/.style={draw, rectangle, dashed, inner sep=2em}}
\tikzset{line/.style={draw, very thick, -latex'}}
\tikzset{    pil/.style={
		->,
		thick,
		shorten <=2pt,
		shorten >=2pt,}}
	
% other stuff
\newcommand{\innermid}{\nonscript\;\delimsize\vert\nonscript\;}
\newcommand{\activatebar}{%
	\begingroup\lccode`\~=`\|
	\lowercase{\endgroup\let~}\innermid 
	\mathcode`|=\string"8000
}
%\usepackage{biblatex}
%\addbibresource{bib_mt.bib}
\usepackage{ulem}
\title{Reconciling Political Feasibility and Efficiency of Climate Policies}
%\title{The Environment, Inequality, and Growth\\ \small{ optimal fiscal policy in an endogenous growth model with inequality and emission targets}}
\author{Sonja Dobkowitz
\thanks{\ {DIW Berlin, sdobkowitz@diw.de, \url{https://sonjadobkowitz.wordpress.com/}.} \newline
	I am deeply grateful to my supervisors Keith Kuester, Pavel Brendler, and Farzad Saidi for their support and guidance. 
	I would like to thank Philippe Aghion, Christian Bayer, Gregor Boehl, Thomas Hintermaier,  Moritz Kuhn, Katrin Millock, Hélène Ollivier,  Janosch Brenzel-Weiss, Rubén Domínguez-Díaz, Paul Schäfer, and Fabian Schmitz for helpful discussions and comments.  
	I further thank participants at the PSE Summer School on Climate Change 2022, the RTG-2281 Retreat 2022, the Bonn Macro Lunch Seminar Winter 2021.
	Financial support by the German Research Foundation (DFG) through RTG-2281 “The macroeconomics of inequality” is gratefully acknowledged.}}
	%University of Bonn\\
%Preliminary and incomplete\\
%First version: January 9, 2022\\
\date{\today %\\ \vspace{1mm} \small{Find the latest version  \href{https://sonjadobkowitz.wordpress.com/job-market-paper/}{here}}
}
\usepackage{graphicx,caption}
%\usepackage{hyperref}
\usepackage[colorlinks,linkcolor=aaltoblue,citecolor=aaltoblue,urlcolor=aaltoblue,unicode=true]{hyperref} %can create hyperlinks. ALWAYS LOAD LAST
\definecolor{aaltoblue}{RGB}{0,94,184}
\usepackage{minitoc}
\setcounter{secttocdepth}{5}
\usetikzlibrary{shapes.geometric}

% for tabular

%\usepackage{array}
\usepackage{makecell}
\usepackage{multirow}
\usepackage{bigdelim}

%propositions etc
\newtheorem{prop}{Proposition}
\newtheorem{corollary}{Corollary}[prop]
\newtheorem{lemma}[prop]{Lemma}

\renewenvironment{abstract}
{\small
	\list{}{
		\setlength{\leftmargin}{0.025\textwidth}%
		\setlength{\rightmargin}{\leftmargin}%
	}%
	\item\relax}
{\endlist}

\begin{document}
%	\includepdf[pages=-]{../titlepage.pdf}
	\maketitle
	\begin{abstract}
		\begin{singlespacing}
			\textbf{Abstract \ }
			Carbon taxes and lump-sum transfers allow for an efficient externality mitigation. However, rebating carbon tax revenues lump sum is unpopular. Households prefer to use carbon tax revenues to subsidize the green sector \citep{Fabre2023Fighting}.
			I show in a simple two-sector model that when carbon tax revenues are not rebated lump sum labor supply is inefficiently high. 
Labor income taxes, thus, qualify as an instrument to reduce this inefficiency.
Using a model of directed technical change, I quantify the optimal policy to meet emission targets. The government has a (i) carbon tax with revenues earmarked for subsidies to the green sector and a (ii) labor income tax available. Throughout the transition, the politically feasible climate policy is best accompanied by a marginal labor tax between 6\% and 9\%.
			%			Despite a compositional effect of tax progressivity in favor of fossil energy,
			%			 the optimal income tax scheme is progressive.
			%			
		%	I show analytically that absent lump-sum transfers, labor supply is inefficiently high.  In the quantitative part of the paper, I study policies targeted at the level of production to accompany socially acceptable yet inefficient climate policies.
			
%			This paper studies the optimal fiscal mix between taxes on carbon and labor income in a model of directed technical change. Carbon taxes direct production toward cleaner activity. Labor taxes reduce production overall. I find that diminishing production via income taxes allows for a lower carbon tax which is in particular costly when the polluting sector is more productive. This policy, however, is only optimal if knowledge spillovers foster green growth. 
%			
%			With endogenous growth but without knowledge spillovers: higher utility from lo
			%; (ii) taxing income lowers work effort when lump-sum transfers are infeasible. Lump-sum rebates of carbon tax proceeds would reduce hours worked to the efficient level. 
%			
%			Some scholars argue for limiting consumption to handle tightening environmental constraints \citep{Schor2005SustainableReductionb, VanVuuren2018AlternativeTechnologies}. Can reductive measures help mitigate environmental externalities?
%			 I show that an environmental tax does not suffice to implement the efficient allocation. Instead, the optimal policy also contains a reductive element: lump-sum transfers.
%			 When such
%			  transfers are  infeasible, the planner uses progressive labor income taxes to decrease work effort. 
%			I quantify the optimal tax progressivity in an endogenous growth model with skill heterogeneity.
%			Despite a compositional effect of tax progressivity in favor of fossil energy,
%			 the optimal income tax scheme is progressive.
%			 Next to the benefits from more leisure, the income tax partly substitutes environmental taxes leading to  higher technology growth today. 
%			 But, substituting carbon taxes with progressive income taxes is only optimal if knowledge spillovers push green growth and mitigate fossil growth tomorrow. 

	\noindent \textit{JEL classification}:  H21, H23, H24, O33, Q54, Q55
			
		\end{singlespacing}
		
		\end{abstract}
	
\thispagestyle{empty}
%\tableofcontents
\clearpage
\setcounter{page}{1}
\section{Introduction}
%\tr{I show that carbon taxes are only efficient if lump-sum transfers are available.}

\begin{comment}
\tr{Think about:
	1) when labor income taxes are not used, then need to have  a higher environmental tax to meet emission limits? \ar Yes, because of advantageous level effect which outweighs recomposing effect of income tax.
	2) When staying at level optimal under the assumption of lump-sum redistribution, but then not redistributing, than absent labor income tax emissions are too high; by how much? Counterfactual}
	
	content...
	\end{comment}

The latest assessment report of the Intergovernmental Panel on Climate Change \citep{IPCC2022} highlights the urgency to reduce greenhouse gas emissions,%relative to the previous report from 2018 \citep{Rogelj2018MitigationDevelopment.}.
\footnote{ \  The report stresses the decreasing likelihood of meeting the Paris Agreement and limiting climate warming to 1.5°. The Paris Agreement of 2015 formulates clear political goals to mitigate climate change. Under this treaty, states have agreed on a legally binding maximum increase in temperature to well below 2°C, preferably 1.5° over pre-industrial levels, and the global community seeks to be climate-neutral in 2050  (see: \url{https://unfccc.int/process-and-meetings/the-paris-agreement/the-paris-agreement}). 
}
and some scholars have pointed to limiting consumption to handle environmental boundaries.\footnote{\ \cite{Schor2005SustainableReductionb} argues for the necessity to limit consumption in the global North through a reduction in working time. \cite{Arrow2004AreMuch} raise the question if today's consumption is too high from a sustainability perspective. \cite{Dasgupta2021}  argues for the impossibility of indefinite growth due to planetary boundaries  \citep{Rockstrom2009AHumanity}. %: acknowledging planetary boundaries, i.e., boundaries which define a state of nature in which humans can safely exist \citep{Rockstrom2009AHumanity}, and that production and consumption produce waste, infinite production would degrade nature in a way that production is impossible.
	 \cite{VanVuuren2018AlternativeTechnologies} study alternative mitigation pathways with lower demand %such as lower energy demand, lower appliance ownership, and meat consumption 
	 in an integrated assessment model motivated by seeking to reduce reliance on carbon capture and storage technologies which entail risks and compete for scarce land. \cite{Bertram2018TargetedScenarios} stress the importance to reduce demand for energy- and material-intense products to alleviate the trade-off between mitigating temperature rises  and the UN sustainability goals%(such as food security, biodiversity protection, and clean water)% (p.11: Shifting towards healthier diets and less energy-and material-intensive consumption patterns appearsto have greatest potential for reducing sustainabilityrisks along a wide range of dimensions)
	. } A reduction in work effort and consumption mitigates pollution by diminishing economic activity. Distortionary fiscal policies qualify as a reductive policy instrument to target the level of production.
However, the literature on environmental policy has focused on compositional policies: environmental taxes. %\citep{Fried2018ClimateAnalysis}. 
Given the exigency to act, this paper addresses the question whether fiscal policies can help meet climate targets. %Using analytical and quantitative methods, I show that reductive policies form part of the optimal environmental policy even absent an additional target.


%This is your core argument for why reductive policy measures may work, so you should mention above that you consider this possibility,  MACHE ICH DAS NICHT MIT DER rESEARCH QUESTION? suggested by its proponents (your "scholars" :-)), and actually show that it works.

In the first part of the paper, I show analytically that once 
labor supply is elastic, reductive policy measures optimally complement the environmental tax. 
The literature has established that, absent any other distortion, an environmental tax equal to the social cost of the externality implements the efficient allocation. 
%Environmental taxes are perceived as a cost-effective way to reduce emissions. 
I demonstrate that this result crucially depends on the use of lump-sum transfers to redistribute environmental tax revenues. Transfers reduce labor supply through an income effect. %Thus, indeed there is a role for reductive policy measures. 
%\textcolor{blue}{This is interesting independent of whether they are feasible or not. Could relate to the fact that there is a discussion how to use revenues. Yet, one might argue that we are always in a setting with distortionary labor income taxes; so that recycling lump-sum is never needed; numbers on size of expected revenues and government spending}
When environmental tax revenues are not redistributed lump sum, environmental taxes are optimally combined with progressive labor income taxes. The use of income taxes as a reductive policy measure is not directly targeted at the externality: the motive for labor taxation emerges from a distortion in labor markets as households feel poorer than the economy is.\footnote{\ This is a novel motive for the use of reductive policies adding to the arguments made in the literature listed in the previous footnote. These are: conflicts with other goals such as the UN sustainability goals, risks associated with carbon capture and storage technology, and planetary boundaries and limits to growth.} %Hence,  % to lower inefficiently high hours worked. 
% I show that redistributing environmental tax revenues through an income tax scheme allows to implement the efficient allocation. The optimal income tax scheme is progressive.
%the optimal environmental policy equalizes the distribution of income as  a side effect.
% The theoretical analysis forms the

In the second part, I scrutinize whether progressive income taxes remain optimal in an endogenous growth model with heterogeneous skills. The government cannot use lump-sum transfers but consumes environmental tax revenues.
 %In the spirit of \cite{Acemoglu2002DirectedChange}, directed technical change may intensify or mitigate these channels thr recomposition. %Second, an overall reduction in labor supply curbing production may lower general research incentives.
 % % more low skill supply, more fossil innovation, more fossil production, and higher low income \ar reduction in the wage premium! 
 The model suggests that the optimal income tax scheme is progressive. The benefits of labor taxation emerge from (i) more leisure and (ii) gains from knowledge spillovers.
 The latter advantage arises as a fossil tax directs research from the non-energy sector to the energy sector.  Energy and non-energy goods are complements in final good production. The literature on directed technical change has shown that when goods are sufficiently complementary a price effect dominates the direction of research \citep{Acemoglu2002DirectedChange, Acemoglu2012TheChange, Hemous2021DirectedEconomics}. Therefore, as the fossil tax makes energy more expensive, the higher price of energy goods pulls research to the energy sector.
 This effect of the environmental tax counters the intention to lower emissions and decreases knowledge spillovers from the non-energy sector. Knowledge spillovers from the non-energy sector, however, are especially valuable, as it is the biggest sector in terms of research processes. 

% 
% On the one hand, a skill bias documented for the green sector \citep{Consoli2016DoCapital} in combination with a relatively more elastic high-skill labor supply causes a higher tax progressivity to recompose the economic structure towards dirty production. On the other hand, a higher labor share in the fossil sector implies a recomposition of the economy towards green production.
%  The skill-recomposition channel dominates and is slightly amplified by a market size effect directing research towards the fossil sector. 
 
% labor income taxes are used to substitute for environmental taxes to realize the gains from knowledge spillovers. 
 
%\textit{I quantify the welfare gains of setting progressive income taxes to equal yyy in consumption equivalent measure. TO BE DONE  }

% relation to literature
I discuss briefly the most important contributions of the paper.
First, the results are relevant for the political and academic debate on how  to recycle environmental tax revenues. The paper points to the importance of lump-sum transfers within the optimal environmental policy as a reductive policy tool; an aspect which appears overlooked in the discussion.%\footnote{\ POLICY debate; \cite{Fried2018TheGenerations}}
When thinking about how to recycle environmental tax revenues other than by lump-sum transfers,  one should take into account alternative reductive tools such as progressive labor income taxes. 
If the reductive part of the environmental policy is neglected, environmental taxes have to be higher to meet emission limits, as I demonstrate in the quantitative exercise.

Second, the results contribute to the academic debate on the so-called \textit{weak double dividend} \citep[for example:][]{LansBovenberg1994EnvironmentalTaxation, LansBovenberg1996OptimalAnalyses}. The hypothesis posits that recycling environmental tax revenues to reduce preexisting tax distortions is advantageous to recycling  revenues as lump-sum transfers. The rationale is that transfers decrease labor supply thereby diminishing the tax base of the income tax. %A conflict between generating government funds and environmental protection arises. 
The findings in the present paper suggest a lower bound on the reduction in distortionary income taxes: when environmental tax revenues are not redistributed lump sum, some reduction in labor supply via distortionary income taxes is in fact efficient from an environmental policy perspective. %In other words, even if environmental tax revenues suffice to satisfy a government revenue requirement, there is a motive for progressive income taxation. 

Third, the findings are especially interesting as the provision of the environmental public good and equity have been perceived as competing targets in the literature. When the poor consume more of the polluting good, a corrective tax is regressive \citep{ Fried2018TheGenerations, Sager2019IncomeCurves}. % \textit{Metcalf 2007, Hassett 2009 as  in Fried 2018}. 
%Second and more indirectly, a fossil tax exerts efficiency costs by lowering labor efforts\footnote{\ The reduction in hours worked is per se not inefficient. The reduction in dirty production reduces the marginal product of labor, which might make a reduction efficient. However, when the government seeks to tax labor income using distortionary policy tools, the reduced labor supply diminishes the tax base of the labor tax making it more costly to redistribute.} which again raises the cost for the government to redistribute \citep{Dobkowitz2022}. 
In contrast to this literature, the present paper provides an argument for progressive income taxes under perfect income-risk sharing suggesting a double dividend of redistribution: equity and efficient externality mitigation. %: equity on the one hand and efficiency gains from less labor as part of the environmental policy.


\paragraph{Literature}


%Second, it connects to the literature connecting environmental and fiscal policies and how to recycle environmental tax revenues. Third, as the paper combines environmental and fiscal policies it naturally connects to the public finance literature. Finally, the results speak to the literature discussing inefficiently high production.
 
%\begin{itemize}
%	\item How to use environmental tax revenues \citep{Fried2018TheGenerations}
%	\item Optimal environmental policy \ar focuses on environmental taxes
%	\item weak double dividend
%	\item to be incorporated: \tr{\cite{Metcalf2003EnvironmentalPollution} why does he find that the optimal pigou tax equals first best when gov spending is satisfied with tax revenues? }
%	\\
%	Williams III: Welfare improvement with xxx \citep{Parry1999WhenMarkets} \tr{is this weak or strong dd?}
%\end{itemize}

The paper relates to four strands of literature. 
%---------------------------------------
%.. optimal environmental policy
%---------------------------------------
Firstly, the paper speaks to the literature on macroeconomic studies of environmental policies. Within this realm, the quantitative analysis connects in particular to the endogenous growth literature. 
In general, these papers focus on environmental taxation and analyze settings with inelastic labor supply so that there is no role for policies targeting the level of production. \cite{Golosov2014OptimalEquilibrium} investigate the optimal carbon tax in a dynamic stochastic general equilibrium model.  
\cite{Acemoglu2012TheChange} discuss with a tractable model of directed technical change limits and possibilities for growth. %They highlight the need for green research subsidies to foster green innovation in combination with carbon taxes to correct for the dynamic spillovers of green innovation not internalized by the research sector.
\cite{Fried2018ClimateAnalysis} extends the framework of the aforementioned paper to a quantitative model. My paper add to the latter an optimal dynamic policy analysis and elastic labor supply. % mainly by introducing cross-sectoral knowledge spillovers and diminishing returns to research. 
%She finds that emission limits can be met at a lower fossil tax when growth is endogenous. 


% OVERVIEW LITERATURE
% Acemoglu 2016 have lump-sum transfers and taxes
% Acemoglu Aghion 2012: lump-sum transfers, no optimal policy
%Golosov: hightlight the need of lump-sum transfers! but exogenous labor supply
% Therefore, the main finding of the present paper, the necessity of reductive policy measures to implement the efficient allocation, complements this literature. 
%Especially, when environmental tax revenues are not redistributed lump-sum in these papers, a variable labor supply would give an argument for labor income taxation. 


%\paragraph{Endogenous growth, elastic labor supply and optimal environmental policy}

\begin{comment}
3/09:	ADD WHEN INTRODUCING HETEROGENEOUS LABOR SUPPLY BAG IN
Secondly, staying within the field of endogenous growth, the paper connects to work examining the interaction of directed technical change and skill heterogeneity. \cite{Acemoglu2002DirectedChange} develops a theory to explain the positive correlation of skill supply and the skill premium: the higher supply of skilled labor raises incentives to innovate in the skill sector. \cite{Loebbing2019NationalChange} introduces fiscal policy into the model to investigate how the equalizing effect of redistribution is amplified through directed technical change.
My paper contributes to these two branches by integrating endogenous and heterogeneous skill supply in an environmental model of directed technical change. These ingredients enable me to analyze labor income taxes through the lens of environmental policies.  

content...
\end{comment}

 %Similar to my paper, a higher tax progressivity changes the relative supply of skills. As low-skill labor is in relative higher supply, low-skill-specific innovation depresses the wage distribution thereby contributing to equity. While the channels are comparable,
% I evaluate the effect of income tax progressivity and endogenous growth on emissions. 
%\cite{Hemous2021DirectedEconomics} provide an overview of models of directed technical change in environmental economics. They argue that a rise in the skill ratio directs innovation towards skill-intense technology when the high- and the low-skill output good are sufficient substitutes. Furthermore, when the two input goods are substitutes, the more advanced sector attracts more innovation. 


%\cite{Oueslati2002EnvironmentalSupply} studies the optimal environmental policy with elastic labor supply and endogenous growth. Yet, he allows for lump-sum transfers of environmental revenues. \textit{He should find something on reduction of hours}: No: capital is the only polluting factor, and labor is the clean factor of production.


%%%--------------------------------------------------------------------
% How to recycle environmental tax revenues: weak double dividend
%%%-------------------------------------------------------------------- 
\
% A big literature has examined potential benefits arising from corrective tax revenues to ameliorate fiscal distortions. The double-dividend literature is concerned with fiscal advantages arising from environmental tax revenues. My results speak directly to the weak double dividen hypothesis which 
%My paper most closely relates to the literature on the weak double-dividend
%\paragraph{Recycling environmental tax revenues}
%\tr{Read:\cite{Freire-Gonzalez2018EnvironmentalReview} yet on strong dd, I assume}
Secondly, this paper is not the first to integrate distortionary fiscal policies into the analysis of environmental policies. The literature discussing how to use environmental tax revenues generally assumes labor supply to be elastic and incorporates fiscal policies. 
The dominant focus of this literature is the weak double dividend of environmental taxes \citep[for instance,][]{Goulder1995EnvironmentalGuide, Bovenberg2002EnvironmentalRegulation, Barrage2019OptimalPolicy}: given an exogenous government funding constraint it is cost saving to recycle environmental tax revenues to lower distortionary labor income taxes as opposed to higher lump-sum transfers. The latter, so the rationale, decreases labor supply through an income effect thereby lowering the tax base of the labor income tax. %Consequently, it becomes more expensive for the government to generate revenues.
Therefore, this literature advocates recycling environmental tax revenues to reduce distortionary fiscal taxes as opposed to lump-sum rebates.
%With its quantitative part, my paper closely relates to \cite{Barrage2019OptimalPolicy} who examines the role of fiscal distortions emerging from an exogenous revenue constraint on the environmental policy in a quantitative framework. She also optimizes jointly over fiscal and environmental policy instruments, but her focus rests on the deviation of the optimal environmental tax from the social costs of carbon.

% my contribution : 1) lower bound on dist income tax; 2) motivation and role of income taxes
Relative to this literature, my paper's contribution is to discuss the existence of an upper bound on the reduction of distortionary income taxes: From an environmental policy perspective, some reduction in labor supply is in fact efficient. However, shrinking labor supply is generally perceived as an inefficiency in this literature and environmental taxes on its own as efficient. The importance of lump-sum transfers to implement the efficient allocation receives less attention.  %To the best of my knowledge, the papers theoretically discussing the weak double-dividend \citep{LansBovenberg1996OptimalAnalyses, Goulder1995EnvironmentalGuide} do not formally derive the result; a possible explanation for why the lower bound on distortionary tax reduction remained unnoticed. 
% The primary distinction of this paper and the weak double-dividend literature is the motive for income taxation. In this literature an exogenous funding constraint motivates the use of distortionary income taxes. In contrast, my model rationalizes a progressive income tax absent an exogenous revenue constraint arising in an otherwise equal set-up from the environmental externality per se as the environmental tax on its own does not establish the efficient allocation. 




 %This becomes clear when environmental tax revenues suffice to meet the government's funding constraint, then labor supply would be inefficiently high when the labor income tax is unused. 
%In contrast to the present paper, the double-dividend literature focuses on non-environmental cost advantages of environmental taxation either via interactions with other taxes and their bases or via their revenues. However, it remains unmentioned that under the assumption of elastic labor supply, which the literature necessarily assumes, the environmental tax alone is not efficient.

%----------------------------------------
%---- optimal revenue recycling 
%---- empirical and quantitative-------
%----------------------------------------
The question of how to use environmental tax revenues has seen a surge in interest recently and diverged from the prominence of fiscal advantages. 
 %They do not constitute a free lunch if efficiency is the goal.\footnote{\ Often, environmental taxes alone seem to be perceived as being able to implement the efficient allocation: \cite{LansBovenberg1999GreenGuide} writes "\textit{Environmental taxes are  generally  an  efficient  instrument  for  protecting  the  environment.}" thereby neglecting the role environmental tax revenue redistribution. Or "\textit{Establishing a price on carbon [...] is well understood to be the most efficient approach for reducing greenhouse gas emissions.}" \citep{Fried2018TheGenerations}. } 
The pros and cons of different recycling means is often assessed using inter- and within generational equity or political feasibility as value measures \citep{Carattini2018, Goulder2019IncomeGroups, VANDERPLOEG2022103966, Kotlikoff2021MakingWin, Carbone2013DeficitImpacts}. Building on the weak double-dividend literature, \cite{Fried2018TheGenerations} compare distinct recycling scenarios investigating the impact on inequality in an overlapping generations model. 
% They find lump-sum transfers to be preferred by the  living generation.
Using German data,   \cite{VANDERPLOEG2022103966} find an equity advantage of lump-sum transfers. % The authors suggest the government to split environmental tax revenues to both lower preexisting tax distortions and as lump-sum transfers. %The present paper employs the first-best allocation as a benchmark to assess distinct recycling methods.
My contribution to this debate is to point to lump-sum redistribution to constitute an integral part of an efficient pollution mitigation. % together with corrective taxes.
If carbon tax revenues are not rebated lump sum, a role for additional policy intervention emerges due to distortions in the labor market.

%\paragraph{Public finance}
Thirdly, the paper contributes to the public finance literature.
An equity-efficiency trade-off is central to this literature.  The benefits of labor taxes and progressivity arise, inter alia, from redistribution. %and from generating government revenues. 
%With concave utility specifications full redistribution is efficient. However, the optimal tax system does not feature full redistribution when labor supply is endogenous. Instead, redistribution is traded off against aggregate output as individuals reduce their labor supply and skill investment in response to labor income taxation 
\citep{Heathcote2017OptimalFramework, Conesa2009TaxingAll, Domeij2004OnTaxes}.
To this literature I add another motive for the use of distortionary fiscal policies: to reduce inefficiently high labor supply in the presence of environmental taxes. 
%One closely related work is \cite{Loebbing2019NationalChange} who studies optimal income taxation in a model of directed technical change. The redistributive effect of tax progressivity is amplified through a compression of the wage rate distribution \textit{to be continued}


Fourthly, the paper relates from its motivation and finding to the discussion on whether production levels are inefficiently high. 
%The finding relates to the literature discussing rationales for the usage of reductive policy measures. 
Other motives for the reduction of consumption arise from
envy \cite{Alvarez-Cuadrado2007EnvyHours}, or a positive externality of leisure \cite{Alesina2005WorkDifferent}. \cite{Arrow2004AreMuch} discuss whether there is a need to reduce consumption levels due to sustainability concerns. 
 The present paper contributes to this literature by identifying another reason for too high labor supply: The externality results from mitigating an environ- mental externality without lump-sum transfers.


\paragraph{Outline}
The remainder of the paper is structured as follows. Section \ref{sec:mod_an} presents the core model and the analytical results. In Section \ref{sec:model2}, I extend and calibrate the model to a quantitative framework.  Results are discussed in Section \ref{sec:res}. Section \ref{sec:con} concludes.

%The remainder of the paper is structured as follows. The next section \ref{sec:mod_an} presents a tractable model which is used to derive the analytical results in section \ref{sec:theory}. In section \ref{sec:model}, I extend the model to a quantitative framework and calibrate it. I present and discuss the quantitative results in section \ref{sec:res}. Section \ref{sec:con} concludes.
%\section{Literature}


\paragraph{Notes \cite{Consoli2016DoCapital}}
\begin{itemize}
	\item in contrast to previous studies, they focus on occupation-level task desciptions
	\item previous work more on the industry level to prox greenness of occupation 
	\item skill and human capital dimension and green jobs; previous work on the effect of environmental regulation abstracted from these quality dimensions
	\item main findings:
	\begin{enumerate}
		\item green occupations exhibit a stronger intensity of high-level cognitive skills (\ar same occupation green version higher cognitive skills)
		\item changing occupations (becoming greener) more formal education, more work experience, more on the job training
	\end{enumerate}
\item method: within SOC 3digit groups compare skill level of occupations identified as emerging due to enviornmental needs, and those jobs transitioning to green tasks
\item SOC3, e.g engineers
\item otherwise, findings could be driven by green jobs beeing differen broader categories. (which I would want to include)
\item \textbf{skill heterogeneity is driven by two dimensions: }
\begin{enumerate}
	\item green occupations cluster in \textbf{high-skill} (\textit{intensive in abstract skills"}) \textbf{macro-occupations} p.1051 (\ar differences in type of jobs); important: non-routine analytical and interactive skills; remaining in mid-skill occupational skills
	\item differences within job types: green version of \textbf{the same job is more skill intense} (this is what they study in this paper! )
\end{enumerate}
\item SOC2 not included occupations: agriculture, public sector
\item sort SOC2 occupations according to average tasks


\begin{itemize}
	\item non-routine analytical: \\
	4.A.2.a.4 (IM) Analyzing data or information
	4.A.2.b.2 (IM) Thinking creatively
	4.A.4.a.1 (IM) Interpreting the meaning of information for others
	\item Non-routine interactive (NRI)
\\
	4.A.4.a.4 (IM) Establishing and maintaining interpersonal relationships
	4.A.4.b.4 (IM) Guiding, directing, and motivating subordinates
	4.A.4.b.5 (IM) Coaching and developing others
	\item Routine cognitive (RC)
\\
	4.C.3.b.4 (CX) Importance of being exact or accurate
	4.C.3.b.7 (CX) Importance of repeating same tasks
	4.C.3.b.8 (CX, reverse) Structured versus unstructured work
	\item Routine manual (RM)
\\
	4.A.3.a.3 (IM) Controlling machines and processes
	4.C.2.d.1.i (CX) Spend time making repetitive motions
	4.C.3.d.3 (CX) Pace determined by speed of equipment
	\item Non-routine manual (NRM)\\ 
	4.A.3.a.4 (IM) Operating vehicles, mechanised devices, or equipment
	4.C.2.d.1.g (CX) Spend time using hands to handle, control or feel objects, tools or controls
	1.A.2.a.2 (IM) Manual dexterity
	1.A1.f.1 (IM) Spatial orientation 
\end{itemize}
\end{itemize}

Green jobs are defined as either recomposing the energy mix towards green energy or as increasing energy efficiency in general: "\textit{reducing the use of fossil fuels, decreasing pollution and greenhouse-gas emissions, increasing the efficiency of energy usage, recycling materials, and developing and adopting renewable sources of energy}"

Idea to include technology on energy efficiency, which contributes to lowering emissions. 
Model so far does not allow for an increase in energy efficiency
\begin{align*}
Y=(E^{\frac{\varepsilon_e-1}{\varepsilon_e}}+N^{\frac{\varepsilon_e-1}{\varepsilon_e}})^\frac{\varepsilon_e}{\varepsilon_e-1}
\end{align*}
what research does in the baseline model is increasing output for the same amounts of input. To allow for an increase in energy efficiency could add
\begin{align*}
Y=((A_eE)^{\frac{\varepsilon_e-1}{\varepsilon_e}}+N^{\frac{\varepsilon_e-1}{\varepsilon_e}})^\frac{\varepsilon_e}{\varepsilon_e-1}
\end{align*}
But, in fact most of the jobs counted as green refer to increasing the use of renewable energy resources. Could also argue that there is an upper bound on the efficiency of energy
\subsection{Models}
\begin{itemize}
\item \cite{Bilbiie2012EndogenousCycles}
\begin{itemize}
\item a model with rep agent
\item investment in the form of stock 
\item innovation as a form of new products
\item one final good sector
\item monopolistic competition
\item homothetic preferences
\end{itemize}
\item \cite{Ravn2006DeepHabits}
\begin{itemize}
 \item habits over average previous consumption of specific good! not over total consumption
 \item rep agent 
 \item habits: marginal utility rises as habits rise \ar could look at what happens as habits are reduced! \ar marginal utility at given consumption level reduces!
 \item more is always better! Plus increases habits \ar I want: that more might not be better after some point
\end{itemize}
\item \cite{McKay2021LumpyPolicy}
\begin{itemize}
\item New Keynesian model with durable and non-durable consumption 
\end{itemize}
\item \cite{Acemoglu2012TheChange}
\begin{itemize}
\item endogenous growth
\item rep agent
\item single labour market
\item no resource use in clean sector! ; abstracts from waste
\item disaster risk!: There is a lower bound on the quality of the environment 
\item environmental externality only affects Utility! So no chance for \textbf{environmental quality} to drive production to zero!
BUT there is a natural resource which is used in production; \textit{How do the two relate?} \ar when environmental quality affects regeneration of exhaustible resource, then there would be some connection, but there is no regeneration of the resource, I think
\item there is degradation of the environment through unsustainable production (only!) and 
\end{itemize}
Functional forms
\begin{align*}
S\in[0,\bar{S}],\ & \text{where}\ \bar{S}\ \text{is the quality of the environment without pollution;}\\
S_v=0 \Rightarrow S_t=0 \forall t\geq v,\ &  0 \ \text{is the point of no return.}\\
\underset{S\rightarrow0}{lim} U(C,S)=-\infty\ & \text{S=0 is a disaster!}\\
\underset{S\rightarrow0}{lim}\frac{\partial U(C,S)}{\partial S}=\infty\ &\\
S_{t+1}= -\xi Y_{dt}+(1+\delta)S_t& \\ 
\text{evolution of environmental quality:} & \text{ falls in dirty production; regeneration rate }\\
 \text{both are exponential relationships}\Rightarrow&\text{ smaller env. quality slower regeneration}\\ 
 &\text{ higher pollution, stronger degradation}
\end{align*}
The dirty sector uses an exploitable resource in the production process
\begin{align*}
Y_{dt}= R_t^{\alpha_2}L_{dt}^{1-\alpha}\int_{0}^{1}A_{dit}^{1-\alpha_1}x_{dit}^{\alpha_1}di
\end{align*}
$R_t$ is the exhaustible resource
\begin{align*}
Q_{t+1}=Q_t-R_t
\end{align*}
they look at a version where the resource is common property (water, air) or owned (Hotelling rule)
\item \cite{Heikkinen2015DegrowthConsumers}: macro model with voluntary reduction in consumption
\item \cite{Borissov2019CarbonDevelopment}: model labour sector in more detail: skill, sectors, and transition
\item \cite{Michaillat2015AggregateUnemployment, Auerbach2021InequalityEconomy} examples of models with economic slack. But both do not feature a satiation point of consumption. 
\item 
\cite{Loebbing2019NationalChange}
\begin{itemize}
	\item studies the welfare effects of a progressive tax reform in a model where skill supply drives the innovation decision of firms
	\item when the rich reduce their labour supply more \ar innovation is directed towards low skills\ar the wage distribution compresses
	\item the overall effect on welfare is mixed since not only do low skill wages catch up but also does the tax base reduce as the high skilled reduce their labour supply
	\item in sum, he finds that, taking directed technical change into account increases the set of welfare improving tax reforms 
	\item first he focuses on the mechanism, second on the optimal tax scheme
	\item optimal tax reform discussed in a comparison to the optimal tax a planenr would choose who perceives technology as fixed
	\item then quantification \ar he finds small responses of labour supply to progressive tax reforms
\end{itemize}
\end{itemize}


\subsection{Motivation}
\begin{itemize}
\item \cite{Schor2005SustainableReduction}
\begin{itemize}
	\item arguments against unlimited growth
	\begin{itemize}
\item hhh
	\end{itemize}
\end{itemize}
\item \cite{Dasgupta2021}
\begin{itemize}
\item emphasises the use of nature as a sink (stock) and as an input to production \ar can the two be combined?
\end{itemize}
\end{itemize}
 %<- contains summaries of potentially relevant papers
\section{Core Model}\label{sec:mod_an}
%\tr{if $tau_f$ was to replicate share of marginal products, then $H^*\geq H_{FB}$; i.e. absent lump-sum transfers (sub-optimal setting)}
%\textbf{Points to be made}
%\begin{enumerate}
%\item the efficient allocation consists of both a recomposing and a scaling element \ar discuss social planner allocation \checkmark
%\item if $tau_f$ was to replicate share of marginal products, then $H^*\geq H_{FB}$; i.e. absent lump-sum transfers (sub-optimal setting) \checkmark
%\item lump-sum transfers implement the efficient reduction in hours worked and ensure the efficient amount of consumption \checkmark
%\item absent lump-sum transfers and income, households work too much  \checkmark; This follows from setting $\tau_\iota>0$.
%\item redistribution through the income tax scheme establishes the efficient allocation if the income tax is progressive \textit{Intuition: in contrast to lump-sum transfers, transferring environmental tax revenues through the income tax scheme constitutes a positive multiplication of labor income \ar this increases labor efforts. The progressive tax counters this tendency.} \checkmark
%\end{enumerate}

This section develops a general model which is at the core of the analytical and quantitative results. 
The model presented in this section is designed as simple as possible to derive the theoretical results. In section \ref{sec:model}, the model is extended to the quantitative framework notably by adding endogenous growth and skill heterogeneity. %investigate the inefficiency arising in hours worked when an environmental externality has to be taken care of.

In the model, the household sector can be described by a representative household. The household faces a consumption and labor supply decision. The final consumption good is a composite of a fossil and a green good. Labor is the only input to production. For simplicity, the green sector does not induce any externality; yet, whenever intermediate goods are no perfect substitutes, final good production is never perfectly green. The core model abstracts from endogenous growth and is static. 

\paragraph{Representative household}
Throughout the paper, household's decisions are static. Each period, the household maximizes its period utility
\begin{align}
U(C,H; F).
\end{align} 

The household derives utility from consumption, $C$, but experiences disutility from hours spent working, $H$. An externality from fossil production, $F$, decreases household utility and is taken as given by the household.
I assume additive separability of consumption, hours, and the externality. I further assume that utility of consumption is increasing and strictly concave. As regards hours and the externality, utility is decreasing and strictly convex.
Utility maximization is subject to a period budget constraint
\begin{align}
	 C= \lambda(wH)^{1-\tau_{\iota}}+T_{ls}. \label{eq:hhbudget}
\end{align}

The variable $w$ indicates the wage rate, and $T_{ls}$ denotes lump-sum transfers from the government.
The planner levies income taxes on labor income using a non-linear tax scheme common in the public finance literature \citep{Heathcote2017OptimalFramework, Benabou2002TaxEfficiency}. The tax scheme is
characterized by (i) a scaling factor, $\lambda$, which determines the level of average tax revenues in the economy and (ii) a measure of the tax progressivity denoted by $\tau_{\iota}$. 
\cite{Heathcote2017OptimalFramework} show that whenever $\tau_{\iota}>0$ the tax scheme is progressive since the marginal tax rate exceeds the average tax rate irrespective of  pre-tax labor income. Hence, average tax payments increase with labor income. An alternative intuition is that when $\tau_{\iota}>0$, the elasticity of post- to pre-tax labor income is smaller unity for all levels of pre-tax labor income.  %\footnote{\ I show that the result is equivalent with a linear tax rate in the appendix.} 

\paragraph{Production}
All sectors of production are perfectly competitive and production functions have decreasing returns to scale. %\footnote{\ \textit{With increasing returns to scale the assumption of perfect competition would be violated. With constant returns to scale, the solution is not unique.}}. The final consumption good, $Y$, is a composite of the fossil, $F$, and the green intermediate good, $G$. 
Intermediate goods are produced from labor, $L_J$, and technology, $A_J$, where $J\in \{F,G\}$ indicates the fossil and the green sector: 
\begin{align}
Y=Y(F, G), \hspace{5mm} F=F(A_F, L_F),\hspace{5mm} G=G(A_G, L_G) \label{eq:prod}
\end{align}

\paragraph{Government}
The government raises income taxes from households and levies ad-valorem sales taxes, $\tau_F$, on fossil producers' revenues $p_FF$, where $p_F$ denotes the price of the fossil good paid by final good producers. Revenues from the income tax and the environmental tax are treated separately by the government. Income tax revenues are fully redistributed through the income tax schedule, while environmental tax revenues are either lump-sum redistributed to households, $T_{ls}$, consumed by the government, $Gov$, or redistributed through the income tax scheme, $T_\iota$:
\begin{align}
\tau_{F}p_FF=T_{ls}+T_\iota+Gov, \hspace{7mm}
0={w H}-\lambda(w H)^{1-\tau_{\iota}}+T_\iota. \label{eq:gov_but}
\end{align}
%Environmental tax revenues are either transferred lump-sum, fully consumed by the government, or transferred through the income tax schedule.

\paragraph{Markets}
The market for labor and the final good both clear: 
\begin{align}
H=L_F+L_G,\ \hspace{5mm} Y=C+Gov. \label{eq:market_clear}
\end{align}
 The final good, $Y$, is the numeraire. In this simple model, labor moves freely between the green and fossil sector. 
%I summarize the equations determining the competitive equilibrium in appendix section \ref{app:model}.
\paragraph{Competitive equilibrium}
In a competitive equilibrium, household behavior is determined by the budget constraint, equation \ref{eq:hhbudget}, and labor supply which follows from the household first order conditions and substituion of $\lambda$ from the government's budget on the income tax:
\begin{align}
-U_H=U_C(1-\tau_{\iota})w\left(1+\frac{T_\iota}{wH}\right). \label{eq:hsup}
\end{align}
Firms choose the quantity of input goods to maximize their profits taking prices as given. The following equations describe this behavior in equilibrium:
\begin{align}
p_G=\frac{\partial Y}{\partial G}, \hspace{5mm}
p_F = \frac{\partial Y}{\partial F}, \hspace{5mm}
w= p_F(1-\tau_F)\frac{\partial F}{\partial L_F}=p_G\frac{\partial G}{\partial L_G}.\label{eq:profmax}
\end{align}

The competitive equilibrium is defined as prices and allocations so that households and firms behave optimally; i.e. equations \ref{eq:hhbudget}, \ref{eq:hsup} and \ref{eq:profmax} hold. Production happens according to \ref{eq:prod}.  Equilibrium prices and the wage rate adjust to clear markets, equations \ref{eq:market_clear}. Finally, the government's budgets are satisfied \ref{eq:gov_but}. Policy variables $\tau_F$ and $\tau_\iota$ are taken as given. 

\section{Theoretic results}\label{sec:theory}
This section derives and discusses the main theoretical results. Section \ref{subsec:sp} defines the efficient allocation. It constitutes a benchmark for the optimal allocation which is discussed in section \ref{subsec:decen_ec}. 
\subsection{Social planner}\label{subsec:sp}
Let the share of fossil to total labor be denoted by $s=L_F/H$. The social planner's problem reads
\begin{align}
\underset{s, H}{\max}\ & U(C,H; F)\\ s.t\ \ & C=Y.
\end{align}
The first order conditions are given by
\begin{align}
wrt.\ s:\hspace{4mm} & U_C \cdot \left(\frp{Y}{F}\frp{F}{s}+\frp{Y}{G}\frp{G}{s}\right)=-U_F\frp{F}{s}, \label{eq:fbs}
\\
wrt.\ H:\hspace{4mm} & U_C\frp{Y}{H}+U_F\frp{F}{H}=-U_H\label{eq:fbh}. 
\end{align}
Where $U_X$ denotes the partial derivative of utility with respect to the variable $X$.
These equations determine the efficient or first-best allocation. 
Absent an externality, $U_F=0$, the efficient distribution of labor across sectors equalizes the marginal product of labor across sectors; compare equation \ref{eq:fbs}. Efficient hours balance the marginal utility gain from consumption and the marginal disutility from working formalized by equation \ref{eq:fbh}. 

When there is an externality, the social planner adjusts the allocation by two modulations: (i) a recomposing and (ii) a scaling one. 
The recomposition is determined by equation \ref{eq:fbs}.
The negative externality of fossil production makes it efficient to adjust the fossil labor share so that  a marginal reallocation of labor to the fossil sector would raise output.\footnote{\ Note that $U_F<0$ by assumption so that the right-hand side is positive and that $\frac{dG}{ds}<0$. }
One can show that the social planner reduces the fossil labor share when the aggregate production function features decreasing returns to scale in its labor inputs, $L_G$ and $L_F$\footnote{\ This follows from assuming that either final good and/or intermediate good production functions are decreasing returns to scale.}.
\begin{comment}
The equation 
\begin{align}
\frac{-U_F}{U_C \frac{dY}{dF}}=1+\frac{\frac{dY}{dG}\frac{dG}{ds}}{\frac{dY}{dF}\frac{dF}{ds}}.
\end{align}
The term on the left-hand side is the social cost of the externality: it measures what the representative household is willing to pay for a further reduction in fossil production. 
\end{comment}

The scaling effect is summarized by equation \ref{eq:fbh}.
First note that equation \ref{eq:fbh} can be rewritten by substituting equation \ref{eq:fbs} and noticing the relation of derivatives with respect to $H$ and $s$.\footnote{\ This is done in more detail for the optimal allocation in appendix section\ref{app:incometax0}. The relation of derivatives are summarized in section \ref{app:dervs_use}.}  
The second first order condition becomes:
\begin{align}\label{eq:fbh_simp}
-U_H=U_C\frac{\partial Y}{\partial G}\frp{G}{L_G}.
\end{align}
Hence, the externality drops from the expression which determines efficient labor. Hours are not chosen in a way to handle the externality. It is rather an indirect effect of externality mitigation which makes an adjustment in hours efficient which I will discuss next.

The recomposition of labor input towards the  green sector reduces the marginal product of labor in the green sector and the utility gains from more labor decline.  This effect has two opposing impacts on efficient labor supply. On the one hand, there is a substitution effect: as leisure becomes less costly, the efficient amount of hours reduces (note that the right-hand side of equation \ref{eq:fbh} is increasing in $H$). On the other hand, the economy becomes poorer in terms of consumption and more work effort might be efficient. This is captured by the term $U_C$ and equivalent to an income effect. 
%In total, which effect dominates depends on the curvature of the utility from consumption, $\theta$. With $\theta>1$ the  lower marginal product of labor decreases the efficient amount of hours worked. 
%Second, the social planner reduces hours worked due to their negative exeternality through fossil production. This effect is introduced by the term $U_F\frac{dF}{dH}<0$. 

Proposition \ref{prop:0} summarizes above discussion
\begin{prop}\label{prop:0}
	Efficient externality mitigation consists of a recomposing and a scaling approach. Efficiency of the scaling effect arises indirectly due to the reduction in the marginal product of labor induced by a recomposition of input factors.
\end{prop}


Depending on the importance of the income effect, efficient hours worked may be higher or lower than  absent an externality. %\footnote{ \ I discuss in the appendix conditions on parameter values when assuming functional forms of the model.}
I will show in the following, that irrespective of whether the social planner de- or increases hours, the decentralized economy always features higher hours when environmental tax revenues are not redistributed lump-sum. 

\begin{comment}
\hrule
One can show that the total effect of a drop in the fossil labor share on hours worked is positive, i.e. $\frac{dh_{FB}}{ds}>0$, if $\theta<\frac{\varepsilon}{\varepsilon-s}$. If the income effect dominates, the social planner increases hours worked as the economy becomes less productive. 
Under the value for $\theta$ suggested by \cite{Boppart2019LaborPerspectiveb}, the efficient scale effect is to increase hours worked. When, however, the substitution effect outweighs or dominates the income effect - as commonly assumed in the public finance literature \citep{Heathcote2017OptimalFramework, LansBovenberg1994EnvironmentalTaxation, LansBovenberg1996OptimalAnalyses} \tr{CHECK this}!.
Nevertheless, the level of hours worked exceeds the efficient level irrespective of $\theta$ when no lump-sum transfers are available. 
When the efficient level of hours increases, though, the fossil labor share reduces even more to outweigh the increase in the externality.

content...
\end{comment}
\subsection{Decentralized economy}\label{subsec:decen_ec}

In today's market economies, a planner to allocate hours worked and consumption does not exist. Instead, governments can revert to tax and transfer instruments to correct for distortions, such as an environmental externality. The question arises if the efficient allocation can be decentralized by the use of taxes and transfers in a competitive economy. And if so, how? Section \ref{subsec:Rams} defines the Ramsey problem. %For now, I assume that the income tax is not available and $\tau_{\iota}=0$, $\lambda=0$.

I show in this section that redistribution of environmental tax revenues are essential to implement the first-best allocation in the competitive equilibrium. Only in combination with lump-sum transfers of  environmental tax revenues does an environmental tax suffice to implement the efficient allocation. %Then the environmental tax equals the social cost of the externality as shown by \textit{PIGOU}. 
When environmental tax revenues are not redistributed lump-sum, hours worked exceed their efficient level, and a role for income taxes to lower hours worked arises. I consider two cases.
In the first case, section \ref{subsec:nolump}, environmental tax revenues are consumed by the government. The optimal policy consists of (i) a progressive labor income tax scheme and (ii) an environmental tax which may deviate from the social cost of the externality. The logic is that labor taxes help to align hours worked closer to the efficient allocation. 
Nevertheless, the efficient allocation is not feasible under this policy regime.

In the second scenario, therefore, I point to an option to implement the efficient allocation even if lump-sum transfers are not available: redistributing environmental tax revenues through the income tax scheme, section \ref{subsec:integrated}.
 I show that, again, the optimal tax scheme is progressive. 
As a consequence, the considered optimal environmental policies which establish the efficient allocation feature - as a side effect - a more equal distribution of income, through either lump-sum transfers or a progressive tax scheme.% \tr{Not sure though if this holds true in progressive scheme as lambda multiplies labor income}).

%\begin{enumerate}
%\item lump-sum transfers important for Pigou tax to implement efficient allocation: Proposition \ref{prop:1}
%\item when transfers are not redistributed: infeasibility of efficient allocation,  role for labor tax, and violation of Pigou principle \ref{prop:2}.
%\item redistribution through income tax scheme with progressive income tax restores efficient allocation \ref{prop:3}
%\end{enumerate}

\subsubsection{Government problem}\label{subsec:Rams}
The government is characterized by a Ramsey planner: it seeks to maximize utility of the representative household but can only revert to tax instruments and transfers to implement the welfare-maximizing allocation. The behavior of firms and households constrain the government's optimization problem. 
The Ramsey problem is defined as
\begin{align}
\underset{s, H}{\max}\ & U(C,H; F)\\ s.t\ \ & (1)\  C=Y-Gov\\ & (2) \ \text{behavior of firms and households}.
\end{align}
The first order conditions differ from the social planner's ones through the derivatives on government revenues, $Gov$:
\begin{align}
wrt.\ s:\hspace{4mm} & U_C\left(\frac{\partial Y}{\partial F}\frac{\partial F}{\partial s}+\frac{\partial Y}{\partial G}\frac{\partial G}{\partial s}-\frac{\partial Gov}{ \partial s}\right)=-U_F\frac{\partial F}{\partial s}, \label{eq:sbs}
\\
wrt.\ H:\hspace{4mm} & U_C\cdot \left(\frac{\partial Y}{\partial H}-\frac{\partial Gov}{\partial H}\right)+U_F\frac{\partial F}{\partial H}=-U_H\label{eq:sbh}. 
\end{align}

%-- paragraph to show that with Gov=0 and lump-sum transfers, the efficient allocation is implemented
When environmental tax revenues are fully redistributed lump-sum, i.e. $Gov=0$, $T_\iota=0$, then an environmental tax equal to the marginal social cost of fossil production\footnote{\ I define and derive the social cost of fossil production in appendix section \ref{app:scp}.} implements the efficient allocation. This observation is known as the \textit{Pigou principle} in the literature. 
To see this, note that equation \ref{eq:sbs} ensures that the social planner's first order condition, equation \ref{eq:fbs}, is satisfied. 
Rewriting equation \ref{eq:fbs} reveals that the Pigou principle holds: %\footnote{\ I derive the social cost of pollution as the price the representative household is willing to pay for a marginal reduction in fossil production. The derivation is exponded in appendix section \ref{sec:mod_an}. 
%	To be precise, social cost of pollution refers to the marginal cost evaluated at the resulting equilibrium allocation.}: The Pigou principle. 
\begin{align}
\underbrace{\frac{-U_F}{U_C\frac{\partial Y}{\partial F}}}_{\text{marginal social cost of fossil production}}=1+\frac{\frac{\partial Y}{\partial G}\frac{\partial G}{\partial s}}{\frac{\partial Y}{\partial F}\frac{\partial F}{\partial s}}=\tau^*_F.
\end{align}
Where the second equality follows from substituting intermediate firms' profit maximization conditions from equations \ref{eq:profmax}. I show in appendix section \ref{app:incometax0} that setting the environmental tax to the social cost of fossil production implies that the second first order condition of the Ramsey planner is satisfied without use of the income tax instrument: $\tau_{\iota}^*=0$. %at $\tau_\iota=0$ when all environmental tax revenues are redistributed lump-sum: $T_{ls}=\tau_{F}p_FF$ (and $Gov=0$ and $T_\iota=0$).

In this paragraph, I briefly discuss the mechanism of the corrective tax.
As discussed previously, absent an externality of production, it is efficient to balance marginal products of labor across sectors. However, when there is an externality, the social planner lowers the fossil share of labor which results in a higher marginal product of labor in the fossil sector. To sustain this gap between marginal products in the competitive equilibrium, the government has to introduce a corrective tax so that market forces do not direct labor towards the sector with the higher marginal product. In other words, the corrective tax is set so that wage rates equalize despite heterogeneous marginal products of labor. As a result of this intervention, the equilibrium wage rate is below the marginal product of labor in the fossil sector.\footnote{\ I formally discuss this statement in appendix section \ref{app:wageMPL}.} These are efficiency costs associated with a use of the environmental tax. They are the source of the competition between environmental good provision and raising government funds or equity alluded to in the literature \citep[e.g.][]{LansBovenberg1994EnvironmentalTaxation}.  
However, from an environmental policy perspective, the adjustment in labor due to a lower marginal product in the green sector is efficient. It mirrors the reduction in the marginal product of green labor in the efficient allocation. It is only when labor constitutes the base of another tax, that the reduction in labor supply becomes costly. 
\subsubsection{Lump-sum transfers and the optimal environmental policy}\label{subsec:nolump}

Another distortion of labor supply occurs when environmental tax revenues are not redistributed lump-sum.
In light of the discussions on how best to recycle environmental tax revenues, this is an important result which I summarize in proposition \ref{prop:1}:

\begin{prop}\label{prop:1}\textbf{Importance of lump-sum redistribution of environmental tax revenues}
	Without lump-sum transfers of environmental tax revenues, hours worked are inefficiently high when $\tau_{F}$ implements the efficient fossil labor share.
%Absent lump-sum transfers and when the  wage rate is non-increasing in equilibrium hours, 
%implementing the efficient share of fossil labor, $s^*=s_{FB}$, via an environmental tax results in inefficiently high hours worked. Lump-sum transfers would serve as a means to lower hours worked via an income channel. %The efficient allocation is infeasible.=> this statement would need to look at the optimal policy, this here is not a statement on the optimal policy
\end{prop}

% The logic is as follows:
%$s^*=s_{FB}$ when tauf implements the efficient allocation. 
% yet, implementing s-efficient without lump-sum transfers results in inefficiently high hours.

 
The proof of proposition \ref{prop:1}, depicted in appendix section \ref{app:nolumpsum_hourshigh}, is informative on the mechanism: The conclusion that $H^*>H_{FB}$, where the subscript $FB$ indicates the first-best allocation, while an asterisk marks the optimal allocation,  follows from consumption in the competitive equilibrium being lower than in the first-best allocation. Lump-sum transfers, thus, imply lower hours worked in the competitive equilibrium through an income effect.

% source of the inefficiency
%Are labor taxes used to cope with the externality, or is it rather that environmental taxation induces a distortion on labor supply?  I will argue in this section taht 
%The distortion in labor supply results from non-redistribution of environmental tax revenues. As it lowers the wage rate to ensure a lower labor share in the fossil sector, the environmental policy positively affects labor supply via an income effect. When lump-sum transfers are not used to counter this mechanism, equilibrium hours are inefficiently high.  %; otherwise, if the labor tax had an advantage in mitigating the externality, it would have been used in the presence of lump-sum transfers. But it is not.

% violation Pigou principle
Choosing the environmental tax to implement the efficient share of fossil production while hours are inefficiently high, most likely  violates the Pigou principle: the environmental tax does not equal the social cost of pollution. One reason is that a higher labor supply increases fossil production above the efficient level; the social cost of pollution increase when the disutility of pollution is convex. Another reason is that household consumption deviates from the efficient level of consumption: if it is below, then the marginal utility of consumption increases diminishing the willingness to pay for a reduction in the externality. 

% discussion: importance to reduce hours worked in context of exogenous emssion limit
When labor supply is endogenous, lump-sum transfers gain in importance for the optimal allocation. When labor supply is fixed, the non-redistribution of environmental tax revenues results in inefficiently low consumption with no further impact on emissions. When labor supply is elastic, however, the lower consumption results in too high hours worked. This is especially important as it aggravates the externality by increasing economic production. When there is an absolute limit on the externality - as is the case for greenhouse gas emissions today - the scale effect could make a stricter environmental tax necessary. 

%\textit{could be} optimal absent means to reduce hours worked. 

% transition to proposition 2: optimal policy
Proposition \ref{prop:1} rationalizes the use of distortionary income taxes as a tool to lower the supply of labor when environmental tax revenues are not redistributed lump-sum. 
The optimal environmental policy then consists of both a progressive income tax scheme and an environmental tax. However, the efficient allocation is infeasible as private consumption is inefficiently low when environmental tax revenues are consumed by the government. Proposition \ref{prop:2} highlights these result.

\begin{prop}\label{prop:2}\textbf{Optimal environmental policy without redistribution of environmental tax revenues}
When environmental tax revenues are not redistributed but instead consumed by the government, i.e., $Gov=\tau_Fp_FF$ and $T_{ls}=T_\iota=0$, then (i) the Pigou principle is violated, and (ii) a motive for labor taxation arises: the optimal income tax scheme is progressive. % if the aggregate production function features decreasing or constant returns to scale. 
The efficient allocation is infeasible.  
\end{prop}

Proof: in appendix sections \ref{app:reiv_tauf} to \ref{app:ineff}.

\paragraph{Optimal environmental tax}
One can show that under the optimal policy the environmental tax equals
	\begin{align}
\tau_{F}^*= SCC+\frac{\partial Gov}{\partial s}\frac{1}{\frac{\partial Y}{\partial F}\frac{\partial F}{\partial s}}.
\end{align}
\begin{comment}
content...

This can be further simplified:
\begin{align}
\tau_F^* = 1-\frac{SCC}{\frp{w}{s}}w. 
\end{align}
A condition for $\tau_F$ to exceed the social cost of the externality reads
\begin{align}
SCC<\frac{1}{1+\frac{w}{\frp{w}{s}}}
\end{align}
\end{comment}
Hence, if government revenues increase with the share of fossil labor, then the optimal environmental tax exceeds the social cost of pollution in equilibrium. 	

With the environmental tax the government intents to reduce the share of fossil labor in equilibrium to lower the externality. 
When environmental tax revenues are consumed by the government and reduce private consumption, there is another mechanism of the environmental tax which the government takes into account. In general, it is welfare increasing to raise private consumption, or, equivalently, to lower government consumption. Since a higher environmental tax implies a lower fossil labor share, a positive relation of fossil labor and government consumption adds to the welfare enhancing effect of the environmental tax. If, in contrast, a lower fossil labor share raises government revenues, the environmental tax has an additional negative effect on social welfare: the Ramsey planner chooses a lower environmental tax. 

%\tr{What mechanisms make government revenues rise with s, which reduce it?}

\paragraph{Optimal labor income tax}
The optimal labor income tax progressivity parameter is given by 
\begin{align}\label{eq:nls_taulopt}
\tau_\iota^*=\frac{1}{w}\frp{w}{s}_{F=\bar{F}}.
\end{align}
%\tr{Checked!!}
In words, the optimal income tax progressivity parameter equals the semi-elasticity of the wage rate in response to a decrease in the green labor share keeping fossil production unchanged. 
Since the environmental tax serves to sustain a gap between marginal products of labor, thereby  reducing the wage rate, a lower labor share in green production increases the wage rate. As a result, the labor income tax scheme is progressive: $\tau_\iota>0$. For a proof and the derivation of the optimal income tax progressivity see appendix section \ref{app:subsub_nltaul}.

Intuitively, the optimal labor income tax scheme is progressive to lower work effort closer to the efficient level. As argued in proposition \ref{prop:1}, hours worked are inefficiently high absent lump-sum transfers, when aggregate production is characterized by decreasing returns to scale. Under the same condition, therefore, the optimal income tax is progressive. 

%\paragraph{Infeasibility of efficient allocation}
%The optimal allocation is inefficient since either consumption is too low or hours work are inefficiently high. The proof is given in appendix section \ref{app:ineff}.
%Since the presence of the environmental tax artificially increases labor in the green sector depressing the wage rate (under the assumption of decreasing returns to scale), the wage rate rises by a reduction of the green labor share. 

\begin{comment}
\paragraph{Complements}
%\tr{Maybe no need to discuss this formally}
Income tax progressivity and the environmental tax
are complements if 
\begin{align}
\frac{d \tau_{\iota}^*}{d \tau_{F}}>0.
\end{align}
Totally differentiating equation \ref{eq:nls_taulopt} yields
\begin{align}
\frac{d \tau_{\iota}^*}{d \tau_F}=-\frac{1}{w^2}\frp{w}{s}\frac{dw}{d \tau_F}+\frac{1}{w}\frac{\partial^2 w}{\partial s^2}\frac{ds}{d \tau_F}
\end{align}
The first summand is positive since the environmental tax reduces the wage rate and a rise in dirty labor share increases the wage rate. The second summand is positive if $\frac{\partial^2 w}{\partial s^2}<0$. 

content...
\end{comment}

%\textit{
%Equation \ref{eq:nls_taulopt} makes clear that environmental taxation and the labor income tax are complements under decreasing returns to scale. When the environmental tax rises, thereby increasing the share of labor allocated to the green sector, the marginal product of green labor decreases further. A marginal reduction in the green labor share would increase the wage rate more the higher the green labor share, hence, the optimal labor tax progressivity increases with the environmental tax. 
%Secondly, the wage rate decreases with $\tau_F$ which as well inflates the optimal labor tax progressivity.}

% In the same section of the appendix, I prove that $\tau_\iota^*>0$ if green production and aggregate production feature constant or decreasing returns to scale and at least one produces with decreasing returns to scale. An assumption satisfied under Cobb-Douglas or constant elasticity of substitution production functions when goods are substitutes. When goods are complements, then...  
 
% \tr{Only direct effect, not a general equilibrium result}
% \begin{corollary}
% Absent income taxes, hours worked are inefficiently high when production features decreasing or constant returns to scale. 
% \end{corollary}
% As a corollary of proposition \ref{prop:2}, it follows that absent income taxes, hours worked are inefficiently high. This follows directly from the houeshold's first order condition:  a positive value of $\tau_\iota$ ireduces the right-hand side. Since the marginal utility is decreasing in hours the left-hand side is a positive function of labor supply. Hence, as the right-hand side decreases, hours diminish.   
% +
 
 \begin{comment}
Complementarity of the two instruments is intuitive. The use of labor taxes is not primarily to handle the environmental externality but instead to cope with a distortion induced by environmental taxation. 
This conclusion is backed by the observation that the externality measure, $U_F$, drops from the planner's first order condition on hours, equation \ref{eq:sbh}, if the environmental externality is taken care of by the environmental tax. Then, labor supply is determined solely by the trade-off between consumption and leisure.\footnote{\ On this argument see the proof on the optimality of $\tau_\iota^*=0$ when lump-sum transfers are available in appendix section \ref{app:incometax0}.} 

content...
\end{comment}
 

%\begin{proof}\textit{With an environmental tax alone, hours worked are inefficiently high } \tr{waiting; to be done}
%\end{proof}
% \tr{Possible interpretation of regressive income taxes: –  then it is more important to increase consumption! }
 

\subsubsection{Optimal policy with combined environmental and fiscal policy}\label{subsec:integrated}

In this subsection, I propose a policy regime which allows to establish the efficient allocation when lump-sum transfers are not available. In this setting, the government redistributes environmental tax revenues through the income tax scheme, the \textit{integrated policy} regime:  
\begin{align}
Gov= wh-\lambda (wH)^{1-\tau_\iota}+\tau_F p_FF.
\end{align}
The budget is balanced and $Gov = 0$ which determines $\lambda=\frac{wH + \tau_F p_F F}{w^{1-\tau_{\iota}}}$. 
Under this regime, the Ramsey planner can replicate the efficient allocation. 
The efficiency result is summarized in proposition \ref{prop:3}

\begin{prop}\label{prop:3}
	If lump-sum transfers are not available, the government can implement the efficient allocation by  transferring environmental tax revenues through the income tax scheme. The optimal tax scheme is progressive and the optimal environmental tax equals the social cost of pollution. %Recomposing and reductive policies are complements in the optimal environmental policy if environmental tax revenues are on the upward slow.
\end{prop}
Proof: in appendix section \ref{app:proofintegrated}. 

	That the optimal environmental tax satisfies the Pigou principle follows straight from the Ramsey planner's first order conditions. Since $Gov=0$ when environmental tax revenues are redistributed through the income tax scheme, the Ramsey planner's first order condition with respect to the fossil labor share ensures that the optimal environmental tax equals the social cost of pollution; compare equation \ref{eq:pigou}. 
	
	%The proofs that the optimal income tax scheme is progressive and that the optimal allocation is efficient\tau^*_f are sketched in appendix section \ref{app:proofintegrated}.
	 The optimal income tax is characterized by
\begin{align}
\tau^*_\iota=1-\frac{wH}{Y}=\frac{\tau_F^*p_FF}{Y}.
\end{align}
This simple relation reveals that the optimal income tax progressivity is positively related to environmental tax revenues.  %The higher the optimal environmental tax, which equals the social cost of pollution in this setting, the higher the optimal income tax progressivity. 
Hence, when environmental tax revenues are on the upward sloping part of the Laffer curve, it holds that  the more the Ramsey planner recomposes production, the more intense the reduction policy has to be. Reductive and recomposing policies complement each other in the integrated policy regime. 
 
\begin{comment}
\begin{prop}
Effect of using progressive income scheme on inequality (maybe as opposed to lump-sum transfers)
\end{prop}

content...
\end{comment}

\begin{comment}
\subsubsection{Discussion in relation to the literature}
These findings relate to the literature on a double dividend of environmental taxation discussed and partly rejected in the seminal paper by \cite{LansBovenberg1994EnvironmentalTaxation}.\footnote{ \ The double dividend of environmental policies refers to the idea that the revenues of environmental taxation can serve to improve on other policy targets such as equity or lowering distortionary fiscal policies. \cite{LansBovenberg1994EnvironmentalTaxation} argue that there is no double dividend because environmental taxes exert efficiency costs which outweigh the gains from lower distortionary income taxes. } The authors, inter alia, argue that recycling environmental tax revenues to lower distortionary fiscal policies has an advantage above recycling environmental tax revenues as lump-sum transfer. The latter would reduce labor supply even more thereby further narrowing the tax base of income taxes.  This result is referred to as the \textit{weak double dividend} hypothesis. The present paper demonstrates that there exists a lower bound up to where a reduction of labor reducing policies is beneficial. To illustrate this point, assume environmental tax revenues are sufficient to cover the government's exogenous revenue constraint. When no lump-sum transfers are available - a necessary assumption to motivate the use of distortionary fiscal policy measures - then, setting distortionary income taxes to zero is not optimal according to this paper's findings. The reason is that some reduction in hours worked is in fact efficient.\footnote{\ This argument refers to proposition \ref{prop:2}, i.e., to a setting where no lump-sum transfers are available. Even without an exogenous funding constraint, labor income taxes are used.}  % labor supply would be inefficiently high. 
%Nevertheless, the \textit{weak double dividend} result states that the  use of environmental tax revenues to lower fiscal policies is advantageous about recycling revenues as lump-sum transfers, because lump-sum transfers would lower the tax base of the income tax even more.
%To implement the efficient allocation, there is no choice of how to use environmental tax revenues and that they are to be perceived as a means to lower hours worked absent other motives for government intervention. To reconcile these two findings, think of the result presented herein as a lower bound on the optimality to diminish the reduction in hours worked through cutting distortionary taxes. 
  
%Although this trade-off between the environmental advantage of lower work effort versus a lower income tax base has implicitly been studied in the double-dividend literature, the efficiency of lump-sum transfers or progressive income taxes has gone unnoticed. 
% Absent other motives of government intervention, such as an exogenous funding condition or inequality, lump-sum transfers should 

%\tr{Caution: in double dividend literature there is a second motive for government intervention... then there are gains, but only up to a certain point}

The finding that redistribution is essential to implement the efficient allocation poses a warning to the literature discussing how to use environmental tax revenues such as \cite{Fried2018TheGenerations}. My results stress that there is no free choice in how to use environmental tax revenues but to redistribute if the government wants to implement the efficient allocation. 

content...
\end{comment}
%\input{impo_modeL_FinN}
%\section{Theoretic Results}
\subsection{An emission target calls for a reduction policy under likely parameter values}

\paragraph{Effect of tax progressivity on energy output ratios}


\subsection{Tax progressivity affects the composition of total output}
In the model, tax progressivity affects the innovation decision due to heterogeneous effects on skill supply. 
The optimal ratio of skills supplied by the household is
\begin{align}
\frac{h_{ht}}{h_{lt}}=\left(\frac{w_{ht}}{w_{lt}}\right)^\frac{1-\tau_{lt}}{\tau_{lt}+\sigma}.
\end{align}
The semi-elasticity of the ratio of aggregate skill supply, defined as $\frac{H_h}{H_l}:=\frac{z_hh_h}{z_lh_l}$, in response to a change in income tax progressivity is then given by
\begin{align}
\frac{d\log\left(\frac{H_h}{H_l}\right)}{d\tau_l}=-\frac{1+\sigma}{(\tau_l+\sigma)^2}\log\left(\frac{w_{h}}{w_l}\right). \end{align}
The direct effect, with fixed prices is negative given a positive wage premium for high skill labour. Hence, a higher tax progressivity implies a decline in the relative supply of high skill labour. 

\paragraph{Effect on the externality }
\begin{prop}Assume that (1) the income share of high-skilled labour exceeds that of low-skilled labour, $\frac{H_lw_l}{H_hw_h}<1$, (2) high-skill labour earns a premium, $\frac{w_h}{w_l}>1$, and (3) energy inputs are substitutes, $\varepsilon_e>1$, then
a rise in income tax progressivity increases the share of fossil energy in the economy when the sum of labour shares in fossil and the green sector is below unity, $\theta_f+\theta_{g}<1$, (intuitively, high skill has a lower income share than low skill). When the sum of income shares of high-skill labour is sufificiently high, and necessarily above unity, a rise in tax progressivity lowers the share of fossil to green energy, $\frac{dlog\left(\frac{F_t}{G_t}\right)}{d\tau_l}<0$.
\end{prop}
\footnote{\textit{Proof}\\ 
	The equation follows from iteratively applying skill demand by labour input producers and the skill market clearing conditions and substituting low-skill supply by the households optimality condition. This gives the following relation of high-skill hours employed in the green sector to total high-skill supply:  
	\begin{align*}
	\frac{h_{hg}}{H_h}=\frac{1-\left(\frac{w_l}{w_h}\right)^\frac{1+\sigma}{\sigma+\tau_{lt}}\frac{z_l}{z_h}\frac{\theta_f}{1-\theta_f}}{1-\frac{\theta_f}{\theta_g}\frac{1-\theta_g}{1-\theta_f}}.
	\end{align*}}
For an intuition consider the output ratio as a function of taxes and the wage ratio in equilibrium:
\begin{align*}
\frac{F_t}{G_t}=\left(\frac{(1-\tau_f)(1-\alpha_f)/(1-\alpha_g)}{\left(\frac{w_l}{w_h}\right)^{\frac{1+\sigma}{\sigma+\tau_l}}\frac{z_l}{z_h}\frac{\theta_f}{1-\theta_f}-\frac{1-\theta_g}{1-\theta_f}}\right)^\frac{1}{\varepsilon_e-1}
\end{align*}

As tax progressivity increases, high-skill supply reduces relative to low-skill supply. 
When high skill has a relatively smaller share in the fossil sector than low-skill in the green sector, the reduction in high skill supply makes 

This translates to an increase in low-skill labour employed in the fossil sector and high skill in the fossil sector rises proportionately leading to a fall in high skill in the green sector by market clearing. When 



\subsection{Growth in the dirty sector has to stop}
Growth in the dirty sector eventually has to stop given the net-zero emission target. Assuming no possibility to increase capture and storage technologies, this would be the case by 2050. Otherwise, it suffices to assume a limit to C02 capture-storage technology. This implies a condition on taxes to counter growth in the green sector. I summarise that result in the following proposition.

\begin{prop}Assume that energy inputs are substitutes. Then, growth in the green sector has to be offset by a rise in the fossil tax. Alternatively, %when high skill is in sufficiently high demand, $\theta_f+\theta_g>>1$, then a rise in low skill supply counteracts a rise in fossil growth, i.e. a more progressive tax. If high skill is in lower demand,  $\theta_f+\theta_g<1$, then a drop in the low-to-high skill ratio, a more regressive tax, can offset green growth. 
	the ratio of low-to-high skill income has to rise, that is, at a positive wage premium for high skill labour a more progressive tax is required. 
\end{prop}

\begin{corollary}
	As a subsidy to green innovation boosts growth in the green sector, it must be counteracted by a stronger corrective tax or a respective change in income tax progressivity. In other words, a green subsidy contributes to growth pressure in fossil energy. The more so the less substitutable goods are.
\end{corollary}

\begin{corollary}
	A higher progressivity of the income tax contributes to keeping fossil production low. A double dividend of redistribution: in addition to lowering inequality it lowers emissions.
\end{corollary}

\tr{\textbf{Next: find an expression for wage ratio as a function of growth rates \ar can discuss exogenous growth case!}}<- potentially relevant stuff
\section{Quantitative model and calibration}\label{sec:model2}

The previous section shows that there is a role for labor taxation when carbon tax revenues are not rebated lump sum. However, the model abstracts from endogenous growth\textemdash an important aspect of today's transitions to green economies. Furthermore, the politically feasible policy which shall be scrutinized is to use of carbon tax revenues as subsidies to green technology usage. In Section \ref{sec_quantmod}, I, therefore, add these aspects to the core model building on \cite{Fried2018ClimateAnalysis}. % in Section \ref{sec_quantmod}. %\footnote{  The result in my framework may differ due to knowledge spillovers and decreasing returns to research. } 
%Then, a motive for income taxation may arise from optimal emission mitigation because the carbon tax deviates from implementing the efficient share of fossil labor. 
%Therefore,  extends the core model to a quantitative framework building on \cite{Fried2018ClimateAnalysis}.
 Section \ref{subsec:calib} calibrates the quantitative model. 

\subsection{Quantitative model}\label{sec_quantmod}
%Either, if the carbon tax is set to decrease fossil research, the labor income tax


The main extensions to the core model are endogenous growth, and a third, non-energy sector. %The latter allows to capture a skill bias in the green sector \citep{Consoli2016DoCapital}. 
%The neutral good is combined with an energy good to form  the final output good. The energy good consists of the dirty, fossil good and green good. The representative household provides two skills: high and low which are used in different shares in the neutral, fossil, and green sector. This extension serves to capture a that the green sector relies more on high-skill labor \citep{Consoli2016DoCapital}.
% Endogenous growth is modeled in form of directed technical change resulting from research enhanced by knowledge spillovers. 
Furthermore, the government maximizes utility of the representative household under the constraint of meeting an exogenous emission limit. This limit attaches social costs to fossil production and replaces the utility costs of fossil in the core model. 
%Appendix \ref{app:quant_mod} provides an overview of all equations determining the competitive equilibrium.
%I study the model for a fixed amount of periods as I do not want to make any assumption on steady growth due to the absolute constraint on fossil production. 

\paragraph{Households}
% the rep agent
A representative household describes the household side.
The household chooses hours spent working, $H_{t}$ and consumption, $C_t$, taking prices as given. The household's problem remains static. Time endowment is given by $\bar{H}$. %Each period, it behaves according to: % The household's problem reads
Labor income of the household is taxed at a constant rate, $\tau_{\iota t}$. The household owns machine producing firms from which it receives profits. It also supplies scientists in a fixed amount: $S$.\footnote{These modeling choices simplify the households budget constraint as profits from firms and scientists' income and subsidies to machine producers cancel. It is common to fix the supply of scientists in the literature on directed technical change in order to simplify the analysis \citep{Acemoglu2012TheChange, Fried2018ClimateAnalysis}. }
%The household receives lump-sum transfers from the government: $T_{\pi t}$ and $T_{lst}$ resulting from (i) confiscating firm profits and (ii) the carbon tax. 
The household behaves according to solving the below each period:
%Scientists, $S_t$, form part of the household. Their size is normalized to one. At the beginning of a period,  the representative household supplies scientists treating their income as part of the household budget. However, to facilitate notation, scientists' income is confiscated by the government. This is not anticipated by the household.\footnote{ Since firm profits are likewise confiscated by the government, firm profits, scientists income, and the subsidy on machine producers cancel from the government consolidated budget. The market imperfection arising from monopolistic competition is corrected for in a non-}
\begin{align*}
	\underset{C_{t}, H_{t}}{\max} & \ \
	u(C_{t},H_t)\\
	s.t.& \ \ p_{t}C_{t}\leq% (1-\tau_{\iota t})(h_{ht}w_{ht}+h_{lt}w_{lt})+T_t\\ 
	(1-\tau_{\iota t})w_tH_t+T_t,\\
	\ &  H_{t}\leq \bar{H}.
\end{align*}
The variables $w_{t}$ and $p_{t}$ indicate prices for labor and the final consumption good. Lump-sum transfers from the labor income tax are denoted by $T_t$. 

%This modeling choice facilitates notation. 
%\begin{align}
%\underset{s_{jt}}{\max}\ \ & w_{jst}s_{jt}-\chi_s \frac{s_{jt}^{1+\sigma_s}}{1+\sigma_s}
%\end{align}

%I assume that all income from science is confiscated by the government to again facilitate notation. The assumption that scientists are risk neutral, introduces an additional externality as scientists do not internalise the social value of their research on society which is shaped by the shadow value of income. The advantage of this specification is that it prevents income tax parameters to affect the supply of scientists allowing to focus on the supply of hours by workers and consumption as the channels through which income taxes affect emissions. \tr{But it would be efficient. } 

%The choice to focus on a representative family enables to abstract from inequality as a motive for government intervention. 
\paragraph{Production}
Production separates into final good production, energy production, interme- diate good production, and the production of machines and the intermediate labor input good. 
The final sector is perfectly competitive combining  non-energy and energy goods according to:
\begin{align*}
	Y_t=\left[\delta_y^\frac{1}{\varepsilon_y}E_{t}^{\frac{\varepsilon_y-1}{\varepsilon_y}}+(1-\delta_y)^\frac{1}{\varepsilon_y}N_{t}^{\frac{\varepsilon_y-1}{\varepsilon_y}}\right]^\frac{\varepsilon_y}{\varepsilon_y-1}.
\end{align*} 
I take the final good as the numeraire and define its price as $p_t=\left[\delta_yp_{Et}^{1-\varepsilon_y}+(1-\delta_y)p_{Nt}^{1-\varepsilon_y}\right]^{\frac{1}{1-\varepsilon_y}}$.
Energy producers perfectly competitively combine fossil and green energy to a composite energy good:
\begin{align*}
	E_t=\left[F_t^\frac{\varepsilon_e-1}{\varepsilon_e}+G_t^\frac{\varepsilon_e-1}{\varepsilon_e}\right]^\frac{\varepsilon_e}{\varepsilon_e-1}.
\end{align*}
The price of energy is determined as  $p_{Et}= \left[(p_{Ft}+\tau_{Ft})^{1-\varepsilon_e}+p_{Gt}^{1-\varepsilon_e}\right]^\frac{1}{{1-\varepsilon_e}}$.
The government levies a sales tax per unit of fossil energy bought by energy producers, $\tau_{Ft}$. This tax is referred to as environmental or carbon tax in this paper. 

Intermediate goods, fossil, $F_t$, green, $G_t$, and non-energy, $N_t$, are again produced in competitive sectors using a sector-specific labor input good and machines. The production function in sector $J\in \{F,G,N\}$ reads
\begin{align*}
	&J_{t}= L_{Jt}^{1-\alpha_J}\int_{0}^{1}A_{Jit}^{1-\alpha_J}x_{Jit}^{\alpha_J} di.
\end{align*}
The variable $A_{Jit}$ indicates the productivity of machine $i$ in sector $J$ at time $t$: $x_{Jit}$. 
Capital shares, $\alpha_J$, are sector specific. 
%A reduction in labor supply, however, does not affect the structure of the economy due to free movement of labor. Importantly, \cite{Fried2018ClimateAnalysis} finds a higher labor share in the fossil sector.Therefore, the green sector cannot profit as much from an increased labor supply as fossil production reduces. This effect mitigates the effectiveness of a carbon tax. 
% as the price for the fossil good increases, demand for fossil reduces. Since the two goods are substitutes, demand for green energy increases. 
Intermediate good producers maximize profits: 
\begin{align*}
	\pi_{Jt}=p_{Jt}J_t-w_{lJt}L_{Jt}-\int_{0}^{1}\left(p_{xJit}-\tau_{sJt}\right)x_{Jit}di,
\end{align*}
where $w_{lJt}$ is the price of sector $J$'s labor input good, $L_{Jt}$, and $p_{xJit}$ denotes the price of machines from producer $i$ in sector $J$. 
$\tau_{sGt}$ is a subsidy on the use of green technologies. 
Only when the firm produces in the green sector, $J=G$, then producers receive a tax on machines, i.e., $\tau_{sFt}=\tau_{sNt}=0$.

%The labor input good of sector $J$ is produced by a perfectly competitive labor industry according to:
%\begin{align*}
%	L_{Jt}=h_{hJt}^{\theta_J}h_{lJt}^{1-\theta_J}.
%\end{align*}
%This additional intermediate industry allows to capture differences in skills by sector and in particular the skill bias of the green sector: $\theta_G>\frac{1}{2}(\theta_F+\theta_N)$. 

Machine producers are imperfect monopolists searching to maximize profits. They choose the price at which to sell their machines to intermediate good producers and decide on the amount of scientists to employ. Demand for machines increases with their productivity. This provides the incentive to invest in research. Irrespective of the sector, the costs of producing one machine is set to one unit of the final output good similar to \cite{Fried2018ClimateAnalysis} and \cite{Acemoglu2012TheChange}. 
Following the same literature, machine producers only receive returns to innovation for one period. Afterwards, patents expire. Machine producer $i$'s profits in sector $J$ are given by
\begin{align*}
	\pi_{xJit}=p_{xJit}(1+\zeta_{Jt})x_{Jit}-x_{Jit}-w_{st}s_{Jit}.
\end{align*}
The government subsidizes machine production by $\zeta_{Jt}$ financed by lump-sum taxes on the household to correct for the monopolistic structure.\footnote{I introduce this policy to allow to abstract from market imperfections as a driver of the results.} 

\paragraph{Research and technology}
Technology growth is driven by research and spillovers. 
The law of motion of technology of machines from firm $i$ in sector $J$ is modeled as
\begin{align*}
	A_{Jit}=A_{Jt-1}\left(1+\gamma\left(\frac{s_{Jit}}{\rho_J}\right)^\eta\left(\frac{A_{t-1}}{A_{Jt-1}}\right)^\phi\right).
\end{align*}
Aggregate technology levels are defined as
\begin{align*}
	A_{Jt}=\int_{0}^{1}A_{Jit}di,\\
	A_{t}=\frac{\rho_FA_{Ft}+\rho_GA_{Gt}+\rho_N A_{Nt}}{\rho_F+\rho_G+\rho_N}.
\end{align*}
The parameters $\rho_J$ capture the number of research processes by sector. This ensures that returns to scale refer to the ratio of scientists to research processes \citep{Fried2018ClimateAnalysis}. 
%The number of research processes is highest in the non-energy sector. Therefore, a reduction in non-energy technology is more costly for growth in other sectors via knowledge spillovers. 
%In the baseline calibration, $\eta$ is smaller unity implying diminishing returns to research within a sector following \cite{Fried2018ClimateAnalysis}. 
Private benefits of research diverge from social ones for two reasons. First, innovation builds on ``the shoulder of giants'' introduced through the term $A_{Jt-1}$, that is, knowledge spills within sectors over time. However, producers do not internalize the effect of today's research on tomorrow's research productivity under one-period patents.  Second, they neither consider knowledge spillovers to other sectors captured by the term $\left(\frac{A_{t-1}}{A_{Jt-1}}\right)^\phi$ with $\phi\geq0$. There are no cross-sectoral knowledge spillovers when $\phi=0$.

The marginal (private) product of research determines the amount of researchers employed. It equals the competitive wage for scientists given by
\begin{align*}
	w_{st}= \frac{\eta \gamma \left(\frac{A_{t-1}}{A_{Jt-1}}\right)^\phi (1-\alpha_J)\alpha_Js_{Jt}^{\eta-1}p_{Jt}J_t}{\rho_J^\eta(1-\tau_{sJt})}.
\end{align*}
%Ceteris paribus, revenues are increasing in labor supply which is affected by the income tax. 
The parameter $\gamma$ governs research productivity. Note that the subsidy, $\tau_{sJt}$, is different from zero only for the green sector.
%The supply of scientists is endogenous in my model. With this choice, I depart from the standard assumption of a fixed supply of scientists in the literature on directed technical change \citep{Acemoglu2012TheChange, Fried2018ClimateAnalysis}.  Modeling the supply of researchers flexibly gives more freedom for the planner to choose lower growth levels: no a-priori fixed amount of research has to be employed. In light of an absolute emission limit, this could be important.
 %Furthermore, I do not assume free movement of scientists which simplifies the numeric calculation when the marginal gains of science diverge. 


%When returns to science are decreasing, $\eta<1$, then there will always be research in the economy and no-growth is not a solution. Market forces increase the marginal returns from research to infinity as the number of scientists approaches zero. Thus, under such a parameter choice, fossil technology continues to grow. To satisfy the emission limit, fossil labor and machine usage have to decline towards zero. 


%Since the aggregate level of research inputs is endogenous, the factors which determine the direction of innovation in other models, also determine the quantity of research demanded in my model. For example, when labor supply in general reduces, a market size effect curbs demand for research in all sectors. %One can show that a regressive income tax is used to boost the supply of research if demand is inefficiently low overall.

%Yet, in equilibrium, this fall in demand is absorbed by changes in the wage rate. Scientists are willing to work the same amount at the lower wage rate since the utility of consumption rises as workers work less.


%\tr{What is the effect on prices?}
%Then input shares across sectors are not constant. 


\paragraph{Markets}
In equilibrium, markets clear. I explicitly model markets for workers, scientists, and the final consumption good:
\begin{align*}
	H_{t}&=L_{Ft}+L_{Gt}+L_{Nt},\\
	S&=s_{Ft}+s_{Gt}+s_{Nt},\\
	C_t&=Y_t-\int_{0}^{1}\left(x_{Fit}+x_{Git}+x_{Nit}\right)di.  %-Gov_t.
\end{align*}
%The government does not redistribute environmental tax revenues and instead consumes the final output good captured by $Gov_t$. 
There is free movement of scientists across sectors, which seems reasonable given the 5-year duration of one period and certain research skills being applicable across sectors \citep{Fried2018ClimateAnalysis}. 

\paragraph{Government}
The government seeks to maximize lifetime utility of the representative household. Each period, the government is constrained by an emission limit, $\Omega_t$, in line with the Paris Agreement.  
It is characterized as a Ramsey planner taking the behavior of firms and households as given and discounting period utility with the household's time discount factor, $\beta$.
The planner chooses time paths for environmental and labor income taxes to solve:%\footnote{ I code the planner's problem using a primal approach going back to \cite{Lucas1983OptimalCapital} where prices and tax instruments are replaced by equilibrium equations describing the competitive equilibrium. It is straight forward to show that the Ramsey allocation is a competitive equilibrium allocation when prices and taxes are chosen adequately.}
\begin{align}
	\underset{\{\tau_{Ft}\}_{t=0}^{\infty},\{\tau_{\iota t}\}_{t=0}^{\infty}}{\max}&\sum_{t=0}^{\infty}\beta^t u(C_{t}, H_{t})%-\chi_s\frac{s_{ft}^{1-\sigma_s}}{{1-\sigma_s}}-\chi_s\frac{s_{gt}^{1-\sigma_s}}{{1-\sigma_s}}-\chi_s\frac{s_{nt}^{1-\sigma_s}}{{1-\sigma_s}}
	\nonumber \\
	s.t.\ \  %& (1)\  \tau_{\iota t}(h_{ht}w_{ht}+h_{lt}w_{lt})=T_t\  \forall \ t\geq 0\\
	&  \omega F_{t} -\delta \leq \Omega_t, \label{eq:emslim} % \ \hspace{3mm} \forall t \in\{0,T\}, 
	\\ %\hspace{3mm} \text{(emission target)}\\
	&  \tau_{\iota t}w_tH_t=T_t, \label{eq:incbud}\\
	&  \tau_{Ft}F_{t}=\tau_{sGt}x_{Gt}.\label{eq:envbud}
\end{align}
subject to the behavior of firms and households, and feasibility\footnote{Feasibility means that the government is constrained by initial technology levels, time endowments of workers and scientists, and production processes prescribed by the model.}. 
%- emission constraint
Constraint \eqref{eq:emslim} is the emission limit. The parameter $\delta$ captures the capacity of the environment to reduce emitted CO$_2$ through sinks, such as forests and moors.\footnote{ In the model, sinks are assumed to be constant. I argue for this choice in Section \ref{sec:modpar}.}  The parameter $\omega$ determines  CO$_2$ emissions per unit of fossil energy produced. %I abstract from other greenhouse gases. This keeps the model simple while accounting for the pollutant the most relevant for mitigation policies.

%- gov budgets
Revenues from income taxation are rebated lump sum, eq. \eqref{eq:incbud}.
Finally, eq. \eqref{eq:envbud} characterizes the politically feasible policy: the government recycles environmental tax revenues as subsidies to the use of green machines.

%\paragraph{Setup of problem: focus on current population; continuation value}

\subsection{Calibration}\label{subsec:calib}

%\tr{Inflation data \url{/home/sonja/Documents/projects/subjective_BN/writing/mainmain}}

Section \ref{sec:ems} derives and discusses the emission target. 
Secton \ref{sec:modpar} calibrates the remaining model parameters.

\subsubsection{Emission target}\label{sec:ems}
To calibrate the emission target, I consider CO$_2$ emissions only and abstract from other greenhouse gasses since carbon is the most important pollutant with the highest mitigation potential \citep[p.29]{IPCC2022}.
%	 WG3 IPCC report (p.37) \textbf{\textit{The trajectory of future CO$_2$ emissions plays a critical role in mitigation, given CO$_2$ long-term impact and dominance in total greenhouse gas forcing}}. Furthermore, \textbf{The main reason is that scenarios reduce non-CO$_2$ greenhouse gas emissions less than CO$_2$ due to a limited mitigation potential (see 3.3.2.2)} p.34 in foxit, 3-26 in chapter 3}.  
The most recent IPCC report \citep{IPCC2022} formulates a reduction of global CO$_2$ emissions in the 2030s by 50\% relative to 2019 and net-zero emissions in the 2050s  as essential to meeting the 1.5°C climate target.\footnote{ ``\textit{Mitigation pathways limiting warming to 1.5°C [...] reach 50\% reductions of CO$_2$ in the 2030s, relative to 2019, then reduce emissions further to reach net zero CO$_2$ emissions in the 2050s [...] (\textnormal{medium confidence}).}" \citep[p.5, Chapter 3]{IPCC2022} }  Furthermore, the report stipulates a remaining global net CO$_2$ budget of 510 GtCO$_2$ %($\approx$ 510,000 million metric tons of CO$_2$) 
from 2020 to the net-zero phase starting from 2050 \citep[p.5, Chapter 3]{IPCC2022}. 
To deduce an emission target for the US, further assumptions on the distribution of mitigation burdens have to be made. I follow \cite{RobiouDuPont2017EquitableGoals} who consider 5 distinct principles of distributive burden sharing. I use an \textit{equal-per-capita} approach according to which emissions per capita shall be equalized across countries. 
 Appendix \ref{app:calib} details the calculation of the emission target. 
Figure \ref{fig:emlimit}  visualizes the resulting emission limit for the US starting from 2020. The value for 2015-2019 refers to observed emissions.

% data
%, 2022: Mitigation pathways compatible with long-term goals. In IPCC, 2022: Climate
% Change 2022: Mitigation of Climate Change. Contribution of Working Group III to the Sixth
% Assessment Report of the Intergovernmental Panel on Climate Change [P.R. Shukla, J. Skea, R.
% Slade, A. Al Khourdajie, R. van Diemen, D. McCollum, M. Pathak, S. Some, P. Vyas, R. Fradera, M.
% Belkacemi, A. Hasija, G. Lisboa, S. Luz, J. Malley, (eds.)]. Cambridge University Press, Cambridge,
% UK and New York, NY, USA. doi: 10.1017/9781009157926.005
% 


%\begin{table}[hh!!!!!]
%	\begin{center}
%		\captionsetup{width=0.9\textwidth}
%		\caption{Net CO$_2$ emission limit for the US by model period}
%		\label{tab:emlimit}
%		\begin{tabular}{l|rrrrrrrr}
	%			\hline 
	%			\hline
	%			Periods&20-24&25-29&30-34&35-39&40-44&45-49&50-80\\
	%			Limits in GtCO$_2$&3.6079&3.5396&3.4798&3.4245&3.3697&3.3164&0\\
	%			\hline \hline
	%			
	%		\end{tabular}
%	\end{center}
%\end{table}	

\begin{figure}
\caption{Net CO$_2$ emission limit in gigatons  (Gt)}\label{fig:emlimit}
%	\graphicspath{{../../codding_model/own_basedOnFried/optimalPol_010922_revision/figures/all_13Sept22_Tplus30/}{../../codding_model/own_basedOnFried/optimalPol_010922_revision/figures/all_13Sept22/}}
\includegraphics[width=0.4\textwidth]{../../../subjective_BN/codding_model/own_basedOnFried/optimalPol_010922_revision/figures/all_13Sept22_Tplus30/Emnet.png}
\end{figure}
%  In summary, I calibrate the net-emission target vector for the period from 2030 to 2080 as 
% $\omega_{2030-2050}$= 2.4899Gt and $\omega_{2050-2080}$= 0Gt.
%\footnote{Another alternative 
%} 
% I assume here that each country contributes to the global reduction by the same percentage of 50\% of its own emissions.\footnote{ Alternatively, one could assume that the global reduction is allocated in the same share as countries contributed to global emissions in 2019. This would result in an even stricter target for the US which contributed almost 20\% to global greenhouse gas emissions in 2019 (based on own calculations where total emissions come from the EIA global greenhouse gas information, to be found here \url{https://www.iea.org/reports/global-energy-review-2021/CO$_2$-emissions}).}
% Starting from 2050, the net-emission target is zero. 
% sinks and emission from fossil sector

\paragraph{Discussion}
The reduction in net CO$_2$ emissions necessary to meet the emission limit relative to 2019 emissions in the US  is substantial. It amounts to around 85\%. The result is not only explained by the global emission limit but also by the US emitting beyond its population share in 2019. In 2019, US emissions accounted for 10.44\% of global net emissions while the population share of the US was 4.3\%. Hence, even without an emission limit, the US would have to reduce emissions according to the \textit{equal-per-capita} principle.

The necessary reduction in net CO$_2$ emissions found in this calibration exceeds political goals. On April 22, 2021, President Biden announced a 50-52\% reduction in net greenhouse gas emissions relative to 2005 levels in 2030 % \footnote{ If pollutants were to be reduced by an equal share, this means a 50-52\% reduction in net CO$_2$ emissions.} 
and net-zero emissions no later than 2050.\footnote{ Source: \href{https://www.whitehouse.gov/briefing-room/statements-releases/2021/04/22/fact-sheet-president-biden-sets-2030-greenhouse-gas-pollution-reduction-target-aimed-at-creating-good-paying-union-jobs-and-securing-u-s-leadership-on-clean-energy-technologies/}{https://www.whitehouse.gov/briefing-room/statements-releases/2021/04/22/}, retrieved 14 September 2022.} 
However, relative to 2019, the planned reduction for 2030 corresponds to a 38\% decline only.
The resulting net emissions in the US would then amount to 103.21 Gt.\footnote{ This calculation assumes emissions where left at 2019-levels until 2030 and then lowered to the Biden target from 2030 to 2050 and net-zero afterwards.} This is roughly 5 times the budget acceptable for the US,  if the global remaining carbon budget was allocated on a \textit{equal-per-capita} basis.\footnote{ The remaining net carbon budget for the US based on its population share is 20.738Gt for the period from 2020 to 2050.} % This amounts to 27\% of emissions which the US would emit if annual emissions equaled 2019 net emissions.}  

\subsubsection{Model parameters}\label{sec:modpar}

\paragraph{Functional forms}
I assume the following functional form of period utility:
\begin{align*}
u(C_t,H_{t}, )= \log(C_t)-\chi\frac{H_{t}^{1+\sigma}}{{1+\sigma}}.
\end{align*}
The log-utility implies constant hours worked over time in a laissez-faire allocation since income and substitution effects cancel. This simplifies the analysis. %\footnote{  On the other hand, recent research has shown that substitution and income effects of the wage rate most likely do not cancel. \cite{Boppart2019LaborPerspectiveb} argue for a slightly higher income effect so that hours fall over time as productivity increases. I plan to conduct a sensitivity analysis by assuming the utility specification suggested in their paper.}

%Most likely, the continuous rise in carbon taxation over time lowers the wage rate and labor supply increases. A lack of lump-sum rebates would most likely aggravate the inefficiency of hours worked. %Nevertheless, as shown in the analytical part for a general utility function, some reductive policy is required to implement the efficient allocation. But, compared to the laissez-faire scenario, hours will rise. 

\paragraph{Parameter values}
To calibrate the model, I proceed in three steps. First, I set certain parameters to values found in the literature. Second, I calibrate the remaining variables requiring that equilibrium conditions and target equations hold. Third, parameters relating production and emissions are chosen. \autoref{tab:calib2} summarizes the calibrated parameter values.

I calibrate the model to the US in the baseline period from 2015 to 2019. Using this calibration approach, it is not ensured that the economy is on a balanced growth path. However, the goal of this paper is to study necessary interventions to meet an absolute emission limit. Therefore, %in contrast to a relative reduction objective, 
it is important to capture whether the economy is transitioning, for example, to %a balanced growth path with
a higher fossil share. The optimal dynamic policy has to counteract these forces. %These transitions are relevant for the dynamic policy. 
%To differentiate model dynamics from policy effects, I take care to interpret results as deviations from the economy without policy intervention. 


In the first step, I mainly rely on \cite{Fried2018ClimateAnalysis} to calibrate the parameters governing research processes, $\eta, \rho_F,\rho_N, \rho_G, \phi, \gamma, S $, and production, $\varepsilon_e, \varepsilon_y, \alpha_F, \alpha_G, \alpha_N$. The labor share in the green sector is remarkably low with $\alpha_G=0.91$. This diminishes the significance of labor supply for green innovation and production. Furthermore, fossil and green energy are no close substitutes with $\varepsilon_e=1.5$ so that the cap on fossil energy cannot be fully substituted for by green energy.
Returns to research are decreasing with $\eta=0.79<1$. This makes extreme distributions of researchers across sectors unproductive. The non-energy sector is the biggest research sector with $\rho_N=1$ and $\rho_F=\rho_G=0.01$. 
The utility parameters, $\beta, \sigma$, are set to $0.984^5$ and $0.75^{-1}$ following \cite{Barrage2019OptimalPolicy} and \cite{Chetty2011AreMargins}, respectively. The business-as-usual policy is set to $\tau_\iota=0.24, \tau_F=0$, where I borrow the tax rate from \cite{Barrage2019OptimalPolicy}. 
%The period over which the government maximizes, T, is chosen to focus on the population living during the transition to the net-zero emission limit. 
%One can think of the T as the periods under the regency of the government. I set T to 11 so that the planner 
%explicitly derives  allocations and polices for 55 years. In their overlapping-generations model, \cite{Kotlikoff2021MakingWin} use the same number to calibrate the working life of a household as it captures the years a household is typically active in economic markets. 
%\tr{ Regency: T=11=55 years, and explicit optimization over T+1 periods. 55 is a suggests to be a sensible number for the explicit optimization interval.  }

In the second step, I calibrate the weight on energy in final good production by matching the average expenditure share on energy relative to GDP over the period from 2015 to 2019 taken from the US Energy Information Administration \citep[][Table 1.7]{EIAEnergy}. The expenditure share equals 6\%. The resulting weight on energy is $\delta_y=0.30$. %\footnote{ Note that in difference to \cite{Fried2018ClimateAnalysis} I raise the weight on intermediate inputs in final production to the power $\frac{1}{\varepsilon_y}$, so that in the limit the function approaches the Leontief specification as $\varepsilon_y\rightarrow 0$ \citep{Herrendorf2014GrowthTransformation}.}
 The disutility of labor, $\chi$, is set to match equilibrium average hours worked to average hours over the period from 2015-2019 drawing from OECD data \citep{OECDHoursworked}, $\chi=9.66$. I normalize total economic time endowment for workers and scientists per day, which I set to 14.5 as found in \cite{Jones1993OptimalGrowth}, to 1. 

 Initial productivity levels follow from normalizing output in the base period to $Y=1$ and matching the ratio of fossil-to-green energy utilization over the years 2015-2019 which equals 7.33 according to \cite[][Table 1.3]{EIAEnergy}. I find that total factor productivities in the baseline period are $A_{N0}^{1-\alpha_N}=1.90$, $A_{F0}^{1-\alpha_F}=4.52$, and $A_{G0}^{1-\alpha_G}=1.17$. %Since the green and fossil energy good are no close substitutes with $\varepsilon_e=1.5$, the fossil sector has to be technologically more advanced to 

Finally, I calibrate the sink capacity to match the average difference between gross and net CO$_2$ emissions over the baseline period from 2015 to 2019. Information on emissions comes from the US Environmental Protection Agency \citep{EPAems}. Since sinks are relevant for all greenhouse gasses, I only use the proportion of total sink capacity which reflects contribution of carbon dioxide to gross greenhouse gas emissions. The resulting sink capacity per model period is $\delta=3.19$GtCO$_2$.\footnote{ I consider this capacity to be constant. This is a simplifying assumption. What is crucial qualitatively is the assumption that sinks are finite. Indeed, natural sinks and carbon capture and storage (CCS) technologies rely on the use of land \citep{VanVuuren2018AlternativeTechnologies} which is in limited supply. In addition, the importance of land for food production makes land even scarcer especially in light of a growing world population.}
The parameter relating CO$_2$ emissions and fossil energy in the base period equals $\omega=345.33$.\footnote{  I perceive the fossil sector in the model as source of all CO$_2$ emissions including, for instance, non-energy use of fuels and incineration of waste.}  

\begin{table}[h!]
\begin{center}
\captionsetup{width=0.9\textwidth}
\caption{ Calibration}
\label{tab:calib2}
\resizebox{5in}{!}{
	\begin{tabular}{c|ll}
		%			\hline \hline
		%			\multicolumn{7}{c}{Calibration based on basic needs}\\
		\hline \hline
		Parameter& Target/Source& \makecell[l]{Value}\\ 
		\hline
		Household&\multicolumn{2}{c}{}\\
		\hline 
		$\sigma$ &  \makecell[l]{\cite{Chetty2011AreMargins}}& $1.33$  \\
		$\chi$ &  \makecell[l]{average hours worked per\\ economic time endowment\\ by worker: 0.34 \citep{OECDHoursworked}}& 9.66 \\
		$\beta$ &  \makecell[l]{\cite{Barrage2019OptimalPolicy}}& 0.93 \\
		$\bar{H}$& \makecell[l]{14.5 hours per day\\ \cite{Jones1993OptimalGrowth}}&1.00 \\
		\hline
		Research&\multicolumn{2}{c}{}
		\\
		\hline 
		$\eta$ & & 0.79 \\
		($\rho_F$, $\rho_G$, $\rho_N$) & & (0.01, 0.01, 1.00) \\
		$\phi$ &\makecell[l]{\cite{Fried2018ClimateAnalysis}} & 0.50 \\
		$S$ && 0.01\\
		$\gamma$ && 3.96\\
		\hline
		Production&\multicolumn{2}{c}{}\\
		\hline
		($\varepsilon_y$, $\varepsilon_e$)&\cite{Fried2018ClimateAnalysis}&(0.05, 1.50)\\			
		$\delta_y$&\makecell[l]{expenditure share \\ on energy \citep{EIAEnergy}}&0.30\\	
		($\alpha_F$, $\alpha_G$, $\alpha_N$)&\cite{Fried2018ClimateAnalysis} &(0.72, 0.91, 0.36)\\
		%\hline
		%$\beta$&\makecell{ annual nominal rate 3\%\\ and annual inflation rate of 2\%}& 0.9903& discount factor\\ 
		\hline
		Initial total factor productivity&\multicolumn{2}{c}{}\\
		\hline
		($A_{F0}^{1-\alpha_F}$, $A_{G0}^{1-\alpha_G}$, $A_{N0}^{1-\alpha_N}$)& energy shares \citep{EIAEnergy} &(4.12, 1.17, 1.90)  \\
		\hline 
		Government&\multicolumn{2}{c}{}\\
		\hline
		$\tau_F$&- &0.00\\
		$\tau_{\iota}$&\cite{Barrage2019OptimalPolicy} &0.24\\
		\hline
		Emissions&\multicolumn{2}{c}{}\\
		\hline
		$\delta$& \makecell[l]{\cite{EPAems}}&3.19\\
		$\omega$& \cite{EPAems}&345.33\\
		\hline \hline
\end{tabular}	}
\end{center}
\end{table}

%According to the IEA, global greenhouse gas emissions from fuel combustion amounted to 34.2 Gt in CO$_2$ equivalents in 2019.\footnote{ Retrieved from \url{https://www.iea.org/reports/global-energy-review-2021/CO$_2$-emissions} on February 2, 2022.} I use the share the US contributed to global emissions in 2019, 19.18\%, to proxy the share in reductions I require the US to contribute to total reductions from 2019 to 2030. 

% procedure


% \textit{Convergence towards equal annual emissions per person} as a fair allocation of reductions. Then US emissions per capita should equal world emissions per capita. 
% I use the UN projected population measure to proxy for future population size.
%  The calibration is done with respect to CO$_2$ emissions. 



%Hence, the smallest adjustment follows from equal budgets per period. 
%I reduce each limit in the same proportion in the 2035-2050 period so that the remaining budget for the US for the period 2020 to 2035 

%This result leads to the following emission limits
%From 2020 to 2035 there is a total budget of net-CO$_2$ emissions of 10.627Gt for the US. From 2035 to 2050 model-period emissions may amount to [2.900, 2.854, 2.809].\footnote{ I use here that in earlier test runs the emission limits have been fully exploited. }

% \clearpage

%\thispagestyle{plain}
% \clearpage
%
%\paragraph{Sources data}
%%\url{https://www.eia.gov/totalenergy/data/monthly/#prices}
%
%Total energy data: 
%For data on skill and premium see references in 
%paper saved in data \citep{Slavik2020WagePremium}
%
%The model is calibrated to parameter values common in the literature. I bestow more care on  calibrating the emission target. 
%I match emissions in the model to emission targets suggested in the IPCC report \citep{Rogelj2018MitigationDevelopment.}. 
%%How to determine the economy in 2050? Should the economy have reached a steady state? or should it be in a transitional path? Maybe no need to specify this...it will be a outcome. All I have to use is that for all years after 2050 net-emissions have to be zero. Whether the economy is on the transitional path or in a steady state is an outcome. 
%The IPCC prescribes net-zero emissions starting from 2050. In 2030 emissions should be between 25 and 30 GtCO$_2$e per year.
%

%

%\clearpage
%\tableofcontents
\section{Quantitative results}\label{sec:res}

In this section, I present and discuss the quantitative results. Under the benchmark policy regime, the government uses carbon tax revenues to subsidize the green sector. Furthermore, a labor income tax is available. 
First, it is informative to consider how a social planner meets the emission limit. Relative to this benchmark, I discuss the optimal allocations a government implements with the ``politically feasible" policy with and without labor income tax in Section \ref{sec:allos}. Section \ref{sec:optpol} discusses the optimal policy.  % Add: comparison to non-exogenous growth => differences in labor tax

%This section depicts results on the optimal policy followed by the implied allocation in the benchmark model where environmental tax revenues are redistributed via the income tax scheme. 

%\begin{itemize}
%	\item emission limit can be obtained at a smaller carbon tax. the economy profits from a more productive ratio of green to fossil energy 
%	\item effect on green technology growth 
%\end{itemize}

\subsection{Efficient and optimal allocations}\label{sec:allos}

\autoref{fig:allo} depicts the efficient, or first-best, allocation, in dashed-gray, the optimal allocation with and without labor taxes, the solid-black and the dashed graph, and the laissez-faire allocation, dotted-black. The x-axis indicates the first year of the 5 year period to which the variable value corresponds. 
To attain the emission limit, the efficient allocation sees a massive increase in the energy ratio from fossil to green energy from close to zero under laissez-faire to around 70 to 1 in the 2070s; see the dashed-gray graph in \autoref{fig:allo} Panel (a). This shift comes at the cost of less consumption relative to a laissez-faire scenario; see Panel (b). The transition is characterized not only by an immediate reduction in consumption but also by a smaller growth rate of consumption. The reason is that the green sector is less productive, but also that a shift to green research entails smaller growth rates due to dynamic knowledge spillovers which make fossil research more productive. 
The reduction in productivity makes it efficient to lower work effort. Even though consumption reduces and the marginal unit of output becomes more valuable\textemdash similar to an income effect\textemdash the social planner prefers lower work effort. The reason is that the marginal product of labor decreases so much\textemdash similar to a substitution effect. Overall, the substitution effect dominates and less work effort becomes desirable. 

The optimal allocation without labor tax, the dashed-black graph, features a strong reduction in the use of fossil fuels. Even more than in the efficient allocation. Despite this less productive use of technologies, the reduction in consumption is mitigated. The reason is that households work more hours than under the social planner allocation. Hence, labor supply is inefficiently high given the reduction in the marginal product of labor.  % As the green sector becomes more productive over time, labor supply rises. 

Allowing for labor income taxes, the government can correct for the inefficiency in labor supply. Panel (c) visualizes how hours move closer to their efficient level when a labor income tax is available; the solid-black graph. Because of the smaller work effort, the level of production declines which allows for a smaller green to fossil energy ratio, hence a more productive allocation of resources, while meeting the emission limit; compare Panel (a). This mitigates the reduction in output as hours decline. 


\begin{figure}[h!!]
	\centering
	\caption{Efficient and Optimal Allocation }\label{fig:allo}
	\begin{minipage}[]{0.45\textwidth}
		\centering{{(a) Green to Fossil Energy}}
		%	\captionsetup{width=.45\linewidth}
		\includegraphics[width=1\textwidth]{../../codding_model/own_Paper/optimalPol_140723_revision/figures/all_30Aug23/Levels_eff2pol_LF_pol2_T_GFF_emnet0_subsres0_knspil0_sigma0_Bop0_util0_lgd1.png}
	\end{minipage}
%	\begin{minipage}[]{0.05\textwidth}
%		\
%	\end{minipage}

\vspace{4mm}
\begin{minipage}[]{0.45\textwidth}
\centering{{(b) Consumption }}
%	\captionsetup{width=.45\linewidth}
\includegraphics[width=1\textwidth]{../../codding_model/own_Paper/optimalPol_140723_revision/figures/all_30Aug23/Levels_eff2pol_LF_pol2_T_C_emnet0_subsres0_knspil0_sigma0_Bop0_util0_lgd0.png}
\end{minipage}

\vspace{4mm}
	\begin{minipage}[]{0.45\textwidth}
		\centering{{(c) Hours Worked }}
		%	\captionsetup{width=.45\linewidth}
		\includegraphics[width=1\textwidth]{../../codding_model/own_Paper/optimalPol_140723_revision/figures/all_30Aug23/Levels_eff2pol_LF_pol2_T_H_emnet0_subsres0_knspil0_sigma0_Bop0_util0_lgd0.png}
	\end{minipage}
%\begin{minipage}[]{0.05\textwidth}
%\
%\end{minipage}
%\begin{minipage}[]{0.45\textwidth}
%	\centering{{(d) Green Technology }}
%	%	\captionsetup{width=.45\linewidth}
%	\includegraphics[width=1\textwidth]{../../codding_model/own_Paper/optimalPol_140723_revision/figures/all_30Aug23/Levels_eff2pol_LF_pol2_T_Ag_emnet0_subsres0_knspil0_sigma0_Bop0_util0_lgd0.png}
%\end{minipage}
\end{figure} 
\newpage

\subsection{Optimal policy}\label{sec:optpol}

What policy implements the optimal allocation? \autoref{fig:optPol} shows the optimal policy in the regime with and without labor income tax. Satisfying the emission limit necessitates a massive tax on carbon which reaches levels close to 4000 US\$ in the 2070s. The jump in 2050 follows from the more stringent net-zero emission limit. Knowledge spillovers from non-fossil to the fossil sector explain the rising pattern of carbon taxes. Knowledge from green and non-energy research makes fossil researches more productive. Market forces lead to a reallocation of researchers to this sector. Overall, a higher carbon tax has to counter this effect. 

The optimal policy is accompanied by a positive tax on labor. The marginal tax on labor is around 6\% before 2050 and jumps to slightly below 9\% in 2050. It gradually reduces to around 7.5\%. These tax rates make up between 25\% to 38\% of the calibrated tax rate of 24\%.  Given the choice of parameters, the substitution effect dominates the income effect and households reduce their labor supply as labor is taxed. In contrast to the policy regime without labor tax, the planner has to set a higher carbon tax especially under the net-zero emission target: The higher level of production requires to meet the emission limit at a higher share of green to fossil energy usage. 

\begin{figure}[h!!]
	\centering
	\caption{Optimal Policy }\label{fig:optPol}
	\begin{minipage}[]{0.45\textwidth}
		\centering{{(a) Tax per ton of CO$_2$ in 2022 US\$\\ \ }}
		%	\captionsetup{width=.45\linewidth}
		\includegraphics[width=1\textwidth]{../../codding_model/own_Paper/optimalPol_140723_revision/figures/all_30Aug23/Optimal_comp_pol2_T_Tauf_emnet0_subsres0_knspil0_sigma0_Bop0_util0_lgd1.png}
	\end{minipage}
	\begin{minipage}[]{0.05\textwidth}
		\
	\end{minipage}
	\begin{minipage}[]{0.45\textwidth}
		\centering{{(b) Labor Income Tax \\ \ }}
		%	\captionsetup{width=.45\linewidth}
		\includegraphics[width=1\textwidth]{../../codding_model/own_Paper/optimalPol_140723_revision/figures/all_30Aug23/Optimal_comp_pol2_T_taul_emnet0_subsres0_knspil0_sigma0_Bop0_util0_lgd0.png}
	\end{minipage}
%	\begin{minipage}[]{0.32\textwidth}
%		\centering{\footnotesize{(b) Labor Tax }}
%		%	\captionsetup{width=.45\linewidth}
%		\includegraphics[width=1\textwidth]{../../codding_model/own_Paper/optimalPol_140723_revision/figures/all_30Aug23/Optimal_comp_pol2_T_taus_emnet0_subsres0_knspil0_sigma0_Bop0_util0_lgd0.png}
%	\end{minipage}
	%\begin{minipage}[]{0.32\textwidth}
	%	\centering{\footnotesize{(c) Net emissions\\ \  }}
	%	%	\captionsetup{width=.45\linewidth}
	%	\includegraphics[width=1\textwidth]{../../codding_model/own_basedOnFried/optimalPol_010922_revision/figures/all_13Sept22_Tplus30/Single_OPT_T_NoTaus_Emnet_regime0_spillover0_knspil0_noskill0_sep0_xgrowth0_extern0_PV1_sizeequ0_GOV0_etaa0.79.png}
	%\end{minipage}
\end{figure} 

\begin{comment}
\begin{figure}[h!!]
	\centering
	\caption{Optimal Policy }\label{fig:optPol_LStrans_Subs}
	\begin{minipage}[]{0.45\textwidth}
		\centering{\footnotesize{(a) Carbon Tax }}
		%	\captionsetup{width=.45\linewidth}
		\includegraphics[width=1\textwidth]{../../codding_model/own_Paper/optimalPol_140723_revision/figures/all_30Aug23/Comp_LSTrans_Subs_T_Tauf_emnet0_subsres0_knspil0_sigma0_Bop0_util0_lgd1.png}
	\end{minipage}
%	\begin{minipage}[]{0.1\textwidth}
%		\
%	\end{minipage}
	\begin{minipage}[]{0.45\textwidth}
		\centering{\footnotesize{(b) Labor Tax }}
		%	\captionsetup{width=.45\linewidth}
		\includegraphics[width=1\textwidth]{../../codding_model/own_Paper/optimalPol_140723_revision/figures/all_30Aug23/Comp_LSTrans_Subs_T_taul_emnet0_subsres0_knspil0_sigma0_Bop0_util0_lgd0.png}
	\end{minipage}
\end{figure} 
\end{comment}
	

%
\subsubsection{Effect of model features on optimal policy}\label{subsec:xgrnsk}

To complete the optimal policy discussion, I address the role of certain model features. 
Figure \ref{fig:comp_mod} presents the optimal policy in a model with exogenous growth, the blue dashed graphs, without skill heterogeneity, the orange dotted graphs, and without knowledge spillovers. The black solid line indicates the results in the benchmark model. 

\paragraph{Endogenous growth}
When growth is exogenous, the optimal income tax progressivity is lower than in the benchmark model during the initial 10 years and higher thereafter (panel (a)). Furthermore, the increase in tax progressivity over time is more pronounced.  

The environmental tax is higher in all periods when growth is exogenous (panel (b)). This finding is in line with \cite{Fried2018ClimateAnalysis} who argues that directed technical change amplifies the recomposing effects of the fossil tax. The higher price for fossil energy increases demand for green energy. Since innovation responds to this shift in demand, the slowdown in fossil production is amplified. 

%The reduction in consumption below the respective efficient allocation is only slightly more than 10\% at maximum whereas consumption falls short of the efficient level by up to 55\% in the benchmark model; figure \ref{fig:comp_mod_allo_dev} in the appendix shows how optimal allocations deviate from the efficient one for the three models studied.

The rationale behind the higher tax progressivity and its attenuated decline in the exogenous growth model is that, first, due to the more aggressive carbon tax, a stronger reduction in labor supply is efficient. Second, when growth does not respond to the change in the high-to-low skill ratio, the compositional effect of a progressive income tax is muted. Furthermore, the social planner chooses an increasing path in hours worked during the net-zero period in the benchmark model. As output declines the income effect makes the social planner value additional output more. This is replicated by the falling tax progressivity. 
The more regressive tax in the initial 10 periods allows for a higher green-to-fossil output ratio. 

%\footnote{\ Compare figures \ref{fig:LF_vs_onlytaul_xgrnsk} and \ref{fig:LF_vs_onlytaul_xgr} which show models with exogenous growth but without and with skill heterogeneity, respectively.}
%When growth is endogenous, the skill-recomposition channel dominates making income taxes less advantageous from an environmental policy perspective.
 %Since the fossil sector relies more on labor than the green sector, the reduction in labor makes fossil production relatively more expensive. This increases the green-to-fossil energy mix. Same holds true for the composition of the final output good which is recomposed towards the less labor intense energy good. However, with endogenous growth, this channel is muted and the adverse recomposing effect of labor income tax progressivity on the green energy share dominates. 
%Overall, the beneficial recomposing effect in the exogenous growth model explains the higher labor tax progressivity and the muted decline over time. 

%Figure \ref{fig:count_taul_xgr} shows how the economies evolve when the optimal income tax from the benchmark model is fed into (1) the exogenous growth model, the black graph, and (2) the exogenous growth model, the blue dashed graph. The respective grey graphs show the allocation in the laissez-faire economy.
%
%
%The reason for the higher tax progressivity in the exogenous growth model relative to the benchmark model is not driven by the effect of income tax progressivity on growth. Growth seems largely unaffected by the income tax; compare panel (e) for the level of growth and panel (f) for the ratio of green to fossil scientists. The effect on innovation is absorbed by wages which increase with the labour income tax; see panels (l) and (m).  
%
%Rather, the higher tax progressivity in the model with exogenous growth arises from the higher environmental tax and the complementarity of the two instruments. This narrative becomes clear when comparing how the social planner adjusts labor supply differently in the two models, see figure \ref{fig:eff_model}. The efficient level of hours worked is lower in the exogenous growth model. Not because it is less costly to reduce hours worked, but because the environmental tax is more aggressive. The channel analyzed in the analytical section.  




%\tr{is it the recomposiing effect or the effect on growth in general? \ar both not present---Cant tell }


\begin{figure}[h!!]
	\centering
	\caption{Optimal policy by model}\label{fig:comp_mod}
	
	\begin{minipage}[]{0.32\textwidth}
		\centering{\footnotesize{(a) Income tax progressivity, $\tau_{\iota t}$\\ \ }}
		%	\captionsetup{width=.45\linewidth}
		\includegraphics[width=1\textwidth]{../../codding_model/own_basedOnFried/optimalPol_010922_revision/figures/all_13Sept22_Tplus30/CompMod1_OPT_T_NoTaus_taul_regime0_spillover0_knspil0_sep0_extern0_PV1_etaa0.79_lgd0.png}
	\end{minipage}
	\begin{minipage}[]{0.32\textwidth}
		\centering{\footnotesize{(b) Environmental tax, $\tau_{Ft}$\\ \ }}
		%	\captionsetup{width=.45\linewidth}
		\includegraphics[width=1\textwidth]{../../codding_model/own_basedOnFried/optimalPol_010922_revision/figures/all_13Sept22_Tplus30/CompMod1_OPT_T_NoTaus_tauf_regime0_spillover0_knspil0_sep0_extern0_PV1_etaa0.79_lgd0.png}
	\end{minipage}
\begin{minipage}[]{0.3\textwidth}
	\includegraphics[width=1\textwidth]{../../codding_model/own_basedOnFried/optimalPol_010922_revision/figures/all_13Sept22_Tplus30/legend_compmod1.png}
\end{minipage}
\end{figure}

\begin{figure}[h!!]
	\centering
	\caption{Optimal policy by model}\label{fig:comp_mod_Tls}
	
	\begin{minipage}[]{0.32\textwidth}
		\centering{\footnotesize{(a) Income tax progressivity, $\tau_{\iota t}$\\ \ }}
		%	\captionsetup{width=.45\linewidth}
		\includegraphics[width=1\textwidth]{../../codding_model/own_basedOnFried/optimalPol_010922_revision/figures/all_13Sept22_Tplus30/CompMod1_OPT_T_NoTaus_taul_regime4_spillover0_knspil0_sep0_extern0_PV1_etaa0.79_lgd0.png}
	\end{minipage}
	\begin{minipage}[]{0.32\textwidth}
		\centering{\footnotesize{(b) Environmental tax, $\tau_{Ft}$\\ \ }}
		%	\captionsetup{width=.45\linewidth}
		\includegraphics[width=1\textwidth]{../../codding_model/own_basedOnFried/optimalPol_010922_revision/figures/all_13Sept22_Tplus30/CompMod1_OPT_T_NoTaus_tauf_regime4_spillover0_knspil0_sep0_extern0_PV1_etaa0.79_lgd0.png}
	\end{minipage}
\begin{minipage}[]{0.3\textwidth}
\includegraphics[width=1\textwidth]{../../codding_model/own_basedOnFried/optimalPol_010922_revision/figures/all_13Sept22_Tplus30/legend_compmod1.png}
\end{minipage}
\end{figure}
\begin{comment}

Despite the more aggressive intervention, consumption in the Ramsey allocation only deviates by -10\% from the efficient allocation over the whole time period considered (panel (a) in figure \ref{fig:comp_mod_allo}). In contrast, in the benchmark model, the deviation of consumption aggravates over time reaching -55\% relative to the respective efficient allocation. 
Interestingly, the recomposing effect of tax progressivity through its impact on the skill ratio is negligible in the exogenous growth model. Indeed, the skill ratio diverges more from the efficient ratio due to the higher tax progressivity (panel (e)). However, since the direction of growth does not respond to the supply of skills, the green-to-fossil energy mix is roughly similar to the efficient one (panel (d)). 

content...
\end{comment}

%- homogenous skill
\paragraph{Skill heterogeneity}
When there is only one type of skill, the optimal income tax progressivity is higher, as well, than in the benchmark model. Yet, the difference is less pronounced compared to the model with exogenous growth. Besides, optimal tax progressivity converges to the one in the benchmark model over time. 
As argued above, endogenous growth masks the advantageous recomposing effect of tax progressivity on the energy mix through the labor-share channel. 

Compared to the benchmark model, the planner does not face a trade-off between too low high-skill supply and too high low-skill supply and can implement hours close to the efficient level.\footnote{ Panel (b) in figure \ref{fig:comp_mod_allo_dev} in the appendix shows deviations from the respective efficient allocation by model. When skill heterogeneity is switched off, labor supply only deviates at maximum by -0.6\% from the efficient allocation. In the benchmark model the same number rises to -3.5\% and +1\% for the high- and the low-skill type. } This explains the higher tax progressivity. 
Furthermore, there is no increase in fossil research as is the case in the benchmark model relative to the laissez-faire allocation.\footnote{\ See figures \ref{fig:LF_vs_onlytaul} and \ref{fig:LF_vs_onlytaul_nsk} which show the effect of the respective optimal  tax progressivity relative to the laissez-faire allocation in the benchmark model and the one with skill homogeneity, respectively.}
%Indeed, the deviation in consumption and growth from the efficient allocation is similar in the model with homogeneous skills to the benchmark model (panels (a) and (g)). 

The fossil tax is higher than in the benchmark alternative throughout, the orange dotted graph in panel (b).
On the contrary to income tax progressivity, the environmental tax diverges more from the benchmark model when there is no skill heterogeneity relative to the alternative with exogenous growth. The motive behind this result is the increased input market for fossil production when there is only one skill type. A smaller low-skill supply, therefore, contributes to lower emissions. I discuss the mechanism in more detail in appendix section \ref{app:count}.

 
 % <- analysis of optimality results with different model features
\section{Conclusion}\label{sec:con}
Some scholars argue that  reductive policies are necessary to handle environmental limits \citep{Schor2005SustainableReductionb, VanVuuren2018AlternativeTechnologies, Bertram2018TargetedScenarios}, and the question has been raised whether consumption is too high \citep{Arrow2004AreMuch}. On the other hand, the focus of environmental policy discussions in economics rests on corrective environmental taxation. In the light of tightening environmental limits \citep{Rockstrom2009AHumanity, IPCC2022}, I study whether labor income taxes - as a reductive policy tool - can help mitigate externalities. 

In the analytical part of the paper, I show in a simple model that labor income taxes are  progressive as part of the optimal environmental policy. %The model does not feature inequality.
% Quantitative results
% baseline model
When environmental tax revenues are not redistributed lump sum, labor supply is inefficiently high. Then, income taxes serve to diminish hours worked closer to the efficient level. The result prevails absent income inequality.


% quantitative
In the second part of the paper, I analyze in a quantitative model with skill heterogeneity and endogenous growth whether the optimal labor income tax remains progressive. Again, there are no equity concerns, but workers are perfectly ensured against income differences. 
The optimal income tax is progressive to reduce inefficiently high hours worked. The quantitative model reveals that income taxes also serve as a substitute for corrective taxes. Knowledge spillovers from the non-energy sector render environmental taxes especially costly. 
Fossil taxes make energy relatively more expensive which directs research from non-energy to energy sectors. As the non-energy sector features the most research processes it is especially important for aggregate technology and knowledge spillovers. Using income taxes instead of fossil taxes to lower emissions allows to direct more research to the non-energy sector and to profit from knowledge spillovers.
In sum, however, the reduction in labor supply outweighs the positive effect on growth and consumption decreases compared to a scenario where no income tax is used. 

In the quantitative setting, the income tax affects the economic structure through two channels. First, because the fossil sector is comparably labor intense, a reduction in labor supply favors the green sector. This mechanism makes a higher tax progressivity optimal. However, the effect vanishes in equilibrium due to endogenous growth.
Second, a skill-recomposition channel makes green energy production more costly compared to fossil production. This effect arises from a skill bias in the green sector and high-skill labor being more responsive to income taxation. 
The second channel dominates the recomposing effect of  income tax progressivity in equilibrium. A market size effect amplifies the skill-recomposition channel directing research to the fossil sector. 

%Initially, the intention not to harm growth too much makes a lower progressivity optimal. As growth in the fossil sector accelerates due to the dynamic structure of endogenous growth too low progressive income taxes conflict with meeting the emission limit. As a result, optimal progressivity increases over time.
%The optimal path of income tax progressivity is decreasing, a feature mainly driven by endogenous growth. As a result, the optimal income tax progressivity and the optimal fossil tax seem to behave like substitutes in the quantitative model. 

%Skill heterogeneity depresses optimal tax progressivity due to the adverse recomposing effect of a lower high-to-low skill labor supply on the green-to-fossil energy ratio. A higher corrective tax is required to meet emission limits when there is only one skill type: with only one skill the supply of fossil-specific inputs increases thereby violating the emission limit.

%% lump-sum transfers
%When environmental tax revenues are redistributed lump-sum, the motive to use labor income taxes to deal with inefficiently high labor supply vanishes. Instead, income taxes serve to boost growth as long as this does not conflict with meeting emission limits. Therefore, they are regressive. 
%\tr{not true! it is rather that the more in research is not worth it given the dynamics! and decreasing utility gains}
%However, the regressivity decreases since more labor supply causes more emissions especially the more progressed the technology. With only a labor income tax as a tool to raise growth, accelerating technology growth is not feasible as it is concomitant with more production and emissions. 

% extensions
In an extension, I am planning to give the Ramsey planner the opportunity to limit working hours directly. The literature advocating a reduction in consumption levels \citep[e.g.,][]{Schor2005SustainableReductionb} proposes a restriction of hours worked as policy instrument to lower the consumption of resources.
Even though advocated in the literature, there is evidence for political difficulties in reducing working hours. In 2020, the French Citizens' Convention on Climate voted against reducing working hours as a measure to handle climate change. Potentially, ignorance about economic consequences is an explanation. The extension would serve to better understand economic consequences. 



\begin{comment}
\paragraph{Extension: What if the low skilled get a higher share \ar they reduce even less \ar more fossil input supply}

Redistribution to households with a higher marginal propensity to consume emissions counteracts the externality. This effect is amplified by a market size effect  of dirty goods. 

content...
\end{comment}

% I plan to discuss results under counterfactual parameter values to elicit the robustness of the main result: the preference of progressive labour taxation above higher fossil taxes. 
%First, the productivity gap between sectors might be driving the results. Second, I will abstract from endogenous growth to learn about the labour-supply-innovation channel as a driver of the optimal policy. Finally, I plan to study how results change as returns to research are increasing within sector. 
%Due to the endogeneity of technological growth in the model, the reduction in work effort fosters less research especially in the non-energy sector.  %However, more hours worked in the Ramsey model fostering research would violate the emission target. As a result, growth in technology and in consumption is inefficiently low in order to meet the emission target. 

\begin{comment}
To shed more light on the main findings, I plan conduct several additional quantitative experiments. First, I want to reduce the size of the emission target, second, I allow for a longer time frame until net-zero emissions have to be reached. The IPCC report states that for a temperature target of 2°C net-zero emissions have to be reached by 2070 only. How does this laxer target affect the importance of labour income taxes. Given the wider time frame, the green sector might be able to catch up and growth could continue. Finally, how does a change in spillovers shape the result? % \textit{(Question: I guess that substitutability is key here! Growth in green implies growths in fossil when goods are no perfect substitutes! )}
content...

%Another central aspect of the paper is the importance of inequality for the optimal environmental policy. How does household heterogeneity in labour supply shape the optimal environmental policy? First, I hypothesise that the skill bias of the green sector makes a less progressive income tax optimal. 
One main result of the paper is reduction of consumption and work effort as an optimal policy. So far, I have assumed that households are passive and preferences are fixed; there is no trade-off between environmental quality and consumption from a household perspective.
In an extension to the baseline model, I plan to depart from the representative agent assumption and explicitly model household heterogeneity. This setting allows to capture a change in household behaviour: A share of households is willing to voluntarily reduce consumption. I provide evidence for such behaviour using a representative Dutch dataset. More than 50\% of households are willing to reduce consumption in order to help the economy. Importantly, these households have a higher likelihood to work in the green sector. How does such a change in behaviour affect the optimal policy? Given the additional reduction in green-specific labour supply, the planner might find it optimal to set a more regressive tax to booster green production and research.    

\end{comment}

%However, data suggests, that households do care, and they express a willingness to reduce consumption.\footnote{\ The data I have studied comes from the Liss Panel, a representative sample of Dutch households, more than 50\% of participants indicate a readiness to change their behaviour to help the environment.} I want to study the effect of such behavioural  change on the optimal policy. Interestingly, households in high-skill jobs are more likely to declare their willingness to reduce. This linkage may intensify the trade-off between reduction and green labour supply. 


%1) BN and inequality
%2) preferences for labour
\begin{comment}
Preferences and the trade-off between leisure and consumption determining household behaviour seem to be key to the results. As argued by \cite{Boppart2019labourPerspectiveb}, the intensive margin of hours worked have been falling steadily over the last 130 years. They argue for the consistency of preferences which feature a slightly higher income effect than substitution effect. In the current model with log-utility and representative family framework,  the substitution effects offset each other. With the preferences suggested in \cite{Boppart2019labourPerspectiveb}, growth would affect hours worked, assumably changing the optimal policy. It could, for instance, be the case, that growth has to be slowed down even more, to prevent too high work efforts and consumption levels. % high-income, high-skill households might increase their labour supply with growth. 

content...
\end{comment}



%Finally, endogenising growth constitutes another interesting trade-off when the impact of fiscal policy is skill specific. 
%As regards growth, it seems reasonable to consider growth as a change in the substitutability of dirty and clean goods in the final consumption good. As it stands now, growth in the dirty sector results in emission growth, ceteris paribus. Growth might instead be associated with a more efficient use of dirty energy sources, so that more output can be generated at lower emissions.
%
%Think about effects of government using revenues for other consumption. Then reducing demand will diminish demand for the final good. 
%Broadly speaking, there are two channels through which distortionary labour taxation affects emissions. First, by affecting households' labour supply decision (efficiency channel) and second in a mechanical way by changing households disposable income. The latter effect cancels out when tax revenues are used by the government to consume the final output good. Allowing the government to recycle revenues in a different way than for final good consumption uncloses another instrument to reduce emissions. 

%Further ideas for extensions: include behavioural aspects: a voluntary reduction in demand, and a lower disutility from working in the green sector.
\begin{comment}
\paragraph{Ways forward}
How to introduce compositional effects:
\begin{enumerate}
	\item 	Utility function: With substitution and income effect not canceling (u(c)=$\frac{c^{1-\gamma}}{1-\gamma},\ \gamma\neq 1$), the wage rate might play a role, depends on GE effects.
	\item endogenising skill supply (rep agent chooses how much skill to supply, but this he already does... / might need to introduce structure as in HSV)
	\item government revenues are not used for final good consumption. Instead,  disposed of/ used for sth useful (this could be an extension and contribute to benefits of progressivity) THINK THIS ONLY CHANGES THE LEVEL TOO!
\end{enumerate}
\paragraph{Point 1 above}
change the utility function in the code to see what happens, if $\frac{Y_d}{Y_c}$ is constant in particular 
\paragraph{Point 3 above}
\textcolor{blue}{2) Government consumption wasted}
Letting the government not consume the final output good may alter the result. 
Now, the aggregate price level is determined endogenously as the goods market does not clear by Walras' law. 

In the equilibrium equations, I drop $p_t=1$ and use goods market clearing instead\\ $Y=c+\psi (x_c+x_d)$.

Blödsinn, only changes level

content...
\end{comment}
%
\section{Going forward: 12 August 22}

Importantly, the equity and the environmental targets of government intervention are perceived as competing goals as both tax instruments exert efficiency costs through a reduction in labor supply.
I argue in this paper that what is perceived as an efficiency cost -  the reduction in labor supply - is part of the optimal environmental policy. Hence, income taxation has a double dividend: an environmental and an equity one.   




\begin{itemize}
	\item to think about: why does the ramsey allocation not achieve the same growth rates as the efficient one? would need additional instruments.
	\item !!! why does the green-to-fossil ratio increase when the labor income tax is progressive in the model with exogenous growth?
	\item compare environmental tax in model without labor income tax and gov consumption to model with lump sum redistribution and without labor income tax \ar how does the environmental tax change? Is it higher?
	\item why this downward sloping tax progressivity? The social planner also chooses an increasing labor supply; could depends on the marginal utility of consumption, but consumption increases. Why does work become more valuable again? 
	Bcs due to rising technology labor becomes more productive. 
	\item analytically derive mechanisms driving direction of innovation in model
	\item check CEV calculation and report results
	\item results without PV to understand dynamics!
	\item include results in other policy regimes\\
	Finally, in section \ref{subsec:comp_lumpsum}, I turn to analyze the optimal allocation under the alternative policy regimes: redistribution of environmental tax revenues via (1) lump-sum transfers and (2) the income tax scheme. 
	%
	\item \tr{integrated policy has an advantage even absent externality when there is endogenous growth!}
	\item why is there the strong deviation in the green to fossil energy ratio in the non-skill version with endogenous growth \ar has to stem from the higher environmental tax? \checkmark 
	\item \tr{Idea: could be that endogenous growth intensifies shift in skill ratio \ar if so, would expect a smaller reaction of skill supply in xgr model (but then the policy is differen!) Would need experiment with same policy; counterfactual}
	\item also why is there a higher environmental tax in the non-skill model? Why is it needed? \checkmark
\end{itemize}
\textbf{extensions:}
\begin{itemize}
	\item other derivation of emission target not reduction by 50\% but weighted by what the country contributes, for instance
	
	\item which  regime is best for equity measured in terms of utility, wages?lump-sum transfers or additional progressive taxes? \ar Evaluate by looking at high and low skill wages. 
	\item think about \textbf{involuntary unemployment} \ar shouldn't there be gains from an overall reduction in labor supply in terms of involuntary unemployment? As households want to work less?
	\item empirical studies motivated through this paper's results?
	\item  What if there is a pre-existing income tax but not gov funding constraint? could still find an increase in labor income tax if optimal level is above initial level
	\item  look at a policy where income taxes are used to fund government spending \ar then this would generate more gov. revenues
	\item commment on whether there is a double dividend from using income taxes
	\item add inequality and heterogenous consumption bundles to model and ask about changes in the distribution of income
\end{itemize} 

\subsection{Literature}

\paragraph{endogenous growth and distortionary taxes}
Fullerton and Kim 2008
\cite{Bovenberg1997EnvironmentalGrowth}
Hettich 1998
Lighart and van der Ploeg 1994
\cite{Loebbing2019NationalChange}


\paragraph{reduction policies and their optimality}
\begin{itemize}
	\item due to social preferences (envy, keeping up with the Joneses, habits) \ar read Layard 2006 on Happiness
	\item due to environmental limits
	\item do not include LIMITS to GROWTH literature (this seems to be a different question )
\end{itemize} 

\paragraph{Pigou and optimal tax deviation}

%%%------------------------------------------------
% Deviation from the pigou principle: The effect of fiscal distortions on the optimal environmental policy
%%%--------------------------------------------------------------------

Another realm the literature which combines fiscal and environmental policy considers is the deviation of the environmental tax from the Pigou principle. In the classic setup, the government faces an exogenous funding condition. This motivates using labor income taxes to generate funds. 
The environmental tax reduces the wage rate - an efficient decline from an environmental perspective - depressing the tax base of the income tax if the uncompensated wage elasticity of labor is positive.  This is why the two motives of government intervention, the environmental externality and generating revenues, compete.\footnote{\ Importantly, using environmental tax revenues to fund the government is more costly as opposed to labor income taxes due to an additional distortion generated from envrionmental taxes: environmental taxes distort commodity in addition to reducing labor supply. This argument rejects the strong double dividend hypothesis and was brought forward by \cite{LansBovenberg1994EnvironmentalTaxation}. } 
In order to satisfy the funding constraint, the optimal environmental tax falls short of the social costs of the externality; the Pigou principle is violates \citep{LansBovenberg1996OptimalAnalyses}. \cite{Barrage2019OptimalPolicy} studies the role of fiscal distortions for the optimal environmental policy in a dynamic setting with climate cycle. 
The results presented in this paper connect to the considerations on whether and why the environmental tax deviates from the social cost of the externality. While it is the second target, i.e., to generate funds, which rationalizes a lower environmental tax, the reason of the deviation in my paper emerges from the non-redistribution of environmental tax revenues. This reduces consumption and thereby utility so that the government seeks to reduce environmental tax revenues.

\subsection{Sensitivity}
I will now briefly discuss sensitivity analyses to the quantitative exercise. 
\subsubsection{Wage elasticity of labor}

Recent papers have examined the wage elasticity of labor. \cite{Boppart2019LaborPerspectiveb} present evidence that hours worked per worker have been falling steadily over time 

\subsubsection{Research subsidy}
but finding should be similar to version without endogenous growth
\subsubsection{Changing emission limits calculation}
The \textit{equal-per-capita} approach is favorable for population high countries like the US. Therefore, in the sensitivity analysis, I rerun the model where US emission limits follow from \textit{Equal cumulative per capita} approach, where countries with historically  high emissions per capita mitigate more.
% \textit{constant-emission-ratio} approach which is achieved by all countries reducing emissions by 50\% in the 2030s relative to 2019. This principle limits US emissions to 2.309Gt in the 2030s. THIS APPROACH IS EVEN MORE FAVOURABLE  TO THE US BECAUSE CURRENTLY THEY EMIT A HIGHER SHARE PER CAPITA THAN THE REST OF THE WORLD. 

\subsubsection{Technology gap}

\paragraph{Sensitivity}
\begin{itemize}
	\item utility specification (Building on Bick can think of European version when substitution effect is stronger)
	
	Since the target of the labor tax in the environmental setting presented here is to align hours worked with their efficient level, results are sensitive to the elasticity of labor with respect to after-tax wages. 
	Quantitative finding to be shaped by income and substitution effect!
	Literature on how households react to changes in income \cite{Bick2018HowImplications} and \cite{Boppart2019LaborPerspectiveb}
	
	\item spillovers across scientists: with positive spillovers potentially no growth \ar then connects to degrowth!
\end{itemize}

\begin{comment}
\section{Model}\label{app:model}

content...


\subsection{Solving the model}
%Demand for intermediate goods determines the price ratio, $\frac{p_g}{p_f}$ in equilibrium, equation \eqref{eq:ana_dem_fin}. 
%Intermediate good market clearing, i.e. substituting intermediate good production functions, equations \eqref{eq:ana_prod_F} and \eqref{eq:ana_prod_G}, in \eqref{eq:ana_dem_fin} yields
%\begin{align}
%\frac{p_g}{p_f}=\frac{A_f}{Ag}\frac{s}{1-s}\frac{1-\varepsilon}{\varepsilon},
%\end{align}
%where I used that $s=\frac{L_F}{h}$ and the labor market clearing condition. Solving the price definition of the final goods price, equation \eqref{eq:ana_pr} for $p_g$ as a function of $\frac{p_g}{p_f}$ and substituting the previous expression for the price ratio gives the price of the clean good in equilibrium as a function of labor shares:
%\begin{align}
%p_g=(1-\varepsilon)\left(\frac{A_f}{A_g}\right)^\varepsilon\left(\frac{s}{1-s}\right)^{1-\varepsilon}.
%\end{align}
%The equilibrium price paid for the clean good 



\subsection{Inefficiency in the wage rate with externality}

The wage rate in the competitive equilibrium is below the marginal product of hours worked. 
The reason is that the lower labor share in the dirty sector has to be sustained by a tax on dirty production. 
Otherwise, market forces would equilibrate the marginal product of labor in both sectors and $\varepsilon=s$. Yet, this disregards the negative externality of dirty production. 

In the competitive equilibrium, hence, the environmental tax serves to sustain the wedge between marginal products of labor across intermediate sectors: it is higher in the dirty and lower in the green sector.

This comes at the cost of lowering the aggregate wage rate in the economy to the marginal product of labor in the green sector. 
Nevertheless, the marginal product of labor in the fossil sector is higher. As a result, the aggregate wage rate, $w$, falls short of the aggregate marginal product of labor in the economy. This mechanism on its own renders labor supply inefficiently low in the competitive economy. However, as shown in proof \ref{prop:1}, the equilibrium hours supplied are inefficiently high in the competitve allocation. 


When the effect of a decreased wage  en gros reduces labor supply, it makes it more costly for the government to generate funds due to a smaller tax base for the income tax. This is the mechanism pointed to by the double dividend literature.


\end{comment}
\clearpage
\appendix
\section{Derivations and proofs}\label{app:derivations}

\subsection{Theory results \ref{sec:mod_an}}
\subsubsection{Useful relations in the simple model}\label{app:dervs_use}
\begin{align*}
\frac{\partial Gov}{\partial s}=\frac{\partial Y}{\partial F}\frac{\partial F}{\partial s}+\frac{\partial Y}{\partial G}\frac{\partial G}{\partial s}-\frac{\partial C}{\partial s}\\
\frac{\partial Gov}{\partial H}=\frac{\partial Y}{\partial F}\frac{\partial F}{\partial s}+\frac{\partial Y}{\partial G}\frac{\partial G}{\partial s}-\frac{\partial C}{\partial H}\\
\frac{\partial Gov}{\partial s}=\frac{\partial Y}{\partial s}-\frac{\partial C}{\partial s}\\
\frac{\partial Gov}{\partial s}=p_f F \frac{\partial \tau_F}{\partial s}+\tau_F F \frac{\partial p_f}{\partial s}+\tau_F p_f \frac{\partial F}{\partial s}\\
%\frac{\partial \tau_F}{\partial s}= -\frac{1-\varepsilon}{\varepsilon}\frac{1}{(1-s)^2}, \\
%\frac{\partial \tau_F}{\partial s}=p_f\frac{1-\varepsilon}{1-\tau_F}\frac{\partial \tau_F}{\partial s}\\
\frac{\partial F}{\partial s}=\frac{F}{s}
\\
\frp{Y}{H}= \frp{Y}{s}\frac{s}{H}+\frp{Y}{G}\frp{G}{Lg}
\\
\frp{G}{H}=-\frac{(1-s)}{H}\frp{G}{s}
\\\frp{G}{s}=-H\frp{G}{L_G}\\
\frp{F}{H}=\frac{s}{H}\frp{F}{s}\\
\frp{F}{s}=H\frp{F}{L_F}
\end{align*}

\subsubsection{Reduction in dirty labor share is efficient}
\begin{proof}
	With a negative externality of dirty production it has to hold that 
	\begin{align}
	\frp{Y}{F}\frp{F}{s}>-\frp{Y}{G}\frp{G}{s},
	\end{align}
	which can be rewritten to 
	\begin{align}\label{eq:mpl_eff}
	\frp{Y}{L_F}>\frp{Y}{L_G}. 
	\end{align}
	In the efficient allocation absent externality, marginal products of dirty and green labor are equalized. 
	Under decreasing returns to scale it holds that the left-hand side is decreasing in $L_F$ and the right-hand side of equation \ref{eq:mpl_eff} is decreasing in $L_G$. Hence, the adjustment to satisfy equation \ref{eq:mpl_eff} relative to the efficient allocation without externality requires a decrease in $L_F$ and/or a rise in $L_G$  .
	This reallocation is achieved by reducing $s$, since $L_F=sH$ and $L_G=(1-s)H$.	
\end{proof}


\begin{comment}
content...
\paragraph{If a reduction in dirty labor share is efficient, then the aggregate production function features decreasing returns to scale in labor}
\begin{proof}
	\textit{The proof rest on the assumption that returns to scale are symmetric across dirty and clean production; either both decreasing or both are non-decreasing.}
It holds by assumption that $s_{FB,E>0}<s_{FB,E=0}$, where $E>0$ indicates that the externality is active. 
Assume by contradiction that the aggregate production function features non-decreasing returns to scale. This implies that:
\begin{align}
\left. \frp{Y}{L_F} \right|_{s_{FB,E>0}}\leq \left. \frp{Y}{L_F} \right|_{s_{FB,E=0}},\\
\left. \frp{Y}{L_G} \right|_{s_{FB,E>0}}\geq \left. \frp{Y}{L_G} \right|_{s_{FB,E=0}}.
\end{align}
When there is no externality, the efficient allocation is characterized by
\begin{align}
\left. \frp{Y}{L_F} \right|_{s_{FB,E=0}}= \left. \frp{Y}{L_G} \right|_{s_{FB,E=0}}.
\end{align}
Using the inequalities above yields
\begin{align}
\left. \frp{Y}{L_F} \right|_{s_{FB,E>0}}\leq \left. \frp{Y}{L_G} \right|_{s_{FB,E>0}}.
\end{align}
This contradicts the optimality condition which requires 
\begin{align}
\left. \frp{Y}{L_F} \right|_{s_{FB,E>0}}> \left. \frp{Y}{L_G} \right|_{s_{FB,E>0}}.
\end{align}
Hence, when a reduction in the dirty labor share is efficient, then the aggregate production function features decreasing returns to scale in both labor input goods. 
\end{proof}
\end{comment}

\subsubsection{The social cost of pollution and the Pigouvian tax rate}\label{app:scp}

The social cost of pollution in my model is defined as the marginal price the representative household is willing to pay for a marginal reduction in dirty production. That is, the household maximises over dirty production for which a market exists.

The household's problem is determined as
\begin{align}
\underset{C,H,F}{\max} U(C,H,F)-\mu \left(C+\tilde{p}_FF-Y(H)\right).
\end{align}
Where $\mu$ is the Lagrange multiplier. Taking the derivative with respect to dirty production  and with respect to consumption yields
\begin{align}
U_F=\mu \tilde{p}_F,\\
U_C=\mu.
\end{align}
Substituting the Lagrange multiplier gives the negative of the equilibrium price the household is willing to pay for a reduction in dirty prodction: $\tilde{p}_F=\frac{U_F}{U_C}$. Since the environmental tax in the model is a percentage of revenues, the price producers pay per unit of dirty production is $\tau_F p_F$. Thus, the social cost of pollution to be deducted from to producers' revenues in percent is $\tau^{Pigou}=\frac{-U_F}{U_Cp_F}$.


\subsubsection{With a positive environmental tax, the wage rate in the competitive equilibrium is below the marginal product of labor}\label{app:wageMPL}

The aggregate marginal product of labor is defined as
\begin{align}
MPL&= \frp{Y}{H}.
\end{align}
This expression can be rewritten using relations of derivatives summarized in \ref{app:dervs_use} as follows.
\begin{align}
&= \frp{Y}{F}\frp{Y}{H}+\frp{Y}{G}\frp{G}{H}\\
&= \frp{Y}{F}\frp{F}{L_F}s+\frp{Y}{G}\frp{G}{L_G}(1-s)\\
&= \frp{Y}{G}\frp{G}{L_G}+ s\left(\frp{Y}{F}\frp{F}{L_F}-\frp{Y}{G}\frp{G}{L_G}\right).\label{eq:mpl_opt}
\end{align}
The term in brackets is positive under the optimal policy as can be seen from the first order condition with respect to $s$, equation \ref{eq:sbs}:
\begin{align}
\frp{Y}{F}\frp{F}{L_F}-\frp{Y}{G}\frp{G}{L_G}=\frac{1}{H}\left(\frp{Y}{F}\frp{F}{s}+\frp{Y}{G}\frp{G}{s}\right)=\frac{1}{H}\left(\frac{-U_F\frp{F}{s}}{U_C}\right)>0.
\end{align}
The inequality holds since the externality of polluting production is negative. %, above expression is positive.
%Therefore, the marginal product of labor in the efficient allocation equals
Now note that the first summand in equation \ref{eq:mpl_opt} is the competitive wage rate.  Hence $w<MPL$.

The gap between the wage rate and the marginal product of labor equals the gap between the marginal products of labor across sectors times the relative size of the dirty sector. 

\subsubsection{Sufficiency of the environmental tax when environmental tax revenues are redistributed lump sum}\label{app:incometax0}

Noticing that $\frac{\partial Y}{\partial H}= \frac{\partial Y}{\partial s}\frac{s}{H}-\frac{\partial Y}{\partial G}\frac{\partial G}{\partial s}\frac{1}{H}$ and that $\frac{\partial F}{\partial H}=\frac{\partial F}{\partial s}\frac{s}{H}$, and substituting equation \ref{eq:sbs} in equation \ref{eq:sbh} yields
\begin{align}\label{eq:pigou}
-U_C \frac{\partial Y}{\partial G}\frp{G}{L_G}=-U_H.
\end{align}
Hence, if the environmental tax is set to guarantee that condition \ref{eq:sbh} holds, then optimal hours worked only trade-off the disutility from labor and the utility from more consumption when environmental tax revenues are redistributed lump-sum.

Equation \ref{eq:pigou} also holds for the social planner allocation simplifying the second first order condition, equation \ref{eq:fbh}.


Substituting $U_H$ from household optimality, equation \ref{eq:hsup}, and the clean sectors' profit maximizing condition from equations \ref{eq:profmax} yields
\begin{align}
1=1-\tau^*_\iota.
\end{align}
Hence, $\tau^*_\iota =0$ from which follows that $\lambda =1$ so that the income tax scheme is a flat tax rate equal to zero; the labor income tax is not used in optimum.

%\subsubsection{Simplifying social planner's first order conditions}
%
%The social planner's first order condition on labor can be rewritten as in the previous section to
%\begin{align}
%-U_H=U_C\frac{\partial Y}{\partial G}\frp{G}{L_G}
%\end{align}
\subsubsection{Proof proposition \ref{prop:1}: Absent lump-sum transfers, hours are inefficiently high under decreasing returns to scale}\label{app:nolumpsum_hourshigh}
\begin{proof}\textit{Absent lump-sum transfers, hours are inefficiently high when the environmental tax implements efficient share of dirty production and the aggregate production function features decreasing returns to scale in labor inputs.}
	
	This proof proceeds by contradiction. 
	Assume by contradiction that $H^*\leq H_{FB}$. 
	It has to hold that 
	\begin{align}
	-U_H^*\leq -U_{H,FB}.
	\end{align} 
	
	Substituting the households' optimal labor supply and the social planner's first order condition for hours, equation \ref{eq:fbh_simp} yields
	\begin{align}\label{eq:prH}
	U_C^*w^* \leq U_{C,FB}\frp{Y_{FB}}{G_{FB}}\frp{G_{FB}}{L_{G,FB}}.
	\end{align}
	
	Rewriting equation \ref{eq:prH} above yields
	\begin{align}
	\frac{U_C^*}{U_{C,FB}}\leq \frac{\frp{Y_{FB}}{G_{FB}}\frp{G_{FB}}{L_{G,FB}}}{\frp{Y^*}{G^*}\frp{G^*}{L^*_{G}}},
	\end{align}
	where I replaced $w^*=\frp{Y^*}{G^*}\frp{G^*}{L^*_{G}}$.
	
	By assumption $s^*=s_{FB}$, $H^*\leq H_{FB}$, and the aggregate production function is increasing in its inputs. It follows that output is higher in the efficient allocation $Y_{FB}\geq Y^*$ and hence $C^*<C_{FB}$, since $Gov>0$ in the competitive equilibrium. By additive separability of the utility function and strict concavity with respect to consumption, we have that $\frac{U_C^*}{U_{C,FB}}>1$.
	
	Now note that $H^*\leq H_{FB}$ implies  $L_G^*\leq L_{G,FB}$, since the dirty labor share is equal. Under decreasing returns to scale of aggregate production to clean labor, it holds that the right-hand side is below or equal unity.Thus,
	\begin{align}
	\frac{U_C^*}{U_{C,FB}}>1\geq \frac{\frp{Y_{FB}}{G_{FB}}\frp{G_{FB}}{L_{G,FB}}}{\frp{Y^*}{G^*}\frp{G^*}{L^*_{G}}}. 
	\end{align}
	A contradiction to the assumption that $H^*\leq H_{FB}$. Hence, it has to hold that $H^*>H_{FB}$. 
\end{proof}


\subsubsection{Derivation $\tau_F^*$ without lump-sum transfers}\label{app:reiv_tauf}
	
Divide the Ramsey planner's first order condition with respect to $s$, equation \ref{eq:sbs}, by $U_C$ and $\frp{Y}{F}\frp{F}{s}$. Solving for $1+\frac{\frac{\partial Y}{\partial G}\frac{\partial G}{\partial s}}{\frac{\partial Y}{\partial F}\frac{\partial F}{\partial s}}$, which equals $\tau_F$, yields the desired result:

\begin{align}
\tau_{F}=SCC + \frp{Gov}{s}.
\end{align}

\begin{comment}
The latter summand can be rewritten to 
\begin{align}
\frp{Gov}{s}= \frp{Y}{s}+H^2 \frp{\left(\frp{Y}{L_G}\right)}{L_G}.
\end{align}
Where under decreasing returns to scale the second summand is negative and the first is positive. \textit{To be continued.} 

content...
\end{comment}
\subsubsection{Derivation $\tau_l$ without lump-sum transfers }\label{app:subsub_nltaul}

\begin{proof}\textit{Absent lump-sum transfers, the optimal income tax scheme is progressive}
Following similar steps as in section \ref{app:incometax0}, the optimal labor income tax progressivity parameter is given by
\begin{align}
\tau_{\iota}^*=\frac{\frac{s}{H}\frac{\partial Gov}{\partial s}- \frac{\partial Gov}{\partial H}}{\frac{\partial Y}{\partial G}\frac{\partial G}{\partial s}\frac{1}{H}}.
\end{align}

	Using the market clearing condition for final output to replace government spending and noticing the relations of derivatives with respect to aggregate labor supply and the dirty labor share, one can write above expression as
	\begin{align}
	\tau_{\iota}=1-\frac{H\frp{C}{H}-s\frp{C}{s}}{wH}.
	\end{align}
	Substituting $\frp{C}{H}=H\frp{w}{H}+w$ and $\frp{C}{s}=H\frp{w}{s}$ from the household's budget constraint gives
	\begin{align}
	\tau_{\iota}=\frac{s}{w}\frp{w}{s}-\frac{H}{w}\frp{w}{H}.
	\end{align}
In a next step, I explicitly solve for $\frp{w}{s}$ and $\frp{w}{H}$, where I use that $w=\frp{Y}{G}\frp{G}{L_G}$ in equilibrium.

\begin{align}
\frp{w}{H}=\left(\frp{G}{L_G}\right)^2\frac{\partial^2Y}{\partial G^2}(1-s)+\frp{Y}{G}\frac{\partial ^2G}{\partial L_G^2}(1-s)+\frp{G}{L_G}\frac{\partial^2 Y}{\partial G \partial F}s\\
%%%%
\frp{w}{s}= \left(\frp{G}{L_G}\right)^2\frac{\partial ^2Y}{\partial G^2}(-H)+\frp{G}{L_G}\frac{\partial ^2Y}{\partial G \partial F}H+\frp{Y}{G}\frac{\partial ^2 G}{\partial L_G^2}(-H)
\end{align}
substituting derivatives and canceling terms yields:
\begin{align}
\tau_\iota= -\frac{H}{w}\frp{\left(\frp{Y}{L_G}\right)}{L_G}.=-\frac{H}{w}\left(\left(\frp{G}{L_G}\right)^2\frac{\partial ^2Y}{\partial G^2}+\frp{Y}{G}\frac{\partial ^2G}{\partial L_G ^2}\right).
\end{align}
Under the assumption of decreasing returns to scale of aggregate production with respect to green labor the term in brackets is negative, and it holds that $\tau_\iota >0$ and the optimal income tax rate is progressive. 

For intuition, note that the right-hand side of the previous expression equals the partial derivative of the wage rate with respect to the dirty labor share under the assumption that dirty production is fixed divided by the wage rate:
\begin{align}
\tau^*_\iota =\left. \frac{1}{w}\frp{w}{s} \right|_{F=\bar{F}}.
\end{align}
%Since the presence of the environmental tax artificially increases labor in the green sector depressing the wage rate (under the assumption of decreasing returns to scale), the wage rate rises by a reduction of the green labor share. 

The equation makes clear that environmental taxation and the labor income tax are complements. When the environmental tax rises, thereby increasing the share of labor allocated to the green sector, the marginal product of green labor decreases further. A marginal reduction in the green labor share would increase the wage rate more the higher the green labor share, hence, the optimal labor tax progressivity increases with the environmental tax. 
Secondly, the wage rate decreases with $\tau_F$ which as well inflates the optimal labor tax progressivity. 
	\end{proof}

\subsubsection{Proof proposition: Infeasibility of efficient allocation}\label{app:ineff}
\begin{proof}\textit{The efficient allocation is infeasible (under the assumption of constant or decreasing returns to scale)}
	To prove this claim, I assume that the government chooses the optimal policy; which is the highest social welfare the Ramsey planner can achieve. I show that the optimal policy does not satisfy the social planner's allocation. Since the social planner could have chosen the Ramsey planner's allocation  but did not, it follows that the social planner's allocation features a higher social welfare.
	
	For the optimal allocation to be efficient, it must be the case that $s^*=s_{FB}$, (i) $C^*=C_{FB}$, and (ii) $H^*=H_{FB}$. I show that, under the assumption that $s^*=s_{FB}$, either (i) or (ii) can hold at a time by demonstrating that assuming (i) violates (ii) and vice versa.
	
	
	
	%\begin{lemma}\textit{$\tau_F=0$ is not optimal}
	%When $\tau_F=0$ then $Gov=0$ and $\frp{Gov}{s}=0$. Furthermore, market forces then imply that the marginal products of labor are equal so that $\frp{Y}{F}\frp{F}{s}=-\frp{Y}{G}\frp{G}{s}$. Substituting this in equation \ref{eq:sbs} yields
	%\begin{align}
	%0=-U_F\frp{F}{s}>0,
	%\end{align}
	%a contradiction. 
	%\end{lemma}
	%
	%\textit{(i) Assume $C^*=C_{FB}$ and $s^*=s_{FB}$:}
	%Since $\tau_F\neq0$, it follows that $Gov>0$ and hence $Y^*=C^*+Gov>Y_{FB}$. Since the allocation of labor is the same in the efficient and the optimal allocation and output is rising in labor, it follows that $H^*>H_{FB}$. 
	%\tr{Missing: if $\tau_F<0$ then $Gov<0$ }
	
	If $s^*=s_{E>0,FB}<s_{E=0,FB}$ then it must be the case that the environmental tax is positive to sustain a gap between marginal productivities in the dirty and the clean sector: $\tau_F>0$. Then, $Gov=\tau_Fp_fF>0$. 
	First assume that (i) holds true: $C^*=C_{FB}$. From the good's market clearing condition and resource constraint of the social planner's problem it follows that
	$Y^*-Gov=C^*=C_{FB}=Y_{FB}$, due to  $Gov>0$ we have that $Y^*>Y_{FB}$. Since hours are the only production input, positively affect output, and $s^*=s_{FB}$ the higher output in the optimal allocation implies that $H^*>H_{FB}$. A violation of condition (ii). 
	
	Assume now that condition (ii) holds: $H^*=H_{FB}$. Since $s^*=s_{FB}$ by assumption it holds that $Y^*=Y_{FB}$ and, by the same argument as before: $Gov>0$. Thus, by the resource and market clearing condition: $C_{FB}=Y_{FB}>Y^*-Gov=C^*$. When labor supply is efficient, then consumption is inefficiently low; condition (i) is violated. 
	
	\begin{comment} (Proof building on first order conditions)
	Assume, 
	The social planner's first order condition on labor supply can be written as
	\begin{align}
	-U_{H, FB}=U_{C, FB}\frp{Y}{G}_{FB}\frp{G}{L_G}_{FB}
	\end{align}
	and optimal labor supply is determined by
	\begin{align}
	-U^*_{H}&=U^*_C(1-\tau_\iota)w
	\end{align}
	Equalizing yields
	\begin{align}
	U_C^*(1-\tau_\iota)w=U_{C,FB}\frp{Y}{G}_{FB}\frp{G}{L_G}_{FB},
	\end{align}
	a condition for optimal labor supply to be efficient. 
	
	In the following, I demonstrate that (i) assuming $C^*=C_{FB}$ violates the condition above and $H^*\neq H_{FB}$ and that (ii) assuming $H^*=H_{FB}$ results in $C^*<C_{FB}$. 
	
	\textit{(i) Assume $C^*=C_{FB}$:}
	then
	\begin{align}
	(1-\tau_\iota)w=\frp{Y}{G}_{FB}\frp{G}{L_G}_{FB}.
	\end{align}
	Assume by contradiction that $H^*=H_{FB}$, since $s^*=s_{FB}$ by assumption, it follows that $w=\frp{Y}{G}_{FB}\frp{G}{L_G}_{FB}$. 
	Since $\tau_\iota\neq 0$ under constant or decreasing returns to scale, it holds that $H^*<H_{FB}$, a contradiction. 
	
	
	%Hence,
	%\begin{align}
	%(1-\tau_\iota)w<\frp{Y}{G}_{FB}\frp{G}{L_G}_{FB}.
	%\end{align}
	%
	%Labor supply in the competitive equilibrium is lower than in the efficient allocation when consumption is equal under the optimal policy. WHY?
	%It follows, that optimal labor supply does not equal its efficient counterpart when optimal consumption is efficient.
	
	\textit{(ii) Assume $H^*=H_{FB}$:} 
	It follows that 
	\begin{align}
	\frac{U_C^*}{U_{C,FB}}=\frac{\frp{Y}{G}_{FB}\frp{G}{L_G}_{FB}}{w}\frac{1}{1-\tau_\iota}=\frac{1}{1-\tau_\iota}>1.
	\end{align}
	From concavity of the utility function it follows that $C^*<C_{FB}$. 
	
	content...
	\end{comment}
\end{proof}

\subsubsection{Proofs proposition \ref{prop:3}}\label{app:proofintegrated}
\begin{proof} \textit{The optimal income tax scheme is progressive}\\ % if the optimal environmental tax is positive.}\\
	Under the new policy, the household's labor supply is determined by
	\begin{align}
	-U_H=\frac{U_C (1-\tau_{\iota})(wH+\tau_F p_fF)}{H}.
	\end{align}
	Expressing the derivatives in the Ramsey planner's first order condition with respect to hours as derivatives with respect to the dirty labor share, $s$, and substituting the first order condition with respect to $s$ yields:
	\begin{align}
	U_C \frp{Y}{G}\frp{G}{L_G}=-U_H.
	\end{align}
	%This equation is equivalent to the social planner's first order condition on hours, equation \ref{eq:fbh}. The optimal policy is to choose
	%\tr{Does this give a hint to why inefficiency without redistribution? The Ramsey planner's foc and household optimality always coincide. But, when Gov does not cancel the two do not coincide! ? the two do not coincide, Bcs consumption is too low so that $U_C$ too high which increases}
	Noticing that $\frp{Y}{G}\frp{G}{L_G}=w$ and replacing household's labor supply condition gives
	\begin{align}
	& w=\frac{(1-\tau_\iota)Y}{H}\\
	\Leftrightarrow\ & \tau_\iota=1-\frac{wH}{Y}. 
	\end{align} 
	Since $Y=C=wH+\tau_Fp_fF$ from the market clearing and household budget constraint, it follows that $wH<Y$ whenever $\tau^*_F>0$. Hence, $\tau_F^*>0$ implies $\tau^*_{\iota}>0$.
	%
	%Observe that $Y\geq MPL \times H$, where $MPL$ stands in for the marginal product of labor, if the aggregate production function features decreasing or constant returns to scale. Under such a production function one can rewrite the last expression as
	%\begin{align}
	%\tau_{\iota}=1-\frac{wH}{Y}\geq 1-\frac{w H}{MPL \times H}
	%\end{align}
	% Note further that the marginal product of labor exceeds the wage rate whenever the environmental tax is different from zero; compare the disucssion in subsection \ref{subsec:Rams}. It follows that the right-hand side is positive, hence
	%\begin{align}
	%\tau_{\iota}>0,
	%\end{align}
	The optimal tax scheme is progressive.
\end{proof}

\begin{proof}\textit{The optimal allocation is efficient}
	
	The idea of this proof is to show that the efficient allocation is attainable for the Ramsey planner. Since the social planner could implement any competitive allocation (which necessarily satisfies the resource constraint) and has the same objective function, the efficient allocation maximizes the Ramsey problem. 
	
	To show that the efficient allocation is feasible, I assume that $s^*=s_{FB}$. Showing that $H^*=H_{FB}$ and $C^*=C_{FB}$ are a solution to the Ramsey problem, proves that the optimal policy implements the efficient allocation for two reasons. First, by the argument in the previous paragraph any competitive allocation is a potential candidate solution to the social planner's problem and the social planner has the same objective function. Second, due to strict concavity of the utility and strict monotonicity of the production function \textit{(so that more input means more output)}, the solution is also unique.
	
	When $H^*=H_{FB}$ then $C^*=C_{FB}$ since $s^*=s_{FB}$ by assumption. It now show that under this allocation optimal labor supply, indeed, is efficient, that is:
	\begin{align}
	U_C^*\frp{Y^*}{G^*}\frp{G^*}{L^*_{G}} = U_{C,FB}\frp{Y_{FB}}{G_{FB}}\frp{G_{FB}}{L_{G,FB}}.
	\end{align}
	
	From the assumed allocation it follows that $U_C^*=U_{C,FB}$ and $\frp{Y_{FB}}{G_{FB}}\frp{G_{FB}}{L_{G,FB}}=\frp{Y^*}{G^*}\frp{G^*}{L^*_{G}}$ and above condition is satisfied. 
	
	It remains to show that under the assumed allocation, $s^*=s_{FB}$ holds true. Since $Gov=0$ the Ramsey planner's first order condition with respect to $s$ equals that of the social planner. Since production and marginal utilities in the optimal allocation equal their counterparts in the efficient allocation, it has to holds that $\tau_F^*$ implements $s^*=s_{FB}$.  
	%Second, efficiency of labor supply, i.e., $H^*=H_{FB}$, as the only solution of the Ramsey planner's problem follows from demonstrating that both (i) $H^*>H_{FB}$ and (ii) $H^*<H_{FB}$ result in a contradiction under the assumption that $s^*=s_{FB}$.
	
	%Assume by contradiction that (i), $H^*>H_{FB}$. 
\end{proof}



%
\section{analytic Model with  functional forms}
\subsection{Competitive equilibrium in simple model}

\begin{align}
\text{Utility}\hspace{5mm}& \frac{C_t^{1-\theta}-1}{1-\theta}-\chi \frac{h_t^{1+\sigma}}{1+\sigma}-\varphi(\omega F)^\eta\\
\text{Budget}\hspace{5mm}& C_t = \lambda_t(w_th_t)^{1-\tau_{\iota t}}\\
\text{optimality HH}\hspace{5mm}& h^{\sigma+\tau_{\iota t}+\theta(1-\tau_{\iota t})}=\lambda_t^{1-\theta}(1-\tau_{\iota t})w_t^{(1-\tau_{\iota t})(1-\theta)}\\
\text{Final Production}\hspace{5mm}&Y=F^{\varepsilon_y}G^{1-
	\varepsilon_y}\\ %\left[F^\frac{\varepsilon_y-1}{\varepsilon_y}+G^\frac{\varepsilon_y-1}{\varepsilon_y}\right]^\frac{\varepsilon_y}{\varepsilon_y-1}\\
%\text{price}\hspace{5mm}&1=p_y= \left(\frac{p_f}{\varepsilon_y}\right)^{\varepsilon_y}\left(\frac{p_g}{1-\varepsilon_y}\right)^{1-\varepsilon_y}\label{eq:ana_pr}\\
\text{Demand clean good}\hspace{5mm}&p_g=(1-\varepsilon)\left(\frac{F}{G}\right)^\varepsilon\label{eq:ana_dem_clean}\\
\text{Demand clean good}\hspace{5mm}&p_f=\varepsilon\left(\frac{F}{G}\right)^{\varepsilon-1}\label{eq:ana_dem_dirty}\\
\text{Production F and G}\hspace{5mm}&F=A_fL_F\label{eq:ana_prod_F}\\\
& G=A_gL_G\label{eq:ana_prod_G}\\
\text{labour demand}\hspace{5mm}& w=p_f(1-\tau_{ft})A_f\\
& w=p_gA_g\\
\text{technology}\hspace{5mm}&A_{ft+1}=(1+\nu_f)A_{ft}\\
&A_{gt+1}=(1+\nu_g)A_{gt}\\
\text{Government}\hspace{5mm}&T=\tau_{f}p_fF
\\
\text{Balanced income tax revenues}\hspace{5mm}&\lambda_t=\frac{w_t h_t}{(w_t h_t)^{1-\tau_{\iota t}}}\\
&E_{net}=\omega F-\delta
\end{align}
\subsubsection{Derivation expression for $h^{FB}$}
Rewriting equation \ref{eq:fbh}, the efficient amount of hours worked can be indirectly expressed as:
\begin{align}
h^{FB}=\frac{1}{\chi^\frac{1}{\sigma}}\left(w_{eff}^{1-\theta}-\frac{dE}{dF}A_f s^{FB}\left(h^{FB}\right)^\theta \right)^\frac{1}{\sigma+\theta}.\label{eq:heff_1}
\end{align}

Note that an explicit expression for $h^{FB}$ follows from equation \ref{eq:fbs} when there is an externality and $\frac{dE}{dF}\neq 0$. Then 

\begin{align}
h^{FB}= \left(\frac{\varepsilon(1-s)-s(1-\varepsilon)}{s(1-s)}\frac{w_{eff}^{1-\theta}}{\frac{dE}{dF}A_f }\right)^\frac{1}{\theta}
\end{align}
and the result follows from substituting the last expression in expression  \ref{eq:heff_1}.

\subsection{Proof: Hours worked with only the efficient share of dirty labor are inefficiently high}

\begin{proof}
	The proof proceeds in two steps. First, I show that the share of labor allocated to the dirty sector is smaller than its efficient level absent externality which is $s=\varepsilon$.
	In the second step, I show that even if the environmental tax is set to the tax which replicates the efficient share of dirty labor, hours worked, denoted by $h_{CE, s^{eff}}$, exceed their efficient level, $h_{FB}$, when neither lump-sum transfers no labour income taxes are available. 
	
	First note that the share of dirty labor is fully determined by the environmental tax. The environmental tax is set to implement a gap between the marginal product of labor between the clean and the dirty sector. The relation follows from labor market clearing and intermediate goods market clearing 
	\begin{align}
	\tau_f = \frac{\varepsilon-s}{(1-s)\varepsilon}\label{eq:tauf}
	\end{align}
	
	\textbf{Step 1:} $\frac{dE}{dF}>0$ \ar $\varepsilon>s$\\
	Rewriting equation \ref{eq:fbs} yields
	\begin{align}
	\frac{\varepsilon(1-s)-s(1-\varepsilon)}{s(1-s)}=\frac{dE}{dF}A_fh^\theta w_{FB}^{1-\theta}.
	\end{align}
	When the externality is negative, i.e., $\frac{dE}{dF}>0$, then the right-hand side is positive.
	Since $s\in(0,1)$ - due to both intermediate goods being necessary to produce the final good and zero consumption is not a solution - the left-hand side is positive when
	\begin{align}
	\varepsilon(1-s)-s(1-\varepsilon)>0,
	\end{align}
	which holds true if and only if $\varepsilon>s$.
	
	\textbf{Step 2:} $\varepsilon>s$ \ar $h_{CE, s^{eff}}>h_{FB}$\\
	I prove the claim by evoking a contradiction to the assumption that $h_{CE, s^{eff}}\leq h_{FB}$. Using equation \ref{eq:hopt} and \ref{eq:heff} the expression becomes
	
	\begin{align}
	&\left(\frac{w^{1-\theta}}{\chi}\right)^{\frac{1}{\sigma+\theta}}\leq \left(\frac{w_{FB}^{1-\theta}}{\chi}\frac{1-\varepsilon}{1-s}\right)^\frac{1}{\sigma+\theta}
	\\
	\Leftrightarrow&\left(\frac{w}{w_{FB}}\right)^{\frac{1-\theta}{\sigma+\theta}}\leq \left(\frac{1-\varepsilon}{1-s}\right)^\frac{1}{\sigma+\theta}
	\end{align}
	
	Note that the ratio of the wage in the competitive economy to the marginal product of labor is $\frac{w}{w_{FB}}=\frac{1-\varepsilon}{1-s}$, which follows from equation \ref{eq:compw} and the definition of $w_{FB}$ under the assumption that the dirty labor share is set to the first best equivalent. Substituting this in the previous equation and rearranging terms yields
	\begin{align}
	\left(\frac{1}{1-\varepsilon}\right)^\frac{\theta}{\sigma+\theta}\leq \left(\frac{1}{1-s}\right)^\frac{\theta}{\sigma+\theta}
	\end{align}
	which holds true whenever
	\begin{align}
	\varepsilon<s.
	\end{align}
	This contradicts $s<\varepsilon$ which has been shown to hold in presence of a negative externality in the dirty sector in step 1. 
\end{proof}

\textit{Intuition:} the fact that the wage rate in the competitive equilibrium understates the marginal product of labor  depresses labor supply due to a substituion effect. When the income effect is more pronounced, that is, $\theta>1$, the low wage rate increases labor supply above the efficient level. When the substitution effect is stronger when $\theta<1$, then the distortion in the wage rate decreases labor supply. 
The neglected contribution to the externality by households in the competitive equilibrium makes hours inefficiently high irrespective of parameter values. 
Hence, with $\theta<1$ the overall distortion in hours worked is mitigated has households the stronger substitution effect offsets part of the inefficient high labor supply due to the neglect of the externality. ¸

In the model, it does so by exactly the same amount as hours contribute to the externality. 
than compensated for by the neglect of the negative effect of hours on the externality due to the concave curvature of the utility function, $\theta>0$. If $\theta=0$, then the two effects would exactly offset. 



\subsection{Proof: lump-sum transfers restore the efficient allocation}
\begin{proof}\label{pr:lst_eff}
	To establish that lump-sum transfers of environmental tax revenues restore the efficient allocation, I first derive the size of lump-sum transfers which implement the efficient level of hours worked given that the efficient dirty labor share is established by choice of the environmental tax. In a second step, I show that this level of transfers coincides with the revenues from the environmental tax when this is set to implement the efficient dirty labor share. 
	These two steps prove that the efficient amount of hours results from lump-sum transferring environmental tax revenues. Finally, I show that the resulting level of consumption is efficient. This completes the proof.
	
	
	
	\textbf{Step 1:} Solve for transfers which implement efficient level of hours\\
	I equalize equation \ref{eq:heff} and \ref{eq:hopt} setting the income tax progressivity, $\tau_{\iota}$, to zero.
	\begin{align}
	\left(\frac{w^{1-\theta}\left(1+\frac{T^*}{wh}\right)^{-\theta}}{\chi}\right)^\frac{1}{\sigma+\theta}=\left(\frac{w_{FB}^{1-\theta}}{\chi}\frac{1-\varepsilon}{1-s}\right)^\frac{1}{\sigma+\theta}.
	\end{align}
	Using the relation of $w_{FB}$ and $w$ established in the previous proof, I can rewrite the right-hand side
	\begin{align}
	&\left(\frac{w^{1-\theta}\left(1+\frac{T^*}{wh}\right)^{-\theta}}{\chi}\right)^\frac{1}{\sigma+\theta}=\left(\frac{w^{1-\theta}}{\chi}\left(\frac{1-s}{1-\varepsilon}\right)^{1-\theta}\frac{1-\varepsilon}{1-s}\right)^\frac{1}{\sigma+\theta}.
	\end{align}
	This step is instructive in showing that transfers will correct for the two inefficiencies in the competitive economy: (i) the too low wage rate captured by the term $\left(\frac{1-s}{1-\varepsilon}\right)^{1-\theta}$, and (ii) the neglect of the effect of hours worked on the externality, captures by $\frac{1-\varepsilon}{1-s}<1$. 
	
	Solving for transfers yields
	\begin{align}
	T^* = \left(\frac{\varepsilon-s}{1-\varepsilon}\right)wh_{FB},
	\end{align}
	
	\textbf{Step 2:} $T^*=\tau_f^*p_fF$\\
	The environmental tax in equilibrium is determined by equation \ref{eq:tauf}: $\tau_f = \frac{\varepsilon-s}{(1-s)\varepsilon}$. Free labor movement enforcing a unique wage rate implies that $p_f=p_g\frac{A_g}{(1-\tau_f)A_f}$. The price for the clean good, $p_g$, in equilibrium, balances clean demand and wages paid: $p_g=\varepsilon^\varepsilon(1-\varepsilon)^{1-\varepsilon}\left(\frac{(1-\tau_f)A_f}{A_g}\right)^\varepsilon$. Dirty output, $F$, is given by $F=A_fsh$. 
	Substituting these expressions in the expression for $T^*$ above and observing that $w=(1-\varepsilon)\left(\frac{A_f}{A_g}\frac{s}{1-s}\right)^\varepsilon A_g$ yields the result.
	
	\textbf{Step 3: } Consumption under $T^*$ is efficient\\
	Trivially, as market clearing has to hold in the competitive equilibrium, it follows that 
	\begin{align}
	C^*=\left(A_f s^*\right)^\varepsilon\left(A_g(1-s^*)\right)^{1-\varepsilon}h^*
	\end{align} 
	Since $T^*$ and $\tau_f^*$ have been set to establish the efficient dirty labor share, $s^*=s_{FB}$ and the efficient level of hours worked, $h^*=h_{FB}$, it follows that $C^*=C_{FB}$. This completes the proof.
\end{proof}

\subsection{Infeasibility of efficient allocation if environmental tax revenues are consumed by the government}

\begin{proof}
	The proof proceeds by construction. First, I assume that the efficient dirty labor share has been implemented and that consumption equals the efficient level. Solving for the competitive level of hours shows that they exceed the efficient level of hours.  Hence, the efficient allocation is not feasible when the government consumes environmental tax revenues since work effort has to be inefficiently high to sustain the first-best level of consumption. 
	
	Working hours to support the efficient level of consumption are given by the market clearing condition  (which holds true with and without income tax scheme)
	\begin{align}
	C_{FB} = \left(A_f s_{FB}\right)^\varepsilon\left(A_g(1-s_{FB})\right)^{1-\varepsilon}h-\tau_f^*p_f^*A_fs_{FB}h
	\end{align}
	Since $s^*=s_{FB}$ by assumption, substituting equilibrium expressions for $\tau_f^*$ and $p_f^*$ used in proof \ref{pr:lst_eff} and solving for $h$ yields
	\begin{align}
	h=\frac{C_{FB}}{w}.
	\end{align}
	Substitution of consumption from the first best allocation, $C_{FB}=\left(A_f s_{FB}\right)^\varepsilon\left(A_g(1-s_{FB})\right)^{1-\varepsilon}h_{FB}$, gives
	\begin{align}
	\frac{h}{h_{FB}}=\frac{w_{FB}}{w}.
	\end{align}
	Since $\varepsilon>s$ the right-hand side, $\frac{w_{FB}}{w}=\frac{1-s}{1-\varepsilon}$, is above unity. Hence, $h>h_{FB}$. 
\end{proof}

\subsection{Proof: redistributing environmental tax revenues through the non-linear income tax scheme restores the efficient allocation. The income tax scheme to support the efficient allocation is progressive.}

\begin{proof}
	Hours under the non-linear tax-scheme policy become
	\begin{align}
	h=\left(\frac{(1-\tau_{\iota})w^{1-\theta}(1+\tau_f p_f\frac{F}{hw})^{1-\theta}}{\chi}\right)^\frac{1}{\sigma+\theta}.
	\end{align}
	
	I assume that the dirty labor share is set to the efficient level, $s=s_{FB}$. This determines $\tau^*_f$ and $p^*_f$.
	It has to be shown that
	\begin{align}
	\left(\frac{(1-\tau_{\iota})w^{1-\theta}\left(1+\tau^*_f p^*_f\frac{F_{FB}}{h_{FB}w}\right)^{1-\theta}}{\chi}\right)^\frac{1}{\sigma+\theta}=\left(\frac{w^{1-\theta}}{\chi}\left(\frac{1-s}{1-\varepsilon}\right)^{1-\theta}\frac{1-\varepsilon}{1-s}\right)^\frac{1}{\sigma+\theta}
	\end{align}
	From proof \ref{pr:lst_eff} step 2 we know that $\frac{\tau_f^*p_fF}{(wh_{FB})}=\left(\frac{\varepsilon-s}{1-\varepsilon}\right)$ and hence $\left(1+\frac{\tau_f^*p_fF}{wh_{FB}}\right)^{-\theta}=\left(\frac{1-s}{1-\varepsilon}\right)^{-\theta}$ and above condition simplifies to
	\begin{align}
	(1-\tau_{\iota})\frac{\varepsilon-s}{1-\varepsilon}=1.
	\end{align}
	Rearranging terms yields
	\begin{align}
	\tau_{\iota}= \frac{\varepsilon-s}{1-s}.
	\end{align}
	Since $\frac{\varepsilon-s}{1-s}\in(0,1)$ when there is a negative externality from dirty production, the optimal tax scheme, which implements the efficient allocation, exists and is progressive. \textit{Note that 1 is an upper bound on the tax scheme progressivity parameter as otherwise the marginal returns to labor would be decreasing in hours worked. It is positive since $s<\varepsilon$.}
\end{proof}

\begin{comment}
\subsection{Numeric results in simple model}
\begin{table}[h!!]
	\caption{Linear tax scheme and lump-sum transfers}\label{tab:lin_lst}
	\begin{tabular}{lllllllll}
		Thetaa & FB hours & FB Pigou & CE hours & CE scc & Opt hours & Opt taul & Opt tauf & Opt scc \\ 
		\hline 
		<1 & 1.192 & 0.99326 & 1.192 & 0.99326 & 1.192 & -3.7748e-15 & 0.99326 & 0.99326 \\ 
		Bop & 0.13601 & 0.99959 & 0.13601 & 0.99959 & 0.13601 & -3.7748e-15 & 0.99959 & 0.99959 \\ 
		log & 0.36434 & 0.99853 & 0.36434 & 0.99853 & 0.36434 & -3.7748e-15 & 0.99853 & 0.99853 \\ 
		\hline 
	\end{tabular}
\end{table}
\begin{table}
	\caption{Linear tax scheme, env. tax revenues not transferred lump-sum}\label{tab:lin_nolst}
	\begin{tabular}{lllllllll}
		Thetaa & FB hours & FB Pigou & CE hours & CE scc & Opt hours & Opt taul & Opt tauf & Opt scc \\ 
		\hline 
		<1 & 1.192 & 0.99326 & 1.2061 & 0.97804 & 1.1706 & 0.049876 & 0.9934 & 0.94584 \\ 
		Bop & 0.13601 & 0.99959 & 0.14026 & 0.96001 & 0.13808 & 0.049876 & 0.99958 & 0.94766 \\ 
		log & 0.36434 & 0.99853 & 0.37243 & 0.97015 & 0.36435 & 0.049876 & 0.99853 & 0.94804 \\ 
		\hline 
	\end{tabular}
\end{table}
\begin{table}[h!!]
	\caption{Baseline model env. revenues transferred via income tax scheme ($\lambda$)}\label{tab:base}
	\begin{tabular}{lllllllll}
		Thetaa & FB hours & FB Pigou & CE hours & CE scc & Opt hours & Opt taul & Opt tauf & Opt scc \\ 
		\hline 
		<1 & 1.192 & 0.99326 & 1.2275 & 1.0056 & 1.192 & 0.049979 & 0.99326 & 0.99326 \\ 
		Bop & 0.13601 & 0.99959 & 0.13811 & 1.0311 & 0.13601 & 0.049979 & 0.99959 & 0.99959 \\ 
		log & 0.36434 & 0.99853 & 0.37243 & 1.0211 & 0.36434 & 0.049979 & 0.99853 & 0.99853 \\ 
		\hline 
	\end{tabular}
\end{table}



Table 1 to 3 compare the efficient allocation to an allocation resulting in the competitive equilibrium when the environmental tax is set to equal the social cost of carbon in the efficient allocation. The rationale being that without any further distortions setting environmental taxes to the social cost of carbon implements the efficient allocation. The last four columns of each table show hours worked, the optimal policy and the social cost of carbon in equilibrium resulting in the Ramsey planner allocation. 

Table \ref{tab:lin_lst} reveals that indeed, setting the corrective tax equal to the social cost of carbon under the social planner implements the first-best allocation when lump-sum transfers are available. The optimal policy chooses zero income taxes. 

The picture changes once no lump-sum transfers are available, compare table \ref{tab:lin_nolst}. In the competitive equilibrium setting the environmental tax to the social costs of carbon under the social planner results in inefficiently high hours worked for all values of $\theta$ considered; compare the columns showing the allocation resulting in the competitive equilibrium when only the efficient dirty share is implemented. 
Theoretically, the labor income tax can be used to establish the 
efficient level of hours worked given that the dirty labor share is efficient. However, since
environmental tax revenues are not redistributed lump-sum, household consumption is lower than under the social planner and the efficient level of hours worked and the efficient dirty labor share feature a lower social welfare in the competitive equilibrium. In other words, a further reduction in labor is too costly in terms of consumption and the optimal labor tax is lower than what would implement efficient hours. \textit{This might change when the household derives utility from government consumption.}

The optimal policy is to set a positive income tax rate; the optimal income tax code is progressive. When the substitution effect outweighs the income effect, i.e., $\theta<1$, then the optimal allocation results in inefficiently \textit{low} hours worked. When the income effect is at least as strong than the substitution effect, that is $\theta\geq 1$, then hours worked remain inefficiently high under the optimal policy. 

Interestingly, when the planner transfers environmental tax revenues through the income tax scheme, table \ref{tab:base}, then the efficient allocation is attainable for all values of $\theta$ considered through a progressive tax scheme. 

Only when the Ramsey planner can implement the efficient level of work, the environmental tax is set to equal the social cost of carbon.   




\textbf{In a nutshell}
\begin{itemize}
	\item hours worked without transfers are always too low even if efficient tax rate is chosen
	\item when hours are not efficient, then the environmental tax does not match the social cost of carbon
	\item when revenues are transferred through the income tax, the planner can implement the efficient allocation with the help of a progressive income tax \textit{(interesting!)}
	\item with $\theta<\frac{\varepsilon}{\varepsilon-s}$ optimal hours worked reduce, otherwise the income effect is too strong and hours worked increase! 
	Nevertheless, the allocation in LF without lump sum transfers always features too high hours worked. 
	\item why does the optimal policy with taul but no lump-sum transfers not implement the efficient level? \ar income taxes are not a measure to implement the efficient allocation; only similar when income and substitution effect cancel. Too high when income effect dominates, too low when substitution effect dominates.
	\ar general consumption tax should neither be able to implement efficient allocation! 
	%\item when there is no income tax, the optimal policy is to set the efficient dirty labour share (compare table \ref{tab:lin_nolst_notaul}). Labor supply is always too high but the optimal tax exceeds the social cost of carbon
\end{itemize}

\end{comment}
%\begin{comment}


%	content...
%\end{comment}

\section{Quantitative model}\label{app:quant_mod}

\subsection{Social planner allocation}\label{app:sp_prob} The solution to the social planner's problem is defined as an allocation \\ $\{L_{Ft}, L_{Gt}, L_{Nt}, x_{Ft}, x_{Gt},  x_{Nt}, C_t,  H_{t}, s_{Ft}, s_{Gt}, s_{Nt} \}$ for each period which maximizes the social welfare function
\vspace{-5mm}
\begin{align*} &\sum_{t=0}^{T}\beta^t u(C_{t}, H_{t})+ PV\\
s.t.\ &  \omega F_{t} -\delta \leq \Omega_t\\
&C_t+x_{Ft}+x_{Gt}+x_{Nt}=Y_t\\
&\text{Law of Motion of technologies and initial productivities}\\
&H_{t}=L_{Ft}+L_{Gt}+L_{Nt}\leq \bar{H},\\
&S=s_{Ft}+s_{Gt}+s_{Nt}.
\end{align*}
Production of $Y_t$ is defined by the equations describing production in the model. %: equations \ref{mod:y} to \ref{mod:LG}.
 It holds that $x_{Jt}=\int_{0}^{1}x_{Jit}di$.
%\subsection{Social planner problem}
%The social planner chooses 
$PV$ stands in for the continuation value of the economy; see \ref{app:PV} for the derivation.

\section{Calibration of the emission limit}\label{app:calib}
In 2019, global net CO$_2$ emissions amounted to 44.25Gt \citep[compare Figure SPM1.a p.11 in ][]{IPCCSPM} yielding a net emission limit in the 2030s of 22.125Gt per annum. The IPCC report is vague on the exact year when the 50\% reduction has to be reached. I assume that the net emission limit becomes binding in 2035 and is active until the limit reduces to zero in 2050. 
To back out the share of the emission limit assigned to the US following the \textit{equal-per-capita} principle, I use population projections from the \cite{UNPOP}. %\footnote{ Retrieved on 4 August, 2022 from \url{https://population.un.org/dataportal/data/indicators/49/locations/900,840/start/2010/end/2100/table/pivotbylocation}.} 
Since the US population share is projected to decline over the period from 2035 to 2050, the emission limit reduces over this period. 

Following this approach, the remaining net CO$_2$ budget for the period 2020-2035 happens to be smaller than total CO$_2$ emissions equal to the limit between 2035-2050. This conflicts with the downward sloping time path of emissions stipulated by the IPCC (p.3-28).\footnote{ The reason is that global net-CO$_2$ emissions are so high that a 50\% reduction starting from 2035 results in more than half the emissions of the remaining carbon budget.}
To generate a non-increasing pattern of the emission limit, I set the global emission limit for the period from 2035 to 2050 to half the CO$_2$ budget. This leaves an equal budget share for the initial 15 years. I assume that in the period from 2020 to 2035 the budget is equally distributed across years to simplify the numeric calculation of the optimal policy. Similar to the derivation for the period from 2035 to 2050, I apply the year-specific population share and sum over those years which form a model period. In sum, this approach amounts to distributing the remaining budget of 510Gt equally over the 30 years until the net-zero limit.

\section{Numerical appendix}\label{app:PV}

Since I cannot solve explicitly for the optimal policy over an infinite horizon, I truncate the problem after period $T$. 
In the literature, utility in periods after $T$ are approximated under the assumption that policy variables are fixed, and the economy reaches a balanced growth path \citep{Barrage2019OptimalPolicy, Jones1993OptimalGrowth}. However, assuming a constant carbon tax would most likely violate the emission limit since the model is designed to reflect market forces describing an economy with green and fossil sectors operating in equilibrium. 


I motivate the design of the continuation value by pretending the planner would hand over the economy to a successor after period $T$. A continuation value, $PV$, in the objective function captures that the planner cares about utility after period $T$. 
This set-up accounts for concerns about economic well-being of future generations in a similar vein than the sustainability criterion proposed by the \cite{UNSUS} by attaching some value to the final technology level.\footnote{ The sustainable development criterion reads "\textit{[...] to ensure that it meets the needs of the present without comprising the ability of future generations to meet their own needs.
	}" (p.24). This is a vague definition.  \cite{Dasgupta2021} p.(332) interprets this criterion as meaning: 
	"\textit{[...] each generation should bequeath to its successor at least as large a productive base as it had inherited from its predecessor. }". 
	However, this cannot be used to derive a sensible condition on the optimization in the present setting since there is no negative growth and technology is the only asset bequeathed to future generations. Thus,
	successors will always have at least as much productive resources as predecessors left. The relation to the future is instead approximated by a future potential to derive utility from consumption given bequeathed technology levels. Natural needs of the future are accounted for through the emission limit. } I approximate the value of future technology levels by assuming constant growth rates.  
To mitigate concerns that the choice of the continuation value drives the results, I experiment with the exact value of explicit optimization periods. I truncate the problem once explicitly adding a further period leaves the optimal allocation numerically unchanged. That is the case after $T=42$, or 210 years. %Then the 
The planner's objective function becomes: 
\begin{align*}
	\underset{\{\tau_{Ft}\}_{t=0}^{T},\{\tau_{\iota t}\}_{t=0}^{T}}{\max}&\sum_{t=0}^{T}\beta^t u(C_{t}, H_{t})
	+PV.
\end{align*}

In more detail, I define the continuation value as the consumption utility over the infinite horizon starting from the last explicit maximization period:
\begin{align*}
	PV=\sum_{s=T+1}^{\infty} \beta^{s}u(C_s, H_{s}).
\end{align*}
I make three simplifying assumptions to derive the continuation value. First, 
I assume that the consumption share, $c_s$, with $C_s=c_sY_s$, is constant from period $T+1$ onward.  Then, consumption grows at the same rate as output. 
Second, as an approximation to future growth, I assume the economy grows at the same rate as in the last explicit optimization period. Third, hours of workers and scientists remain at their value in the last explicit optimization period. %\footnote{ Note that I am not making the assumption that the future economy is best described by the continuation value; it rather serves as a proxy to measure the value of passed technology levels. } 
Let $\gamma_{yT}=\frac{Y_{T}}{Y_{T-1}}-1$. Under above assumptions, I can rewrite future consumption as $C_s=(1+\gamma_{yT})^{s-T}C_{T}$.
Given the functional form
\begin{align*}
	u(C_s)= \frac{C_s^{1-\theta}}{1-\theta},
\end{align*}
the continuation value reduces to
\begin{align*}
	PV= \beta^{T}\left(\frac{1}{1-\beta (1+\gamma_{yT})^{1-\theta}}\frac{C_{T}^{1-\theta}}{1-\theta}+ \frac{1}{1-\beta}u(H_{T})\right).
\end{align*}
Where $u(C_s, H_{s})=u(c_s)+u(H_{T})$.
When $\theta\underset{\lim}{\rightarrow} 1$,  the first summand in brackets becomes
\begin{align*}
	\frac{1}{1-\beta}\log(C_{T}).
\end{align*}

%\section{Degrowth and end to growth in the literature}
\begin{itemize}
	\item \cite{Dasgupta2021}\ar impossibility to grow indefinitely \ar need to reduce to not surpass safe operating space
	\item \cite{Schor2005SustainableReduction}
	\item \cite{VanVuuren2018AlternativeTechnologies}: limit to carbon capture and storage technologies; if output growth requires fossil energy, than infinite growth would need infinite storage; to reduce dependence on this technology beneficial to reduce demand 
	\item \cite{Bertram2018TargetedScenarios}: reduction in demand to simultaneously meet emission targets and sustainability goals (global acceptability)\ar reduction of energy demand alleviates competition between reaching emission limits and sustainability goals (zero hunger, affordability of energy); to lower sustainability risk;
	\\   change demand as a parameter in  model; motivation: taking global inequality into account alternative measures (mitigation policies) to carbon taxes  become optimal. These include lifestyle changes \textbf{in addition to sector-specific carbon taxes!} (25\% lower energy demand and -20\% lower demand for agricultural products )
\end{itemize}

In his review, \cite{Dasgupta2021} attempts to construct an economics of biodiversity. By taking nature's maintenance services, that is, services without which human activity and live would not be possible, as a constituent of total factor productivity, economic growth is no longer disconnected from planetary boundaries (p.137).\footnote{\ The term \textit{planetary boundaries} has been coined by \cite{Rockstrom2009AHumanity} who use it to refer to a state of nature in which humans can safely exist.} 
Papers on environmental economics acknowledge that natural conditions are important to production by assuming tipping points \citep{Acemoglu2012TheChange} or rising temperatures affecting output as in \textit{cite Nordhaus1994, Stern 2006} \cite{Barrage2019OptimalPolicy}.
In addition, \cite{Dasgupta2021} accounts for the waste resulting from production and consumption which again requires nature for 

\cite{Dasgupta2021} argues that infinite GDP growth is impossible given planetary boundaries (p. 47), i.e., the safe space for humans to exist, and that waste which degrades the environment is always positively related to output. 

My project relates by looking at one particular boundary: the one on carbon emissions through a limited carbon budget postulated in the natural sciences \citep{IPCC2022, Rockstrom2009AHumanity}. The model is designed to allow for a stop to technological and consumption growth. 

 
%\clearpage
\appendix
\section{Natural Scientific Background}
\begin{comment}
\subsection{Why output (growth) reduction might be optimal}
It is a vibrant debate whether technological process will result in a production technology that is perfectly clean in that it does not exert any environmental externality. 
\begin{itemize}
	%\item \underline{Extensions to technology in \cite{Acemoglu2012TheChange} }
	%\begin{itemize}
	\item \underline{externality of ``clean'' sector} \citep[see also][]{Dasgupta2021, Brock2005ChapterEmpirics}
	\begin{itemize}
		\item[-] renewable/ non-fossil fuels \ar externalities in production process are present e.g. production of solar panels uses toxic inputs \citep{Yue2014DomesticAnalysis}; non-fossil fuel nitrogen generation (e.g., biomass burning to clear land) important ($\approx$ 50\%) \citep{Song2021ImportantEmissions}; low but chronical levels of nitrogen cause species extinctions \citep{Clark2008LossGrasslands}
		\item[-] waste (after use) \ar depends on recycling technology %\ar recycling system for solar panels not profitable enough today
		%	\item[-] substitutability of nature in production (input sources eg. fossil vs. non-fossil fuels)
		%\end{itemize}
		%\item Irreversibilities already before thresholds are hit (e.g. species extinction)
		
	\end{itemize}
	%\item greenhouse gases: Carbon dioxide $CO_2$ (vast majority), Nitrous oxide $N_2O$, methane $CH_4$
	%\item stock of nature globally determined
	\item \underline{parallel positive trend in demand} (population growth, rebound effect) that outperforms increase in clean technology growth \small{(no long-run issue if perfectly clean technology exists)}
	\item \normalsize{\underline{objective function}:} \cite{Arrow2004AreMuch}(Journal of Economic Perspectives) \ar using a sustainability measure they provide evidence that consumption is too high (= not leaving enough natural resources for future generations)
	\item \underline{risk, ambiguity}
	\item if have to meet climate target in short run, might need to lower production to do so; or it might be better in terms of inequality?
\end{itemize}

content...
\end{comment}
\subsection{Greenhouse-gas emissions and the Paris Agreement}

Two alternatives exist to specifiy the relation between the environment and production: (i) a broad approach considering natures use as a sink and as a resource, and all relevant pollutants. 
In order to determine \textit{relevant}, I refer to the planetary boundaries discussed in \cite{Rockstrom2009AHumanity}. (ii) a more specific approach that focuses on greenhouse gas emissions in particular which allows to draw on emission goals specified by country. Paris agreement goal: `'\textit{Climate neutral world by the mid-century}'' (source \url{https://unfccc.int/process-and-meetings/the-paris-agreement/the-paris-agreement}). In 2020 countries had to submit plansfor a \textit{long-term low ghg emissions} (LT-LEDS) where long term means mid-century (I assume). In the EU member states have submitted \textit{integrated national-energy and climate plans} (NECPS) (source \url{https://ec.europa.eu/info/energy-climate-change-environment/implementation-eu-countries/energy-and-climate-governance-and-reporting/national-long-term-strategies_en}). According to this source, emissions occur in the following fields: 
\textit{emission reductions and enhancements of removals in individual sectors, including \textbf{electricity, industry, transport, the heating and cooling and buildings sector (residential and tertiary), agriculture, waste and land use, land-use change and forestry (LULUCF)}}; the website also contains documents on country specific plans and actions

\subsection{Modelling choice: external emission target}\label{app:emission_climate_targets}
There is a multitude of uncertainties shaping the relation of production, on the one hand, and nature and climate warming, on the other hand. These uncertainties relate to (i) the technological possibilities to reduce emissions in the future and (ii) the relation of emissions and the climate. 

In the Paris Agreement clear political goals have been formulated in 2015. Under this treaty, states have agreed on a legally binding maximum increase in temperature to well below 2°C, preferably 1.5° over pre-industrial levels, and the global community seeks to be climate-neutral in 2050  (compare:\\ \url{https://unfccc.int/process-and-meetings/the-paris-agreement/the-paris-agreement}). 

\paragraph{Uncertainty 1): Emissions $\rightarrow$ temperature}
Carbon dioxide has been the focus of the literature integrating climate change and economic models \citep[such as,][]{Golosov2014OptimalEquilibrium,Barrage2019OptimalPolicy}. 
The (geo-)physical mechanisms which determine the interrelation between carbon emissions and temperature changes are highly uncertain and complex. For example, (1) there is no good understanding of the relation of CO2 and the climate as the temperature rises to certain limits, (2)  feedback of the Earth system, such as permafrost thawing, has to be taken into account, as well as (3) interactions of carbon with non-CO2 emissions, (\citep[][p.96, 2nd paragraph]{Rogelj2018MitigationDevelopment.}).  Uncertainty also surrounds the regeneration rate of the environment \citep{Acemoglu2012TheChange} and irreversibilities (might CITE Hassler handbook chapter here). In a quantitative study on optimal environmental policies, hence, a lot of assumptions and simplifications have to be made. 

In chapter 2 of the
\textit{IPCC Special Report} \citep{Rogelj2018MitigationDevelopment.}, scientists quantify emission pathways to meet the 1.5°C goal of the Paris Agreement by carefully taking uncertainties and the complex geophysical processes into account. I use these limits on emissions as constraints to the government's objective function. This approach is clearly policy relevant, while at the same time reduces the need to make (geophysical) assumptions. Furthermore, it allows me to take other important non-CO2 emissions into account, too. (\textit{Look at the discussion of integrated assessment models in \cite{Hassler2016EnvironmentalMacroeconomics} for the advantages of integrating a simplified carbon cycle into macro models (\ar dynamics) })
%\tr{\ar In a nutshell, I take from these reports the emission reduction pathways. I do have to make assumptions on the possibilities of technological innovations. No carbon cycle needed but less assumptions have to be made.}

\paragraph{Uncertainty 2: Technological progress $\rightarrow$ emissions}
An important modelling uncertainty remains: what degree of emission reduction can be achieved by technological progress in the specified time frame? I use different specifications of technological possibilities: (i) a scenario where technological progress is sufficient to reduce emissions to zero until 2050 at current consumption levels per capita, (ii) and one where innovation steps are insufficient.
Also look at different modelling approaches to technological change: (i) sector-specific innovations, (ii) porgress on the substitutability of clean and dirty input goods.

\paragraph{Uncertainty 2: regeneration rate of nature}

\paragraph{Uncertainty 2: Substitutability of natural capital in welfare}
Related to technological possibilities is the substitutability of clean and dirty production. 

A key reference is \cite{Cohen2019AnnualSubstitutable}. They argue for a limit of substitutability of \textit{natural capital} (defined as the stock of renewable and nonrenewable resources including minerals, soils, plants, animals, water, air, and energy). The role of natural capital for welfare: resources for production, absorption of waste, basic-life support, direct conrtibution to human welfare \ar No perfect substitutability: Humans cannot live without natural resources. 

BUT: my model so far is about greenhouse-gas emissions only.

Definition \textit{sustainability} common in economics: maintaining a non-decreasing level of welfare across generations. (Sonja: this shouls include provision of natural services: human-friendly climate; how to know the utility function of future generations? Eg if habits are relevant, than they might be happy with less consumption). As regards renewables, it must be ensured that renewability is maintained. It is about the substitutability of natural inputs to the welfare function. \textit{Sonja: (1) How should we know how substitutable biodiversity is for humanity if it is about pleasure derived from living in a diverse world. (2) If it is about other factors of biodiversity, such as, maintaining more basal services for human live, such as safety, there might be less disagreement; more certainty on the importance also for future generations; a utility approach but based on objectively defined values: basic needs.}

\cite{Cohen2019AnnualSubstitutable} make the following distinction: \textit{within-input substitution}: =resources used in welfare production: within-input substitution allows to reduce environmental \textbf{impacts} if the same type of input can be obtained from different sources (e.g.: energy: from emission-low sources instead of emission high sources \tr{\ar This is substitution of dirty with clean goods}); \textit{between-input substitution}: = from energy to manufactured capital;\tr{ includes more efficient energy use (\textbf{with the same amount of energy, more can be produced.})} , recycling, reforestation.

\textbf{Sonja:} within-input substitution: \ar same input but different source (progress= less emissions for energy); between-input substitution: 
\ar sector specific production functions are cobb-douglas \ar unit elasticity of input goods.


\textit{Brundtland Comission} definition of sustainable development: \textit{development that meets the \textbf{needs} of the present without compromising the ability of future generations to meet their own needs}.
If various forms of capital are substitutable (i.e. can use manufactured capital, human capital to replace nature) then production only depends on the total capital stock. Then, economic growth is said to be \textit{weakly sustainable} when the total capital stock is non-decreasing; i..e aggregate savings rate is above depreciation rate on all forms of capital. 

\textit{Strong sustainability view}:  a minimum of natural capital must be sustained as it provides non-substitutable inputs to utility. Then, long-run growth must be able to maintain a natural capital stock.  
\ar Define $Y$ broadly to incorporate other necessary consumption goods: stable climate, breathable air, food, water; but these are partially not traded in markets, rather public goods. Then, better to model as public goods; \textbf{Or as another input in final good production}.

\begin{align*}
Y= \left[\left(\underbrace{\left(Y_c^{\frac{\varepsilon_c-1}{\varepsilon_c}}+Y_d^{\frac{\varepsilon_c-1}{\varepsilon_c}}\right)^{\frac{\varepsilon_c}{\varepsilon_c-1}}}_{\text{consumption  good}}\right)^{\frac{\varepsilon_o-1}{\varepsilon_o}}+(\underbrace{N-\bar{N}}_{\text{natural capital}})^{\frac{\varepsilon_o-1}{\varepsilon_o}}\right]^{\frac{\varepsilon_o}{\varepsilon_o-1}}
\end{align*} 
where $\varepsilon_o$  governs the substitutability of natural capital and consumption in the final good; it translates into the substitutability discussed in \cite{Cohen2019AnnualSubstitutable}. For the basic needs level ins terms of natural capital holds: $\bar{N}>0$ so that nature is a necessary good. The variable $N$ determines the consumption of natural capital, such as breathing air, stable climate.
Model technological growth on substitutability between natural capital (including nature as a waste), and other production inputs.  

 
%%%%%%%%%%%%%%%%%%%%%%%%%%%%%%%%%%%%%%%
\subsection{Greenhouse gas emissions Data}
To calibrate the relation of economic production and emissions, I use data from the EPA \url{https://www.epa.gov/newsreleases/latest-inventory-us-greenhouse-gas-emissions-and-sinks-shows-long-term-reductions-0}.
In 2019, the US greenhouse gas emissions amounted to 6,558 million metric tons of carbon dioxide equivalents; that is, 6.558Gt. 

Global greenhouse gas emissions amounted to 34.2 Gt in C02 equivalents in 2019. There was a decline in 2020 (presumably due to the pandemic, and a rebound in 2021 by 5\%)Found here: \url{https://www.iea.org/reports/greenhouse-gas-emissions-from-energy-overview/global-ghg-emissions}; the Global Energy review of the iea is to  be found here: \url{https://www.iea.org/reports/global-energy-review-2021/co2-emissions}.

Natural sinks are, for example, forests, vegetation, soils. 
The epa report (\url{https://www.epa.gov/ghgemissions/inventory-us-greenhouse-gas-emissions-and-sinks-1990-2019}) includes information on sinks. 

Net emissions after taking sinks into account are estimated by the epa to 5,769.1 in 
\paragraph{Translation metric ton to gigatonne}
1.000.000.000 metric tons are 1 gigatonne 

\paragraph{Demand and Production approach}
The OECD, \url{https://www.oecd.org/sti/ind/carbondioxideemissionsembodiedininternationaltrade.htm} differentaites between a demand and a production-side approach to determine emissions on country level! I am now using the production approach. 

For now, I assume that the global emission target is given in gros emissions. I match the US gros emission target so that contribution in 2019 equals contribution to the global gros reduction. 

\subsection{Radiative Forcing}
\ar  leads to uncertainty in temperature response to emissions
(from \url{https://climate.mit.edu/explainers/radiative-forcing})

Radiative forcing is what happens when the amount of energy that enters the Earth’s atmosphere is different from the amount of energy that leaves it. Energy travels in the form of radiation: solar radiation entering the atmosphere from the sun, and infrared radiation exiting as heat. \textbf{If more radiation is entering Earth than leaving—as is happening today—then the atmosphere will warm up.} This is called radiative forcing because the difference in energy can force changes in the Earth’s climate.
Heat in, Heat out

Sunlight is always shining on half of the Earth’s surface. Some of this sunlight (about 30 percent) is reflected back to space. The rest is absorbed by the planet. But as with any warm object sitting in cold surroundings—and space is a very cold place—some energy from Earth is always radiating back out into space as heat.

Radiative forcing measures how much energy is coming in from the sun, compared to how much is leaving. The analysis needed to pin down this exact number is very complicated. \textbf{Many factors, including clouds, polar ice, and the physical properties of gases in the atmosphere, have an effect on this balancing act}, and each has its own level of uncertainty and its own difficulties in being precisely measured. However, we do know that today, more heat is coming in than going out.


Before the industrial era, radiative forcing was in very close balance, and the Earth’s average temperature was more or less stable. For this reason, researchers calculate radiative forcing based on a “baseline” year sometime before the beginning of world industrialization. For example, the Intergovernmental Panel on Climate Change uses 1750 as a baseline year.

Compared to this baseline, radiative forcing can directly measure the ways recent human activities have changed the planet’s climate. \textbf{The biggest change has been the greenhouse gases we have added to the atmosphere, which keep heat from escaping the Earth.} But there have been other changes too. For example, by \textbf{cutting down forests, we have exposed more of the Earth’s surface to sunlight. If that surface is darker than the forest cover, the Earth absorbs more solar radiation;} where it’s lighter, like in the arctic, more sunlight is reflected back into space.

Humans are also adding small particles called\textbf{ aerosols} to the air, from smokestacks, airplanes, and the tailpipes of cars. Aerosols make radiative forcing especially hard to measure, because their effects are highly complex and can work both ways. For example, bright aerosols (like sulfates from coal-burning) can help cool the atmosphere by reflecting light, while dark aerosols (like black carbon from diesel exhausts) absorb heat and lead to warming.

Finally, measures of radiative forcing also include any natural effects that have changed since the baseline year, such as changes in the sun’s output (which has caused a little more warming) and aerosols released into the atmosphere by volcanoes (which cause temporary cooling).

\subsection{Methane emissions: second important ghg}
(source \url{https://climate.mit.edu/ask-mit/why-do-we-compare-methane-carbon-dioxide-over-100-year-timeframe-are-we-underrating})
Methane is a colorless, odorless gas that’s produced both by nature (such as in wetlands when plants decompose underwater) and in industry (for example, natural gas is mostly made of methane). 
Methane is a quick fading (on average it tstays 10 years in the atmoshpere) ghg. But it is far more damaging than CO2, which sticks longer in the atmosphere; for centuries. 
Methane traps 100times more heat than CO2; Co2 closes the gap over time when  methane has broken down. Old standard: look at warming effect of methane over 100 years; but that is too long a horizon as climate change advances. 

Sources of methane: natural gas production, decompostation of plants under water. 
Methane. CH4

\textbf{(from NASA \url{https://svs.gsfc.nasa.gov/4799})}

Methane is a powerful greenhouse gas that traps heat 28 times more effectively than carbon dioxide over a 100-year timescale. Concentrations of methane have increased by more than 150\% since industrial activities and intensive agriculture began. After carbon dioxide, methane is responsible for about 23\% of climate change in the twentieth century. Methane is produced under conditions where little to no oxygen is available. About 30\% of methane emissions are produced by wetlands, including ponds, lakes and rivers. Another 20\% is produced by agriculture, due to a combination of livestock, waste management and rice cultivation. Activities related to oil, gas, and coal extraction release an additional 30\%. The remainder of methane emissions come from minor sources such as wildfire, biomass burning, permafrost, termites, dams, and the ocean. Scientists around the world are working to better understand the budget of methane with the ultimate goals of reducing greenhouse gas emissions and improving prediction of environmental change. For additional information, see the Global Methane Budget.
\subsection{Natural gas}
(source \url{https://www.eia.gov/energyexplained/natural-gas/})
fossil energy source! remains of plants and animals

natural gas can be produced from shale and other types of sedimentary rock formations by forcing water, chemicals, and sand down a well under high pressure (fracking); tis breaks up the formation, and releases the natural gas from the rock. 
\subsection{Energy sources}
(source: eia, \url{https://www.eia.gov/energyexplained/what-is-energy/sources-of-energy.php})

\subsection{Bioenergy with Carbon Capture and Storage (BECCS)}
source. \url{https://en.wikipedia.org/wiki/Bioenergy_with_carbon_capture_and_storage}

 is the process of extracting bioenergy from biomass and capturing and storing the carbon, thereby removing it from the atmosphere. Energy is extracted in useful forms (electricity, heat, biofuels, etc.) as the biomass is utilized through combustion, fermentation, pyrolysis or other conversion methods.
Wide deployment of BECCS is constrained by cost and availability of biomass.

\subsection{Biosequestration} capture and storage of the atmospheric co2 by continual or enhanced biological processes.; reforestation

\subsection{IPCC report 2018 \citep{Rogelj2018MitigationDevelopment.}}
\begin{itemize}
\item literature focused on demand side: 
\begin{itemize}
	\item \cite{Arrow2004AreMuch}: raise the question if consumption is too high
	\item 
	\cite{Bertram2018TargetedScenarios}:   change demand as a parameter in  model; motivation: taking global inequality into account alternative meausres (mitigation policies) to carbon taxes  become optimal. These include lifestyle changes \textbf{in addition to sector-specific carbon taxes!} (25\% lower energy demand and -20\% lower demand for agricultural products )
	\item \cite{Grubler2018ATechnologies} argue for a low increase in demand (projections)
	\item \cite{Liu2018SocioeconomicC}: lifestyle changes become more and more important under more stringent 1.5°C goal 
	\item \cite{VanVuuren2018AlternativeTechnologies}: analyse lifestyle changes motivated by uncertainty and risks surrounding CDRs (carbon dioxide removal) and their competition for land, yet many mitigation pathways to meet Paris Agreement rely on these technologies.
	
	\textit{This paper  explores  a  set  of  what-if  scenarios  that  explore  these  alter-native  assumptions,  and  analyses  the  extent  to  which  they  reduce  the  need  for  CDR.  The  evaluated  measures  (Table  1)  have  been  mentioned in scientific literature and could possibly limit CDR use. } (p. 2) \ar hence: assuming a lower demand what does this imply?
	
	\textit{The  lifestyle  change  scenario  (LiStCh)  assumes  a  radi-cal  value  shift  towards  more  environmentally  friendly  behaviour,  including a healthy, low-meat diet, changes in transport habits and a reduction of heating and cooling levels at homes. Such a shift could be motivated by both environmental and health con} (p.2)
	
	Their models also take other forcers into account 
	
	\textit{ Given  the  possible  disadvantages  of  BECCS,  it  is  important  to  seriously  discuss  and  appraise such alternative pathways. This could focus on issues such as feasibility, social acceptance, associated costs and benefits, requir-ing input from other scientific disciplines to complement the model-based  scenarios. } p. vorletzte
	
	How they introduce a reduction in demand: 
	\begin{itemize}
\item reduced meat consumption; higher consumption of pulses and oilcrops
\item reduction in food waste
(by households and during production process)
\item transport changes: (1) reduced volume of transport, (2) reduction of energy intense modes of transport, \ar (3)  less private vehicle use and increased car sharing; less motorised options; also implies \ar (4) lower air travel demand
\item less residential energy use
\item reduced appliance ownership (a maximum of two per household) and use (less standby, smarter use)!
\item lower demand for plastic and chemicals 
	\end{itemize}
\end{itemize}
\end{itemize}

\section{Labour}
\subsection{Elasticity of labour}
\cite{Bick2018HowImplications}
\begin{itemize}
	\item \ar informative on the behaviour of individuals in the cross-section
	\item from the abstract: \textit{Within
	countries, hours worked per worker are also decreasing in the individual wage for most countries, though in the richest countries,
	hours worked are flat or increasing in the wage.}
\item there is heterogeneity in the wage elasticity of labour across countries! \ar could test theory in data? \ar those countries identified by \cite{Bick2018HowImplications} as having a negative relation of wage and hours worked \ar different effect of fiscal policy on green labour supply 
\item will imply heterogeneity in policy recommendation!:
\begin{itemize}
	\item poorer countries: negative relationship of individual wage and hours worked \ar higher income tax \ar lower wage \ar work more hours; 
	\ar income effect dominates! 
	\item richer countries:  positive relationship; higher income tax \ar lower wage \ar lower hours worked!
	\ar substitution effect dominates!!
\end{itemize}
\item over time hours worked per adult in the US have been falling. 
\item similar patterns when looking at the cross-section of countries
\item extensive margin: employment shares are falling with GDP for poor to medium countries, modest increase when country is rich
\end{itemize}
\cite{Boppart2019LaborPerspectiveb}
\begin{itemize}
	\item a long run perspective, but in the end I am also looking at a long run model!
	\item only look at the intensive margin: hours worked per worker! 
	\item \textbf{they argue for a higher income effect in the long run} BUT ON AGGREGATE 
	\item They look at a time dimension, this is what I am looking at too.
\end{itemize}
\ar GOAL: on the one hand match the cross-sectional wage elasticity, and on the other hand match the elasticity over time
\cite{LansBovenberg1994EnvironmentalTaxation}
\begin{itemize}
	\item they assume an upward sloping labour-supply curve \ar as the wage rate rises, households work more \ar as the wage rate falls, households work less
	\item an increase in the pollution tax reduces employment if the \textbf{uncompensated wage elasticity of labor supply} is positive \ar i.e. the substitution effect exceeds the income effect
\end{itemize}

\subsection{Skill and environment}
\tr{\textbf{Question} How is skill accumulated here? Retraining to higher skill level or long run decision based on education?\ar pre-job entry; How calibrated?\ar not a quantitative model; Inequality?\ar choice variable skill differences due to OLG structure...but only live for one period... not sure where both skill levels result from when households are all the same...}
\begin{itemize}
	\item \cite{Vona2018EnvironmentalExploration}
	\begin{itemize}
		\item differentiate also occupations\ar engineering skills and managering skills; this  paper less on skill but more on jobs?
		\item ``\textit{we identify two core sets of green skills for which green jobs differ from non-green jobs: engineering skills, and managerial skills }'' (p.2) \ar these skills (=jobs) are especially used in green occupations; \ar jobs which are used heavily in the green sector (p.3)
		\item environmental regulatory policies increase demand for some green skills (engineering, managing); no impact on employment in general
		\item ``\textit{The adjustment costs from job losses can be exacerbated when the skill profile of expanding jobs does not match the skill profile of contracting jobs}'' (p.3)
		\item ``\textit{while the skill gap between green and brown jobs within the same occupational group is generally small }'' (p.4) \ar model as neutral jobs, ``\textit{... exceptions emerge: largest skill gaps occur in construction and extraction occupations [...] important for climate change}''
		\item transitioning to greener production requires a transition in skills
	\end{itemize}
	\item \cite{Borissov2019CarbonDevelopment}
	\begin{itemize}
		\item skills are crucial a determinant of green growth as the labour required for green production is special: higher skills/ higher human capital accumulation
		\item on this they cite policy recommendation papers and \cite{Vona2018EnvironmentalExploration} which they cite as "green skills are closely related to the design, production, management and monitoring of technology and conclude that education emerges as a critical ingredient in the policy mix to promote sustainable economic growth"
		\item their focus seems not to lie on inequality!
		\item their idea: requiring non-developed countries to reduce emissions \ar higher carbon taxes \ar increase in human capital investment since the green sector uses high-skilled labour \ar economic growth! (\textit{how measured?}) \ar a win-win situation
		\item North-South knowledge spillover: if the carbon tax leads to knowledge growth in the green sector this might spread to the South even if the South itself does not levy a pollution tax
		\item human capital accumulation is the driver of growth in this model (not innovations! no directed technical change)
		\item no a-priori inequality in their model! Skill is a choice variable! 
		\item "clean sectors  tend to be skill intensive"
		\item positive intergenerational spill over in skills! (longer-run effects)
		\item positive effect of human capital accumulation on TFP see Lucas 1988, Glomm and Ravikumar 1992
		\item \textbf{incentives to human capital accumulation through carbon taxes!}
		\item \textbf{Model}
		\begin{itemize}
			\item model of successive generations: OLG
			\item no preference heterogeneity, acquiring education costs labour income
			\item skilled, unskilled is a discrete choice
			\item spill over of skills within country
			\item production only requires labour skilled and unskilled,
			\item sector specific goods are perfect substitutes! they produce the same stuff
			\item general production function, but also Cobb-Douglas, then MPhigh skill relative to MP lowskill in clean sector is higher in the clean than in the dirty sector if income share of high skilled in clean is higher than in dirty. \ar calculate relative marginal product of skilled labour in clean and dirty sector
			\item section 5: endogenous growth version: higher share of high-skilled \ar higher growth\ar carbon tax implies higher growth!
			\item carbon tax leads to growth! in their model due to intensified skill accumulation
		\end{itemize}
	\end{itemize}
	
	\item \cite{Consoli2016DoCapital}
	\begin{itemize}
		\item findings: labour force characteristics of green and non-green jobs
		\begin{itemize}
			\item green jobs: high-level cognitive and interpersonal, higher level of formal education, more work experience and on-the-job training compared to non-green; 
			\item use O*NET (Occupational Information Network) comprising 905 occupations
			\item in new occupations which emerge due to a higher demand of green skills on-the-job trainging is a distinctive feature but not in already existing occupations
			\item review policy effects on employment in literature; comment that distinction between job characteristics in these papers are missing
			\item green occupations: (estimates are within SOC3 digit occupations, that is, they are conditional differences in expectations; not unconditional, in the paper I would want to take macro occupational differences into account too, its not only about being green but also that these green jobs are within some specific group: driven by heterogeneity in the average skill content in the macro group ); and green occupations are more within high skilled occupational groups, on the other hand I also only need to focus on occupations where a distinction between green and non-green can be made, and another sector that is neutral (quantitative analysis)\ar the conditional estimates fit well
			\item findings:
			\begin{itemize}
				\item 
				significantly more non-routine tasks and significantly less routine tasks than in the non-green counterparts, p.1052
				\item 19 percent more years of eductaion ($\approx$ 13 weeks), 43\% more years of experience ( $\approx$ 10 months at the mean), 41\% more years of trainnig,($\approx$ 15 weeks); p.1053
				\item green enhanced jobs are more exposed to all measures of technology (p.1053) 
			\end{itemize}
		\end{itemize}
	\end{itemize}
\end{itemize}

\section{Model}
\subsection{Model as in \cite{Fried2018ClimateAnalysis}}
\begin{align*}
\text{Household}\\
& C_t=w_{lft}L_{lft}+w_{lgt}L_{lgt}+w_{lnt}L_{lnt}+w_{sft}S_{sft}+w_{sgt}S_{sgt}+w_{snt}S_{snt}+\\ &\int_{0}^{1}\pi_{fit}+\pi_{git}+\pi_{nit}di+T_t\\
\text{Final good producer optimality}&\\
\text{Optimality}\ \vspace{4mm}& \delta_yY_t^\frac{1}{\varepsilon_y}E_t^{-\frac{1}{\varepsilon_e}}=p_{Et}\\
& (1-\delta_y)Y_t^\frac{1}{\varepsilon_y}N_t^{-\frac{1}{\varepsilon_e}}=p_{Nt}\\
&
p_{Et}E_t^\frac{1}{\varepsilon_e}\tilde{F}_t^{-\frac{1}{\varepsilon_e}}=p_{\tilde{F}t}\\
& p_{Et}E_t^\frac{1}{\varepsilon_e}G_t^{-\frac{1}{\varepsilon_e}}=p_{Gt}\\
& p_{\tilde{F}t}\tilde{F}_t^\frac{1}{\varepsilon_f}\delta_fF_t^{-\frac{1}{\varepsilon_f}}=p_{Ft}+\tau_{f}\\
& p_{\tilde{F}t}\tilde{F}_t^\frac{1}{\varepsilon_f}(1-\delta_f)O_t^{-\frac{1}{\varepsilon_f}}=p_{Ot}+\tau_{o}\\
\text{Definitions prices}\ \vspace{4mm}&
p_{yt}= \left[\delta_y^{\varepsilon_y}p_{Et}^{1-\varepsilon_y}+(1-\delta_y)^{\varepsilon_y}p_{Nt}^{1-\varepsilon_y}\right]^\frac{1}{1-\varepsilon_y}\\
& p_{Et}= \left[p_{\tilde{F}t}^{1-\varepsilon_e}+p_{Gt}^{1-\varepsilon_y}\right]^\frac{1}{1-\varepsilon_e}\\
& p_{\tilde{F}t}= \left[\delta_f^{\varepsilon_f}(p_{Ft}+\tau_{f})^{1-\varepsilon_f}+(1-\delta_f)^{\varepsilon_f}(p_{nt}+\tau_{o})^{1-\varepsilon_f}\right]^\frac{1}{1-\varepsilon_f}\\
\text{Production}\ \vspace{4mm}& 
Y_t=\left(\delta_yE_t^\frac{\varepsilon_y-1}{\varepsilon_y}+(1-\delta_y)N_t^\frac{\varepsilon_y-1}{\varepsilon_y}\right)^\frac{\varepsilon_y}{\varepsilon_y-1}\\
&E_t=\left(\tilde{F}_t^\frac{\varepsilon_e-1}{\varepsilon_e}+G_t^\frac{\varepsilon_e-1}{\varepsilon_e}\right)^\frac{\varepsilon_e}{\varepsilon_e-1}\\
&\tilde{F}_t=\left(\delta_fF_t^\frac{\varepsilon_f-1}{\varepsilon_f}+(1-\delta_f)O_t^\frac{\varepsilon_f-1}{\varepsilon_f}\right)^\frac{\varepsilon_f}{\varepsilon_f-1}\\
\text{Intermediate good producers}\\
\text{Production}\ \vspace{4mm}& F_t= (\alpha_f^2p_{Ft})^\frac{ \alpha_f}{1-\alpha_f}A_{ft}L_{ft}\\
&N_t= (\alpha_n^2p_{Nt})^\frac{ \alpha_n}{1-\alpha_n}A_{nt}L_{nt}\\
&G_t= (\alpha_g^2p_{Gt})^\frac{ \alpha_g}{1-\alpha_g}A_{gt}L_{gt}\\
\text{ Definition agg. Technology}\ \vspace{4mm}&
A_t= \frac{\rho_gA_{gt}+\rho_nA_{nt}+\rho_fA_{ft}}{\rho_n+\rho_g+\rho_f}
\end{align*}

\begin{align*}
\text{Labour demand}\\
& w_{lft}=p_{Ft}(1-\alpha_f)(\alpha_f^2p_{Ft})^\frac{\alpha_f}{1-\alpha_f}A_{ft}\\
& w_{lnt}=p_{Nt}(1-\alpha_n)(\alpha_n^2p_{Nt})^\frac{\alpha_n}{1-\alpha_n}A_{nt}\\
& w_{lgt}=p_{Gt}(1-\alpha_g)(\alpha_g^2p_{Gt})^\frac{\alpha_g}{1-\alpha_g}A_{gt}\\
\text{Mashine demand}\ \vspace{4mm}\\
&x_{fit}= \left(\alpha_f^2 p_{Ft}\right)^\frac{1}{1-\alpha_f}L_{ft}A_{fit}\\
&x_{nit}= \left(\alpha_n^2 p_{Nt}\right)^\frac{1}{1-\alpha_n}L_{nt}A_{nit}\\
&x_{git}= \left(\alpha_g^2 p_{Gt}\right)^\frac{1}{1-\alpha_g}L_{gt}A_{git}\\
\text{Mashine producers}\\
\text{Price setting}\ \vspace{4mm}&p_{fit}^x=\frac{1}{\alpha_f}\\
&p_{nit}^x=\frac{1}{\alpha_n}\\
&p_{git}^x=\frac{1}{\alpha_g}\\ 
\text{Demand Scientists}\ \vspace{4mm}&
w_{sft}=\frac{\eta \gamma \alpha_f A_{ft-1}^{1-\phi}A_{t-1}^{\phi}\left(\frac{S_{ft}}{\rho_f}\right)^{\eta}p_{Ft}F_t}{\frac{1}{1-\alpha_f}S_{ft}A_{ft}}\\
&
w_{snt}=\frac{\eta \gamma \alpha_n A_{nt-1}^{1-\phi}A_{t-1}^{\phi}\left(\frac{S_{nt}}{\rho_n}\right)^{\eta}p_{Nt}N_t}{\frac{1}{1-\alpha_n}S_{nt}A_{nt}}\\
&
w_{sgt}=\frac{\eta \gamma \alpha_g A_{gt-1}^{1-\phi}A_{t-1}^{\phi}\left(\frac{S_{gt}}{\rho_g}\right)^{\eta}p_{Gt}G_t}{\frac{1}{1-\alpha_g}S_{gt}A_{gt}}\\
\text{Innovation}\ \vspace{4mm}&
A_{fit}=A_{ft-1}\left(1+\gamma\left(\frac{S_{fit}}{\rho_f}\right)^{\eta}\left(\frac{A_{t-1}}{A_{ft-1}}\right)^{\phi}\right) \\
&
A_{nit}=A_{nt-1}\left(1+\gamma\left(\frac{S_{nit}}{\rho_n}\right)^{\eta}\left(\frac{A_{t-1}}{A_{nt-1}}\right)^{\phi}\right) \\
&
A_{git}=A_{gt-1}\left(1+\gamma\left(\frac{S_{git}}{\rho_g}\right)^{\eta}\left(\frac{A_{t-1}}{A_{gt-1}}\right)^{\phi}\right) \\
\text{Markets}&\\
&S_{ft}+S_{nt}+S_{gt}=S\\
&L_{ft}+L_{nt}+L_{gt}=L\\
&C_{t}+\int_{0}^{1} x_{fit}+x_{nit}+x_{git}d_i+P_{Ot}O_t=Y_t\\
&P_{Ot} \text{taken as given, imports}
\end{align*}

\subsection{Balanced growth path in \cite{Fried2018ClimateAnalysis}}
She assumes that the ratio of prices is constant and energy prices themselves are constant (p.103) (\textit{that sounds wrong}). Assuming that the spillover effect is sufficiently strong (that is, $\phi$ is large) then a balanced growth path may exist on which the ratio of green and fossil technology, $A_g/A_f$, is constant. This ratio being constant follows from constant price ratios! 

In equilibrium it has to hold that 
\begin{align*}
P_{Gt}G_t= P_{\tilde{F}t}\tilde{F}\left(\frac{P_{\tilde{F}t}}{P_{Gt}}\right)^{\varepsilon_e-1}.
\end{align*} 

\subsection{Model}
\begin{align}
\text{\textbf{Household}}& \max \frac{C_t^{1-\theta}}{1-\theta}-\chi\frac{z_hh_{ht}^{1+\sigma}+z_lh_{lt}^{1+\sigma}}{1+\sigma}-\chi_s\frac{S^{1+\sigma}}{{1+\sigma}} %when z is also to the power of 1+sigma than, the higher zh the lower hours supplied! Not reasonable
\\
\text{Budget}\ \vspace{4mm}& C_t=z_h \lambda_t (w_{ht}h_{ht})^{1-\tau_{lt}}+z_l \lambda_t (w_{lt}h_{lt})^{1-\tau_{lt}}+T^{Gov}_t\\
\text{Optimality}\ \vspace{4mm}
& C_t^{-\theta}= \mu_tp_{yt}\\
& \chi h_{ht}^{\sigma}=\mu_t \lambda_t(1-\tau_{lt})w_{ht}^{1-\tau_{lt}}h_{ht}^{-\tau_{lt}}-\gamma_{ht}/z_h\\
& \chi h_{lt}^{\sigma}=\mu_t \lambda_t(1-\tau_{lt})w_{lt}^{1-\tau_{lt}}h_{lt}^{-\tau_{lt}}-\gamma_{lt}/(1-z_h)\\
%&( h_{st})^{\sigma}=\mu_t \lambda_t(1-\tau_{lt})w_{st}^{1-\tau_{lt}}h_{st}^{-\tau_{lt}}\\
\Rightarrow\ \ & \frac{h_{ht}}{h_{lt}}=\left(\frac{w_{ht}}{w_{lt}}\right)^{\frac{1-\tau_{lt}}{{\sigma+\tau_{lt}}}}\ \text{(Interior solution)}
\\
& \chi_s S^\sigma =\mu w_s-\gamma_{st}\ \text{(scientist income confiscated by gov.)}\\
\text{\textbf{Final good and Energy producers}}&\\
\text{Optimality}\ \vspace{4mm}&
\frac{E_t}{N_t}=\frac{\delta_y}{(1-\delta_y)}\left(\frac{p_{Nt}}{p_{Et}}\right)^{\varepsilon_y} p_{yt}\delta_yY_t^\frac{1}{\varepsilon_y}E_t^{-\frac{1}{\varepsilon_e}}=p_{Et}\\
& p_{yt}(1-\delta_y)Y_t^\frac{1}{\varepsilon_y}N_t^{-\frac{1}{\varepsilon_e}}=p_{Nt}\\
&
p_{Et}E_t^\frac{1}{\varepsilon_e}{F}_t^{-\frac{1}{\varepsilon_e}}=p_{{F}t}\\
& p_{Et}E_t^\frac{1}{\varepsilon_e}G_t^{-\frac{1}{\varepsilon_e}}=p_{Gt}\\
%& p_{\tilde{F}t}\tilde{F}_t^\frac{1}{\varepsilon_f}\delta_fF_t^{-\frac{1}{\varepsilon_f}}=p_{Ft}+\tau_{f}\\
%& p_{\tilde{F}t}\tilde{F}_t^\frac{1}{\varepsilon_f}(1-\delta_f)O_t^{-\frac{1}{\varepsilon_f}}=p_{Ot}+\tau_{o}
%\\
\text{Definitions prices}\ \vspace{4mm}&
p_{yt}= \left[\delta_y^{\varepsilon_y}p_{Et}^{1-\varepsilon_y}+(1-\delta_y)^{\varepsilon_y}p_{Nt}^{1-\varepsilon_y}\right]^\frac{1}{1-\varepsilon_y}\\
& p_{Et}= \left[p_{{F}t}^{1-\varepsilon_e}+p_{Gt}^{1-\varepsilon_y}\right]^\frac{1}{1-\varepsilon_e}\\
%& p_{\tilde{F}t}= \left[\delta_f^{\varepsilon_f}(p_{Ft}+\tau_{f})^{1-\varepsilon_f}+(1-\delta_f)^{\varepsilon_f}(p_{nt}+\tau_{o})^{1-\varepsilon_f}\right]^\frac{1}{1-\varepsilon_f}
%\\
\text{Production}\ \vspace{4mm}& 
Y_t=\left(\delta_yE_t^\frac{\varepsilon_y-1}{\varepsilon_y}+(1-\delta_y)N_t^\frac{\varepsilon_y-1}{\varepsilon_y}\right)^\frac{\varepsilon_y}{\varepsilon_y-1}\\
&E_t=\left({F}_t^\frac{\varepsilon_e-1}{\varepsilon_e}+G_t^\frac{\varepsilon_e-1}{\varepsilon_e}\right)^\frac{\varepsilon_e}{\varepsilon_e-1}\\
%&\tilde{F}_t=\left(\delta_fF_t^\frac{\varepsilon_f-1}{\varepsilon_f}+(1-\delta_f)O_t^\frac{\varepsilon_f-1}{\varepsilon_f}\right)^\frac{\varepsilon_f}{\varepsilon_f-1}\\
\text{\textbf{Intermediate good producers}}&\\
\text{Production}\ \vspace{4mm}& F_t= (\alpha_f(p_{Ft}(1-\tau_{ft})))^\frac{ \alpha_f}{1-\alpha_f}A_{ft}L_{ft}\\
&N_t= (\alpha_np_{Nt})^\frac{ \alpha_n}{1-\alpha_n}A_{nt}L_{nt}\\
&G_t= (\alpha_gp_{Gt})^\frac{ \alpha_g}{1-\alpha_g}A_{gt}L_{gt}\\
%\end{align}
%
%\begin{align}
\text{Labour demand}\label{eq:lab_demand}\\
& w_{lft}=(p_{Ft}(1-\tau_{ft}))^\frac{1}{1-\alpha_f}(1-\alpha_f)(\alpha_f)^\frac{\alpha_f}{1-\alpha_f}A_{ft}\\
& w_{lnt}=p_{Nt}^\frac{1}{1-\alpha_n}(1-\alpha_n)(\alpha_n)^\frac{\alpha_n}{1-\alpha_n}A_{nt}\\
& w_{lgt}=p_{Gt}^\frac{1}{1-\alpha_g}(1-\alpha_g)(\alpha_g)^\frac{\alpha_g}{1-\alpha_g}A_{gt}
\\
\text{Machine demand}&
\\
&x_{fit}= \left(\alpha_f p_{Ft}(1-\tau_{ft})\right)^\frac{1}{1-\alpha_f}L_{ft}A_{fit}\\
&x_{nit}= \left(\alpha_n p_{Nt}\right)^\frac{1}{1-\alpha_n}L_{nt}A_{nit}\\
&x_{git}= \left(\alpha_g p_{Gt}\right)^\frac{1}{1-\alpha_g}L_{gt}A_{git}
\\
\text{\textbf{Labour producers}}&
\\
\text{Production}\ \vspace{4mm}& L_{ft}=h_{hft}^{\theta_{f}}h_{lft}^{1-\theta_{f}}\\
& L_{nt}=h_{hnt}^{\theta_{n}}h_{lnt}^{1-\theta_{n}}\\
& L_{gt}=h_{hgt}^{\theta_{g}}h_{lgt}^{1-\theta_{g}}\\
\ \\
\text{Optimality}\ \vspace{4mm}& h_{hft}= \theta_{f}L_{ft}\frac{w_{lft}}{w_{ht}}\label{eq:opt_lab_pro}\\
& h_{hnt}= \theta_{n}L_{nt}\frac{w_{lnt}}{w_{ht}}\\
& h_{hgt}= \theta_{g}L_{gt}\frac{w_{lgt}}{w_{ht}}\\
& h_{lft}= (1-\theta_{f})L_{ft}\frac{w_{lft}}{w_{lt}}\label{eq:opt_lab_pro_low}\\
& h_{lnt}= (1-\theta_{n}) L_{nt}\frac{w_{lnt}}{w_{lt}}\\
& h_{lgt}= (1-\theta_{g}) L_{gt}\frac{w_{lgt}}{w_{lt}}\\
%\end{align}
%
%\begin{align}
\text{\textbf{Machine producers}}\\
\text{Price setting}\ \vspace{4mm}&p_{fit}^x=\frac{1}{\alpha_f(1+\zeta_f)}\\
&p_{nit}^x=\frac{1}{\alpha_n(1+\zeta_n)}\\
&p_{git}^x=\frac{1}{\alpha_g(1+\zeta_g)}
\\ 
\text{Demand Scientists}\ \vspace{4mm}&
w_{sft}=\frac{\eta \gamma A_{ft-1}^{1-\phi}A_{t-1}^{\phi}\left(\frac{S_{ft}}{\rho_f}\right)^{\eta}p_{Ft}(1-\tau_{ft})F_t}{\frac{1}{1-\alpha_f}S_{ft}A_{ft}}\\
&
w_{snt}=\frac{\eta \gamma  A_{nt-1}^{1-\phi}A_{t-1}^{\phi}\left(\frac{S_{nt}}{\rho_n}\right)^{\eta}p_{Nt}N_t}{\frac{1}{1-\alpha_n}S_{nt}A_{nt}}\\
&
w_{sgt}=\frac{\eta \gamma  A_{gt-1}^{1-\phi}A_{t-1}^{\phi}\left(\frac{S_{gt}}{\rho_g}\right)^{\eta}p_{Gt}G_t}{\frac{1}{1-\alpha_g}S_{gt}A_{gt}(1-\tau_{st})}\\
\text{Innovation}\ \vspace{4mm}&
A_{fit}=A_{ft-1}\left(1+\gamma\left(\frac{S_{fit}}{\rho_f}\right)^{\eta}\left(\frac{A_{t-1}}{A_{ft-1}}\right)^{\phi}\right) \\
&
A_{nit}=A_{nt-1}\left(1+\gamma\left(\frac{S_{nit}}{\rho_n}\right)^{\eta}\left(\frac{A_{t-1}}{A_{nt-1}}\right)^{\phi}\right) \\
&
A_{git}=A_{gt-1}\left(1+\gamma\left(\frac{S_{git}}{\rho_g}\right)^{\eta}\left(\frac{A_{t-1}}{A_{gt-1}}\right)^{\phi}\right) \\
%\text{Demand Scientists}\ \vspace{4mm}&
%w_{sft}=\frac{\eta \gamma \alpha_f A_{ft-1}\left(\frac{S_{ft}}{\rho_f}\right)^{\eta}p_{Ft}F_t}{\frac{1}{1-\alpha_f}S_{ft}A_{ft}}\label{eq:demand_sc}\\
%&
%w_{snt}=\frac{\eta \gamma \alpha_n A_{nt-1}\left(\frac{S_{nt}}{\rho_n}\right)^{\eta}p_{Nt}N_t}{\frac{1}{1-\alpha_n}S_{nt}A_{nt}}\\
%&
%w_{sgt}=\frac{\eta \gamma \alpha_g A_{gt-1}\left(\frac{S_{gt}}{\rho_g}\right)^{\eta}p_{Gt}G_t}{\frac{1}{1-\alpha_g}S_{gt}A_{gt}}\\
%\text{Innovation}\ \vspace{4mm}&
%A_{fit}=A_{ft-1}\left(1+\gamma\left(\frac{S_{fit}}{\rho_f}\right)^{\eta}\right) \\
%&
%A_{nit}=A_{nt-1}\left(1+\gamma\left(\frac{S_{nit}}{\rho_n}\right)^{\eta}\right) \\
%&
%A_{git}=A_{gt-1}\left(1+\gamma\left(\frac{S_{git}}{\rho_g}\right)^{\eta}\right) \\
%&A_t=\max\{A_{nt}, A_{ft}, A_{gt}\}\\`
%DROP THE FOLLOWING AS IT DOES NOT GROW AT A CONSTANT RATE IF SHARES ARE CHANGING! &A_t= \frac{\rho_fA_{ft}+\rho_gA_{gt}+\rho_nA_{nt}}{\rho_f+\rho_n+\rho_g}\\
\text{\textbf{Government}}&\\
&T_t=\int_{0}^{1}\pi_{fit}+\pi_{git}+\pi_{nit}di+z_h(w_{ht}h_{ht}-\lambda_t(w_{ht}h_{ht})^{(1-\tau_{lt})})\\&+z_l(w_{lt}h_{lt}-\lambda_t(w_{lt}h_{lt})^{(1-\tau_{lt})})+w_{st}s_{ft}+w_{st}s_{gt}+w_{st}s_{nt}\\ &+\tau_{ct}p_{ft}(\omega_FF_t)-w_{sgt}\tau_{st}s_{gt} \\ &+\int_{0}^{1} p_{fit}\zeta_{ft}x_{fit}+p_{git}\zeta_{gt}x_{git}+ p_{hit}\zeta_{ht}x_{hit}di\\
\text{with}\ \vspace{4mm}&\zeta_{jt}=\frac{1-\alpha_j}{\alpha_j} \\
\text{simplified}\ \vspace{4mm} & T_t= z_h(w_{ht}h_{ht}-\lambda_t(w_{ht}h_{ht})^{(1-\tau_{lt})})\\&+z_l(w_{lt}h_{lt}-\lambda_t(w_{lt}h_{lt})^{(1-\tau_{lt})})+\tau_{ct}p_{ft}(\omega_FF_t) \\
\text{\textbf{Markets}}&\\
& S_{ft}+ S_{nt}+ S_{gt}=S_t\\
&h_{hft}+h_{hnt}+h_{hgt}=z_{h} h_{ht}\\
&h_{lft}+h_{lnt}+h_{lgt}=z_{l} h_{lt}\\
&C_{t}+\int_{0}^{1} x_{fit}+x_{nit}+x_{git}d_i=Y_t
\end{align}

\subsection{BGP: which can accomodate a trend in hours and diverging productivity shares}
\textbf{\tr{But this version assumes constant labour shares \ar this implies constant mashine growth and labour growth in each sector. Then again, on the BGP labour supply reduces, I only assume this affects sector labour input proportionately. There may also not be any growth in machines. }}
From the optimality condition in skill demand by labour producing firms, equations \ref{eq:opt_lab_pro} and \ref{eq:opt_lab_pro_low}, and the price paid by intermediate good producers in sector $j\in\{F, G, N\}$, $w_{ljt}$, equations \ref{eq:lab_demand}, yields the sector specific demand for high and low skill as a function of output in this sector. Substituting these equations in the intermediate good production function determines technology as a function of prices:
\begin{align}
A_{jt} = \left[\alpha_j^{2\frac{\alpha_j}{1-\alpha_j}}p_{jt}^\frac{1}{1-\alpha_j}(1-\alpha_j)\left(\frac{1-\theta_j}{w_{lt}}\right)^{1-\theta_j}\left(\frac{\theta_j}{w_{ht}}\right)^{\theta_j}\right] ^{-1}
\end{align}
Under the assumption of a stable wage premium, one can detrend technological progress as:
\begin{align}
\hat{A_{jt}}:=\frac{A_{jt}p_{jt}^\frac{1}{1-\alpha_j}}{w_{ht}}= \left[\alpha_j^{2\frac{\alpha_j}{1-\alpha_j}}(1-\alpha_j)\left(1-\theta_j\right)^{1-\theta_j}\theta_j^{\theta_j}\left(\frac{w_{ht}}{w_{lt}}\right)^{1-\theta_j}\right] ^{-1}\label{eq:A_det}
\end{align}
Both wage rates for high and low skill labour hence grow at the rate
\begin{align}
\gamma_{w}=\left(\frac{p_{jt+1}}{p_{jt}}\right)^\frac{1}{1-\alpha_j}\frac{A_{jt+1}}{A_{jt}}-1  \ \forall \ j. 
\end{align}
Hence, free skill movement across labour input firms, implies that on a BGP
\begin{align}
\left(\frac{p_{gt+1}}{p_{gt}}\right)^\frac{1}{1-\alpha_g}\frac{A_{gt+1}}{A_{gt}}=\left(\frac{p_{ft+1}}{p_{ft}}\right)^\frac{1}{1-\alpha_f}\frac{A_{ft+1}}{A_{ft}}=\left(\frac{p_{nt+1}}{p_{nt}}\right)^\frac{1}{1-\alpha_n}\frac{A_{nt+1}}{A_{nt}}. \label{eq:const_prA} 
\end{align}

\tr{Drop assumption that input shares are constant}
These conditions ensure that relative expenditures on intermediate goods are constant \textbf{whenever the labour input ratios are constant}. Substituting intermeidate good production into the expenditure ratio $p_{ft}F_t/(p_{gt}G_t)$  yields
\begin{align}
&\frac{p_{ft+1}^\frac{1}{1-\alpha_f}A_{ft+1}L_{ft+1}}{p_{gt+1}^\frac{1}{1-\alpha_g}A_{gt+1}L_{gt+1}}= \frac{p_{ft}^\frac{1}{1-\alpha_f}A_{ft}L_{ft}}{p_{gt}^\frac{1}{1-\alpha_g}A_{gt}L_{gt}}\\
&\Leftrightarrow \frac{L_{ft+1}}{L_{gt+1}}=\frac{L_{ft}}{L_{gt}},
\end{align}
where the second line follows from \ref{eq:const_prA}. 

Note that the assumption that the ratio of technological progress is constant over time is necessary to have aggregate technology, $A_t$, as defined in Fried constant. I don't want to make this assumption to allow for zero growth in the fossil sector. Therefore, I define the leading technology as 
\begin{align}
A_t= \max\{A_{nt}, A_{gt}, A_{ft}\}
\end{align}
Since on the BGP each technology growths at a constant rate, the leading technology growths at a constant rate  whenever the fastest growing technology is also the leading one in levels.

The assumption of a stable wage premium together with a constant progressivity parameter ensures that relative skill supply on the BGP is stable, too: 
\begin{align}
\frac{H_{ht}}{H_{lt}}=\left(\frac{w_{ht}}{w_{lt}}\right)^\frac{1-\tau_{lt}}{1+\sigma}\frac{z_h}{z_l}.
\end{align}

A constant wage ratio also implies that relative skill employment in each sector is constant, $\frac{h_{hj}}{h_{lf}}$, which follows from skill demand by labour producers. 


Observe that skill shares employed in each sector, $\frac{h_{pjt}}{H_{pt}} \forall \ j\ \text{and} \ p\in\{h,l\}$, are constant given constant expenditure ratios. 
To see this, substitute demand for skill inputs by labour producing firms in the market clearing for high skill labour. Substituting labour demand by intermediate good producers yields
\begin{align}
w_{ht}H_{ht}= \left(\underbrace{\theta_f (1-\alpha_f)\frac{p_{ft}F_t}{p_{gt}G_t}+\theta_{g}(1-\alpha_g)+\theta_n(1-\alpha_n)\frac{p_{nt}N_t}{p_{gt}G_t}}_{:=\vartheta_t}\right)p_{gt}G_t
\end{align}
Replacing $G_t$ by its production function and substituting $L_{gt}$ using the production function and optimal skill ratios implies
\begin{align}
\frac{h_{lgt}}{H_{ht}}=\left(\frac{1}{\vartheta_t\alpha_g^{2\frac{\alpha_g}{1-\alpha_g}}\left(\frac{\theta_g}{1-\theta_g}\frac{w_{lt}}{w_{ht}}\right)^{\theta_{g}}}\right)\frac{w_{ht}}{A_{gt}p_{gt}^\frac{1}{1-\alpha_g}}.
\end{align}
Since expenditure shares are constant, so is the first multiplier. The second multiplier is constant following equation \ref{eq:A_det}. Since $\frac{H_h}{H_l}$ is constant by households optimality condition, above equation implies that $\frac{h_{lgt}}{H_{lt}}$ is time invariant. Note that I assume that skill shares (relative to total skill supplied) are stable on the BGP. 
\paragraph{Scientists and the marginal gains from innovation}
Scientists are in fixed supply. They receive the competitive wage rate and consume their income. Their income is not taxed and they do not receive transfers. This is to avoid redistribution from scientists to households and vice versa.  (In an extension could assume preferential tax schemes for scientists.) Could also add income from scientists to representative household but let it not be taxed. 
%The amount of scientists on the BGP is flexible in order to keep sector-specific technological growth constant. This follows from the law of motion from technologym let $\gamma_{Aj}$ denote technology growth in sector j:
%\begin{align}
%S_{jt}=\left(\frac{\gamma_{Aj}}{\gamma}\right)^\frac{1}{\eta}\rho_j{A_{t-1}}^{\frac{-\phi}{\eta}}. \label{eq:scien}
%\end{align}
%The amount of scientists varies with overall productivity. For $\eta>0$ and $\phi>0$ the higher output the lower the amount of scientists employed. 
%The marginal profit of innovation positively depends on aggregate technology. 


In equilibrium, wages for scientists, i.e. the marginal product of innovations, are determined by equations \ref{eq:demand_sc}. 
Free movement of scientists requires that
\begin{align}
\left(\frac{S_{gt}}{S_{ft}}\right)^{\eta-1}= \frac{(1-\alpha_f)\alpha_f}{(1-\alpha_g)\alpha_g}\frac{p_{ft}F_t}{p_{gt}G_t}\left(\frac{\rho_g}{\rho_f}\right)^\eta.
\end{align}
Note, that in equilibrium, the equation states that the higher the ratio of scientists in sector g relative to sector f, the higher the technology gap in favour of sector g, under the assumption of increasing returns to scale, $\eta>1$. Increasing returns to scale in innovation seem reasonable, for example, synergy effects from teamwork. Creativity benefits from communication. Increasing returns can be perceived as positive spill-over effects within a sector. 


Substituting equation \ref{eq:scien}, yields a condition on the sector-specific spillover of innovation $\eta$ so that the ratio of workers across sectors is constant on the BGP.
It has to hold that
\begin{align}
\frac{L_{gt}}{L_{ft}}= \left(\frac{\gamma_{AF}}{\gamma_{AG}}\right)^{\eta-1}\left(\frac{1+\gamma_{AG}}{1+\gamma_{AF}}\right)^\phi \left(\frac{A_{ft-1}}{A_{gt-1}}\right)^{\phi(\eta-2)}\frac{\rho_g}{\rho_f}\frac{(1-\alpha_f)\alpha_f^{2\frac{\alpha_f}{1-\alpha_f}}}{(1-\alpha_g)\alpha_g^{2\frac{\alpha_g}{1-\alpha_g}}}\frac{p_{ft}^\frac{1}{1-\alpha_f}A_{ft}}{p_{gt}^\frac{1}{1-\alpha_g}A_{gt}}.
\end{align}

All terms are stable except for $\left(\frac{A_{ft-1}}{A_{gt-1}}\right)^{\phi(\eta-2)}$. Hence, for a balanced growth to have constant worker ratios across sectors, there have to be increasing returns to scientist within sectors and $\eta=2$.
 
Alternatively, one might want to abstain from stable employment ratios. However, then expenditure shares would not be constant, which again is consistent with structural change. When expenditure shares are not constant, then skill moves across sectors. 
However, my take on the BGP in this model is far in the future, after all transitions across sectors have taken place. Note, that I do not need to assume a BGP from 2050 onwards. 
 

\begin{comment}
%content...
\subsection{Equilibrium conditions: own model}

\begin{align*}
\text{\textbf{Household solved:}} \hspace{50mm}& \\
\text{FOCs labour supply}\hspace{4mm}&  %\log(H_t)=\frac{1}{1+\sigma}\log(1-\tau_{lt})\\
H_t=(1-\tau_{lt})^\frac{1}{1+\sigma}\\
\ \hspace{4mm} & %\log(w_{ht})=\log(w_{lt})+\log(\zeta)\\
w_{ht}=\zeta w_{lt}\\
\text{Budget}\hspace{4mm}&  %\log(c_t)= \log(\lambda_t)+ (1-\tau_{lt})\left[\frac{1}{1+\sigma}\log(1-\tau_{lt})+\log(w_{lt})\right]\\
c_t= \lambda_t (H_tw_{lt})^{(1-\tau_{lt})}\\
\text{definition}\  H_t\hspace{4mm} & %\log(H_t)=\log(h_{lt}+\zeta h_{ht})\\
H_t=\zeta h_{ht}+h_{lt}
\\
\text{\textbf{General Household Problem:}} \hspace{50mm}& \\
\text{FOC consumption}\hspace{4mm}& Mu_{ct}=p_t\mu_t\\
\text{FOC low skill}\hspace{4mm} & -Mu_{h_lt}=\mu_t \frac{\partial I_t}{\partial h_{lt}}\\
\text{FOC high skill}\hspace{4mm} & -Mu_{h_ht}=\mu_t \frac{\partial I_t}{\partial h_{ht}}\\
\text{Budget}\hspace{4mm}& c_tp_t= I_t\\
\text{definition}\  H_t\hspace{4mm} & H_t=\zeta h_{ht}+h_{lt}\\
\text{\textbf{Labour sectors:}}\hspace{50mm}&\\
\text{Production clean labour input} \hspace{4mm}& L_{ct}=l_{hct}^{\theta_c}l_{lct}^{1-\theta_c}\\ 
\text{Production dirty labour input} \hspace{4mm}& L_{dt}=l_{hdt}^{\theta_d}l_{ldt}^{1-\theta_d}\\
%
\text{Demand high skill clean sector}\hspace{4mm}&l_{hct}= \left(\frac{p_{cLt}}{w_{ht}}\right)^{\frac{1}{1-\theta_c}}\theta_c^{\frac{1}{1-\theta_c}}l_{lct}\\
%
\text{Demand low skill clean sector } \hspace{4mm}&l_{lct}= \left(\frac{p_{cLt}}{w_{lt}}\right)^{\frac{1}{\theta_c}}(1-\theta_c)^{\frac{1}{\theta_c}}l_{hct}\\
%
\text{Demand high skill dirty sector} \hspace{4mm}&l_{hdt}= \left(\frac{p_{dLt}}{w_{ht}}\right)^{\frac{1}{1-\theta_d}}\theta_d^{\frac{1}{1-\theta_d}}l_{ldt}\\
%
\text{Demand low skill dirty sector } \hspace{4mm}&l_{ldt}= \left(\frac{p_{dLt}}{w_{lt}}\right)^{\frac{1}{\theta_d}}(1-\theta_d)^{\frac{1}{\theta_d}}l_{hdt}\\
\text{\textbf{Government}}\hspace{50mm}& \nonumber\\
\text{Budget}\hspace{4mm}& G_t=H_tw_{lt}-\lambda_t(H_t w_{lt})^{(1-\tau_{lt})}
\\
\text{\textbf{Technology:}}\hspace{50mm}&\\
\text{Clean sector}\hspace{4mm}& A_{ict+1}=(1+\upsilon_{ct})A_{ict}\\
\text{Dirty sector}\hspace{4mm}& A_{idt+1}=(1+\upsilon_{dt})A_{idt}\\
%\text{Progress bound}\hspace{4mm}& \upsilon_{ct}+\upsilon_{dt}=\Upsilon\\
\text{Definition average clean technology}\hspace{4mm}& A_{ct}=\int_0^1A_{ict}di\\
\text{Definition average dirty technology}\hspace{4mm}& A_{dt}=\int_0^1A_{idt}di
\end{align*}

\begin{align}
\text{\textbf{Production:}} \hspace{4mm}
\text{\textbf{Final Good Producer}}&\\
\text{Profit maximisation}\hspace{4mm} & Y_{nt}=\left(\frac{p_{ct}}{p_{dt}}\right)^\varepsilon Y_{ct}\\
\text{Production}\hspace{4mm} & Y_t=\left[Y_{ct}^{\frac{\varepsilon-1}{\varepsilon}}+Y_{dt}^{\frac{\varepsilon-1}{\varepsilon}}\right]^{\frac{\varepsilon}{\varepsilon-1}}\\
\text{Price}\hspace{4mm}& p_t:=\left[p_{ct}^{1-\varepsilon}+p_{dt}^{1-\varepsilon}\right]^{\frac{1}{1-\varepsilon}}\\
\text{\textbf{Clean Sector}}\\
\text{Production}\hspace{4mm}& Y_{ct}=L^{1-\alpha}_{ct}\int_{0}^{1}A^{1-\alpha}_{ict}x_{ict}^{\alpha}di=  \left(\alpha\frac{p_{ct}}{\psi}\right)^{\frac{\alpha}{1-\alpha}}A_{ct} L_{ct} \label{eqbm:outputc}
\\ & =x_{ct}^{\alpha}\left(A_{ct}L_{ct}\right)^{1-\alpha} \\ 
\text{labour demand}\hspace{4mm} & p_{cLt} =
(1-\alpha)\left(\frac{\alpha}{\psi}\right)^\frac{\alpha}{1- \alpha}p_{ct}^\frac{1}{1-\alpha}A_{ct} \label{eqbm:labc} \\
\text{machine demand}\hspace{4mm} & x_{ict} = \left(\alpha\frac{ p_{ct}}{p_{ict}}\right)^\frac{1}{1-\alpha}A_{ict}L_{ct}\\
& x_{ct}:=\int_{0}^{1}x_{ict} di= \left(\alpha\frac{p_{ct}}{\psi}\right)^\frac{1}{1-\alpha}A_{ct}L_{ct}\\
%
\text{Supply machines (price)}\hspace{4mm}& p_{ict}=\psi \\
%
\text{\textbf{Dirty Sector}}\\
\text{Production}\hspace{4mm} & Y_{dt}=L^{1-\alpha}_{dt}\int_{0}^{1}A^{1-\alpha}_{idt}x_{idt}^{\alpha}di=  \left(\alpha\frac{p_{dt}}{\psi}\right)^{\frac{\alpha}{1-\alpha}}A_{dt} L_{dt}\label{eqbm:outputd}\\ & =x_{dt}^{\alpha}\left(A_{dt}L_{dt}\right)^{1-\alpha} \\ 
\text{labour demand}\hspace{4mm} & p_{dLt} =
(1-\alpha)\left(\frac{\alpha}{\psi}\right)^\frac{\alpha}{1- \alpha}p_{dt}^\frac{1}{1-\alpha}A_{dt}\label{eqbm:labd}\\
\text{machine demand}\hspace{4mm} & x_{idt} = \left(\alpha\frac{ p_{dt}}{p_{idt}}\right)^\frac{1}{1-\alpha}A_{idt}L_{dt}\\
& x_{dt}:=\int_{0}^{1}x_{idt} di= \left(\alpha\frac{p_{dt}}{\psi}\right)^\frac{1}{1-\alpha}A_{dt}L_{dt}\\
\text{Supply machines (price)}\hspace{4mm}& p_{idt}=\psi\\
\text{\textbf{Market clearing:}}\hspace{50mm}& \nonumber\\
\text{Final Good}\hspace{4mm}& Y_{t}=c_t+\psi\left(\int_{0}^1x_{idt}di+\int_{0}^1x_{ict}di\right)+G_t%\psi \left(\int_0^1\left(\alpha\frac{p_{dt}}{\psi}\right)^\frac{1}{1-\alpha}A_{idt}L_{dt}+\int_0^1\left(\alpha\frac{p_{ct}}{\psi}\right)^\frac{1}{1-\alpha}A_{ict}L_{ct}\right)
\\
%& \ (\text{Numeraire}\ \  p_t=1)\\
\text{high skill}\hspace{4mm}& l_{hct}+l_{hdt}=h_{ht}\\
\text{low skill}\hspace{4mm}&l_{lct}+l_{ldt}=h_{lt}
\end{align}
\end{comment}


\begin{comment}
\subsection{Equilibrium conditions: Simplified model with Lc=hh, ld=hl}

\begin{align*}
\text{\textbf{Household solved:}} \hspace{50mm}& \\
\text{FOCs labour supply}\hspace{4mm}&  %\log(H_t)=\frac{1}{1+\sigma}\log(1-\tau_{lt})\\
H_t=(1-\tau_{lt})^\frac{1}{1+\sigma}\\
\ \hspace{4mm} & %\log(w_{ht})=\log(w_{lt})+\log(\zeta)\\
w_{ht}=\zeta w_{lt}\\
\text{Budget}\hspace{4mm}&  %\log(c_t)= \log(\lambda_t)+ (1-\tau_{lt})\left[\frac{1}{1+\sigma}\log(1-\tau_{lt})+\log(w_{lt})\right]\\
c_t= \lambda_t (H_tw_{lt})^{(1-\tau_{lt})}\\
\text{definition}\  H_t\hspace{4mm} & %\log(H_t)=\log(h_{lt}+\zeta h_{ht})\\
H_t=\zeta h_{ht}+h_{lt}
\\
%\text{\textbf{Labour sectors:}}\hspace{50mm}&\\
%\text{Production clean labour input} \hspace{4mm}& L_{ct}=h_{ht}\\ 
%\text{Production dirty labour input} \hspace{4mm}& L_{dt}=h_{lt}\\
%
\text{\textbf{Government}}\hspace{50mm}& \nonumber\\
\text{Budget}\hspace{4mm}& G_t=H_tw_{lt}-\lambda_t(H_t w_{lt})^{(1-\tau_{lt})}
\\
\text{\textbf{Technology:}}\hspace{50mm}&\\
\text{Clean sector}\hspace{4mm}& A_{ict+1}=(1+\upsilon_{ct})A_{ict}\\
\text{Dirty sector}\hspace{4mm}& A_{idt+1}=(1+\upsilon_{dt})A_{idt}\\
%\text{Progress bound}\hspace{4mm}& \upsilon_{ct}+\upsilon_{dt}=\Upsilon\\
\text{Definition average clean technology}\hspace{4mm}& A_{ct}=\int_0^1A_{ict}di\\
\text{Definition average dirty technology}\hspace{4mm}& A_{dt}=\int_0^1A_{idt}di
\end{align*}

\begin{align*}
\text{\textbf{Production:}} \hspace{4mm}
\text{\textbf{Final Good Producer}}&\\
\text{Profit maximisation}\hspace{4mm} & Y_{nt}=\left(\frac{p_{ct}}{p_{dt}}\right)^\varepsilon Y_{ct}\\
\text{Production}\hspace{4mm} & Y_t=\left[Y_{ct}^{\frac{\varepsilon-1}{\varepsilon}}+Y_{dt}^{\frac{\varepsilon-1}{\varepsilon}}\right]^{\frac{\varepsilon}{\varepsilon-1}}\\
\text{Price}\hspace{4mm}& p_t:=\left[p_{ct}^{1-\varepsilon}+p_{dt}^{1-\varepsilon}\right]^{\frac{1}{1-\varepsilon}}\\
\text{\textbf{Clean Sector}}\\
\text{Production}\hspace{4mm}& Y_{ct}=L^{1-\alpha}_{ct}\int_{0}^{1}A^{1-\alpha}_{ict}x_{ict}^{\alpha}di=  \left(\alpha\frac{p_{ct}}{\psi}\right)^{\frac{\alpha}{1-\alpha}}A_{ct} L_{ct}
\\ & =x_{ct}^{\alpha}\left(A_{ct}L_{ct}\right)^{1-\alpha} \\ 
\text{labour demand}\hspace{4mm} & w_{ht} =
(1-\alpha)\left(\frac{\alpha}{\psi}\right)^\frac{\alpha}{1- \alpha}p_{ct}^\frac{1}{1-\alpha}A_{ct}\\
\text{machine demand}\hspace{4mm} & x_{ict} = \left(\alpha\frac{ p_{ct}}{p_{ict}}\right)^\frac{1}{1-\alpha}A_{ict}L_{ct}\\
& x_{ct}:=\int_{0}^{1}x_{ict} di= \left(\alpha\frac{p_{ct}}{\psi}\right)^\frac{1}{1-\alpha}A_{ct}L_{ct}\\
%
\text{Supply machines (price)}\hspace{4mm}& p_{ict}=\psi \\
%
\text{\textbf{Dirty Sector}}\\
\text{Production}\hspace{4mm} & Y_{dt}=L^{1-\alpha}_{dt}\int_{0}^{1}A^{1-\alpha}_{idt}x_{idt}^{\alpha}di=  \left(\alpha\frac{p_{dt}}{\psi}\right)^{\frac{\alpha}{1-\alpha}}A_{dt} L_{dt}\\ & =x_{dt}^{\alpha}\left(A_{dt}L_{dt}\right)^{1-\alpha} \\ 
\text{labour demand}\hspace{4mm} & w_{lt} =
(1-\alpha)\left(\frac{\alpha}{\psi}\right)^\frac{\alpha}{1- \alpha}p_{dt}^\frac{1}{1-\alpha}A_{dt}\\
\text{machine demand}\hspace{4mm} & x_{idt} = \left(\alpha\frac{ p_{dt}}{p_{idt}}\right)^\frac{1}{1-\alpha}A_{idt}L_{dt}\\
& x_{dt}:=\int_{0}^{1}x_{idt} di= \left(\alpha\frac{p_{dt}}{\psi}\right)^\frac{1}{1-\alpha}A_{dt}L_{dt}\\
\text{Supply machines (price)}\hspace{4mm}& p_{idt}=\psi\\
\text{\textbf{Market clearing:}}\hspace{50mm}& \nonumber\\
\text{Final Good}\hspace{4mm}& Y_{t}=c_t+\psi\left(\int_{0}^1x_{idt}di+\int_{0}^1x_{ict}di\right)+G_t%\psi \left(\int_0^1\left(\alpha\frac{p_{dt}}{\psi}\right)^\frac{1}{1-\alpha}A_{idt}L_{dt}+\int_0^1\left(\alpha\frac{p_{ct}}{\psi}\right)^\frac{1}{1-\alpha}A_{ict}L_{ct}\right)
\\
%& \ (\text{Numeraire}\ \  p_t=1)\\
\text{high skill}\hspace{4mm}& L_{ct}=h_{ht}\\
\text{low skill}\hspace{4mm}&L_{dt}=h_{lt}
\end{align*}

\section{Solution of tractable model}\label{app:solu}
Define
\begin{align*}
	\tilde{\kappa}:=\ &\frac{(1-\theta_c)(1-\theta_d)\left[\left(\frac{A_c}{A_d}\right)^{(1-\alpha)(1-\varepsilon)}\zeta^{-(\theta_c-\theta_d)(1-\alpha)(1-\varepsilon)}\tilde{\chi}+1\right]}{(1-\theta_d)+(1-\theta_c)\left[\left(\frac{A_c}{A_d}\right)^{(1-\alpha)(1-\varepsilon)}\zeta^{-(\theta_c-\theta_d)(1-\alpha)(1-\varepsilon)}\tilde{\chi}\right]}\\
	\gamma_j:=\ & \left(\frac{\theta_j}{\zeta(1-\theta_j)}\right)^{\theta_j}\\
	z_j:=\ &\theta_j^{\theta_j}(1-\theta_j)^{1-\theta_j} \\
	\chi:=\ &% \frac{(1-\theta_d)(1-\theta_c)}{\theta_c(1-\theta_d)-\theta_d(1-\theta_c)}
	\frac{(1-\theta_d)(1-\theta_c)}{\theta_c-\theta_d}\\
	\tilde{\chi}: =\ &  (\theta_c^{\theta_c}\theta_d^{-\theta_d})^{(1-\alpha) (1-\varepsilon)}(1-\theta_c)^{-\theta_c-(1-\theta_c)(\alpha+\varepsilon(1-\alpha))}(1-\theta_d)^{\theta_d+(1-\theta_d)(\alpha+\varepsilon(1-\alpha))}
\end{align*}
From profit maximisation by labour input good producers follows that the price of the labour input good relative to the skill-specific wage rate is constant. Substituting demand for low skill in the clean sector into the demand for high skill yields

\begin{align*}
	w_{h}^{\frac{1}{1-\theta_c}}w_l^{\frac{1}{\theta_c}}= p_{cL}^\frac{1}{(1-\theta_c)\theta_c}\theta_c^\frac{1}{1-\theta_c}(1-\theta_c)^\frac{1}{\theta_c}.
\end{align*}
Multiplying the left-hand side with $(w_h/w_h)^\frac{1}{\theta_c}$ and
using the FOC governing skill supply $w_h/w_l=\zeta$, it holds that

\begin{align}\label{eq:constant}
%	& \zeta^\frac{-1}{\theta_c}w_h^\frac{1}{(1-\theta_c)\theta_c}= p_{cL}^\frac{1}{(1-\theta_c)\theta_c}\theta_c^\frac{1}{1-\theta_c}(1-\theta_c)^\frac{1}{\theta_c}\nonumber\\
%	\Leftrightarrow\ 
& \frac{p_{cL}}{w_h}= \frac{\zeta^{-(1-\theta_c)}}{z_c}.
\end{align}
%\noindent \tr{Note: this result does not rely on the claim that the labour input good is constant.}

Analogously to \ref{eq:constant}, it follows that
\begin{align}
	\frac{p_{cL}}{w_l}&=\frac{\zeta^{\theta_c}}{z_c}\label{eq:pcl_wl}\\
	\frac{p_{dL}}{w_l}&=\frac{\zeta^{\theta_d}}{z_d}%\ \Leftrightarrow\ w_l= p_{dL}\zeta^{-\theta_d}\theta_d^{\theta_d}(1-\theta_d)^{1-\theta_d}
	\label{eq:pdl_wl}\\
	\frac{p_{dL}}{w_h}&=\frac{\zeta^{-(1-\theta_d)}}{z_d}.
\end{align}
Therefore, the optimal skill input ratios in the labour good production are given by
\begin{align}\label{eq:inputr}
	\frac{l_{hc}}{l_{lc}}=\frac{\theta_c}{\zeta (1-\theta_c)} \hspace{2mm} \text{and}\hspace{3mm} \frac{l_{hd}}{l_{ld}}=\frac{\theta_d}{\zeta (1-\theta_d)}.
\end{align}
This is the common result that  factor shares 
% this refers to (wh lhc)/(wl llc)
are constant over time with a Cobb-Douglas production function. 
Imposing labour market clearing for both skills and optimal skill demand yields 
\begin{align}
	&l_{ld}=\chi\left(\frac{1}{1-\theta_c}h_l-H\right)\label{eq:lld}\\ %\frac{\theta_c}{1-\theta_c}\chi h_l-\chi \zeta h_h,\\
	& l_{lc}=\chi \left(H-\frac{1}{1-\theta_d}h_l\right)\label{eq:llc} %\\
%	with \ & \chi:= \frac{(1-\theta_d)(1-\theta_c)}{\theta_c(1-\theta_d)-\theta_d(1-\theta_c)}=\frac{(1-\theta_d)(1-\theta_c)}{\theta_c-\theta_d}.
	%& l_{hc}= \frac{\theta_c}{\zeta (1-\theta_c)}l_{lc}\\
	%& l_{hd}=\frac{\theta_d}{\zeta (1-\theta_d)}l_{ld}
\end{align}
Labour good supply follows from the labour input good's production function and optimal skill inputs, equations \ref{eq:inputr}, as
\begin{align}
	L_c&=\gamma_cl_{lc}\label{eq:lab_inputc} \\
	L_d&=\gamma_dl_{ld}.\label{eq:lab_inputd}
\end{align}
%\tr{Note that now policy can affect inflation/ relative prices through changes in labour supply---NOPE: cancels}
A relation of the relative price in equilibrium results from equating demand for the labour input goods, equations \ref{eqbm:labc} and \ref{eqbm:labd}, % (which relates the price for the labour input good and the price for the sector-specific final good), 
demand for low skill input by labour producers, equations \ref{eq:pcl_wl} and \ref{eq:pdl_wl}, and free movement of skills: 
\begin{align}\label{eq:price_ratio_labourinput}
	\frac{p_c}{p_d}= \left(\frac{A_d}{A_c}\frac{z_d}{z_c}\zeta^{\theta_c-\theta_d}\right)^{1-\alpha}& \text{(optimality labour input production)}
\end{align}
%where
%\begin{align*}
%	z_j=\theta_j^{\theta_j}(1-\theta_j)^{1-\theta_j}
%\end{align*}
Together with the definition of the aggregate price level and the choice of $Y$ as numeraire, equation \ref{eq:price_ratio_labourinput} determines sector-specific prices as a function of parameters and productivity:
%\begin{align*}
%	p_c= \left(1+\left(\frac{\gamma_c}{\gamma_d}\frac{A_c}{A_d}\frac{l_{lc}}{l_{ld}}\right)^{\frac{(1-\alpha)(1-\varepsilon)}{\alpha+\varepsilon(1-\alpha)}}\right)^{-\frac{1}{1-\varepsilon}}.
%\end{align*}
%Substituting equation \ref{eq:lldllc} gives the price of the clean good in equilibrium as
\begin{align}
	p_c%& = \frac{1}{\left(1+\left(\frac{A_c}{A_d}\right)^{(1-\alpha)(1-\varepsilon)}\left(\frac{z_c}{z_d}\right)^{(1-\alpha)(1-\varepsilon)}\zeta^{-(\theta_c-\theta_d)(1-\alpha)(1-\varepsilon)}\right)^{\frac{1}{1-\varepsilon}}}\\
	&= \left(\frac{\left(A_dz_d\zeta^{\theta_c}\right)^{(1-\alpha)(1-\varepsilon)}}{\left(A_dz_d\zeta^{\theta_c}\right)^{(1-\alpha)(1-\varepsilon)}+\left(A_cz_c\zeta^{\theta_d}\right)^{(1-\alpha)(1-\varepsilon)}}\right)^{\frac{1}{1-\varepsilon}}\label{eq:eq_pc}\\
%\end{align}
%%and using equation \ref{eq:price_ratio_labourinput} yields
%\begin{align}
	p_d%& =\frac{1}{\left(\left(\frac{A_d}{A_c}\right)^{(1-\alpha)(1-\varepsilon)}\left(\frac{z_d}{z_c}\right)^{(1-\alpha)(1-\varepsilon)}\zeta^{(\theta_c-\theta_d)(1-\alpha)(1-\varepsilon)}+1\right)^{\frac{1}{1-\varepsilon}}}\\
	&= \left(\frac{\left(A_cz_c\zeta^{\theta_d}\right)^{(1-\alpha)(1-\varepsilon)}}{\left(A_dz_d\zeta^{\theta_c}\right)^{(1-\alpha)(1-\varepsilon)}+\left(A_cz_c\zeta^{\theta_d}\right)^{(1-\alpha)(1-\varepsilon)}}\right)^{\frac{1}{1-\varepsilon}} \label{eq:eq_pd}
\end{align}

\paragraph{Skill allocation}
To solve for the equilibrium ratio of skill inputs in the clean and dirty sector, I substitute labour input, equations \ref{eq:lab_inputc} and \ref{eq:lab_inputd}, in the sector-specific production functions, \ref{eqbm:outputc} and \ref{eqbm:outputd}. Exploiting demand for sector goods, $Y_d=\left(\frac{p_c}{p_d}\right)^\varepsilon Y_c$, and the price ratio in equilibrium pinned down by equation \ref{eq:price_ratio_labourinput} yields
%\begin{align}\label{eq:price_ratio_output}

%\end{align}
%Substituting equation \ref{eq:price_ratio_output} into equation \ref{eq:price_ratio_labourinput} determines the equilibrium ratio of low-skill input in the dirty to the clean sector: 
\begin{align}
	\frac{p_c}{p_d} =&\left(\frac{\gamma_d}{\gamma_c}\frac{A_d}{A_c}\frac{l_{ld}}{l_{lc}}\right)^{\frac{1-\alpha}{\alpha+\varepsilon(1-\alpha)}}\\ %& \text{(Demand for sector-specific goods)}\\
\Leftrightarrow\ 	\frac{l_{ld}}{l_{lc}}=&%\left(\frac{A_c}{A_d}\right)^{(1-\alpha)(1-\varepsilon)}\frac{\gamma_c}{\gamma_d}\left(\frac{z_d}{z_c}\right)^{\alpha+\varepsilon(1-\alpha)}\zeta^{(\theta_c-\theta_d)(\alpha+\varepsilon(1-\alpha))}\nonumber\\	=&
	\left(\frac{A_c}{A_d}\right)^{(1-\alpha)(1-\varepsilon)}\left(\zeta^{\theta_c-\theta_d}\frac{\gamma_c}{\gamma_d}\frac{z_d}{z_c}\right)^{\alpha+\varepsilon(1-\alpha)}\label{eq:lldllc}%\\
	%	\text{where}&\\
	%	\tilde{\chi}= &\  (\theta_c^{\theta_c}\theta_d^{-\theta_d})^{(1-\alpha) (1-\varepsilon)}(1-\theta_c)^{-\theta_c-(1-\theta_c)(\alpha+\varepsilon(1-\alpha))}(1-\theta_d)^{\theta_d+(1-\theta_d)(\alpha+\varepsilon(1-\alpha))}\nonumber
\end{align}

\paragraph{Skill supply}
Using equations \ref{eq:lld}, \ref{eq:llc}, and \ref{eq:lldllc} one can solve for $h_l$ as a function of total skill supply in equilibrium
\begin{align}
	h_l= \underbrace{\frac{(1-\theta_c)(1-\theta_d)\left[\left(\frac{A_c}{A_d}\right)^{(1-\alpha)(1-\varepsilon)}\zeta^{-(\theta_c-\theta_d)(1-\alpha)(1-\varepsilon)}\tilde{\chi}+1\right]}{(1-\theta_d)+(1-\theta_c)\left[\left(\frac{A_c}{A_d}\right)^{(1-\alpha)(1-\varepsilon)}\zeta^{-(\theta_c-\theta_d)(1-\alpha)(1-\varepsilon)}\tilde{\chi}\right]}}_{:=\tilde{\kappa}}H
\end{align}
Now, one can solve for labour input and sector-specific output as a function of tax progessivity  in equilibrium. 
$L_c$ and $L_d$ are
\begin{align}
	L_c&= \gamma_c \chi \left(1-\frac{\tilde{\kappa}}{1-\theta_d}\right)H\\
	L_d&= \gamma_d \chi \left(\frac{\tilde{\kappa}}{1-\theta_c}-1\right)H=\zeta^{-\theta_d}z_dp_d^{1-\varepsilon}H
\end{align}
This solves the model, since, in equilibrium,  $H$ is a function of parameters and policy variables only. 
Replacing dirty labour input and machines in dirty production leads to the expression for dirty output growth used in the text. 
\begin{comment}
\paragraph{Summary of equilibrium equations}
\begin{align*}
H=\ & (1-\tau_l)^{\frac{1}{1+\sigma}}\\
h_l=\ & \tilde{\kappa}H\\
L_c=\ & \gamma_c \chi \left(1-\frac{\tilde{\kappa}}{1-\theta_d}\right)H
\\
L_d=\ & \gamma_d \chi \left(\frac{\tilde{\kappa}}{1-\theta_c}-1\right)H%= \zeta^{-\theta_d}z_d\frac{\left(A_cz_c\zeta^{\theta_d}\right)^{(1-\alpha)(1-\varepsilon)}}{\left(A_dz_d\zeta^{\theta_c}\right)^{(1-\alpha)(1-\varepsilon)}+\left(A_cz_c\zeta^{\theta_d}\right)^{(1-\alpha)(1-\varepsilon)}}H
=\zeta^{-\theta_d}z_dp_d^{1-\varepsilon}H\\
p_d=\ &\left(\frac{\left(A_cz_c\zeta^{\theta_d}\right)^{(1-\alpha)(1-\varepsilon)}}{\left(A_dz_d\zeta^{\theta_c}\right)^{(1-\alpha)(1-\varepsilon)}+\left(A_cz_c\zeta^{\theta_d}\right)^{(1-\alpha)(1-\varepsilon)}}\right)^{\frac{1}{1-\varepsilon}} \\
p_c=\ & \left(\frac{\left(A_dz_d\zeta^{\theta_c}\right)^{(1-\alpha)(1-\varepsilon)}}{\left(A_dz_d\zeta^{\theta_c}\right)^{(1-\alpha)(1-\varepsilon)}+\left(A_cz_c\zeta^{\theta_d}\right)^{(1-\alpha)(1-\varepsilon)}}\right)^{\frac{1}{1-\varepsilon}}
\end{align*}

content...
\end{comment}
\section{Balanced Growth Path}

The model features structural transformation stemming from price effects (\cite{Ngai2007StructuralGrowth}, Baumol (1967)), since heterogeneous growth rates result in relative price changes over time. %A shown by \cite{Ngai2007StructuralGrowth}, the model features a balanced-growth path with certain parameter values: 
For certain parameter values the model exhibits a generalised balanced growth path\footnote{\ 
In contrast to a balanced growth path, which is commonly defined by constant growth in all variables, a GBGP is less strict and certain variables are allowed to grow at non-constant rates. The literature on structural transformation commonly reverts to this concept as transitions across sectors are essential to this literature.}.
\cite{Ngai2007StructuralGrowth} show that with goods being complements, employment shares shift to sectors with lower TFP growth; eventually, all labour is in the sector with the lowest TFP. In the present model, this is the clean sector. 

\section{Model Isomorphic to model with investment and rented capital}
The model is isomorphic to a model with (instantaneously productive) investment and full depreciation: 
\begin{align*}
I_t&=\psi(x_{dt}+x_{ct})\\
(LOM capital) \ K_t&=I_t= I_{ct}+I_{dt}
\end{align*}
That is, the capital good is produced by the following technology
\begin{align*}
x_{ijt}=\frac{I_{ijt}}{\psi}
\end{align*}
Machine producing firms rent the investment good, $I_{jt}$ and pay the real rate. They maximise over the choice of investment, i.e. capital, to borrow:
\begin{align*}
\underset{I_{ijt}}{\max}\hspace{2mm}p_{ijt}x_{ijt}-r_tI_{ijt}
\end{align*}
Profit maximisation of machine producing firms yields
\begin{align*}
\frac{p_{ijt}}{\psi}=r_t
\end{align*}
Free movement of capital and homogeneity of production costs imply that machine prices are equal across firms and sectors. 

Imposing market clearing for investment, $I_t=\int_{0}^{1}I_{idt}di+\int_{0}^{1}I_{ict}di$, and market clearing for machines yield a condition for the real rate in equilibrium
\begin{align*}
r_t=\alpha \psi^{-\alpha}\left(\frac{p_{dt}^{\frac{1}{1-\alpha}}A_{dt}L_{dt}+p_{ct}^{\frac{1}{1-\alpha}}A_{ct}L_{ct}}{K_t}\right)^{1-\alpha}
\end{align*}

\section{Results}
\begin{figure}[h!!]
	\centering
	\caption{Business as usual versus laissez-faire, substitutes, additional variables }\label{fig:onlyBAU_add}
	
	\begin{minipage}[]{0.32\textwidth}
		\centering{\footnotesize{(a) Clean output, $y_c$ }}
		%	\captionsetup{width=.45\linewidth}
		\includegraphics[width=1\textwidth]{../../codding_model/Own/figures/Rep_agent/staticBAU_LF_separate_yc_periods59_eppsilon4.00_zeta1.40_Ad08_Ac04_thetac0.70_thetad0.56_HetGrowth1_tauul0.181_util0_withtarget0_lgd0.png}
	\end{minipage}
	\begin{minipage}[]{0.32\textwidth}
		\centering{\footnotesize{(b) Dirty output, $y_d$}}
		%	\captionsetup{width=.45\linewidth}
		\includegraphics[width=1\textwidth]{../../codding_model/Own/figures/Rep_agent/staticBAU_LF_separate_yd_periods59_eppsilon4.00_zeta1.40_Ad08_Ac04_thetac0.70_thetad0.56_HetGrowth1_tauul0.181_util0_withtarget0_lgd0.png}
	\end{minipage}
	\begin{minipage}[]{0.32\textwidth}
		\centering{\footnotesize{(c) Labour input clean, $L_c$ }}
		%	\captionsetup{width=.45\linewidth}
		\includegraphics[width=1\textwidth]{../../codding_model/Own/figures/Rep_agent/staticBAU_LF_separate_Lc_periods59_eppsilon4.00_zeta1.40_Ad08_Ac04_thetac0.70_thetad0.56_HetGrowth1_tauul0.181_util0_withtarget0_lgd0.png}
	\end{minipage}
	\begin{minipage}[]{0.32\textwidth}
		\centering{\footnotesize{(d) Labour input dirty, $L_d$ }}
		%	\captionsetup{width=.45\linewidth}
		\includegraphics[width=1\textwidth]{../../codding_model/Own/figures/Rep_agent/staticBAU_LF_separate_Ld_periods59_eppsilon4.00_zeta1.40_Ad08_Ac04_thetac0.70_thetad0.56_HetGrowth1_tauul0.181_util0_withtarget0_lgd0.png}
	\end{minipage}
\begin{minipage}[]{0.32\textwidth}
	\centering{\footnotesize{(e) Machines clean, $x_c$}}
	%	\captionsetup{width=.45\linewidth}
	\includegraphics[width=1\textwidth]{../../codding_model/Own/figures/Rep_agent/staticBAU_LF_separate_xc_periods59_eppsilon4.00_zeta1.40_Ad08_Ac04_thetac0.70_thetad0.56_HetGrowth1_tauul0.181_util0_withtarget0_lgd0.png}
\end{minipage}
	\begin{minipage}[]{0.32\textwidth}
		\centering{\footnotesize{(f) Machines dirty, $x_d$}}
		%	\captionsetup{width=.45\linewidth}
		\includegraphics[width=1\textwidth]{../../codding_model/Own/figures/Rep_agent/staticBAU_LF_separate_xd_periods59_eppsilon4.00_zeta1.40_Ad08_Ac04_thetac0.70_thetad0.56_HetGrowth1_tauul0.181_util0_withtarget0_lgd0.png}
	\end{minipage}
\end{figure}

\begin{figure}[h!!]
	\centering
	\caption{Business as usual versus laissez-faire, complements, additional variables }\label{fig:onlyBAU_comp_add}
		\begin{minipage}[]{0.32\textwidth}
		\centering{\footnotesize{(a) Clean output }}
		%	\captionsetup{width=.45\linewidth}
		\includegraphics[width=1\textwidth]{../../codding_model/Own/figures/Rep_agent/staticBAU_LF_separate_yc_periods59_eppsilon0.40_zeta1.40_Ad08_Ac04_thetac0.70_thetad0.56_HetGrowth1_tauul0.181_util0_withtarget0_lgd0.png}
	\end{minipage}
	\begin{minipage}[]{0.32\textwidth}
		\centering{\footnotesize{(b) Dirty output }}
		%	\captionsetup{width=.45\linewidth}
		\includegraphics[width=1\textwidth]{../../codding_model/Own/figures/Rep_agent/staticBAU_LF_separate_yd_periods59_eppsilon0.40_zeta1.40_Ad08_Ac04_thetac0.70_thetad0.56_HetGrowth1_tauul0.181_util0_withtarget0_lgd0.png}
	\end{minipage}
	\begin{minipage}[]{0.32\textwidth}
		\centering{\footnotesize{(c) Labour input clean, $L_c$ }}
		%	\captionsetup{width=.45\linewidth}
		\includegraphics[width=1\textwidth]{../../codding_model/Own/figures/Rep_agent/staticBAU_LF_separate_Lc_periods59_eppsilon0.40_zeta1.40_Ad08_Ac04_thetac0.70_thetad0.56_HetGrowth1_tauul0.181_util0_withtarget0_lgd0.png}
	\end{minipage}
	\begin{minipage}[]{0.32\textwidth}
		\centering{\footnotesize{(d) Labour input dirty, $L_d$ }}
		%	\captionsetup{width=.45\linewidth}
		\includegraphics[width=1\textwidth]{../../codding_model/Own/figures/Rep_agent/staticBAU_LF_separate_Ld_periods59_eppsilon0.40_zeta1.40_Ad08_Ac04_thetac0.70_thetad0.56_HetGrowth1_tauul0.181_util0_withtarget0_lgd0.png}
	\end{minipage}
	\begin{minipage}[]{0.32\textwidth}
		\centering{\footnotesize{(e) Machines clean, $x_c$}}
		%	\captionsetup{width=.45\linewidth}
		\includegraphics[width=1\textwidth]{../../codding_model/Own/figures/Rep_agent/staticBAU_LF_separate_xc_periods59_eppsilon0.40_zeta1.40_Ad08_Ac04_thetac0.70_thetad0.56_HetGrowth1_tauul0.181_util0_withtarget0_lgd0.png}
	\end{minipage}
	\begin{minipage}[]{0.32\textwidth}
		\centering{\footnotesize{(f) Machines dirty, $x_d$}}
		%	\captionsetup{width=.45\linewidth}
		\includegraphics[width=1\textwidth]{../../codding_model/Own/figures/Rep_agent/staticBAU_LF_separate_xd_periods59_eppsilon0.40_zeta1.40_Ad08_Ac04_thetac0.70_thetad0.56_HetGrowth1_tauul0.181_util0_withtarget0_lgd0.png}
	\end{minipage}
\end{figure}

\begin{figure}[h!!]
	\centering
	\caption{Optimal allocation with emission target, complements, additional variables }\label{fig:optallo_comp_onlyR_add}
	\begin{minipage}[]{0.32\textwidth}
	\centering{\footnotesize{(a) Labour input clean, $L_c$ }}
	%	\captionsetup{width=.45\linewidth}
	\includegraphics[width=1\textwidth]{../../codding_model/Own/figures/Rep_agent/staticonlyRam_separate_Lc_periods59_eppsilon0.40_zeta1.40_Ad08_Ac04_thetac0.70_thetad0.56_HetGrowth1_tauul0.181_util0_withtarget1_lgd0.png}
\end{minipage}
	\begin{minipage}[]{0.32\textwidth}
		\centering{\footnotesize{(b) Labour input dirty, $L_d$ }}
		%	\captionsetup{width=.45\linewidth}
		\includegraphics[width=1\textwidth]{../../codding_model/Own/figures/Rep_agent/staticonlyRam_separate_Ld_periods59_eppsilon0.40_zeta1.40_Ad08_Ac04_thetac0.70_thetad0.56_HetGrowth1_tauul0.181_util0_withtarget1_lgd0.png}
	\end{minipage}
	\begin{minipage}[]{0.32\textwidth}
		\centering{\footnotesize{(c) Machines clean, $x_c$}}
		%	\captionsetup{width=.45\linewidth}
		\includegraphics[width=1\textwidth]{../../codding_model/Own/figures/Rep_agent/staticonlyRam_separate_xc_periods59_eppsilon0.40_zeta1.40_Ad08_Ac04_thetac0.70_thetad0.56_HetGrowth1_tauul0.181_util0_withtarget1_lgd0.png}
	\end{minipage}
\begin{minipage}[]{0.32\textwidth}
\centering{\footnotesize{(d) Machines dirty, $x_d$}}
%	\captionsetup{width=.45\linewidth}
\includegraphics[width=1\textwidth]{../../codding_model/Own/figures/Rep_agent/staticonlyRam_separate_xd_periods59_eppsilon0.40_zeta1.40_Ad08_Ac04_thetac0.70_thetad0.56_HetGrowth1_tauul0.181_util0_withtarget1_lgd0.png}
\end{minipage}
\begin{minipage}[]{0.32\textwidth}
	\centering{\footnotesize{(e) Price clean good, $p_c$}}
	%	\captionsetup{width=.45\linewidth}
	\includegraphics[width=1\textwidth]{../../codding_model/Own/figures/Rep_agent/staticonlyRam_separate_pc_periods59_eppsilon0.40_zeta1.40_Ad08_Ac04_thetac0.70_thetad0.56_HetGrowth1_tauul0.181_util0_withtarget1_lgd0.png}
\end{minipage}
	\begin{minipage}[]{0.32\textwidth}
		\centering{\footnotesize{(f) Price dirty good, $p_d$}}
		%	\captionsetup{width=.45\linewidth}
		\includegraphics[width=1\textwidth]{../../codding_model/Own/figures/Rep_agent/staticonlyRam_separate_pd_periods59_eppsilon0.40_zeta1.40_Ad08_Ac04_thetac0.70_thetad0.56_HetGrowth1_tauul0.181_util0_withtarget1_lgd0.png}
	\end{minipage}
	\begin{minipage}[]{0.32\textwidth}
	\centering{\footnotesize{(g) $\lambda$}}
	%	\captionsetup{width=.45\linewidth}
	\includegraphics[width=1\textwidth]{../../codding_model/Own/figures/Rep_agent/staticonlyRam_separate_lambdaa_periods59_eppsilon0.40_zeta1.40_Ad08_Ac04_thetac0.70_thetad0.56_HetGrowth1_tauul0.181_util0_withtarget1_lgd0.png}
\end{minipage}
\end{figure}



\section{Skill supply}
\paragraph{Effect of $\tau_l$ on skill investment}
From the definition of $H$ it has to hold that 
\begin{align}
&1=\frac{dh_l}{dH}+\zeta \frac{dh_h}{dH}\label{eq:ident} \\
\Leftrightarrow\ & \frac{dh_h}{dH}=\frac{1-\frac{dh_l}{dH}}{\zeta}.\label{eq:resp}
\end{align}
Using this equation, one can show that high skill supply is relatively more responsive to changes in total effective hours worked, i.e.,  $\frac{dh_h}{dH}>\frac{dh_l}{dH}$, if one excludes the case that high skill supply reduces as effective hours increase.\footnote{\ Proof: Suppose   $\frac{dh_h}{dH}>0$. Now, assume by contradiction that low skill supply is relatively more responsive. Hence, $\frac{dh_h}{dH}<\frac{dh_l}{dH}$. Using equation \ref{eq:resp}, one gets that $\frac{dh_l}{dH}>1+\zeta$. Replacing this inequality in the identity \ref{eq:ident}, it follows that $0>\zeta[1+\frac{dh_h}{dH}]$. Since $\zeta>1$ by assumption, it has to hold that $\frac{dh_h}{dH}<-1$ which contradicts the premise that $\frac{dh_h}{dH}>0$. } Thus, as the household reduces total effective hours supplied, the reduction in high skilled hours is higher. \tr{This should be due to the marginal utility from less high skill is higher than from less low skill hours.} This should show up in general equilibrium effects... but relative wages are fixed. 


\section{Calculations, partially wrong}
\noindent\rule[1ex]{\textwidth}{1pt}

\paragraph{Progressivity and emission targets}\tr{This result also rests on wrong premisses, but needs to be replicated!}
The constraint on emissions in the government's objective function implies that $Y_{dt}=\frac{\delta}{\kappa}$, thus, $(1+g_{ydt})=1$, for all time periods starting from 2050, $t\geq 30$. 

From the dirty sector's production function and equation \ref{eq:inf_d} we have that
\begin{align}
&\frac{Y_{d}'}{Y_d}=(1+\pi_d)^{\frac{\alpha}{1-\alpha}}(1+\upsilon_{d})\label{eq:gyd}\\
\Leftrightarrow\ &(1+\upsilon_{d})^{\frac{(1-\alpha)(1-\tau_l-\varepsilon)}{(1-\tau_l)-(\varepsilon(1-\alpha)+\alpha)}}=1\label{eq:def_taul}
\end{align}
The inflation rate in equation \ref{eq:gyd} captures the role of machine demand by the dirty sector. When the price at which dirty firms can sell their output is high, they demand more machines. A positive inflation, therefore, implies a rise in dirty output.

At the same time, a rise in the dirty good's price reduces demand. This counteracting mechanism is accounted for in equation \ref{eq:def_taul}. 
As will be shown below, this mechanism ensures that the government can target dirty sector production through tax progressivity. 

First, I establish an optimal policy result. Assume that the government cannot set the growth rate in the dirty sector, then equation \ref{eq:def_taul} defines $\tau_l$ on a balanced growth path.

\begin{prop}[Optimal tax progressivity]
	Assume growth of the dirty technology, $\upsilon_{d}$, is exogenously determined. 
	Then, to comply with the Paris Agreement, the government has to set the tax progressivity parameter, $\tau_l$, to $\tau^*_l=1-\varepsilon$ for $\varepsilon\neq 1$ (as otherwise the exponent in \ref{eq:def_taul} is not defined under the optimal tax rate.).
	When goods are complements, the optimal tax system is progressive. If goods are substitutes, the optimal tax system is regressive.
\end{prop}

The intuition is, that by choosing tax progressivity, the government affects price inflation in the dirty sector; compare equation \ref{eq:inf_d}. 
The result implies that inflation in the dirty sector under the optimal policy is negative when there is positive growth in dirty technology. The demand for machines has to decline by the same rate as technology growths for dirty output to be constant.  

Can this be an equilibrium as the price for dirty products declines?

How does tax progressivity affect inflation? 
First note that at a flat tax, the inflation rate is independent of the sector-specific technological growth rate. This is due to offsetting mechanisms.\tr{Continue}


\tr{Start from effect on HH:} 
(1) A rise in $\tau_l$ reduces disposable income and aggregate demand falls. This is a mechanical result from a higher tax rate, and a reduction in aggregate hours supplied.  
(1) For the dirty sector to demand labour, the costs of the labour input good has to balance its marginal product which positively depends on technological progress and the sector specific price. 

\noindent\rule[1ex]{\textwidth}{1pt} 


\noindent\rule[1ex]{\textwidth}{1pt}

\textcolor{blue}{below is wrong since the aggregate price level is not constant when G is disposed off.}
Define sector-specific inflation as: $1+\pi_{j}=\frac{p'_j}{p_j}$.
Using the definition of the aggregate price level, final good production, and optimality conditions in the clean sector, one can show that 
\begin{align}\label{eq:agg_supply}
\frac{Y'}{Y}= (1+\pi_c)^{\frac{\varepsilon(1-\alpha)+\alpha}{1-\alpha}}(1+\upsilon_{c}), \hspace{3mm} \text{(Supply side)}
\end{align}
since $\frac{L_{c}'}{L_c}=1$.
Using goods market clearance, the budget condition, and the FOC for total skill supply, it follows that 
\begin{align}\label{eq:agg_demand}
\frac{Y'}{Y}= \left(\frac{w'_h}{w_h}\right)^{1-\tau_l}. \hspace{3mm} \text{(Demand side)}\tr{\text{wrong, misses gov expenditures... or let lambdaa adjust, and machine production}}
\end{align}
Demand for the labour input good implies that 
\begin{align}\label{eq:labour income}
\frac{p'_{cL}}{p_{cL}}= (1+\pi_c)^\frac{1}{1-\alpha}(1+\upsilon_{c})
\end{align}
(independent of growth in $L_c$).

Multiplying both sides with $\left(\frac{w_h'}{w_h}\right)^{-1}$, using equation \ref{eq:agg_growth}, and that $\frac{p_{cL}}{w_h}$ is constant, it follows that 

\begin{align}
\frac{\frac{p'_{cL}}{w'_h}}{\frac{p_{cL}}{w_h}}= (1+\pi_c)^\frac{1}{1-\alpha}(1+\upsilon_{c})\left(\frac{Y'}{Y}\right)^{-\frac{1}{1-\tau_l}}=1.
\end{align}

Above equation determines inflation in the clean sector:
\begin{align}\label{eq:inf_c}
1+\pi_c=(1+\upsilon_{c})^{\frac{\tau_l(1-\alpha)}{(1-\tau_l)-\varepsilon(1-\alpha)-\alpha}}.
\end{align}
By symmetry of (i) how goods enter the production fo the final good and of (ii) sectors, it also holds that 
\begin{align}\label{eq:inf_d}
1+\pi_d=(1+\upsilon_{d})^{\frac{\tau_l(1-\alpha)}{(1-\tau_l)-\varepsilon(1-\alpha)-\alpha}}.
\end{align}


\noindent \textbf{(Aggregate output result, less relevant for main story)}

Hence, 
\begin{align}\label{eq:agg_growth}
(1+g_y)=\frac{Y'}{Y}=(1+\upsilon_{c})^\frac{(1-\tau_l)[1-(\varepsilon(1-\alpha)+\alpha)]}{(1-\tau_l)-(\varepsilon(1-\alpha)+\alpha)}.
\end{align}
Equation \ref{eq:agg_growth} implies the following proposition:
\begin{prop}[aggregate growth]
	\textit{For a proportional tax system, $\tau_l=0$, aggregate growth equals growth in the clean sector. 
		When the tax system is progressive\footnote{\ In the sense defined in \cite{Heathcote2017OptimalFramework}.}, $\tau_l>0$, then aggregate growth exceeds technology growth in the clean sector. When the tax rate is regressive, $\tau_l<0$, aggregate growth is smaller than technology growth in the clean sector. }
\end{prop}
\tr{Have to understand why.} 
With a flat tax system there is no inflation in the clean sector; compare equation \ref{eq:inf_c}. When the tax system is progressive, ...

\paragraph{Proof: labour input good constant}
%\textit{Check that the labour input good is constant:} 

First note that $\frac{l_{hc}}{l_{lc}}$ is constant over time. 
From the FOC governing high skill demand in the clean sector and equation \ref{eq:constant} we have:

\begin{align*}
\frac{l_{hc}}{l_{lc}}=\left(\frac{p_{cL}}{w_h}\theta_c\right)^{\frac{1}{1-\theta_c}}= constant.
\end{align*}

Substitution into the production function of the clean labour input good yields

\begin{align*}
\frac{L'_c}{L_c}=\frac{l_{lc}'}{l_{lc}}.
\end{align*}

\tr{To be continued.}

\textbf{\tr{To be shown next:  How $\tau_l$ affects (1) skill supply (level) and (2) externality. }}
\\

\paragraph{Overview BGP compatible growth rates}
\begin{align}
\frac{Y'}{Y}\\
\frac{w_h'}{w_h}=\frac{w_l'}{w_l}\\
tbc
\end{align}
\textbf{Conditions for BGP to exist}
Next to assuming no transition of labour input goods across sectors, a joint condition on tax progressivity and substitutability of sector goods ensures that output growth in both sectors is positive which has to be the case as otherwise production of one good tends to zero which cannot be an equilibrium when goods are no perfect substitutes.
Hence, on a BGP it has to hold that 
\begin{align*}
\frac{(1-\alpha)(1-\tau_l-\varepsilon)}{(1-\tau_l)-(\varepsilon(1-\alpha)+\alpha)}>0.
\end{align*}
There are two possible ranges of parameter values reads
\begin{align*}
\text{either}\\	&(1-\tau_l)>\varepsilon\hspace{4mm}&\text{if goods are substitutes} ;\\ \text{and}\ \ & (1-\tau_l)>\varepsilon(1-\alpha)+\alpha\hspace{4mm}&\text{if goods are complements}\\
\text{or}\\	&(1-\tau_l)<\varepsilon\hspace{4mm}&\text{if goods are complements} ;\\ \text{and}\ \ & (1-\tau_l)<\varepsilon(1-\alpha)+\alpha\hspace{4mm}&\text{if goods are substitutes}
\end{align*}
Focus on the case that goods are substitutes. Then the condition in the \textit{either}-statement prevents the government from choosing a progressive tax system, since $\tau_l<1-\varepsilon<0$.
Analogously, when goods are complements, the \textit{or}-statement excludes regressive tax systems.
I, therefore, ensure that when goods are complements, it holds that $\tau_l<(1-\alpha)(1-\varepsilon)$. When they are substitutes, it holds that $\tau_l>(1-\alpha)(1-\varepsilon)$.


\tr{Remaining problem: prices are not constant on BGP with fixed growth rates...Look at literature on positive trend inflation...}

\begin{comment}
\textbf{Below wrong Because wrong hl used}
From here,  equilibrium conditions determine prices $p_{dL}, p_{cL}$. Using \ref{eq:constant} skill wages follow. Together with the FOC on hours supply, wages determine aggregate demand. Imposing goods market clearing and using equations \ref{eq:lab_inputc} and \ref{eq:lab_inputd}, determines low skill hour demand in equilibrium.

\begin{align*}
h_l=\left( \frac{1}{\left(\frac{\alpha}{\psi}\right)^{\frac{\alpha}{1-\alpha}}\left[\left(p_c^\frac{\alpha}{1-\alpha}\chi_c A_c\right)^\frac{\varepsilon-1}{\varepsilon}+\left(p_d^\frac{\alpha}{1-\alpha}\chi_d A_d\right)^\frac{\varepsilon-1}{\varepsilon}\right]^\frac{\varepsilon}{\varepsilon-1}}\right)\ \lambda \left(H w_l\right)^{1-\tau_l}.
\end{align*}

Knowing $h_l$, the variables $L_c, \ L_d, \ h_h, \ l_{lc}, l_{ld}, l_{hc}, l_{hd}$ follow. 

Output of the clean and dirty sector read
\begin{align}
%Y_d& =  \frac{\chi_d A_d}{\left[\left(\left(\chi_d A_d\right)^\frac{\alpha}{\alpha+\varepsilon(1-\alpha)}\left(\chi_c A_c\right)^\frac{\varepsilon(1-\alpha)}{\alpha+\varepsilon(1-\alpha)}\right)^\frac{\varepsilon-1}{\varepsilon}+\left(\chi_d A_d\right)^\frac{\varepsilon-1}{\varepsilon}\right]^\frac{\varepsilon}{\varepsilon-1}} \lambda (H w_l)^{1-\tau_l}\\
&Y_d = \left(\frac{1}{\left(\frac{\chi_c A_c}{\chi_d A_d}\right)^{\frac{(\varepsilon-1)(1-\alpha)}{\alpha+\varepsilon(1-\alpha)}}+1}\right)^\frac{\varepsilon}{\varepsilon-1}\lambda (H w_l)^{1-\tau_l}\\
& Y_c= \left(\frac{1}{1+\left(\frac{\chi_d A_d}{\chi_c A_c}\right)^{\frac{(\varepsilon-1)(1-\alpha)}{\alpha+\varepsilon(1-\alpha)}}}\right)^\frac{\varepsilon}{\varepsilon-1}\lambda (H w_l)^{1-\tau_l}.
\end{align}

The government can affect dirty production by lowering aggregate demand. Note that $\chi_c,\ \chi_d$ are functions of the disutility from high skill labour supply, $\zeta$. As a result, the elasticity of diryt and clean output to tax progressivity is asymmetric.

\end{comment}<- Important notes on nature and on Balanced growth path in model based on fried
%\section{Guideline Computations}
\begin{enumerate}
	\item calibrate initial situation to data, using observed tax rates (this would be a competitive equilibrium with taxes as given)
	\item find BGP; BGP exists when $c^*_s+c_n^*\geq \bar{c}$; that is optimal allocation without penalty satisfies basic needs; from this point onwards there are no reallocations across sectors and sectors grow at a constant rate, this is equivalent to the solution of the problem without penalty term \ar \textbf{Need to solve Ramsey problem for BGP absent penalty term}
	Could also solve for the BGP in Ramsey model numerically: get model equations and set growth corrected variables to constant values\\
	Follow \cite{Jones1993OptimalGrowth}:
	\begin{enumerate}
		\item fix assumed SS tax rates and transfers relative to output
		\item calculate ss values of consumption/output, other variables relative to output (constant in ss)
		\item make end corrections to Ramsey problem which is explicitly solved up to period T given the values from point 1 and 2 above
		\item iterate until guess in 1 matches with solution for ss value 
	\end{enumerate}
end corrections are derived analytically.
	\item for competitive equilibrium follow \cite{Acemoglu2008CapitalGrowth}: 
	\begin{enumerate}
		\item analytically or numerically calculate BGP values
		\item initial values and parameters match to data (including tax rates and transfers)
		\item use shooting (or relaxation) algorithm to find solution, i.e. sequence of allocations that solve two boundary value problem: shooting algorithm finds initial conditions that s.t. ss values are matched. 
		\item proof uniqueness of transition path? 
	\end{enumerate}
	
	they write \begin{quote}
		The previous subsection demonstrated that there exists a unique CGP with  nonbalanced  sectoral  growth;  that  is,  there  is  aggregate  output growth at a constant rate together with differential sectoral growth and reallocation of factors of production across sectors. We now investigate whether the competitive equilibrium will approach the CGP. 
	\end{quote} 
\ar From where my economy starts today with calibrated ws, and taxes (a constant growth path), does it converge to a new constant growth path where ws is higher? 
\item in my model growth rates of sectoral consumption are not constant over time! Households reallocate shares as they get richer
\end{enumerate}

Simple version without growth
\begin{enumerate}
	\item economy is in SS today, as households income does not change their consumption is fixed, period t=0
	\item then in period t=1 ws rises (1)\ar what is the optimal policy when ws rises starting from calibrated tax rates; (2)\ar what is the new ss and how does the economy converge?
	\item[\ar] I know initial and end conditions, the shock system is also known a priori, then use shooting or relaxation algorithm to calculate transition
\end{enumerate}
%-------------------------------------
\clearpage
\bibliography{../../../bib_2_0}
\addcontentsline{toc}{section}{References}
\end{document}