
\subsubsection{Effect of model features on optimal policy}\label{subsec:xgrnsk}

To complete the optimal policy discussion, I address the role of certain model features. 
Figure \ref{fig:comp_mod} presents the optimal policy in a model with exogenous growth, the blue dashed graphs, without skill heterogeneity, the orange dotted graphs, and without knowledge spillovers. The black solid line indicates the results in the benchmark model. 

\paragraph{Endogenous growth}
When growth is exogenous, the optimal income tax progressivity is lower than in the benchmark model during the initial 10 years and higher thereafter (panel (a)). Furthermore, the increase in tax progressivity over time is more pronounced.  

The environmental tax is higher in all periods when growth is exogenous (panel (b)). This finding is in line with \cite{Fried2018ClimateAnalysis} who argues that directed technical change amplifies the recomposing effects of the fossil tax. The higher price for fossil energy increases demand for green energy. Since innovation responds to this shift in demand, the slowdown in fossil production is amplified. 

%The reduction in consumption below the respective efficient allocation is only slightly more than 10\% at maximum whereas consumption falls short of the efficient level by up to 55\% in the benchmark model; figure \ref{fig:comp_mod_allo_dev} in the appendix shows how optimal allocations deviate from the efficient one for the three models studied.

The rationale behind the higher tax progressivity and its attenuated decline in the exogenous growth model is that, first, due to the more aggressive carbon tax, a stronger reduction in labor supply is efficient. Second, when growth does not respond to the change in the high-to-low skill ratio, the compositional effect of a progressive income tax is muted. Furthermore, the social planner chooses an increasing path in hours worked during the net-zero period in the benchmark model. As output declines the income effect makes the social planner value additional output more. This is replicated by the falling tax progressivity. 
The more regressive tax in the initial 10 periods allows for a higher green-to-fossil output ratio. 

%\footnote{\ Compare figures \ref{fig:LF_vs_onlytaul_xgrnsk} and \ref{fig:LF_vs_onlytaul_xgr} which show models with exogenous growth but without and with skill heterogeneity, respectively.}
%When growth is endogenous, the skill-recomposition channel dominates making income taxes less advantageous from an environmental policy perspective.
 %Since the fossil sector relies more on labor than the green sector, the reduction in labor makes fossil production relatively more expensive. This increases the green-to-fossil energy mix. Same holds true for the composition of the final output good which is recomposed towards the less labor intense energy good. However, with endogenous growth, this channel is muted and the adverse recomposing effect of labor income tax progressivity on the green energy share dominates. 
%Overall, the beneficial recomposing effect in the exogenous growth model explains the higher labor tax progressivity and the muted decline over time. 

%Figure \ref{fig:count_taul_xgr} shows how the economies evolve when the optimal income tax from the benchmark model is fed into (1) the exogenous growth model, the black graph, and (2) the exogenous growth model, the blue dashed graph. The respective grey graphs show the allocation in the laissez-faire economy.
%
%
%The reason for the higher tax progressivity in the exogenous growth model relative to the benchmark model is not driven by the effect of income tax progressivity on growth. Growth seems largely unaffected by the income tax; compare panel (e) for the level of growth and panel (f) for the ratio of green to fossil scientists. The effect on innovation is absorbed by wages which increase with the labour income tax; see panels (l) and (m).  
%
%Rather, the higher tax progressivity in the model with exogenous growth arises from the higher environmental tax and the complementarity of the two instruments. This narrative becomes clear when comparing how the social planner adjusts labor supply differently in the two models, see figure \ref{fig:eff_model}. The efficient level of hours worked is lower in the exogenous growth model. Not because it is less costly to reduce hours worked, but because the environmental tax is more aggressive. The channel analyzed in the analytical section.  




%\tr{is it the recomposiing effect or the effect on growth in general? \ar both not present---Cant tell }


\begin{figure}[h!!]
	\centering
	\caption{Optimal policy by model}\label{fig:comp_mod}
	
	\begin{minipage}[]{0.32\textwidth}
		\centering{\footnotesize{(a) Income tax progressivity, $\tau_{\iota t}$\\ \ }}
		%	\captionsetup{width=.45\linewidth}
		\includegraphics[width=1\textwidth]{../../codding_model/own_basedOnFried/optimalPol_010922_revision/figures/all_13Sept22_Tplus30/CompMod1_OPT_T_NoTaus_taul_regime0_spillover0_knspil0_sep0_extern0_PV1_etaa0.79_lgd0.png}
	\end{minipage}
	\begin{minipage}[]{0.32\textwidth}
		\centering{\footnotesize{(b) Environmental tax, $\tau_{Ft}$\\ \ }}
		%	\captionsetup{width=.45\linewidth}
		\includegraphics[width=1\textwidth]{../../codding_model/own_basedOnFried/optimalPol_010922_revision/figures/all_13Sept22_Tplus30/CompMod1_OPT_T_NoTaus_tauf_regime0_spillover0_knspil0_sep0_extern0_PV1_etaa0.79_lgd0.png}
	\end{minipage}
\begin{minipage}[]{0.3\textwidth}
	\includegraphics[width=1\textwidth]{../../codding_model/own_basedOnFried/optimalPol_010922_revision/figures/all_13Sept22_Tplus30/legend_compmod1.png}
\end{minipage}
\end{figure}

\begin{figure}[h!!]
	\centering
	\caption{Optimal policy by model}\label{fig:comp_mod_Tls}
	
	\begin{minipage}[]{0.32\textwidth}
		\centering{\footnotesize{(a) Income tax progressivity, $\tau_{\iota t}$\\ \ }}
		%	\captionsetup{width=.45\linewidth}
		\includegraphics[width=1\textwidth]{../../codding_model/own_basedOnFried/optimalPol_010922_revision/figures/all_13Sept22_Tplus30/CompMod1_OPT_T_NoTaus_taul_regime4_spillover0_knspil0_sep0_extern0_PV1_etaa0.79_lgd0.png}
	\end{minipage}
	\begin{minipage}[]{0.32\textwidth}
		\centering{\footnotesize{(b) Environmental tax, $\tau_{Ft}$\\ \ }}
		%	\captionsetup{width=.45\linewidth}
		\includegraphics[width=1\textwidth]{../../codding_model/own_basedOnFried/optimalPol_010922_revision/figures/all_13Sept22_Tplus30/CompMod1_OPT_T_NoTaus_tauf_regime4_spillover0_knspil0_sep0_extern0_PV1_etaa0.79_lgd0.png}
	\end{minipage}
\begin{minipage}[]{0.3\textwidth}
\includegraphics[width=1\textwidth]{../../codding_model/own_basedOnFried/optimalPol_010922_revision/figures/all_13Sept22_Tplus30/legend_compmod1.png}
\end{minipage}
\end{figure}
\begin{comment}

Despite the more aggressive intervention, consumption in the Ramsey allocation only deviates by -10\% from the efficient allocation over the whole time period considered (panel (a) in figure \ref{fig:comp_mod_allo}). In contrast, in the benchmark model, the deviation of consumption aggravates over time reaching -55\% relative to the respective efficient allocation. 
Interestingly, the recomposing effect of tax progressivity through its impact on the skill ratio is negligible in the exogenous growth model. Indeed, the skill ratio diverges more from the efficient ratio due to the higher tax progressivity (panel (e)). However, since the direction of growth does not respond to the supply of skills, the green-to-fossil energy mix is roughly similar to the efficient one (panel (d)). 

content...
\end{comment}

%- homogenous skill
\paragraph{Skill heterogeneity}
When there is only one type of skill, the optimal income tax progressivity is higher, as well, than in the benchmark model. Yet, the difference is less pronounced compared to the model with exogenous growth. Besides, optimal tax progressivity converges to the one in the benchmark model over time. 
As argued above, endogenous growth masks the advantageous recomposing effect of tax progressivity on the energy mix through the labor-share channel. 

Compared to the benchmark model, the planner does not face a trade-off between too low high-skill supply and too high low-skill supply and can implement hours close to the efficient level.\footnote{ Panel (b) in figure \ref{fig:comp_mod_allo_dev} in the appendix shows deviations from the respective efficient allocation by model. When skill heterogeneity is switched off, labor supply only deviates at maximum by -0.6\% from the efficient allocation. In the benchmark model the same number rises to -3.5\% and +1\% for the high- and the low-skill type. } This explains the higher tax progressivity. 
Furthermore, there is no increase in fossil research as is the case in the benchmark model relative to the laissez-faire allocation.\footnote{\ See figures \ref{fig:LF_vs_onlytaul} and \ref{fig:LF_vs_onlytaul_nsk} which show the effect of the respective optimal  tax progressivity relative to the laissez-faire allocation in the benchmark model and the one with skill homogeneity, respectively.}
%Indeed, the deviation in consumption and growth from the efficient allocation is similar in the model with homogeneous skills to the benchmark model (panels (a) and (g)). 

The fossil tax is higher than in the benchmark alternative throughout, the orange dotted graph in panel (b).
On the contrary to income tax progressivity, the environmental tax diverges more from the benchmark model when there is no skill heterogeneity relative to the alternative with exogenous growth. The motive behind this result is the increased input market for fossil production when there is only one skill type. A smaller low-skill supply, therefore, contributes to lower emissions. I discuss the mechanism in more detail in appendix section \ref{app:count}.

 
 