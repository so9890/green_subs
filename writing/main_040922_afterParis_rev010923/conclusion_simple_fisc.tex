\newpage
\section{Conclusion}\label{sec:con}

Worldwide policy efforts to implement climate targets fall short of required efforts \citep{IPCC2022}. Political feasibility of policies is a crucial feature of successful climate policies. On the other hand, popular policies do not coincide with efficient ones. \cite{Fabre2023Fighting} document that using carbon taxes to subsidize green technology usage attracts around 70\% more support than recycling carbon tax revenues lump sum. However, the latter forms part of an efficient policy mitigation. Therefore, the question arises how to reconcile feasible and efficient climate policies.


In the first part of the paper, I show in a simple model that the optimal environmental policy features lump-sum rebates of environmental tax revenues. When lump-sum transfers are not available, households' labor supply is inefficiently high. To correct for this inefficiency, distortive labor income taxation qualifies as a suited policy instrument. % The intention is to mitigate distortions on the labor market arising from the lack of lump-sum rebates. %The model does not feature inequality.
% Quantitative results
% baseline model
%When environmental tax revenues are not redistributed lump sum, labor supply is inefficiently high. Then, income taxes serve to diminish hours worked closer to the efficient level. The result prevails absent income inequality.


% quantitative
In the second part, I analyze the optimal policy in a quantitative model with  endogenous growth\textemdash the state-of-the-art model to study green transitions. The government seeks to maximize welfare but is constrained by an exogenous emission limit. 
I find that a politically feasible carbon-tax and green subsidy policy is best combined with a positive tax on labor income. The marginal labor income tax rates vary between 6\% and 9\% during the transition. These values correspond to between 25\% and 38\% of current US tax rates.


% extensions
In light of the political feasibility motivation of the paper, it is questionable whether taxing labor is a more popular policy.  As a potentially more accepted policy, I plan to replace labor income taxes with a direct adjustment of working hours \citep{Alvarez-Cuadrado2007EnvyHours}. 

%In an extension, I am planning to give the Ramsey planner the opportunity to limit working hours directly. The literature advocating a reduction in consumption levels \citep[e.g.,][]{Schor2005SustainableReductionb} proposes a restriction of hours worked as policy instrument to lower the consumption of resources.
%Even though advocated in the literature, there is evidence for political difficulties in reducing working hours. In 2020, the French Citizens' Convention on Climate voted against reducing working hours as a measure to handle climate change. Potentially, ignorance of economic consequences is an explanation. The extension would serve to better understand these consequences.

%Secondly, the model features log-utility of consumption in order to keep labor supply unaffected from the scale of the income tax scheme. A substitution and income effect offset each other as the wage rate rises.  However, recent evidence exists showing empirically that working hours reduce over time with productivity \citep{Boppart2019LaborPerspectiveb}. An income effect dominates the substitution effect. A lack of lump-sum transfers may then exacerbate the need for progressive income taxes. 

%Thirdly, it would be interesting to investigate whether there is a role for progressive income taxes in the optimal environmental policy when green subsidies are available. \cite{Acemoglu2012TheChange} point to the importance of green subsidies within the optimal environmental policy. The use of progressive taxes to boost growth might be redundant. % Then again, subsidies may boost labor demand aggravating the inefficiency in working hours. 
%Secondly, the model abstracts from income inequality and government funding constraints which constitute traditional motives for income taxation.  Integrating these aspects into the model 
