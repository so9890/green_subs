\section{Core Model and Theoretic Results}\label{sec:mod_an}
%\tr{if $tau_f$ was to replicate share of marginal products, then $H^*\geq H_{FB}$; i.e. absent lump-sum transfers (sub-optimal setting)}
%\textbf{Points to be made}
%\begin{enumerate}
%\item the efficient allocation consists of both a recomposing and a scaling element \ar discuss social planner allocation \checkmark
%\item if $tau_f$ was to replicate share of marginal products, then $H^*\geq H_{FB}$; i.e. absent lump-sum transfers (sub-optimal setting) \checkmark
%\item lump-sum transfers implement the efficient reduction in hours worked and ensure the efficient amount of consumption \checkmark
%\item absent lump-sum transfers and income, households work too much  \checkmark; This follows from setting $\tau_\iota>0$.
%\item redistribution through the income tax scheme establishes the efficient allocation if the income tax is progressive \textit{Intuition: in contrast to lump-sum transfers, transferring environmental tax revenues through the income tax scheme constitutes a positive multiplication of labor income \ar this increases labor efforts. The progressive tax counters this tendency.} \checkmark
%\end{enumerate}

This section develops a tractable model  to derive the theoretic results. To focus on the role of lump-sum rebates in the optimal policy, I assume that the government consumes carbon tax revenues. I show that, absent lump-sum transfers, solely relying on a carbon tax is not efficient: households supply too much labor. Therefore, the government would revert to taxing labor income to adjust work effort closer to the efficient level.
%When carbon tax revenues are used as green  subsidies, lump-sum transfers, which form part of an efficient externality mitigation, are missing. This makes households relatively poorer and their labor supply 
% In section \ref{sec:model}, the model is extended to the quantitative framework notably by adding endogenous growth and skill heterogeneity. %investigate the inefficiency arising in hours worked when an environmental externality has to be taken care of.
\subsection{Model}
The representative household faces a consumption and labor supply decision. The final consumption good is a composite of a fossil and a green good. Labor is the only input to production. The fossil sector causes an environmental externality.\footnote{ For simplicity, the green sector does not induce any externality; yet, whenever intermediate goods are no perfect substitutes, final good production is never perfectly green.} There is no growth, and the model is static.

\paragraph{Representative household}
Throughout the paper, the household's decision is static. Each period, the household maximizes its period utility
\begin{align*}
	U(C,H; F).
\end{align*} 
The household derives utility from consumption, $C$, but experiences disutility from hours worked, $H$. An externality from fossil production, $F$, decreases household utility. The level of fossil production is taken as given by the household.
I assume additive separability of consumption, hours, and the externality. Utility of consumption is increasing and strictly concave. Utility is decreasing and strictly convex in hours worked and fossil production.
Utility maximization is subject to a period budget constraint:
\begin{align}
	C= (1-\tau_{\iota})wH+T+T_F. \label{eq:hhbudget}
\end{align}
The variable $w$ indicates the wage rate.  Lump-sum transfers from the government are denoted by $T$. The marginal tax rate is given by $\tau_{\iota}$. Since the use of the labor income tax in this paper is motivated to target the level of production it is sufficient to study a linear income tax scheme.\footnote{ In fact, theoretic results remain unchanged under a non-linear tax scheme.}
The variable $T_F$ captures lump-sum rebates from the carbon tax.

\paragraph{Production}
All sectors of production are perfectly competitive, and production functions have decreasing returns to scale. %\footnote{ \textit{With increasing returns to scale the assumption of perfect competition would be violated. With constant returns to scale, the solution is not unique.}}. The final consumption good, $Y$, is a composite of the fossil, $F$, and the green intermediate good, $G$. 
Intermediate goods, indicated by $J\in \{F,G\}$ for fossil and green, are produced from the labor input good, $L_J$, using technology, $A_J$. The variable $Y$ stands in for final output and is the numeraire. Production is given by:
\begin{align}
	Y=Y(F, G), \hspace{5mm} F=F(A_F, L_F),\hspace{5mm} G=G(A_G, L_G). \label{eq:prod}
\end{align}

\paragraph{Government}
The government raises income taxes from households and levies an environ- mental tax, $\tau_F$, per unit of fossil energy bought by final good producers. The environmental tax, thus, is modeled in parallel to a carbon tax which poses a price on emissions. Carbon tax revenues are either consumed by the government, $Gov$, or rebated to households lump sum: $T_F$. Revenues from the linear income tax are redistributed lump-sum. \begin{align}
	\tau_{F}F=Gov+T_F, \hspace{7mm}
	T=\tau_{\iota}{w H}. \label{eq:gov_but}
\end{align}
%The scaling parameter $\lambda$ adjusts to balance the income tax scheme. 
%Environmental tax revenues are either transferred lump-sum, fully consumed by the government, or transferred through the income tax schedule.

\paragraph{Markets}
Markets for labor and the final good clear: 
\begin{align}
	H=L_F+L_G,\ \hspace{5mm} Y=C+Gov. \label{eq:market_clear}
\end{align}
%I summarize the eq.s determining the competitive equilibrium in appendix Section \ref{app:model}.
\paragraph{Competitive equilibrium}
In a competitive equilibrium, household behavior is determined by the budget constraint, eq. \eqref{eq:hhbudget}, and labor supply which follows from the household's first order conditions:
\begin{align}
	-U_H=U_C(1-\tau_{\iota})w. \label{eq:hsup}
\end{align}
Firms choose the quantity of input goods to maximize their profits taking prices as given. The following equations describe this behavior in equilibrium:
\begin{align}
	p_G=\frac{\partial Y}{\partial G}, \hspace{5mm}
	p_F +\tau_{F} = \frac{\partial Y}{\partial F}, \hspace{5mm}
	w= p_F\frac{\partial F}{\partial L_F}=p_G\frac{\partial G}{\partial L_G}.\label{eq:profmax}
\end{align}

The competitive equilibrium is defined as prices and allocations so that households and firms behave optimally; i.e. eqs. \eqref{eq:hhbudget}, \eqref{eq:hsup}, and \eqref{eq:profmax} hold. Production happens according to eqs. \eqref{eq:prod}.  Equilibrium prices and the wage rate adjust to clear markets, eqs. \eqref{eq:market_clear}. Finally, the government's budgets are satisfied eqs. \eqref{eq:gov_but}. Policy variables $\tau_F$ and $\tau_\iota$ are taken as given. 

\subsection{Theoretic results}\label{sec:theory}
This section derives and discusses the main theoretical results. Section \ref{subsec:sp} defines the efficient allocation. It constitutes a benchmark for the optimal allocation which is discussed in section \ref{subsec:decen_ec}. 
\subsubsection{Social planner}\label{subsec:sp}
Let the share of fossil to total labor be denoted by $s=L_F/H$. The social planner's problem reads
\begin{align}
\underset{s, H}{\max}\ & U(C,H; F)\\ s.t\ \ & C=Y.
\end{align}
The first order conditions are given by
\begin{align}
wrt.\ s:\hspace{4mm} & U_C \cdot \left(\frp{Y}{F}\frp{F}{s}+\frp{Y}{G}\frp{G}{s}\right)=-U_F\frp{F}{s}, \label{eq:fbs}
\\
wrt.\ H:\hspace{4mm} & U_C\frp{Y}{H}+U_F\frp{F}{H}=-U_H\label{eq:fbh}. 
\end{align}
Where $U_X$ denotes the partial derivative of utility with respect to the variable $X$.
These equations determine the efficient or first-best allocation. 
Absent an externality, $U_F=0$, the efficient distribution of labor across sectors equalizes the marginal product of labor across sectors; compare eq. \eqref{eq:fbs}. Efficient hours balance the marginal utility gain from consumption and the marginal disutility from working formalized by eq. \eqref{eq:fbh}. 

When there is an externality, the social planner adjusts the allocation by two modulations: (i) a recomposing and (ii) a scaling one. 
The recomposition is determined by eq. \eqref{eq:fbs}.
The negative externality of fossil production makes it efficient to adjust the fossil labor share so that  a marginal reallocation of labor to the fossil sector would raise output.\footnote{ Note that $U_F<0$ by assumption so that the right-hand side is positive and that $\frac{dG}{ds}<0$. }
I show in Appendix \ref{app:redeffs} that the social planner reduces the fossil labor share when the aggregate production function features decreasing returns to scale in its labor inputs, $L_G$ and $L_F$.
\begin{comment}
The eq. 
\begin{align}
\frac{-U_F}{U_C \frac{dY}{dF}}=1+\frac{\frac{dY}{dG}\frac{dG}{ds}}{\frac{dY}{dF}\frac{dF}{ds}}.
\end{align}
The term on the left-hand side is the social cost of the externality: it measures what the representative household is willing to pay for a further reduction in fossil production. 
\end{comment}

The scaling effect is summarized by eq. \eqref{eq:fbh}.
First note that eq. \eqref{eq:fbh} can be rewritten by substituting eq. \eqref{eq:fbs} and noticing the relation of derivatives with respect to $H$ and $s$.\footnote{ This is done in more detail for the optimal allocation in Appendix \ref{app:incometax0}. The relation of derivatives are summarized in Appendix \ref{app:dervs_use}.}  
The second first order condition becomes:
\begin{align}\label{eq:fbh_simp}
-U_H=U_C\frac{\partial Y}{\partial G}\frp{G}{L_G}.
\end{align}
Hence, the externality drops from the expression which determines efficient labor. Hours are not chosen in a way to handle the externality. It is rather an indirect effect of externality mitigation which makes an adjustment in hours efficient which I will discuss next.

The recomposition of labor input towards the  green sector reduces the marginal product of labor in the green sector and the utility gains from more labor decline.  This effect has two opposing impacts on efficient labor supply. On the one hand, there is a substitution effect: as leisure becomes less costly, the efficient amount of hours reduces (note that the right-hand side of eq. \eqref{eq:fbh} is increasing in $H$). On the other hand, the economy becomes poorer in terms of consumption and more work effort might be efficient. This is captured by the term $U_C$ and equivalent to an income effect. 
%In total, which effect dominates depends on the curvature of the utility from consumption, $\theta$. With $\theta>1$ the  lower marginal product of labor decreases the efficient amount of hours worked. 
%Second, the social planner reduces hours worked due to their negative exeternality through fossil production. This effect is introduced by the term $U_F\frac{dF}{dH}<0$. 
Proposition \ref{prop:0} summarizes above discussion
\begin{prop}\label{prop:0}
	Efficient externality mitigation consists of a recomposing and a scaling approach. Efficiency of the scaling effect arises indirectly due to the reduction in the marginal product of labor induced by a recomposition of input factors.
\end{prop}


Depending on the importance of the income effect, efficient hours worked may be higher or lower than  absent an externality. %\footnote{ \ I discuss in the appendix conditions on parameter values when assuming functional forms of the model.}
I will show in the following, that irrespective of whether the social planner de- or increases hours, the decentralized economy always features higher hours when environmental tax revenues are not redistributed lump-sum. 

\begin{comment}
\hrule
One can show that the total effect of a drop in the fossil labor share on hours worked is positive, i.e. $\frac{dh_{FB}}{ds}>0$, if $\theta<\frac{\varepsilon}{\varepsilon-s}$. If the income effect dominates, the social planner increases hours worked as the economy becomes less productive. 
Under the value for $\theta$ suggested by \cite{Boppart2019LaborPerspectiveb}, the efficient scale effect is to increase hours worked. When, however, the substitution effect outweighs or dominates the income effect - as commonly assumed in the public finance literature \citep{Heathcote2017OptimalFramework, LansBovenberg1994EnvironmentalTaxation, LansBovenberg1996OptimalAnalyses} \tr{CHECK this}!.
Nevertheless, the level of hours worked exceeds the efficient level irrespective of $\theta$ when no lump-sum transfers are available. 
When the efficient level of hours increases, though, the fossil labor share reduces even more to outweigh the increase in the externality.

content...
\end{comment}
\subsubsection{Decentralized economy}\label{subsec:decen_ec}

Governments use tax and transfer instruments to correct for distortions, such as an environ- mental externality. The question arises if the efficient allocation can be decentralized by the use of taxes and transfers in a competitive economy. %For now, I assume that the income tax is not available and $\tau_{\iota}=0$, $\lambda=0$.
I argue in this section that rebates of environmental tax revenues are essential to implement the first-best allocation in the competitive equilibrium. % Only in combination with lump-sum transfers of  environmental tax revenues does an environmental tax suffice to implement the efficient allocation. %Then the environmental tax equals the social cost of the externality as shown by \textit{PIGOU}. 
%When environmental tax revenues are not redistributed lump-sum, hours worked exceed their efficient level, and a role for income taxes to lower hours worked arises. %I consider two cases.
%In the first case, section \ref{subsec:nolump}, environmental tax revenues are consumed by the government. The optimal policy consists of (i) a progressive labor income tax scheme and (ii) an environmental tax which may deviate from the social cost of the externality. The logic is that labor taxes help to align hours worked closer to the efficient allocation. Nevertheless, the efficient allocation is not feasible under this policy regime.

%In the second scenario, therefore, I point to an option to implement the efficient allocation even if lump-sum transfers are not available: redistributing environmental tax revenues through the income tax scheme, section \ref{subsec:integrated}.I show that, again, the optimal tax scheme is progressive. As a consequence, the considered optimal environmental policies which establish the efficient allocation feature - as a side effect - a more equal distribution of income, through either lump-sum transfers or a progressive tax scheme.% \tr{Not sure though if this holds true in progressive scheme as lambda multiplies labor income}).

%\begin{enumerate}
%\item lump-sum transfers important for Pigou tax to implement efficient allocation: Proposition \ref{prop:1}
%\item when transfers are not redistributed: infeasibility of efficient allocation,  role for labor tax, and violation of Pigou principle \ref{prop:2}.
%\item redistribution through income tax scheme with progressive income tax restores efficient allocation \ref{prop:3}
%\end{enumerate}

The government is characterized by a Ramsey planner: it seeks to maximize utility of the representative household but can only revert to tax instruments and transfers to implement the welfare-maximizing allocation. The behavior of firms and households constrains the government's optimization problem:

\begin{align}
\underset{s, H}{\max}\ &\ U(C,H; F)\\ s.t.\ \ & \  C=Y-Gov,
\end{align}
subject to the behavior of firms and households.
The first order conditions differ from the social planner's ones through the derivatives on government revenues, $Gov$:
\begin{align}
wrt.\ s:\hspace{4mm} & U_C\left(\frac{\partial Y}{\partial F}\frac{\partial F}{\partial s}+\frac{\partial Y}{\partial G}\frac{\partial G}{\partial s}-\frac{\partial Gov}{ \partial s}\right)=-U_F\frac{\partial F}{\partial s}, \label{eq:sbs}
\\
wrt.\ H:\hspace{4mm} & U_C\cdot \left(\frac{\partial Y}{\partial H}-\frac{\partial Gov}{\partial H}\right)+U_F\frac{\partial F}{\partial H}=-U_H\label{eq:sbh}. 
\end{align}

%-- paragraph to show that with Gov=0 and lump-sum transfers, the efficient allocation is implemented
When environmental tax revenues are fully redistributed lump-sum, i.e. $Gov=0$, an environmental tax equal to the marginal social cost of fossil production\footnote{ I define and derive the social cost of fossil production in Appendix  \ref{app:scp}.} implements the efficient allocation.\footnote{Appendix \ref{app:incometax0} proofs this claim.} This observation is known as the \textit{Pigou principle} in the literature. 
To see this, note that eq. \eqref{eq:sbs} ensures that the social planner's first order condition, eq. \eqref{eq:fbs}, is satisfied. 
Rewriting eq. \eqref{eq:fbs} reveals that the Pigou principle holds: %\footnote{ I derive the social cost of pollution as the price the representative household is willing to pay for a marginal reduction in fossil production. The derivation is exponded in appendix section \ref{sec:mod_an}. 
%	To be precise, social cost of pollution refers to the marginal cost evaluated at the resulting equilibrium allocation.}: The Pigou principle. 
\begin{align}
\underbrace{\frac{-U_F}{U_C\frac{\partial Y}{\partial F}}}_{\text{marginal social cost of fossil production}}=\left(1+\frac{\frac{\partial Y}{\partial G}\frac{\partial G}{\partial s}}{\frac{\partial Y}{\partial F}\frac{\partial F}{\partial s}}\right)\frac{\partial Y}{\partial F}=\tau^*_F.
\end{align}
Where the second equality follows from substituting intermediate firms' profit maximization conditions from eqs. \eqref{eq:profmax}. %\footnote{I show in appendix section \ref{app:incometax0} that setting the environmental tax to the social cost of fossil production implies that the second first order condition of the Ramsey planner is satisfied without use of the income tax instrument: $\tau_{\iota}^*=0$.} %at $\tau_\iota=0$ when all environmental tax revenues are redistributed lump-sum: $T_{ls}=\tau_{F}p_FF$ (and $Gov=0$ and $T_\iota=0$).


Absent an externality of production, it is efficient to balance marginal products of labor across sectors.
When there is an externality, the social planner lowers the share of labor in the fossil sector. As a result, the marginal product of labor in this sector increases. It falls in the green sector. To sustain this gap between marginal products in the competitive equilibrium, the government has to introduce a corrective tax. Otherwise, market forces would direct labor towards the sector with the higher marginal product. Consequently, the equilibrium wage rate is below the marginal product of labor.\footnote{ I formally discuss this statement in Appendix \ref{app:wageMPL}.} 

%In this paragraph, I briefly discuss the mechanism of the corrective tax.As discussed previously, absent an externality of production, it is efficient to balance marginal products of labor across sectors. However, when there is an externality, the social planner lowers the fossil share of labor which results in a higher marginal product of labor in the fossil sector. To sustain this gap between marginal products in the competitive equilibrium, the government has to introduce a corrective tax so that market forces do not direct labor towards the sector with the higher marginal product. In other words, the corrective tax is set so that wage rates equalize despite heterogeneous marginal products of labor. As a result of this intervention, the equilibrium wage rate is below the marginal product of labor in the fossil sector.\footnote{ I formally discuss this statement in appendix section \ref{app:wageMPL}.} These are efficiency costs associated with a use of the environmental tax. They are the source of the competition between environmental good provision and raising government funds or equity alluded to in the literature \citep[e.g.][]{LansBovenberg1994EnvironmentalTaxation}.  However, from an environmental policy perspective, the adjustment in labor due to a lower marginal product in the green sector is efficient. It mirrors the reduction in the marginal product of green labor in the efficient allocation. It is only when labor constitutes the base of another tax, that the reduction in labor supply becomes costly. 
\subsubsection{Lump-sum transfers and the optimal environmental policy}\label{subsec:nolump}

A distortion of labor supply occurs when environmental tax revenues are not redistributed lump-sum.
In light of the discussions on how best to recycle environmental tax revenues, this is an important result which I summarize in proposition \ref{prop:1}:

\begin{prop}\label{prop:1}\textbf{Importance of lump-sum transfers of environmental tax revenues}
	Without lump-sum transfers of environmental tax revenues, hours worked are inefficiently high when $\tau_{F}$ implements the efficient fossil labor share.
%Absent lump-sum transfers and when the  wage rate is non-increasing in equilibrium hours, 
%implementing the efficient share of fossil labor, $s^*=s_{FB}$, via an environmental tax results in inefficiently high hours worked. Lump-sum transfers would serve as a means to lower hours worked via an income channel. %The efficient allocation is infeasible.=> this statement would need to look at the optimal policy, this here is not a statement on the optimal policy
\end{prop}

% The logic is as follows:
%$s^*=s_{FB}$ when tauf implements the efficient allocation. 
% yet, implementing s-efficient without lump-sum transfers results in inefficiently high hours.

 
The proof of proposition \ref{prop:1}, depicted in appendix section \ref{app:nolumpsum_hourshigh}, is informative on the mechanism: The conclusion that $H^*>H_{FB}$\textemdash where the subscript $FB$ indicates the first-best allocation, while an asterisk marks the optimal allocation\textemdash follows from consumption in the competitive equilibrium being lower than in the first-best allocation. Lump-sum transfers, thus, imply lower hours worked in the competitive equilibrium through an income effect.

% source of the inefficiency
%Are labor taxes used to cope with the externality, or is it rather that environmental taxation induces a distortion on labor supply?  I will argue in this section taht 
%The distortion in labor supply results from non-redistribution of environmental tax revenues. As it lowers the wage rate to ensure a lower labor share in the fossil sector, the environmental policy positively affects labor supply via an income effect. When lump-sum transfers are not used to counter this mechanism, equilibrium hours are inefficiently high.  %; otherwise, if the labor tax had an advantage in mitigating the externality, it would have been used in the presence of lump-sum transfers. But it is not.

% violation Pigou principle
%Choosing the environmental tax to implement the efficient share of fossil production while hours are inefficiently high, most likely  violates the Pigou principle: the environmental tax does not equal the social cost of pollution. One reason is that a higher labor supply increases fossil production above the efficient level; the social cost of pollution increase when the disutility of pollution is convex. Another reason is that household consumption deviates from the efficient level of consumption: if it is below, then the marginal utility of consumption increases diminishing the willingness to pay for a reduction in the externality. 

% discussion: importance to reduce hours worked in context of exogenous emssion limit
When labor supply is endogenous, lump-sum transfers gain in importance for the optimal allocation. When labor supply is fixed, the non-redistribution of environmental tax revenues results in inefficiently low consumption with no further impact on emissions. When labor supply is elastic, however, the lower consumption results in too high hours worked. This is especially important as it aggravates the externality by increasing economic production. When there is an absolute limit on the externality\textemdash as is the case for greenhouse gas emissions today\textemdash the scale effect could make a stricter environmental tax necessary. 

%\textit{could be} optimal absent means to reduce hours worked. 

% transition to proposition 2: optimal policy
Proposition \ref{prop:1} rationalizes the use of distortionary income taxes as a tool to lower the supply of labor when environmental tax revenues are not redistributed lump-sum.
In the following, I quantify the optimal labor income tax to accompany the politically feasible policy regime: a carbon tax with subsidies to the green sector. 

