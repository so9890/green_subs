\clearpage
\setcounter{page}{1}
\section{Introduction}
%\tr{I show that carbon taxes are only efficient if lump-sum transfers are available.}

\begin{comment}
\tr{Think about:
	1) when labor income taxes are not used, then need to have  a higher environmental tax to meet emission limits? \ar Yes, because of advantageous level effect which outweighs recomposing effect of income tax.
	2) When staying at level optimal under the assumption of lump-sum redistribution, but then not redistributing, than absent labor income tax emissions are too high; by how much? Counterfactual}
	
	content...
	\end{comment}

%\paragraph{new introduction 28/08/23}
The latest assessment report of the Intergovernmental Panel on Climate Change \citep{IPCC2022} highlights the urgency to reduce greenhouse gas emissions.%relative to the previous report from 2018 \citep{Rogelj2018MitigationDevelopment.}.
\footnote{ \  The report stresses the decreasing likelihood of meeting the Paris Agreement and limiting climate warming to 1.5°. The Paris Agreement of 2015 formulates clear political goals to mitigate climate change. Under this treaty, states have agreed on a legally binding maximum increase in temperature to well below 2°C, preferably 1.5° over pre-industrial levels, and the global community seeks to be climate-neutral in 2050  (see: \url{https://unfccc.int/process-and-meetings/the-paris-agreement/the-paris-agreement}). 
} In theory, it is well known that a carbon tax in combination with lump-sum transfers allows to reduce emissions at the lowest costs.\footnote{\cite{Acemoglu2012TheChange} show that  in a model of directed technical change the efficient policy also features research subsidies to complement a carbon tax. The subsidy serves to make firms internalize dynamic spillovers in research important for a green transition.}
 However, the slow implementation of climate policies points to political obstacles in following an efficient policy. In a global survey, \cite{Fabre2023Fighting} find that the acceptance of lump-sum transfers of environmental tax revenues is  low. The use of carbon tax revenues to subsidize green technology, instead, sees a 70\% higher support in high-income countries. Albeit politically feasible, this policy tuple is not efficient. In this paper, I ask what policies should accompany the ``politically feasible" policy to reduce the costs of a green transition. 

To this end, I study a model economy with elastic labor supply. I show analytically that absent lump-sum transfers of carbon tax revenues, labor supply is inefficiently high. The lack of lump-sum transfers makes households poorer so that the lower income spurs their willingness to work. Thus, distortive labor income taxes may help diminish work effort.
In a next step, I use a model of endogenous growth to quantify the optimal policy to meet climate targets. Next to the politically feasible policy of carbon taxes and green subsidies, the government can choose to tax labor income. I find that the optimal policy features a marginal labor income tax rate between 6\% and 9\% reducing hours worked by up to 3.9\%. The carbon tax necessary to meet emission limits is up to 4.6\% higher when labor is not taxed since the level of economic activity remains high.

% relation to literature
These results inform the political and academic debate on the use of environmental tax revenues 
\citep[e.g.][]{Baker2017TheDividends, Fried2018TheGenerations, Carattini2018}. More precisely, the paper points to the importance of lump-sum transfers within the optimal environmental policy to lower labor supply; an aspect which appears overlooked. %\footnote{\ POLICY debate; \cite{Fried2018TheGenerations}}
When lump-sum transfers do not form part of the environmental policy, governments should take into account alternative tools to cope with too high work effort. If they do not, carbon taxes have to be higher to meet emission limits due to a higher level of production and, thus, emissions.

%Second, the results contribute to the academic debate on the so-called \textit{weak double dividend} \citep[for example:][]{LansBovenberg1994EnvironmentalTaxation, LansBovenberg1996OptimalAnalyses}. The hypothesis posits that recycling environmental tax revenues to reduce preexisting tax distortions is advantageous to transferring  revenues lump sum. The rationale is that transfers decrease labor supply thereby diminishing the tax base of the income tax. %A conflict between generating government funds and environmental protection arises. 
%The present paper's finding suggests a lower bound on the reduction in distortionary income taxes: when environmental tax revenues are not redistributed lump sum, some reduction in labor supply via distortionary income taxes is in fact efficient from an environmental policy perspective. %In other words, even if environmental tax revenues suffice to satisfy a government revenue requirement, there is a motive for progressive income taxation. 


More in detail, I start by studying a simple two-sector model with endogenous labor supply.  There are two intermediate sectors of production, one of which induces a negative environmental externality. The environmental externality is the only distortion motivating government action. Once 
labor supply is elastic, reductive policy measures optimally comple- ment the environmental tax. 
The literature has established that, absent any other distortion, an environmental tax equal to the social cost of the externality implements the efficient allocation. 
%Environmental taxes are perceived as a cost-effective way to reduce emissions. 
I demonstrate that this result crucially depends on the use of lump-sum transfers of environmental tax revenues. Absent transfers, labor supply remains inefficiently high through an income effect. Distortive labor income taxation emerges as a suitable policy instrument to correct for this inefficiency.


 Building on this insight, I turn to scrutinize the role of labor income taxes in a more realistic, quantitative set-up following \cite{Fried2018ClimateAnalysis}. The model contains the state-of-the-art feature to study dynamic environmental policies: directed technical change.  A final consumption good is produced from energy and non-energy goods. The energy good, in turn, is composed of green and fossil energy. The fossil sector exerts emissions. Imperfectly monopolistic producers of machinery invest in research to increase the productivity of their machines. Machines are used in the intermediate sectors: non-energy, fossil, and green energy.  The model builds on the directed technical change framework developed in \cite{Acemoglu2012TheChange}, where innovation profits from past technology levels within a sector (\textit{within-sector knowledge spillovers}). In addition to their model, returns to research decrease in the number of scientists employed within a sector, and some knowledge spills across sectors (\textit{cross-sectoral knowledge spillovers}).
 
Relative to the study in \cite{Fried2018ClimateAnalysis}, the current paper adds the consideration of an optimal policy. The optimal policy accounts for a gradually declining net emission limit that eventually turns to zero in 2050.
 The government is bound to recycle environmental tax revenues as subsidies to the green sector. In addition, I give the planner the option to tax or subsidize labor through a linear tax scheme.\footnote{I abstract from the classical motives for labor taxation such as inequality or an exogenous condition on government funds. This choice allows to highlight the contribution of this paper to spotlight labor income taxes as an integral part of efficient and feasible climate change mitigation. }
 I solve for the optimal path of carbon and labor taxes using the numerical approach by \cite{Jones1993OptimalGrowth} and \cite{Barrage2019OptimalPolicy}.

I calibrate the model to the US economy in the period from 2015 to 2019. Then, I feed an exogenous emission limit into the model that is based on the path of the global target provided by the \cite{IPCC2022}. The emission limit for the US used in the analysis stipulates a reduction by 84.5\% in net emissions  in 2020 relative to 2019-levels. The value increases to 85.6\% in 2045.\footnote{ These targets are more than twice as big than the reduction prescribed by the Biden administration amounting to 38\% relative to 2019-emissions. Source:  \href{https://www.whitehouse.gov/briefing-room/statements-releases/2021/04/22/fact-sheet-president-biden-sets-2030-greenhouse-gas-pollution-reduction-target-aimed-at-creating-good-paying-union-jobs-and-securing-u-s-leadership-on-clean-energy-technologies/}{https://www.whitehouse.gov/briefing-room/statements-releases/2021/04/22/}, retrieved 14 September 2022.} In 2050, the emission limit further reduces to net zero.
In the quantitative exercise, the planner optimally sets the carbon and the labor tax to maximize welfare.


I find that the optimal policy path is characterized by a positive tax on labor income in all periods. Initially, before the net-zero emission limit becomes binding, the optimal labor income tax is around 6\%. As the emission target falls to net-zero, the labor tax rises to slightly below 9\%. These marginal tax rates make up between 25\% and 38\% of current tax rates of 24\% in the US. Hours worked diminish by up to 4.6\%. The use of the labor income tax to lower work effort allows for a more productive yet more polluting allocation of labor across sectors while meeting emission targets. This is achieved by taxing carbon less relative to a policy regime without labor income tax. 

\begin{comment}
Third, the findings are especially interesting as the provision of the environmental public good and equity have been perceived as competing targets in the literature. When the poor consume more of the polluting good, a corrective tax is regressive \citep{ Fried2018TheGenerations, Sager2019IncomeCurves}. % \textit{Metcalf 2007, Hassett 2009 as  in Fried 2018}. 
%Second and more indirectly, a fossil tax exerts efficiency costs by lowering labor efforts\footnote{\ The reduction in hours worked is per se not inefficient. The reduction in dirty production reduces the marginal product of labor, which might make a reduction efficient. However, when the government seeks to tax labor income using distortionary policy tools, the reduced labor supply diminishes the tax base of the labor tax making it more costly to redistribute.} which again raises the cost for the government to redistribute \citep{Dobkowitz2022}. 
In contrast to this literature, the present paper provides an argument for progressive income taxes under perfect income-risk sharing suggesting a double dividend of redistribution: equity and efficient externality mitigation. %: equity on the one hand and efficiency gains from less labor as part of the environmental policy.

content...
\end{comment}

\input{litt2_impo}

\paragraph{Outline}
The remainder of the paper is structured as follows. Section \ref{sec:mod_an} presents the core model and the analytical results. In Section \ref{sec:model2}, I extend and calibrate the model to a quantitative framework.  Results are discussed in Section \ref{sec:res}. Section \ref{sec:con} concludes.

%The remainder of the paper is structured as follows. The next section \ref{sec:mod_an} presents a tractable model which is used to derive the analytical results in section \ref{sec:theory}. In section \ref{sec:model}, I extend the model to a quantitative framework and calibrate it. I present and discuss the quantitative results in section \ref{sec:res}. Section \ref{sec:con} concludes.