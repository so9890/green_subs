\clearpage
\setcounter{page}{1}
\section{Introduction}
%\tr{I show that carbon taxes are only efficient if lump-sum transfers are available.}

\begin{comment}
\tr{Think about:
	1) when labor income taxes are not used, then need to have  a higher environmental tax to meet emission limits? \ar Yes, because of advantageous level effect which outweighs recomposing effect of income tax.
	2) When staying at level optimal under the assumption of lump-sum redistribution, but then not redistributing, than absent labor income tax emissions are too high; by how much? Counterfactual}
	
	content...
	\end{comment}

The latest assessment report of the Intergovernmental Panel on Climate Change \citep{IPCC2022} highlights the urgency to reduce greenhouse gas emissions,%relative to the previous report from 2018 \citep{Rogelj2018MitigationDevelopment.}.
\footnote{ \  The report stresses the decreasing likelihood of meeting the Paris Agreement and limiting climate warming to 1.5°. The Paris Agreement of 2015 formulates clear political goals to mitigate climate change. Under this treaty, states have agreed on a legally binding maximum increase in temperature to well below 2°C, preferably 1.5° over pre-industrial levels, and the global community seeks to be climate-neutral in 2050  (see: \url{https://unfccc.int/process-and-meetings/the-paris-agreement/the-paris-agreement}). 
}
and some scholars have pointed to limiting consumption to handle environmental boundaries.\footnote{\ \cite{Schor2005SustainableReductionb} argues for the necessity to limit consumption in the global North through a reduction in working time. \cite{Arrow2004AreMuch} raise the question if today's consumption is too high from a sustainability perspective. \cite{Dasgupta2021}  argues for the impossibility of indefinite growth due to planetary boundaries  \citep{Rockstrom2009AHumanity}. %: acknowledging planetary boundaries, i.e., boundaries which define a state of nature in which humans can safely exist \citep{Rockstrom2009AHumanity}, and that production and consumption produce waste, infinite production would degrade nature in a way that production is impossible.
	 \cite{VanVuuren2018AlternativeTechnologies} study alternative mitigation pathways with lower demand %such as lower energy demand, lower appliance ownership, and meat consumption 
	 in an integrated assessment model motivated by seeking to reduce reliance on carbon capture and storage technologies which entail risks and compete for scarce land. \cite{Bertram2018TargetedScenarios} stress the importance to reduce demand for energy- and material-intense products to alleviate the trade-off between mitigating temperature rises  and the UN sustainability goals%(such as food security, biodiversity protection, and clean water)% (p.11: Shifting towards healthier diets and less energy-and material-intensive consumption patterns appearsto have greatest potential for reducing sustainabilityrisks along a wide range of dimensions)
	. } A reduction in work effort and consumption mitigates pollution by diminishing economic activity. Distortionary fiscal policies qualify as a reductive policy instrument to target the level of production.
However, the literature on environmental policy has focused on compositional policies: environmental taxes. %\citep{Fried2018ClimateAnalysis}. 
Given the exigency to act, this paper addresses the question whether fiscal policies can help meet climate targets. %Using analytical and quantitative methods, I show that reductive policies form part of the optimal environmental policy even absent an additional target.


%This is your core argument for why reductive policy measures may work, so you should mention above that you consider this possibility,  MACHE ICH DAS NICHT MIT DER rESEARCH QUESTION? suggested by its proponents (your "scholars" :-)), and actually show that it works.

In the first part of the paper, I show analytically that once 
labor supply is elastic, reductive policy measures optimally complement the environmental tax. 
The literature has established that, absent any other distortion, an environmental tax equal to the social cost of the externality implements the efficient allocation. 
%Environmental taxes are perceived as a cost-effective way to reduce emissions. 
I demonstrate that this result crucially depends on the use of lump-sum transfers to redistribute environmental tax revenues. Transfers reduce labor supply through an income effect. %Thus, indeed there is a role for reductive policy measures. 
%\textcolor{blue}{This is interesting independent of whether they are feasible or not. Could relate to the fact that there is a discussion how to use revenues. Yet, one might argue that we are always in a setting with distortionary labor income taxes; so that recycling lump-sum is never needed; numbers on size of expected revenues and government spending}
When environmental tax revenues are not redistributed lump sum, environmental taxes are optimally combined with progressive labor income taxes. The use of income taxes as a reductive policy measure is not directly targeted at the externality: the motive for labor taxation emerges from a distortion in labor markets as households feel poorer than the economy is.\footnote{\ This is a novel motive for the use of reductive policies adding to the arguments made in the literature listed in the previous footnote. These are: conflicts with other goals such as the UN sustainability goals, risks associated with carbon capture and storage technology, and planetary boundaries and limits to growth.} %Hence,  % to lower inefficiently high hours worked. 
% I show that redistributing environmental tax revenues through an income tax scheme allows to implement the efficient allocation. The optimal income tax scheme is progressive.
%the optimal environmental policy equalizes the distribution of income as  a side effect.
% The theoretical analysis forms the

In the second part, I scrutinize whether progressive income taxes remain optimal in an endogenous growth model with heterogeneous skills. The government cannot use lump-sum transfers but consumes environmental tax revenues.
 %In the spirit of \cite{Acemoglu2002DirectedChange}, directed technical change may intensify or mitigate these channels thr recomposition. %Second, an overall reduction in labor supply curbing production may lower general research incentives.
 % % more low skill supply, more fossil innovation, more fossil production, and higher low income \ar reduction in the wage premium! 
 The model suggests that the optimal income tax scheme is progressive. The benefits of labor taxation emerge from (i) more leisure and (ii) gains from knowledge spillovers.
 The latter advantage arises as a fossil tax directs research from the non-energy sector to the energy sector.  Energy and non-energy goods are complements in final good production. The literature on directed technical change has shown that when goods are sufficiently complementary a price effect dominates the direction of research \citep{Acemoglu2002DirectedChange, Acemoglu2012TheChange, Hemous2021DirectedEconomics}. Therefore, as the fossil tax makes energy more expensive, the higher price of energy goods pulls research to the energy sector.
 This effect of the environmental tax counters the intention to lower emissions and decreases knowledge spillovers from the non-energy sector. Knowledge spillovers from the non-energy sector, however, are especially valuable, as it is the biggest sector in terms of research processes. 

% 
% On the one hand, a skill bias documented for the green sector \citep{Consoli2016DoCapital} in combination with a relatively more elastic high-skill labor supply causes a higher tax progressivity to recompose the economic structure towards dirty production. On the other hand, a higher labor share in the fossil sector implies a recomposition of the economy towards green production.
%  The skill-recomposition channel dominates and is slightly amplified by a market size effect directing research towards the fossil sector. 
 
% labor income taxes are used to substitute for environmental taxes to realize the gains from knowledge spillovers. 
 
%\textit{I quantify the welfare gains of setting progressive income taxes to equal yyy in consumption equivalent measure. TO BE DONE  }

% relation to literature
I discuss briefly the most important contributions of the paper.
First, the results are relevant for the political and academic debate on how  to recycle environmental tax revenues. The paper points to the importance of lump-sum transfers within the optimal environmental policy as a reductive policy tool; an aspect which appears overlooked in the discussion.%\footnote{\ POLICY debate; \cite{Fried2018TheGenerations}}
When thinking about how to recycle environmental tax revenues other than by lump-sum transfers,  one should take into account alternative reductive tools such as progressive labor income taxes. 
If the reductive part of the environmental policy is neglected, environmental taxes have to be higher to meet emission limits, as I demonstrate in the quantitative exercise.

Second, the results contribute to the academic debate on the so-called \textit{weak double dividend} \citep[for example:][]{LansBovenberg1994EnvironmentalTaxation, LansBovenberg1996OptimalAnalyses}. The hypothesis posits that recycling environmental tax revenues to reduce preexisting tax distortions is advantageous to recycling  revenues as lump-sum transfers. The rationale is that transfers decrease labor supply thereby diminishing the tax base of the income tax. %A conflict between generating government funds and environmental protection arises. 
The findings in the present paper suggest a lower bound on the reduction in distortionary income taxes: when environmental tax revenues are not redistributed lump sum, some reduction in labor supply via distortionary income taxes is in fact efficient from an environmental policy perspective. %In other words, even if environmental tax revenues suffice to satisfy a government revenue requirement, there is a motive for progressive income taxation. 

Third, the findings are especially interesting as the provision of the environmental public good and equity have been perceived as competing targets in the literature. When the poor consume more of the polluting good, a corrective tax is regressive \citep{ Fried2018TheGenerations, Sager2019IncomeCurves}. % \textit{Metcalf 2007, Hassett 2009 as  in Fried 2018}. 
%Second and more indirectly, a fossil tax exerts efficiency costs by lowering labor efforts\footnote{\ The reduction in hours worked is per se not inefficient. The reduction in dirty production reduces the marginal product of labor, which might make a reduction efficient. However, when the government seeks to tax labor income using distortionary policy tools, the reduced labor supply diminishes the tax base of the labor tax making it more costly to redistribute.} which again raises the cost for the government to redistribute \citep{Dobkowitz2022}. 
In contrast to this literature, the present paper provides an argument for progressive income taxes under perfect income-risk sharing suggesting a double dividend of redistribution: equity and efficient externality mitigation. %: equity on the one hand and efficiency gains from less labor as part of the environmental policy.


\paragraph{Literature}


%Second, it connects to the literature connecting environmental and fiscal policies and how to recycle environmental tax revenues. Third, as the paper combines environmental and fiscal policies it naturally connects to the public finance literature. Finally, the results speak to the literature discussing inefficiently high production.
 
%\begin{itemize}
%	\item How to use environmental tax revenues \citep{Fried2018TheGenerations}
%	\item Optimal environmental policy \ar focuses on environmental taxes
%	\item weak double dividend
%	\item to be incorporated: \tr{\cite{Metcalf2003EnvironmentalPollution} why does he find that the optimal pigou tax equals first best when gov spending is satisfied with tax revenues? }
%	\\
%	Williams III: Welfare improvement with xxx \citep{Parry1999WhenMarkets} \tr{is this weak or strong dd?}
%\end{itemize}

The paper relates to four strands of literature. 
%---------------------------------------
%.. optimal environmental policy
%---------------------------------------
Firstly, the paper speaks to the literature on macroeconomic studies of environmental policies. Within this realm, the quantitative analysis connects in particular to the endogenous growth literature. 
In general, these papers focus on environmental taxation and analyze settings with inelastic labor supply so that there is no role for policies targeting the level of production. \cite{Golosov2014OptimalEquilibrium} investigate the optimal carbon tax in a dynamic stochastic general equilibrium model.  
\cite{Acemoglu2012TheChange} discuss with a tractable model of directed technical change limits and possibilities for growth. %They highlight the need for green research subsidies to foster green innovation in combination with carbon taxes to correct for the dynamic spillovers of green innovation not internalized by the research sector.
\cite{Fried2018ClimateAnalysis} extends the framework of the aforementioned paper to a quantitative model. My paper add to the latter an optimal dynamic policy analysis and elastic labor supply. % mainly by introducing cross-sectoral knowledge spillovers and diminishing returns to research. 
%She finds that emission limits can be met at a lower fossil tax when growth is endogenous. 


% OVERVIEW LITERATURE
% Acemoglu 2016 have lump-sum transfers and taxes
% Acemoglu Aghion 2012: lump-sum transfers, no optimal policy
%Golosov: hightlight the need of lump-sum transfers! but exogenous labor supply
% Therefore, the main finding of the present paper, the necessity of reductive policy measures to implement the efficient allocation, complements this literature. 
%Especially, when environmental tax revenues are not redistributed lump-sum in these papers, a variable labor supply would give an argument for labor income taxation. 


%\paragraph{Endogenous growth, elastic labor supply and optimal environmental policy}

\begin{comment}
3/09:	ADD WHEN INTRODUCING HETEROGENEOUS LABOR SUPPLY BAG IN
Secondly, staying within the field of endogenous growth, the paper connects to work examining the interaction of directed technical change and skill heterogeneity. \cite{Acemoglu2002DirectedChange} develops a theory to explain the positive correlation of skill supply and the skill premium: the higher supply of skilled labor raises incentives to innovate in the skill sector. \cite{Loebbing2019NationalChange} introduces fiscal policy into the model to investigate how the equalizing effect of redistribution is amplified through directed technical change.
My paper contributes to these two branches by integrating endogenous and heterogeneous skill supply in an environmental model of directed technical change. These ingredients enable me to analyze labor income taxes through the lens of environmental policies.  

content...
\end{comment}

 %Similar to my paper, a higher tax progressivity changes the relative supply of skills. As low-skill labor is in relative higher supply, low-skill-specific innovation depresses the wage distribution thereby contributing to equity. While the channels are comparable,
% I evaluate the effect of income tax progressivity and endogenous growth on emissions. 
%\cite{Hemous2021DirectedEconomics} provide an overview of models of directed technical change in environmental economics. They argue that a rise in the skill ratio directs innovation towards skill-intense technology when the high- and the low-skill output good are sufficient substitutes. Furthermore, when the two input goods are substitutes, the more advanced sector attracts more innovation. 


%\cite{Oueslati2002EnvironmentalSupply} studies the optimal environmental policy with elastic labor supply and endogenous growth. Yet, he allows for lump-sum transfers of environmental revenues. \textit{He should find something on reduction of hours}: No: capital is the only polluting factor, and labor is the clean factor of production.


%%%--------------------------------------------------------------------
% How to recycle environmental tax revenues: weak double dividend
%%%-------------------------------------------------------------------- 
\
% A big literature has examined potential benefits arising from corrective tax revenues to ameliorate fiscal distortions. The double-dividend literature is concerned with fiscal advantages arising from environmental tax revenues. My results speak directly to the weak double dividen hypothesis which 
%My paper most closely relates to the literature on the weak double-dividend
%\paragraph{Recycling environmental tax revenues}
%\tr{Read:\cite{Freire-Gonzalez2018EnvironmentalReview} yet on strong dd, I assume}
Secondly, this paper is not the first to integrate distortionary fiscal policies into the analysis of environmental policies. The literature discussing how to use environmental tax revenues generally assumes labor supply to be elastic and incorporates fiscal policies. 
The dominant focus of this literature is the weak double dividend of environmental taxes \citep[for instance,][]{Goulder1995EnvironmentalGuide, Bovenberg2002EnvironmentalRegulation, Barrage2019OptimalPolicy}: given an exogenous government funding constraint it is cost saving to recycle environmental tax revenues to lower distortionary labor income taxes as opposed to higher lump-sum transfers. The latter, so the rationale, decreases labor supply through an income effect thereby lowering the tax base of the labor income tax. %Consequently, it becomes more expensive for the government to generate revenues.
Therefore, this literature advocates recycling environmental tax revenues to reduce distortionary fiscal taxes as opposed to lump-sum rebates.
%With its quantitative part, my paper closely relates to \cite{Barrage2019OptimalPolicy} who examines the role of fiscal distortions emerging from an exogenous revenue constraint on the environmental policy in a quantitative framework. She also optimizes jointly over fiscal and environmental policy instruments, but her focus rests on the deviation of the optimal environmental tax from the social costs of carbon.

% my contribution : 1) lower bound on dist income tax; 2) motivation and role of income taxes
Relative to this literature, my paper's contribution is to discuss the existence of an upper bound on the reduction of distortionary income taxes: From an environmental policy perspective, some reduction in labor supply is in fact efficient. However, shrinking labor supply is generally perceived as an inefficiency in this literature and environmental taxes on its own as efficient. The importance of lump-sum transfers to implement the efficient allocation receives less attention.  %To the best of my knowledge, the papers theoretically discussing the weak double-dividend \citep{LansBovenberg1996OptimalAnalyses, Goulder1995EnvironmentalGuide} do not formally derive the result; a possible explanation for why the lower bound on distortionary tax reduction remained unnoticed. 
% The primary distinction of this paper and the weak double-dividend literature is the motive for income taxation. In this literature an exogenous funding constraint motivates the use of distortionary income taxes. In contrast, my model rationalizes a progressive income tax absent an exogenous revenue constraint arising in an otherwise equal set-up from the environmental externality per se as the environmental tax on its own does not establish the efficient allocation. 




 %This becomes clear when environmental tax revenues suffice to meet the government's funding constraint, then labor supply would be inefficiently high when the labor income tax is unused. 
%In contrast to the present paper, the double-dividend literature focuses on non-environmental cost advantages of environmental taxation either via interactions with other taxes and their bases or via their revenues. However, it remains unmentioned that under the assumption of elastic labor supply, which the literature necessarily assumes, the environmental tax alone is not efficient.

%----------------------------------------
%---- optimal revenue recycling 
%---- empirical and quantitative-------
%----------------------------------------
The question of how to use environmental tax revenues has seen a surge in interest recently and diverged from the prominence of fiscal advantages. 
 %They do not constitute a free lunch if efficiency is the goal.\footnote{\ Often, environmental taxes alone seem to be perceived as being able to implement the efficient allocation: \cite{LansBovenberg1999GreenGuide} writes "\textit{Environmental taxes are  generally  an  efficient  instrument  for  protecting  the  environment.}" thereby neglecting the role environmental tax revenue redistribution. Or "\textit{Establishing a price on carbon [...] is well understood to be the most efficient approach for reducing greenhouse gas emissions.}" \citep{Fried2018TheGenerations}. } 
The pros and cons of different recycling means is often assessed using inter- and within generational equity or political feasibility as value measures \citep{Carattini2018, Goulder2019IncomeGroups, VANDERPLOEG2022103966, Kotlikoff2021MakingWin, Carbone2013DeficitImpacts}. Building on the weak double-dividend literature, \cite{Fried2018TheGenerations} compare distinct recycling scenarios investigating the impact on inequality in an overlapping generations model. 
% They find lump-sum transfers to be preferred by the  living generation.
Using German data,   \cite{VANDERPLOEG2022103966} find an equity advantage of lump-sum transfers. % The authors suggest the government to split environmental tax revenues to both lower preexisting tax distortions and as lump-sum transfers. %The present paper employs the first-best allocation as a benchmark to assess distinct recycling methods.
My contribution to this debate is to point to lump-sum redistribution to constitute an integral part of an efficient pollution mitigation. % together with corrective taxes.
If carbon tax revenues are not rebated lump sum, a role for additional policy intervention emerges due to distortions in the labor market.

%\paragraph{Public finance}
Thirdly, the paper contributes to the public finance literature.
An equity-efficiency trade-off is central to this literature.  The benefits of labor taxes and progressivity arise, inter alia, from redistribution. %and from generating government revenues. 
%With concave utility specifications full redistribution is efficient. However, the optimal tax system does not feature full redistribution when labor supply is endogenous. Instead, redistribution is traded off against aggregate output as individuals reduce their labor supply and skill investment in response to labor income taxation 
\citep{Heathcote2017OptimalFramework, Conesa2009TaxingAll, Domeij2004OnTaxes}.
To this literature I add another motive for the use of distortionary fiscal policies: to reduce inefficiently high labor supply in the presence of environmental taxes. 
%One closely related work is \cite{Loebbing2019NationalChange} who studies optimal income taxation in a model of directed technical change. The redistributive effect of tax progressivity is amplified through a compression of the wage rate distribution \textit{to be continued}


Fourthly, the paper relates from its motivation and finding to the discussion on whether production levels are inefficiently high. 
%The finding relates to the literature discussing rationales for the usage of reductive policy measures. 
Other motives for the reduction of consumption arise from
envy \cite{Alvarez-Cuadrado2007EnvyHours}, or a positive externality of leisure \cite{Alesina2005WorkDifferent}. \cite{Arrow2004AreMuch} discuss whether there is a need to reduce consumption levels due to sustainability concerns. 
 The present paper contributes to this literature by identifying another reason for too high labor supply: The externality results from mitigating an environ- mental externality without lump-sum transfers.


\paragraph{Outline}
The remainder of the paper is structured as follows. Section \ref{sec:mod_an} presents the core model and the analytical results. In Section \ref{sec:model2}, I extend and calibrate the model to a quantitative framework.  Results are discussed in Section \ref{sec:res}. Section \ref{sec:con} concludes.

%The remainder of the paper is structured as follows. The next section \ref{sec:mod_an} presents a tractable model which is used to derive the analytical results in section \ref{sec:theory}. In section \ref{sec:model}, I extend the model to a quantitative framework and calibrate it. I present and discuss the quantitative results in section \ref{sec:res}. Section \ref{sec:con} concludes.