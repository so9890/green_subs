%\tableofcontents
\section{Quantitative results}\label{sec:res}

In this section, I present and discuss the quantitative results. Under the benchmark policy regime, the government uses carbon tax revenues to subsidize the green sector. Furthermore, a labor income tax is available. 
First, it is informative to consider how a social planner meets the emission limit. Relative to this benchmark, I discuss the optimal allocations a government implements with the ``politically feasible" policy with and without labor income tax in Section \ref{sec:allos}. Section \ref{sec:optpol} discusses the optimal policy.  % Add: comparison to non-exogenous growth => differences in labor tax

%This section depicts results on the optimal policy followed by the implied allocation in the benchmark model where environmental tax revenues are redistributed via the income tax scheme. 

%\begin{itemize}
%	\item emission limit can be obtained at a smaller carbon tax. the economy profits from a more productive ratio of green to fossil energy 
%	\item effect on green technology growth 
%\end{itemize}

\subsection{Efficient and optimal allocations}\label{sec:allos}

\autoref{fig:allo} depicts the efficient, or first-best, allocation, in dashed-gray, the optimal allocation with and without labor taxes, the solid-black and the dashed graph, and the laissez-faire allocation, dotted-black. The x-axis indicates the first year of the 5 year period to which the variable value corresponds. 
To attain the emission limit, the efficient allocation sees a massive increase in the energy ratio from fossil to green energy from close to zero under laissez-faire to around 70 to 1 in the 2070s; see the dashed-gray graph in \autoref{fig:allo} Panel (a). This shift comes at the cost of less consumption relative to a laissez-faire scenario; see Panel (b). The transition is characterized not only by an immediate reduction in consumption but also by a smaller growth rate of consumption. The reason is that the green sector is less productive, but also that a shift to green research entails smaller growth rates due to dynamic knowledge spillovers which make fossil research more productive. 
The reduction in productivity makes it efficient to lower work effort. Even though consumption reduces and the marginal unit of output becomes more valuable\textemdash similar to an income effect\textemdash the social planner prefers lower work effort. The reason is that the marginal product of labor decreases so much\textemdash similar to a substitution effect. Overall, the substitution effect dominates and less work effort becomes desirable. 

The optimal allocation without labor tax, the dashed-black graph, features a strong reduction in the use of fossil fuels. Even more than in the efficient allocation. Despite this less productive use of technologies, the reduction in consumption is mitigated. The reason is that households work more hours than under the social planner allocation. Hence, labor supply is inefficiently high given the reduction in the marginal product of labor.  % As the green sector becomes more productive over time, labor supply rises. 

Allowing for labor income taxes, the government can correct for the inefficiency in labor supply. Panel (c) visualizes how hours move closer to their efficient level when a labor income tax is available; the solid-black graph. Because of the smaller work effort, the level of production declines which allows for a smaller green to fossil energy ratio, hence a more productive allocation of resources, while meeting the emission limit; compare Panel (a). This mitigates the reduction in output as hours decline. 


\begin{figure}[h!!]
	\centering
	\caption{Efficient and Optimal Allocation }\label{fig:allo}
	\begin{minipage}[]{0.45\textwidth}
		\centering{{(a) Green to Fossil Energy}}
		%	\captionsetup{width=.45\linewidth}
		\includegraphics[width=1\textwidth]{../../codding_model/own_Paper/optimalPol_140723_revision/figures/all_30Aug23/Levels_eff2pol_LF_pol2_T_GFF_emnet0_subsres0_knspil0_sigma0_Bop0_util0_lgd1.png}
	\end{minipage}
%	\begin{minipage}[]{0.05\textwidth}
%		\
%	\end{minipage}

\vspace{4mm}
\begin{minipage}[]{0.45\textwidth}
\centering{{(b) Consumption }}
%	\captionsetup{width=.45\linewidth}
\includegraphics[width=1\textwidth]{../../codding_model/own_Paper/optimalPol_140723_revision/figures/all_30Aug23/Levels_eff2pol_LF_pol2_T_C_emnet0_subsres0_knspil0_sigma0_Bop0_util0_lgd0.png}
\end{minipage}

\vspace{4mm}
	\begin{minipage}[]{0.45\textwidth}
		\centering{{(c) Hours Worked }}
		%	\captionsetup{width=.45\linewidth}
		\includegraphics[width=1\textwidth]{../../codding_model/own_Paper/optimalPol_140723_revision/figures/all_30Aug23/Levels_eff2pol_LF_pol2_T_H_emnet0_subsres0_knspil0_sigma0_Bop0_util0_lgd0.png}
	\end{minipage}
%\begin{minipage}[]{0.05\textwidth}
%\
%\end{minipage}
%\begin{minipage}[]{0.45\textwidth}
%	\centering{{(d) Green Technology }}
%	%	\captionsetup{width=.45\linewidth}
%	\includegraphics[width=1\textwidth]{../../codding_model/own_Paper/optimalPol_140723_revision/figures/all_30Aug23/Levels_eff2pol_LF_pol2_T_Ag_emnet0_subsres0_knspil0_sigma0_Bop0_util0_lgd0.png}
%\end{minipage}
\end{figure} 
\newpage

\subsection{Optimal policy}\label{sec:optpol}

What policy implements the optimal allocation? \autoref{fig:optPol} shows the optimal policy in the regime with and without labor income tax. Satisfying the emission limit necessitates a massive tax on carbon which reaches levels close to 4000 US\$ in the 2070s. The jump in 2050 follows from the more stringent net-zero emission limit. Knowledge spillovers from non-fossil to the fossil sector explain the rising pattern of carbon taxes. Knowledge from green and non-energy research makes fossil researches more productive. Market forces lead to a reallocation of researchers to this sector. Overall, a higher carbon tax has to counter this effect. 

The optimal policy is accompanied by a positive tax on labor. The marginal tax on labor is around 6\% before 2050 and jumps to slightly below 9\% in 2050. It gradually reduces to around 7.5\%. These tax rates make up between 25\% to 38\% of the calibrated tax rate of 24\%.  Given the choice of parameters, the substitution effect dominates the income effect and households reduce their labor supply as labor is taxed. In contrast to the policy regime without labor tax, the planner has to set a higher carbon tax especially under the net-zero emission target: The higher level of production requires to meet the emission limit at a higher share of green to fossil energy usage. 

\begin{figure}[h!!]
	\centering
	\caption{Optimal Policy }\label{fig:optPol}
	\begin{minipage}[]{0.45\textwidth}
		\centering{{(a) Tax per ton of CO$_2$ in 2022 US\$\\ \ }}
		%	\captionsetup{width=.45\linewidth}
		\includegraphics[width=1\textwidth]{../../codding_model/own_Paper/optimalPol_140723_revision/figures/all_30Aug23/Optimal_comp_pol2_T_Tauf_emnet0_subsres0_knspil0_sigma0_Bop0_util0_lgd1.png}
	\end{minipage}
	\begin{minipage}[]{0.05\textwidth}
		\
	\end{minipage}
	\begin{minipage}[]{0.45\textwidth}
		\centering{{(b) Labor Income Tax \\ \ }}
		%	\captionsetup{width=.45\linewidth}
		\includegraphics[width=1\textwidth]{../../codding_model/own_Paper/optimalPol_140723_revision/figures/all_30Aug23/Optimal_comp_pol2_T_taul_emnet0_subsres0_knspil0_sigma0_Bop0_util0_lgd0.png}
	\end{minipage}
%	\begin{minipage}[]{0.32\textwidth}
%		\centering{\footnotesize{(b) Labor Tax }}
%		%	\captionsetup{width=.45\linewidth}
%		\includegraphics[width=1\textwidth]{../../codding_model/own_Paper/optimalPol_140723_revision/figures/all_30Aug23/Optimal_comp_pol2_T_taus_emnet0_subsres0_knspil0_sigma0_Bop0_util0_lgd0.png}
%	\end{minipage}
	%\begin{minipage}[]{0.32\textwidth}
	%	\centering{\footnotesize{(c) Net emissions\\ \  }}
	%	%	\captionsetup{width=.45\linewidth}
	%	\includegraphics[width=1\textwidth]{../../codding_model/own_basedOnFried/optimalPol_010922_revision/figures/all_13Sept22_Tplus30/Single_OPT_T_NoTaus_Emnet_regime0_spillover0_knspil0_noskill0_sep0_xgrowth0_extern0_PV1_sizeequ0_GOV0_etaa0.79.png}
	%\end{minipage}
\end{figure} 

\begin{comment}
\begin{figure}[h!!]
	\centering
	\caption{Optimal Policy }\label{fig:optPol_LStrans_Subs}
	\begin{minipage}[]{0.45\textwidth}
		\centering{\footnotesize{(a) Carbon Tax }}
		%	\captionsetup{width=.45\linewidth}
		\includegraphics[width=1\textwidth]{../../codding_model/own_Paper/optimalPol_140723_revision/figures/all_30Aug23/Comp_LSTrans_Subs_T_Tauf_emnet0_subsres0_knspil0_sigma0_Bop0_util0_lgd1.png}
	\end{minipage}
%	\begin{minipage}[]{0.1\textwidth}
%		\
%	\end{minipage}
	\begin{minipage}[]{0.45\textwidth}
		\centering{\footnotesize{(b) Labor Tax }}
		%	\captionsetup{width=.45\linewidth}
		\includegraphics[width=1\textwidth]{../../codding_model/own_Paper/optimalPol_140723_revision/figures/all_30Aug23/Comp_LSTrans_Subs_T_taul_emnet0_subsres0_knspil0_sigma0_Bop0_util0_lgd0.png}
	\end{minipage}
\end{figure} 
\end{comment}
	
