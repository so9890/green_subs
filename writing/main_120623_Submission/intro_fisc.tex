
\tableofcontents
\section{Introduction}
An equity-efficiency trade-off is central to the discussion of optimal labour income taxation and tax progressivity in the public finance literature.  The benefits of labour taxes and progressivity arise from redistribution. With concave utility specifications full redistribution is efficient. However, the optimal tax system does not feature full redistribution when labour supply is endogenous. Instead, redistribution is traded off against aggregate output as individuals reduce their labour supply and skill investment in response to labour income taxation. 

%\paragraph{An environmental perspective}

 Adding environmental externalities to the classical public finance framework changes the perception of efficiency costs. Instead of merely reducing welfare, direct benefits through a reduction of the externality arise by lowering output. 
Yet, in general, in the presence of corrective tax instruments and the assumption of a clean production alternative, %which direct production to a non-polluting alternative, however,
distortionary labour taxes are not employed as environmental policy instruments.\footnote{ NOTE: e.g Compare my paper on sustainable production and demand where there is a clean alternative available and no environmental disaster.} 
However, technological progress might not suffice to meet climate targets while leaving per capita consumption (growth) unchanged. This is especially relevant given the tight time frame foreseen by the Paris Agreement.\footnote{\ Under this treaty, states have agreed on limiting temperature rise to well below 2°C, preferably 1.5°C and to achieve climate neutrality by mid-century \url{https://unfccc.int/process-and-meetings/the-paris-agreement/the-paris-agreement}. } Indeed, in an report by the Intergovernmental Panel on Climate Change (IPCC)\footnote{\ A body of the United Nations established to assess the science related to climate change.},  \cite{Rogelj2018MitigationDevelopment.} advocate additional demand-side measures to lower demand for energy and greenhouse gas intense products (p.97 bottom) to meet agreed-on climate targets.\footnote{\tr{Notes: I am unsure if distortionary taxes would indeed arise as an optimal environmental policy tool when corrective taxes in both sectors are available. Maybe need some other argument, e.g. not available, why?; some political argument; inequality might be another if poor households consume a bigger polluting share; Changing already established policy measures less difficult than introducing new taxes?...I would make this framing dependent on the findings. If I focus on tax progressivity:  progressivity could lower consumption of consumption-rich households in particular which could contain some sort of advantage. A general consumption tax might serve as a baseline corrective tax and the explicit corrective tax accounts for the gap between the vat consumption tax and the social cost of the dirtier sector. }}

%\paragraph{Consumption reduction (degrowth!)}
% 
% Growth drag may result in papers which study the trade-off between consumption growth and the environment \citep{Acemoglu2012TheChange, Stokey1998AreGrowth, Jones2016LifeGrowth}. %\citep{Rockstrom2009AHumanity}\footnote{\ In their article, \cite{Rockstrom2009AHumanity} argue that several planetary boundaries have already been surpassed. Uncertainties about their interaction are another argument in favour of more cautious environmental policies. } such as biodiversity and nitrogen xxx.
%Taken this broader view on environmental pollution through economic activity, it becomes more questionable, if a perfectly clean technology exists or can ever be innovated.  

%In sum, the need to reduce environmental externalities rather quickly and the potential non-existence of perfectly green technologies could make it optimal to use classical fiscal tax instruments for environmental concerns through the efficiency channel. 

%\textit{So far: efficiency-equity channel; plus on efficiency side}
%\paragraph{Trade-offs: environmental externality}
 Consumption reduction in affluent countries has also been promoted by some (non-economic) scholars \citep{Schor2005SustainableReduction, Pullinger2014WorkingDesign, Arrow2004AreMuch}. But, the general equilibrium effects are less well understood.
While having an advantageous direct effect on the externality, counteracting indirect effects arise in a general equilibrium framework. Proponents of a reduction policy especially focus on consumption by the rich which consume a higher amount of natural resources.\footnote{\ Note Sonja: Abstracting from inequality, would it still be best to reduce consumption by the rich when the poor have a higher marginal propensity to consume dirty? \textit{Could be an important aspect in the model}. Not in the baseline, look at it in an extension...}
This concern could add to the benefits of tax progressivity.
However, targeting these households in particular for environmental reasons, in contrast, will lower the supply of high skilled labour.\footnote{\ The relation of labour tax progressivity and skill investment has been studied by \cite{Heathcote2017OptimalFramework}.} Yet, these skills are essentially important in greener sectors of the economy \citep{Consoli2016DoCapital}. As a result, dirty production becomes relatively cheaper and the dirty share of production rises. 

%The literature on optimal environmental policy knows the 
Endogenising the direction of technological progress might add to the adverse negative effects of a reduction policy:
Innovations might shift towards the more polluting good in response to a higher progressivity. The reduction in the cleaner sector's labour input good diminishes the relative market size of this sector. The market effect shifts innovation towards the bigger dirty sector. On the other hand, the higher price of cleaner products makes innovation in this sector more profitable, the price effect. 
One contribution of the paper is to discuss optimal fiscal policy in an endogenous growth setting. 


%\paragraph{Research question}
Given these counteracting forces, this paper seeks to answer whether a more progressive labour tax is indeed optimal for environmental reasons. 
%\tr{Why not use one corrective tax in each sector? Political argument? }

%\paragraph{Empirical motivation}
The key channel inducing adverse consequences of a reduction policy on the externality hinges on the skill-bias in the cleaner sector \textbf{and in the innovation sector}. % and that consumption-rich households are high skill households. \ar that they are richer is an equilibrium effect 
For the US, \cite{Consoli2016DoCapital} find that within broader occupational groups greener occupations are skill-biased: on average, the degree of non-routine tasks is higher as are formal education, work experience, and on-the-job training. I link this evidence to the type of skills employed in the dirty and the cleaner sector in the model. %\ar might be able to abstract from CRRA utility function and to focus on the skill accumulation dimension...
% \cite{Bowen2018CharacterisingComposition} estimate a relatively low difference in skills between jobs and argue that this gap can be closed quickly by on-the-job trainings. 
\cite{Borissov2019CarbonDevelopment}
build an endogeneous skill accumulation model incorporating the skill-bias in the green sector. They motivate their study by citing policy recommendation papers and \cite{Vona2018EnvironmentalExploration}: "\textit{green skills are closely related to the design, production, management and monitoring of technology and conclude that education emerges as a critical ingredient in the policy mix to promote sustainable economic growth}". Hence, skills are relevant in two levels for cleaner production: (1) for the innovation process of clean technologies (development), and (2) their management (operation). (\cite{Fried2018ClimateAnalysis} endogenises researchers!); \ar when introducing technological change reduction policies might become especially important. 

%\paragraph{Model}
I study the effects of labour-tax progressivity  on the environmental externality in a tractable general equilibrium model with directed technical change \tr{(might drop endogenous growth from the baseline model depending on findings, see Roadmap in section \ref{sec:rm})}. The model builds on \cite{Heathcote2017OptimalFramework} and \cite{Acemoglu2012TheChange}.
In contrast to \cite{Heathcote2017OptimalFramework}, I abstract from idiosyncratic risk and the life-cycle component to focus on the medium to long run and the environment. 
The planner still faces the trade-off between equity and skill accumulation. 

The government chooses tax instruments to maximise a 
social welfare function. But, the choice is constrained by emission targets consistent with the Paris Agreement. The advantage of this approach is that it suffers less from  model misspesifications due to  uncertainties about how emissions affect the environment. Furthermore, it is closely related to the current political debate. Compare appendix section \ref{app:emission_climate_targets} for a more in depth discussion of this aspect.  


An important modelling uncertainty remains: what degree of emission reduction can be achieved by technological progress in the specified time frame? I will give careful attention to this aspect in course of the paper: analytically and quantitatively.\footnote{\ In the quantitative part, I use different specifications of technological possibilities: (i) a scenario where technological progress is sufficient to reduce emissions to zero until 2050 at current consumption levels, (ii) and one where it is not. \textit{ Look at how others estimate innovation steps. Still questionable if can draw conclusion from past innovations to future possibilities of innovations. }}


Three different formulations of the planner's objective function seem interesting: (1) a Utilitarian social welfare function, (2) the minimisation of the variance of reductions in household utility (which is meant to proxy for social tension), and (3) an objective function to minimise the consumption reduction of the poorest in a Rawlsian spirit. \tr{(I will start to work with the Utilitarian version and only add (2) and (3) if necessary, i.e., tax progressivity is unaffected by the environmental constraint.)}


%\paragraph{Exercise}
The paper is divided into two parts: an analytical and a quantitative one.
In the first part, I derive  conditions which shape the relation of production and emissions so that fiscal policies are employed for environmental reasons in presence  of corrective taxes.  
In the second part, I examine the optimal policy and transitions to the new steady state from a realistically calibrated current state of the economy in a quantitative version of the model\footnote{\ For example, as in  \cite{Fried2018ClimateAnalysis} I add additional features to the innovation process such as spillovers across sectors and a third neutral sector of production.}.

% Two possible setups seem interesting. (1) The government maximises a \textit{(possibly pareto-weighted, utilitarian)} social welfare function. This version allows to account for other planetary boundaries potentially interacted with the dominant climate boundary and carbon emissions. (2)
%(\textit{\textbf{Note:}}(3) A reduction policy can also become optimal when climate policies as required to meet the targets are socially unfeasible. For example, poor households consume a higher share of polluting goods, reducing overall consumption through labour taxes, instead, could be more favourable in terms of inequality. )

I perform two quantitative experiments to scrutinise the contribution of directed technological change and skill heterogeneity on the optimal policy. First, I rerun the analysis in versions of the model where either or both channels are shut down.  Second, I introduce the optimal policy of the amended model, e.g. the  version without skill heterogeneity, into the model with skill heterogeneity to learn how this aspect shapes the effect of the optimal policy on the externality. %\footnote{\ \tr{Note: These quantitative exercises }}

%\paragraph{Sensitivity}
First, I adjust the model for different specifications of the skill accumulation process which is essential in shaping the adverse effects of reduction policies on the environment.\footnote{\ For example, skill can be modelled as generating intergenerational spill-overs within households as in \cite{Borissov2019CarbonDevelopment}.} Second, I allow for additional heterogeneity in preferences: some households voluntarily reduce their consumption.\footnote{\ \tr{Compare my other proposal on voluntary reduction.: Suggestive evidence for $\underline{a\ rise}$ in the share of households which voluntarily reduce their consumption (of new products) and working hours. Yet, voluntary reduction seems to be uncorrelated with skills.}} \textit{Heterogeneity in the marginal propensity to consume could affect the role of fiscal policy on the externality potentially reducing its effectiveness.}\footnote{\ Could also study special utility functions which incorporate the arguments put forward by advocates of reduction policies such as: habits, social preferences, or  spill-overs of leisure \citep[][\textit{to be read}]{Alesina2005WorkDifferent}. These aspects imply inefficient high levels of consumption and work effort even absent an environmental externality.  Hence, they will add to the benefits of reducing consumption. 
}
Third, instead of fiscal policies the government has a policy tool to lower the maximum hours worked per worker and period \citep[\textit{compare}][]{Alvarez-Cuadrado2007EnvyHours}. Such a policy is closer to what proponents of a reduction policy suggest. % Third, more evolved modelling of the innovation process as in \cite{Fried2018ClimateAnalysis} is introduced \textit{(add this to the main quantitative part)}. 
\tr{These are some ideas, will decide while working on the paper what extensions to focus on.}
\\

\noindent\rule[1ex]{\textwidth}{1pt}


\paragraph{The below is an addendum in case I want to add capital taxation...} \ 
\\

\noindent
\textit{below relatively close to \cite{Conesa2009TaxingAll}} 
Another prominent debate in the public finance literature centres on the optimal size of the capital tax. The optimality of zero capital taxation has been argued for in the seminal work by Judd (1985) and affirmed by others.  However, as discussed by \cite{Conesa2009TaxingAll}, the optimality of a zero capital tax hinges upon the endogeneity of labour supply. The capital tax rises when allowing for life-cycle aspects. In this setting, it is optimal to tax capital heavily and to accept a reduction in consumption at a modest reduction of hours worked for the benefit of better insurance and redistribution. Taking environmental externalities into account, the optimal capital tax may even be higher due to its advantageous effect on the externality.

\cite{Domeij2004OnTaxes} discuss a reduction of capital taxes in a model with idiosyncratic productivity shocks which imply  wealth inequality. Their focus rests on the distributional effect which arise in the transition from the benchmark to a lower capital tax. They find huge adverse distributional effects over the transition since a linear labour tax is less progressive.  (\textit{check this}) The distributional effects outweigh the benefits of a higher capital stock and output. 

\tr{(I would first want to look at labour tax progressivity as I can build on the tractable model by \cite{Heathcote2017OptimalFramework}.)}
\\

\noindent\rule[1ex]{\textwidth}{1pt}

\paragraph{Literature}
\begin{itemize}
\item Public finance literature on optimal labour tax (progressivity) or capital tax (advantage of labour tax progressivity: there is a tractable model already available; capital tax with the zero capital tax finding seems interesting and important though...But requires a different model with capital) \citep{Heathcote2017OptimalFramework, Conesa2009TaxingAll, Domeij2004OnTaxes}
\item Optimal environmental policy with and without directed technical change \citep{Acemoglu2012TheChange, Acemoglu2016TransitionTechnology, Fried2018ClimateAnalysis, Barrage2019OptimalPolicy, Golosov2014OptimalEquilibrium, Hassler2016EnvironmentalMacroeconomics}
\item potentially: limits to growth \citep{Stokey1998AreGrowth, Jones2016LifeGrowth, Arrow2004AreMuch}
\item building on the following literature: \begin{itemize}
\item skills and green production: empirical \citep{Consoli2016DoCapital, Bowen2018CharacterisingComposition, Borissov2019CarbonDevelopment}; growth model with skill accumulation \citep{Borissov2019CarbonDevelopment}
\item literature on the environment and overconsumption: (i) economics \citep{Dasgupta2021, Brock2005ChapterEmpirics, Arrow2004AreMuch, Cohen2019AnnualSubstitutable}, (ii) natural sciences \citep{ Rockstrom2009AHumanity, Rogelj2018MitigationDevelopment.}
\item literature advocating reduction policies \citep{Schor2005SustainableReduction, Pullinger2014WorkingDesign}
%Some (non-economist) scholars argue for the necessity to reduce current consumption growth in developed economies due to threats to planetary boundaries \citep{Arrow2004AreMuch, Schor2005SustainableReduction}. 

\end{itemize}
\end{itemize}

\section{\tr{Roadmap}}\label{sec:rm}

\paragraph{To do next 21/01/22}
\ \\
\textbf{write 10 page paper with main ingredients: well motivated, simple rep agent model, what is my innovation: tax instrument and targets, stylized results (no focus on calibration)} \ar to hand in to conferences in February
\begin{itemize}
	\item simple simple model: rep agent, externality in form of target
	\item motivate to look at linear taxes/(maybe tax progressivity later)
	\item two sectors with two different labour inputs
	\item think about emission and climate targets! 
	\item there is no corrective tax (what can the government attain with policies that are already well established?)
\end{itemize}
\paragraph{To do sometime }
\begin{itemize}
	\item read papers on consumption tax
	\item read \cite{Acemoglu2002DirectedChange} \ar how is technological progress modeled, how is model applied to empirical examples
	\item how to model skills? What is important empirically? On the job training? or occupational choice?
\end{itemize}
\begin{enumerate}
	\item[0.] where/how do skills become important for a transition to cleaner production? (a) in innovation itself (engineering skills), (b) in using the new technology (managerial skills) \citep{Vona2018EnvironmentalExploration}
\item answer research question in model without directed technical change, analytically and quantitatively. This is basically an extension to \cite{Heathcote2017OptimalFramework} to include (i) two sectors, (ii) a corrective tax, and (iii) a climate target in the planner's objective function. (note that HSV in baseline is a static model)
\\ further questions: 
\begin{itemize}
	\item \tr{are there public finance papers which look at the distribution of skills \textbf{across sectors}?}
\end{itemize}
\item add directed technical change \ar if leaves findings unchanged, write up as extension; if this changes finding importantly (e.g. qualitatively) then include into baseline model
\begin{itemize}
	\item distribution of skills and technological change will interact with each other... 
	\item have to read more about directed technical change and skill bias!
	\item references
\begin{itemize}
	\item Benabou 2002 (\ar no directed technical change) and 2005 \ar skill bias of technological change: technological change captured by the elasticity of substitution of skill factors; \textbf{could argue that when it comes to clean vs dirty production it is about innovations to reduce emissions}; \tr{sector specific technology is key}
	\item \cite{Acemoglu2002DirectedChange}
\end{itemize}
\end{itemize}
\end{enumerate}