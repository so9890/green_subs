\documentclass[12pt]{article}
\usepackage[utf8]{inputenc}
\usepackage{xcolor}
\usepackage{graphicx}
\usepackage{listings}
\usepackage{epstopdf}
\usepackage{etoc}
\usepackage{pdfpages}
\usepackage[capposition=top]{floatrow}
\usepackage{pdflscape} % landsacpe package
% set font to times
%\usepackage{mathptmx} % times!!! 
%\usepackage[T1]{fontenc}
\usepackage{amsmath}
\usepackage{amsthm}
\usepackage{soul}
\usepackage[left=2.5cm, right=2.5cm, top=2.5cm, bottom =2.5cm]{geometry}
\usepackage{natbib}
%\usepackage[natbibapa]{apacite}
%\usepackage{apacite}
%\bibliographystyle{apacite}
\bibliographystyle{apa}
%\renewcommand{\footnotesize}{\fontsize{10pt}{11pt}\selectfont}
\usepackage[onehalfspacing]{setspace}
\usepackage{listings}
\renewcommand{\figurename}{\textbf{Figure}}
\renewcommand{\hat}{\widehat}
\usepackage[bf]{caption}
\usepackage{tikz}
%\begin{comment}
%\usepackage[headsepline,footsepline]{scrlayer-scrpage} % has to come before package!!! otherwise option clash
%\usepackage{scrlayer-scrpage}
%\pagestyle{scrheadings} % kopfzeile/ fußzeile
%\clearpairofpagestyles
%\ohead{}
%\ihead{\textit{Redistribution, Demand and  Sustainable Production}}
%\cfoot{\thepage}
%\pagestyle{plain} % comment this one to have header
%\end{comment}
\allowdisplaybreaks
\usepackage{comment}
\usepackage{siunitx}
\usepackage{textcomp}
\definecolor{sonja}{cmyk}{0.9,0,0.3,0}
%\definecolor{purple}{model}{color-spec}
\usepackage{amssymb}
\newcommand{\ar}{$\Rightarrow$ \ }
\newcommand{\frp}[2]{\frac{\partial{#1}}{\partial{#2}}}
\newcommand{\tr}[1]{\textcolor{red}{#1}}
\newcommand{\vlt}[1]{\textcolor{violet}{#1}}
\newcommand{\bl}[1]{\textcolor{blue}{#1}}
\newcommand{\sn}[1]{\textcolor{sonja}{#1}}
%%% TIKZS
\usepackage{tikz}
\usetikzlibrary{mindmap,trees}
\usetikzlibrary{backgrounds}
\tikzstyle{every edge}=  [fill=orange]  
\usetikzlibrary{tikzmark}
\usetikzlibrary{decorations.markings}
\usepackage{tikz-cd}
\usetikzlibrary{arrows,calc,fit}
\tikzset{mainbox/.style={draw=sonja, text=black, fill=white, ellipse, rounded corners, thick, node distance=5em, text width=8em, text centered, minimum height=3.5em}}
\tikzset{mainboxbig/.style={draw=sonja, text=black, fill=white, ellipse, rounded corners, thick, node distance=5em, text width=13em, text centered, minimum height=3.5em}}
\tikzset{dummybox/.style={draw=none, text=black , rectangle, rounded corners, thick, node distance=4em, text width=20em, text centered, minimum height=3.5em}}
\tikzset{box/.style={draw , rectangle, rounded corners, thick, node distance=7em, text width=8em, text centered, minimum height=3.5em}}
\tikzset{container/.style={draw, rectangle, dashed, inner sep=2em}}
\tikzset{line/.style={draw, very thick, -latex'}}
\tikzset{    pil/.style={
		->,
		thick,
		shorten <=2pt,
		shorten >=2pt,}}

% other stuff
\newcommand{\innermid}{\nonscript\;\delimsize\vert\nonscript\;}
\newcommand{\activatebar}{%
	\begingroup\lccode`\~=`\|
	\lowercase{\endgroup\let~}\innermid 
	\mathcode`|=\string"8000
}
%\usepackage{biblatex}
%\addbibresource{bib_mt.bib}
\usepackage{ulem}
\title{On the Role of Fiscal Policies in the Optimal Environmental Policy}
%\title{The Environment, Inequality, and Growth\\ \small{ optimal fiscal policy in an endogenous growth model with inequality and emission targets}}
\date{Sonja Dobkowitz\\ Bonn Graduate School of Economics\\ %University of Bonn\\
	\vspace{1mm}
	%Preliminary and incomplete\\
	%First version: January 9, 2022\\
	This version: \today }
\usepackage{graphicx,caption}
%\usepackage{hyperref}
\usepackage[colorlinks,linkcolor=aaltoblue,citecolor=aaltoblue,urlcolor=aaltoblue,unicode=true]{hyperref} %can create hyperlinks. ALWAYS LOAD LAST
\definecolor{aaltoblue}{RGB}{0,94,184}
\usepackage{minitoc}
\setcounter{secttocdepth}{5}
\usetikzlibrary{shapes.geometric}

% for tabular

%\usepackage{array}
\usepackage{makecell}
\usepackage{multirow}
\usepackage{bigdelim}

%propositions etc
\newtheorem{prop}{Proposition}
\newtheorem{corollary}{Corollary}[prop]
\newtheorem{lemma}[prop]{Lemma}

\renewenvironment{abstract}
{\small
	\list{}{
		\setlength{\leftmargin}{0.025\textwidth}%
		\setlength{\rightmargin}{\leftmargin}%
	}%
	\item\relax}
{\endlist}
\begin{document}
	%	\includepdf[pages=-]{../titlepage.pdf}
	\maketitle
	\begin{abstract}
		\begin{singlespacing}
			\textbf{Abstract \ }
			Some scholars call for reductive policies to handle tightening environmental limits. I show that, indeed, when labor supply is elastic, 
			the optimal policy consists of both a recomposing and a reductive element: an environmental tax does not suffice to implement the efficient allocation. Lump-sum transfers complement environmental taxes by reducing labor supply through an income effect.
			When environmental tax revenues are not redistributed lump sum, the planner uses progressive income taxes to decrease work effort.
			Therefore, the optimal environmental policy - as a byproduct - increases equity either through lump-sum transfers or progressive income taxation.
			I quantify the optimal income tax in an endogenous growth model with skill heterogeneity. The optimal income tax is progressive even though this reduces growth and increases the share of fossil to green energy usage due to a skill bias in the green sector.
			%	The model suggests that the use of income taxes when environmental tax revenues are not redistributed raises social welfare by 0.1\% over the 60 years from 2020 to 2080.
			%The model suggests that integration of environmental and fiscal policy when no lump-sum transfers are available raises social welfare by 0.1\% over the 60 years from 2020 to 2080.
			%			To reach climate targets, the International Panel on Climate Change has identified net-zero emissions by 2050 an essential element. I show that progressive income taxes are optimally used in concert with corrective taxes to satisfy an absolute emission limit.
			%			On the one hand, progressive income taxation reduces labour efforts as leisure becomes relatively cheaper. The overall reduction in production lowers emissions. On the other hand, a progressive income tax (i) reduces general research efforts through a market size effect  and (ii) recomposes the structure of the economy away from green energy through a skill-supply channel. Both effects call for a more regressive tax to foster a green transition. For a reasonable calibration, I find that the reduction effect dominates and the optimal income tax schedule is progressive. The welfare advantage of income tax progressivity arises from a reduction of inefficiently high hours worked. 
			%The model suggests that including progressive income taxes as a tool to lower emissions accounts for a rise in social welfare by 0.1\% over the 60 years from 2020 to 2080.  % as relative supply of the through a skill-supply channel. As richer, high-skilled workers reduce their labour supply more in response to a more progressive tax, low-skill labour becomes more abundant. 
			
			%			Natural scientists have identified the reduction of demand %for energy and land %thus, a change in lifestyle 
			%			as an important contributor to meeting global climate targets. However, a general equilibrium analysis of reduction policies is missing.
			%		%	I study the general equilibrium effects of reduction policies: such as income taxes, a restriction of hours worked, or consumption taxes. 
			%		A higher labour income tax progressivity can achieve such a reduction as it lowers labour supply. 
			%		Then again, tax progressivity alters the relative skill supply. As the green sector is relatively more skill-biased, the economy recomposes production towards the dirty sector. What is the optimal policy when the government has to meet an exogenous emission and demand target?
			%This additional benefit of income taxes changes the  equity-efficiency trade-off which classically determines optimal fiscal policies. 
			%To answer the question, I build a model of directed technical change and skill heterogeneity.
			
			
			%In a set up with representative agent, the necessity to meet emission targets makes the optimal income tax highly progressive. The more goods are substitutes the higher the optimal tax progressivity. When goods are complements, the more slowly growing clean sector dampens production in the dirty sector making a lower progressivity sufficient. 
		%	\noindent \textit{JEL classification}: E71, H21, H23,  O11, O13, Q58
			
		\end{singlespacing}
		
	\end{abstract}
	%\tableofcontents
	%\section{Introduction}

%\begin{quote}
%"Mitigation pathways limiting warming to 1.5°C [...]  reduce emissions further to reach net zero $CO_2$ emissions in the 2050s [...] \textit{(medium confidence)}."
%\end{quote}

The latest reports of the International Panel on Climate Change (IPCC) \citep{ IPCC2022, Rogelj2018MitigationDevelopment.} highlights the importance of an absolute emission limit to comply with the Paris Agreement on limiting temperature rise to 1.5°C or well below 2°C.\footnote{ \ The Paris Agreement of 2015 formulates clear political goals to mitigate climate change. Under this treaty, states have agreed on a legally binding maximum increase in temperature to well below 2°C, preferably 1.5° over pre-industrial levels, and the global community seeks to be climate-neutral in 2050  (compare:\\ \url{https://unfccc.int/process-and-meetings/the-paris-agreement/the-paris-agreement}). 
} 
The economics literature on environmental policy has by and large allowed for a trade-off between consumption and pollution \citep{Barrage2019OptimalPolicy, Golosov2014OptimalEquilibrium} or studied relative emission targets \citep{Fried2018ClimateAnalysis}. 
The presence of an absolute emission target may change the optimal environmental policy, as it poses a limit on growth in fossil energy usage.
%Depending on the substitutability of green and fossil energy and the velocity of the green sector to grow, the absolute emission target may, first, pose a limit to consumption growth and, second, make 
In particular, untraditional policy measures in addition to corrective taxes may become optimal. %\textit{ (WHY THIS DIFFERENCE? Also with externalities in consumption pollution cannot be compensated for by consumption as the marginal utility of consumption reduces. BUT STILL MORE IS BETTER! SO IT CAN BE COMPENSATED! )} 

%For a reasonably calibrated endogenous growth model,
 I find that the optimal labour income tax is progressive when accounting for an absolute emission target even though corrective taxes on fossil energy are available. This finding highlights the importance of policy measures targeted at a \textit{reduction} of production in tandem with \textit{recomposing} policies such as carbon taxes to mitigate climate change. % Then again, I present data indicating a voluntary reduction in household consumption. Given this behavioural change, the optimal income tax progressivity could become regressive in order to boost high-skill labour supply. 

%MODEL
To investigate the effect of an exogenous emission target on the optimal policy, I study an endogenous growth model building on \cite{Fried2018ClimateAnalysis}. 
%Calibration

% Quantitative Experiment and Results
The main finding is that progressive income taxes are optimally used in concert with carbon taxes to meet emission targets. Indeed, the emission target could be satisfied without income taxation by use of higher carbon taxes, yet, at lower welfare.  
Meeting emission targets with a fossil tax alone, hours supplied are inefficiently high. Due to the cap on fossil energy and green and fossil energy being no perfect substitutes, the additional work effort is not sufficiently compensated for by rising consumption.  
In fact, by use of a more regressive tax the government could subsidise research. Nevertheless, the planner optimally reduces labour supply thereby forfeiting higher growth rates in the green sector. %However, in presence of the emission target, higher work effort at higher productivity would violate the emission target. 

\paragraph{Literature}
The paper relates to two strands of literature. Firstly, it contributes to the literature on environmental policy and directed technical change \citep[e.g.][]{Acemoglu2012TheChange, Acemoglu2016TransitionTechnology} by focusing on unconventional policy measures which is justified given the urgent nature of climate change mitigation.  The paper is most closely related to \cite{Fried2018ClimateAnalysis} extending the analysis by (i) an optimal policy analysis, (ii) studying an absolute emission target, and (iii) providing the government with an additional policy tool: income taxes.
%\\
%{Limits to and endogenous growth}
%The paper also relates to the endogenous growth literature by inclu

Secondly, the paper connects to the literature on public policy which focuses on an efficiency-equity trade-off \citep{Heathcote2017OptimalFramework, Loebbing2019NationalChange}. In this project, instead, the reduction in labour effort induced by distortionary labour taxes has an advantageous effect: it reduces emissions. On the other hand, there is a recomposing effect which counteracts the reduction of emissions. This reduction-recomposition trade-off shapes the optimal tax progressivity in the present paper. 

%Third, the finding that hours worked are inefficiently high relates the paper to the literature on inefficiently high work efforts. Generally, too high hours supplied arise from some negative externality of consumption such as  a keeping-up-with the Joneses motive or envy \cite{Alvarez-Cuadrado2007EnvyHours}. \cite{Arrow2004AreMuch} also discuss the question of too high consumption levels.
\tr{Is there literature on optimal income taxes and environmental taxes? chosen jointly}

Thirdly, the paper relates to the literature discussing optimal environmental taxation in a distortionary fiscal setting \citep{Bovenberg1997EnvironmentalGrowth, Barrage2019OptimalPolicy, LansBovenberg1994EnvironmentalTaxation}. Redistribution and environmental protection arise as competing targets in the optimal policy. The reason is that the environmental tax reduces labour efforts as it diminishes the returns to labour or makes consumption more expensive.  

However, in contrast to this literature, the present paper 


\tr{Which tax schedule is closer to reality? }
\paragraph{Outline} Section \ref{sec:model} lays out the model which is calibrated in section \ref{sec:calib}. I subsequently show and discuss results in section \ref{sec:res}. Section \ref{sec:con} concludes. 
	%\clearpage
\setcounter{page}{1}
\section{Introduction}
%\tr{I show that carbon taxes are only efficient if lump-sum transfers are available.}

\begin{comment}
\tr{Think about:
	1) when labor income taxes are not used, then need to have  a higher environmental tax to meet emission limits? \ar Yes, because of advantageous level effect which outweighs recomposing effect of income tax.
	2) When staying at level optimal under the assumption of lump-sum redistribution, but then not redistributing, than absent labor income tax emissions are too high; by how much? Counterfactual}
	
	content...
	\end{comment}

The latest assessment report of the Intergovernmental Panel on Climate Change \citep{IPCC2022} highlights the urgency to reduce greenhouse gas emissions,%relative to the previous report from 2018 \citep{Rogelj2018MitigationDevelopment.}.
\footnote{ \  The report stresses the decreasing likelihood of meeting the Paris Agreement and limiting climate warming to 1.5°. The Paris Agreement of 2015 formulates clear political goals to mitigate climate change. Under this treaty, states have agreed on a legally binding maximum increase in temperature to well below 2°C, preferably 1.5° over pre-industrial levels, and the global community seeks to be climate-neutral in 2050  (see: \url{https://unfccc.int/process-and-meetings/the-paris-agreement/the-paris-agreement}). 
}
and some scholars have pointed to limiting consumption to handle environmental boundaries.\footnote{\ \cite{Schor2005SustainableReductionb} argues for the necessity to limit consumption in the global North through a reduction in working time. \cite{Arrow2004AreMuch} raise the question if today's consumption is too high from a sustainability perspective. \cite{Dasgupta2021}  argues for the impossibility of indefinite growth due to planetary boundaries  \citep{Rockstrom2009AHumanity}. %: acknowledging planetary boundaries, i.e., boundaries which define a state of nature in which humans can safely exist \citep{Rockstrom2009AHumanity}, and that production and consumption produce waste, infinite production would degrade nature in a way that production is impossible.
	 \cite{VanVuuren2018AlternativeTechnologies} study alternative mitigation pathways with lower demand %such as lower energy demand, lower appliance ownership, and meat consumption 
	 in an integrated assessment model motivated by seeking to reduce reliance on carbon capture and storage technologies which entail risks and compete for scarce land. \cite{Bertram2018TargetedScenarios} stress the importance to reduce demand for energy- and material-intense products to alleviate the trade-off between mitigating temperature rises  and the UN sustainability goals%(such as food security, biodiversity protection, and clean water)% (p.11: Shifting towards healthier diets and less energy-and material-intensive consumption patterns appearsto have greatest potential for reducing sustainabilityrisks along a wide range of dimensions)
	. } A reduction in work effort and consumption mitigates pollution by diminishing economic activity. Distortionary fiscal policies qualify as a reductive policy instrument to target the level of production.
However, the literature on environmental policy has focused on compositional policies: environmental taxes. %\citep{Fried2018ClimateAnalysis}. 
Given the exigency to act, this paper addresses the question whether fiscal policies can help meet climate targets. %Using analytical and quantitative methods, I show that reductive policies form part of the optimal environmental policy even absent an additional target.


%This is your core argument for why reductive policy measures may work, so you should mention above that you consider this possibility,  MACHE ICH DAS NICHT MIT DER rESEARCH QUESTION? suggested by its proponents (your "scholars" :-)), and actually show that it works.

In the first part of the paper, I show analytically that once 
labor supply is elastic, reductive policy measures optimally complement the environmental tax. 
The literature has established that, absent any other distortion, an environmental tax equal to the social cost of the externality implements the efficient allocation. 
%Environmental taxes are perceived as a cost-effective way to reduce emissions. 
I demonstrate that this result crucially depends on the use of lump-sum transfers to redistribute environmental tax revenues. Transfers reduce labor supply through an income effect. %Thus, indeed there is a role for reductive policy measures. 
%\textcolor{blue}{This is interesting independent of whether they are feasible or not. Could relate to the fact that there is a discussion how to use revenues. Yet, one might argue that we are always in a setting with distortionary labor income taxes; so that recycling lump-sum is never needed; numbers on size of expected revenues and government spending}
When environmental tax revenues are not redistributed lump sum, environmental taxes are optimally combined with progressive labor income taxes. The use of income taxes as a reductive policy measure is not directly targeted at the externality: the motive for labor taxation emerges from a distortion in labor markets as households feel poorer than the economy is.\footnote{\ This is a novel motive for the use of reductive policies adding to the arguments made in the literature listed in the previous footnote. These are: conflicts with other goals such as the UN sustainability goals, risks associated with carbon capture and storage technology, and planetary boundaries and limits to growth.} %Hence,  % to lower inefficiently high hours worked. 
% I show that redistributing environmental tax revenues through an income tax scheme allows to implement the efficient allocation. The optimal income tax scheme is progressive.
%the optimal environmental policy equalizes the distribution of income as  a side effect.
% The theoretical analysis forms the

In the second part, I scrutinize whether progressive income taxes remain optimal in an endogenous growth model with heterogeneous skills. The government cannot use lump-sum transfers but consumes environmental tax revenues.
 %In the spirit of \cite{Acemoglu2002DirectedChange}, directed technical change may intensify or mitigate these channels thr recomposition. %Second, an overall reduction in labor supply curbing production may lower general research incentives.
 % % more low skill supply, more fossil innovation, more fossil production, and higher low income \ar reduction in the wage premium! 
 The model suggests that the optimal income tax scheme is progressive. The benefits of labor taxation emerge from (i) more leisure and (ii) gains from knowledge spillovers.
 The latter advantage arises as a fossil tax directs research from the non-energy sector to the energy sector.  Energy and non-energy goods are complements in final good production. The literature on directed technical change has shown that when goods are sufficiently complementary a price effect dominates the direction of research \citep{Acemoglu2002DirectedChange, Acemoglu2012TheChange, Hemous2021DirectedEconomics}. Therefore, as the fossil tax makes energy more expensive, the higher price of energy goods pulls research to the energy sector.
 This effect of the environmental tax counters the intention to lower emissions and decreases knowledge spillovers from the non-energy sector. Knowledge spillovers from the non-energy sector, however, are especially valuable, as it is the biggest sector in terms of research processes. 

% 
% On the one hand, a skill bias documented for the green sector \citep{Consoli2016DoCapital} in combination with a relatively more elastic high-skill labor supply causes a higher tax progressivity to recompose the economic structure towards dirty production. On the other hand, a higher labor share in the fossil sector implies a recomposition of the economy towards green production.
%  The skill-recomposition channel dominates and is slightly amplified by a market size effect directing research towards the fossil sector. 
 
% labor income taxes are used to substitute for environmental taxes to realize the gains from knowledge spillovers. 
 
%\textit{I quantify the welfare gains of setting progressive income taxes to equal yyy in consumption equivalent measure. TO BE DONE  }

% relation to literature
I discuss briefly the most important contributions of the paper.
First, the results are relevant for the political and academic debate on how  to recycle environmental tax revenues. The paper points to the importance of lump-sum transfers within the optimal environmental policy as a reductive policy tool; an aspect which appears overlooked in the discussion.%\footnote{\ POLICY debate; \cite{Fried2018TheGenerations}}
When thinking about how to recycle environmental tax revenues other than by lump-sum transfers,  one should take into account alternative reductive tools such as progressive labor income taxes. 
If the reductive part of the environmental policy is neglected, environmental taxes have to be higher to meet emission limits, as I demonstrate in the quantitative exercise.

Second, the results contribute to the academic debate on the so-called \textit{weak double dividend} \citep[for example:][]{LansBovenberg1994EnvironmentalTaxation, LansBovenberg1996OptimalAnalyses}. The hypothesis posits that recycling environmental tax revenues to reduce preexisting tax distortions is advantageous to recycling  revenues as lump-sum transfers. The rationale is that transfers decrease labor supply thereby diminishing the tax base of the income tax. %A conflict between generating government funds and environmental protection arises. 
The findings in the present paper suggest a lower bound on the reduction in distortionary income taxes: when environmental tax revenues are not redistributed lump sum, some reduction in labor supply via distortionary income taxes is in fact efficient from an environmental policy perspective. %In other words, even if environmental tax revenues suffice to satisfy a government revenue requirement, there is a motive for progressive income taxation. 

Third, the findings are especially interesting as the provision of the environmental public good and equity have been perceived as competing targets in the literature. When the poor consume more of the polluting good, a corrective tax is regressive \citep{ Fried2018TheGenerations, Sager2019IncomeCurves}. % \textit{Metcalf 2007, Hassett 2009 as  in Fried 2018}. 
%Second and more indirectly, a fossil tax exerts efficiency costs by lowering labor efforts\footnote{\ The reduction in hours worked is per se not inefficient. The reduction in dirty production reduces the marginal product of labor, which might make a reduction efficient. However, when the government seeks to tax labor income using distortionary policy tools, the reduced labor supply diminishes the tax base of the labor tax making it more costly to redistribute.} which again raises the cost for the government to redistribute \citep{Dobkowitz2022}. 
In contrast to this literature, the present paper provides an argument for progressive income taxes under perfect income-risk sharing suggesting a double dividend of redistribution: equity and efficient externality mitigation. %: equity on the one hand and efficiency gains from less labor as part of the environmental policy.


\paragraph{Literature}


%Second, it connects to the literature connecting environmental and fiscal policies and how to recycle environmental tax revenues. Third, as the paper combines environmental and fiscal policies it naturally connects to the public finance literature. Finally, the results speak to the literature discussing inefficiently high production.
 
%\begin{itemize}
%	\item How to use environmental tax revenues \citep{Fried2018TheGenerations}
%	\item Optimal environmental policy \ar focuses on environmental taxes
%	\item weak double dividend
%	\item to be incorporated: \tr{\cite{Metcalf2003EnvironmentalPollution} why does he find that the optimal pigou tax equals first best when gov spending is satisfied with tax revenues? }
%	\\
%	Williams III: Welfare improvement with xxx \citep{Parry1999WhenMarkets} \tr{is this weak or strong dd?}
%\end{itemize}

The paper relates to four strands of literature. 
%---------------------------------------
%.. optimal environmental policy
%---------------------------------------
Firstly, the paper speaks to the literature on macroeconomic studies of environmental policies. Within this realm, the quantitative analysis connects in particular to the endogenous growth literature. 
In general, these papers focus on environmental taxation and analyze settings with inelastic labor supply so that there is no role for policies targeting the level of production. \cite{Golosov2014OptimalEquilibrium} investigate the optimal carbon tax in a dynamic stochastic general equilibrium model.  
\cite{Acemoglu2012TheChange} discuss with a tractable model of directed technical change limits and possibilities for growth. %They highlight the need for green research subsidies to foster green innovation in combination with carbon taxes to correct for the dynamic spillovers of green innovation not internalized by the research sector.
\cite{Fried2018ClimateAnalysis} extends the framework of the aforementioned paper to a quantitative model. My paper add to the latter an optimal dynamic policy analysis and elastic labor supply. % mainly by introducing cross-sectoral knowledge spillovers and diminishing returns to research. 
%She finds that emission limits can be met at a lower fossil tax when growth is endogenous. 


% OVERVIEW LITERATURE
% Acemoglu 2016 have lump-sum transfers and taxes
% Acemoglu Aghion 2012: lump-sum transfers, no optimal policy
%Golosov: hightlight the need of lump-sum transfers! but exogenous labor supply
% Therefore, the main finding of the present paper, the necessity of reductive policy measures to implement the efficient allocation, complements this literature. 
%Especially, when environmental tax revenues are not redistributed lump-sum in these papers, a variable labor supply would give an argument for labor income taxation. 


%\paragraph{Endogenous growth, elastic labor supply and optimal environmental policy}

\begin{comment}
3/09:	ADD WHEN INTRODUCING HETEROGENEOUS LABOR SUPPLY BAG IN
Secondly, staying within the field of endogenous growth, the paper connects to work examining the interaction of directed technical change and skill heterogeneity. \cite{Acemoglu2002DirectedChange} develops a theory to explain the positive correlation of skill supply and the skill premium: the higher supply of skilled labor raises incentives to innovate in the skill sector. \cite{Loebbing2019NationalChange} introduces fiscal policy into the model to investigate how the equalizing effect of redistribution is amplified through directed technical change.
My paper contributes to these two branches by integrating endogenous and heterogeneous skill supply in an environmental model of directed technical change. These ingredients enable me to analyze labor income taxes through the lens of environmental policies.  

content...
\end{comment}

 %Similar to my paper, a higher tax progressivity changes the relative supply of skills. As low-skill labor is in relative higher supply, low-skill-specific innovation depresses the wage distribution thereby contributing to equity. While the channels are comparable,
% I evaluate the effect of income tax progressivity and endogenous growth on emissions. 
%\cite{Hemous2021DirectedEconomics} provide an overview of models of directed technical change in environmental economics. They argue that a rise in the skill ratio directs innovation towards skill-intense technology when the high- and the low-skill output good are sufficient substitutes. Furthermore, when the two input goods are substitutes, the more advanced sector attracts more innovation. 


%\cite{Oueslati2002EnvironmentalSupply} studies the optimal environmental policy with elastic labor supply and endogenous growth. Yet, he allows for lump-sum transfers of environmental revenues. \textit{He should find something on reduction of hours}: No: capital is the only polluting factor, and labor is the clean factor of production.


%%%--------------------------------------------------------------------
% How to recycle environmental tax revenues: weak double dividend
%%%-------------------------------------------------------------------- 
\
% A big literature has examined potential benefits arising from corrective tax revenues to ameliorate fiscal distortions. The double-dividend literature is concerned with fiscal advantages arising from environmental tax revenues. My results speak directly to the weak double dividen hypothesis which 
%My paper most closely relates to the literature on the weak double-dividend
%\paragraph{Recycling environmental tax revenues}
%\tr{Read:\cite{Freire-Gonzalez2018EnvironmentalReview} yet on strong dd, I assume}
Secondly, this paper is not the first to integrate distortionary fiscal policies into the analysis of environmental policies. The literature discussing how to use environmental tax revenues generally assumes labor supply to be elastic and incorporates fiscal policies. 
The dominant focus of this literature is the weak double dividend of environmental taxes \citep[for instance,][]{Goulder1995EnvironmentalGuide, Bovenberg2002EnvironmentalRegulation, Barrage2019OptimalPolicy}: given an exogenous government funding constraint it is cost saving to recycle environmental tax revenues to lower distortionary labor income taxes as opposed to higher lump-sum transfers. The latter, so the rationale, decreases labor supply through an income effect thereby lowering the tax base of the labor income tax. %Consequently, it becomes more expensive for the government to generate revenues.
Therefore, this literature advocates recycling environmental tax revenues to reduce distortionary fiscal taxes as opposed to lump-sum rebates.
%With its quantitative part, my paper closely relates to \cite{Barrage2019OptimalPolicy} who examines the role of fiscal distortions emerging from an exogenous revenue constraint on the environmental policy in a quantitative framework. She also optimizes jointly over fiscal and environmental policy instruments, but her focus rests on the deviation of the optimal environmental tax from the social costs of carbon.

% my contribution : 1) lower bound on dist income tax; 2) motivation and role of income taxes
Relative to this literature, my paper's contribution is to discuss the existence of an upper bound on the reduction of distortionary income taxes: From an environmental policy perspective, some reduction in labor supply is in fact efficient. However, shrinking labor supply is generally perceived as an inefficiency in this literature and environmental taxes on its own as efficient. The importance of lump-sum transfers to implement the efficient allocation receives less attention.  %To the best of my knowledge, the papers theoretically discussing the weak double-dividend \citep{LansBovenberg1996OptimalAnalyses, Goulder1995EnvironmentalGuide} do not formally derive the result; a possible explanation for why the lower bound on distortionary tax reduction remained unnoticed. 
% The primary distinction of this paper and the weak double-dividend literature is the motive for income taxation. In this literature an exogenous funding constraint motivates the use of distortionary income taxes. In contrast, my model rationalizes a progressive income tax absent an exogenous revenue constraint arising in an otherwise equal set-up from the environmental externality per se as the environmental tax on its own does not establish the efficient allocation. 




 %This becomes clear when environmental tax revenues suffice to meet the government's funding constraint, then labor supply would be inefficiently high when the labor income tax is unused. 
%In contrast to the present paper, the double-dividend literature focuses on non-environmental cost advantages of environmental taxation either via interactions with other taxes and their bases or via their revenues. However, it remains unmentioned that under the assumption of elastic labor supply, which the literature necessarily assumes, the environmental tax alone is not efficient.

%----------------------------------------
%---- optimal revenue recycling 
%---- empirical and quantitative-------
%----------------------------------------
The question of how to use environmental tax revenues has seen a surge in interest recently and diverged from the prominence of fiscal advantages. 
 %They do not constitute a free lunch if efficiency is the goal.\footnote{\ Often, environmental taxes alone seem to be perceived as being able to implement the efficient allocation: \cite{LansBovenberg1999GreenGuide} writes "\textit{Environmental taxes are  generally  an  efficient  instrument  for  protecting  the  environment.}" thereby neglecting the role environmental tax revenue redistribution. Or "\textit{Establishing a price on carbon [...] is well understood to be the most efficient approach for reducing greenhouse gas emissions.}" \citep{Fried2018TheGenerations}. } 
The pros and cons of different recycling means is often assessed using inter- and within generational equity or political feasibility as value measures \citep{Carattini2018, Goulder2019IncomeGroups, VANDERPLOEG2022103966, Kotlikoff2021MakingWin, Carbone2013DeficitImpacts}. Building on the weak double-dividend literature, \cite{Fried2018TheGenerations} compare distinct recycling scenarios investigating the impact on inequality in an overlapping generations model. 
% They find lump-sum transfers to be preferred by the  living generation.
Using German data,   \cite{VANDERPLOEG2022103966} find an equity advantage of lump-sum transfers. % The authors suggest the government to split environmental tax revenues to both lower preexisting tax distortions and as lump-sum transfers. %The present paper employs the first-best allocation as a benchmark to assess distinct recycling methods.
My contribution to this debate is to point to lump-sum redistribution to constitute an integral part of an efficient pollution mitigation. % together with corrective taxes.
If carbon tax revenues are not rebated lump sum, a role for additional policy intervention emerges due to distortions in the labor market.

%\paragraph{Public finance}
Thirdly, the paper contributes to the public finance literature.
An equity-efficiency trade-off is central to this literature.  The benefits of labor taxes and progressivity arise, inter alia, from redistribution. %and from generating government revenues. 
%With concave utility specifications full redistribution is efficient. However, the optimal tax system does not feature full redistribution when labor supply is endogenous. Instead, redistribution is traded off against aggregate output as individuals reduce their labor supply and skill investment in response to labor income taxation 
\citep{Heathcote2017OptimalFramework, Conesa2009TaxingAll, Domeij2004OnTaxes}.
To this literature I add another motive for the use of distortionary fiscal policies: to reduce inefficiently high labor supply in the presence of environmental taxes. 
%One closely related work is \cite{Loebbing2019NationalChange} who studies optimal income taxation in a model of directed technical change. The redistributive effect of tax progressivity is amplified through a compression of the wage rate distribution \textit{to be continued}


Fourthly, the paper relates from its motivation and finding to the discussion on whether production levels are inefficiently high. 
%The finding relates to the literature discussing rationales for the usage of reductive policy measures. 
Other motives for the reduction of consumption arise from
envy \cite{Alvarez-Cuadrado2007EnvyHours}, or a positive externality of leisure \cite{Alesina2005WorkDifferent}. \cite{Arrow2004AreMuch} discuss whether there is a need to reduce consumption levels due to sustainability concerns. 
 The present paper contributes to this literature by identifying another reason for too high labor supply: The externality results from mitigating an environ- mental externality without lump-sum transfers.


\paragraph{Outline}
The remainder of the paper is structured as follows. Section \ref{sec:mod_an} presents the core model and the analytical results. In Section \ref{sec:model2}, I extend and calibrate the model to a quantitative framework.  Results are discussed in Section \ref{sec:res}. Section \ref{sec:con} concludes.

%The remainder of the paper is structured as follows. The next section \ref{sec:mod_an} presents a tractable model which is used to derive the analytical results in section \ref{sec:theory}. In section \ref{sec:model}, I extend the model to a quantitative framework and calibrate it. I present and discuss the quantitative results in section \ref{sec:res}. Section \ref{sec:con} concludes.
	\section{Introduction}
	The latest assessment report of the Intergovernmental Panel on Climate Change \citep{IPCC2022} highlights the urgency to reduce greenhouse gas emissions,%relative to the previous report from 2018 \citep{Rogelj2018MitigationDevelopment.}.
\footnote{ \  The report stresses the decreasing likelihood of meeting the Paris Agreement and limiting climate warming to 1.5°. The Paris Agreement of 2015 formulates clear political goals to mitigate climate change. Under this treaty, states have agreed on a legally binding maximum increase in temperature to well below 2°C, preferably 1.5° over pre-industrial levels, and the global community seeks to be climate-neutral in 2050  (see: \url{https://unfccc.int/process-and-meetings/the-paris-agreement/the-paris-agreement}). 
}
and some scholars have pointed to limiting consumption to handle environmental boundaries.\footnote{\ \cite{Schor2005SustainableReductionb} argues for the necessity to limit consumption in the global North through a reduction in working time. \cite{Arrow2004AreMuch} raise the question if today's consumption is too high from a sustainability perspective. \cite{Dasgupta2021}  argues for the impossibility of indefinite growth due to planetary boundaries  \citep{Rockstrom2009AHumanity}. %: acknowledging planetary boundaries, i.e., boundaries which define a state of nature in which humans can safely exist \citep{Rockstrom2009AHumanity}, and that production and consumption produce waste, infinite production would degrade nature in a way that production is impossible.
	 \cite{VanVuuren2018AlternativeTechnologies} study alternative mitigation pathways with lower demand %such as lower energy demand, lower appliance ownership, and meat consumption 
	 in an integrated assessment model motivated by seeking to reduce reliance on carbon capture and storage technologies which entail risks and compete for scarce land. \cite{Bertram2018TargetedScenarios} stress the importance to reduce demand for energy- and material-intense products to alleviate the trade-off between mitigating temperature rises  and the UN sustainability goals%(such as food security, biodiversity protection, and clean water)% (p.11: Shifting towards healthier diets and less energy-and material-intensive consumption patterns appearsto have greatest potential for reducing sustainabilityrisks along a wide range of dimensions)
	. } A reduction in work effort and consumption mitigates pollution by diminishing economic activity. Distortionary fiscal policies qualify as a reductive policy instrument to target the level of production.
However, the literature on environmental policy has focused on compositional policies: environmental taxes. %\citep{Fried2018ClimateAnalysis}. 
Given the exigency to act, this paper addresses the question whether fiscal policies can help meet climate targets. %Using analytical and quantitative methods, I show that reductive policies form part of the optimal environmental policy even absent an additional target.


%This is your core argument for why reductive policy measures may work, so you should mention above that you consider this possibility,  MACHE ICH DAS NICHT MIT DER rESEARCH QUESTION? suggested by its proponents (your "scholars" :-)), and actually show that it works.

In the first part of the paper, I show analytically that once 
labor supply is elastic, reductive policy measures optimally complement the environmental tax. 
The literature has established that, absent any other distortion, an environmental tax equal to the social cost of the externality implements the efficient allocation. 
%Environmental taxes are perceived as a cost-effective way to reduce emissions. 
I demonstrate that this result crucially depends on the use of lump-sum transfers to redistribute environmental tax revenues. Transfers reduce labor supply through an income effect. %Thus, indeed there is a role for reductive policy measures. 
%\textcolor{blue}{This is interesting independent of whether they are feasible or not. Could relate to the fact that there is a discussion how to use revenues. Yet, one might argue that we are always in a setting with distortionary labor income taxes; so that recycling lump-sum is never needed; numbers on size of expected revenues and government spending}
When environmental tax revenues are not redistributed lump sum, environmental taxes are optimally combined with progressive labor income taxes. The use of income taxes as a reductive policy measure is not directly targeted at the externality: the motive for labor taxation emerges from a distortion in labor markets as households feel poorer than the economy is.\footnote{\ This is a novel motive for the use of reductive policies adding to the arguments made in the literature listed in the previous footnote. These are: conflicts with other goals such as the UN sustainability goals, risks associated with carbon capture and storage technology, and planetary boundaries and limits to growth.} %Hence,  % to lower inefficiently high hours worked. 
% I show that redistributing environmental tax revenues through an income tax scheme allows to implement the efficient allocation. The optimal income tax scheme is progressive.
%the optimal environmental policy equalizes the distribution of income as  a side effect.
% The theoretical analysis forms the

In the second part, I scrutinize whether progressive income taxes remain optimal in an endogenous growth model with heterogeneous skills. The government cannot use lump-sum transfers but consumes environmental tax revenues.
 %In the spirit of \cite{Acemoglu2002DirectedChange}, directed technical change may intensify or mitigate these channels thr recomposition. %Second, an overall reduction in labor supply curbing production may lower general research incentives.
 % % more low skill supply, more fossil innovation, more fossil production, and higher low income \ar reduction in the wage premium! 
 The model suggests that the optimal income tax scheme is progressive. The benefits of labor taxation emerge from (i) more leisure and (ii) gains from knowledge spillovers.
 The latter advantage arises as a fossil tax directs research from the non-energy sector to the energy sector.  Energy and non-energy goods are complements in final good production. The literature on directed technical change has shown that when goods are sufficiently complementary a price effect dominates the direction of research \citep{Acemoglu2002DirectedChange, Acemoglu2012TheChange, Hemous2021DirectedEconomics}. Therefore, as the fossil tax makes energy more expensive, the higher price of energy goods pulls research to the energy sector.
 This effect of the environmental tax counters the intention to lower emissions and decreases knowledge spillovers from the non-energy sector. Knowledge spillovers from the non-energy sector, however, are especially valuable, as it is the biggest sector in terms of research processes. 

% 
% On the one hand, a skill bias documented for the green sector \citep{Consoli2016DoCapital} in combination with a relatively more elastic high-skill labor supply causes a higher tax progressivity to recompose the economic structure towards dirty production. On the other hand, a higher labor share in the fossil sector implies a recomposition of the economy towards green production.
%  The skill-recomposition channel dominates and is slightly amplified by a market size effect directing research towards the fossil sector. 
 
% labor income taxes are used to substitute for environmental taxes to realize the gains from knowledge spillovers. 
 
%\textit{I quantify the welfare gains of setting progressive income taxes to equal yyy in consumption equivalent measure. TO BE DONE  }

% relation to literature
I discuss briefly the most important contributions of the paper.
First, the results are relevant for the political and academic debate on how  to recycle environmental tax revenues. The paper points to the importance of lump-sum transfers within the optimal environmental policy as a reductive policy tool; an aspect which appears overlooked in the discussion.%\footnote{\ POLICY debate; \cite{Fried2018TheGenerations}}
When thinking about how to recycle environmental tax revenues other than by lump-sum transfers,  one should take into account alternative reductive tools such as progressive labor income taxes. 
If the reductive part of the environmental policy is neglected, environmental taxes have to be higher to meet emission limits, as I demonstrate in the quantitative exercise.

Second, the results contribute to the academic debate on the so-called \textit{weak double dividend} \citep[for example:][]{LansBovenberg1994EnvironmentalTaxation, LansBovenberg1996OptimalAnalyses}. The hypothesis posits that recycling environmental tax revenues to reduce preexisting tax distortions is advantageous to recycling  revenues as lump-sum transfers. The rationale is that transfers decrease labor supply thereby diminishing the tax base of the income tax. %A conflict between generating government funds and environmental protection arises. 
The findings in the present paper suggest a lower bound on the reduction in distortionary income taxes: when environmental tax revenues are not redistributed lump sum, some reduction in labor supply via distortionary income taxes is in fact efficient from an environmental policy perspective. %In other words, even if environmental tax revenues suffice to satisfy a government revenue requirement, there is a motive for progressive income taxation. 

Third, the findings are especially interesting as the provision of the environmental public good and equity have been perceived as competing targets in the literature. When the poor consume more of the polluting good, a corrective tax is regressive \citep{ Fried2018TheGenerations, Sager2019IncomeCurves}. % \textit{Metcalf 2007, Hassett 2009 as  in Fried 2018}. 
%Second and more indirectly, a fossil tax exerts efficiency costs by lowering labor efforts\footnote{\ The reduction in hours worked is per se not inefficient. The reduction in dirty production reduces the marginal product of labor, which might make a reduction efficient. However, when the government seeks to tax labor income using distortionary policy tools, the reduced labor supply diminishes the tax base of the labor tax making it more costly to redistribute.} which again raises the cost for the government to redistribute \citep{Dobkowitz2022}. 
In contrast to this literature, the present paper provides an argument for progressive income taxes under perfect income-risk sharing suggesting a double dividend of redistribution: equity and efficient externality mitigation. %: equity on the one hand and efficiency gains from less labor as part of the environmental policy.

	\paragraph{Outline} Section \ref{sec:mod_an} presents the model. I derive and discuss the results in section \ref{sec:theory}. Section \ref{sec:con} concludes.
	\section{Core Model}\label{sec:mod_an}
%\tr{if $tau_f$ was to replicate share of marginal products, then $H^*\geq H_{FB}$; i.e. absent lump-sum transfers (sub-optimal setting)}
%\textbf{Points to be made}
%\begin{enumerate}
%\item the efficient allocation consists of both a recomposing and a scaling element \ar discuss social planner allocation \checkmark
%\item if $tau_f$ was to replicate share of marginal products, then $H^*\geq H_{FB}$; i.e. absent lump-sum transfers (sub-optimal setting) \checkmark
%\item lump-sum transfers implement the efficient reduction in hours worked and ensure the efficient amount of consumption \checkmark
%\item absent lump-sum transfers and income, households work too much  \checkmark; This follows from setting $\tau_\iota>0$.
%\item redistribution through the income tax scheme establishes the efficient allocation if the income tax is progressive \textit{Intuition: in contrast to lump-sum transfers, transferring environmental tax revenues through the income tax scheme constitutes a positive multiplication of labor income \ar this increases labor efforts. The progressive tax counters this tendency.} \checkmark
%\end{enumerate}

This section develops a general model which is at the core of the analytical and quantitative results. 
The model presented in this section is designed as simple as possible to derive the theoretical results. In section \ref{sec:model}, the model is extended to the quantitative framework notably by adding endogenous growth and skill heterogeneity. %investigate the inefficiency arising in hours worked when an environmental externality has to be taken care of.

In the model, the household sector can be described by a representative household. The household faces a consumption and labor supply decision. The final consumption good is a composite of a fossil and a green good. Labor is the only input to production. For simplicity, the green sector does not induce any externality; yet, whenever intermediate goods are no perfect substitutes, final good production is never perfectly green. The core model abstracts from endogenous growth and is static. 

\paragraph{Representative household}
Throughout the paper, household's decisions are static. Each period, the household maximizes its period utility
\begin{align}
U(C,H; F).
\end{align} 

The household derives utility from consumption, $C$, but experiences disutility from hours spent working, $H$. An externality from fossil production, $F$, decreases household utility and is taken as given by the household.
I assume additive separability of consumption, hours, and the externality. I further assume that utility of consumption is increasing and strictly concave. As regards hours and the externality, utility is decreasing and strictly convex.
Utility maximization is subject to a period budget constraint
\begin{align}
	 C= \lambda(wH)^{1-\tau_{\iota}}+T_{ls}. \label{eq:hhbudget}
\end{align}

The variable $w$ indicates the wage rate, and $T_{ls}$ denotes lump-sum transfers from the government.
The planner levies income taxes on labor income using a non-linear tax scheme common in the public finance literature \citep{Heathcote2017OptimalFramework, Benabou2002TaxEfficiency}. The tax scheme is
characterized by (i) a scaling factor, $\lambda$, which determines the level of average tax revenues in the economy and (ii) a measure of the tax progressivity denoted by $\tau_{\iota}$. 
\cite{Heathcote2017OptimalFramework} show that whenever $\tau_{\iota}>0$ the tax scheme is progressive since the marginal tax rate exceeds the average tax rate irrespective of  pre-tax labor income. Hence, average tax payments increase with labor income. An alternative intuition is that when $\tau_{\iota}>0$, the elasticity of post- to pre-tax labor income is smaller unity for all levels of pre-tax labor income.  %\footnote{\ I show that the result is equivalent with a linear tax rate in the appendix.} 

\paragraph{Production}
All sectors of production are perfectly competitive and production functions have decreasing returns to scale. %\footnote{\ \textit{With increasing returns to scale the assumption of perfect competition would be violated. With constant returns to scale, the solution is not unique.}}. The final consumption good, $Y$, is a composite of the fossil, $F$, and the green intermediate good, $G$. 
Intermediate goods are produced from labor, $L_J$, and technology, $A_J$, where $J\in \{F,G\}$ indicates the fossil and the green sector: 
\begin{align}
Y=Y(F, G), \hspace{5mm} F=F(A_F, L_F),\hspace{5mm} G=G(A_G, L_G) \label{eq:prod}
\end{align}

\paragraph{Government}
The government raises income taxes from households and levies ad-valorem sales taxes, $\tau_F$, on fossil producers' revenues $p_FF$, where $p_F$ denotes the price of the fossil good paid by final good producers. Revenues from the income tax and the environmental tax are treated separately by the government. Income tax revenues are fully redistributed through the income tax schedule, while environmental tax revenues are either lump-sum redistributed to households, $T_{ls}$, consumed by the government, $Gov$, or redistributed through the income tax scheme, $T_\iota$:
\begin{align}
\tau_{F}p_FF=T_{ls}+T_\iota+Gov, \hspace{7mm}
0={w H}-\lambda(w H)^{1-\tau_{\iota}}+T_\iota. \label{eq:gov_but}
\end{align}
%Environmental tax revenues are either transferred lump-sum, fully consumed by the government, or transferred through the income tax schedule.

\paragraph{Markets}
The market for labor and the final good both clear: 
\begin{align}
H=L_F+L_G,\ \hspace{5mm} Y=C+Gov. \label{eq:market_clear}
\end{align}
 The final good, $Y$, is the numeraire. In this simple model, labor moves freely between the green and fossil sector. 
%I summarize the equations determining the competitive equilibrium in appendix section \ref{app:model}.
\paragraph{Competitive equilibrium}
In a competitive equilibrium, household behavior is determined by the budget constraint, equation \ref{eq:hhbudget}, and labor supply which follows from the household first order conditions and substituion of $\lambda$ from the government's budget on the income tax:
\begin{align}
-U_H=U_C(1-\tau_{\iota})w\left(1+\frac{T_\iota}{wH}\right). \label{eq:hsup}
\end{align}
Firms choose the quantity of input goods to maximize their profits taking prices as given. The following equations describe this behavior in equilibrium:
\begin{align}
p_G=\frac{\partial Y}{\partial G}, \hspace{5mm}
p_F = \frac{\partial Y}{\partial F}, \hspace{5mm}
w= p_F(1-\tau_F)\frac{\partial F}{\partial L_F}=p_G\frac{\partial G}{\partial L_G}.\label{eq:profmax}
\end{align}

The competitive equilibrium is defined as prices and allocations so that households and firms behave optimally; i.e. equations \ref{eq:hhbudget}, \ref{eq:hsup} and \ref{eq:profmax} hold. Production happens according to \ref{eq:prod}.  Equilibrium prices and the wage rate adjust to clear markets, equations \ref{eq:market_clear}. Finally, the government's budgets are satisfied \ref{eq:gov_but}. Policy variables $\tau_F$ and $\tau_\iota$ are taken as given. 

\section{Theoretic results}\label{sec:theory}
This section derives and discusses the main theoretical results. Section \ref{subsec:sp} defines the efficient allocation. It constitutes a benchmark for the optimal allocation which is discussed in section \ref{subsec:decen_ec}. 
\subsection{Social planner}\label{subsec:sp}
Let the share of fossil to total labor be denoted by $s=L_F/H$. The social planner's problem reads
\begin{align}
\underset{s, H}{\max}\ & U(C,H; F)\\ s.t\ \ & C=Y.
\end{align}
The first order conditions are given by
\begin{align}
wrt.\ s:\hspace{4mm} & U_C \cdot \left(\frp{Y}{F}\frp{F}{s}+\frp{Y}{G}\frp{G}{s}\right)=-U_F\frp{F}{s}, \label{eq:fbs}
\\
wrt.\ H:\hspace{4mm} & U_C\frp{Y}{H}+U_F\frp{F}{H}=-U_H\label{eq:fbh}. 
\end{align}
Where $U_X$ denotes the partial derivative of utility with respect to the variable $X$.
These equations determine the efficient or first-best allocation. 
Absent an externality, $U_F=0$, the efficient distribution of labor across sectors equalizes the marginal product of labor across sectors; compare equation \ref{eq:fbs}. Efficient hours balance the marginal utility gain from consumption and the marginal disutility from working formalized by equation \ref{eq:fbh}. 

When there is an externality, the social planner adjusts the allocation by two modulations: (i) a recomposing and (ii) a scaling one. 
The recomposition is determined by equation \ref{eq:fbs}.
The negative externality of fossil production makes it efficient to adjust the fossil labor share so that  a marginal reallocation of labor to the fossil sector would raise output.\footnote{\ Note that $U_F<0$ by assumption so that the right-hand side is positive and that $\frac{dG}{ds}<0$. }
One can show that the social planner reduces the fossil labor share when the aggregate production function features decreasing returns to scale in its labor inputs, $L_G$ and $L_F$\footnote{\ This follows from assuming that either final good and/or intermediate good production functions are decreasing returns to scale.}.
\begin{comment}
The equation 
\begin{align}
\frac{-U_F}{U_C \frac{dY}{dF}}=1+\frac{\frac{dY}{dG}\frac{dG}{ds}}{\frac{dY}{dF}\frac{dF}{ds}}.
\end{align}
The term on the left-hand side is the social cost of the externality: it measures what the representative household is willing to pay for a further reduction in fossil production. 
\end{comment}

The scaling effect is summarized by equation \ref{eq:fbh}.
First note that equation \ref{eq:fbh} can be rewritten by substituting equation \ref{eq:fbs} and noticing the relation of derivatives with respect to $H$ and $s$.\footnote{\ This is done in more detail for the optimal allocation in appendix section\ref{app:incometax0}. The relation of derivatives are summarized in section \ref{app:dervs_use}.}  
The second first order condition becomes:
\begin{align}\label{eq:fbh_simp}
-U_H=U_C\frac{\partial Y}{\partial G}\frp{G}{L_G}.
\end{align}
Hence, the externality drops from the expression which determines efficient labor. Hours are not chosen in a way to handle the externality. It is rather an indirect effect of externality mitigation which makes an adjustment in hours efficient which I will discuss next.

The recomposition of labor input towards the  green sector reduces the marginal product of labor in the green sector and the utility gains from more labor decline.  This effect has two opposing impacts on efficient labor supply. On the one hand, there is a substitution effect: as leisure becomes less costly, the efficient amount of hours reduces (note that the right-hand side of equation \ref{eq:fbh} is increasing in $H$). On the other hand, the economy becomes poorer in terms of consumption and more work effort might be efficient. This is captured by the term $U_C$ and equivalent to an income effect. 
%In total, which effect dominates depends on the curvature of the utility from consumption, $\theta$. With $\theta>1$ the  lower marginal product of labor decreases the efficient amount of hours worked. 
%Second, the social planner reduces hours worked due to their negative exeternality through fossil production. This effect is introduced by the term $U_F\frac{dF}{dH}<0$. 

Proposition \ref{prop:0} summarizes above discussion
\begin{prop}\label{prop:0}
	Efficient externality mitigation consists of a recomposing and a scaling approach. Efficiency of the scaling effect arises indirectly due to the reduction in the marginal product of labor induced by a recomposition of input factors.
\end{prop}


Depending on the importance of the income effect, efficient hours worked may be higher or lower than  absent an externality. %\footnote{ \ I discuss in the appendix conditions on parameter values when assuming functional forms of the model.}
I will show in the following, that irrespective of whether the social planner de- or increases hours, the decentralized economy always features higher hours when environmental tax revenues are not redistributed lump-sum. 

\begin{comment}
\hrule
One can show that the total effect of a drop in the fossil labor share on hours worked is positive, i.e. $\frac{dh_{FB}}{ds}>0$, if $\theta<\frac{\varepsilon}{\varepsilon-s}$. If the income effect dominates, the social planner increases hours worked as the economy becomes less productive. 
Under the value for $\theta$ suggested by \cite{Boppart2019LaborPerspectiveb}, the efficient scale effect is to increase hours worked. When, however, the substitution effect outweighs or dominates the income effect - as commonly assumed in the public finance literature \citep{Heathcote2017OptimalFramework, LansBovenberg1994EnvironmentalTaxation, LansBovenberg1996OptimalAnalyses} \tr{CHECK this}!.
Nevertheless, the level of hours worked exceeds the efficient level irrespective of $\theta$ when no lump-sum transfers are available. 
When the efficient level of hours increases, though, the fossil labor share reduces even more to outweigh the increase in the externality.

content...
\end{comment}
\subsection{Decentralized economy}\label{subsec:decen_ec}

In today's market economies, a planner to allocate hours worked and consumption does not exist. Instead, governments can revert to tax and transfer instruments to correct for distortions, such as an environmental externality. The question arises if the efficient allocation can be decentralized by the use of taxes and transfers in a competitive economy. And if so, how? Section \ref{subsec:Rams} defines the Ramsey problem. %For now, I assume that the income tax is not available and $\tau_{\iota}=0$, $\lambda=0$.

I show in this section that redistribution of environmental tax revenues are essential to implement the first-best allocation in the competitive equilibrium. Only in combination with lump-sum transfers of  environmental tax revenues does an environmental tax suffice to implement the efficient allocation. %Then the environmental tax equals the social cost of the externality as shown by \textit{PIGOU}. 
When environmental tax revenues are not redistributed lump-sum, hours worked exceed their efficient level, and a role for income taxes to lower hours worked arises. I consider two cases.
In the first case, section \ref{subsec:nolump}, environmental tax revenues are consumed by the government. The optimal policy consists of (i) a progressive labor income tax scheme and (ii) an environmental tax which may deviate from the social cost of the externality. The logic is that labor taxes help to align hours worked closer to the efficient allocation. 
Nevertheless, the efficient allocation is not feasible under this policy regime.

In the second scenario, therefore, I point to an option to implement the efficient allocation even if lump-sum transfers are not available: redistributing environmental tax revenues through the income tax scheme, section \ref{subsec:integrated}.
 I show that, again, the optimal tax scheme is progressive. 
As a consequence, the considered optimal environmental policies which establish the efficient allocation feature - as a side effect - a more equal distribution of income, through either lump-sum transfers or a progressive tax scheme.% \tr{Not sure though if this holds true in progressive scheme as lambda multiplies labor income}).

%\begin{enumerate}
%\item lump-sum transfers important for Pigou tax to implement efficient allocation: Proposition \ref{prop:1}
%\item when transfers are not redistributed: infeasibility of efficient allocation,  role for labor tax, and violation of Pigou principle \ref{prop:2}.
%\item redistribution through income tax scheme with progressive income tax restores efficient allocation \ref{prop:3}
%\end{enumerate}

\subsubsection{Government problem}\label{subsec:Rams}
The government is characterized by a Ramsey planner: it seeks to maximize utility of the representative household but can only revert to tax instruments and transfers to implement the welfare-maximizing allocation. The behavior of firms and households constrain the government's optimization problem. 
The Ramsey problem is defined as
\begin{align}
\underset{s, H}{\max}\ & U(C,H; F)\\ s.t\ \ & (1)\  C=Y-Gov\\ & (2) \ \text{behavior of firms and households}.
\end{align}
The first order conditions differ from the social planner's ones through the derivatives on government revenues, $Gov$:
\begin{align}
wrt.\ s:\hspace{4mm} & U_C\left(\frac{\partial Y}{\partial F}\frac{\partial F}{\partial s}+\frac{\partial Y}{\partial G}\frac{\partial G}{\partial s}-\frac{\partial Gov}{ \partial s}\right)=-U_F\frac{\partial F}{\partial s}, \label{eq:sbs}
\\
wrt.\ H:\hspace{4mm} & U_C\cdot \left(\frac{\partial Y}{\partial H}-\frac{\partial Gov}{\partial H}\right)+U_F\frac{\partial F}{\partial H}=-U_H\label{eq:sbh}. 
\end{align}

%-- paragraph to show that with Gov=0 and lump-sum transfers, the efficient allocation is implemented
When environmental tax revenues are fully redistributed lump-sum, i.e. $Gov=0$, $T_\iota=0$, then an environmental tax equal to the marginal social cost of fossil production\footnote{\ I define and derive the social cost of fossil production in appendix section \ref{app:scp}.} implements the efficient allocation. This observation is known as the \textit{Pigou principle} in the literature. 
To see this, note that equation \ref{eq:sbs} ensures that the social planner's first order condition, equation \ref{eq:fbs}, is satisfied. 
Rewriting equation \ref{eq:fbs} reveals that the Pigou principle holds: %\footnote{\ I derive the social cost of pollution as the price the representative household is willing to pay for a marginal reduction in fossil production. The derivation is exponded in appendix section \ref{sec:mod_an}. 
%	To be precise, social cost of pollution refers to the marginal cost evaluated at the resulting equilibrium allocation.}: The Pigou principle. 
\begin{align}
\underbrace{\frac{-U_F}{U_C\frac{\partial Y}{\partial F}}}_{\text{marginal social cost of fossil production}}=1+\frac{\frac{\partial Y}{\partial G}\frac{\partial G}{\partial s}}{\frac{\partial Y}{\partial F}\frac{\partial F}{\partial s}}=\tau^*_F.
\end{align}
Where the second equality follows from substituting intermediate firms' profit maximization conditions from equations \ref{eq:profmax}. I show in appendix section \ref{app:incometax0} that setting the environmental tax to the social cost of fossil production implies that the second first order condition of the Ramsey planner is satisfied without use of the income tax instrument: $\tau_{\iota}^*=0$. %at $\tau_\iota=0$ when all environmental tax revenues are redistributed lump-sum: $T_{ls}=\tau_{F}p_FF$ (and $Gov=0$ and $T_\iota=0$).

In this paragraph, I briefly discuss the mechanism of the corrective tax.
As discussed previously, absent an externality of production, it is efficient to balance marginal products of labor across sectors. However, when there is an externality, the social planner lowers the fossil share of labor which results in a higher marginal product of labor in the fossil sector. To sustain this gap between marginal products in the competitive equilibrium, the government has to introduce a corrective tax so that market forces do not direct labor towards the sector with the higher marginal product. In other words, the corrective tax is set so that wage rates equalize despite heterogeneous marginal products of labor. As a result of this intervention, the equilibrium wage rate is below the marginal product of labor in the fossil sector.\footnote{\ I formally discuss this statement in appendix section \ref{app:wageMPL}.} These are efficiency costs associated with a use of the environmental tax. They are the source of the competition between environmental good provision and raising government funds or equity alluded to in the literature \citep[e.g.][]{LansBovenberg1994EnvironmentalTaxation}.  
However, from an environmental policy perspective, the adjustment in labor due to a lower marginal product in the green sector is efficient. It mirrors the reduction in the marginal product of green labor in the efficient allocation. It is only when labor constitutes the base of another tax, that the reduction in labor supply becomes costly. 
\subsubsection{Lump-sum transfers and the optimal environmental policy}\label{subsec:nolump}

Another distortion of labor supply occurs when environmental tax revenues are not redistributed lump-sum.
In light of the discussions on how best to recycle environmental tax revenues, this is an important result which I summarize in proposition \ref{prop:1}:

\begin{prop}\label{prop:1}\textbf{Importance of lump-sum redistribution of environmental tax revenues}
	Without lump-sum transfers of environmental tax revenues, hours worked are inefficiently high when $\tau_{F}$ implements the efficient fossil labor share.
%Absent lump-sum transfers and when the  wage rate is non-increasing in equilibrium hours, 
%implementing the efficient share of fossil labor, $s^*=s_{FB}$, via an environmental tax results in inefficiently high hours worked. Lump-sum transfers would serve as a means to lower hours worked via an income channel. %The efficient allocation is infeasible.=> this statement would need to look at the optimal policy, this here is not a statement on the optimal policy
\end{prop}

% The logic is as follows:
%$s^*=s_{FB}$ when tauf implements the efficient allocation. 
% yet, implementing s-efficient without lump-sum transfers results in inefficiently high hours.

 
The proof of proposition \ref{prop:1}, depicted in appendix section \ref{app:nolumpsum_hourshigh}, is informative on the mechanism: The conclusion that $H^*>H_{FB}$, where the subscript $FB$ indicates the first-best allocation, while an asterisk marks the optimal allocation,  follows from consumption in the competitive equilibrium being lower than in the first-best allocation. Lump-sum transfers, thus, imply lower hours worked in the competitive equilibrium through an income effect.

% source of the inefficiency
%Are labor taxes used to cope with the externality, or is it rather that environmental taxation induces a distortion on labor supply?  I will argue in this section taht 
%The distortion in labor supply results from non-redistribution of environmental tax revenues. As it lowers the wage rate to ensure a lower labor share in the fossil sector, the environmental policy positively affects labor supply via an income effect. When lump-sum transfers are not used to counter this mechanism, equilibrium hours are inefficiently high.  %; otherwise, if the labor tax had an advantage in mitigating the externality, it would have been used in the presence of lump-sum transfers. But it is not.

% violation Pigou principle
Choosing the environmental tax to implement the efficient share of fossil production while hours are inefficiently high, most likely  violates the Pigou principle: the environmental tax does not equal the social cost of pollution. One reason is that a higher labor supply increases fossil production above the efficient level; the social cost of pollution increase when the disutility of pollution is convex. Another reason is that household consumption deviates from the efficient level of consumption: if it is below, then the marginal utility of consumption increases diminishing the willingness to pay for a reduction in the externality. 

% discussion: importance to reduce hours worked in context of exogenous emssion limit
When labor supply is endogenous, lump-sum transfers gain in importance for the optimal allocation. When labor supply is fixed, the non-redistribution of environmental tax revenues results in inefficiently low consumption with no further impact on emissions. When labor supply is elastic, however, the lower consumption results in too high hours worked. This is especially important as it aggravates the externality by increasing economic production. When there is an absolute limit on the externality - as is the case for greenhouse gas emissions today - the scale effect could make a stricter environmental tax necessary. 

%\textit{could be} optimal absent means to reduce hours worked. 

% transition to proposition 2: optimal policy
Proposition \ref{prop:1} rationalizes the use of distortionary income taxes as a tool to lower the supply of labor when environmental tax revenues are not redistributed lump-sum. 
The optimal environmental policy then consists of both a progressive income tax scheme and an environmental tax. However, the efficient allocation is infeasible as private consumption is inefficiently low when environmental tax revenues are consumed by the government. Proposition \ref{prop:2} highlights these result.

\begin{prop}\label{prop:2}\textbf{Optimal environmental policy without redistribution of environmental tax revenues}
When environmental tax revenues are not redistributed but instead consumed by the government, i.e., $Gov=\tau_Fp_FF$ and $T_{ls}=T_\iota=0$, then (i) the Pigou principle is violated, and (ii) a motive for labor taxation arises: the optimal income tax scheme is progressive. % if the aggregate production function features decreasing or constant returns to scale. 
The efficient allocation is infeasible.  
\end{prop}

Proof: in appendix sections \ref{app:reiv_tauf} to \ref{app:ineff}.

\paragraph{Optimal environmental tax}
One can show that under the optimal policy the environmental tax equals
	\begin{align}
\tau_{F}^*= SCC+\frac{\partial Gov}{\partial s}\frac{1}{\frac{\partial Y}{\partial F}\frac{\partial F}{\partial s}}.
\end{align}
\begin{comment}
content...

This can be further simplified:
\begin{align}
\tau_F^* = 1-\frac{SCC}{\frp{w}{s}}w. 
\end{align}
A condition for $\tau_F$ to exceed the social cost of the externality reads
\begin{align}
SCC<\frac{1}{1+\frac{w}{\frp{w}{s}}}
\end{align}
\end{comment}
Hence, if government revenues increase with the share of fossil labor, then the optimal environmental tax exceeds the social cost of pollution in equilibrium. 	

With the environmental tax the government intents to reduce the share of fossil labor in equilibrium to lower the externality. 
When environmental tax revenues are consumed by the government and reduce private consumption, there is another mechanism of the environmental tax which the government takes into account. In general, it is welfare increasing to raise private consumption, or, equivalently, to lower government consumption. Since a higher environmental tax implies a lower fossil labor share, a positive relation of fossil labor and government consumption adds to the welfare enhancing effect of the environmental tax. If, in contrast, a lower fossil labor share raises government revenues, the environmental tax has an additional negative effect on social welfare: the Ramsey planner chooses a lower environmental tax. 

%\tr{What mechanisms make government revenues rise with s, which reduce it?}

\paragraph{Optimal labor income tax}
The optimal labor income tax progressivity parameter is given by 
\begin{align}\label{eq:nls_taulopt}
\tau_\iota^*=\frac{1}{w}\frp{w}{s}_{F=\bar{F}}.
\end{align}
%\tr{Checked!!}
In words, the optimal income tax progressivity parameter equals the semi-elasticity of the wage rate in response to a decrease in the green labor share keeping fossil production unchanged. 
Since the environmental tax serves to sustain a gap between marginal products of labor, thereby  reducing the wage rate, a lower labor share in green production increases the wage rate. As a result, the labor income tax scheme is progressive: $\tau_\iota>0$. For a proof and the derivation of the optimal income tax progressivity see appendix section \ref{app:subsub_nltaul}.

Intuitively, the optimal labor income tax scheme is progressive to lower work effort closer to the efficient level. As argued in proposition \ref{prop:1}, hours worked are inefficiently high absent lump-sum transfers, when aggregate production is characterized by decreasing returns to scale. Under the same condition, therefore, the optimal income tax is progressive. 

%\paragraph{Infeasibility of efficient allocation}
%The optimal allocation is inefficient since either consumption is too low or hours work are inefficiently high. The proof is given in appendix section \ref{app:ineff}.
%Since the presence of the environmental tax artificially increases labor in the green sector depressing the wage rate (under the assumption of decreasing returns to scale), the wage rate rises by a reduction of the green labor share. 

\begin{comment}
\paragraph{Complements}
%\tr{Maybe no need to discuss this formally}
Income tax progressivity and the environmental tax
are complements if 
\begin{align}
\frac{d \tau_{\iota}^*}{d \tau_{F}}>0.
\end{align}
Totally differentiating equation \ref{eq:nls_taulopt} yields
\begin{align}
\frac{d \tau_{\iota}^*}{d \tau_F}=-\frac{1}{w^2}\frp{w}{s}\frac{dw}{d \tau_F}+\frac{1}{w}\frac{\partial^2 w}{\partial s^2}\frac{ds}{d \tau_F}
\end{align}
The first summand is positive since the environmental tax reduces the wage rate and a rise in dirty labor share increases the wage rate. The second summand is positive if $\frac{\partial^2 w}{\partial s^2}<0$. 

content...
\end{comment}

%\textit{
%Equation \ref{eq:nls_taulopt} makes clear that environmental taxation and the labor income tax are complements under decreasing returns to scale. When the environmental tax rises, thereby increasing the share of labor allocated to the green sector, the marginal product of green labor decreases further. A marginal reduction in the green labor share would increase the wage rate more the higher the green labor share, hence, the optimal labor tax progressivity increases with the environmental tax. 
%Secondly, the wage rate decreases with $\tau_F$ which as well inflates the optimal labor tax progressivity.}

% In the same section of the appendix, I prove that $\tau_\iota^*>0$ if green production and aggregate production feature constant or decreasing returns to scale and at least one produces with decreasing returns to scale. An assumption satisfied under Cobb-Douglas or constant elasticity of substitution production functions when goods are substitutes. When goods are complements, then...  
 
% \tr{Only direct effect, not a general equilibrium result}
% \begin{corollary}
% Absent income taxes, hours worked are inefficiently high when production features decreasing or constant returns to scale. 
% \end{corollary}
% As a corollary of proposition \ref{prop:2}, it follows that absent income taxes, hours worked are inefficiently high. This follows directly from the houeshold's first order condition:  a positive value of $\tau_\iota$ ireduces the right-hand side. Since the marginal utility is decreasing in hours the left-hand side is a positive function of labor supply. Hence, as the right-hand side decreases, hours diminish.   
% +
 
 \begin{comment}
Complementarity of the two instruments is intuitive. The use of labor taxes is not primarily to handle the environmental externality but instead to cope with a distortion induced by environmental taxation. 
This conclusion is backed by the observation that the externality measure, $U_F$, drops from the planner's first order condition on hours, equation \ref{eq:sbh}, if the environmental externality is taken care of by the environmental tax. Then, labor supply is determined solely by the trade-off between consumption and leisure.\footnote{\ On this argument see the proof on the optimality of $\tau_\iota^*=0$ when lump-sum transfers are available in appendix section \ref{app:incometax0}.} 

content...
\end{comment}
 

%\begin{proof}\textit{With an environmental tax alone, hours worked are inefficiently high } \tr{waiting; to be done}
%\end{proof}
% \tr{Possible interpretation of regressive income taxes: –  then it is more important to increase consumption! }
 

\subsubsection{Optimal policy with combined environmental and fiscal policy}\label{subsec:integrated}

In this subsection, I propose a policy regime which allows to establish the efficient allocation when lump-sum transfers are not available. In this setting, the government redistributes environmental tax revenues through the income tax scheme, the \textit{integrated policy} regime:  
\begin{align}
Gov= wh-\lambda (wH)^{1-\tau_\iota}+\tau_F p_FF.
\end{align}
The budget is balanced and $Gov = 0$ which determines $\lambda=\frac{wH + \tau_F p_F F}{w^{1-\tau_{\iota}}}$. 
Under this regime, the Ramsey planner can replicate the efficient allocation. 
The efficiency result is summarized in proposition \ref{prop:3}

\begin{prop}\label{prop:3}
	If lump-sum transfers are not available, the government can implement the efficient allocation by  transferring environmental tax revenues through the income tax scheme. The optimal tax scheme is progressive and the optimal environmental tax equals the social cost of pollution. %Recomposing and reductive policies are complements in the optimal environmental policy if environmental tax revenues are on the upward slow.
\end{prop}
Proof: in appendix section \ref{app:proofintegrated}. 

	That the optimal environmental tax satisfies the Pigou principle follows straight from the Ramsey planner's first order conditions. Since $Gov=0$ when environmental tax revenues are redistributed through the income tax scheme, the Ramsey planner's first order condition with respect to the fossil labor share ensures that the optimal environmental tax equals the social cost of pollution; compare equation \ref{eq:pigou}. 
	
	%The proofs that the optimal income tax scheme is progressive and that the optimal allocation is efficient\tau^*_f are sketched in appendix section \ref{app:proofintegrated}.
	 The optimal income tax is characterized by
\begin{align}
\tau^*_\iota=1-\frac{wH}{Y}=\frac{\tau_F^*p_FF}{Y}.
\end{align}
This simple relation reveals that the optimal income tax progressivity is positively related to environmental tax revenues.  %The higher the optimal environmental tax, which equals the social cost of pollution in this setting, the higher the optimal income tax progressivity. 
Hence, when environmental tax revenues are on the upward sloping part of the Laffer curve, it holds that  the more the Ramsey planner recomposes production, the more intense the reduction policy has to be. Reductive and recomposing policies complement each other in the integrated policy regime. 
 
\begin{comment}
\begin{prop}
Effect of using progressive income scheme on inequality (maybe as opposed to lump-sum transfers)
\end{prop}

content...
\end{comment}

\begin{comment}
\subsubsection{Discussion in relation to the literature}
These findings relate to the literature on a double dividend of environmental taxation discussed and partly rejected in the seminal paper by \cite{LansBovenberg1994EnvironmentalTaxation}.\footnote{ \ The double dividend of environmental policies refers to the idea that the revenues of environmental taxation can serve to improve on other policy targets such as equity or lowering distortionary fiscal policies. \cite{LansBovenberg1994EnvironmentalTaxation} argue that there is no double dividend because environmental taxes exert efficiency costs which outweigh the gains from lower distortionary income taxes. } The authors, inter alia, argue that recycling environmental tax revenues to lower distortionary fiscal policies has an advantage above recycling environmental tax revenues as lump-sum transfer. The latter would reduce labor supply even more thereby further narrowing the tax base of income taxes.  This result is referred to as the \textit{weak double dividend} hypothesis. The present paper demonstrates that there exists a lower bound up to where a reduction of labor reducing policies is beneficial. To illustrate this point, assume environmental tax revenues are sufficient to cover the government's exogenous revenue constraint. When no lump-sum transfers are available - a necessary assumption to motivate the use of distortionary fiscal policy measures - then, setting distortionary income taxes to zero is not optimal according to this paper's findings. The reason is that some reduction in hours worked is in fact efficient.\footnote{\ This argument refers to proposition \ref{prop:2}, i.e., to a setting where no lump-sum transfers are available. Even without an exogenous funding constraint, labor income taxes are used.}  % labor supply would be inefficiently high. 
%Nevertheless, the \textit{weak double dividend} result states that the  use of environmental tax revenues to lower fiscal policies is advantageous about recycling revenues as lump-sum transfers, because lump-sum transfers would lower the tax base of the income tax even more.
%To implement the efficient allocation, there is no choice of how to use environmental tax revenues and that they are to be perceived as a means to lower hours worked absent other motives for government intervention. To reconcile these two findings, think of the result presented herein as a lower bound on the optimality to diminish the reduction in hours worked through cutting distortionary taxes. 
  
%Although this trade-off between the environmental advantage of lower work effort versus a lower income tax base has implicitly been studied in the double-dividend literature, the efficiency of lump-sum transfers or progressive income taxes has gone unnoticed. 
% Absent other motives of government intervention, such as an exogenous funding condition or inequality, lump-sum transfers should 

%\tr{Caution: in double dividend literature there is a second motive for government intervention... then there are gains, but only up to a certain point}

The finding that redistribution is essential to implement the efficient allocation poses a warning to the literature discussing how to use environmental tax revenues such as \cite{Fried2018TheGenerations}. My results stress that there is no free choice in how to use environmental tax revenues but to redistribute if the government wants to implement the efficient allocation. 

content...
\end{comment}
%	\input{impo_model_fin}
%	\subsection{Calibration}\label{subsec:calib}

%\tr{Inflation data \url{/home/sonja/Documents/projects/subjective_BN/writing/mainmain}}

Section \ref{sec:ems} derives and discusses the emission target. 
Secton \ref{sec:modpar} calibrates the remaining model parameters.

\subsubsection{Emission target}\label{sec:ems}
To calibrate the emission target, I consider CO$_2$ emissions only and abstract from other greenhouse gasses since carbon is the most important pollutant with the highest mitigation potential \citep[p.29]{IPCC2022}.
%	 WG3 IPCC report (p.37) \textbf{\textit{The trajectory of future CO$_2$ emissions plays a critical role in mitigation, given CO$_2$ long-term impact and dominance in total greenhouse gas forcing}}. Furthermore, \textbf{The main reason is that scenarios reduce non-CO$_2$ greenhouse gas emissions less than CO$_2$ due to a limited mitigation potential (see 3.3.2.2)} p.34 in foxit, 3-26 in chapter 3}.  
The most recent IPCC report \citep{IPCC2022} formulates a reduction of global CO$_2$ emissions in the 2030s by 50\% relative to 2019 and net-zero emissions in the 2050s  as essential to meeting the 1.5°C climate target.\footnote{ ``\textit{Mitigation pathways limiting warming to 1.5°C [...] reach 50\% reductions of CO$_2$ in the 2030s, relative to 2019, then reduce emissions further to reach net zero CO$_2$ emissions in the 2050s [...] (\textnormal{medium confidence}).}" \citep[p.5, Chapter 3]{IPCC2022} }  Furthermore, the report stipulates a remaining global net CO$_2$ budget of 510 GtCO$_2$ %($\approx$ 510,000 million metric tons of CO$_2$) 
from 2020 to the net-zero phase starting from 2050 \citep[p.5, Chapter 3]{IPCC2022}. 
To deduce an emission target for the US, further assumptions on the distribution of mitigation burdens have to be made. I follow \cite{RobiouDuPont2017EquitableGoals} who consider 5 distinct principles of distributive burden sharing. I use an \textit{equal-per-capita} approach according to which emissions per capita shall be equalized across countries. 
 Appendix \ref{app:calib} details the calculation of the emission target. 
Figure \ref{fig:emlimit}  visualizes the resulting emission limit for the US starting from 2020. The value for 2015-2019 refers to observed emissions.

% data
%, 2022: Mitigation pathways compatible with long-term goals. In IPCC, 2022: Climate
% Change 2022: Mitigation of Climate Change. Contribution of Working Group III to the Sixth
% Assessment Report of the Intergovernmental Panel on Climate Change [P.R. Shukla, J. Skea, R.
% Slade, A. Al Khourdajie, R. van Diemen, D. McCollum, M. Pathak, S. Some, P. Vyas, R. Fradera, M.
% Belkacemi, A. Hasija, G. Lisboa, S. Luz, J. Malley, (eds.)]. Cambridge University Press, Cambridge,
% UK and New York, NY, USA. doi: 10.1017/9781009157926.005
% 


%\begin{table}[hh!!!!!]
%	\begin{center}
%		\captionsetup{width=0.9\textwidth}
%		\caption{Net CO$_2$ emission limit for the US by model period}
%		\label{tab:emlimit}
%		\begin{tabular}{l|rrrrrrrr}
	%			\hline 
	%			\hline
	%			Periods&20-24&25-29&30-34&35-39&40-44&45-49&50-80\\
	%			Limits in GtCO$_2$&3.6079&3.5396&3.4798&3.4245&3.3697&3.3164&0\\
	%			\hline \hline
	%			
	%		\end{tabular}
%	\end{center}
%\end{table}	

\begin{figure}
\caption{Net CO$_2$ emission limit in gigatons  (Gt)}\label{fig:emlimit}
%	\graphicspath{{../../codding_model/own_basedOnFried/optimalPol_010922_revision/figures/all_13Sept22_Tplus30/}{../../codding_model/own_basedOnFried/optimalPol_010922_revision/figures/all_13Sept22/}}
\includegraphics[width=0.4\textwidth]{../../../subjective_BN/codding_model/own_basedOnFried/optimalPol_010922_revision/figures/all_13Sept22_Tplus30/Emnet.png}
\end{figure}
%  In summary, I calibrate the net-emission target vector for the period from 2030 to 2080 as 
% $\omega_{2030-2050}$= 2.4899Gt and $\omega_{2050-2080}$= 0Gt.
%\footnote{Another alternative 
%} 
% I assume here that each country contributes to the global reduction by the same percentage of 50\% of its own emissions.\footnote{ Alternatively, one could assume that the global reduction is allocated in the same share as countries contributed to global emissions in 2019. This would result in an even stricter target for the US which contributed almost 20\% to global greenhouse gas emissions in 2019 (based on own calculations where total emissions come from the EIA global greenhouse gas information, to be found here \url{https://www.iea.org/reports/global-energy-review-2021/CO$_2$-emissions}).}
% Starting from 2050, the net-emission target is zero. 
% sinks and emission from fossil sector

\paragraph{Discussion}
The reduction in net CO$_2$ emissions necessary to meet the emission limit relative to 2019 emissions in the US  is substantial. It amounts to around 85\%. The result is not only explained by the global emission limit but also by the US emitting beyond its population share in 2019. In 2019, US emissions accounted for 10.44\% of global net emissions while the population share of the US was 4.3\%. Hence, even without an emission limit, the US would have to reduce emissions according to the \textit{equal-per-capita} principle.

The necessary reduction in net CO$_2$ emissions found in this calibration exceeds political goals. On April 22, 2021, President Biden announced a 50-52\% reduction in net greenhouse gas emissions relative to 2005 levels in 2030 % \footnote{ If pollutants were to be reduced by an equal share, this means a 50-52\% reduction in net CO$_2$ emissions.} 
and net-zero emissions no later than 2050.\footnote{ Source: \href{https://www.whitehouse.gov/briefing-room/statements-releases/2021/04/22/fact-sheet-president-biden-sets-2030-greenhouse-gas-pollution-reduction-target-aimed-at-creating-good-paying-union-jobs-and-securing-u-s-leadership-on-clean-energy-technologies/}{https://www.whitehouse.gov/briefing-room/statements-releases/2021/04/22/}, retrieved 14 September 2022.} 
However, relative to 2019, the planned reduction for 2030 corresponds to a 38\% decline only.
The resulting net emissions in the US would then amount to 103.21 Gt.\footnote{ This calculation assumes emissions where left at 2019-levels until 2030 and then lowered to the Biden target from 2030 to 2050 and net-zero afterwards.} This is roughly 5 times the budget acceptable for the US,  if the global remaining carbon budget was allocated on a \textit{equal-per-capita} basis.\footnote{ The remaining net carbon budget for the US based on its population share is 20.738Gt for the period from 2020 to 2050.} % This amounts to 27\% of emissions which the US would emit if annual emissions equaled 2019 net emissions.}  

\subsubsection{Model parameters}\label{sec:modpar}

\paragraph{Functional forms}
I assume the following functional form of period utility:
\begin{align*}
u(C_t,H_{t}, )= \log(C_t)-\chi\frac{H_{t}^{1+\sigma}}{{1+\sigma}}.
\end{align*}
The log-utility implies constant hours worked over time in a laissez-faire allocation since income and substitution effects cancel. This simplifies the analysis. %\footnote{  On the other hand, recent research has shown that substitution and income effects of the wage rate most likely do not cancel. \cite{Boppart2019LaborPerspectiveb} argue for a slightly higher income effect so that hours fall over time as productivity increases. I plan to conduct a sensitivity analysis by assuming the utility specification suggested in their paper.}

%Most likely, the continuous rise in carbon taxation over time lowers the wage rate and labor supply increases. A lack of lump-sum rebates would most likely aggravate the inefficiency of hours worked. %Nevertheless, as shown in the analytical part for a general utility function, some reductive policy is required to implement the efficient allocation. But, compared to the laissez-faire scenario, hours will rise. 

\paragraph{Parameter values}
To calibrate the model, I proceed in three steps. First, I set certain parameters to values found in the literature. Second, I calibrate the remaining variables requiring that equilibrium conditions and target equations hold. Third, parameters relating production and emissions are chosen. \autoref{tab:calib2} summarizes the calibrated parameter values.

I calibrate the model to the US in the baseline period from 2015 to 2019. Using this calibration approach, it is not ensured that the economy is on a balanced growth path. However, the goal of this paper is to study necessary interventions to meet an absolute emission limit. Therefore, %in contrast to a relative reduction objective, 
it is important to capture whether the economy is transitioning, for example, to %a balanced growth path with
a higher fossil share. The optimal dynamic policy has to counteract these forces. %These transitions are relevant for the dynamic policy. 
%To differentiate model dynamics from policy effects, I take care to interpret results as deviations from the economy without policy intervention. 


In the first step, I mainly rely on \cite{Fried2018ClimateAnalysis} to calibrate the parameters governing research processes, $\eta, \rho_F,\rho_N, \rho_G, \phi, \gamma, S $, and production, $\varepsilon_e, \varepsilon_y, \alpha_F, \alpha_G, \alpha_N$. The labor share in the green sector is remarkably low with $\alpha_G=0.91$. This diminishes the significance of labor supply for green innovation and production. Furthermore, fossil and green energy are no close substitutes with $\varepsilon_e=1.5$ so that the cap on fossil energy cannot be fully substituted for by green energy.
Returns to research are decreasing with $\eta=0.79<1$. This makes extreme distributions of researchers across sectors unproductive. The non-energy sector is the biggest research sector with $\rho_N=1$ and $\rho_F=\rho_G=0.01$. 
The utility parameters, $\beta, \sigma$, are set to $0.984^5$ and $0.75^{-1}$ following \cite{Barrage2019OptimalPolicy} and \cite{Chetty2011AreMargins}, respectively. The business-as-usual policy is set to $\tau_\iota=0.24, \tau_F=0$, where I borrow the tax rate from \cite{Barrage2019OptimalPolicy}. 
%The period over which the government maximizes, T, is chosen to focus on the population living during the transition to the net-zero emission limit. 
%One can think of the T as the periods under the regency of the government. I set T to 11 so that the planner 
%explicitly derives  allocations and polices for 55 years. In their overlapping-generations model, \cite{Kotlikoff2021MakingWin} use the same number to calibrate the working life of a household as it captures the years a household is typically active in economic markets. 
%\tr{ Regency: T=11=55 years, and explicit optimization over T+1 periods. 55 is a suggests to be a sensible number for the explicit optimization interval.  }

In the second step, I calibrate the weight on energy in final good production by matching the average expenditure share on energy relative to GDP over the period from 2015 to 2019 taken from the US Energy Information Administration \citep[][Table 1.7]{EIAEnergy}. The expenditure share equals 6\%. The resulting weight on energy is $\delta_y=0.30$. %\footnote{ Note that in difference to \cite{Fried2018ClimateAnalysis} I raise the weight on intermediate inputs in final production to the power $\frac{1}{\varepsilon_y}$, so that in the limit the function approaches the Leontief specification as $\varepsilon_y\rightarrow 0$ \citep{Herrendorf2014GrowthTransformation}.}
 The disutility of labor, $\chi$, is set to match equilibrium average hours worked to average hours over the period from 2015-2019 drawing from OECD data \citep{OECDHoursworked}, $\chi=9.66$. I normalize total economic time endowment for workers and scientists per day, which I set to 14.5 as found in \cite{Jones1993OptimalGrowth}, to 1. 

 Initial productivity levels follow from normalizing output in the base period to $Y=1$ and matching the ratio of fossil-to-green energy utilization over the years 2015-2019 which equals 7.33 according to \cite[][Table 1.3]{EIAEnergy}. I find that total factor productivities in the baseline period are $A_{N0}^{1-\alpha_N}=1.90$, $A_{F0}^{1-\alpha_F}=4.52$, and $A_{G0}^{1-\alpha_G}=1.17$. %Since the green and fossil energy good are no close substitutes with $\varepsilon_e=1.5$, the fossil sector has to be technologically more advanced to 

Finally, I calibrate the sink capacity to match the average difference between gross and net CO$_2$ emissions over the baseline period from 2015 to 2019. Information on emissions comes from the US Environmental Protection Agency \citep{EPAems}. Since sinks are relevant for all greenhouse gasses, I only use the proportion of total sink capacity which reflects contribution of carbon dioxide to gross greenhouse gas emissions. The resulting sink capacity per model period is $\delta=3.19$GtCO$_2$.\footnote{ I consider this capacity to be constant. This is a simplifying assumption. What is crucial qualitatively is the assumption that sinks are finite. Indeed, natural sinks and carbon capture and storage (CCS) technologies rely on the use of land \citep{VanVuuren2018AlternativeTechnologies} which is in limited supply. In addition, the importance of land for food production makes land even scarcer especially in light of a growing world population.}
The parameter relating CO$_2$ emissions and fossil energy in the base period equals $\omega=345.33$.\footnote{  I perceive the fossil sector in the model as source of all CO$_2$ emissions including, for instance, non-energy use of fuels and incineration of waste.}  

\begin{table}[h!]
\begin{center}
\captionsetup{width=0.9\textwidth}
\caption{ Calibration}
\label{tab:calib2}
\resizebox{5in}{!}{
	\begin{tabular}{c|ll}
		%			\hline \hline
		%			\multicolumn{7}{c}{Calibration based on basic needs}\\
		\hline \hline
		Parameter& Target/Source& \makecell[l]{Value}\\ 
		\hline
		Household&\multicolumn{2}{c}{}\\
		\hline 
		$\sigma$ &  \makecell[l]{\cite{Chetty2011AreMargins}}& $1.33$  \\
		$\chi$ &  \makecell[l]{average hours worked per\\ economic time endowment\\ by worker: 0.34 \citep{OECDHoursworked}}& 9.66 \\
		$\beta$ &  \makecell[l]{\cite{Barrage2019OptimalPolicy}}& 0.93 \\
		$\bar{H}$& \makecell[l]{14.5 hours per day\\ \cite{Jones1993OptimalGrowth}}&1.00 \\
		\hline
		Research&\multicolumn{2}{c}{}
		\\
		\hline 
		$\eta$ & & 0.79 \\
		($\rho_F$, $\rho_G$, $\rho_N$) & & (0.01, 0.01, 1.00) \\
		$\phi$ &\makecell[l]{\cite{Fried2018ClimateAnalysis}} & 0.50 \\
		$S$ && 0.01\\
		$\gamma$ && 3.96\\
		\hline
		Production&\multicolumn{2}{c}{}\\
		\hline
		($\varepsilon_y$, $\varepsilon_e$)&\cite{Fried2018ClimateAnalysis}&(0.05, 1.50)\\			
		$\delta_y$&\makecell[l]{expenditure share \\ on energy \citep{EIAEnergy}}&0.30\\	
		($\alpha_F$, $\alpha_G$, $\alpha_N$)&\cite{Fried2018ClimateAnalysis} &(0.72, 0.91, 0.36)\\
		%\hline
		%$\beta$&\makecell{ annual nominal rate 3\%\\ and annual inflation rate of 2\%}& 0.9903& discount factor\\ 
		\hline
		Initial total factor productivity&\multicolumn{2}{c}{}\\
		\hline
		($A_{F0}^{1-\alpha_F}$, $A_{G0}^{1-\alpha_G}$, $A_{N0}^{1-\alpha_N}$)& energy shares \citep{EIAEnergy} &(4.12, 1.17, 1.90)  \\
		\hline 
		Government&\multicolumn{2}{c}{}\\
		\hline
		$\tau_F$&- &0.00\\
		$\tau_{\iota}$&\cite{Barrage2019OptimalPolicy} &0.24\\
		\hline
		Emissions&\multicolumn{2}{c}{}\\
		\hline
		$\delta$& \makecell[l]{\cite{EPAems}}&3.19\\
		$\omega$& \cite{EPAems}&345.33\\
		\hline \hline
\end{tabular}	}
\end{center}
\end{table}

%According to the IEA, global greenhouse gas emissions from fuel combustion amounted to 34.2 Gt in CO$_2$ equivalents in 2019.\footnote{ Retrieved from \url{https://www.iea.org/reports/global-energy-review-2021/CO$_2$-emissions} on February 2, 2022.} I use the share the US contributed to global emissions in 2019, 19.18\%, to proxy the share in reductions I require the US to contribute to total reductions from 2019 to 2030. 

% procedure


% \textit{Convergence towards equal annual emissions per person} as a fair allocation of reductions. Then US emissions per capita should equal world emissions per capita. 
% I use the UN projected population measure to proxy for future population size.
%  The calibration is done with respect to CO$_2$ emissions. 



%Hence, the smallest adjustment follows from equal budgets per period. 
%I reduce each limit in the same proportion in the 2035-2050 period so that the remaining budget for the US for the period 2020 to 2035 

%This result leads to the following emission limits
%From 2020 to 2035 there is a total budget of net-CO$_2$ emissions of 10.627Gt for the US. From 2035 to 2050 model-period emissions may amount to [2.900, 2.854, 2.809].\footnote{ I use here that in earlier test runs the emission limits have been fully exploited. }

% \clearpage

%\thispagestyle{plain}
% \clearpage
%
%\paragraph{Sources data}
%%\url{https://www.eia.gov/totalenergy/data/monthly/#prices}
%
%Total energy data: 
%For data on skill and premium see references in 
%paper saved in data \citep{Slavik2020WagePremium}
%
%The model is calibrated to parameter values common in the literature. I bestow more care on  calibrating the emission target. 
%I match emissions in the model to emission targets suggested in the IPCC report \citep{Rogelj2018MitigationDevelopment.}. 
%%How to determine the economy in 2050? Should the economy have reached a steady state? or should it be in a transitional path? Maybe no need to specify this...it will be a outcome. All I have to use is that for all years after 2050 net-emissions have to be zero. Whether the economy is on the transitional path or in a steady state is an outcome. 
%The IPCC prescribes net-zero emissions starting from 2050. In 2030 emissions should be between 25 and 30 GtCO$_2$e per year.
%

%

%\clearpage
%	\section{Quantitative results}\label{sec:res}

In this section, I present and discuss the quantitative results.
Subsection \ref{subsec:mr} depicts the optimal policy under the baseline policy regime: environmental tax revenues are consumed by the government. Subsection \ref{subsec:dis} discusses the results in comparison to the efficient allocation focusing on the role of labor income taxes.
%I focus on analyzing the mechanisms and welfare benefits from integrating the income tax scheme into the environmental policy. I also discuss the costs of not using lump-sum transfers.


\subsection{Results}\label{subsec:mr}
%This section depicts results on the optimal policy followed by the implied allocation in the benchmark model where environmental tax revenues are redistributed via the income tax scheme. 

\begin{figure}[h!!]
	\centering
	\caption{Optimal Policy }\label{fig:optPol}
	\begin{minipage}[]{0.4\textwidth}
		\centering{\footnotesize{(a) Income tax progressivity, $\tau_{lt}$}}
		%	\captionsetup{width=.45\linewidth}
		\includegraphics[width=1\textwidth]{../../codding_model/own_basedOnFried/optimalPol_190722_tidiedUp/figures/all_July22/taul_SingleAltPolOPT_T_NoTaus_regime3_spillover0_noskill0_sep1_xgrowth0_etaa0.79.png}
	\end{minipage}
	\begin{minipage}[]{0.1\textwidth}
		\
	\end{minipage}
	\begin{minipage}[]{0.4\textwidth}
		\centering{\footnotesize{(b) Environmental tax, $\tau_{ft}$ }}
		%	\captionsetup{width=.45\linewidth}
		\includegraphics[width=1\textwidth]{../../codding_model/own_basedOnFried/optimalPol_190722_tidiedUp/figures/all_July22/tauf_SingleAltPolOPT_T_NoTaus_regime3_spillover0_noskill0_sep1_xgrowth0_etaa0.79.png}
	\end{minipage}
\end{figure} 


To meet the emission limits suggested by the IPCC, the optimal income tax is progressive for all periods between 2030 and 2080; see panel (a) in figure \ref{fig:optPol}.  
% optimal taul over time
% -0.0153   -0.0142    0.0949    0.0958    0.0967    0.0978    0.0926    0.0937    0.0949    0.0962    0.0976    0.0992
As an emission limit becomes active in 2030, the optimal income tax progressivity jumps to $\tau_{\iota t}=0.095$ and increases to $\tau_{\iota t}=0.098$ in 2045, the last period before the net-zero limit. As the emission limit diminishes to net-zero in 2050, optimal tax progressivity reduces slightly to above $0.093$ and gradually increases in the subsequent years to close to but below $0.1$. Overall, the optimal tax progressivity during constrained periods is approximately  around half the size found for the US in \cite{Heathcote2017OptimalFramework}: $\tau_{l}=0.181$.\footnote{\ 
In the period without emission limit from 2020 to 2030, the optimal income tax is slightly regressive to boost growth as will be discussed below.}

%
%0.0000    0.0000    0.6806    0.6821    0.6835    0.6849    0.9344    0.9346    0.9347    0.9348    0.9349    0.9350
Consider panel (b). The optimal fossil tax displays a step pattern. In 2030, the environmental tax jumps to around 68\% as the emission target is to reduce emissions by 50\% relative to 2019. Over the years from 2030 to 2045 there is a small gradual increse. As the emission target declines to net-zero emissions in 2050, the optimal tax on fossil sales accelerates to 93\% and gradually increases afterwards. 
These figures underline the optimality of the integration of fiscal policy instruments into the environmental policy.

Figure \ref{fig:optAll} depicts the optimal allocation. Limiting emissions in line with the Paris Agreement is concomitant with both a reduction and recomposition of consumption and production over time. 
Panel (a) shows consumption which reduces significantly when new emission limits become active, in 2030 and in 2050, but starting from the new low levels it growths modestly. Labor effort of both skill types reduces as stricter emission targets are enforced with the exception of high skill which slightly increases as the net-zero limit becomes binding; panel (b). The rise in high-skilled hours can be explained by the drop in income tax progressivity. Hours of low-skill workers appear constant over time after the initial reduction in 2030.  In comparison to hours supplied by low-skilled workers, high-skilled workers reduce hours more as the first emission limit gets introduced in 2030; compare panel (c) which shows the ratio of hours worked by high to low skill workers. Yet, the small rise in high-skill hours in 2050 causes an increase of the high-to-low skill ratio. The increase is decaying in the subsequent years.

The rise in consumption after each reduction is driven by technological progress in all sectors; compare panel (d) which shows growth rates by sector and as aggregate in per cent. 
The green sector sees a rise in technological progress, the dashed black line, while growth in the fossil and the non-energy sector is positive, yet diminishing over time. Overall, aggregate growth is positive but decreasing; compare the grey dashed graph. 
Summing up the last two paragraphs, the emission target is best achieved with more leisure at higher technology levels in all sectors. 

The optimal allocation of scientists over time nicely captures the combination of reductive and recomposing policies; see panel (e). There is a recomposition towards the green sector: while research in the non-energy and the fossil sector decrease over time, green research effort rises. Yet, overall, the amount of scientists reduces; compare the gray graph which depicts the sum of researchers across sectors.  
Finally, labor input goods are redirected towards the green sector; see panel (f). 

\begin{figure}[h!!]
	\centering
	\caption{Optimal Allocation }\label{fig:optAll}
	
	
	\begin{minipage}[]{0.32\textwidth}
		\centering{\footnotesize{(a) Consumption}}
		%	\captionsetup{width=.45\linewidth}
		\includegraphics[width=1\textwidth]{../../codding_model/own_basedOnFried/optimalPol_190722_tidiedUp/figures/all_July22/C_SingleAltPolOPT_T_NoTaus_regime3_spillover0_noskill0_sep1_xgrowth0_etaa0.79.png}
	\end{minipage}
	\begin{minipage}[]{0.32\textwidth}
		\centering{\footnotesize{(b) Hours worked }}
		%	\captionsetup{width=.45\linewidth}
		\includegraphics[width=1\textwidth]{../../codding_model/own_basedOnFried/optimalPol_190722_tidiedUp/figures/all_July22/SingleJointTOT_regime3_OPT_T_NoTaus_Labour_spillover0_noskill0_sep1_xgrowth0_extern0_etaa0.79_lgd1.png}
	\end{minipage}
	\begin{minipage}[]{0.32\textwidth}
		\centering{\footnotesize{(c) High-to-low-skill ratio hours}}
		%	\captionsetup{width=.45\linewidth}
		\includegraphics[width=1\textwidth]{../../codding_model/own_basedOnFried/optimalPol_190722_tidiedUp/figures/all_July22/hhhl_SingleAltPolOPT_T_NoTaus_regime3_spillover0_noskill0_sep1_xgrowth0_etaa0.79.png}
	\end{minipage}
	\begin{minipage}[]{0.32\textwidth}
		\centering{\footnotesize{\ \\ (d) Technology growth}}
		%	\captionsetup{width=.45\linewidth}
		\includegraphics[width=1\textwidth]{../../codding_model/own_basedOnFried/optimalPol_190722_tidiedUp/figures/all_July22/SingleJointTOT_regime3_OPT_T_NoTaus_Growth_spillover0_noskill0_sep1_xgrowth0_extern0_etaa0.79_lgd1.png}
	\end{minipage}
%\begin{minipage}[]{0.32\textwidth}
%	\centering{\footnotesize{\ \\ (d) Technology growth}}
%	%	\captionsetup{width=.45\linewidth}
%	\includegraphics[width=1\textwidth]{../../codding_model/own_basedOnFried/optimalPol_190722_tidiedUp/figures/all_July22/SingleJointTOT_regime0_OPT_T_NoTaus_Growth_spillover0_noskill0_sep1_xgrowth0_extern0_etaa0.79_lgd1.png}
%\end{minipage}
	\begin{minipage}[]{0.32\textwidth}
		\centering{\footnotesize{\ \\(e) Scientists }}
		%	\captionsetup{width=.45\linewidth}
		\includegraphics[width=1\textwidth]{../../codding_model/own_basedOnFried/optimalPol_190722_tidiedUp/figures/all_July22/SingleJointTOT_regime3_OPT_T_NoTaus_Science_spillover0_noskill0_sep1_xgrowth0_extern0_etaa0.79_lgd1.png}
	\end{minipage}
	\begin{minipage}[]{0.32\textwidth}
		\centering{\footnotesize{\ \\(f) Labor input}}
		%	\captionsetup{width=.45\linewidth}
		\includegraphics[width=1\textwidth]{../../codding_model/own_basedOnFried/optimalPol_190722_tidiedUp/figures/all_July22/SingleJointTOT_regime3_OPT_T_NoTaus_LabourInp_spillover0_noskill0_sep1_xgrowth0_extern0_etaa0.79_lgd1.png}
	\end{minipage}
\end{figure} 



\subsection{Discussion}\label{subsec:dis}
%\tr{Questions}
%\begin{itemize}
%	\item why progressive tax? and why the drop in progressivity in 2050? (a means to boost high-skill supply and keeping low skill stable)
%	\item what are the costs of the progressive tax
%\end{itemize}
The discussion of the optimal policy centers on the question what drives the optimal policy. In the first subsection, \ref{subsec:sp_q}, I compare the optimal to the efficient allocation. In subsection \ref{subsec:notaul}, I contrast the optimal allocation under the benchmark regime with income tax to a scenario where no income tax is available. This comparison is informative on the benefits of an income tax.
Finally, I turn to the optimal allocation when lump-sum transfers are used in subsection \ref{subsec:comp_lumpsum}. 
%In subsection \ref{subsec:simpler}, I discuss the results when the benchmark model is simplified: that is, assuming exogenous growth and/or skill homogeneity.

%\begin{enumerate}
%	\item What is the goal of policy intervention? \ar social planner allocation
%	\item (Benefits) What is different when no integrated policy is run and instead revs consumed by government \ar Benefits of an integrated policy
%	\item double dividend literature: use of labor income tax when all env tax revenues are consumed by the government.
%	\item (Costs) What cannot be reached by integrated policy as compared to lump-sum transfers: is taul used for different purpose? without endogenous growth should be zero; eg. can use taul to boost growth as lump-sum transfers take care of labor supply 
%	\item What could be reached if there was no trade-off with heterogenous skills or growth? no heterogeneous skills, no endogenous growth \ar how does the optimal policy differ?
%\end{enumerate}

\subsubsection{Social planner allocation}\label{subsec:sp_q}
As a benchmark to the Ramsey planner allocation, I present the social planner's allocation. The efficient allocation can be perceived as the allocation the Ramsey planner seeks to implement. However, it may not be able to achieve the efficient allocation due to the reliance on tax instruments. Figure \ref{fig:fb_opt} depicts the efficient and the optimal allocation by the black-solid and the orange-dashed graphs, respectively. 
\begin{figure}[h!!]
	\centering
	\caption{Comparison to efficient allocation }\label{fig:fb_opt}
	
	\begin{minipage}[]{0.32\textwidth}
		\centering{\footnotesize{(a) Consumption}}
		%	\captionsetup{width=.45\linewidth}
		\includegraphics[width=1\textwidth]{../../codding_model/own_basedOnFried/optimalPol_190722_tidiedUp/figures/all_July22/C_CompEffOPT_T_NoTaus_regime3_opteff_spillover0_noskill0_sep1_xgrowth0_countec0_etaa0.79_lgd1_lff0.png}
	\end{minipage}
	\begin{minipage}[]{0.32\textwidth}
	\centering{\footnotesize{(b) High skill hours worked}}
	%	\captionsetup{width=.45\linewidth}
	\includegraphics[width=1\textwidth]{../../codding_model/own_basedOnFried/optimalPol_190722_tidiedUp/figures/all_July22/hh_CompEffOPT_T_NoTaus_regime3_opteff_spillover0_noskill0_sep1_xgrowth0_countec0_etaa0.79_lgd0_lff0.png}
\end{minipage}
	\begin{minipage}[]{0.32\textwidth}
	\centering{\footnotesize{(c) Low skill hours worked}}
	%	\captionsetup{width=.45\linewidth}
	\includegraphics[width=1\textwidth]{../../codding_model/own_basedOnFried/optimalPol_190722_tidiedUp/figures/all_July22/hl_CompEffOPT_T_NoTaus_regime3_opteff_spillover0_noskill0_sep1_xgrowth0_countec0_etaa0.79_lgd0_lff0.png}
\end{minipage}

	\begin{minipage}[]{0.32\textwidth}
	\centering{\footnotesize{(d) Aggregate growth}}
	%	\captionsetup{width=.45\linewidth}
	\includegraphics[width=1\textwidth]{../../codding_model/own_basedOnFried/optimalPol_190722_tidiedUp/figures/all_July22/gAagg_CompEffOPT_T_NoTaus_regime3_opteff_spillover0_noskill0_sep1_xgrowth0_countec0_etaa0.79_lgd0_lff0.png}
\end{minipage}
\begin{minipage}[]{0.32\textwidth}
	\centering{\footnotesize{(e) Energy mix, $\frac{G}{F}$}}
	%	\captionsetup{width=.45\linewidth}
	\includegraphics[width=1\textwidth]{../../codding_model/own_basedOnFried/optimalPol_190722_tidiedUp/figures/all_July22/GFF_CompEffOPT_T_NoTaus_regime3_opteff_spillover0_noskill0_sep1_xgrowth0_countec0_etaa0.79_lgd0_lff0.png}
\end{minipage}
\begin{minipage}[]{0.32\textwidth}
	\centering{\footnotesize{(f) Utility}}
	%	\captionsetup{width=.45\linewidth}
	\includegraphics[width=1\textwidth]{../../codding_model/own_basedOnFried/optimalPol_190722_tidiedUp/figures/all_July22/SWF_CompEffOPT_T_NoTaus_regime3_opteff_spillover0_noskill0_sep1_xgrowth0_countec0_etaa0.79_lgd0_lff0.png}
\end{minipage}
%\begin{minipage}[]{0.32\textwidth}
%	\centering{\footnotesize{(f) Utility Scientists}}
%	%	\captionsetup{width=.45\linewidth}
%	\includegraphics[width=1\textwidth]{../../codding_model/own_basedOnFried/optimalPol_190722_tidiedUp/figures/all_July22/Utilsci_CompEffOPT_T_NoTaus_regime3_opteff_spillover0_noskill0_sep1_xgrowth0_countec0_etaa0.79_lgd0_lff0.png}
%\end{minipage}
%\begin{minipage}[]{0.32\textwidth}
%	\centering{\footnotesize{(f) Utility labour}}
%	%	\captionsetup{width=.45\linewidth}
%	\includegraphics[width=1\textwidth]{../../codding_model/own_basedOnFried/optimalPol_190722_tidiedUp/figures/all_July22/Utillab_CompEffOPT_T_NoTaus_regime3_opteff_spillover0_noskill0_sep1_xgrowth0_countec0_etaa0.79_lgd0_lff0.png}
%\end{minipage}
%\begin{minipage}[]{0.32\textwidth}
%	\centering{\footnotesize{(f) Utility con}}
%	%	\captionsetup{width=.45\linewidth}
%	\includegraphics[width=1\textwidth]{../../codding_model/own_basedOnFried/optimalPol_190722_tidiedUp/figures/all_July22/Utilcon_CompEffOPT_T_NoTaus_regime3_opteff_spillover0_noskill0_sep1_xgrowth0_countec0_etaa0.79_lgd0_lff0.png}
%\end{minipage}
\end{figure}

The social planner allocation, too, is a combination of recomposing and reductive measures. 
The social planner reduces consumption less than in the Ramsey planner allocation in order to reach emission limits; compare panel (a). There is a reduction in hours worked for high- and low-skilled labor over time as emission limits become stricter. Relative to a scenario without emission limit the efficient level of hours also reduces; see panels (b) and (c) in figure \ref{fig:eff_with_notarget} in appendix section \ref{app:eff_notarg} which compares the efficient allocation absent emission limit to the one with emission limit.\footnote{\ For the given calibration, the planner reduces hours worked once there is an emission limit. Since with log-utility the income and substitution effect cancel, the social planner does not increase hours worked to compensate for lower output. Compare the discussion of the efficient allocation in section \ref{sec:theory}.} 
The optimal allocation mimics the reduction in hours worked; yet, for the high-skilled the reduction is too strong starting from 2030. The gap between the efficient and the optimal level of high-skilled hours reduces as tax progressivity declines in 2050. Optimal labor supply of the low skilled is inefficiently high for all periods considered.

Higher growth rates contribute to more consumption in the efficient allocation; see panel (d) showing aggregate growth. While the Ramsey planner can boost growth through the labor income tax only, the social planner can directly allocate scientists. The ratio of green to fossil energy, depicted in panel (e), moves similarly in the efficient and the optimal allocation. This illustrates the recomposing quality of the efficient allocation to cope with stricter emission limits. However, the Ramsey planner chooses a lower green-to-fossil energy mix starting from 2050. 

Utility of the representative household in the efficient and the optimal allocation is decreasing overall, panel (f). The reduction happens in steps when a tighter emission limit becomes active. Utility is rising when the emission limit is stable but more so under the social planner. Recall that utility is net of environmental concerns solely determined by consumption and leisure in the present framework. In the periods absent emission limit, from 2020 to 2030, utility under the optimal policy exceeds utility in the efficient allocation. This counterintuitive behaviour highlights the dynamic structure of the model: the social planner forgoes this higher utility level to profit from a higher technology level in future periods.

%\subsubsection{Comparison to other policy regimes}
%\tr{To be rewritten}
%How does the optimal allocation and especially its relation to the efficient allocation change under alternative policy scenarios?
%In this section, I discuss two policy alternations which have already been discussed in the analytical section. First, a version where environmental tax revenues are consumed by the government and no labor income tax scheme is available, henceforth referred to as \textit{separate policy}. The comparison of this scenario serves to assess the benefits of an integrated environmental-fiscal policy when no lump-sum transfers are available. 
%Second, I look at the optimal allocation 

\subsubsection{The role of income taxes}\label{subsec:notaul}
How does the availability of an income tax change the optimal allocation? 
In figure \ref{fig:comp_nored}, I contrast the optimal allocation (i) without income tax scheme, orange-dotted graph, with (ii) a regime with the option to use a labor income tax. The solid black graph depicts the efficient allocation. %Finally, I discuss how endogenous growth and skill heterogeneity shape the optimal income tax. 


In comparison to a policy scenario without income tax, the availability of an income tax allows to more closely resemble the efficient levels of labor, panels (b) and (c). %In total, the utility level of the representative household is at least as close to the efficient level for all time periods considered. 
The benefits of the progressive income tax, more leisure, come at the cost of less consumption, panel (a), a reduction in growth, panel (d), and a lower green-to-fossil energy mix, panel (e). Overall, the gains from a higher income tax exceed its costs. \textit{\tr{(Do CEV, work in progress)}}

The optimal environmental tax is only negligibly smaller when the income tax is available, compare panel (b) in figure \ref{fig:comp_nored_pol}. 
The minimal adjustment in the optimal environmental tax does not suggest  that the two tax instruments are substitutes in that the income tax is used to target the emission limit. There would be argument in favor of such a substitution effect given the model's framework: while environmental tax revenues are not redistributed to households, the income tax is redistributed, thereby keeping consumption high. Despite this advantage, there is no evidence for a substitution of policy instruments. This rather points to environmental taxes and labor income taxes being complements in the optimal environmental policy: the first is targets the externality, while the second handles the inefficiency in labor markets. 
Only in the period from 2030 to 2050 the environmental tax necessary to meet emission limits is slightly smaller which can be rationalized by a lower level of production due to the decline in labor supply induced by the income tax.

These results speak to the weak double-dividend literature. %When the government consumes environmental tax revenues, hours worked are inefficiently high. 
The weak double-dividend result posits that when environmental tax revenues suffice to cover all government funding requirements, it would be optimal to lower distortionary income taxes. The results presented herein, however, show that there is a lower bound. Lowering distortionary income taxes too much results in inefficiently high hours worked. Hence, even though there is no motive to fund government expenses  labor income taxation is not zero due to the environmental externality.
%Indeed, this reduces consumption further away from the efficient level, but, hours worked are aligned closer to the efficient level, panels (b) and (c). Next to consumption, the planner also forfeits an advantageous green-to-fossil energy ratio, panel (e). 

%The use of a progressive labor income tax contributes minimally to meeting the emission limit as can be seen by scrutinizing the optimal environmental tax, panel (b) in figure \ref{fig:comp_nored_pol}: when income taxes can be used, the environmental tax is lower. Still, the difference is minimal, supporting the thesis of complementarity of income and environmental taxes. 
%Environmental tax revenues are lower as the tax rate reduces, and income taxes reduce labor supply and hence the tax base of the environmental tax. 
%Even though labor income taxes have the advantage of being redistributed to households and lowering the externality, they are not used to substitute environmental tax revenues.\footnote{\ This might be a motive to prefer labor income taxes as an instrument to reduce emissions since labor income tax revenues are redistributed back to the household while environmental tax revenues are not in this setting. Nevertheless, the observation that the environmental tax only adjusts slightly once an income tax tool is available points to the advantage of environmental taxes in handling too high emissions.
%}

%\tr{Could find that the labor income tax is used to reduce emissions because it does not reduce consumption! \ar maybe better to compare to a scenario where government revenues have to be equal to the env. tax revenues in the separate scenario, and then check if labor income taxes are still used? But this would reduce government revenues from the exogenous target! What would be a good comparison?}

\begin{figure}[h!!]
	\centering
	\caption{Comparison optimal allocation with and without income tax}\label{fig:comp_nored}
	
	\begin{minipage}[]{0.32\textwidth}
		\centering{\footnotesize{(a) Consumption}}
		%	\captionsetup{width=.45\linewidth}
		\includegraphics[width=1\textwidth]{../../codding_model/own_basedOnFried/optimalPol_190722_tidiedUp/figures/all_July22/C_DDCompEffOPT_T_NoTaus_pol3_spillover0_noskill0_sep1_xgrowth0_etaa0.79_lgd1_lff0.png}
	\end{minipage}
	\begin{minipage}[]{0.32\textwidth}
		\centering{\footnotesize{(b) High skill hours worked}}
		%	\captionsetup{width=.45\linewidth}
		\includegraphics[width=1\textwidth]{../../codding_model/own_basedOnFried/optimalPol_190722_tidiedUp/figures/all_July22/hh_DDCompEffOPT_T_NoTaus_pol3_spillover0_noskill0_sep1_xgrowth0_etaa0.79_lgd0_lff0.png}
	\end{minipage}
	\begin{minipage}[]{0.32\textwidth}
		\centering{\footnotesize{(c) Low skill hours worked}}
		%	\captionsetup{width=.45\linewidth}
		\includegraphics[width=1\textwidth]{../../codding_model/own_basedOnFried/optimalPol_190722_tidiedUp/figures/all_July22/hl_DDCompEffOPT_T_NoTaus_pol3_spillover0_noskill0_sep1_xgrowth0_etaa0.79_lgd0_lff0.png}
	\end{minipage}
	\begin{minipage}[]{0.32\textwidth}
		\centering{\footnotesize{(d) Aggregate growth}}
		%	\captionsetup{width=.45\linewidth}
		\includegraphics[width=1\textwidth]{../../codding_model/own_basedOnFried/optimalPol_190722_tidiedUp/figures/all_July22/gAagg_DDCompEffOPT_T_NoTaus_pol3_spillover0_noskill0_sep1_xgrowth0_etaa0.79_lgd0_lff0.png}
	\end{minipage}
	\begin{minipage}[]{0.32\textwidth}
		\centering{\footnotesize{(e) Energy mix, $\frac{G}{F}$}}
		%	\captionsetup{width=.45\linewidth}
		\includegraphics[width=1\textwidth]{../../codding_model/own_basedOnFried/optimalPol_190722_tidiedUp/figures/all_July22/GFF_DDCompEffOPT_T_NoTaus_pol3_spillover0_noskill0_sep1_xgrowth0_etaa0.79_lgd0_lff0.png}
	\end{minipage}
	\begin{minipage}[]{0.32\textwidth}
		\centering{\footnotesize{(f)Utility}}
		%	\captionsetup{width=.45\linewidth}
		\includegraphics[width=1\textwidth]{../../codding_model/own_basedOnFried/optimalPol_190722_tidiedUp/figures/all_July22/SWF_DDCompEffOPT_T_NoTaus_pol3_spillover0_noskill0_sep1_xgrowth0_etaa0.79_lgd0_lff0.png}
	\end{minipage}
\end{figure}

\begin{figure}[h!!]
	\centering
	\caption{Optimal policy with and without income tax}\label{fig:comp_nored_pol}
	
	\begin{minipage}[]{0.32\textwidth}
		\centering{\footnotesize{(a) Income tax progressivity, $\tau_{\iota t}$ }}
		%	\captionsetup{width=.45\linewidth}
		\includegraphics[width=1\textwidth]{../../codding_model/own_basedOnFried/optimalPol_190722_tidiedUp/figures/all_July22/taul_DDCompEffOPT_T_NoTaus_pol3_spillover0_noskill0_sep1_xgrowth0_etaa0.79_lgd1_lff0.png}
	\end{minipage}
	\begin{minipage}[]{0.32\textwidth}
		\centering{\footnotesize{(b) Environmental tax, $\tau_{ft}$}}
		%	\captionsetup{width=.45\linewidth}
		\includegraphics[width=1\textwidth]{../../codding_model/own_basedOnFried/optimalPol_190722_tidiedUp/figures/all_July22/tauf_DDCompEffOPT_T_NoTaus_pol3_spillover0_noskill0_sep1_xgrowth0_etaa0.79_lgd0_lff0.png}
	\end{minipage}
	\begin{minipage}[]{0.32\textwidth}
		\centering{\footnotesize{(c) Government consumption }}
		%	\captionsetup{width=.45\linewidth}
		\includegraphics[width=1\textwidth]{../../codding_model/own_basedOnFried/optimalPol_190722_tidiedUp/figures/all_July22/GovCon_DDCompEffOPT_T_NoTaus_pol3_spillover0_noskill0_sep1_xgrowth0_etaa0.79_lgd0_lff0.png}
	\end{minipage}
\end{figure}


%\subsubsection{Explaining the pattern of the optimal income tax progressivity parameter}
\begin{figure}[h!!]
	\centering
	\caption{Optimal tax progressivity in amended models }\label{fig:optPol_nogr_nosk}
	\begin{minipage}[]{0.32\textwidth}
		\centering{\footnotesize{(a) Baseline model}}
		%	\captionsetup{width=.45\linewidth}
		\includegraphics[width=1\textwidth]{../../codding_model/own_basedOnFried/optimalPol_190722_tidiedUp/figures/all_July22/taul_SingleAltPolOPT_T_NoTaus_regime3_spillover0_noskill0_sep1_xgrowth0_etaa0.79.png}
	\end{minipage}
	\begin{minipage}[]{0.32\textwidth}
	\centering{\footnotesize{(b) Exogenous growth}}
	%	\captionsetup{width=.45\linewidth}
	\includegraphics[width=1\textwidth]{../../codding_model/own_basedOnFried/optimalPol_190722_tidiedUp/figures/all_July22/taul_SingleAltPolOPT_T_NoTaus_regime3_spillover0_noskill0_sep1_xgrowth1_etaa0.79.png}
\end{minipage}
\begin{minipage}[]{0.32\textwidth}
	\centering{\footnotesize{(c) Skill homogeneity}}
	%	\captionsetup{width=.45\linewidth}
	\includegraphics[width=1\textwidth]{../../codding_model/own_basedOnFried/optimalPol_190722_tidiedUp/figures/all_July22/taul_SingleAltPolOPT_T_NoTaus_regime3_spillover0_noskill1_sep1_xgrowth0_etaa0.79.png}
\end{minipage}

\begin{minipage}[]{0.32\textwidth}
	\centering{\footnotesize{(c) endogenous growth}}
	%	\captionsetup{width=.45\linewidth}
	\includegraphics[width=1\textwidth]{../../codding_model/own_basedOnFried/optimalPol_190722_tidiedUp/figures/all_July22/taul_SingleAltPolOPT_NOT_NoTaus_regime3_spillover0_noskill0_sep1_xgrowth0_etaa0.79.png}
\end{minipage}
\begin{minipage}[]{0.32\textwidth}
	\centering{\footnotesize{(c) exogenous growth, no target}}
	%	\captionsetup{width=.45\linewidth}
	\includegraphics[width=1\textwidth]{../../codding_model/own_basedOnFried/optimalPol_190722_tidiedUp/figures/all_July22/taul_SingleAltPolOPT_NOT_NoTaus_regime3_spillover0_noskill0_sep1_xgrowth1_etaa0.79.png}
\end{minipage}
\end{figure} 

% growth only enhanced as long as feasible
Figure \ref{fig:optPol_nogr_nosk} contrasts the optimal tax progressivity parameter in the baseline model, panel (a), in a model with exogenous growth, panel (b), and a model without skill heterogeneity, panel (c).
First note that even with exogenous growth the optimal income tax is progressive and approximately similarly in size. Hence, it is not that income taxes are used to lower emissions through a market-size effect. The main driver is the inefficiency in labor supply which makes the optimal income tax progressive. However, endogenous growth shapes the structure of income tax progressivity. 
While progressivity of the income tax is increasing when growth is endogenous, it is constant or even decreasing in a model with exogenous growth. 
As a conclusion, income tax progressivity in the baseline model is depressed to profit from more growth. In other words, a more regressive income tax schedule is chosen to boost growth. Nevertheless, regressivity decreases over time as accelerating growth conflicts with the emission limit.  
% The costs a higher environmental tax would incur to meet the emission limit at higher labor effort and growth exceed the benefits of more growth and consumption. 

Due to skill heterogeneity, both the regressivity of the tax in periods without an emission limit and the progressivity in periods with emission limit is more pronounced; see panel (c). When the planner has to satisfy an emission limit, it can choose a higher progressivity as there is no adverse recomposing effect on green-to-fossil energy use.  
\begin{comment}
\paragraph{Comparison integrated policy to separate policy}

Consider figure \ref{fig:bench_nored_notaul}. The figure presents the optimal allocation in the integrated policy scenario,  the orange-dashed graph, the optimal allocation under the separate policy, the blue-dotted graph, and the efficient allocation, the black-solid graph.

In comparison to a policy scenario where environmental tax revenues are not redistributed, the integrated policy closer resembles the efficient allocation in terms of consumption, panel (a) and of labor, panels (b) and (c). %In total, the utility level of the representative household is at least as close to the efficient level for all time periods considered. 
The benefits of an integrated-policy regime come at the cost of a lower green-to-fossil energy mix, panel (e), and a reduction in growth, panel (d). Nevertheless, if a planner could choose between the two regimes, it would select the integrated-policy regime. The gains from the integrated regime amount to xxx. \tr{Do CEV}

Interestingly, the optimal environmental tax is only negligibly smaller in the integrated-policy regime. This suggests, that environmental taxes and labor income taxes are complements in the optimal environmental policy to lower inefficiently high hours worked. Only in the period from 2030 to 2050 the environmental tax necessary to meet emission limits is slightly smaller which can be rationalized by a lower level of production.\footnote{\ Absent an emission limit before 2030, the optimal environmental tax is slightly negative to subsidize fossil research which again spills over to research in the other sectors. }  

\begin{figure}[h!!]
	\centering
	\caption{Comparison to separate policy scenario; \tr{drop efficient from tauf graph }}\label{fig:bench_nored_notaul}
	
	\begin{minipage}[]{0.32\textwidth}
		\centering{\footnotesize{(a) Consumption}}
		%	\captionsetup{width=.45\linewidth}
		\includegraphics[width=1\textwidth]{../../codding_model/own_basedOnFried/optimalPol_190722_tidiedUp/figures/all_July22/C_CompEffOPT_T_NoTaus_pol2_spillover0_noskill0_sep1_xgrowth0_etaa0.79_lgd1_lff0.png}
	\end{minipage}
	\begin{minipage}[]{0.32\textwidth}
		\centering{\footnotesize{(b) High skill hours worked}}
		%	\captionsetup{width=.45\linewidth}
		\includegraphics[width=1\textwidth]{../../codding_model/own_basedOnFried/optimalPol_190722_tidiedUp/figures/all_July22/hh_CompEffOPT_T_NoTaus_pol2_spillover0_noskill0_sep1_xgrowth0_etaa0.79_lgd0_lff0.png}
	\end{minipage}
	\begin{minipage}[]{0.32\textwidth}
		\centering{\footnotesize{(c) Low skill hours worked}}
		%	\captionsetup{width=.45\linewidth}
		\includegraphics[width=1\textwidth]{../../codding_model/own_basedOnFried/optimalPol_190722_tidiedUp/figures/all_July22/hl_CompEffOPT_T_NoTaus_pol2_spillover0_noskill0_sep1_xgrowth0_etaa0.79_lgd0_lff0.png}
	\end{minipage}
	\begin{minipage}[]{0.32\textwidth}
		\centering{\footnotesize{(d) Aggregate growth}}
		%	\captionsetup{width=.45\linewidth}
		\includegraphics[width=1\textwidth]{../../codding_model/own_basedOnFried/optimalPol_190722_tidiedUp/figures/all_July22/gAagg_CompEffOPT_T_NoTaus_pol2_spillover0_noskill0_sep1_xgrowth0_etaa0.79_lgd0_lff0.png}
	\end{minipage}
	\begin{minipage}[]{0.32\textwidth}
		\centering{\footnotesize{(e) Energy mix, $\frac{G}{F}$}}
		%	\captionsetup{width=.45\linewidth}
		\includegraphics[width=1\textwidth]{../../codding_model/own_basedOnFried/optimalPol_190722_tidiedUp/figures/all_July22/GFF_CompEffOPT_T_NoTaus_pol2_spillover0_noskill0_sep1_xgrowth0_etaa0.79_lgd0_lff0.png}
	\end{minipage}
	%	\begin{minipage}[]{0.32\textwidth}
	%	\centering{\footnotesize{(f) Utility}}
	%	%	\captionsetup{width=.45\linewidth}
	%	\includegraphics[width=1\textwidth]{../../codding_model/own_basedOnFried/optimalPol_190722_tidiedUp/figures/all_July22/SWF_CompEffOPT_T_NoTaus_pol2_spillover0_noskill0_sep1_xgrowth0_etaa0.79_lgd0_lff0.png}
	%\end{minipage}
	\begin{minipage}[]{0.32\textwidth}
		\centering{\footnotesize{(f) Environmental tax, $\tau_{ft}$}}
		%	\captionsetup{width=.45\linewidth}
		\includegraphics[width=1\textwidth]{../../codding_model/own_basedOnFried/optimalPol_190722_tidiedUp/figures/all_July22/tauf_CompEffOPT_T_NoTaus_pol2_spillover0_noskill0_sep1_xgrowth0_etaa0.79_lgd0_lff0.png}
	\end{minipage}
\end{figure}


	content...
\end{comment}

\subsubsection{Optimal policy and allocation with lump-sum transfers}\label{subsec:comp_lumpsum}

How do lump-sum transfers change the role of labor income taxes?\footnote{\ In appendix section \tr{To be added}, I present the optimal allocation under a policy regime where environmental tax revenues are redistributed through the income tax scheme. This scenario is relevant when the government wants to redistribute environmental tax revenues but lump-sum transfers are not feasible.}
 According to the theory in section \ref{sec:mod_an}, the use of lump-sum taxes should (i)  allow to attain an allocation closer to the efficient one and (ii) deprive the income tax scheme of its use as reductive environmental policy tool. Indeed, the optimal allocation under lump-sum transfers is much closer to the efficient one. 
 Yet, the emission limit still shapes the optimal income tax due to endogenous growth. 
  %This is so despite the advantageous recomposing effect of regressive income taxes through skill supply.
  
% YES, one can speak of a separation of environmental and fiscal policies as the goal of income taxes is to boost or lower growth in the first place. We also dont speak of the environmental tax being targeted at 

\begin{figure}[h!!]
	\centering
	\caption{Comparison integrated regime and regime lump-sum transfers}\label{fig:bench_lumpsum}
	
	\begin{minipage}[]{0.32\textwidth}
		\centering{\footnotesize{(a) Consumption}}
		%	\captionsetup{width=.45\linewidth}
		\includegraphics[width=1\textwidth]{../../codding_model/own_basedOnFried/optimalPol_190722_tidiedUp/figures/all_July22/C_CompEffOPT_T_NoTaus_bb3_pol4_spillover0_noskill0_sep1_xgrowth0_etaa0.79_lgd1_lff0.png}
	\end{minipage}
	\begin{minipage}[]{0.32\textwidth}
		\centering{\footnotesize{(b) High skill hours worked}}
		%	\captionsetup{width=.45\linewidth}
		\includegraphics[width=1\textwidth]{../../codding_model/own_basedOnFried/optimalPol_190722_tidiedUp/figures/all_July22/hh_CompEffOPT_T_NoTaus_bb3_pol4_spillover0_noskill0_sep1_xgrowth0_etaa0.79_lgd0_lff0.png}
	\end{minipage}
	\begin{minipage}[]{0.32\textwidth}
		\centering{\footnotesize{(c) Low skill hours worked}}
		%	\captionsetup{width=.45\linewidth}
		\includegraphics[width=1\textwidth]{../../codding_model/own_basedOnFried/optimalPol_190722_tidiedUp/figures/all_July22/hl_CompEffOPT_T_NoTaus_bb3_pol4_spillover0_noskill0_sep1_xgrowth0_etaa0.79_lgd0_lff0.png}
	\end{minipage}
	\begin{minipage}[]{0.32\textwidth}
		\centering{\footnotesize{(d) Aggregate growth}}
		%	\captionsetup{width=.45\linewidth}
		\includegraphics[width=1\textwidth]{../../codding_model/own_basedOnFried/optimalPol_190722_tidiedUp/figures/all_July22/gAagg_CompEffOPT_T_NoTaus_bb3_pol4_spillover0_noskill0_sep1_xgrowth0_etaa0.79_lgd0_lff0.png}
	\end{minipage}
	\begin{minipage}[]{0.32\textwidth}
		\centering{\footnotesize{(e) Energy mix, $\frac{G}{F}$}}
		%	\captionsetup{width=.45\linewidth}
		\includegraphics[width=1\textwidth]{../../codding_model/own_basedOnFried/optimalPol_190722_tidiedUp/figures/all_July22/GFF_CompEffOPT_T_NoTaus_bb3_pol4_spillover0_noskill0_sep1_xgrowth0_etaa0.79_lgd0_lff0.png}
	\end{minipage}
	\begin{minipage}[]{0.32\textwidth}
		\centering{\footnotesize{(f) Utility}}
		%	\captionsetup{width=.45\linewidth}
		\includegraphics[width=1\textwidth]{../../codding_model/own_basedOnFried/optimalPol_190722_tidiedUp/figures/all_July22/SWF_CompEffOPT_T_NoTaus_bb3_pol4_spillover0_noskill0_sep1_xgrowth0_etaa0.79_lgd0_lff0.png}
	\end{minipage}
\end{figure}

 Figure \ref{fig:bench_lumpsum} contrasts the efficient allocation, the black solid graphs, the allocation under the benchmark policy, the orange-dashed graphs, and the optimal allocation when lump-sum transfers are available, the blue-dotted graphs. 
When lump-sum transfers of environmental tax revenues are in the policy set, the Ramsey planner can implement hours worked closer to the efficient allocation, panels (b) and (c). Consumption under the regime with lump-sum transfers, as well, mirrors the efficient level more closely see panel (a). 
Growth in the scenario with lump-sum transfers is at least as high as under the benchmark policy due to the use of regressive income taxes to accelerate growth. % I DONT KNOW WHY: However, the green-to-fossil energy ratio is slightly higher under the benchmark policy under the net-zero emission limit. The recomposing mechanism of the income tax via a relatively higher supply of high-skilled labor contributes to this finding. 
Utility gains from the availability of lump-sum transfers overall seem sizable especially under the net-zero emission limit; compare panel (f).

\begin{figure}[h!!]
	\centering
	\caption{Optimal policy in integrated regime and with lump-sum transfers}\label{fig:bench_lumpsum_pol}
	
	\begin{minipage}[]{0.32\textwidth}
		\centering{\footnotesize{(a) Income tax progressivity, $\tau_{\iota t}$}}
		%	\captionsetup{width=.45\linewidth}
		\includegraphics[width=1\textwidth]{../../codding_model/own_basedOnFried/optimalPol_190722_tidiedUp/figures/all_July22/comp_bb3_notaul4_OPT_T_NoTaus_taul_spillover0_noskill0_sep1_xgrowth0_etaa0.79_lgd1.png}
	\end{minipage}
	\begin{minipage}[]{0.32\textwidth}
		\centering{\footnotesize{(b) Environmental tax, $\tau_{ft}$}}
		%	\captionsetup{width=.45\linewidth}
		\includegraphics[width=1\textwidth]{../../codding_model/own_basedOnFried/optimalPol_190722_tidiedUp/figures/all_July22/comp_bb3_notaul4_OPT_T_NoTaus_tauf_spillover0_noskill0_sep1_xgrowth0_etaa0.79_lgd0.png}
	\end{minipage}
	\begin{minipage}[]{0.32\textwidth}
		\centering{\footnotesize{(c) Lump-sum transfers}}
		%	\captionsetup{width=.45\linewidth}
		\includegraphics[width=1\textwidth]{../../codding_model/own_basedOnFried/optimalPol_190722_tidiedUp/figures/all_July22/comp_bb3_notaul4_OPT_T_NoTaus_Tls_spillover0_noskill0_sep1_xgrowth0_etaa0.79_lgd0.png}
	\end{minipage}
\end{figure}


% finding 1) income tax to boost growth, 2) no use of recomposing effect of income tax
Figure \ref{fig:bench_lumpsum_pol} shows the optimal policy when lump-sum transfers are available. 
Now, the optimal income tax scheme is regressive. Since lump-sum transfers ensure a reduction in labor supply, the inefficiency in labor supply as a motive for progressive income taxes vanishes.
Instead, the motive to boost growth and consumption dominates: absent endogenous growth, the income tax remains untouched.\footnote{\ Compare results in the model with exogenous growth depicted in figure \ref{fig:lumpsum_xgr_vglNotaul} in appendix section \ref{app:lumps}.}  In fact, the allocation attained in the model without endogenous growth but lump-sum transfers is similar to the efficient one at a zero income tax progressivity; compare figure \ref{fig:lumpsum_xgr_vglNotaul}. Hence, the environmental policy does not use income tax regressivity to recompose production towards green energy by subsidizing high-skill supply.
 
 \tr{Has to be rewritten: it is rather, that the costs of more research seem to be too high in terms of disutility.}
Nevertheless, note, though, that regressivity of the optimal income tax reduces over time; compare the orange-dashed graph in panel (a) in figure \ref{fig:bench_lumpsum_pol}. This points to the income tax, again, being shaped by the environmental externality.
This is so even though consumption is inefficiently low and higher growth rates would be efficiency improving. I conclude from this observation that it is the environmental externality which makes it optimal to forfeit growth. 
The conflict between growth and emissions, however, does not seem to arise from growth itself, since the social planner satisfies the emission limit at higher growth rates. Instead, the issue stems from labor supply as a means to boost growth in the competitive economy. 
Then, the decreasing pattern of income tax regressivity can be rationalized as follows: as growth increases, more labor means more production and hence emissions. Therefore, boost growth becomes more costly in terms of emissions. Only growth which is concomitant with lower production is feasible in the market economy.\footnote{\ If the Ramsey planner had tools at hand to boost growth without necessarily boosting production, more growth would be optimal. \tr{\textit{Does a research subsidy imply more production?}}}

%It suggests itself that the conflict with growth does not stem from growth itself but rather that it is fostered in the competitive economy through a market size effect, that is, more production, since the social planner chooses higher technology levels. This would speak against progressive income taxes as growth accelerates. Thus, the environmental indeed prevents usage of the income tax to boost demand, but I conclude that it is not used with the goal to reduce emissions through the endogenous growth channel. Since growth as such does not pose the conflict to the emission limit. 
I conclude from this discussion that it is solely the environmental tax and not the income tax which addresses the environmental externality when lump-sum transfers are available.

\subsection{Sensitivity}
I will now briefly discuss sensitivity analyses to the quantitative exercise. 
\subsubsection{Wage elasticity of labor}

Recent papers have examined the wage elasticity of labor. \cite{Boppart2019LaborPerspectiveb} present evidence that hours worked per worker have been falling steadily over time 

\subsection{Research subsidy}
but finding should be similar to version without endogenous growth
\subsection{Changing emission limits calculation}
\subsection{Technology gap}
	\section{Conclusion}\label{sec:con}
Some scholars argue that  reductive policies are necessary to handle environmental limits \citep{Schor2005SustainableReductionb, VanVuuren2018AlternativeTechnologies, Bertram2018TargetedScenarios}, and the question has been raised whether consumption is too high \citep{Arrow2004AreMuch}. On the other hand, the focus of environmental policy discussions in economics rests on corrective environmental taxation. In the light of tightening environmental limits \citep{Rockstrom2009AHumanity, IPCC2022}, I study whether labor income taxes - as a reductive policy tool - can help mitigate externalities. 

In the analytical part of the paper, I show in a simple model that labor income taxes are  progressive as part of the optimal environmental policy. %The model does not feature inequality.
% Quantitative results
% baseline model
When environmental tax revenues are not redistributed lump sum, labor supply is inefficiently high. Then, income taxes serve to diminish hours worked closer to the efficient level. The result prevails absent income inequality.


% quantitative
In the second part of the paper, I analyze in a quantitative model with skill heterogeneity and endogenous growth whether the optimal labor income tax remains progressive. Again, there are no equity concerns, but workers are perfectly ensured against income differences. 
The optimal income tax is progressive to reduce inefficiently high hours worked. The quantitative model reveals that income taxes also serve as a substitute for corrective taxes. Knowledge spillovers from the non-energy sector render environmental taxes especially costly. 
Fossil taxes make energy relatively more expensive which directs research from non-energy to energy sectors. As the non-energy sector features the most research processes it is especially important for aggregate technology and knowledge spillovers. Using income taxes instead of fossil taxes to lower emissions allows to direct more research to the non-energy sector and to profit from knowledge spillovers.
In sum, however, the reduction in labor supply outweighs the positive effect on growth and consumption decreases compared to a scenario where no income tax is used. 

In the quantitative setting, the income tax affects the economic structure through two channels. First, because the fossil sector is comparably labor intense, a reduction in labor supply favors the green sector. This mechanism makes a higher tax progressivity optimal. However, the effect vanishes in equilibrium due to endogenous growth.
Second, a skill-recomposition channel makes green energy production more costly compared to fossil production. This effect arises from a skill bias in the green sector and high-skill labor being more responsive to income taxation. 
The second channel dominates the recomposing effect of  income tax progressivity in equilibrium. A market size effect amplifies the skill-recomposition channel directing research to the fossil sector. 

%Initially, the intention not to harm growth too much makes a lower progressivity optimal. As growth in the fossil sector accelerates due to the dynamic structure of endogenous growth too low progressive income taxes conflict with meeting the emission limit. As a result, optimal progressivity increases over time.
%The optimal path of income tax progressivity is decreasing, a feature mainly driven by endogenous growth. As a result, the optimal income tax progressivity and the optimal fossil tax seem to behave like substitutes in the quantitative model. 

%Skill heterogeneity depresses optimal tax progressivity due to the adverse recomposing effect of a lower high-to-low skill labor supply on the green-to-fossil energy ratio. A higher corrective tax is required to meet emission limits when there is only one skill type: with only one skill the supply of fossil-specific inputs increases thereby violating the emission limit.

%% lump-sum transfers
%When environmental tax revenues are redistributed lump-sum, the motive to use labor income taxes to deal with inefficiently high labor supply vanishes. Instead, income taxes serve to boost growth as long as this does not conflict with meeting emission limits. Therefore, they are regressive. 
%\tr{not true! it is rather that the more in research is not worth it given the dynamics! and decreasing utility gains}
%However, the regressivity decreases since more labor supply causes more emissions especially the more progressed the technology. With only a labor income tax as a tool to raise growth, accelerating technology growth is not feasible as it is concomitant with more production and emissions. 

% extensions
In an extension, I am planning to give the Ramsey planner the opportunity to limit working hours directly. The literature advocating a reduction in consumption levels \citep[e.g.,][]{Schor2005SustainableReductionb} proposes a restriction of hours worked as policy instrument to lower the consumption of resources.
Even though advocated in the literature, there is evidence for political difficulties in reducing working hours. In 2020, the French Citizens' Convention on Climate voted against reducing working hours as a measure to handle climate change. Potentially, ignorance about economic consequences is an explanation. The extension would serve to better understand economic consequences. 



\begin{comment}
\paragraph{Extension: What if the low skilled get a higher share \ar they reduce even less \ar more fossil input supply}

Redistribution to households with a higher marginal propensity to consume emissions counteracts the externality. This effect is amplified by a market size effect  of dirty goods. 

content...
\end{comment}

% I plan to discuss results under counterfactual parameter values to elicit the robustness of the main result: the preference of progressive labour taxation above higher fossil taxes. 
%First, the productivity gap between sectors might be driving the results. Second, I will abstract from endogenous growth to learn about the labour-supply-innovation channel as a driver of the optimal policy. Finally, I plan to study how results change as returns to research are increasing within sector. 
%Due to the endogeneity of technological growth in the model, the reduction in work effort fosters less research especially in the non-energy sector.  %However, more hours worked in the Ramsey model fostering research would violate the emission target. As a result, growth in technology and in consumption is inefficiently low in order to meet the emission target. 

\begin{comment}
To shed more light on the main findings, I plan conduct several additional quantitative experiments. First, I want to reduce the size of the emission target, second, I allow for a longer time frame until net-zero emissions have to be reached. The IPCC report states that for a temperature target of 2°C net-zero emissions have to be reached by 2070 only. How does this laxer target affect the importance of labour income taxes. Given the wider time frame, the green sector might be able to catch up and growth could continue. Finally, how does a change in spillovers shape the result? % \textit{(Question: I guess that substitutability is key here! Growth in green implies growths in fossil when goods are no perfect substitutes! )}
content...

%Another central aspect of the paper is the importance of inequality for the optimal environmental policy. How does household heterogeneity in labour supply shape the optimal environmental policy? First, I hypothesise that the skill bias of the green sector makes a less progressive income tax optimal. 
One main result of the paper is reduction of consumption and work effort as an optimal policy. So far, I have assumed that households are passive and preferences are fixed; there is no trade-off between environmental quality and consumption from a household perspective.
In an extension to the baseline model, I plan to depart from the representative agent assumption and explicitly model household heterogeneity. This setting allows to capture a change in household behaviour: A share of households is willing to voluntarily reduce consumption. I provide evidence for such behaviour using a representative Dutch dataset. More than 50\% of households are willing to reduce consumption in order to help the economy. Importantly, these households have a higher likelihood to work in the green sector. How does such a change in behaviour affect the optimal policy? Given the additional reduction in green-specific labour supply, the planner might find it optimal to set a more regressive tax to booster green production and research.    

\end{comment}

%However, data suggests, that households do care, and they express a willingness to reduce consumption.\footnote{\ The data I have studied comes from the Liss Panel, a representative sample of Dutch households, more than 50\% of participants indicate a readiness to change their behaviour to help the environment.} I want to study the effect of such behavioural  change on the optimal policy. Interestingly, households in high-skill jobs are more likely to declare their willingness to reduce. This linkage may intensify the trade-off between reduction and green labour supply. 


%1) BN and inequality
%2) preferences for labour
\begin{comment}
Preferences and the trade-off between leisure and consumption determining household behaviour seem to be key to the results. As argued by \cite{Boppart2019labourPerspectiveb}, the intensive margin of hours worked have been falling steadily over the last 130 years. They argue for the consistency of preferences which feature a slightly higher income effect than substitution effect. In the current model with log-utility and representative family framework,  the substitution effects offset each other. With the preferences suggested in \cite{Boppart2019labourPerspectiveb}, growth would affect hours worked, assumably changing the optimal policy. It could, for instance, be the case, that growth has to be slowed down even more, to prevent too high work efforts and consumption levels. % high-income, high-skill households might increase their labour supply with growth. 

content...
\end{comment}



%Finally, endogenising growth constitutes another interesting trade-off when the impact of fiscal policy is skill specific. 
%As regards growth, it seems reasonable to consider growth as a change in the substitutability of dirty and clean goods in the final consumption good. As it stands now, growth in the dirty sector results in emission growth, ceteris paribus. Growth might instead be associated with a more efficient use of dirty energy sources, so that more output can be generated at lower emissions.
%
%Think about effects of government using revenues for other consumption. Then reducing demand will diminish demand for the final good. 
%Broadly speaking, there are two channels through which distortionary labour taxation affects emissions. First, by affecting households' labour supply decision (efficiency channel) and second in a mechanical way by changing households disposable income. The latter effect cancels out when tax revenues are used by the government to consume the final output good. Allowing the government to recycle revenues in a different way than for final good consumption uncloses another instrument to reduce emissions. 

%Further ideas for extensions: include behavioural aspects: a voluntary reduction in demand, and a lower disutility from working in the green sector.
\begin{comment}
\paragraph{Ways forward}
How to introduce compositional effects:
\begin{enumerate}
	\item 	Utility function: With substitution and income effect not canceling (u(c)=$\frac{c^{1-\gamma}}{1-\gamma},\ \gamma\neq 1$), the wage rate might play a role, depends on GE effects.
	\item endogenising skill supply (rep agent chooses how much skill to supply, but this he already does... / might need to introduce structure as in HSV)
	\item government revenues are not used for final good consumption. Instead,  disposed of/ used for sth useful (this could be an extension and contribute to benefits of progressivity) THINK THIS ONLY CHANGES THE LEVEL TOO!
\end{enumerate}
\paragraph{Point 1 above}
change the utility function in the code to see what happens, if $\frac{Y_d}{Y_c}$ is constant in particular 
\paragraph{Point 3 above}
\textcolor{blue}{2) Government consumption wasted}
Letting the government not consume the final output good may alter the result. 
Now, the aggregate price level is determined endogenously as the goods market does not clear by Walras' law. 

In the equilibrium equations, I drop $p_t=1$ and use goods market clearing instead\\ $Y=c+\psi (x_c+x_d)$.

Blödsinn, only changes level

content...
\end{comment}
	\clearpage
\appendix
\section{Derivations and proofs}\label{app:derivations}

\subsection{Theory results \ref{sec:mod_an}}
\subsubsection{Useful relations in the simple model}\label{app:dervs_use}
\begin{align*}
\frac{\partial Gov}{\partial s}=\frac{\partial Y}{\partial F}\frac{\partial F}{\partial s}+\frac{\partial Y}{\partial G}\frac{\partial G}{\partial s}-\frac{\partial C}{\partial s}\\
\frac{\partial Gov}{\partial H}=\frac{\partial Y}{\partial F}\frac{\partial F}{\partial s}+\frac{\partial Y}{\partial G}\frac{\partial G}{\partial s}-\frac{\partial C}{\partial H}\\
\frac{\partial Gov}{\partial s}=\frac{\partial Y}{\partial s}-\frac{\partial C}{\partial s}\\
\frac{\partial Gov}{\partial s}=p_f F \frac{\partial \tau_F}{\partial s}+\tau_F F \frac{\partial p_f}{\partial s}+\tau_F p_f \frac{\partial F}{\partial s}\\
%\frac{\partial \tau_F}{\partial s}= -\frac{1-\varepsilon}{\varepsilon}\frac{1}{(1-s)^2}, \\
%\frac{\partial \tau_F}{\partial s}=p_f\frac{1-\varepsilon}{1-\tau_F}\frac{\partial \tau_F}{\partial s}\\
\frac{\partial F}{\partial s}=\frac{F}{s}
\\
\frp{Y}{H}= \frp{Y}{s}\frac{s}{H}+\frp{Y}{G}\frp{G}{Lg}
\\
\frp{G}{H}=-\frac{(1-s)}{H}\frp{G}{s}
\\\frp{G}{s}=-H\frp{G}{L_G}\\
\frp{F}{H}=\frac{s}{H}\frp{F}{s}\\
\frp{F}{s}=H\frp{F}{L_F}
\end{align*}

\subsubsection{Reduction in dirty labor share is efficient}
\begin{proof}
	With a negative externality of dirty production it has to hold that 
	\begin{align}
	\frp{Y}{F}\frp{F}{s}>-\frp{Y}{G}\frp{G}{s},
	\end{align}
	which can be rewritten to 
	\begin{align}\label{eq:mpl_eff}
	\frp{Y}{L_F}>\frp{Y}{L_G}. 
	\end{align}
	In the efficient allocation absent externality, marginal products of dirty and green labor are equalized. 
	Under decreasing returns to scale it holds that the left-hand side is decreasing in $L_F$ and the right-hand side of equation \ref{eq:mpl_eff} is decreasing in $L_G$. Hence, the adjustment to satisfy equation \ref{eq:mpl_eff} relative to the efficient allocation without externality requires a decrease in $L_F$ and/or a rise in $L_G$  .
	This reallocation is achieved by reducing $s$, since $L_F=sH$ and $L_G=(1-s)H$.	
\end{proof}


\begin{comment}
content...
\paragraph{If a reduction in dirty labor share is efficient, then the aggregate production function features decreasing returns to scale in labor}
\begin{proof}
	\textit{The proof rest on the assumption that returns to scale are symmetric across dirty and clean production; either both decreasing or both are non-decreasing.}
It holds by assumption that $s_{FB,E>0}<s_{FB,E=0}$, where $E>0$ indicates that the externality is active. 
Assume by contradiction that the aggregate production function features non-decreasing returns to scale. This implies that:
\begin{align}
\left. \frp{Y}{L_F} \right|_{s_{FB,E>0}}\leq \left. \frp{Y}{L_F} \right|_{s_{FB,E=0}},\\
\left. \frp{Y}{L_G} \right|_{s_{FB,E>0}}\geq \left. \frp{Y}{L_G} \right|_{s_{FB,E=0}}.
\end{align}
When there is no externality, the efficient allocation is characterized by
\begin{align}
\left. \frp{Y}{L_F} \right|_{s_{FB,E=0}}= \left. \frp{Y}{L_G} \right|_{s_{FB,E=0}}.
\end{align}
Using the inequalities above yields
\begin{align}
\left. \frp{Y}{L_F} \right|_{s_{FB,E>0}}\leq \left. \frp{Y}{L_G} \right|_{s_{FB,E>0}}.
\end{align}
This contradicts the optimality condition which requires 
\begin{align}
\left. \frp{Y}{L_F} \right|_{s_{FB,E>0}}> \left. \frp{Y}{L_G} \right|_{s_{FB,E>0}}.
\end{align}
Hence, when a reduction in the dirty labor share is efficient, then the aggregate production function features decreasing returns to scale in both labor input goods. 
\end{proof}
\end{comment}

\subsubsection{The social cost of pollution and the Pigouvian tax rate}\label{app:scp}

The social cost of pollution in my model is defined as the marginal price the representative household is willing to pay for a marginal reduction in dirty production. That is, the household maximises over dirty production for which a market exists.

The household's problem is determined as
\begin{align}
\underset{C,H,F}{\max} U(C,H,F)-\mu \left(C+\tilde{p}_FF-Y(H)\right).
\end{align}
Where $\mu$ is the Lagrange multiplier. Taking the derivative with respect to dirty production  and with respect to consumption yields
\begin{align}
U_F=\mu \tilde{p}_F,\\
U_C=\mu.
\end{align}
Substituting the Lagrange multiplier gives the negative of the equilibrium price the household is willing to pay for a reduction in dirty prodction: $\tilde{p}_F=\frac{U_F}{U_C}$. Since the environmental tax in the model is a percentage of revenues, the price producers pay per unit of dirty production is $\tau_F p_F$. Thus, the social cost of pollution to be deducted from to producers' revenues in percent is $\tau^{Pigou}=\frac{-U_F}{U_Cp_F}$.


\subsubsection{With a positive environmental tax, the wage rate in the competitive equilibrium is below the marginal product of labor}\label{app:wageMPL}

The aggregate marginal product of labor is defined as
\begin{align}
MPL&= \frp{Y}{H}.
\end{align}
This expression can be rewritten using relations of derivatives summarized in \ref{app:dervs_use} as follows.
\begin{align}
&= \frp{Y}{F}\frp{Y}{H}+\frp{Y}{G}\frp{G}{H}\\
&= \frp{Y}{F}\frp{F}{L_F}s+\frp{Y}{G}\frp{G}{L_G}(1-s)\\
&= \frp{Y}{G}\frp{G}{L_G}+ s\left(\frp{Y}{F}\frp{F}{L_F}-\frp{Y}{G}\frp{G}{L_G}\right).\label{eq:mpl_opt}
\end{align}
The term in brackets is positive under the optimal policy as can be seen from the first order condition with respect to $s$, equation \ref{eq:sbs}:
\begin{align}
\frp{Y}{F}\frp{F}{L_F}-\frp{Y}{G}\frp{G}{L_G}=\frac{1}{H}\left(\frp{Y}{F}\frp{F}{s}+\frp{Y}{G}\frp{G}{s}\right)=\frac{1}{H}\left(\frac{-U_F\frp{F}{s}}{U_C}\right)>0.
\end{align}
The inequality holds since the externality of polluting production is negative. %, above expression is positive.
%Therefore, the marginal product of labor in the efficient allocation equals
Now note that the first summand in equation \ref{eq:mpl_opt} is the competitive wage rate.  Hence $w<MPL$.

The gap between the wage rate and the marginal product of labor equals the gap between the marginal products of labor across sectors times the relative size of the dirty sector. 

\subsubsection{Sufficiency of the environmental tax when environmental tax revenues are redistributed lump sum}\label{app:incometax0}

Noticing that $\frac{\partial Y}{\partial H}= \frac{\partial Y}{\partial s}\frac{s}{H}-\frac{\partial Y}{\partial G}\frac{\partial G}{\partial s}\frac{1}{H}$ and that $\frac{\partial F}{\partial H}=\frac{\partial F}{\partial s}\frac{s}{H}$, and substituting equation \ref{eq:sbs} in equation \ref{eq:sbh} yields
\begin{align}\label{eq:pigou}
-U_C \frac{\partial Y}{\partial G}\frp{G}{L_G}=-U_H.
\end{align}
Hence, if the environmental tax is set to guarantee that condition \ref{eq:sbh} holds, then optimal hours worked only trade-off the disutility from labor and the utility from more consumption when environmental tax revenues are redistributed lump-sum.

Equation \ref{eq:pigou} also holds for the social planner allocation simplifying the second first order condition, equation \ref{eq:fbh}.


Substituting $U_H$ from household optimality, equation \ref{eq:hsup}, and the clean sectors' profit maximizing condition from equations \ref{eq:profmax} yields
\begin{align}
1=1-\tau^*_\iota.
\end{align}
Hence, $\tau^*_\iota =0$ from which follows that $\lambda =1$ so that the income tax scheme is a flat tax rate equal to zero; the labor income tax is not used in optimum.

%\subsubsection{Simplifying social planner's first order conditions}
%
%The social planner's first order condition on labor can be rewritten as in the previous section to
%\begin{align}
%-U_H=U_C\frac{\partial Y}{\partial G}\frp{G}{L_G}
%\end{align}
\subsubsection{Proof proposition \ref{prop:1}: Absent lump-sum transfers, hours are inefficiently high under decreasing returns to scale}\label{app:nolumpsum_hourshigh}
\begin{proof}\textit{Absent lump-sum transfers, hours are inefficiently high when the environmental tax implements efficient share of dirty production and the aggregate production function features decreasing returns to scale in labor inputs.}
	
	This proof proceeds by contradiction. 
	Assume by contradiction that $H^*\leq H_{FB}$. 
	It has to hold that 
	\begin{align}
	-U_H^*\leq -U_{H,FB}.
	\end{align} 
	
	Substituting the households' optimal labor supply and the social planner's first order condition for hours, equation \ref{eq:fbh_simp} yields
	\begin{align}\label{eq:prH}
	U_C^*w^* \leq U_{C,FB}\frp{Y_{FB}}{G_{FB}}\frp{G_{FB}}{L_{G,FB}}.
	\end{align}
	
	Rewriting equation \ref{eq:prH} above yields
	\begin{align}
	\frac{U_C^*}{U_{C,FB}}\leq \frac{\frp{Y_{FB}}{G_{FB}}\frp{G_{FB}}{L_{G,FB}}}{\frp{Y^*}{G^*}\frp{G^*}{L^*_{G}}},
	\end{align}
	where I replaced $w^*=\frp{Y^*}{G^*}\frp{G^*}{L^*_{G}}$.
	
	By assumption $s^*=s_{FB}$, $H^*\leq H_{FB}$, and the aggregate production function is increasing in its inputs. It follows that output is higher in the efficient allocation $Y_{FB}\geq Y^*$ and hence $C^*<C_{FB}$, since $Gov>0$ in the competitive equilibrium. By additive separability of the utility function and strict concavity with respect to consumption, we have that $\frac{U_C^*}{U_{C,FB}}>1$.
	
	Now note that $H^*\leq H_{FB}$ implies  $L_G^*\leq L_{G,FB}$, since the dirty labor share is equal. Under decreasing returns to scale of aggregate production to clean labor, it holds that the right-hand side is below or equal unity.Thus,
	\begin{align}
	\frac{U_C^*}{U_{C,FB}}>1\geq \frac{\frp{Y_{FB}}{G_{FB}}\frp{G_{FB}}{L_{G,FB}}}{\frp{Y^*}{G^*}\frp{G^*}{L^*_{G}}}. 
	\end{align}
	A contradiction to the assumption that $H^*\leq H_{FB}$. Hence, it has to hold that $H^*>H_{FB}$. 
\end{proof}


\subsubsection{Derivation $\tau_F^*$ without lump-sum transfers}\label{app:reiv_tauf}
	
Divide the Ramsey planner's first order condition with respect to $s$, equation \ref{eq:sbs}, by $U_C$ and $\frp{Y}{F}\frp{F}{s}$. Solving for $1+\frac{\frac{\partial Y}{\partial G}\frac{\partial G}{\partial s}}{\frac{\partial Y}{\partial F}\frac{\partial F}{\partial s}}$, which equals $\tau_F$, yields the desired result:

\begin{align}
\tau_{F}=SCC + \frp{Gov}{s}.
\end{align}

\begin{comment}
The latter summand can be rewritten to 
\begin{align}
\frp{Gov}{s}= \frp{Y}{s}+H^2 \frp{\left(\frp{Y}{L_G}\right)}{L_G}.
\end{align}
Where under decreasing returns to scale the second summand is negative and the first is positive. \textit{To be continued.} 

content...
\end{comment}
\subsubsection{Derivation $\tau_l$ without lump-sum transfers }\label{app:subsub_nltaul}

\begin{proof}\textit{Absent lump-sum transfers, the optimal income tax scheme is progressive}
Following similar steps as in section \ref{app:incometax0}, the optimal labor income tax progressivity parameter is given by
\begin{align}
\tau_{\iota}^*=\frac{\frac{s}{H}\frac{\partial Gov}{\partial s}- \frac{\partial Gov}{\partial H}}{\frac{\partial Y}{\partial G}\frac{\partial G}{\partial s}\frac{1}{H}}.
\end{align}

	Using the market clearing condition for final output to replace government spending and noticing the relations of derivatives with respect to aggregate labor supply and the dirty labor share, one can write above expression as
	\begin{align}
	\tau_{\iota}=1-\frac{H\frp{C}{H}-s\frp{C}{s}}{wH}.
	\end{align}
	Substituting $\frp{C}{H}=H\frp{w}{H}+w$ and $\frp{C}{s}=H\frp{w}{s}$ from the household's budget constraint gives
	\begin{align}
	\tau_{\iota}=\frac{s}{w}\frp{w}{s}-\frac{H}{w}\frp{w}{H}.
	\end{align}
In a next step, I explicitly solve for $\frp{w}{s}$ and $\frp{w}{H}$, where I use that $w=\frp{Y}{G}\frp{G}{L_G}$ in equilibrium.

\begin{align}
\frp{w}{H}=\left(\frp{G}{L_G}\right)^2\frac{\partial^2Y}{\partial G^2}(1-s)+\frp{Y}{G}\frac{\partial ^2G}{\partial L_G^2}(1-s)+\frp{G}{L_G}\frac{\partial^2 Y}{\partial G \partial F}s\\
%%%%
\frp{w}{s}= \left(\frp{G}{L_G}\right)^2\frac{\partial ^2Y}{\partial G^2}(-H)+\frp{G}{L_G}\frac{\partial ^2Y}{\partial G \partial F}H+\frp{Y}{G}\frac{\partial ^2 G}{\partial L_G^2}(-H)
\end{align}
substituting derivatives and canceling terms yields:
\begin{align}
\tau_\iota= -\frac{H}{w}\frp{\left(\frp{Y}{L_G}\right)}{L_G}.=-\frac{H}{w}\left(\left(\frp{G}{L_G}\right)^2\frac{\partial ^2Y}{\partial G^2}+\frp{Y}{G}\frac{\partial ^2G}{\partial L_G ^2}\right).
\end{align}
Under the assumption of decreasing returns to scale of aggregate production with respect to green labor the term in brackets is negative, and it holds that $\tau_\iota >0$ and the optimal income tax rate is progressive. 

For intuition, note that the right-hand side of the previous expression equals the partial derivative of the wage rate with respect to the dirty labor share under the assumption that dirty production is fixed divided by the wage rate:
\begin{align}
\tau^*_\iota =\left. \frac{1}{w}\frp{w}{s} \right|_{F=\bar{F}}.
\end{align}
%Since the presence of the environmental tax artificially increases labor in the green sector depressing the wage rate (under the assumption of decreasing returns to scale), the wage rate rises by a reduction of the green labor share. 

The equation makes clear that environmental taxation and the labor income tax are complements. When the environmental tax rises, thereby increasing the share of labor allocated to the green sector, the marginal product of green labor decreases further. A marginal reduction in the green labor share would increase the wage rate more the higher the green labor share, hence, the optimal labor tax progressivity increases with the environmental tax. 
Secondly, the wage rate decreases with $\tau_F$ which as well inflates the optimal labor tax progressivity. 
	\end{proof}

\subsubsection{Proof proposition: Infeasibility of efficient allocation}\label{app:ineff}
\begin{proof}\textit{The efficient allocation is infeasible (under the assumption of constant or decreasing returns to scale)}
	To prove this claim, I assume that the government chooses the optimal policy; which is the highest social welfare the Ramsey planner can achieve. I show that the optimal policy does not satisfy the social planner's allocation. Since the social planner could have chosen the Ramsey planner's allocation  but did not, it follows that the social planner's allocation features a higher social welfare.
	
	For the optimal allocation to be efficient, it must be the case that $s^*=s_{FB}$, (i) $C^*=C_{FB}$, and (ii) $H^*=H_{FB}$. I show that, under the assumption that $s^*=s_{FB}$, either (i) or (ii) can hold at a time by demonstrating that assuming (i) violates (ii) and vice versa.
	
	
	
	%\begin{lemma}\textit{$\tau_F=0$ is not optimal}
	%When $\tau_F=0$ then $Gov=0$ and $\frp{Gov}{s}=0$. Furthermore, market forces then imply that the marginal products of labor are equal so that $\frp{Y}{F}\frp{F}{s}=-\frp{Y}{G}\frp{G}{s}$. Substituting this in equation \ref{eq:sbs} yields
	%\begin{align}
	%0=-U_F\frp{F}{s}>0,
	%\end{align}
	%a contradiction. 
	%\end{lemma}
	%
	%\textit{(i) Assume $C^*=C_{FB}$ and $s^*=s_{FB}$:}
	%Since $\tau_F\neq0$, it follows that $Gov>0$ and hence $Y^*=C^*+Gov>Y_{FB}$. Since the allocation of labor is the same in the efficient and the optimal allocation and output is rising in labor, it follows that $H^*>H_{FB}$. 
	%\tr{Missing: if $\tau_F<0$ then $Gov<0$ }
	
	If $s^*=s_{E>0,FB}<s_{E=0,FB}$ then it must be the case that the environmental tax is positive to sustain a gap between marginal productivities in the dirty and the clean sector: $\tau_F>0$. Then, $Gov=\tau_Fp_fF>0$. 
	First assume that (i) holds true: $C^*=C_{FB}$. From the good's market clearing condition and resource constraint of the social planner's problem it follows that
	$Y^*-Gov=C^*=C_{FB}=Y_{FB}$, due to  $Gov>0$ we have that $Y^*>Y_{FB}$. Since hours are the only production input, positively affect output, and $s^*=s_{FB}$ the higher output in the optimal allocation implies that $H^*>H_{FB}$. A violation of condition (ii). 
	
	Assume now that condition (ii) holds: $H^*=H_{FB}$. Since $s^*=s_{FB}$ by assumption it holds that $Y^*=Y_{FB}$ and, by the same argument as before: $Gov>0$. Thus, by the resource and market clearing condition: $C_{FB}=Y_{FB}>Y^*-Gov=C^*$. When labor supply is efficient, then consumption is inefficiently low; condition (i) is violated. 
	
	\begin{comment} (Proof building on first order conditions)
	Assume, 
	The social planner's first order condition on labor supply can be written as
	\begin{align}
	-U_{H, FB}=U_{C, FB}\frp{Y}{G}_{FB}\frp{G}{L_G}_{FB}
	\end{align}
	and optimal labor supply is determined by
	\begin{align}
	-U^*_{H}&=U^*_C(1-\tau_\iota)w
	\end{align}
	Equalizing yields
	\begin{align}
	U_C^*(1-\tau_\iota)w=U_{C,FB}\frp{Y}{G}_{FB}\frp{G}{L_G}_{FB},
	\end{align}
	a condition for optimal labor supply to be efficient. 
	
	In the following, I demonstrate that (i) assuming $C^*=C_{FB}$ violates the condition above and $H^*\neq H_{FB}$ and that (ii) assuming $H^*=H_{FB}$ results in $C^*<C_{FB}$. 
	
	\textit{(i) Assume $C^*=C_{FB}$:}
	then
	\begin{align}
	(1-\tau_\iota)w=\frp{Y}{G}_{FB}\frp{G}{L_G}_{FB}.
	\end{align}
	Assume by contradiction that $H^*=H_{FB}$, since $s^*=s_{FB}$ by assumption, it follows that $w=\frp{Y}{G}_{FB}\frp{G}{L_G}_{FB}$. 
	Since $\tau_\iota\neq 0$ under constant or decreasing returns to scale, it holds that $H^*<H_{FB}$, a contradiction. 
	
	
	%Hence,
	%\begin{align}
	%(1-\tau_\iota)w<\frp{Y}{G}_{FB}\frp{G}{L_G}_{FB}.
	%\end{align}
	%
	%Labor supply in the competitive equilibrium is lower than in the efficient allocation when consumption is equal under the optimal policy. WHY?
	%It follows, that optimal labor supply does not equal its efficient counterpart when optimal consumption is efficient.
	
	\textit{(ii) Assume $H^*=H_{FB}$:} 
	It follows that 
	\begin{align}
	\frac{U_C^*}{U_{C,FB}}=\frac{\frp{Y}{G}_{FB}\frp{G}{L_G}_{FB}}{w}\frac{1}{1-\tau_\iota}=\frac{1}{1-\tau_\iota}>1.
	\end{align}
	From concavity of the utility function it follows that $C^*<C_{FB}$. 
	
	content...
	\end{comment}
\end{proof}

\subsubsection{Proofs proposition \ref{prop:3}}\label{app:proofintegrated}
\begin{proof} \textit{The optimal income tax scheme is progressive}\\ % if the optimal environmental tax is positive.}\\
	Under the new policy, the household's labor supply is determined by
	\begin{align}
	-U_H=\frac{U_C (1-\tau_{\iota})(wH+\tau_F p_fF)}{H}.
	\end{align}
	Expressing the derivatives in the Ramsey planner's first order condition with respect to hours as derivatives with respect to the dirty labor share, $s$, and substituting the first order condition with respect to $s$ yields:
	\begin{align}
	U_C \frp{Y}{G}\frp{G}{L_G}=-U_H.
	\end{align}
	%This equation is equivalent to the social planner's first order condition on hours, equation \ref{eq:fbh}. The optimal policy is to choose
	%\tr{Does this give a hint to why inefficiency without redistribution? The Ramsey planner's foc and household optimality always coincide. But, when Gov does not cancel the two do not coincide! ? the two do not coincide, Bcs consumption is too low so that $U_C$ too high which increases}
	Noticing that $\frp{Y}{G}\frp{G}{L_G}=w$ and replacing household's labor supply condition gives
	\begin{align}
	& w=\frac{(1-\tau_\iota)Y}{H}\\
	\Leftrightarrow\ & \tau_\iota=1-\frac{wH}{Y}. 
	\end{align} 
	Since $Y=C=wH+\tau_Fp_fF$ from the market clearing and household budget constraint, it follows that $wH<Y$ whenever $\tau^*_F>0$. Hence, $\tau_F^*>0$ implies $\tau^*_{\iota}>0$.
	%
	%Observe that $Y\geq MPL \times H$, where $MPL$ stands in for the marginal product of labor, if the aggregate production function features decreasing or constant returns to scale. Under such a production function one can rewrite the last expression as
	%\begin{align}
	%\tau_{\iota}=1-\frac{wH}{Y}\geq 1-\frac{w H}{MPL \times H}
	%\end{align}
	% Note further that the marginal product of labor exceeds the wage rate whenever the environmental tax is different from zero; compare the disucssion in subsection \ref{subsec:Rams}. It follows that the right-hand side is positive, hence
	%\begin{align}
	%\tau_{\iota}>0,
	%\end{align}
	The optimal tax scheme is progressive.
\end{proof}

\begin{proof}\textit{The optimal allocation is efficient}
	
	The idea of this proof is to show that the efficient allocation is attainable for the Ramsey planner. Since the social planner could implement any competitive allocation (which necessarily satisfies the resource constraint) and has the same objective function, the efficient allocation maximizes the Ramsey problem. 
	
	To show that the efficient allocation is feasible, I assume that $s^*=s_{FB}$. Showing that $H^*=H_{FB}$ and $C^*=C_{FB}$ are a solution to the Ramsey problem, proves that the optimal policy implements the efficient allocation for two reasons. First, by the argument in the previous paragraph any competitive allocation is a potential candidate solution to the social planner's problem and the social planner has the same objective function. Second, due to strict concavity of the utility and strict monotonicity of the production function \textit{(so that more input means more output)}, the solution is also unique.
	
	When $H^*=H_{FB}$ then $C^*=C_{FB}$ since $s^*=s_{FB}$ by assumption. It now show that under this allocation optimal labor supply, indeed, is efficient, that is:
	\begin{align}
	U_C^*\frp{Y^*}{G^*}\frp{G^*}{L^*_{G}} = U_{C,FB}\frp{Y_{FB}}{G_{FB}}\frp{G_{FB}}{L_{G,FB}}.
	\end{align}
	
	From the assumed allocation it follows that $U_C^*=U_{C,FB}$ and $\frp{Y_{FB}}{G_{FB}}\frp{G_{FB}}{L_{G,FB}}=\frp{Y^*}{G^*}\frp{G^*}{L^*_{G}}$ and above condition is satisfied. 
	
	It remains to show that under the assumed allocation, $s^*=s_{FB}$ holds true. Since $Gov=0$ the Ramsey planner's first order condition with respect to $s$ equals that of the social planner. Since production and marginal utilities in the optimal allocation equal their counterparts in the efficient allocation, it has to holds that $\tau_F^*$ implements $s^*=s_{FB}$.  
	%Second, efficiency of labor supply, i.e., $H^*=H_{FB}$, as the only solution of the Ramsey planner's problem follows from demonstrating that both (i) $H^*>H_{FB}$ and (ii) $H^*<H_{FB}$ result in a contradiction under the assumption that $s^*=s_{FB}$.
	
	%Assume by contradiction that (i), $H^*>H_{FB}$. 
\end{proof}


	%-------------------------------------
	\clearpage
	\bibliography{../../../bib_2_0}
	\addcontentsline{toc}{section}{References}
\end{document}