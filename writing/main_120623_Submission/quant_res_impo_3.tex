\section{Quantitative results}\label{sec:res}

This section presents and discusses the quantitative results. 
In Section \ref{subsec:exp}, I use the model to learn how a constant carbon tax affects the economy and how it interacts with a tax on labor income. Section \ref{subsec:meetlim} calculates how high a carbon tax is necessary to meet the emission limit. I find that an increasing carbon tax is necessary to counter market forces directing production and research towards the fossil sector. 
Section \ref{subsec:mr} goes one step further asking  how the government can optimally satisfy the emission limit using carbon and labor income taxes. Results show that a combination of the two instruments is optimal throughout. 

%I focus on analyzing the mechanisms and welfare benefits from integrating the income tax scheme into the environmental policy. I also discuss the costs of not using lump-sum transfers.




\subsection{Optimal policy}\label{subsec:mr}


This section seeks to answer the question how a benevolent planner optimally attains the emission limit. After showing the results in Section \ref{sec:optres}, I discuss the motive behind the optimal policy in Section \ref{subsec:dis}. The optimal policy consists of a combination of labor income and carbon taxes. The reason is that when the carbon tax is used to target the direction of research, the wage rate no longer captures the social costs of labor. The labor tax boosts or curbs labor supply to correct for the externality of work through emissions. 
%This section depicts results on the optimal policy followed by the implied allocation in the benchmark model where environmental tax revenues are redistributed via the income tax scheme. 

\subsubsection{Results}\label{sec:optres}
Figure \ref{fig:optPol} depicts the optimal policy.
To meet the emission limit suggested by the IPCC, the optimal policy taxes labor until 2044 (Panel (a)). The labor tax  turns into a subsidy from 2045 onward, $\tau_{\iota t}<0$. 
\begin{figure}[h!!]
	\centering
	\caption{Optimal policy }\label{fig:optPol}
	\begin{subfigure}{0.4\textwidth}
		\caption{Average marginal income tax rate }
		%	\captionsetup{width=.45\linewidth}
		\includegraphics[width=1\textwidth]{dTaulAv_OPT_T_NoTaus_COMPtaul_regime4_spillover0_knspil0_noskill0_sep0_xgrowth0_PV1_etaa0.79_lgd0.png}
	\end{subfigure}
\begin{minipage}[]{0.1\textwidth}
	\
\end{minipage}
	\begin{subfigure}{0.4\textwidth}
		\caption{Tax per ton of carbon in 2022 US\$ }
		%	\captionsetup{width=.45\linewidth}
		\includegraphics[width=1\textwidth]{Single_periods12_OPT_T_NoTaus_Tauf_regime4_spillover0_knspil0_noskill0_sep0_xgrowth0_extern0_PV1_sizeequ0_GOV0_etaa0.79.png}
	\end{subfigure}
\floatfoot{Notes: {The x-axis indicates the first year of the 5 year period to which the variable value corresponds. A vertical line indicates when the net-zero emission limit becomes binding.}}
\end{figure} 
Consider now Panel (b). The optimal carbon tax increases over time and jumps to a higher level when the net-zero emission limit is introduced in 2050.
In 2020, the carbon tax equals US\$987 and rises steadily to US\$1,326 in the 2045-2049 period.  As the emission limit declines to net-zero, the tax rapidly surges to US\$2,833 and gradually increases afterwards reaching US\$3,2186 in 2070-2074. 
%\paragraph{Efficient and optimal allocation}\label{subsec:notaul}

Figure \ref{fig:optAll_percLf_dyn} presents adjustments of key variables under the first-best (efficient) and the second-best (optimal) policy relative to the laissez-faire allocation.\footnote{ I formulate the social planner's problem in Appendix \ref{app:sp_prob}. }  Dashed graphs show the efficient and solid graphs the optimal allocation.
The efficient allocation is chosen by a social planner who picks allocations irrespective of market forces. It can be perceived as the allocation the Ramsey planner intents to implement. However, she may not be able to achieve the efficient allocation due to the reliance on (a limited number of) tax instruments and market forces.\footnote{ Figure \ref{fig:LF} in Appendix \ref{app:quant_res_opt}  shows the laissez-faire, the efficient, and the optimal allocation in levels.}

\begin{figure}[h!!!]
	\centering \caption{Efficient and optimal allocation relative to laissez-faire}
\label{fig:optAll_percLf_dyn}
	\begin{subfigure}[]{1\textwidth}	
		\centering\footnotesize{\textbf{In percentage deviation from laissez-faire}}\\ \vspace{2mm}
	\begin{subfigure}[]{0.4\textwidth}
		\caption{Consumption}
		%	\captionsetup{width=.45\linewidth}
		\includegraphics[width=1\textwidth]{C_PercentageLFDyn_Target_regime4_knspil0_spillover0_noskill0_sep0_xgrowth0_PV1_etaa0.79_lgd1.png}
	\end{subfigure}
\begin{minipage}[]{0.1\textwidth}
	\ 
\end{minipage}
	\begin{subfigure}[]{0.4\textwidth}
				\caption{Average hours worked }
		%	\captionsetup{width=.45\linewidth}
		\includegraphics[width=1\textwidth]{Hagg_PercentageLFDyn_Target_regime4_knspil0_spillover0_noskill0_sep0_xgrowth0_PV1_etaa0.79_lgd0.png}
	\end{subfigure}

\vspace{3mm}
\begin{subfigure}[]{0.4\textwidth}
				\caption{Green-to-fossil energy ratio in k}
	%	\captionsetup{width=.45\linewidth}
	\includegraphics[width=1\textwidth]{GFF_PercentageLFDyn_Target_regime4_knspil0_spillover0_noskill0_sep0_xgrowth0_PV1_etaa0.79_lgd0.png}
\end{subfigure}
\begin{minipage}[]{0.1\textwidth}
	\ 
\end{minipage}
\begin{subfigure}[]{0.4\textwidth}
			\caption{ Non-energy scientists share}
	%	\captionsetup{width=.45\linewidth}
	\includegraphics[width=1\textwidth]{snS_PercentageLFDyn_Target_regime4_knspil0_spillover0_noskill0_sep0_xgrowth0_PV1_etaa0.79_lgd0.png}
\end{subfigure}
\end{subfigure}

\vspace{3mm}
\begin{subfigure}[]{1\textwidth}
	\centering	\footnotesize{{\textbf{In levels}}}\\ \vspace{2mm}
\begin{subfigure}[]{0.4\textwidth}
				\caption{Fossil scientists}
	%	\captionsetup{width=.45\linewidth}
	\includegraphics[width=1\textwidth]{sff_CompEffOPT_T_NoTaus_regime4_opteff_knspil0_spillover0_noskill0_sep0_xgrowth0_countec0_PV1_etaa0.79_lgd1_lff1.png}
\end{subfigure}
\begin{minipage}[]{0.1\textwidth}
\ 
\end{minipage}
\begin{subfigure}[]{0.4\textwidth}
			\caption{Green scientists}
%	\captionsetup{width=.45\linewidth}
\includegraphics[width=1\textwidth]{sg_CompEffOPT_T_NoTaus_regime4_opteff_knspil0_spillover0_noskill0_sep0_xgrowth0_countec0_PV1_etaa0.79_lgd0_lff1.png}
\end{subfigure}
%\begin{subfigure}[]{0.4\textwidth}
%	\caption{Fossil scientists}
%	%	\captionsetup{width=.45\linewidth}
%	\includegraphics[width=1\textwidth]{sn_CompEffOPT_T_NoTaus_regime4_opteff_knspil0_spillover0_noskill0_sep0_xgrowth0_countec0_PV1_etaa0.79_lgd1_lff1.png}
%\end{subfigure}
\end{subfigure}
	\floatfoot{Notes: { Panels (a) to (d) show the percentage deviation of the allocation resulting under the optimal policy, the black solid graph, and the efficient allocation, the black dashed graph, relative to the laissez-faire allocation. Panels (e) and (f) show fossil and green scientists in levels. The dotted graph refers to the laissez-faire allocation. A vertical line indicates when the net-zero emission limit becomes binding.
	}}
\end{figure} 


%\begin{figure}[h!!!]
%	\centering \caption{Efficient and optimal allocation relative to laissez-faire: 	new model}
%	\label{fig:optAll_percLf_dyn}
%	\begin{subfigure}[]{1\textwidth}	
%		\centering\footnotesize{\textbf{In percentage deviation from laissez-faire}}\\ \vspace{2mm}
%		\begin{subfigure}[]{0.4\textwidth}
%			\caption{Consumption}
%			%	\captionsetup{width=.45\linewidth}
%			\includegraphics[width=1\textwidth]{NC_C_PercentageLFDyn_T_regime4_spillover0_noskill0_sep0_xgrowth0_PV1_etaa0.79_lgd1.png}
%		\end{subfigure}
%		\begin{minipage}[]{0.1\textwidth}
%			\ 
%		\end{minipage}
%		\begin{subfigure}[]{0.4\textwidth}
%			\caption{Average hours worked }
%			%	\captionsetup{width=.45\linewidth}
%			\includegraphics[width=1\textwidth]{NC_Hagg_PercentageLFDyn_T_regime4_spillover0_noskill0_sep0_xgrowth0_PV1_etaa0.79_lgd0.png}
%		\end{subfigure}
%		
%		\vspace{3mm}
%		\begin{subfigure}[]{0.4\textwidth}
%			\caption{Green-to-fossil energy ratio in k}
%			%	\captionsetup{width=.45\linewidth}
%			\includegraphics[width=1\textwidth]{NC_GFF_PercentageLFDyn_T_regime4_spillover0_noskill0_sep0_xgrowth0_PV1_etaa0.79_lgd0.png}
%		\end{subfigure}
%		\begin{minipage}[]{0.1\textwidth}
%			\ 
%		\end{minipage}
%		\begin{subfigure}[]{0.4\textwidth}
%			\caption{ Non-energy scientists share}
%			%	\captionsetup{width=.45\linewidth}
%			\includegraphics[width=1\textwidth]{NC_snS_PercentageLFDyn_T_regime4_spillover0_noskill0_sep0_xgrowth0_PV1_etaa0.79_lgd0.png}
%		\end{subfigure}
%	\end{subfigure}
%	
%	\vspace{3mm}
%	\begin{subfigure}[]{1\textwidth}
%		\centering	\footnotesize{{\textbf{In levels}}}\\ \vspace{2mm}
%		\begin{subfigure}[]{0.4\textwidth}
%			\caption{Fossil scientists}
%			%	\captionsetup{width=.45\linewidth}
%			\includegraphics[width=1\textwidth]{NC_sff_CompEffT_regime4_opteff_spillover0_noskill0_sep0_xgrowth0_countec0_PV1_etaa0.79_lgd1_lff1.png}
%		\end{subfigure}
%		\begin{minipage}[]{0.1\textwidth}
%			\ 
%		\end{minipage}
%		\begin{subfigure}[]{0.4\textwidth}
%			\caption{Green scientists}
%			%	\captionsetup{width=.45\linewidth}
%			\includegraphics[width=1\textwidth]{NC_sg_CompEffT_regime4_opteff_spillover0_noskill0_sep0_xgrowth0_countec0_PV1_etaa0.79_lgd0_lff1.png}
%		\end{subfigure}
%		%\begin{subfigure}[]{0.4\textwidth}
%		%	\caption{Fossil scientists}
%		%	%	\captionsetup{width=.45\linewidth}
%		%	\includegraphics[width=1\textwidth]{sn_CompEffOPT_T_NoTaus_regime4_opteff_knspil0_spillover0_noskill0_sep0_xgrowth0_countec0_PV1_etaa0.79_lgd1_lff1.png}
%		%\end{subfigure}
%	\end{subfigure}
%	\floatfoot{Notes: \footnotesize{ Panels (a) to (d) show the percentage deviation of the allocation resulting under the optimal policy, the black solid graph, and the efficient allocation, the black dashed graph, relative to the laissez-faire allocation. Panels (e) and (f) show fossil and green scientists in levels. The dotted graph refers to the laissez-faire allocation. A vertical line indicates when the net-zero emission limit becomes binding.
%	}}
%\end{figure} 
%
%\clearpage
%%
% Labor supply
The social planner attains the emission limit while increasing consumption and decreasing labor  relative to laissez-faire (Panels (a) and (b) in Figure \ref{fig:optAll_percLf_dyn}). This allocation is achieved by a higher research effort on aggregate: average hours of scientists roughly double in all periods. The social planner decreases the share of non-energy scientists (Panel (d)). More research effort in the energy sector is efficient. Within the energy sector, a higher level of fossil scientists as compared to green scientists characterizes the efficient allocation (Panels (e) and (f)). As the emission limit becomes stricter, the ratio of green-to-fossil scientists increases.
The social planner can sustain high growth rates\textemdash especially in the fossil sector\textemdash and simultaneously meet the emission limit by choosing a lower energy share to GDP and a higher ratio of green-to-fossil energy (Panel (c)).  I will discuss in the next section why this allocation of researchers is efficient.


Under the optimal policy, in contrast, consumption reduces relative to the laissez-faire allocation. Average hours of scientists fall slightly by approximately 0.1\%. The optimal policy implements a lower share of energy research, and the number of fossil scientists remains close to zero (Panels (d) and (e)). The ratio of green to fossil scientists tends to infinity. 
In the competitive economy, a rise in fossil research has to be induced via demand. A higher demand for fossil fuels, however,  conflicts with the emission target. A trade-off between growth and emission mitigation exists.  In fact, the optimal allocation falls short of both the efficient green-to-fossil energy ratio and the efficient allocation of researchers. %Thus, implementing the emission target is costly in terms of R\&D investment and growth.

%%%%%%%%%%%%%%%%%%%%%%%%%%%%%%%%%%%%%%%%%%%%%%%%%%%%%%%%%%%%%%%%%%%%%%%%%%%%%%%%%%%%
%% DISCUSSION 
%%%%%%%%%%%%%%%%%%%%%%%%%%%%%%%%%%%%%%%%%%%%%%%%%%%%%%%%%%%%%%%%%%%%%%%%%%%%%%%%%%%%

\subsubsection{Discussion}\label{subsec:dis}

 What explains the optimal policy?  To answer this question, I, first, consider the social planner allocation without cross-sectoral knowledge spillovers. Knowing the reasons behind the efficient allocation enables us to better understand the use of policy instruments. Second, I look at how the optimal allocation with labor income tax differs from the optimal allocation when no income tax is available. 
Finally, I conduct a counterfactual experiment where only the optimal carbon tax is implemented. This allows to decompose the effect of the carbon and the labor income tax. 

\paragraph{Efficient and optimal allocation without cross-sectoral knowledge spillovers}
Figure \ref{fig:optAll_percLf_dyn_noKN} depicts deviations of the efficient and the optimal allocation from laissez-faire in the model without cross-sectoral knowledge spillovers.
Absent knowledge spillovers, the social planner raises research efforts  by a factor of 3.5 compared to 2 in the benchmark model, and hours worked reduce less and increase more over time (Panel (b)). Nevertheless, consumption grows less than in the model with knowledge spillovers (Panel (a)). The reason is that fossil research is no longer valuable in a green future absent cross-sectoral knowledge spillovers. 

 When knowledge cannot spill from conventional to the green sector, meeting the emission limit requires a strong rise in green relative to fossil research. Almost no energy research happens in the fossil sector (Panel (c)), and the social planner raises green research effort (Panel (d)). Due to the knowledge advantage in the fossil sector, this extreme allocation reduces overall research productivity and, hence, consumption growth. Yet, it is efficient because it takes into account dynamic spillovers in the green sector.\footnote{ Figure \ref{fig:optAll_percLf_dyn_noKN_add} in Appendix \ref{app:quant_res_opt} depicts the ratio of green-to-fossil energy and the adjustment in the share of non-energy scientists absent knowledge spillovers. When no knowledge spills to the non-energy sector, the planner raises the share of non-energy scientists initially to boost consumption. Once the net-zero emission limit binds, the social planner reduces the share of non-energy researchers relative to laissez-faire. Then, energy research becomes more important to mitigate the costs of the emission limit.}
 
The rise in working hours over time under the social planner reflects the slow down in consumption. Consumption becomes so valuable, that hours have to rise. 
Hence, knowledge spillovers diminish the costs of implementing the emission limit since they allow to mitigate decreasing returns to research in the green sector. 


\begin{figure}[h!!!]
	\centering 	\caption{Efficient and optimal allocation: no knowledge spillovers}\label{fig:optAll_percLf_dyn_noKN}
		\begin{subfigure}[]{1\textwidth}	
		\centering\footnotesize{\textbf{In percentage deviation from laissez-faire}}\\ \vspace{2mm}
		\begin{subfigure}[]{0.4\textwidth}
			\caption{Consumption}
			%	\captionsetup{width=.45\linewidth}
			\includegraphics[width=1\textwidth]{C_PercentageLFDyn_Target_regime4_knspil1_spillover0_noskill0_sep0_xgrowth0_PV1_etaa0.79_lgd1.png}
		\end{subfigure}
		\begin{minipage}[]{0.1\textwidth}
			\ 
		\end{minipage}
		\begin{subfigure}[]{0.4\textwidth}
			\caption{Average hours worked }
			%	\captionsetup{width=.45\linewidth}
			\includegraphics[width=1\textwidth]{Hagg_PercentageLFDyn_Target_regime4_knspil1_spillover0_noskill0_sep0_xgrowth0_PV1_etaa0.79_lgd0.png}
		\end{subfigure}
	\end{subfigure}
	
	\vspace{3mm}
			\begin{subfigure}[]{1\textwidth}	
		\centering\footnotesize{\textbf{In levels}}\\ \vspace{2mm}
\begin{subfigure}[]{0.4\textwidth}
	\caption{Fossil scientists}
	%	\captionsetup{width=.45\linewidth}
	\includegraphics[width=1\textwidth]{sff_CompEffOPT_T_NoTaus_regime4_opteff_knspil1_spillover0_noskill0_sep0_xgrowth0_countec0_PV1_etaa0.79_lgd1_lff1.png}
\end{subfigure}
	\begin{minipage}[]{0.1\textwidth}
		\ 
	\end{minipage}
	\begin{subfigure}[]{0.4\textwidth}
		\caption{Green scientists}
		%	\captionsetup{width=.45\linewidth}
		\includegraphics[width=1\textwidth]{sg_CompEffOPT_T_NoTaus_regime4_opteff_knspil1_spillover0_noskill0_sep0_xgrowth0_countec0_PV1_etaa0.79_lgd0_lff1.png}
	\end{subfigure}	
%	\begin{subfigure}[]{0.4\textwidth}
%	\caption{Non-energy scientists}
%	%	\captionsetup{width=.45\linewidth}
%	\includegraphics[width=1\textwidth]{sn_CompEffOPT_T_NoTaus_regime4_opteff_knspil1_spillover0_noskill0_sep0_xgrowth0_countec0_PV1_etaa0.79_lgd0_lff1.png}
%\end{subfigure}	
\end{subfigure}
\end{figure} 


\paragraph{Comparison to carbon-tax-only policy regime}
How is the income tax used to achieve a more efficient allocation?
This section analysis the benefits of the policy regime with income tax, the {combined} regime, as opposed to a {carbon-tax-only} regime where $\tau_{\iota t}=0$.  
Panel (a)  in Figure \ref{fig:efftaul} presents percentage deviations of variables under the combined policy relative to the carbon-tax-only regime. In both regimes, tax instruments are chosen optimally. In the period from 2020 to 2044, the carbon tax is lower when a labor income tax can be used (Panel (a i)). Recall that in the exact same period, the labor income tax is used to tax labor (Panel (a), Figure \ref{fig:optPol}). From 2045 onward, the carbon tax exceeds its counterpart when no labor tax is available. Now, the government subsidizes labor. Thus, labor income taxes and carbon taxes act as substitutes.

By setting a lower carbon tax and taxing labor, the government achieves a higher share of fossil scientists in the periods before the net-zero emission limit (Panel (a ii)). 
% the fossil and non-energy sector keep consumption high; see Panels (c) and (d). 
%To achieve the higher technology levels, the planner accepts a smaller green-to-fossil ratio today. % which result in lower utility levels tomorrow. %
Hence, in terms of the allocation of research, the optimal policy comes closer to the efficient allocation. Yet, to do so, it forfeits an advantageous green-to-fossil energy ratio.\footnote{ \ Both, a smaller carbon tax and a tax on labor contribute to the adverse energy ratio. Figure \ref{fig:efftaul_GFF} in Appendix \ref{app:quant_res_opt} shows deviations of the green-to-fossil energy ratio.} This observation highlights the trade-off between more fossil research and lower fossil demand. 

When the net-zero emission limit binds, the carbon tax under the combined policy is higher than in the carbon-tax-only scenario.
 The  stricter emission limit makes the use of a labor income tax to lower emission too costly. Therefore, the government cannot engineer a higher share of fossil-to-green scientists but uses the carbon tax to meet the emission limit. Within this limit, it becomes optimal to direct more research to the green sector which is achieved by an even higher carbon tax. This is in spirit of the finding in \cite{Acemoglu2012TheChange}: absent a research subsidy, the carbon tax is used to take into account the path dependency of research, i.e. the gains from green research today by boosting green productivity tomorrow.
 
 Indeed, the optimal policy in the model without knowledge spillovers subsidizes labor throughout; see Figure \ref{fig:opt_TLs_noKN}. In this case, future green growth does not profit from fossil growth today. Therefore, carbon is taxed higher to foster more green research right from the beginning.\footnote{ The deviation in the carbon tax is minimal. For better visibility, consider Figure \ref{fig:opt_TLs_noKN_app} in Appendix \ref{app:quant_res_opt}.} By doing so, the optimal policy takes into account the path dependency of research. As regards the level of the carbon tax, it is smaller than in the model with knowledge spillovers. Knowledge spillovers to the fossil sector call for a higher and gradually increasing carbon tax.\footnote{ On this point, see the discussion in Section \ref{subsec:exp}.}
% Hence, making up for earlier fossil growth in later periods by ``overtaxing'' carbon, seems not to be the sole driver of the higher carbon tax. C}  

\begin{figure}[h!!!]
	\centering
	\caption{Optimal policy without knowledge spillovers}\label{fig:opt_TLs_noKN}
	\begin{subfigure}{0.4\textwidth}
		\caption{Average marginal income tax rate }
		%	\captionsetup{width=.45\linewidth}
		\includegraphics[width=1\textwidth]{dTaulAv_OPT_T_NoTaus_COMPtaul_regime4_spillover0_knspil1_noskill0_sep0_xgrowth0_PV1_etaa0.79_lgd0.png}
	\end{subfigure}
	\begin{minipage}[]{0.1\textwidth}
		\
	\end{minipage}
	\begin{subfigure}{0.4\textwidth}
		\caption{Tax per ton of carbon in 2022 US\$}
		%	\captionsetup{width=.45\linewidth}
		\includegraphics[width=1\textwidth]{Tauf_OPT_T_NoTaus_COMPtaul_regime4_spillover0_knspil1_noskill0_sep0_xgrowth0_PV1_etaa0.79_lgd0.png}
	\end{subfigure}
	\floatfoot{Notes: \footnotesize{ The figure shows the optimal policy in the model without knowledge spillovers. Solid graphs refer to the combined policy regime, and dashed graphs to the carbon-tax-only regime. A vertical line indicates when the net-zero emission limit becomes binding.}}
\end{figure} 


In the next two paragraphs, I decompose the effect of the combined policy regime relative to the carbon-tax-only regime into the effect of (i) the adjusted carbon tax, and (ii) the change in the labor income tax. % Under the assumption that the government first adjusts the carbon tax and then implements the labor tax, this decomposes the  aggregate effect into the effect of the policy regime as a whole.  

\paragraph{Effect of adjustment in carbon tax}
Panel (b) in Figure \ref{fig:efftaul} depicts the deviation of the allocation when only the optimal carbon tax is implemented\textemdash that is, the labor income tax is fixed at $\tau_{\iota t}=0$\textemdash relative to the carbon-tax-only scenario. The lower carbon tax almost fully accounts for the deviation of the ratio of green-to-fossil research (Panel (b i)). Average hours worked remain largely unchanged by the change in the carbon tax (Panel (b ii)). But, a bigger labor share allocated to the fossil sector raises the externality associated with work. As a result, the allocation with the lower carbon tax alone would violate the emission limit. 

%Under the net-zero emission limit, average hours are close to the level under the carbon-tax-only allocation. Yet, they are lower due to a reduction in the wage rate induced by the higher carbon tax. As shown in the analytical section, a smaller share of fossil production is concomitant with a lower wage rate. The reason is that more labor is allocated to the green sector where the marginal product of labor is smaller. 

\paragraph{Effect of labor income tax}

Panel (c) in Figure \ref{fig:efftaul} compares the allocation under the combined policy to the one where only the carbon tax is set to its optimal value and the labor income tax is kept at zero. The difference is, thus, explained by income taxation. 

The labor income tax contributes to a smoother allocation of green-to-fossil scientists albeit minimally (Panel (c i)). 
Instead, the labor income tax is more important to adjust average hours worked (Panel (c ii)). Before the net-zero emission limit, it reduces labor supply contributing to a reduction in emissions. The smaller carbon tax implies a distortion in the labor market: households do not internalize the negative effect of their work effort on emissions. 

The higher carbon tax under the net-zero emission limit, in contrast, results in too low a wage rate discouraging labor supply. In other words, households act as if their work was associated with more social costs than it actually is. The labor subsidy ensures that labor supply rises. It, thereby, raises fossil production. Yet, the emission limit remains satisfied due to the higher carbon tax. % \textit{Emissions are below their optimal social level}

An alternative explanation for the use of labor income taxes could be to stimulate research activity. 
Indeed, the progressive income tax fosters fossil research in early periods through its effect on the skill ratio. However, even in a model with homogeneous skills, the labor income tax is part of the optimal policy. Nevertheless, in this counterfactual model, the labor income tax does not affect research effort.\footnote{ Compare Figure \ref{fig:opt_Count_homskill} in Appendix \ref{app:quant_res_opt}.}  The reason is that a higher demand for research is absorbed by a higher wage rate for scientists, as discussed in section \ref{subsec:exp}. 
\begin{figure}[h!!!]
	\centering
	\caption{Decomposing effect of combined policy}\label{fig:efftaul}
	\begin{subfigure}{1\textwidth}
		\caption{\textbf{Deviation of combined policy from carbon-tax-only policy in percent}}
		\vspace{3mm}
	\begin{subfigure}{0.4\textwidth}
		\centering{(i) Carbon tax}
		%	\captionsetup{width=.45\linewidth}
		\includegraphics[width=1\textwidth]{Tauf_OPT_T_NoTaus_COMPtaulPer_regime4_spillover0_knspil0_noskill0_sep0_xgrowth0_PV1_etaa0.79.png}
	\end{subfigure}
\begin{minipage}[]{0.1\textwidth}
\
\end{minipage}
\begin{subfigure}{0.4\textwidth}
	\centering{(ii) Green-to-fossil scientists}
	%	\captionsetup{width=.45\linewidth}
	\includegraphics[width=1\textwidth]{sgsff_OPT_T_NoTaus_COMPtaulPer_regime4_spillover0_knspil0_noskill0_sep0_xgrowth0_PV1_etaa0.79.png}
\end{subfigure}
\end{subfigure}

\vspace{3mm}
	\begin{subfigure}{1\textwidth}
	\caption{\textbf{Deviation only optimal carbon tax from carbon-tax-only policy in percent}}
	\vspace{3mm}
	\begin{subfigure}{0.4\textwidth}
		\centering{(i) Green-to-fossil scientists}
		%	\captionsetup{width=.45\linewidth}
		\includegraphics[width=1\textwidth]{CountTAUF_CTOPer_Opt_target_sgsff_nsk0_xgr0_knspil0_regime4_spillover0_sep0_extern0_PV1_etaa0.79.png}
	\end{subfigure}
	\begin{minipage}[]{0.1\textwidth}
		\
	\end{minipage}
	\begin{subfigure}{0.4\textwidth}
		\centering{(ii) Average hours worked}
		%	\captionsetup{width=.45\linewidth}
		\includegraphics[width=1\textwidth]{CountTAUF_CTOPer_Opt_target_Hagg_nsk0_xgr0_knspil0_regime4_spillover0_sep0_extern0_PV1_etaa0.79.png}
	\end{subfigure}
\end{subfigure}

\vspace{3mm}
\begin{subfigure}{1\textwidth}
	\caption{\textbf{Deviation of combined policy from only optimal carbon tax in percent}}\label{fig:opt_Count}
			\vspace{3mm}
	\begin{subfigure}{0.4\textwidth}
		\centering{(i) Green-to-fossil scientists}
		%	\captionsetup{width=.45\linewidth}
		\includegraphics[width=1\textwidth]{CountTAUFPerDif_Opt_target_sgsff_nsk0_xgr0_knspil0_regime4_spillover0_sep0_extern0_PV1_etaa0.79.png}
	\end{subfigure}	
	\begin{minipage}[]{0.1\textwidth}
		\
	\end{minipage}	
	\begin{subfigure}{0.4\textwidth}
		\centering{(ii) Average hours worked}
		%	\captionsetup{width=.45\linewidth}
		\includegraphics[width=1\textwidth]{CountTAUFPerDif_Opt_target_Hagg_nsk0_xgr0_knspil0_regime4_spillover0_sep0_extern0_PV1_etaa0.79.png}
	\end{subfigure}
\end{subfigure}

	\floatfoot{Notes: \footnotesize{ Graphs show the percentage deviations of the variable under the combined policy regime where the planner can choose income tax progressivity and the carbon-tax-only regime where the income tax scheme is non-distortive, $\tau_{\iota t}=0$. A vertical line indicates the introduction of the net-zero emission limit. }}
\end{figure} 

 
\clearpage