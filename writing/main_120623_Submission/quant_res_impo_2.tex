\section{Quantitative results}\label{sec:res}

In this section, I present and discuss the quantitative results.
Subsection \ref{subsec:mr} depicts the optimal policy under the baseline policy regime: environmental tax revenues are consumed by the government. Subsection \ref{subsec:dis} discusses the results in comparison to the efficient allocation focusing on the role of labor income taxes.
%I focus on analyzing the mechanisms and welfare benefits from integrating the income tax scheme into the environmental policy. I also discuss the costs of not using lump-sum transfers.


\subsection{Results}\label{subsec:mr}
%This section depicts results on the optimal policy followed by the implied allocation in the benchmark model where environmental tax revenues are redistributed via the income tax scheme. 

\begin{figure}[h!!]
	\centering
	\caption{Optimal Policy }\label{fig:optPol}
	\begin{minipage}[]{0.4\textwidth}
		\centering{\footnotesize{(a) Income tax progressivity, $\tau_{lt}$}}
		%	\captionsetup{width=.45\linewidth}
		\includegraphics[width=1\textwidth]{../../codding_model/own_basedOnFried/optimalPol_190722_tidiedUp/figures/all_July22/taul_SingleAltPolOPT_T_NoTaus_regime3_spillover0_noskill0_sep1_xgrowth0_etaa0.79.png}
	\end{minipage}
	\begin{minipage}[]{0.1\textwidth}
		\
	\end{minipage}
	\begin{minipage}[]{0.4\textwidth}
		\centering{\footnotesize{(b) Environmental tax, $\tau_{ft}$ }}
		%	\captionsetup{width=.45\linewidth}
		\includegraphics[width=1\textwidth]{../../codding_model/own_basedOnFried/optimalPol_190722_tidiedUp/figures/all_July22/tauf_SingleAltPolOPT_T_NoTaus_regime3_spillover0_noskill0_sep1_xgrowth0_etaa0.79.png}
	\end{minipage}
\end{figure} 


To meet the emission limits suggested by the IPCC, the optimal income tax is progressive for all periods between 2030 and 2080; see panel (a) in figure \ref{fig:optPol}.  
% optimal taul over time
% -0.0153   -0.0142    0.0949    0.0958    0.0967    0.0978    0.0926    0.0937    0.0949    0.0962    0.0976    0.0992
As an emission limit becomes active in 2030, the optimal income tax progressivity jumps to $\tau_{\iota t}=0.095$ and increases to $\tau_{\iota t}=0.098$ in 2045, the last period before the net-zero limit. As the emission limit diminishes to net-zero in 2050, optimal tax progressivity reduces slightly to above $0.093$ and gradually increases in the subsequent years to close to but below $0.1$. Overall, the optimal tax progressivity during constrained periods is approximately  around half the size found for the US in \cite{Heathcote2017OptimalFramework}: $\tau_{l}=0.181$.\footnote{\ 
In the period without emission limit from 2020 to 2030, the optimal income tax is slightly regressive to boost growth as will be discussed below.}

%
%0.0000    0.0000    0.6806    0.6821    0.6835    0.6849    0.9344    0.9346    0.9347    0.9348    0.9349    0.9350
Consider panel (b). The optimal fossil tax displays a step pattern. In 2030, the environmental tax jumps to around 68\% as the emission target is to reduce emissions by 50\% relative to 2019. Over the years from 2030 to 2045 there is a small gradual increse. As the emission target declines to net-zero emissions in 2050, the optimal tax on fossil sales accelerates to 93\% and gradually increases afterwards. 
These figures underline the optimality of the integration of fiscal policy instruments into the environmental policy.

Figure \ref{fig:optAll} depicts the optimal allocation. Limiting emissions in line with the Paris Agreement is concomitant with both a reduction and recomposition of consumption and production over time. 
Panel (a) shows consumption which reduces significantly when new emission limits become active, in 2030 and in 2050, but starting from the new low levels it growths modestly. Labor effort of both skill types reduces as stricter emission targets are enforced with the exception of high skill which slightly increases as the net-zero limit becomes binding; panel (b). The rise in high-skilled hours can be explained by the drop in income tax progressivity. Hours of low-skill workers appear constant over time after the initial reduction in 2030.  In comparison to hours supplied by low-skilled workers, high-skilled workers reduce hours more as the first emission limit gets introduced in 2030; compare panel (c) which shows the ratio of hours worked by high to low skill workers. Yet, the small rise in high-skill hours in 2050 causes an increase of the high-to-low skill ratio. The increase is decaying in the subsequent years.

The rise in consumption after each reduction is driven by technological progress in all sectors; compare panel (d) which shows growth rates by sector and as aggregate in per cent. 
The green sector sees a rise in technological progress, the dashed black line, while growth in the fossil and the non-energy sector is positive, yet diminishing over time. Overall, aggregate growth is positive but decreasing; compare the grey dashed graph. 
Summing up the last two paragraphs, the emission target is best achieved with more leisure at higher technology levels in all sectors. 

The optimal allocation of scientists over time nicely captures the combination of reductive and recomposing policies; see panel (e). There is a recomposition towards the green sector: while research in the non-energy and the fossil sector decrease over time, green research effort rises. Yet, overall, the amount of scientists reduces; compare the gray graph which depicts the sum of researchers across sectors.  
Finally, labor input goods are redirected towards the green sector; see panel (f). 

\begin{figure}[h!!]
	\centering
	\caption{Optimal Allocation }\label{fig:optAll}
	
	
	\begin{minipage}[]{0.32\textwidth}
		\centering{\footnotesize{(a) Consumption}}
		%	\captionsetup{width=.45\linewidth}
		\includegraphics[width=1\textwidth]{../../codding_model/own_basedOnFried/optimalPol_190722_tidiedUp/figures/all_July22/C_SingleAltPolOPT_T_NoTaus_regime3_spillover0_noskill0_sep1_xgrowth0_etaa0.79.png}
	\end{minipage}
	\begin{minipage}[]{0.32\textwidth}
		\centering{\footnotesize{(b) Hours worked }}
		%	\captionsetup{width=.45\linewidth}
		\includegraphics[width=1\textwidth]{../../codding_model/own_basedOnFried/optimalPol_190722_tidiedUp/figures/all_July22/SingleJointTOT_regime3_OPT_T_NoTaus_Labour_spillover0_noskill0_sep1_xgrowth0_extern0_etaa0.79_lgd1.png}
	\end{minipage}
	\begin{minipage}[]{0.32\textwidth}
		\centering{\footnotesize{(c) High-to-low-skill ratio hours}}
		%	\captionsetup{width=.45\linewidth}
		\includegraphics[width=1\textwidth]{../../codding_model/own_basedOnFried/optimalPol_190722_tidiedUp/figures/all_July22/hhhl_SingleAltPolOPT_T_NoTaus_regime3_spillover0_noskill0_sep1_xgrowth0_etaa0.79.png}
	\end{minipage}
	\begin{minipage}[]{0.32\textwidth}
		\centering{\footnotesize{\ \\ (d) Technology growth}}
		%	\captionsetup{width=.45\linewidth}
		\includegraphics[width=1\textwidth]{../../codding_model/own_basedOnFried/optimalPol_190722_tidiedUp/figures/all_July22/SingleJointTOT_regime3_OPT_T_NoTaus_Growth_spillover0_noskill0_sep1_xgrowth0_extern0_etaa0.79_lgd1.png}
	\end{minipage}
%\begin{minipage}[]{0.32\textwidth}
%	\centering{\footnotesize{\ \\ (d) Technology growth}}
%	%	\captionsetup{width=.45\linewidth}
%	\includegraphics[width=1\textwidth]{../../codding_model/own_basedOnFried/optimalPol_190722_tidiedUp/figures/all_July22/SingleJointTOT_regime0_OPT_T_NoTaus_Growth_spillover0_noskill0_sep1_xgrowth0_extern0_etaa0.79_lgd1.png}
%\end{minipage}
	\begin{minipage}[]{0.32\textwidth}
		\centering{\footnotesize{\ \\(e) Scientists }}
		%	\captionsetup{width=.45\linewidth}
		\includegraphics[width=1\textwidth]{../../codding_model/own_basedOnFried/optimalPol_190722_tidiedUp/figures/all_July22/SingleJointTOT_regime3_OPT_T_NoTaus_Science_spillover0_noskill0_sep1_xgrowth0_extern0_etaa0.79_lgd1.png}
	\end{minipage}
	\begin{minipage}[]{0.32\textwidth}
		\centering{\footnotesize{\ \\(f) Labor input}}
		%	\captionsetup{width=.45\linewidth}
		\includegraphics[width=1\textwidth]{../../codding_model/own_basedOnFried/optimalPol_190722_tidiedUp/figures/all_July22/SingleJointTOT_regime3_OPT_T_NoTaus_LabourInp_spillover0_noskill0_sep1_xgrowth0_extern0_etaa0.79_lgd1.png}
	\end{minipage}
\end{figure} 



\subsection{Discussion}\label{subsec:dis}
%\tr{Questions}
%\begin{itemize}
%	\item why progressive tax? and why the drop in progressivity in 2050? (a means to boost high-skill supply and keeping low skill stable)
%	\item what are the costs of the progressive tax
%\end{itemize}
The discussion of the optimal policy centers on the question what drives the optimal policy. In the first subsection, \ref{subsec:sp_q}, I compare the optimal to the efficient allocation. In subsection \ref{subsec:notaul}, I contrast the optimal allocation under the benchmark regime with income tax to a scenario where no income tax is available. This comparison is informative on the benefits of an income tax.
Finally, I turn to the optimal allocation when lump-sum transfers are used in subsection \ref{subsec:comp_lumpsum}. 
%In subsection \ref{subsec:simpler}, I discuss the results when the benchmark model is simplified: that is, assuming exogenous growth and/or skill homogeneity.

%\begin{enumerate}
%	\item What is the goal of policy intervention? \ar social planner allocation
%	\item (Benefits) What is different when no integrated policy is run and instead revs consumed by government \ar Benefits of an integrated policy
%	\item double dividend literature: use of labor income tax when all env tax revenues are consumed by the government.
%	\item (Costs) What cannot be reached by integrated policy as compared to lump-sum transfers: is taul used for different purpose? without endogenous growth should be zero; eg. can use taul to boost growth as lump-sum transfers take care of labor supply 
%	\item What could be reached if there was no trade-off with heterogenous skills or growth? no heterogeneous skills, no endogenous growth \ar how does the optimal policy differ?
%\end{enumerate}

\subsubsection{Social planner allocation}\label{subsec:sp_q}
As a benchmark to the Ramsey planner allocation, I present the social planner's allocation. The efficient allocation can be perceived as the allocation the Ramsey planner seeks to implement. However, it may not be able to achieve the efficient allocation due to the reliance on tax instruments. Figure \ref{fig:fb_opt} depicts the efficient and the optimal allocation by the black-solid and the orange-dashed graphs, respectively. 
\begin{figure}[h!!]
	\centering
	\caption{Comparison to efficient allocation }\label{fig:fb_opt}
	
	\begin{minipage}[]{0.32\textwidth}
		\centering{\footnotesize{(a) Consumption}}
		%	\captionsetup{width=.45\linewidth}
		\includegraphics[width=1\textwidth]{../../codding_model/own_basedOnFried/optimalPol_190722_tidiedUp/figures/all_July22/C_CompEffOPT_T_NoTaus_regime3_opteff_spillover0_noskill0_sep1_xgrowth0_countec0_etaa0.79_lgd1_lff0.png}
	\end{minipage}
	\begin{minipage}[]{0.32\textwidth}
	\centering{\footnotesize{(b) High skill hours worked}}
	%	\captionsetup{width=.45\linewidth}
	\includegraphics[width=1\textwidth]{../../codding_model/own_basedOnFried/optimalPol_190722_tidiedUp/figures/all_July22/hh_CompEffOPT_T_NoTaus_regime3_opteff_spillover0_noskill0_sep1_xgrowth0_countec0_etaa0.79_lgd0_lff0.png}
\end{minipage}
	\begin{minipage}[]{0.32\textwidth}
	\centering{\footnotesize{(c) Low skill hours worked}}
	%	\captionsetup{width=.45\linewidth}
	\includegraphics[width=1\textwidth]{../../codding_model/own_basedOnFried/optimalPol_190722_tidiedUp/figures/all_July22/hl_CompEffOPT_T_NoTaus_regime3_opteff_spillover0_noskill0_sep1_xgrowth0_countec0_etaa0.79_lgd0_lff0.png}
\end{minipage}

	\begin{minipage}[]{0.32\textwidth}
	\centering{\footnotesize{(d) Aggregate growth}}
	%	\captionsetup{width=.45\linewidth}
	\includegraphics[width=1\textwidth]{../../codding_model/own_basedOnFried/optimalPol_190722_tidiedUp/figures/all_July22/gAagg_CompEffOPT_T_NoTaus_regime3_opteff_spillover0_noskill0_sep1_xgrowth0_countec0_etaa0.79_lgd0_lff0.png}
\end{minipage}
\begin{minipage}[]{0.32\textwidth}
	\centering{\footnotesize{(e) Energy mix, $\frac{G}{F}$}}
	%	\captionsetup{width=.45\linewidth}
	\includegraphics[width=1\textwidth]{../../codding_model/own_basedOnFried/optimalPol_190722_tidiedUp/figures/all_July22/GFF_CompEffOPT_T_NoTaus_regime3_opteff_spillover0_noskill0_sep1_xgrowth0_countec0_etaa0.79_lgd0_lff0.png}
\end{minipage}
\begin{minipage}[]{0.32\textwidth}
	\centering{\footnotesize{(f) Utility}}
	%	\captionsetup{width=.45\linewidth}
	\includegraphics[width=1\textwidth]{../../codding_model/own_basedOnFried/optimalPol_190722_tidiedUp/figures/all_July22/SWF_CompEffOPT_T_NoTaus_regime3_opteff_spillover0_noskill0_sep1_xgrowth0_countec0_etaa0.79_lgd0_lff0.png}
\end{minipage}
%\begin{minipage}[]{0.32\textwidth}
%	\centering{\footnotesize{(f) Utility Scientists}}
%	%	\captionsetup{width=.45\linewidth}
%	\includegraphics[width=1\textwidth]{../../codding_model/own_basedOnFried/optimalPol_190722_tidiedUp/figures/all_July22/Utilsci_CompEffOPT_T_NoTaus_regime3_opteff_spillover0_noskill0_sep1_xgrowth0_countec0_etaa0.79_lgd0_lff0.png}
%\end{minipage}
%\begin{minipage}[]{0.32\textwidth}
%	\centering{\footnotesize{(f) Utility labour}}
%	%	\captionsetup{width=.45\linewidth}
%	\includegraphics[width=1\textwidth]{../../codding_model/own_basedOnFried/optimalPol_190722_tidiedUp/figures/all_July22/Utillab_CompEffOPT_T_NoTaus_regime3_opteff_spillover0_noskill0_sep1_xgrowth0_countec0_etaa0.79_lgd0_lff0.png}
%\end{minipage}
%\begin{minipage}[]{0.32\textwidth}
%	\centering{\footnotesize{(f) Utility con}}
%	%	\captionsetup{width=.45\linewidth}
%	\includegraphics[width=1\textwidth]{../../codding_model/own_basedOnFried/optimalPol_190722_tidiedUp/figures/all_July22/Utilcon_CompEffOPT_T_NoTaus_regime3_opteff_spillover0_noskill0_sep1_xgrowth0_countec0_etaa0.79_lgd0_lff0.png}
%\end{minipage}
\end{figure}

The social planner allocation, too, is a combination of recomposing and reductive measures. 
The social planner reduces consumption less than in the Ramsey planner allocation in order to reach emission limits; compare panel (a). There is a reduction in hours worked for high- and low-skilled labor over time as emission limits become stricter. Relative to a scenario without emission limit the efficient level of hours also reduces; see panels (b) and (c) in figure \ref{fig:eff_with_notarget} in appendix section \ref{app:eff_notarg} which compares the efficient allocation absent emission limit to the one with emission limit.\footnote{\ For the given calibration, the planner reduces hours worked once there is an emission limit. Since with log-utility the income and substitution effect cancel, the social planner does not increase hours worked to compensate for lower output. Compare the discussion of the efficient allocation in section \ref{sec:theory}.} 
The optimal allocation mimics the reduction in hours worked; yet, for the high-skilled the reduction is too strong starting from 2030. The gap between the efficient and the optimal level of high-skilled hours reduces as tax progressivity declines in 2050. Optimal labor supply of the low skilled is inefficiently high for all periods considered.

Higher growth rates contribute to more consumption in the efficient allocation; see panel (d) showing aggregate growth. While the Ramsey planner can boost growth through the labor income tax only, the social planner can directly allocate scientists. The ratio of green to fossil energy, depicted in panel (e), moves similarly in the efficient and the optimal allocation. This illustrates the recomposing quality of the efficient allocation to cope with stricter emission limits. However, the Ramsey planner chooses a lower green-to-fossil energy mix starting from 2050. 

Utility of the representative household in the efficient and the optimal allocation is decreasing overall, panel (f). The reduction happens in steps when a tighter emission limit becomes active. Utility is rising when the emission limit is stable but more so under the social planner. Recall that utility is net of environmental concerns solely determined by consumption and leisure in the present framework. In the periods absent emission limit, from 2020 to 2030, utility under the optimal policy exceeds utility in the efficient allocation. This counterintuitive behaviour highlights the dynamic structure of the model: the social planner forgoes this higher utility level to profit from a higher technology level in future periods.

%\subsubsection{Comparison to other policy regimes}
%\tr{To be rewritten}
%How does the optimal allocation and especially its relation to the efficient allocation change under alternative policy scenarios?
%In this section, I discuss two policy alternations which have already been discussed in the analytical section. First, a version where environmental tax revenues are consumed by the government and no labor income tax scheme is available, henceforth referred to as \textit{separate policy}. The comparison of this scenario serves to assess the benefits of an integrated environmental-fiscal policy when no lump-sum transfers are available. 
%Second, I look at the optimal allocation 

\subsubsection{The role of income taxes}\label{subsec:notaul}
How does the availability of an income tax change the optimal allocation? 
In figure \ref{fig:comp_nored}, I contrast the optimal allocation (i) without income tax scheme, orange-dotted graph, with (ii) a regime with the option to use a labor income tax. The solid black graph depicts the efficient allocation. %Finally, I discuss how endogenous growth and skill heterogeneity shape the optimal income tax. 


In comparison to a policy scenario without income tax, the availability of an income tax allows to more closely resemble the efficient levels of labor, panels (b) and (c). %In total, the utility level of the representative household is at least as close to the efficient level for all time periods considered. 
The benefits of the progressive income tax, more leisure, come at the cost of less consumption, panel (a), a reduction in growth, panel (d), and a lower green-to-fossil energy mix, panel (e). Overall, the gains from a higher income tax exceed its costs. \textit{\tr{(Do CEV, work in progress)}}

The optimal environmental tax is only negligibly smaller when the income tax is available, compare panel (b) in figure \ref{fig:comp_nored_pol}. 
The minimal adjustment in the optimal environmental tax does not suggest  that the two tax instruments are substitutes in that the income tax is used to target the emission limit. There would be argument in favor of such a substitution effect given the model's framework: while environmental tax revenues are not redistributed to households, the income tax is redistributed, thereby keeping consumption high. Despite this advantage, there is no evidence for a substitution of policy instruments. This rather points to environmental taxes and labor income taxes being complements in the optimal environmental policy: the first is targets the externality, while the second handles the inefficiency in labor markets. 
Only in the period from 2030 to 2050 the environmental tax necessary to meet emission limits is slightly smaller which can be rationalized by a lower level of production due to the decline in labor supply induced by the income tax.

These results speak to the weak double-dividend literature. %When the government consumes environmental tax revenues, hours worked are inefficiently high. 
The weak double-dividend result posits that when environmental tax revenues suffice to cover all government funding requirements, it would be optimal to lower distortionary income taxes. The results presented herein, however, show that there is a lower bound. Lowering distortionary income taxes too much results in inefficiently high hours worked. Hence, even though there is no motive to fund government expenses  labor income taxation is not zero due to the environmental externality.
%Indeed, this reduces consumption further away from the efficient level, but, hours worked are aligned closer to the efficient level, panels (b) and (c). Next to consumption, the planner also forfeits an advantageous green-to-fossil energy ratio, panel (e). 

%The use of a progressive labor income tax contributes minimally to meeting the emission limit as can be seen by scrutinizing the optimal environmental tax, panel (b) in figure \ref{fig:comp_nored_pol}: when income taxes can be used, the environmental tax is lower. Still, the difference is minimal, supporting the thesis of complementarity of income and environmental taxes. 
%Environmental tax revenues are lower as the tax rate reduces, and income taxes reduce labor supply and hence the tax base of the environmental tax. 
%Even though labor income taxes have the advantage of being redistributed to households and lowering the externality, they are not used to substitute environmental tax revenues.\footnote{\ This might be a motive to prefer labor income taxes as an instrument to reduce emissions since labor income tax revenues are redistributed back to the household while environmental tax revenues are not in this setting. Nevertheless, the observation that the environmental tax only adjusts slightly once an income tax tool is available points to the advantage of environmental taxes in handling too high emissions.
%}

%\tr{Could find that the labor income tax is used to reduce emissions because it does not reduce consumption! \ar maybe better to compare to a scenario where government revenues have to be equal to the env. tax revenues in the separate scenario, and then check if labor income taxes are still used? But this would reduce government revenues from the exogenous target! What would be a good comparison?}

\begin{figure}[h!!]
	\centering
	\caption{Comparison optimal allocation with and without income tax}\label{fig:comp_nored}
	
	\begin{minipage}[]{0.32\textwidth}
		\centering{\footnotesize{(a) Consumption}}
		%	\captionsetup{width=.45\linewidth}
		\includegraphics[width=1\textwidth]{../../codding_model/own_basedOnFried/optimalPol_190722_tidiedUp/figures/all_July22/C_DDCompEffOPT_T_NoTaus_pol3_spillover0_noskill0_sep1_xgrowth0_etaa0.79_lgd1_lff0.png}
	\end{minipage}
	\begin{minipage}[]{0.32\textwidth}
		\centering{\footnotesize{(b) High skill hours worked}}
		%	\captionsetup{width=.45\linewidth}
		\includegraphics[width=1\textwidth]{../../codding_model/own_basedOnFried/optimalPol_190722_tidiedUp/figures/all_July22/hh_DDCompEffOPT_T_NoTaus_pol3_spillover0_noskill0_sep1_xgrowth0_etaa0.79_lgd0_lff0.png}
	\end{minipage}
	\begin{minipage}[]{0.32\textwidth}
		\centering{\footnotesize{(c) Low skill hours worked}}
		%	\captionsetup{width=.45\linewidth}
		\includegraphics[width=1\textwidth]{../../codding_model/own_basedOnFried/optimalPol_190722_tidiedUp/figures/all_July22/hl_DDCompEffOPT_T_NoTaus_pol3_spillover0_noskill0_sep1_xgrowth0_etaa0.79_lgd0_lff0.png}
	\end{minipage}
	\begin{minipage}[]{0.32\textwidth}
		\centering{\footnotesize{(d) Aggregate growth}}
		%	\captionsetup{width=.45\linewidth}
		\includegraphics[width=1\textwidth]{../../codding_model/own_basedOnFried/optimalPol_190722_tidiedUp/figures/all_July22/gAagg_DDCompEffOPT_T_NoTaus_pol3_spillover0_noskill0_sep1_xgrowth0_etaa0.79_lgd0_lff0.png}
	\end{minipage}
	\begin{minipage}[]{0.32\textwidth}
		\centering{\footnotesize{(e) Energy mix, $\frac{G}{F}$}}
		%	\captionsetup{width=.45\linewidth}
		\includegraphics[width=1\textwidth]{../../codding_model/own_basedOnFried/optimalPol_190722_tidiedUp/figures/all_July22/GFF_DDCompEffOPT_T_NoTaus_pol3_spillover0_noskill0_sep1_xgrowth0_etaa0.79_lgd0_lff0.png}
	\end{minipage}
	\begin{minipage}[]{0.32\textwidth}
		\centering{\footnotesize{(f)Utility}}
		%	\captionsetup{width=.45\linewidth}
		\includegraphics[width=1\textwidth]{../../codding_model/own_basedOnFried/optimalPol_190722_tidiedUp/figures/all_July22/SWF_DDCompEffOPT_T_NoTaus_pol3_spillover0_noskill0_sep1_xgrowth0_etaa0.79_lgd0_lff0.png}
	\end{minipage}
\end{figure}

\begin{figure}[h!!]
	\centering
	\caption{Optimal policy with and without income tax}\label{fig:comp_nored_pol}
	
	\begin{minipage}[]{0.32\textwidth}
		\centering{\footnotesize{(a) Income tax progressivity, $\tau_{\iota t}$ }}
		%	\captionsetup{width=.45\linewidth}
		\includegraphics[width=1\textwidth]{../../codding_model/own_basedOnFried/optimalPol_190722_tidiedUp/figures/all_July22/taul_DDCompEffOPT_T_NoTaus_pol3_spillover0_noskill0_sep1_xgrowth0_etaa0.79_lgd1_lff0.png}
	\end{minipage}
	\begin{minipage}[]{0.32\textwidth}
		\centering{\footnotesize{(b) Environmental tax, $\tau_{ft}$}}
		%	\captionsetup{width=.45\linewidth}
		\includegraphics[width=1\textwidth]{../../codding_model/own_basedOnFried/optimalPol_190722_tidiedUp/figures/all_July22/tauf_DDCompEffOPT_T_NoTaus_pol3_spillover0_noskill0_sep1_xgrowth0_etaa0.79_lgd0_lff0.png}
	\end{minipage}
	\begin{minipage}[]{0.32\textwidth}
		\centering{\footnotesize{(c) Government consumption }}
		%	\captionsetup{width=.45\linewidth}
		\includegraphics[width=1\textwidth]{../../codding_model/own_basedOnFried/optimalPol_190722_tidiedUp/figures/all_July22/GovCon_DDCompEffOPT_T_NoTaus_pol3_spillover0_noskill0_sep1_xgrowth0_etaa0.79_lgd0_lff0.png}
	\end{minipage}
\end{figure}


%\subsubsection{Explaining the pattern of the optimal income tax progressivity parameter}
\begin{figure}[h!!]
	\centering
	\caption{Optimal tax progressivity in amended models }\label{fig:optPol_nogr_nosk}
	\begin{minipage}[]{0.32\textwidth}
		\centering{\footnotesize{(a) Baseline model}}
		%	\captionsetup{width=.45\linewidth}
		\includegraphics[width=1\textwidth]{../../codding_model/own_basedOnFried/optimalPol_190722_tidiedUp/figures/all_July22/taul_SingleAltPolOPT_T_NoTaus_regime3_spillover0_noskill0_sep1_xgrowth0_etaa0.79.png}
	\end{minipage}
	\begin{minipage}[]{0.32\textwidth}
	\centering{\footnotesize{(b) Exogenous growth}}
	%	\captionsetup{width=.45\linewidth}
	\includegraphics[width=1\textwidth]{../../codding_model/own_basedOnFried/optimalPol_190722_tidiedUp/figures/all_July22/taul_SingleAltPolOPT_T_NoTaus_regime3_spillover0_noskill0_sep1_xgrowth1_etaa0.79.png}
\end{minipage}
\begin{minipage}[]{0.32\textwidth}
	\centering{\footnotesize{(c) Skill homogeneity}}
	%	\captionsetup{width=.45\linewidth}
	\includegraphics[width=1\textwidth]{../../codding_model/own_basedOnFried/optimalPol_190722_tidiedUp/figures/all_July22/taul_SingleAltPolOPT_T_NoTaus_regime3_spillover0_noskill1_sep1_xgrowth0_etaa0.79.png}
\end{minipage}

\begin{minipage}[]{0.32\textwidth}
	\centering{\footnotesize{(c) endogenous growth}}
	%	\captionsetup{width=.45\linewidth}
	\includegraphics[width=1\textwidth]{../../codding_model/own_basedOnFried/optimalPol_190722_tidiedUp/figures/all_July22/taul_SingleAltPolOPT_NOT_NoTaus_regime3_spillover0_noskill0_sep1_xgrowth0_etaa0.79.png}
\end{minipage}
\begin{minipage}[]{0.32\textwidth}
	\centering{\footnotesize{(c) exogenous growth, no target}}
	%	\captionsetup{width=.45\linewidth}
	\includegraphics[width=1\textwidth]{../../codding_model/own_basedOnFried/optimalPol_190722_tidiedUp/figures/all_July22/taul_SingleAltPolOPT_NOT_NoTaus_regime3_spillover0_noskill0_sep1_xgrowth1_etaa0.79.png}
\end{minipage}
\end{figure} 

% growth only enhanced as long as feasible
Figure \ref{fig:optPol_nogr_nosk} contrasts the optimal tax progressivity parameter in the baseline model, panel (a), in a model with exogenous growth, panel (b), and a model without skill heterogeneity, panel (c).
First note that even with exogenous growth the optimal income tax is progressive and approximately similarly in size. Hence, it is not that income taxes are used to lower emissions through a market-size effect. The main driver is the inefficiency in labor supply which makes the optimal income tax progressive. However, endogenous growth shapes the structure of income tax progressivity. 
While progressivity of the income tax is increasing when growth is endogenous, it is constant or even decreasing in a model with exogenous growth. 
As a conclusion, income tax progressivity in the baseline model is depressed to profit from more growth. In other words, a more regressive income tax schedule is chosen to boost growth. Nevertheless, regressivity decreases over time as accelerating growth conflicts with the emission limit.  
% The costs a higher environmental tax would incur to meet the emission limit at higher labor effort and growth exceed the benefits of more growth and consumption. 

Due to skill heterogeneity, both the regressivity of the tax in periods without an emission limit and the progressivity in periods with emission limit is more pronounced; see panel (c). When the planner has to satisfy an emission limit, it can choose a higher progressivity as there is no adverse recomposing effect on green-to-fossil energy use.  
\begin{comment}
\paragraph{Comparison integrated policy to separate policy}

Consider figure \ref{fig:bench_nored_notaul}. The figure presents the optimal allocation in the integrated policy scenario,  the orange-dashed graph, the optimal allocation under the separate policy, the blue-dotted graph, and the efficient allocation, the black-solid graph.

In comparison to a policy scenario where environmental tax revenues are not redistributed, the integrated policy closer resembles the efficient allocation in terms of consumption, panel (a) and of labor, panels (b) and (c). %In total, the utility level of the representative household is at least as close to the efficient level for all time periods considered. 
The benefits of an integrated-policy regime come at the cost of a lower green-to-fossil energy mix, panel (e), and a reduction in growth, panel (d). Nevertheless, if a planner could choose between the two regimes, it would select the integrated-policy regime. The gains from the integrated regime amount to xxx. \tr{Do CEV}

Interestingly, the optimal environmental tax is only negligibly smaller in the integrated-policy regime. This suggests, that environmental taxes and labor income taxes are complements in the optimal environmental policy to lower inefficiently high hours worked. Only in the period from 2030 to 2050 the environmental tax necessary to meet emission limits is slightly smaller which can be rationalized by a lower level of production.\footnote{\ Absent an emission limit before 2030, the optimal environmental tax is slightly negative to subsidize fossil research which again spills over to research in the other sectors. }  

\begin{figure}[h!!]
	\centering
	\caption{Comparison to separate policy scenario; \tr{drop efficient from tauf graph }}\label{fig:bench_nored_notaul}
	
	\begin{minipage}[]{0.32\textwidth}
		\centering{\footnotesize{(a) Consumption}}
		%	\captionsetup{width=.45\linewidth}
		\includegraphics[width=1\textwidth]{../../codding_model/own_basedOnFried/optimalPol_190722_tidiedUp/figures/all_July22/C_CompEffOPT_T_NoTaus_pol2_spillover0_noskill0_sep1_xgrowth0_etaa0.79_lgd1_lff0.png}
	\end{minipage}
	\begin{minipage}[]{0.32\textwidth}
		\centering{\footnotesize{(b) High skill hours worked}}
		%	\captionsetup{width=.45\linewidth}
		\includegraphics[width=1\textwidth]{../../codding_model/own_basedOnFried/optimalPol_190722_tidiedUp/figures/all_July22/hh_CompEffOPT_T_NoTaus_pol2_spillover0_noskill0_sep1_xgrowth0_etaa0.79_lgd0_lff0.png}
	\end{minipage}
	\begin{minipage}[]{0.32\textwidth}
		\centering{\footnotesize{(c) Low skill hours worked}}
		%	\captionsetup{width=.45\linewidth}
		\includegraphics[width=1\textwidth]{../../codding_model/own_basedOnFried/optimalPol_190722_tidiedUp/figures/all_July22/hl_CompEffOPT_T_NoTaus_pol2_spillover0_noskill0_sep1_xgrowth0_etaa0.79_lgd0_lff0.png}
	\end{minipage}
	\begin{minipage}[]{0.32\textwidth}
		\centering{\footnotesize{(d) Aggregate growth}}
		%	\captionsetup{width=.45\linewidth}
		\includegraphics[width=1\textwidth]{../../codding_model/own_basedOnFried/optimalPol_190722_tidiedUp/figures/all_July22/gAagg_CompEffOPT_T_NoTaus_pol2_spillover0_noskill0_sep1_xgrowth0_etaa0.79_lgd0_lff0.png}
	\end{minipage}
	\begin{minipage}[]{0.32\textwidth}
		\centering{\footnotesize{(e) Energy mix, $\frac{G}{F}$}}
		%	\captionsetup{width=.45\linewidth}
		\includegraphics[width=1\textwidth]{../../codding_model/own_basedOnFried/optimalPol_190722_tidiedUp/figures/all_July22/GFF_CompEffOPT_T_NoTaus_pol2_spillover0_noskill0_sep1_xgrowth0_etaa0.79_lgd0_lff0.png}
	\end{minipage}
	%	\begin{minipage}[]{0.32\textwidth}
	%	\centering{\footnotesize{(f) Utility}}
	%	%	\captionsetup{width=.45\linewidth}
	%	\includegraphics[width=1\textwidth]{../../codding_model/own_basedOnFried/optimalPol_190722_tidiedUp/figures/all_July22/SWF_CompEffOPT_T_NoTaus_pol2_spillover0_noskill0_sep1_xgrowth0_etaa0.79_lgd0_lff0.png}
	%\end{minipage}
	\begin{minipage}[]{0.32\textwidth}
		\centering{\footnotesize{(f) Environmental tax, $\tau_{ft}$}}
		%	\captionsetup{width=.45\linewidth}
		\includegraphics[width=1\textwidth]{../../codding_model/own_basedOnFried/optimalPol_190722_tidiedUp/figures/all_July22/tauf_CompEffOPT_T_NoTaus_pol2_spillover0_noskill0_sep1_xgrowth0_etaa0.79_lgd0_lff0.png}
	\end{minipage}
\end{figure}


	content...
\end{comment}

\subsubsection{Optimal policy and allocation with lump-sum transfers}\label{subsec:comp_lumpsum}

How do lump-sum transfers change the role of labor income taxes?\footnote{\ In appendix section \tr{To be added}, I present the optimal allocation under a policy regime where environmental tax revenues are redistributed through the income tax scheme. This scenario is relevant when the government wants to redistribute environmental tax revenues but lump-sum transfers are not feasible.}
 According to the theory in section \ref{sec:mod_an}, the use of lump-sum taxes should (i)  allow to attain an allocation closer to the efficient one and (ii) deprive the income tax scheme of its use as reductive environmental policy tool. Indeed, the optimal allocation under lump-sum transfers is much closer to the efficient one. 
 Yet, the emission limit still shapes the optimal income tax due to endogenous growth. 
  %This is so despite the advantageous recomposing effect of regressive income taxes through skill supply.
  
% YES, one can speak of a separation of environmental and fiscal policies as the goal of income taxes is to boost or lower growth in the first place. We also dont speak of the environmental tax being targeted at 

\begin{figure}[h!!]
	\centering
	\caption{Comparison integrated regime and regime lump-sum transfers}\label{fig:bench_lumpsum}
	
	\begin{minipage}[]{0.32\textwidth}
		\centering{\footnotesize{(a) Consumption}}
		%	\captionsetup{width=.45\linewidth}
		\includegraphics[width=1\textwidth]{../../codding_model/own_basedOnFried/optimalPol_190722_tidiedUp/figures/all_July22/C_CompEffOPT_T_NoTaus_bb3_pol4_spillover0_noskill0_sep1_xgrowth0_etaa0.79_lgd1_lff0.png}
	\end{minipage}
	\begin{minipage}[]{0.32\textwidth}
		\centering{\footnotesize{(b) High skill hours worked}}
		%	\captionsetup{width=.45\linewidth}
		\includegraphics[width=1\textwidth]{../../codding_model/own_basedOnFried/optimalPol_190722_tidiedUp/figures/all_July22/hh_CompEffOPT_T_NoTaus_bb3_pol4_spillover0_noskill0_sep1_xgrowth0_etaa0.79_lgd0_lff0.png}
	\end{minipage}
	\begin{minipage}[]{0.32\textwidth}
		\centering{\footnotesize{(c) Low skill hours worked}}
		%	\captionsetup{width=.45\linewidth}
		\includegraphics[width=1\textwidth]{../../codding_model/own_basedOnFried/optimalPol_190722_tidiedUp/figures/all_July22/hl_CompEffOPT_T_NoTaus_bb3_pol4_spillover0_noskill0_sep1_xgrowth0_etaa0.79_lgd0_lff0.png}
	\end{minipage}
	\begin{minipage}[]{0.32\textwidth}
		\centering{\footnotesize{(d) Aggregate growth}}
		%	\captionsetup{width=.45\linewidth}
		\includegraphics[width=1\textwidth]{../../codding_model/own_basedOnFried/optimalPol_190722_tidiedUp/figures/all_July22/gAagg_CompEffOPT_T_NoTaus_bb3_pol4_spillover0_noskill0_sep1_xgrowth0_etaa0.79_lgd0_lff0.png}
	\end{minipage}
	\begin{minipage}[]{0.32\textwidth}
		\centering{\footnotesize{(e) Energy mix, $\frac{G}{F}$}}
		%	\captionsetup{width=.45\linewidth}
		\includegraphics[width=1\textwidth]{../../codding_model/own_basedOnFried/optimalPol_190722_tidiedUp/figures/all_July22/GFF_CompEffOPT_T_NoTaus_bb3_pol4_spillover0_noskill0_sep1_xgrowth0_etaa0.79_lgd0_lff0.png}
	\end{minipage}
	\begin{minipage}[]{0.32\textwidth}
		\centering{\footnotesize{(f) Utility}}
		%	\captionsetup{width=.45\linewidth}
		\includegraphics[width=1\textwidth]{../../codding_model/own_basedOnFried/optimalPol_190722_tidiedUp/figures/all_July22/SWF_CompEffOPT_T_NoTaus_bb3_pol4_spillover0_noskill0_sep1_xgrowth0_etaa0.79_lgd0_lff0.png}
	\end{minipage}
\end{figure}

 Figure \ref{fig:bench_lumpsum} contrasts the efficient allocation, the black solid graphs, the allocation under the benchmark policy, the orange-dashed graphs, and the optimal allocation when lump-sum transfers are available, the blue-dotted graphs. 
When lump-sum transfers of environmental tax revenues are in the policy set, the Ramsey planner can implement hours worked closer to the efficient allocation, panels (b) and (c). Consumption under the regime with lump-sum transfers, as well, mirrors the efficient level more closely see panel (a). 
Growth in the scenario with lump-sum transfers is at least as high as under the benchmark policy due to the use of regressive income taxes to accelerate growth. % I DONT KNOW WHY: However, the green-to-fossil energy ratio is slightly higher under the benchmark policy under the net-zero emission limit. The recomposing mechanism of the income tax via a relatively higher supply of high-skilled labor contributes to this finding. 
Utility gains from the availability of lump-sum transfers overall seem sizable especially under the net-zero emission limit; compare panel (f).

\begin{figure}[h!!]
	\centering
	\caption{Optimal policy in integrated regime and with lump-sum transfers}\label{fig:bench_lumpsum_pol}
	
	\begin{minipage}[]{0.32\textwidth}
		\centering{\footnotesize{(a) Income tax progressivity, $\tau_{\iota t}$}}
		%	\captionsetup{width=.45\linewidth}
		\includegraphics[width=1\textwidth]{../../codding_model/own_basedOnFried/optimalPol_190722_tidiedUp/figures/all_July22/comp_bb3_notaul4_OPT_T_NoTaus_taul_spillover0_noskill0_sep1_xgrowth0_etaa0.79_lgd1.png}
	\end{minipage}
	\begin{minipage}[]{0.32\textwidth}
		\centering{\footnotesize{(b) Environmental tax, $\tau_{ft}$}}
		%	\captionsetup{width=.45\linewidth}
		\includegraphics[width=1\textwidth]{../../codding_model/own_basedOnFried/optimalPol_190722_tidiedUp/figures/all_July22/comp_bb3_notaul4_OPT_T_NoTaus_tauf_spillover0_noskill0_sep1_xgrowth0_etaa0.79_lgd0.png}
	\end{minipage}
	\begin{minipage}[]{0.32\textwidth}
		\centering{\footnotesize{(c) Lump-sum transfers}}
		%	\captionsetup{width=.45\linewidth}
		\includegraphics[width=1\textwidth]{../../codding_model/own_basedOnFried/optimalPol_190722_tidiedUp/figures/all_July22/comp_bb3_notaul4_OPT_T_NoTaus_Tls_spillover0_noskill0_sep1_xgrowth0_etaa0.79_lgd0.png}
	\end{minipage}
\end{figure}


% finding 1) income tax to boost growth, 2) no use of recomposing effect of income tax
Figure \ref{fig:bench_lumpsum_pol} shows the optimal policy when lump-sum transfers are available. 
Now, the optimal income tax scheme is regressive. Since lump-sum transfers ensure a reduction in labor supply, the inefficiency in labor supply as a motive for progressive income taxes vanishes.
Instead, the motive to boost growth and consumption dominates: absent endogenous growth, the income tax remains untouched.\footnote{\ Compare results in the model with exogenous growth depicted in figure \ref{fig:lumpsum_xgr_vglNotaul} in appendix section \ref{app:lumps}.}  In fact, the allocation attained in the model without endogenous growth but lump-sum transfers is similar to the efficient one at a zero income tax progressivity; compare figure \ref{fig:lumpsum_xgr_vglNotaul}. Hence, the environmental policy does not use income tax regressivity to recompose production towards green energy by subsidizing high-skill supply.
 
 \tr{Has to be rewritten: it is rather, that the costs of more research seem to be too high in terms of disutility.}
Nevertheless, note, though, that regressivity of the optimal income tax reduces over time; compare the orange-dashed graph in panel (a) in figure \ref{fig:bench_lumpsum_pol}. This points to the income tax, again, being shaped by the environmental externality.
This is so even though consumption is inefficiently low and higher growth rates would be efficiency improving. I conclude from this observation that it is the environmental externality which makes it optimal to forfeit growth. 
The conflict between growth and emissions, however, does not seem to arise from growth itself, since the social planner satisfies the emission limit at higher growth rates. Instead, the issue stems from labor supply as a means to boost growth in the competitive economy. 
Then, the decreasing pattern of income tax regressivity can be rationalized as follows: as growth increases, more labor means more production and hence emissions. Therefore, boost growth becomes more costly in terms of emissions. Only growth which is concomitant with lower production is feasible in the market economy.\footnote{\ If the Ramsey planner had tools at hand to boost growth without necessarily boosting production, more growth would be optimal. \tr{\textit{Does a research subsidy imply more production?}}}

%It suggests itself that the conflict with growth does not stem from growth itself but rather that it is fostered in the competitive economy through a market size effect, that is, more production, since the social planner chooses higher technology levels. This would speak against progressive income taxes as growth accelerates. Thus, the environmental indeed prevents usage of the income tax to boost demand, but I conclude that it is not used with the goal to reduce emissions through the endogenous growth channel. Since growth as such does not pose the conflict to the emission limit. 
I conclude from this discussion that it is solely the environmental tax and not the income tax which addresses the environmental externality when lump-sum transfers are available.

\subsection{Sensitivity}
I will now briefly discuss sensitivity analyses to the quantitative exercise. 
\subsubsection{Wage elasticity of labor}

Recent papers have examined the wage elasticity of labor. \cite{Boppart2019LaborPerspectiveb} present evidence that hours worked per worker have been falling steadily over time 

\subsection{Research subsidy}
but finding should be similar to version without endogenous growth
\subsection{Changing emission limits calculation}
\subsection{Technology gap}