\section{Conclusion}\label{sec:con3}
The latest IPCC report \citep{IPCC2022} stresses the necessity to transition to net-zero emissions in order to meet climate goals. The economics literature has largely focused on the use of carbon taxes to lower emissions.  However, labor income taxes may contribute to lowering emissions by affecting the level of production. I ask what is the optimal fiscal mix of taxes on carbon and labor income in a transition toward net-zero emissions?

I build an endogenous growth model in which an emission limit renders fossil energy socially costly. I find that the optimal policy always chooses a combination of carbon and labor income taxes. 
In the run-up to the net-zero emission limit, the optimal policy chooses a smaller carbon tax that on its own is insufficient to meet the emission limit.  A labor tax, therefore, serves to diminish emissions by lowering labor supply and thus overall production. The rationale is that the smaller carbon tax allows the economy to benefit from fossil research. Fossil research is more productive as it can draw from a deep pool of knowledge that has been generated in the past. In addition, fossil knowledge remains valuable in a green future since green innovation can learn from 
knowledge advances on fossil-based technologies.

Once the net-zero emission limit binds, policy implications differ. Now, the optimal policy taxes carbon extensively. A subsidy on labor serves to stabilize production.
Under the  more stringent emission target, using a labor income tax to lower emissions becomes too costly. Then, it is optimal to set an even higher carbon tax to foster green research. More green research today stimulates green research tomorrow through dynamic spillovers of knowledge. As a side effect of the higher carbon tax, a more beneficial ratio of green-to-fossil energy is obtained. This allows to boost production overall while meeting the emission limit. The scaling of the economy is achieved by subsidizing labor. 
%The rationale is a distortion in the labor market arising from an excessive use of the carbon tax to also discourage fossil research. As a side effect of the high carbon tax the wage rate declines.  A subsidy on labor mitigates this effect.

%In the short run, the policy implications differ.  The carbon tax should rise less.  However, a smaller carbon tax alone would violate emission target. 

The quantitative results show that the optimal allocation deviates substantially from the efficient one when only carbon and income taxes are available. Absent knowledge spillovers, \cite{Acemoglu2012TheChange} point to green research subsidies as an essential tool to implement the efficient allocation. With knowledge spillovers, however, my results suggest that a subsidy on fossil research might be beneficial. An important aspect of a green transition is societal acceptance. Investigating additional instruments to reduce the costs of meeting emission targets, therefore, is an important direction for future research.

\clearpage
% extensions
%In an extension, I plan to give the government the opportunity to limit working hours. The literature advocating a reduction in consumption levels \citep[e.g.,][]{Schor2005SustainableReductionb} proposes a restriction of hours worked as policy instrument to lower the consumption of resources. Even though discussed in the literature, there is evidence for political difficulties in reducing working hours. In 2020, the French Citizens' Convention on Climate voted against reducing working hours to handle climate change. Potentially, ignorance of economic consequences is an explanation. %The extension would serve to better understand these consequences.

%Thirdly, it would be inter´ndant. % Then again, subsidies may boost labor demand aggravating the inefficiency in working hours. 
%Secondly, the model abstracts from income inequality and government funding constraints which constitute traditional motives for income taxation.  Integrating these aspects into the model 
