\section{Introduction}

%\begin{quote}
%"Mitigation pathways limiting warming to 1.5°C [...]  reduce emissions further to reach net zero $CO_2$ emissions in the 2050s [...] \textit{(medium confidence)}."
%\end{quote}

The latest reports of the International Panel on Climate Change (IPCC) \citep{ IPCC2022, Rogelj2018MitigationDevelopment.} highlights the importance of an absolute emission limit to comply with the Paris Agreement on limiting temperature rise to 1.5°C or well below 2°C.\footnote{ \ The Paris Agreement of 2015 formulates clear political goals to mitigate climate change. Under this treaty, states have agreed on a legally binding maximum increase in temperature to well below 2°C, preferably 1.5° over pre-industrial levels, and the global community seeks to be climate-neutral in 2050  (compare:\\ \url{https://unfccc.int/process-and-meetings/the-paris-agreement/the-paris-agreement}). 
} 
The economics literature on environmental policy has by and large allowed for a trade-off between consumption and pollution \citep{Barrage2019OptimalPolicy, Golosov2014OptimalEquilibrium} or studied relative emission targets \citep{Fried2018ClimateAnalysis}. 
The presence of an absolute emission target may change the optimal environmental policy, as it poses a limit on growth in fossil energy usage.
%Depending on the substitutability of green and fossil energy and the velocity of the green sector to grow, the absolute emission target may, first, pose a limit to consumption growth and, second, make 
In particular, untraditional policy measures in addition to corrective taxes may become optimal. %\textit{ (WHY THIS DIFFERENCE? Also with externalities in consumption pollution cannot be compensated for by consumption as the marginal utility of consumption reduces. BUT STILL MORE IS BETTER! SO IT CAN BE COMPENSATED! )} 

%For a reasonably calibrated endogenous growth model,
 I find that the optimal labour income tax is progressive when accounting for an absolute emission target even though corrective taxes on fossil energy are available. This finding highlights the importance of policy measures targeted at a \textit{reduction} of production in tandem with \textit{recomposing} policies such as carbon taxes to mitigate climate change. % Then again, I present data indicating a voluntary reduction in household consumption. Given this behavioural change, the optimal income tax progressivity could become regressive in order to boost high-skill labour supply. 

%MODEL
To investigate the effect of an exogenous emission target on the optimal policy, I study an endogenous growth model building on \cite{Fried2018ClimateAnalysis}. 
%Calibration

% Quantitative Experiment and Results
The main finding is that progressive income taxes are optimally used in concert with carbon taxes to meet emission targets. Indeed, the emission target could be satisfied without income taxation by use of higher carbon taxes, yet, at lower welfare.  
Meeting emission targets with a fossil tax alone, hours supplied are inefficiently high. Due to the cap on fossil energy and green and fossil energy being no perfect substitutes, the additional work effort is not sufficiently compensated for by rising consumption.  
In fact, by use of a more regressive tax the government could subsidise research. Nevertheless, the planner optimally reduces labour supply thereby forfeiting higher growth rates in the green sector. %However, in presence of the emission target, higher work effort at higher productivity would violate the emission target. 

\paragraph{Literature}
The paper relates to two strands of literature. Firstly, it contributes to the literature on environmental policy and directed technical change \citep[e.g.][]{Acemoglu2012TheChange, Acemoglu2016TransitionTechnology} by focusing on unconventional policy measures which is justified given the urgent nature of climate change mitigation.  The paper is most closely related to \cite{Fried2018ClimateAnalysis} extending the analysis by (i) an optimal policy analysis, (ii) studying an absolute emission target, and (iii) providing the government with an additional policy tool: income taxes.
%\\
%{Limits to and endogenous growth}
%The paper also relates to the endogenous growth literature by inclu

Secondly, the paper connects to the literature on public policy which focuses on an efficiency-equity trade-off \citep{Heathcote2017OptimalFramework, Loebbing2019NationalChange}. In this project, instead, the reduction in labour effort induced by distortionary labour taxes has an advantageous effect: it reduces emissions. On the other hand, there is a recomposing effect which counteracts the reduction of emissions. This reduction-recomposition trade-off shapes the optimal tax progressivity in the present paper. 

%Third, the finding that hours worked are inefficiently high relates the paper to the literature on inefficiently high work efforts. Generally, too high hours supplied arise from some negative externality of consumption such as  a keeping-up-with the Joneses motive or envy \cite{Alvarez-Cuadrado2007EnvyHours}. \cite{Arrow2004AreMuch} also discuss the question of too high consumption levels.
\tr{Is there literature on optimal income taxes and environmental taxes? chosen jointly}

Thirdly, the paper relates to the literature discussing optimal environmental taxation in a distortionary fiscal setting \citep{Bovenberg1997EnvironmentalGrowth, Barrage2019OptimalPolicy, LansBovenberg1994EnvironmentalTaxation}. Redistribution and environmental protection arise as competing targets in the optimal policy. The reason is that the environmental tax reduces labour efforts as it diminishes the returns to labour or makes consumption more expensive.  

However, in contrast to this literature, the present paper 


\tr{Which tax schedule is closer to reality? }
\paragraph{Outline} Section \ref{sec:model} lays out the model which is calibrated in section \ref{sec:calib}. I subsequently show and discuss results in section \ref{sec:res}. Section \ref{sec:con} concludes. 