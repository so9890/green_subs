\section{Literature}


\paragraph{Notes \cite{Consoli2016DoCapital}}
\begin{itemize}
	\item in contrast to previous studies, they focus on occupation-level task desciptions
	\item previous work more on the industry level to prox greenness of occupation 
	\item skill and human capital dimension and green jobs; previous work on the effect of environmental regulation abstracted from these quality dimensions
	\item main findings:
	\begin{enumerate}
		\item green occupations exhibit a stronger intensity of high-level cognitive skills (\ar same occupation green version higher cognitive skills)
		\item changing occupations (becoming greener) more formal education, more work experience, more on the job training
	\end{enumerate}
\item method: within SOC 3digit groups compare skill level of occupations identified as emerging due to enviornmental needs, and those jobs transitioning to green tasks
\item SOC3, e.g engineers
\item otherwise, findings could be driven by green jobs beeing differen broader categories. (which I would want to include)
\item \textbf{skill heterogeneity is driven by two dimensions: }
\begin{enumerate}
	\item green occupations cluster in \textbf{high-skill} (\textit{intensive in abstract skills"}) \textbf{macro-occupations} p.1051 (\ar differences in type of jobs); important: non-routine analytical and interactive skills; remaining in mid-skill occupational skills
	\item differences within job types: green version of \textbf{the same job is more skill intense} (this is what they study in this paper! )
\end{enumerate}
\item SOC2 not included occupations: agriculture, public sector
\item sort SOC2 occupations according to average tasks


\begin{itemize}
	\item non-routine analytical: \\
	4.A.2.a.4 (IM) Analyzing data or information
	4.A.2.b.2 (IM) Thinking creatively
	4.A.4.a.1 (IM) Interpreting the meaning of information for others
	\item Non-routine interactive (NRI)
\\
	4.A.4.a.4 (IM) Establishing and maintaining interpersonal relationships
	4.A.4.b.4 (IM) Guiding, directing, and motivating subordinates
	4.A.4.b.5 (IM) Coaching and developing others
	\item Routine cognitive (RC)
\\
	4.C.3.b.4 (CX) Importance of being exact or accurate
	4.C.3.b.7 (CX) Importance of repeating same tasks
	4.C.3.b.8 (CX, reverse) Structured versus unstructured work
	\item Routine manual (RM)
\\
	4.A.3.a.3 (IM) Controlling machines and processes
	4.C.2.d.1.i (CX) Spend time making repetitive motions
	4.C.3.d.3 (CX) Pace determined by speed of equipment
	\item Non-routine manual (NRM)\\ 
	4.A.3.a.4 (IM) Operating vehicles, mechanised devices, or equipment
	4.C.2.d.1.g (CX) Spend time using hands to handle, control or feel objects, tools or controls
	1.A.2.a.2 (IM) Manual dexterity
	1.A1.f.1 (IM) Spatial orientation 
\end{itemize}
\end{itemize}

Green jobs are defined as either recomposing the energy mix towards green energy or as increasing energy efficiency in general: "\textit{reducing the use of fossil fuels, decreasing pollution and greenhouse-gas emissions, increasing the efficiency of energy usage, recycling materials, and developing and adopting renewable sources of energy}"

Idea to include technology on energy efficiency, which contributes to lowering emissions. 
Model so far does not allow for an increase in energy efficiency
\begin{align*}
Y=(E^{\frac{\varepsilon_e-1}{\varepsilon_e}}+N^{\frac{\varepsilon_e-1}{\varepsilon_e}})^\frac{\varepsilon_e}{\varepsilon_e-1}
\end{align*}
what research does in the baseline model is increasing output for the same amounts of input. To allow for an increase in energy efficiency could add
\begin{align*}
Y=((A_eE)^{\frac{\varepsilon_e-1}{\varepsilon_e}}+N^{\frac{\varepsilon_e-1}{\varepsilon_e}})^\frac{\varepsilon_e}{\varepsilon_e-1}
\end{align*}
But, in fact most of the jobs counted as green refer to increasing the use of renewable energy resources. Could also argue that there is an upper bound on the efficiency of energy
\subsection{Models}
\begin{itemize}
\item \cite{Bilbiie2012EndogenousCycles}
\begin{itemize}
\item a model with rep agent
\item investment in the form of stock 
\item innovation as a form of new products
\item one final good sector
\item monopolistic competition
\item homothetic preferences
\end{itemize}
\item \cite{Ravn2006DeepHabits}
\begin{itemize}
 \item habits over average previous consumption of specific good! not over total consumption
 \item rep agent 
 \item habits: marginal utility rises as habits rise \ar could look at what happens as habits are reduced! \ar marginal utility at given consumption level reduces!
 \item more is always better! Plus increases habits \ar I want: that more might not be better after some point
\end{itemize}
\item \cite{McKay2021LumpyPolicy}
\begin{itemize}
\item New Keynesian model with durable and non-durable consumption 
\end{itemize}
\item \cite{Acemoglu2012TheChange}
\begin{itemize}
\item endogenous growth
\item rep agent
\item single labour market
\item no resource use in clean sector! ; abstracts from waste
\item disaster risk!: There is a lower bound on the quality of the environment 
\item environmental externality only affects Utility! So no chance for \textbf{environmental quality} to drive production to zero!
BUT there is a natural resource which is used in production; \textit{How do the two relate?} \ar when environmental quality affects regeneration of exhaustible resource, then there would be some connection, but there is no regeneration of the resource, I think
\item there is degradation of the environment through unsustainable production (only!) and 
\end{itemize}
Functional forms
\begin{align*}
S\in[0,\bar{S}],\ & \text{where}\ \bar{S}\ \text{is the quality of the environment without pollution;}\\
S_v=0 \Rightarrow S_t=0 \forall t\geq v,\ &  0 \ \text{is the point of no return.}\\
\underset{S\rightarrow0}{lim} U(C,S)=-\infty\ & \text{S=0 is a disaster!}\\
\underset{S\rightarrow0}{lim}\frac{\partial U(C,S)}{\partial S}=\infty\ &\\
S_{t+1}= -\xi Y_{dt}+(1+\delta)S_t& \\ 
\text{evolution of environmental quality:} & \text{ falls in dirty production; regeneration rate }\\
 \text{both are exponential relationships}\Rightarrow&\text{ smaller env. quality slower regeneration}\\ 
 &\text{ higher pollution, stronger degradation}
\end{align*}
The dirty sector uses an exploitable resource in the production process
\begin{align*}
Y_{dt}= R_t^{\alpha_2}L_{dt}^{1-\alpha}\int_{0}^{1}A_{dit}^{1-\alpha_1}x_{dit}^{\alpha_1}di
\end{align*}
$R_t$ is the exhaustible resource
\begin{align*}
Q_{t+1}=Q_t-R_t
\end{align*}
they look at a version where the resource is common property (water, air) or owned (Hotelling rule)
\item \cite{Heikkinen2015DegrowthConsumers}: macro model with voluntary reduction in consumption
\item \cite{Borissov2019CarbonDevelopment}: model labour sector in more detail: skill, sectors, and transition
\item \cite{Michaillat2015AggregateUnemployment, Auerbach2021InequalityEconomy} examples of models with economic slack. But both do not feature a satiation point of consumption. 
\item 
\cite{Loebbing2019NationalChange}
\begin{itemize}
	\item studies the welfare effects of a progressive tax reform in a model where skill supply drives the innovation decision of firms
	\item when the rich reduce their labour supply more \ar innovation is directed towards low skills\ar the wage distribution compresses
	\item the overall effect on welfare is mixed since not only do low skill wages catch up but also does the tax base reduce as the high skilled reduce their labour supply
	\item in sum, he finds that, taking directed technical change into account increases the set of welfare improving tax reforms 
	\item first he focuses on the mechanism, second on the optimal tax scheme
	\item optimal tax reform discussed in a comparison to the optimal tax a planenr would choose who perceives technology as fixed
	\item then quantification \ar he finds small responses of labour supply to progressive tax reforms
\end{itemize}
\end{itemize}


\subsection{Motivation}
\begin{itemize}
\item \cite{Schor2005SustainableReduction}
\begin{itemize}
	\item arguments against unlimited growth
	\begin{itemize}
\item hhh
	\end{itemize}
\end{itemize}
\item \cite{Dasgupta2021}
\begin{itemize}
\item emphasises the use of nature as a sink (stock) and as an input to production \ar can the two be combined?
\end{itemize}
\end{itemize}
