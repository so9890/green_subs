\subsection{Hypothesised outcome}
How do I expect the optimal steady state to differ from the laissez-faire one? 
In the representative agent model, the government faces a trade-off  between efficiency and the externality. 
On the one hand, the distortionary labour tax reduces output and thereby the externality of production. On the other hand, it reduces utility from consumption.

Allowing for two skill types and a skill bias of the cleaner sector adds an additional layer to the effect of labour taxes on the environment. Instead of merely reducing output there is also a recomposition effect. 
In response to the labour tax, the household reduces its labour supply. Since the high-skilled labour earns a higher wage rate, unskilled labour becomes more attractive to the representative agent. \tr{This effect is not there because all household members consume the same amount.} The lower supply of skilled labour increases production costs of the cleaner sector. The price of the cleaner good increases. Hence, the share of clean to dirty output falls. This indirect recomposition effect counteracts the direct reduction of the externality. 

I hypothesise that under this assumption growth in the clean sector, too, will have to stop. Why? Consider that only the clean sector growths, then the price for clean goods has to fall so that the final good sector  continues demand the supply of the clean good. The price will be driven towards zero. Which cannot be an equilibrium solution since the clean sector would stop producing as costs exceed revenues (only if marginal production costs tend to zero but they don't as labour exerts disutility).

How can then be there a role for distortionary labour taxes if the government can choose growth rates? Maybe not. But once growth is endogenous? Maybe during the transition? 
Maybe because reducing labour supply is better in terms of utility? 

Some rationale for setting hours restriction? 