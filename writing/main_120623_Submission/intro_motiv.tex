The Intergovernmental Panel on Climate Change highlights the need to transition to net-zero emissions by 2050. The literature has by and large focused on carbon taxes to mitigate environmental pollution. Carbon taxes direct demand towards less polluting goods. When knowledge accumulation is endogenous, the carbon tax also shifts research across sectors. A labor income tax, in contrast, affects the level of emissions by diminishing work effort. When a labor income tax is available, the carbon tax can be targeted more towards directing research. 
What is the optimal fiscal mix to implement the emission limit?

I answer this question in a model of directed technical change. The government chooses labor income taxes and a carbon tax. Carbon tax revenues are rebated lump sum. I find that the optimal policy mix consists of a carbon tax in combination with a tax on labor in early years. This policy enables the economy to profit from growth in the fossil sector while the tax on labor reduces work effort to meet the emission limit. When the net-zero emission limit binds, the economy has to reduce fossil research more. The optimal carbon tax rises. A subsidy on labor stabilizes output. 

The role of labor income taxes arises from the contracting effect of lump-sum rebates. 
When the carbon tax is also targeted at directing research, carbon tax revenues and hence lump-sum transfers change, too. When, for instance, the carbon tax is set lower in order not to reduce fossil research too much, lump-sum transfers decline. This stimulates economic activity as households are eager to work more. As a result, production and thereby emissions increase. To counter this side effect on emissions, the government taxes labor. 


%The optimal policy mix crucially depends on the accumulation of knowledge. If knowledge is exogenous, the fiscal mix involves a carbon tax only. With knowledge accumulation, the carbon tax also directs research activity. The less sector-specific knowledge is, the more should the government tax carbon to prevent the fossil sector from growing too much through spillovers from other sectors. On the other hand, knowledge spillovers from the fossil sector allow the economy to profit from fossil research in early years while meeting the net-zero emission limit later on. 


\begin{comment}
The latest assessment report of the Intergovernmental Panel on Climate Change \citep{IPCC2022} highlights the urgency to reduce greenhouse gas emissions,%relative to the previous report from 2018 \citep{Rogelj2018MitigationDevelopment.}.
\footnote{ \  The report stresses the decreasing likelihood of meeting the Paris Agreement and limiting climate warming to 1.5°. The Paris Agreement of 2015 formulates clear political goals to mitigate climate change. Under this treaty, states have agreed on a legally binding maximum increase in temperature to well below 2°C, preferably 1.5° over pre-industrial levels, and the global community seeks to be climate-neutral in 2050  (see: \url{https://unfccc.int/process-and-meetings/the-paris-agreement/the-paris-agreement}). 
}
and some scholars have pointed to limiting consumption to handle environmental boundaries.\footnote{\ \cite{Schor2005SustainableReductionb} argues for the necessity to limit consumption in the global North through a reduction in working time. \cite{Arrow2004AreMuch} raise the question if today's consumption is too high from a sustainability perspective. \cite{Dasgupta2021}  argues for the impossibility of indefinite growth due to planetary boundaries  \citep{Rockstrom2009AHumanity}. %: acknowledging planetary boundaries, i.e., boundaries which define a state of nature in which humans can safely exist \citep{Rockstrom2009AHumanity}, and that production and consumption produce waste, infinite production would degrade nature in a way that production is impossible.
	 \cite{VanVuuren2018AlternativeTechnologies} study alternative mitigation pathways with lower demand %such as lower energy demand, lower appliance ownership, and meat consumption 
	 in an integrated assessment model motivated by seeking to reduce reliance on carbon capture and storage technologies which entail risks and compete for scarce land. \cite{Bertram2018TargetedScenarios} stress the importance to reduce demand for energy- and material-intense products to alleviate the trade-off between mitigating temperature rises  and the UN sustainability goals%(such as food security, biodiversity protection, and clean water)% (p.11: Shifting towards healthier diets and less energy-and material-intensive consumption patterns appearsto have greatest potential for reducing sustainabilityrisks along a wide range of dimensions)
	. } A reduction in work effort and consumption mitigates pollution by diminishing economic activity. Distortionary fiscal policies qualify as a reductive policy instrument to target the level of production.
However, the literature on environmental policy has focused on compositional policies: environmental taxes. %\citep{Fried2018ClimateAnalysis}. 
Given the exigency to act, this paper addresses the question whether fiscal policies can help meet climate targets. %Using analytical and quantitative methods, I show that reductive policies form part of the optimal environmental policy even absent an additional target.

content...
\end{comment}


%This is your core argument for why reductive policy measures may work, so you should mention above that you consider this possibility,  MACHE ICH DAS NICHT MIT DER rESEARCH QUESTION? suggested by its proponents (your "scholars" :-)), and actually show that it works.

In the first part of the paper, I show analytically that once 
labor supply is elastic, reductive policy measures optimally complement the environmental tax. 
The literature has established that, absent any other distortion, an environmental tax equal to the social cost of the externality implements the efficient allocation. 
%Environmental taxes are perceived as a cost-effective way to reduce emissions. 
I demonstrate that this result crucially depends on the use of lump-sum transfers to redistribute environmental tax revenues. Transfers reduce labor supply through an income effect. %Thus, indeed there is a role for reductive policy measures. 
%\textcolor{blue}{This is interesting independent of whether they are feasible or not. Could relate to the fact that there is a discussion how to use revenues. Yet, one might argue that we are always in a setting with distortionary labor income taxes; so that recycling lump-sum is never needed; numbers on size of expected revenues and government spending}
When environmental tax revenues are not redistributed lump sum, environmental taxes are optimally combined with progressive labor income taxes. The use of income taxes as a reductive policy measure is not directly targeted at the externality: the motive for labor taxation emerges from a distortion in labor markets as households feel poorer than the economy is. %\footnote{\ This is a novel motive for the use of reductive policies adding to the arguments made in the literature listed in the previous footnote. These are: conflicts with other goals such as the UN sustainability goals, risks associated with carbon capture and storage technology, and planetary boundaries and limits to growth.} 
Hence, progressive income taxes complement environmental taxes. %Hence,  % to lower inefficiently high hours worked. 
% I show that redistributing environmental tax revenues through an income tax scheme allows to implement the efficient allocation. The optimal income tax scheme is progressive.
%the optimal environmental policy equalizes the distribution of income as  a side effect.
% The theoretical analysis forms the

In the second part, I scrutinize whether progressive income taxes remain optimal in an endogenous growth model with heterogeneous skills. The government cannot use lump-sum transfers but redistributes environmental tax revenues through the income tax scheme. The environmental externality is modeled as an exogenous limit on net emissions. 
 %In the spirit of \cite{Acemoglu2002DirectedChange}, directed technical change may intensify or mitigate these channels thr recomposition. %Second, an overall reduction in labor supply curbing production may lower general research incentives.
 % % more low skill supply, more fossil innovation, more fossil production, and higher low income \ar reduction in the wage premium! 
 The model suggests that the optimal income tax scheme is progressive. The benefits of labor taxation emerge from (i) more leisure and (ii) gains from technology growth.
 The economy, however, can only profit from the latter advantage when knowledge spillovers across sectors are strong enough. Then, augmented growth in the fossil and non-energy sector in early periods boosts green and mitigates fossil growth in future periods. Here, progressive income taxes emerge as a substitute to environmental taxes. Their use as substitutes is limited in scope: once the emission limit narrows to net zero, there is no room for the government to substitute carbon taxes with progressive income taxes.
 
% 
% On the one hand, a skill bias documented for the green sector \citep{Consoli2016DoCapital} in combination with a relatively more elastic high-skill labor supply causes a higher tax progressivity to recompose the economic structure towards dirty production. On the other hand, a higher labor share in the fossil sector implies a recomposition of the economy towards green production.
%  The skill-recomposition channel dominates and is slightly amplified by a market size effect directing research towards the fossil sector. 
 
% labor income taxes are used to substitute for environmental taxes to realize the gains from knowledge spillovers. 
 
%\textit{I quantify the welfare gains of setting progressive income taxes to equal yyy in consumption equivalent measure. TO BE DONE  }

% relation to literature
I discuss briefly the most important contributions of the paper.
First, the results are relevant for the political and academic debate on how  to use environmental tax revenues %Current debates on how to optimally use environmental tax revenues in politics \citep{Baker2017TheDividends} and academia \citep[e.g.][]{Fried2018TheGenerations, Carattini2018} motivate to focus on this policy regime. 
 \citep[e.g.][]{Baker2017TheDividends, Fried2018ClimateAnalysis, Carattini2018}. The paper points to the importance of lump-sum transfers within the optimal environmental policy as a reductive policy tool; an aspect which appears overlooked in the discussion. %\footnote{\ POLICY debate; \cite{Fried2018TheGenerations}}
When thinking about how to recycle environmental tax revenues other than by lump-sum transfers,  one should take into account alternative reductive tools such as progressive labor income taxes. 
If the reductive part of the environmental policy is neglected, environmental taxes have to be higher to meet emission limits, as I demonstrate in the quantitative exercise.

Second, the results contribute to the academic debate on the so-called \textit{weak double dividend} \citep[for example:][]{LansBovenberg1994EnvironmentalTaxation, LansBovenberg1996OptimalAnalyses}. The hypothesis posits that recycling environmental tax revenues to reduce preexisting tax distortions is advantageous to transferring  revenues lump sum. The rationale is that transfers decrease labor supply thereby diminishing the tax base of the income tax. %A conflict between generating government funds and environmental protection arises. 
The finding in the present paper suggests a lower bound on the reduction in distortionary income taxes: when environmental tax revenues are not redistributed lump sum, some reduction in labor supply via distortionary income taxes is in fact efficient from an environmental policy perspective. %In other words, even if environmental tax revenues suffice to satisfy a government revenue requirement, there is a motive for progressive income taxation. 

Third, the findings are especially interesting as the provision of the environmental public good and equity have been perceived as competing targets in the literature. When the poor consume more of the polluting good, a corrective tax is regressive \citep{ Fried2018TheGenerations, Sager2019IncomeCurves}. % \textit{Metcalf 2007, Hassett 2009 as  in Fried 2018}. 
%Second and more indirectly, a fossil tax exerts efficiency costs by lowering labor efforts\footnote{\ The reduction in hours worked is per se not inefficient. The reduction in dirty production reduces the marginal product of labor, which might make a reduction efficient. However, when the government seeks to tax labor income using distortionary policy tools, the reduced labor supply diminishes the tax base of the labor tax making it more costly to redistribute.} which again raises the cost for the government to redistribute \citep{Dobkowitz2022}. 
In contrast to this literature, the present paper provides an argument for progressive income taxes under perfect income-risk sharing suggesting a double dividend of redistribution: equity and efficient externality mitigation. %: equity on the one hand and efficiency gains from less labor as part of the environmental policy.
