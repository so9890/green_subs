\section{Core model and theoretic results}\label{sec:mod_an}

This section develops a tractable model  to derive the theoretic results. I show that scaling the level of production is part of the efficient environmental policy. Yet, absent endogenous growth, this is fully implemented by the use of an environmental tax and lump-sum transfers. There is no role for labor income taxes.

\subsection{Model}
The representative household faces a consumption and labor supply decision. The final consumption good is a composite of a fossil and a green good. Labor is the only input to production. The fossil sector causes an environmental externality.\footnote{ For simplicity, the green sector does not induce any externality; yet, whenever intermediate goods are no perfect substitutes, final good production is never perfectly green.} There is no growth, and the model is static.

\paragraph{Representative household}
Throughout the paper, the household's decision is static. Each period, the household maximizes its period utility
\begin{align*}
U(C,H; F).
\end{align*} 

The household derives utility from consumption, $C$, but experiences disutility from hours worked, $H$. An externality from fossil production, $F$, decreases household utility. The level of fossil production is taken as given by the household.
I assume additive separability of consumption, hours, and the externality. Utility of consumption is increasing and strictly concave. Utility is decreasing and strictly convex in hours worked and fossil production.
Utility maximization is subject to a period budget constraint:
\begin{align}
	 C= \lambda(wH)^{1-\tau_{\iota}}+T_{ls}. \label{eq:hhbudget}
\end{align}
The variable $w$ indicates the wage rate.  Lump-sum transfers from the government are denoted by $T_{ls}$.
The government levies income taxes on labor income using a non-linear tax scheme common in the public finance literature \citep{Heathcote2017OptimalFramework, Benabou2002TaxEfficiency}. The tax scheme is
characterized by (i) a scaling factor, $\lambda$, which determines the level of average tax revenues in the economy, and (ii) a measure of tax progressivity denoted by $\tau_{\iota}$. 
\cite{Heathcote2017OptimalFramework} show that whenever $\tau_{\iota}>0$, the tax scheme is progressive since the marginal tax rate exceeds the average tax rate irrespective of  pre-tax labor income. Hence, average tax payments increase with labor income.\footnote{ An alternative intuition is that when $\tau_{\iota}>0$, the elasticity of post- to pre-tax  income is smaller unity for all levels of pre-tax income.  } %\footnote{ I show that the result is equivalent with a linear tax rate in the appendix.} 
With a representative household, $\tau_{\iota}$ can be understood as an instrument to regulate labor supply and, thus, the overall level of production. When $\tau_{\iota}<0$, the government subsidizes labor, with $\tau_{\iota}>0$, it discourages labor. 

\paragraph{Production}
All sectors of production are perfectly competitive, and production functions have decreasing returns to scale. %\footnote{ \textit{With increasing returns to scale the assumption of perfect competition would be violated. With constant returns to scale, the solution is not unique.}}. The final consumption good, $Y$, is a composite of the fossil, $F$, and the green intermediate good, $G$. 
Intermediate goods, indicated by $J\in \{F,G\}$ for fossil and green, are produced from the labor input good, $L_J$, using technology, $A_J$. The variable $Y$ stands in for final output and is the numeraire. Production is given by:
\begin{align}
Y=Y(F, G), \hspace{5mm} F=F(A_F, L_F),\hspace{5mm} G=G(A_G, L_G). \label{eq:prod}
\end{align}

\paragraph{Government}
The government raises income taxes from households and levies an environmental tax, $\tau_F$, per unit of fossil energy bought by final good producers. The environmental tax, thus, is modeled in parallel to a carbon tax which poses a price on emissions. Revenues from the income tax and the environmental tax are treated separately by the government. Income tax revenues are fully redistributed through the income tax schedule. Environmental tax revenues are rebated lump sum to households:
\begin{align}
\tau_{F}F=T_{ls}, \hspace{7mm}
0={w H}-\lambda(w H)^{1-\tau_{\iota}}. \label{eq:gov_but}
\end{align}
The scaling parameter $\lambda$ adjusts to balance the income tax scheme. 
%Environmental tax revenues are either transferred lump-sum, fully consumed by the government, or transferred through the income tax schedule.

\paragraph{Markets}
Markets for labor and the final good both clear: 
\begin{align}
H=L_F+L_G,\ \hspace{5mm} Y=C. \label{eq:market_clear}
\end{align}
%I summarize the eq.s determining the competitive equilibrium in appendix Section \ref{app:model}.
\paragraph{Competitive equilibrium}
In a competitive equilibrium, household behavior is determined by the budget constraint, eq. \eqref{eq:hhbudget}, and labor supply which follows from the household's first order conditions and substitution of $\lambda$ from the government's budget on the income tax:
\begin{align}
-U_H=U_C(1-\tau_{\iota})w. \label{eq:hsup}
\end{align}
Firms choose the quantity of input goods to maximize their profits taking prices as given. The following equations describe this behavior in equilibrium:
\begin{align}
p_G=\frac{\partial Y}{\partial G}, \hspace{5mm}
p_F +\tau_{F} = \frac{\partial Y}{\partial F}, \hspace{5mm}
w= p_F\frac{\partial F}{\partial L_F}=p_G\frac{\partial G}{\partial L_G}.\label{eq:profmax}
\end{align}

The competitive equilibrium is defined as prices and allocations so that households and firms behave optimally; i.e. eqs. \eqref{eq:hhbudget}, \eqref{eq:hsup}, and \eqref{eq:profmax} hold. Production happens according to eqs. \eqref{eq:prod}.  Equilibrium prices and the wage rate adjust to clear markets, eqs. \eqref{eq:market_clear}. Finally, the government's budgets are satisfied eqs. \eqref{eq:gov_but}. Policy variables $\tau_F$ and $\tau_\iota$ are taken as given. 

\subsection{Theoretic results}\label{sec:theory}
Section \ref{subsec:sp2} defines and discusses the efficient allocation. It constitutes a benchmark for the optimal allocation discussed in Section \ref{subsec:decen_ec}. 

\subsubsection{Social planner}\label{subsec:sp2}
Let the share of fossil to total labor be denoted by $s=\frac{L_F}{H}$. The social planner's problem reads
\begin{align*}
\underset{s, H}{\max}\ & U(C,H; F)\\ s.t.\ \ & C=Y.
\end{align*}
The first order conditions are given by
\begin{align}
wrt. s:\hspace{4mm} & U_C \cdot \left(\frp{Y}{F}\frp{F}{s}+\frp{Y}{G}\frp{G}{s}\right)=-U_F\frp{F}{s}, \label{eq:fbs2}\\
wrt. H:\hspace{4mm}& U_C\frp{Y}{H}+U_F\frp{F}{H}=-U_H. \label{eq:fbh}
\end{align}
Where $U_X$ denotes the partial derivative of utility with respect to the variable $X$.
These equations determine the efficient or first-best allocation. 
Absent an externality, $U_F=0$, the efficient distribution of labor equalizes the marginal product of labor across sectors; compare eq. \eqref{eq:fbs2}. Efficient hours balance the marginal utility gain from consumption and the marginal disutility from working, as formalized by eq. \eqref{eq:fbh}. 

When there is an externality, the social planner adjusts the allocation in two ways: (i) a compositional adjustment, that targets the share of fossil production, and (ii) a scaling adjustment amending the level of production. 
The compositional adjustment is determined by eq. \eqref{eq:fbs2}.
The negative externality of fossil production makes it efficient to alter the share of fossil labor so that  a marginal reallocation of labor to the fossil sector would raise output.\footnote{ Note that $U_F<0$ by assumption so that the right-hand side is positive and that $\frac{dG}{ds}<0$. Hence,  in the efficient allocation, the marginal product of labor in the fossil sector is higher than in the green sector.} %Hence,$\frp{Y}{F}\frp{F}{s}>-\frp{Y}{G}\frp{G}{s}$ is efficient. }
I show in Appendix \ref{app:redeffs} that the social planner reduces the fossil labor share when the aggregate production function features decreasing returns to scale in its labor inputs, $L_G$ and $L_F$.


The scaling effect is summarized by eq. \eqref{eq:fbh}.
First note that eq. \eqref{eq:fbh} can be rewritten by substituting eq. \eqref{eq:fbs2} and noticing the relation of derivatives with respect to $H$ and $s$.\footnote{ This is done in more detail for the optimal allocation in Appendix \ref{app:incometax0}. Relations of derivatives are summarized in Appendix \ref{app:dervs_use}.}  
The second first order condition becomes:
\begin{align}\label{eq:fbh_simp}
-U_H=U_C\frac{\partial Y}{\partial G}\frp{G}{L_G}.
\end{align}
Hence, the efficient level of the externality lowers hours as if the marginal product of labor was equal to the marginal product of labor in the clean sector.

The recomposition of labor towards the  green sector reduces the average marginal product of labor in the economy. An additional unit of labor results in a smaller increase in consumption.  This effect has two opposing impacts on the efficient level of labor. On the one hand, there is a substitution effect: as leisure becomes less costly, the efficient amount of hours reduces (note that the right-hand side of eq. \eqref{eq:fbh} is increasing in $H$). On the other hand, the economy becomes poorer in terms of consumption, and more work effort might be efficient. This is captured by the term $U_C$ and equivalent to an income effect. 
%In total, which effect dominates depends on the curvature of the utility from consumption, $\theta$. With $\theta>1$ the  lower marginal product of labor decreases the efficient amount of hours worked. 
%Second, the social planner reduces hours worked due to their negative exeternality through fossil production. This effect is introduced by the term $U_F\frac{dF}{dH}<0$. 
Proposition \ref{prop:0} summarizes this discussion.
\begin{prop}\label{prop:0}
	Efficient externality mitigation consists of a compositional and a scaling adjustment. 
\end{prop}


Depending on the importance of the income effect, efficient hours worked may be higher or lower than  absent an externality. I will show in the following, however, that there is no role for labor income taxation in implementing the efficient allocation. In fact, under the optimal policy, the wage rate is set so that households internalize the effect of work effort on emissions. %\footnote{ \ I discuss in the appendix conditions on parameter values when assuming functional forms of the model.}
%I will show in the following, that irrespective of whether the social planner de- or increases hours, the decentralized economy always features higher hours when environmental tax revenues are not redistributed lump-sum. 


\subsubsection{Decentralized economy}\label{subsec:decen_ec}

Governments use tax and transfer instruments to correct for distortions, such as an environmental externality. The question arises if the efficient allocation can be decentralized by the use of taxes and transfers in a competitive economy.  %For now, I assume that the income tax is not available and $\tau_{\iota}=0$, $\lambda=0$.

%I show in this section that lump-sum redistribution of environmental tax revenues is essential to implement the first-best allocation in the competitive equilibrium. Only in combination with lump-sum transfers of  environmental tax revenues does an environmental tax suffice to implement the efficient allocation. %Then the environmental tax equals the social cost of the externality as shown by \textit{PIGOU}. 
%When environmental tax revenues are not redistributed lump-sum, hours worked exceed their efficient level, and a role for income taxes to lower hours worked arises. I consider two cases.

%\begin{enumerate}
%\item lump-sum transfers important for Pigou tax to implement efficient allocation: Proposition \ref{prop:1}
%\item when transfers are not redistributed: infeasibility of efficient allocation,  role for labor tax, and violation of Pigou principle \ref{prop:2}.
%\item redistribution through income tax scheme with progressive income tax restores efficient allocation \ref{prop:3}
%\end{enumerate}

%\subsubsection{Government problem}\label{subsec:Rams}
The government is characterized by a Ramsey planner who maximizes utility of the representative household by use of tax and transfer instruments. The behavior of firms and households constrain the government's optimization problem. 
The Ramsey problem is defined as
\begin{align*}
\underset{s, H}{\max}\ & U(C,H; F)\\ s.t.\ \ &  C=Y,
\end{align*}
subject to the behavior of households and firms.
The first order conditions are equivalent to the social planner ones:
\begin{align}
wrt.\ s:\hspace{4mm} & U_C\cdot\left(\frac{\partial Y}{\partial F}\frac{\partial F}{\partial s}+\frac{\partial Y}{\partial G}\frac{\partial G}{\partial s}\right)=-U_F\frac{\partial F}{\partial s}, \label{eq:sbs}
\\
wrt.\ H:\hspace{4mm} & U_C\frac{\partial Y}{\partial H}+U_F\frac{\partial F}{\partial H}=-U_H\label{eq:sbh}. 
\end{align}
%-- paragraph to show that with Gov=0 and lump-sum transfers, the efficient allocation is implemented
When environmental tax revenues are fully redistributed lump sum, an environmental tax equal to the marginal social cost of fossil production implements the efficient allocation.\footnote{ I define and derive the social cost of fossil production in Appendix \ref{app:scp}.} This observation is known as the \textit{Pigou principle} in the literature. 
To see this, note that eq. \eqref{eq:sbs} ensures that the social planner's first order condition, eq. \eqref{eq:fbs2}, is satisfied. 
Rewriting eq. \eqref{eq:fbs2} reveals that the Pigou principle holds: %\footnote{ I derive the social cost of pollution as the price the representative household is willing to pay for a marginal reduction in fossil production. The derivation is exponded in appendix Section \ref{sec:mod_an}. 
%	To be precise, social cost of pollution refers to the marginal cost evaluated at the resulting equilibrium allocation.}: The Pigou principle. 
\begin{align*}
\underbrace{\frac{-U_F}{U_C}}_{\text{marginal social cost of fossil production}}=\left(1+\frac{\frac{\partial Y}{\partial G}\frac{\partial G}{\partial s}}{\frac{\partial Y}{\partial F}\frac{\partial F}{\partial s}}\right)\frac{\partial Y}{\partial F}=\tau^*_F.
\end{align*}
Where the second equality follows from substituting firms' profit maximization conditions from eqs. \eqref{eq:profmax}.

Absent an externality of production, it is efficient to balance marginal products of labor across sectors.
When there is an externality, the social planner lowers the share of labor in the fossil sector. As a result, the marginal product of labor in this sector increases. It falls in the green sector. To sustain this gap between marginal products in the competitive equilibrium, the government has to introduce a corrective tax. Otherwise, market forces would direct labor towards the sector with the higher marginal product. Consequently, the equilibrium wage rate is below the marginal product of labor.\footnote{ I formally discuss this statement in Appendix \ref{app:wageMPL}.} 

Setting the environmental tax equal to the social cost of fossil production implies that the second first order condition of the Ramsey planner, eq. \eqref{eq:sbh}, is satisfied without use of the income tax instrument: $\tau_{\iota}^*=0$. 
The reason is, that in this case, the wage rate reflects the marginal social costs of hours through raising emissions. I show in Appendix \ref{app:incometax0} that the wage rate can be written as:
\begin{align*}
w = \frp{Y}{H}+\frac{U_F}{U_C}\frp{F}{H}.
\end{align*}
Since $U_F<0$, the second summand reduces the wage rate beyond the marginal product of labor in the economy.
Therefore, households internalize the marginal social costs of the externality of hours worked in their labor supply decision. Relative to no policy intervention, labor supply declines. Proposition \ref{prop:1} condenses this result.

\begin{prop}\label{prop:1}
	The efficient allocation is implemented by an environmental tax and lump-sum transfers.  When the environmental tax implements the efficient share of dirty labor, the wage rate fully captures the marginal effect of hours worked on the externality. There is no role for distortive labor income taxation, $\tau_{\iota}^*=0$.
\end{prop}
Proof: Appendix \ref{app:incometax0}. 

%As discussed previously,  However,

%Due to this effect of the environmental tax on the wage rate, lump-sum transfers and environmental taxes alone suffice to implement the efficient level of hours worked. 

