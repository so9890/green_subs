\subsection{A constant carbon tax}\label{subsec:exp}


We are now equipped to study how a carbon tax  equal to US\$185 (in 2020 prices) affects the economy. The value reflects the social costs of carbon calculated by a joint research effort led by \textit{Resources for the Future} (RFF), an independent research institution, and the University of Berkeley \citep{Rennert2022ComprehensiveCO2}.\footnote{  For comparability, the social cost of carbon equal US\$203.5 in 2022 prices.}


\begin{figure}[h!!]
	\centering
	\caption{A constant carbon tax equal to US\$185 (2020 prices) per ton of carbon  }\label{fig:Leveltauf_nsk0_xgr0_know}		
	\begin{subfigure}[]{0.4\textwidth}
		\caption{Net CO$_2$ emissions in Gt \\ \ }
		%	\captionsetup{width=.45\linewidth}
		\includegraphics[width=1\textwidth]{CompTauf_bytaul_Reg5_Emnet_spillover0_nsk0_xgr0_knspil0_sep0_LFlimit0_emsbase0_countec0_GovRev0_etaa0.79_lgd1.png}
	\end{subfigure}	
	\begin{minipage}[]{0.1\textwidth}
		\
	\end{minipage}
	\begin{subfigure}[]{0.4\textwidth}
		\caption{Fossil energy in percentage deviation from business-as-usual}
		%	\captionsetup{width=.45\linewidth}
		\includegraphics[width=1\textwidth]{PerdifNoTauf_regime5_CompTaul_F_spillover0_nsk0_xgr0_knspil0_sep0_LFlimit0_emsbase0_countec0_GovRev0_etaa0.79_lgd0.png}
	\end{subfigure}
	%\begin{subfigure}[]{0.4\textwidth}
	%\caption{Emnet}
	%%	\captionsetup{width=.45\linewidth}
	%\includegraphics[width=1\textwidth]{PerdifNoTauf_regime5_CompTaul_Emnet_spillover0_nsk0_xgr0_knspil0_sep0_LFlimit0_emsbase0_countec0_GovRev0_etaa0.79_lgd0.png}
	%\end{subfigure}
	\floatfoot{Notes: Panel (a) shows levels of net emissions under a constant carbon tax equal to US\$185 (in 2020 prices) for a world with progressive income taxation at $\tau_{\iota}=0.181$, the solid graph, and without income taxation, $\tau_{\iota}=0$. The thin dotted graph shows the emission limit suggested by the IPCC. Panel (b) presents the percentage deviation from the business-as-usual policy in the economy (i) with and (ii) without income tax by the solid and dashed graphs, respectively.}
\end{figure} 

\paragraph{Static effect of a carbon tax}
% 1) reallocation of demand by energy producers and by final good producers.
Figure \ref{fig:Leveltauf_nsk0_xgr0_know} shows the effect of a constant carbon tax in a world with and without labor income tax represented by the solid and the dashed graphs, respectively. In this and all following figures, the x-axis indicates the first year of the 5-year period to which the variable value corresponds.\footnote{ Figure \ref{fig:Leveltauf_nsk0_xgr0_add} in Appendix \ref{app:polexp_cc} shows research related and other variables.}  


Panel (a) depicts how net emissions evolve under the carbon tax. The level of emissions is smaller in presence of a labor income tax with $\tau_{\iota}=0.181$.   A carbon tax of 185\$ per ton diminishes emissions by around 46\% initially relative to the business-as-usual (BAU) policy; i.e., without carbon tax. However, net emissions exceed the emission limit  derived previously; see the thin dotted line in Panel (a). Panel (b) shows the percentage deviation from the BAU allocation for fossil energy.

A carbon tax operates as follows. As energy producers face a higher price for fossil energy, they lower demand for fossil and rise demand for green energy. Fossil production falls, and green production rises.
The tax on fossil goods also increases the price for energy goods relative to non-energy goods on impact. Final good producers recompose their inputs towards non-energy goods. The energy share to GDP declines.  But, the recomposition is limited as energy and non-energy goods are complements. 

% 2) effect on research
The shift in demand by energy and final good producers induces a reallocation of research. In the model, the direction of research is determined by a market size effect, a price effect, and knowledge spillovers. 
A market size effect directs research to the sector with the bigger market; i.e., higher output. A price effect runs in the contrary direction rendering research in more expensive sectors more profitable. Which effect dominates depends on the degree of substitutability of goods \citep{Acemoglu2002DirectedChange, Hemous2021DirectedEconomics}. Cross-sectoral knowledge spillovers make research in less productive sectors more profitable.

Since green and fossil goods are sufficient substitutes, the market size effect dominates the price effect. As demand for the green good increases, profitability of research in  this sector rises. In contrast, research in the fossil sector falls. This makes the green good even cheaper contributing to an increase in the green-to-fossil energy ratio.
Non-energy research falls because knowledge spillovers to the energy sector direct research away from the non-energy sector.\footnote{ I examine the effect of a carbon tax on non-energy and energy research in Appendix \ref{app:polexp_cc}. Contrary to theory, the price effect does not dominate. It does not direct research to the more expensive good. The reason are heterogeneous labor shares which hamper a supply-side effect. Assume sectors share the same input good. As demand for the more expensive good falls, the cheaper sector can produce even cheaper because of a higher supply of input goods. This amplifies the price difference in goods. When the supply-side effect is muted since sectors have different production functions, the price difference may not be big enough to direct research to the more expensive good. The market size effect dominates directing research to the sector with the bigger market: in this case, non-energy goods. Yet, cross-sectoral knowledge spillovers from the non-energy to the energy sector are pivotal. All in all, the share of energy research increases.}

\paragraph{Dynamics}

%%%%%%%%%%%%%%%%%%%%%%%%%%%%%%%%%%%%%
%% knowledge spillovers
%%%%%%%%%%%%%%%%%%%%%%%%%%%%%%%%%%%%%
Over time, the effectiveness of the carbon tax to lower fossil production declines from 40.8\% to 39.3\%. Hence, to meet a net-zero emission limit, a continuous intervention is necessary. 
This finding is in contrast to the result by \cite{Acemoglu2012TheChange} who abstract from cross-sectoral knowledge spillovers and heterogeneity in labor shares. They conclude that when dirty and clean goods are sufficient substitutes, a temporary intervention suffices to prevent too high pollution. In contrast, in the present model, cross-sectoral knowledge spillovers and heterogeneous labor shares 
call for a continuous rise in the carbon tax to keep fossil energy from rising. 

Consider, first, the effect of cross-sectoral knowledge spillovers.
Initially, the carbon tax reduces research in the fossil sector, however, as green technology advances, knowledge spillovers from the green sector make fossil research more profitable again, and demand for fossil scientists resurges.
It is not only that a constant amount of researchers becomes more productive but also a change in the equilibrium level of fossil researchers which intensifies the effect of knowledge spillovers. This mechanism explains the quick rise in emissions under a constant carbon tax.\footnote{ Panel (a) in Figure \ref{fig:Leveltauf_nsk0_xgr0_noknow} shows the effect of a constant carbon tax in a model without cross-sectoral knowledge spillovers, $\phi=0$. The rise in net emissions over time is muted, and a constant carbon tax becomes more effective over time in reducing fossil production.  Absent cross-sectoral knowledge spillovers, more research is allocated to the green and non-energy sector.} 
Therefore, when knowledge spillovers are strong, reducing emissions to net-zero requires a continues intensification of environmental intervention. In its extreme, growth may have to stop eventually in order to prevent the fossil sector from growing too much.

%%%%%%%%%%%%%%%%%%%%%%%%%%%%%%%%%%%%%%%%%%%
% \paragraph{Role of heterogeneous labor shares}
%%%%%%%%%%%%%%%%%%%%%%%%%%%%%%%%%%%%%%%%%%%%%%
Consider now the effect of heterogeneous labor shares. The green sector has the smallest labor share. Labor is more important in the fossil and most important in the non-energy sector. This heterogeneity lowers the effectiveness of the carbon tax through a supply-side channel. 
A reduction in demand for labor in the fossil sector eases labor costs of the green sector. When, however, the green sector only uses a small share of labor, the higher labor supply does not lower green production costs as much, and the green good remains more expensive. The share of green energy and labor rises less. This weakens the effectiveness of directed technical change to foster green energy production. 
Panel (b) in Figure \ref{fig:Leveltauf_nsk0_xgr0_noknow} displays the behavior of key variables in a model with homogeneous labor shares across sectors.\footnote{ In this counterfactual calibration, I set capital shares equal across sectors to the average in the baseline calibration: $\alpha_g=\alpha_f=\alpha_n=0.66$. } 

%Now, absent a carbon tax, a smaller labor share in the green sector raises the share of labor allocated to the fossil sector over time. As the fossil sector becomes more productive, the marginal product of labor in the fossil sector increases more and labor transitions to the fossil sector. This mechanism offsets the effectiveness of a constant carbon tax over time. In contrast, when labor shares are equal, the effectiveness of the carbon tax to lower fossil production increases over time. 



%Endogenous growth intensifies this adverse effect of heterogeneous capital shares: as the market size of green goods is depressed, the carbon tax does not boost green research as much as with equal capital shares. 
%In sum, heterogeneous labor shares imply an increasing path of emissions over time under the carbon tax. 
%This feature of the economy also calls for a carbon tax increasing over time.

Panel (c) in Figure \ref{fig:Leveltauf_nsk0_xgr0_noknow} shows the result in a model variation without cross-sectoral knowledge spillovers and with equal labor shares. A constant carbon tax suffices to lower emissions over time. Then, endogenous growth directs research away from the fossil sector so that emissions continuously decline.  This finding is consistent with the result in \cite{Acemoglu2012TheChange}. %:  when clean and dirty goods are sufficiently substitutable, eventually, no further intervention may be necessary to satisfy environmental limits. % consistent in the sense that it does not conflict with

\paragraph{Effect of the income tax}

A progressive income tax lowers the level of emissions; compare the solid and the dashed graph in Panel (a) in Figure \ref{fig:Leveltauf_nsk0_xgr0_know}. As labor supply reduces, output shrinks, and emissions fall. 
However, there are compositional effects of a progressive income tax which (i) affect the economic structure and (ii) interact with the effectiveness of the carbon tax. I will explain each statement in turn. 

% \paragraph{Compositional effect of progressive income tax}
% compositional effect
A progressive income tax lowers the green-to-fossil energy ratio and diminishes the energy share in GDP. %\footnote{ Figure \ref{fig:Efftaul_nsk0_xgr0_know} in Appendix \ref{app:polexp_cc} displays the effect of a progressive income tax in presence of a constant carbon tax. }
The compositional effect of a progressive income tax originates from the asymmetric reaction of high- and low-skill workers. 
Tax progressivity affects labor supply via an income and a substitution effect. 
The income effect is similar across skill types due to the family structure of the household side.
The substitution effect, in contrast, is more pronounced for high-skill workers.
There are two reasons. First, post-tax income falls more the higher pre-tax income as progressivity rises. 
Second,  I assume that the marginal value of leisure rises with hours worked. Since the high skill work more, they require a higher wage rate to be compensated for an additional unit of labor. Hence, as tax progressivity rises, and the after-tax wage rate falls, the high skill reduce their labor supply more.  %\tr{The effect of a marginal increase in income tax progressivity intensifies with the level of pre-tax income.}
%I assume a constant Frisch elasticity of hours with respect to the wage rate. Hence, the responsiveness of labor with respect to the after-tax wage rate varies with hours worked.\footnote{ The percentage change in hours by high- and low-skilled workers to a percentage change in the wage rate is equivalent. Then, more hours worked initially imply a stronger reduction in absolute terms in response to a percentage change in the wage rate. }
Overall, the high-to-low skill ratio declines. 


Now, note that green production is skill biased as opposed to fossil production.
As a  consequence, green production becomes more expensive, while fossil production gets cheaper when tax progressivity rises. The price of non-energy goods, which are less skill-intense than energy goods, falls, too. Therefore, energy producers substitute fossil for green energy, and final good producers turn to non-energy goods. The former effect raises, the latter diminishes emissions.

% research

Research responds to the change in demand and in prices. First, non-energy research is less profitable due to its smaller price. Since non-energy and energy goods are complements, the amount of non-energy goods does not rise sufficiently to raise machine producers' profits despite the smaller price. As a result, research turns to the energy sector. The share of non-energy researchers reduces, albeit minimally. %; see Panel \eqref{figpan:nonre} in Figure \ref{fig:Efftaul_nsk0_xgr0_know}. %\footnote{ The effect of the progressive income tax also prevails with heterogeneous labor shares and absent knowledge spillovers. The price-effect dominates the direction of research.} 
Second, focusing solely on the allocation of researchers between the fossil and green sector, the relatively higher supply of low-skill labor raises the market size of the fossil good. Since intermediate energy goods are sufficient substitutes, the market size effect dominates the price effect and research shifts from green to fossil.
Although the share of green-to-fossil research drops, the absolute amount of green scientists increases. The reallocation of research to energy goods in general implies a rise in green researchers. 

While the labor tax affects the composition of research due to skill heterogeneity, there is no effect on aggregate research activity. 
To see this, I consider the model with homogeneous skills, then the labor tax has no compositional effect. In this model, the equilibrium amount of scientists remains unchanged by a progressive income tax. Indeed, demand for innovation reduces in response to a progressive income tax since less labor is available to work with technology. However, at the same time, scientists are willing to accept a lower wage rate since consumption of the household reduces. In equilibrium, the reduction in demand is absorbed by a change in the wage rate, and the level of aggregate research remains unchanged. 
% This finding is important to be kept in mind. It allows to differentiate the motive for labor taxation in the optimal policy analysis between  targeting (i) research and (ii) labor supply. 



\paragraph{Interaction of income and carbon taxes}
The compositional effect of a tax on labor interacts with the impact of the carbon tax.
Quantitatively, the effect of the carbon tax seems largely unaffected by the value of income tax progressivity.
Yet, there is a smaller reduction in fossil energy visible in Panel (b) in Figure \ref{fig:Leveltauf_nsk0_xgr0_know}.
This discrepancy emerges from the effect of a carbon tax on skill supply which is affected by the income tax. 

The carbon tax changes the skill premium since demand for green-specific high-skill labor increases. High-skill hours in equilibrium rise, while hours of low-skill workers reduce.  A progressive income tax lessens the effect of changes in the wage rate on labor supply. The reason is that the elasticity of after-tax labor income with respect to pre-tax labor income diminishes with a higher tax progressivity.\footnote{ Consider Panel (f) in Figure \ref{fig:Leveltauf_nsk0_xgr0_add} in Appendix \ref{app:polexp_cc}.} This mutes the supply response in the skill ratio to the carbon tax, and production costs of the green good remain high.


\subsection{Meeting the emission limit}\label{subsec:meetlim}

The previous section makes apparent that, first, the carbon tax suggested by the RFF does not cause emissions in line with the IPCC's emission limit.  Second, it shows that model dynamics call for an increasing carbon tax. In this section, I calculate the necessary carbon tax to meet the emission limit. I compare the resulting tax and allocations for the policy regimes with labor income tax (``combined policy") to a ``carbon-tax-only'' policy.


\begin{figure}[h!!]
	\centering
	\caption{Meeting the emission limit with and without preexisting labor tax }\label{fig:Limit_nsk0_xgr0_know}
	\begin{subfigure}[]{1\textwidth}
		\centering\footnotesize{\textbf{In levels}}\\ \vspace{2mm}
		\begin{subfigure}[]{0.4\textwidth}
			\caption{Tax per ton of carbon in 2022 US\$}
			%	\captionsetup{width=.45\linewidth}
			\includegraphics[width=1\textwidth]{CompTauf_bytaul_Reg5_Tauf_spillover0_nsk0_xgr0_knspil0_sep0_LFlimit1_emsbase0_countec0_GovRev0_etaa0.79_lgd1.png}
		\end{subfigure}	
		\begin{minipage}[]{0.1\textwidth}
			\
		\end{minipage}
		\begin{subfigure}[]{0.4\textwidth}
			\caption{Green-to-fossil energy ratio}
			%	\captionsetup{width=.45\linewidth}
			\includegraphics[width=1\textwidth]{CompTauf_bytaul_Reg5_GFF_spillover0_nsk0_xgr0_knspil0_sep0_LFlimit1_emsbase0_countec0_GovRev0_etaa0.79_lgd0.png}
		\end{subfigure}	 
	\end{subfigure}		
	
	\vspace{3mm}
	\begin{subfigure}[]{1\textwidth}
		\centering\footnotesize{\textbf{In percentage deviation from carbon-tax-only regime}}\\ \vspace{2mm}
		\begin{subfigure}[]{0.4\textwidth}
			\caption{Carbon tax}
			%	\captionsetup{width=.45\linewidth}
			\includegraphics[width=1\textwidth]{CompTaufPER_bytaul_Reg5_Tauf_spillover0_nsk0_xgr0_knspil0_sep0_LFlimit1_emsbase0_countec0_GovRev0_etaa0.79_lgd0.png} 
		\end{subfigure}
		\begin{minipage}[]{0.1\textwidth}
			\
		\end{minipage}
		\begin{subfigure}[]{0.4\textwidth}
			\caption{Green-to-fossil scientists}
			%	\captionsetup{width=.45\linewidth}
			\includegraphics[width=1\textwidth]{CompTaufPER_bytaul_Reg5_sgsff_spillover0_nsk0_xgr0_knspil0_sep0_LFlimit1_emsbase0_countec0_GovRev0_etaa0.79_lgd0.png} 
		\end{subfigure}		
	\end{subfigure}			
	\floatfoot{Notes: Panels (a) and (b) show variables in levels when the emission limit has to be met (i) in a scenario without income tax, $\tau_{\iota}=0$, dashed graphs, and (ii) in a scenario with labor income tax, $\tau_{\iota}=0.181$, solid graphs. Panel (c) and (d) depict percentage deviation in variables in the combined scenario to the carbon-tax-only scenario. A vertical line indicates when the net-zero emission limit becomes binding.}
\end{figure} 

% necessary carbon tax
%         0.8892    0.9516    1.0138    1.0765    1.1402    1.2047    2.6180    2.6986    2.7810    2.8651    2.9507    3.0376

Figure \ref{fig:Limit_nsk0_xgr0_know} shows the results.
Consider Panel (a). The necessary carbon tax ranges from US\$889  in the 2020-2024 period to around US\$2,951 in the 2070-2074 period (both in 2022 prices). The carbon tax is lower when labor is taxed: the deviation of the carbon tax reaches -10\% in initial periods but diminishes over time to approximately -7.25\% (Panel (c)).\footnote{ The smaller deviation under the net-zero emission limit results primarily from the tightness of the emission limit itself and not the presence of the labor income tax. Abstracting from all model features discussed earlier, the carbon tax nevertheless approaches the one without progressive income tax as the emission limit gets tighter. Hence, the more stringent emission limit calls for a more aggressive carbon tax despite the reductive effect of labor taxation.}

The combined policy results in a lower green-to-fossil energy ratio (Panel (b)) and a higher energy share to GDP. Yet, the reduction in economic activity induced by the labor income tax ensures that the emission limit is satisfied. On the upside, the combined policy enables a smoother distribution of energy scientists (Panel (d)). While the carbon tax induces a shift of researchers to the green sector, the labor tax lowers the green-to-fossil research share. 
In sum, the combined policy allows for a smoother distribution of scientists across sectors. This diminishes the reduction in scientists' productivity due to the higher research productivity in the fossil sector from within-sector knowledge spillovers.\footnote{ I discuss the effects of cross-sectoral knowledge spillovers, heterogeneous skills, heterogeneous labor shares, and endogenous growth in Appendix \ref{app:modvar}. }
