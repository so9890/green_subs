% 15 minutes teaching philosophy 

%he next 15 minutes are devoted to presenting your teaching philosophy, 
%including your experience in terms of courses and methods as well as
%intentions and general principles of your pedagogical concept. 

\section*{Teaching Philosophy}

\begin{frame}{Outline}
	\vspace{-15mm}\hspace{-10mm}
	\begin{minipage}{1\textwidth}
\begin{itemize}
	\item[] Experience
	\vspace{6mm}
	\item[] General Principles
	\vspace{6mm}
	\item[] Intention
\end{itemize}

\end{minipage}
\end{frame}
\section*{Experience}
\begin{frame}{Experience: my teaching in the past}
	\begin{itemize}[<+->]
		\item Tutor undergraduate \textit{Introductory  Macroeconomics}, several times
		\vspace{2mm}
		\begin{itemize}
			\item[-] discussion of problem sets
		\end{itemize}
	\vspace{3mm}
		\item Teaching Assistant undergraduate \textit{Monetary Theory and Policy} 
		\vspace{2mm}		
		\begin{itemize}
			\item[-] offered room to ask questions 
		%	\item[-] assistence for exam preparation 
			\item[-] during the COVID-19 pandemic for those with special needs
			\item[-] helped design final exam
		\end{itemize}
	\end{itemize}
\end{frame}

\section*{General Principles}

\begin{frame}{Outline}
\begin{itemize}[<+->]
	\item[1.] What to teach: content 
	\item[2.] How to teach: learning environment
	\item[\ar 3.] Choice and design of methods
\end{itemize}
\end{frame}
\begin{frame}{Content: goals of education}
	\pause
	\centering
	\vspace{-4mm}
	\begin{itemize}[<+->]
\item[] \hspace{-10mm} \centering \textbf{Education in general}
\vspace{2mm}
\begin{itemize}
	\item To enable students:
	\begin{itemize}
		\item[-]  to develop personally 
		%	\item  to be successful in life % to know what they want
		\vspace{1mm}
		\item[-]  to take on responsibility within society %develop the capacity to be responsible a sense for society\textit{Verantwortung }
	\end{itemize}
\end{itemize}
%\vspace{2mm}
%\item[] \hspace{-9.4mm} \centering \textbf{University education}
%\begin{itemize}
%	\item to discover own \textbf{research} interests 
%	%	\item critical, independent thinking
%	% Make a choice in what direction to develop
%	\item to enable students to make \textbf{competent} contributions to society
%\end{itemize}
\vspace{2mm}
\item[] \hspace{-10mm} \centering \textbf{Economic university eduction}
\vspace{2mm}
\begin{itemize}
	\item To enable students to understand and assess economic phenomena %(undergraduate)
			\vspace{1mm}
		\item[\ar] \alert{Content dimension 1: make expert economics \textbf{knowledge}
	accessible}
		\vspace{1mm}
		\item To enable students to develop and answer economic research questions
		%	\item capability independent knowledge acquisition %(graduate/ doctoral students)
				\vspace{1mm}
	\item Contribute to overall goals of education
			\vspace{1mm}
	\item[\ar]  \alert{Content dimension 2: \textbf{competencies}} 
\end{itemize}
	\end{itemize}
\end{frame}


\begin{frame}{Content: competencies}
	\pause
		\begin{figure}[h]
		\vspace{-12.4mm}
		\centering
		\begin{tikzpicture}[auto,scale=.63, transform shape]
			\node[circllB] (B) at (-7,15) {\huge\textbf{ Critical Thinking}\ \ \ };
			\pause
			\node[circllB] (A) at (-2,19) {\huge\textbf{Communication}};
			\pause 
			\node[circllB] (C) at (3,15) {\huge\textbf{{\ \ \ \  Collaboration\ \ \ }} };
			\pause
			\node[circllB] (D) at (-2,11) {\\ \ \\ \  \huge\textbf{{Creativity}}\\ \ };
			%	\node[circll,fill=black!1] (E) at (-2,15) {\alert{\textbf{Effective Research}}\\ \textbf{\alert{Social Responsibility}}\\ \textbf{\alert{Personal Development}}} ;
			
			\end{tikzpicture}	
	\end{figure}	
\end{frame}

\addtocounter{framenumber}{-1}
\begin{frame}{Content: competencies}
	\begin{figure}[h]
		\vspace{-12.4mm}
		\centering
		\begin{tikzpicture}[auto,scale=.63, transform shape]
			\node[circllB] (B) at (-7,15) {\huge\textbf{ Critical Thinking}\ \ \ };
			\node[circllB] (A) at (-2,19) {\huge\textbf{Communication}}; 
			\node[circllB] (C) at (3,15) {\huge\textbf{{\ \ \ \  Collaboration\ \ \ }} };
			\node[circllB] (D) at (-2,11) {\\ \ \\ \  \huge\textbf{{Creativity}}\\ \ };
				\node[circll,fill=black!1] (E) at (-2,15) {\alert{\textbf{Effective Research}}\\ \textbf{\alert{Social Responsibility}}\\ \textbf{\alert{Personal Development}}} ;
			
		\end{tikzpicture}	
	\end{figure}	
\end{frame}



\begin{frame}{How to teach: learning environment} % ausgestaltung
	\vspace{-2.5mm}
	\pause
	\begin{itemize}% \ar
%		\begin{itemize}
%			\item[-] real world examples \item[-] students choose specialization fields
%		\end{itemize} 
		\item<+-> Create an \alert{inclusive} learning environment % in the sense of belonging in the classroom 
		\begin{itemize}[<+->]
			\item[-] use of inclusive language/ choice of examples
			\item[-] give time to think before welcoming answers
			\item[-] acknowledgment of merits
			\item[-] ask students for feedback 
		\end{itemize}
	\vspace{1mm}
		\item<+-> \alert{Mentoring} tailored to \alert{individual needs}
		\begin{itemize}[<+->]
			\item[a)] provide opportunities to deepen knowledge beyond class content 
			\begin{itemize}
				\item[-] information on relevant journal articles, data sets
			\end{itemize}
			\item[b)]  active mentoring of minorities and those with special needs
			\begin{itemize}
				\item[-] make videos of lectures available %, offer hybrid format 
				\item[-] monitor learning success to be able to support where necessary early on 
				\item[-] ensure easily approachable %/ be available to discuss questions outside class
			\end{itemize}
		\end{itemize}		
		\vspace{1mm}		
		\item<+-> Stimulate \alert{intrinsic interest} in the topic; e.g.: 
		\begin{itemize}[<+->]
			\item[-] real world examples/ projects % assessing environmental policies currently discussed
		%	\item free choice of topic 
		\end{itemize}
	\end{itemize}
\end{frame}

\begin{frame}
	\vspace{5mm}
	\centering
	\begin{tikzpicture}[auto,scale=.7, transform shape]
		% QUESTION
		%	 	  \node[] (A) at (-10,9) {Question:\ \ \ \ \ \ \ \ \ };
		\node[] (A) at (0,9) {\huge \textbf{\alert{Methods}} };
		\pause 
		\node[modus] (B) at (-7,7) {\textbf{``Conveying''}\\ \textbf{Expert Knowledge}};
		\node[modus] (E) at (-2.7,7) {\textbf{\hyperlink{backhh}{{Critical}}}\\ {\textbf{Thinking}}};
		% Dimensions
		\node[modus] (C) at (9.2,7) {\textbf{\hyperlink{backhh}{{Collaboration}}}\\ \ };
		\node[modus] (D) at (0.8,7) {\textbf{\hyperlink{backhh}{{Creativity}}}\\ \ };
		\node[modus] (D) at (4.8,7) {\textbf{\hyperlink{backhh}{{Communication}}}\\ \ };
		
		%%%%%%%%%%%%%%%%%%%%%%%%%%%%%%%%%%%%%%%%%%%%%%%%%%%%%%%%%%%%%%%%%%%%%%%
		%	  \node[] (A) at (-10,3) {Projects: \ \ \ \ \ \ \ \ \ };	
		\node[elli, fill=black!1, draw=black!1] (G) at (-2.4,2.5) {\ \ \bb{\textbf{Specialization exercises}:}\\ \ \ \bb{Defining research question or topic} }; % could be essay, presentations
		\node[elli, fill=black!1, draw=black!1] (F) at (-6.9,3.8) { \ \ \ \ \bb{\textbf{Lectures}}\ \ \ \ };
		\node[elli, fill=black!1, draw=black!1] (I) at (4.8,3) {\bb{{\textbf{Presentations and}  \textbf{essays}}}};
		%		\node[elli, fill=black!1] (I) at (4.8,2.8) {\ \ \ \ \ \ \ \ \ \ \textbf{Essays}\ \ \ \ \ \ \ \ \ \ };
		\node[elli, fill=black!1, draw=black!1] (I) at (4.8,1) { \ \ \ \ \ \ \ \ \ \ \ \ \ \ \ \bb{\textbf{Role plays}}\ \ \ \ \ \ \ \ \ \ \ \ \ \ \  };
		\node[elli, fill=black!1, draw=black!1] (I) at (0.8,-1) {\ \ \ \ \ \ \ \ \ \ \ \ \ \ \ \ \ \ \ \ \ \ \ \ \ \ \ \ \ \ \ \   \ \ \ \ \ \ \ \ \ \  \bb{\textbf{Real world, hands-on projects}}\ \ \ \ \ \ \ \ \ \ \ \ \ \ \ \ \ \ \ \ \ \ \ \ \  };
		\node[elli, fill=black!1, draw=black!1] (G) at (-4.6,0) {\bb{Inviting a \textbf{multitude of perspectives}}};
		\begin{pgfonlayer}{bg}    % select the background layer
			\node[elli, fill=black!1, draw=black!1]  (I) at (8.5,4.5) {\ \ \ \ \  \bb{\textbf{Discussions}}\ \ \ \ \  };
			\node[elli, fill=black!1, draw=black!1] (H) at (-0.5,4.5) {\ \ \ \ \ \ \ \ \ \ \ \ \ \ \ \ \ \ \bb{\textbf{Flipping the classroom}}\ \ \ \ \ \ \ \ \ \ \ \ \ \ \ \ \ \ };
		\end{pgfonlayer}
		
	\end{tikzpicture}
\end{frame}

\addtocounter{framenumber}{-1}

\begin{frame}
	\vspace{5mm}
	\centering
	\begin{tikzpicture}[auto,scale=.7, transform shape]
		% QUESTION
		%	 	  \node[] (A) at (-10,9) {Question:\ \ \ \ \ \ \ \ \ };
		\node[] (A) at (0,9) {\huge \textbf{\alert{Methods}} }; 
		\node[modus] (B) at (-7,7) {\textbf{``Conveying''}\\ \textbf{Expert Knowledge}};
		\node[modus] (E) at (-2.7,7) {\textbf{\hyperlink{backhh}{{Critical}}}\\ {\textbf{Thinking}}};
		% Dimensions
		\node[modus] (C) at (9.2,7) {\textbf{\hyperlink{backhh}{{Collaboration}}}\\ \ };
		\node[modus] (D) at (0.8,7) {\textbf{\hyperlink{backhh}{{Creativity}}}\\ \ };
		\node[modus] (D) at (4.8,7) {\textbf{\hyperlink{backhh}{{Communication}}}\\ \ };
		
		%%%%%%%%%%%%%%%%%%%%%%%%%%%%%%%%%%%%%%%%%%%%%%%%%%%%%%%%%%%%%%%%%%%%%%%
		%	  \node[] (A) at (-10,3) {Projects: \ \ \ \ \ \ \ \ \ };	
		
		\node[elli, fill=black!1] (F) at (-6.9,3.8) { \ \ \ \ \textbf{Lectures}\ \ \ \ };
		\node[elli, fill=black!1] (G) at (-2.4,2.5) {\ \ \textbf{Specialization exercises}:\\ \ \ Defining research question or topic }; % could be essay, presentations
		\node[elli, fill=black!1] (I) at (4.8,3) {\textbf{Presentations and}  \textbf{essays}};		
		\node[elli, fill=black!1] (I) at (9.2,2) {\ \ \ \ \  \textbf{Group work}\ \ \ \ \  };
		%		\pause
		\node[elli, fill=black!1] (G) at (-4.6,0) {Inviting a \textbf{multitude of perspectives}};
		
		%	\pause
		%		\node[draw=none] (I) at (4.8,2.8) {\ \ \ \ \ \ \ \ \ \ \bb{\textbf{Essays}}\ \ \ \ \ \ \ \ \ \ };
		\node[elli, fill=black!1] (I) at (4.8,1) { \ \ \ \ \ \ \ \ \ \ \ \ \ \ \ {{\textbf{Role plays}}}\ \ \ \ \ \ \ \ \ \ \ \ \ \ \  };
		\begin{pgfonlayer}{bg}    % select the background layer
			\node[elli, fill=black!1]  (I) at (8.5,4.5) {\ \ \ \ \  {\textbf{Discussions}}\ \ \ \ \  };
			\node[elli, fill=black!1] (H) at (-0.5,4.5) {\ \ \ \ \ \ \ \ \ \ \ \ \ \ \ \ \ \ {{\textbf{Flipping the classroom}}}\ \ \ \ \ \ \ \ \ \ \ \ \ \ \ \ \ \ };
			\node[elli, fill=black!1] (I) at (0.8,-1) {\ \ \ \ \ \ \ \ \ \ \ \ \ \ \ \ \ \ \ \ \ \ \ \ \ \ \ \ \ \ \ \   \ \ \ \ \ \ \ \ \ \ {\textbf{Real world, hands-on projects}}\ \ \ \ \ \ \ \ \ \ \ \ \ \ \ \ \ \ \ \ \ \ \ \ \  };
		\end{pgfonlayer}
		
	\end{tikzpicture}
\end{frame}

\addtocounter{framenumber}{-1}

\begin{frame}
	\vspace{5mm}
	\centering
	\begin{tikzpicture}[auto,scale=.7, transform shape]
		% QUESTION
		%	 	  \node[] (A) at (-10,9) {Question:\ \ \ \ \ \ \ \ \ };
		\node[] (A) at (0,9) {\huge \textbf{\alert{Methods}} }; 
		\node[modus] (B) at (-7,7) {\textbf{``Conveying''}\\ \textbf{Expert Knowledge}};
		\node[modus] (E) at (-2.7,7) {\textbf{\hyperlink{backhh}{{Critical}}}\\ {\textbf{Thinking}}};
		% Dimensions
		\node[modus] (C) at (9.2,7) {\textbf{\hyperlink{backhh}{{Collaboration}}}\\ \ };
		\node[modus] (D) at (0.8,7) {\textbf{\hyperlink{backhh}{{Creativity}}}\\ \ };
		\node[modus] (D) at (4.8,7) {\textbf{\hyperlink{backhh}{{Communication}}}\\ \ };
		
		%%%%%%%%%%%%%%%%%%%%%%%%%%%%%%%%%%%%%%%%%%%%%%%%%%%%%%%%%%%%%%%%%%%%%%%
		%	  \node[] (A) at (-10,3) {Projects: \ \ \ \ \ \ \ \ \ };	
		
		\node[elli, fill=black!1] (F) at (-6.9,3.8) { \ \ \ \ \textbf{Lectures}\ \ \ \ };
		\node[elli, fill=black!1] (G) at (-2.4,2.5) {\ \ \textbf{Specialization exercises}:\\ \ \ Defining research question or topic }; % could be essay, presentations
		\node[elli, fill=black!1] (I) at (4.8,3) {\textbf{Presentations and}  \textbf{essays}};		
		\node[elli, fill=black!1] (I) at (9.2,2) {\ \ \ \ \  \textbf{Group work}\ \ \ \ \  };
		%		\pause
		\node[elli, fill=black!1] (G) at (-4.6,0) {Inviting a \textbf{multitude of perspectives}};
		
		%	\pause
		%		\node[draw=none] (I) at (4.8,2.8) {\ \ \ \ \ \ \ \ \ \ \bb{\textbf{Essays}}\ \ \ \ \ \ \ \ \ \ };
		\node[elli, fill=black!1] (I) at (4.8,1) { \ \ \ \ \ \ \ \ \ \ \ \ \ \ \ {{\textbf{Role plays}}}\ \ \ \ \ \ \ \ \ \ \ \ \ \ \  };
		\begin{pgfonlayer}{bg}    % select the background layer
			\node[elli, fill=black!1]  (I) at (8.5,4.5) {\ \ \ \ \  {\textbf{Discussions}}\ \ \ \ \  };
			\node[elli, fill=black!1] (H) at (-0.5,4.5) {\ \ \ \ \ \ \ \ \ \ \ \ \ \ \ \ \ \ {\alert{\textbf{Flipping the classroom}}}\ \ \ \ \ \ \ \ \ \ \ \ \ \ \ \ \ \ };
			\node[elli, fill=black!1] (I) at (0.8,-1) {\ \ \ \ \ \ \ \ \ \ \ \ \ \ \ \ \ \ \ \ \ \ \ \ \ \ \ \ \ \ \ \   \ \ \ \ \ \ \ \ \ \ {\textbf{Real world, hands-on projects}}\ \ \ \ \ \ \ \ \ \ \ \ \ \ \ \ \ \ \ \ \ \ \ \ \  };
		\end{pgfonlayer}
		
	\end{tikzpicture}
\end{frame}



\begin{frame}{Flipping the classroom}
	\pause
	\begin{itemize}[<+->]
		\item Depending on size of course: let students \alert{collaboratively} decide on related sub-fields they want to learn about
		\item Preparation of class
		\begin{itemize}
			\item[-] independent \alert{knowledge acquisition}
			\item[-] \alert{critical} assessment of information
		\end{itemize}
		\item Students can be \alert{creative} on how to \alert{communicate} information to classmates
		\end{itemize}
\end{frame}


\begin{frame}
	\vspace{0mm}
	\hfill \hyperlink{int}{\tiny{$\rightarrow$ intention}}
	\vspace{3mm}
	\centering
	\begin{tikzpicture}[auto,scale=.7, transform shape]
		% QUESTION
		%	 	  \node[] (A) at (-10,9) {Question:\ \ \ \ \ \ \ \ \ };
		\node[] (A) at (0,9) {\huge \textbf{\alert{Methods}} }; 
		\node[modus] (B) at (-7,7) {\textbf{``Conveying''}\\ \textbf{Expert Knowledge}};
		\node[modus] (E) at (-2.7,7) {\textbf{\hyperlink{backhh}{{Critical}}}\\ {\textbf{Thinking}}};
		% Dimensions
		\node[modus] (C) at (9.2,7) {\textbf{\hyperlink{backhh}{{Collaboration}}}\\ \ };
		\node[modus] (D) at (0.8,7) {\textbf{\hyperlink{backhh}{{Creativity}}}\\ \ };
		\node[modus] (D) at (4.8,7) {\textbf{\hyperlink{backhh}{{Communication}}}\\ \ };
		
		%%%%%%%%%%%%%%%%%%%%%%%%%%%%%%%%%%%%%%%%%%%%%%%%%%%%%%%%%%%%%%%%%%%%%%%
		%	  \node[] (A) at (-10,3) {Projects: \ \ \ \ \ \ \ \ \ };	
		
		\node[elli, fill=black!1] (F) at (-6.9,3.8) { \ \ \ \ \textbf{Lectures}\ \ \ \ };
		\node[elli, fill=black!1] (G) at (-2.4,2.5) {\ \ \textbf{Specialization exercises}:\\ \ \ Defining research question or topic }; % could be essay, presentations
		\node[elli, fill=black!1] (I) at (4.8,3) {\textbf{Presentations and}  \textbf{essays}};		
		\node[elli, fill=black!1] (I) at (9.2,2) {\ \ \ \ \  \textbf{Group work}\ \ \ \ \  };
		%		\pause
		\node[elli, fill=black!1] (G) at (-4.6,0) {Inviting a \textbf{multitude of perspectives}};
		
		%	\pause
		%		\node[draw=none] (I) at (4.8,2.8) {\ \ \ \ \ \ \ \ \ \ \bb{\textbf{Essays}}\ \ \ \ \ \ \ \ \ \ };
		\node[elli, fill=black!1] (I) at (4.8,1) { \ \ \ \ \ \ \ \ \ \ \ \ \ \ \ {{\textbf{Role plays}}}\ \ \ \ \ \ \ \ \ \ \ \ \ \ \  };
		\begin{pgfonlayer}{bg}    % select the background layer
			\node[elli, fill=black!1]  (I) at (8.5,4.5) {\ \ \ \ \  {\textbf{Discussions}}\ \ \ \ \  };
			\node[elli, fill=black!1] (H) at (-0.5,4.5) {\ \ \ \ \ \ \ \ \ \ \ \ \ \ \ \ \ \ {{\textbf{Flipping the classroom}}}\ \ \ \ \ \ \ \ \ \ \ \ \ \ \ \ \ \ };
			\node[elli, fill=black!1] (I) at (0.8,-1) {\ \ \ \ \ \ \ \ \ \ \ \ \ \ \ \ \ \ \ \ \ \ \ \ \ \ \ \ \ \ \ \   \ \ \ \ \ \ \ \ \ \ {\textbf{Real world, hands-on projects}}\ \ \ \ \ \ \ \ \ \ \ \ \ \ \ \ \ \ \ \ \ \ \ \ \  };
		\end{pgfonlayer}
		
	\end{tikzpicture}
\end{frame}



\addtocounter{framenumber}{-1}

\begin{frame}
\vspace{0mm}
\hfill \hyperlink{int}{\tiny{$\rightarrow$ intention}}
\vspace{3mm}
	\centering
	\begin{tikzpicture}[auto,scale=.7, transform shape]
		% QUESTION
		%	 	  \node[] (A) at (-10,9) {Question:\ \ \ \ \ \ \ \ \ };
		\node[] (A) at (0,9) {\huge \textbf{\alert{Methods}} }; 
		\node[modus] (B) at (-7,7) {\textbf{``Conveying''}\\ \textbf{Expert Knowledge}};
		\node[modus] (E) at (-2.7,7) {\textbf{\hyperlink{backhh}{{Critical}}}\\ {\textbf{Thinking}}};
		% Dimensions
		\node[modus] (C) at (9.2,7) {\textbf{\hyperlink{backhh}{{Collaboration}}}\\ \ };
		\node[modus] (D) at (0.8,7) {\textbf{\hyperlink{backhh}{{Creativity}}}\\ \ };
		\node[modus] (D) at (4.8,7) {\textbf{\hyperlink{backhh}{{Communication}}}\\ \ };
		
		%%%%%%%%%%%%%%%%%%%%%%%%%%%%%%%%%%%%%%%%%%%%%%%%%%%%%%%%%%%%%%%%%%%%%%%
		%	  \node[] (A) at (-10,3) {Projects: \ \ \ \ \ \ \ \ \ };	
		
		\node[elli, fill=black!1] (F) at (-6.9,3.8) { \ \ \ \ \textbf{Lectures}\ \ \ \ };
		\node[elli, fill=black!1] (G) at (-2.4,2.5) {\ \ \textbf{Specialization exercises}:\\ \ \ Defining research question or topic }; % could be essay, presentations
		\node[elli, fill=black!1] (I) at (4.8,3) {\textbf{Presentations and}  \textbf{essays}};		
		\node[elli, fill=black!1] (I) at (9.2,2) {\ \ \ \ \  \textbf{Group work}\ \ \ \ \  };
		%		\pause
		\node[elli, fill=black!1] (G) at (-4.6,0) {Inviting a \textbf{multitude of perspectives}};
		
		%	\pause
		%		\node[draw=none] (I) at (4.8,2.8) {\ \ \ \ \ \ \ \ \ \ \bb{\textbf{Essays}}\ \ \ \ \ \ \ \ \ \ };
		\node[elli, fill=black!1] (I) at (4.8,1) { \ \ \ \ \ \ \ \ \ \ \ \ \ \ \ {\alert{\textbf{Role plays}}}\ \ \ \ \ \ \ \ \ \ \ \ \ \ \  };
		\begin{pgfonlayer}{bg}    % select the background layer
			\node[elli, fill=black!1]  (I) at (8.5,4.5) {\ \ \ \ \  \alert{\textbf{Discussions}}\ \ \ \ \  };
			\node[elli, fill=black!1] (H) at (-0.5,4.5) {\ \ \ \ \ \ \ \ \ \ \ \ \ \ \ \ \ \ {{\textbf{Flipping the classroom}}}\ \ \ \ \ \ \ \ \ \ \ \ \ \ \ \ \ \ };
			\node[elli, fill=black!1] (I) at (0.8,-1) {\ \ \ \ \ \ \ \ \ \ \ \ \ \ \ \ \ \ \ \ \ \ \ \ \ \ \ \ \ \ \ \   \ \ \ \ \ \ \ \ \ \ {\textbf{Real world, hands-on projects}}\ \ \ \ \ \ \ \ \ \ \ \ \ \ \ \ \ \ \ \ \ \ \ \ \  };
		\end{pgfonlayer}
		
	\end{tikzpicture}
\end{frame}


\begin{frame}{Discussions}
	% design of discussions as to incorporate other methods!
	\pause
	\begin{itemize}[<+->]
		\item Provocative statements to \alert{facilitate participation} in the discussion
	%	\item use of inclusive language and examples
		\item Ask students to prepare questions on the material prior to class	(\alert{critical thinking})	
		\item {Role plays}
		\begin{itemize}
			\item[-] ask for counterarguments  or assign roles explicitly
		%	\item[-] to lower threshold of participation
			\item[-] \alert{change of perspective}
		\end{itemize}
	\item Discussions/role plays in \alert{groups}
	\end{itemize}
\end{frame}


\begin{frame}
	\vspace{5mm}
	\centering
	\begin{tikzpicture}[auto,scale=.7, transform shape]
		% QUESTION
		%	 	  \node[] (A) at (-10,9) {Question:\ \ \ \ \ \ \ \ \ };
		\node[] (A) at (0,9) {\huge \textbf{\alert{Methods}} }; 
		\node[modus] (B) at (-7,7) {\textbf{``Conveying''}\\ \textbf{Expert Knowledge}};
		\node[modus] (E) at (-2.7,7) {\textbf{\hyperlink{backhh}{{Critical}}}\\ {\textbf{Thinking}}};
		% Dimensions
		\node[modus] (C) at (9.2,7) {\textbf{\hyperlink{backhh}{{Collaboration}}}\\ \ };
		\node[modus] (D) at (0.8,7) {\textbf{\hyperlink{backhh}{{Creativity}}}\\ \ };
		\node[modus] (D) at (4.8,7) {\textbf{\hyperlink{backhh}{{Communication}}}\\ \ };
		
		%%%%%%%%%%%%%%%%%%%%%%%%%%%%%%%%%%%%%%%%%%%%%%%%%%%%%%%%%%%%%%%%%%%%%%%
		%	  \node[] (A) at (-10,3) {Projects: \ \ \ \ \ \ \ \ \ };	
		
		\node[elli, fill=black!1] (F) at (-6.9,3.8) { \ \ \ \ \textbf{Lectures}\ \ \ \ };
		\node[elli, fill=black!1] (G) at (-2.4,2.5) {\ \ \textbf{Specialization exercises}:\\ \ \ Defining research question or topic }; % could be essay, presentations
		\node[elli, fill=black!1] (I) at (4.8,3) {\textbf{Presentations and}  \textbf{essays}};		
		\node[elli, fill=black!1] (I) at (9.2,2) {\ \ \ \ \  \textbf{Group work}\ \ \ \ \  };
		%		\pause
		\node[elli, fill=black!1] (G) at (-4.6,0) {Inviting a \textbf{multitude of perspectives}};
		
		%	\pause
		%		\node[draw=none] (I) at (4.8,2.8) {\ \ \ \ \ \ \ \ \ \ \bb{\textbf{Essays}}\ \ \ \ \ \ \ \ \ \ };
		\node[elli, fill=black!1] (I) at (4.8,1) { \ \ \ \ \ \ \ \ \ \ \ \ \ \ \ {\textbf{Role plays}}\ \ \ \ \ \ \ \ \ \ \ \ \ \ \  };
		\begin{pgfonlayer}{bg}    % select the background layer
			\node[elli, fill=black!1]  (I) at (8.5,4.5) {\ \ \ \ \  {\textbf{Discussions}}\ \ \ \ \  };
			\node[elli, fill=black!1] (H) at (-0.5,4.5) {\ \ \ \ \ \ \ \ \ \ \ \ \ \ \ \ \ \ {{\textbf{Flipping the classroom}}}\ \ \ \ \ \ \ \ \ \ \ \ \ \ \ \ \ \ };
			\node[elli, fill=black!1] (I) at (0.8,-1) {\ \ \ \ \ \ \ \ \ \ \ \ \ \ \ \ \ \ \ \ \ \ \ \ \ \ \ \ \ \ \ \   \ \ \ \ \ \ \ \ \ \ {\textbf{Real world, hands-on projects}}\ \ \ \ \ \ \ \ \ \ \ \ \ \ \ \ \ \ \ \ \ \ \ \ \  };
		\end{pgfonlayer}
		
	\end{tikzpicture}
\end{frame}


%
%\begin{frame}{How to convey competencies and expert knowledge: methods}
%	
%	\begin{enumerate}
%		\item<+-> Provide room for development and stimulate interest
%		\begin{enumerate}
%	\item<+-> {make economics accessible: knowledge transfer} % creating the room
%	\item<+-> {give opportunities to specialize based on own interests}
%	%			%			\item provide opportunities to shape specialization content of classes
%\end{enumerate}
%		\item<+-> Invite multitudes of perspectives
%%		\begin{itemize}
%%			\item[] \textcolor{black!1}{classroom discussions}
%%			\item[] \textcolor{black!1}{plays\\ \ }
%%		\end{itemize}
%		\item<+-> Active mentoring of minorities and those with special needs
%	\end{enumerate}
%\end{frame}






%\begin{frame}{Real World Projects}
%	\begin{itemize}
%		\item either: students take up role of government seeking to 
%		\item data work on country 
%		\item 
%	\end{itemize}
%\end{frame}


\begin{comment}


\begin{frame}{How to achieve these goals?: {Role of teacher} and \alert{methods}}
	
	\begin{enumerate}
		\item Provide room for development and  stimulate interest
				\begin{enumerate}
					\item {make economics accessible: knowledge transfer} % creating the room
						\item {provide opportunities to specialize based on own interests}
			%			%			\item provide opportunities to shape specialization content of classes
					\end{enumerate}
		\item Invite multitudes of perspectives
		%		\begin{itemize}
			%			\item[] \textcolor{black!1}{classroom discussions}
			%			\item[] \textcolor{black!1}{plays\\ \ }
			%		\end{itemize}
		\item Active mentoring of minorities and those with special needs
	\end{enumerate}
\end{frame}


\begin{frame}{How to achieve these goals?: Role of teacher and\alert{ methods}}
	\begin{enumerate}
		\item \alert{\textbf{Provide room for development and stimulate interest}}
		\begin{enumerate}
			\item make economics accessible: \alert{knowledge transfer} % being able to think critically about economics and economic research
			\begin{itemize}
				\item[-] lectures
				\item[-] real world examples
				\item[-] hands-on problem sets
			\end{itemize}
			\item provide \alert{opportunities to specialize} based on own interests %\\ %(individual work/ shape specialization content of classes)
			\begin{itemize}
				\item[-] opportunity to shape specialization content of classes
				\item[-] flipping the classroom; student presentations
				\item[-] opportunity to define content of essays/projects %hands on (data/programming) exercises
				\item[-] ask students to prepare questions prior to lectures
			\end{itemize}
			%			\item provide opportunities to shape specialization content of classes
		\end{enumerate}
		\item Invite various perspectives
%		\begin{itemize}
%			\item[-] classroom discussions 
%			\item[-] encourage group work
%			\item[-] plays
%			\item[-] interdisciplinary approach % e.g. moral philosophy when it comes to discussing welfare functions
%		\end{itemize}
		\item Active mentoring of minorities and those with special needs
%				\begin{itemize}
%			\item[-] availability of videos of lectures/ hybrid format
%			\item[-] monitor learning success to be able to support where necessary
%			\item[-] ensure easily approachable by students and actively approach students to learn about their needs
%			\item[-] use of inclusive language
%			\item[-] be available to discuss questions
%		\end{itemize}
	\end{enumerate}
\end{frame}



\begin{frame}{How to achieve these goals?: Role of teacher \alert{and methods}}
	\begin{enumerate}
		\item Provide room for development and stimulate interest
%		\begin{enumerate}
%			\item make economics accessible: knowledge transfer % creating the room
%			\item provide opportunities to specialize based on own interests\\ (individual work/ shape specialization content of classes)
%			\begin{itemize}
%				\item flipping the classroom; student presentations
%				\item essays/ hands on (data/programming) exercises
%				\item ask students to prepare questions prior to lecture
%			\end{itemize}
%			%			\item provide opportunities to shape specialization content of classes
%		\end{enumerate}
		\item \alert{\textbf{Invite various perspectives}}
		\begin{itemize}[<+->]
			\item[-] classroom discussions
			\begin{itemize}
				\item[-] start with provocative statement
				\item[-] explicitly  ask for counterarguments
				\item[-] assign roles
			\end{itemize}
			\item[-] plays (in small groups)
			\item[-] encourage group work in- and outside the classroom
			\item[-] interdisciplinary approach
			\begin{itemize}
				\item[-] highlight where other disciplines become relevant; e.g. natural sciences in environmental economics, philosophy for welfare/ utility functions
				\item[-] alternative approaches from other fields in- and outside of economics; e.g. behavioral econ and macroeconomics
			\end{itemize}
			 % e.g. moral philosophy when it comes to discussing welfare functions
		\end{itemize}
		\item Active mentoring of minorities and those with special needs
%		\begin{itemize}
%			\item[-] availability of videos of lectures/ hybrid format
%			\item[-] monitor learning success to be able to support where necessary
%			\item[-] ensure easily approachable by students and actively approach students to learn about their needs
%			\item[-] use of inclusive language
%			\item[-] be available to discuss questions
%		\end{itemize}
	\end{enumerate}
\end{frame}

\begin{frame}{How to achieve these goals?: Role of teacher \alert{and methods}}
	\begin{enumerate}
		\item Provide room for development and stimulate interest
		%		\begin{enumerate}
			%			\item make economics accessible: knowledge transfer % creating the room
			%			\item provide opportunities to specialize based on own interests\\ (individual work/ shape specialization content of classes)
			%			\begin{itemize}
				%				\item flipping the classroom; student presentations
				%				\item essays/ hands on (data/programming) exercises
				%				\item ask students to prepare questions prior to lecture
				%			\end{itemize}
			%			%			\item provide opportunities to shape specialization content of classes
			%		\end{enumerate}
		\item Invite various perspectives
%		\begin{itemize}
%			\item[-] classroom discussions 
%			\item[-] encourage group work
%			\item[-] plays
%			\item[-] interdisciplinary approach % e.g. moral philosophy when it comes to discussing welfare functions
%		\end{itemize}
		\item \alert{\textbf{Active mentoring of minorities and those with special needs}}
		\begin{itemize}[<+->]
			\item[-] availability of videos of lectures/ hybrid format
			\item[-] use of inclusive language
			\item[-] inviting participation in class; see discussions, preparing questions prior to lecture
			\item[-] monitor learning success to be able to support where necessary; e.g. online questionnaires during semester
			\item[-] ensure easily approachable by students and actively approach students to learn about their needs
			\item[-] be available to discuss questions
		\end{itemize}
	\end{enumerate}
\end{frame}

content...
\end{comment}

\hypertarget{int}{}
\section*{Intention}
\begin{frame}{Intention: my teaching in the future}
\vspace{10mm}

\begin{itemize}[<+->]
	\item All levels
	\begin{itemize}
		\item[-] Macroeconomics
		\item[-] Monetary and Fiscal Policy
		\item[-] Environmental Economics 
	\end{itemize}
\vspace{2mm}
	\item Master's/ Ph.D. level
	\begin{itemize}
		\item[-] Reading groups and groups to discuss own research
		\item[-] Political Economy of Climate Change
		\item[-] \alert{Environmental Policies and Public Finance from a Macroeconomic Perspective}
	\end{itemize}
\end{itemize}

\vspace{16mm}
\hfill \hyperlink{end}{\tiny{$\rightarrow$ end}}
\end{frame}
\begin{frame}{Example Course}
	\vspace{-2mm}
\alert{\textbf{Environmental Policies and Public Finance from a Macroeconomic Perspective}}
	\pause
\begin{enumerate}[<+->]
	\item Natural sciences aspect of climate change \footnotesize{\citep{Hassler2016EnvironmentalMacroeconomics, Hsiang2018AnScience}} \normalsize
	\vspace{1mm}
	\item Macro models to study environmental policies \\ \footnotesize{\citep{Acemoglu2012TheChange, Golosov2014OptimalEquilibrium, Acemoglu2016TransitionTechnology, Fried2018ClimateAnalysis, Barrage2019OptimalPolicy}} \normalsize
		\vspace{1mm}
	\item Public finance \footnotesize{\citep{ Domeij2004OnTaxes, Conesa2009TaxingAll, Heathcote2017OptimalFramework}} \normalsize
		\vspace{1mm}
	\item Public finance and environmental policies
	\begin{itemize}
		\item[-] exogenous government funding condition \\ \footnotesize{\citep{LansBovenberg1994EnvironmentalTaxation, Goulder1995EnvironmentalGuide, Barrage2019OptimalPolicy}}
		\item[-] distributive effects of environmental policies \\ \footnotesize{\citep{Fried2018TheGenerations, Goulder2019IncomeGroups, Kotlikoff2021MakingWin}}
	\end{itemize}
	\vspace{1mm}
\item Classical fiscal policies as environmental policy instrument
\begin{itemize}
		\item[-] inequality \footnotesize{\citep{Jacobs2019RedistributionCurves, Dobkowitz2022, Douenne2022OptimalHouseholds}} \normalsize
		\item[-] endogenous growth \footnotesize{\citep{Dobkowitz2022b}}
\end{itemize}
\end{enumerate}
\end{frame}


\begin{frame}
	\hypertarget{end}{}
	\huge \textbf{Thank you!} \\
	\vspace{5mm}
	\normalsize
	{Happy to continue the discussion: \alert{sonja.dobkowitz@uni-bonn.de}}
\end{frame}


