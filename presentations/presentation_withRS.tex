\documentclass[11pt,aspectratio=169]{beamer}

%\mode<handout>
%{
	%	\usepackage{pgf}
	%	\usepackage{pgfpages}
	%	
	%	\pgfpagesdeclarelayout{4 on 1 boxed}
	%	{
		%		\edef\pgfpageoptionheight{\the\paperheight} 
		%		\edef\pgfpageoptionwidth{\the\paperwidth}
		%		\edef\pgfpageoptionborder{0pt}
		%	}
	%	{
		%		\pgfpagesphysicalpageoptions
		%		{%
			%			logical pages=4,%
			%			physical height=\pgfpageoptionheight,%
			%			physical width=\pgfpageoptionwidth%
			%		}
		%		\pgfpageslogicalpageoptions{1}
		%		{%
			%			border code=\pgfsetlinewidth{2pt}\pgfstroke,%
			%			border shrink=\pgfpageoptionborder,%
			%			resized width=.5\pgfphysicalwidth,%
			%			resized height=.5\pgfphysicalheight,%
			%			center=\pgfpoint{.25\pgfphysicalwidth}{.75\pgfphysicalheight}%
			%		}%
		%		\pgfpageslogicalpageoptions{2}
		%		{%
			%			border code=\pgfsetlinewidth{2pt}\pgfstroke,%
			%			border shrink=\pgfpageoptionborder,%
			%			resized width=.5\pgfphysicalwidth,%
			%			resized height=.5\pgfphysicalheight,%
			%			center=\pgfpoint{.75\pgfphysicalwidth}{.75\pgfphysicalheight}%
			%		}%
		%		\pgfpageslogicalpageoptions{3}
		%		{%
			%			border code=\pgfsetlinewidth{2pt}\pgfstroke,%
			%			border shrink=\pgfpageoptionborder,%
			%			resized width=.5\pgfphysicalwidth,%
			%			resized height=.5\pgfphysicalheight,%
			%			center=\pgfpoint{.25\pgfphysicalwidth}{.25\pgfphysicalheight}%
			%		}%
		%		\pgfpageslogicalpageoptions{4}
		%		{%
			%			border code=\pgfsetlinewidth{2pt}\pgfstroke,%
			%			border shrink=\pgfpageoptionborder,%
			%			resized width=.5\pgfphysicalwidth,%
			%			resized height=.5\pgfphysicalheight,%
			%			center=\pgfpoint{.75\pgfphysicalwidth}{.25\pgfphysicalheight}%
			%		}%
		%	}
	%	
	%	
	%	\pgfpagesuselayout{4 on 1 boxed}[a4paper, border shrink=5mm, landscape]
	%	\nofiles
	%}
\usefonttheme[onlymath]{serif}
\usetheme[outer/progressbar=foot,
%outer/numbering=none
]{metropolis}
\setbeamertemplate{caption}{\raggedright\insertcaption\par}
\setbeamercolor{frametitle}{bg={}, fg=black!80}
\definecolor{myorange}{rgb}{0.8500, 0.3250, 0.0980}
\setbeamercolor{alerted text}{bg={}, fg=myorange }
\setbeamercolor{block title}{bg=black!10, fg=black}
\setbeamercolor{block body}{bg=black!10, fg=black}
\setbeamercolor{block frame}{bg=black, fg=black}
\setbeamertemplate{blocks}[rounded]
\setbeamertemplate{blocks}[framed]
%\usecolortheme{seahorse}
\usepackage[utf8]{inputenc}
\usepackage[english]{babel}
%\usepackage[T1]{fontenc}
\newcommand{\tr}[1]{\textcolor{blue}{#1}}
\usepackage{amsmath}
\usepackage{amsfonts}
\usepackage{amssymb}
\usepackage{mathtools}
\usepackage{calc}
\usepackage{soul}
\setbeamercolor{headerCol}{fg=blue!30,bg=black!80}
\setbeamercolor{bodyCol}{fg=black}
\usepackage{graphicx}
\usepackage{xcolor}
\usepackage{appendix}
\usepackage{hyperref}
\usepackage{natbib}
\usepackage{comment}
\usepackage{setspace}
\renewcommand{\bibsection}{}
\bibliographystyle{apa} 
% have to run bibtex mydocument.aux after first run to generate bbl file. 
\usepackage{appendixnumberbeamer}
\usepackage{xcolor}
\usepackage{subcaption}
%table
\usepackage{makecell}
\usepackage{multirow}
\usepackage{bigdelim}

\newif\ifabbreviation
\pretocmd{\thebibliography}{\abbreviationfalse}{}{}
\AtBeginDocument{\abbreviationtrue}
\DeclareRobustCommand\acroauthor[2]{%
	\ifabbreviation #2\else #1 (\mbox{#2})\fi}

\usepackage[customcolors]{hf-tikz}
\definecolor{sonja}{cmyk}{1.5,0,0.9,0.3}
%\definecolor{blue}{cmyk}{0,1,0,0}
\hfsetfillcolor{black!10}
\hfsetbordercolor{black}

\usepackage{tikz}
\usetikzlibrary{tikzmark}
\usetikzlibrary{decorations.markings}
\usepackage{tikz-cd}
\usetikzlibrary{arrows,calc,fit}
\tikzset{mainbox/.style={draw=white, text=white, fill=gray, rectangle, rounded corners, thick, node distance=7em, text width=8em, text centered, minimum height=3.5em}}
\tikzset{dummybox/.style={draw=none, text=white , rectangle, rounded corners, thick, node distance=7em, text width=8em, text centered, minimum height=3.5em}}
\tikzset{box/.style={draw , rectangle, rounded corners, thick, node distance=7em, text width=8em, text centered, minimum height=3.5em}}
\tikzset{container/.style={draw, rectangle, dashed, inner sep=2em}}
\tikzset{line/.style={draw, very thick, -latex'}}
\tikzset{    pil/.style={
		->,
		thick,
		shorten <=2pt,
		shorten >=2pt,}}
\tikzstyle{vecArrow} = [thick, decoration={markings,mark=at position
	1 with {\arrow[semithick]{open triangle 60}}},
double distance=1.4pt, shorten >= 5.5pt,
preaction = {decorate},
postaction = {draw,line width=1.4pt, white,shorten >= 4.5pt}]
\usetikzlibrary{shapes}
\renewcommand{\figurename}{}

%TITLE
\author[Sonja Dobkowitz]{\small Sonja Dobkowitz\\ \footnotesize{University of Bonn%, RTG 2281 The Macroeconomics of Inequality}
}\\ }
%\institute[University of Bonn]{}
\title{Meeting Climate Targets: The Optimal Fiscal Policy Mix}

\newcommand{\ar}{$\Rightarrow$ \ }

%\addtobeamertemplate{navigation symbols}{}{%
%    \usebeamerfont{footline}%
%    \usebeamercolor[fg]{footline}%
%    \hspace{1em}%
%   \insertframenumber/\inserttotalframenumber
%}

%\institute{University of Bonn} 
\date{\small{ Presentation\\ December 18, 2022 }} 
%\subject{} 
\begin{document}

\tikzstyle{modus}=[rectangle,inner sep=5mm,align=center, draw]
\tikzstyle{dialog}=[diamond, align=center, draw]
\tikzstyle{sphere}=[circle, align=center, dotted, minimum size=3cm, draw]
\tikzstyle{circll}=[circle, align=center, minimum size=3cm, draw]
{\setbeamertemplate{footline}{}
	\begin{frame}
		\titlepage
	\end{frame}
}
%\addtocounter{framenumber}{-1}

% {\setbeamertemplate{footline}{}
	% \begin{frame}{Content}
		% \vspace{4mm}
		% \tableofcontents
		% \end{frame}
	% }
%\addtocounter{framenumber}{-1}


%---------------------------------------
%            Intro
%---------------------------------------

%\begin{frame}{Motivation}
%	
%	\begin{itemize}[<+-| alert@+>]
	%		\setbeamercolor{alerted text}{} %change the font color
	%		\setbeamerfont{alerted text}{}
	%		\item 
	%		\item meeting climate targets requires a limit on emissions  \citep{IPCC2022}
	%		\vspace{3mm}
	%		\item carbon taxes  direct  (i) demand towards emission-low alternatives\\ \hspace{23mm} \underline{and} (ii) research across sectors
	%		\vspace{3mm}
	%		\item labor income taxes affect the level of production 
	%		\vspace{3mm}
	%		\item \textbf{What is the optimal policy mix to meet the emission target?}
	%	\end{itemize}
%\end{frame}

	\begin{frame}{Motivation}
		
		\begin{itemize}[<+-| alert@+>]
			\setbeamercolor{alerted text}{} %change the font color
			\setbeamerfont{alerted text}{}
			%	\item we are facing  environmental limits 
			%	\vspace{3mm}
			\item Meeting climate targets requires a limit on emissions \citep{IPCC2022}
			\vspace{3mm}
			\item Carbon taxes  direct  (i) demand towards low-emission alternatives ...
			\vspace{3mm}
			\item ... \underline{and} (ii) research across sectors
			\vspace{2mm}
			\begin{itemize}
				\item[-] important when research subsidies are unavailable \small{\citep{Acemoglu2012TheChange}}	
				%			\item[-] if want to foster \textbf{green} research
				%			\ar higher carbon tax \ar % but reduces returns to labor %\ar
				%			costly in terms of output % reduces share of fossil energy 
				%			\item[-] if want to foster \textbf{fossil} research \ar smaller carbon tax \ar but too high emissions
			\end{itemize}
			\vspace{2mm}
			\item Labor income tax can be used to counter side effects of carbon tax 
			
			\vspace{3mm}
			\item \textbf{What is the optimal policy mix to meet the emission target?}
		\end{itemize}
	\end{frame}
	



\begin{frame}{This paper}
	\vspace{-2mm}
	\begin{itemize}
		\item<+-> Quantitative model of \alert{directed technical change}  building on \cite{Fried2018ClimateAnalysis}
		\vspace{2mm}
		\item<+->   The government   chooses the \alert{path of carbon and income taxes} to maximize welfare\vspace{2mm}
		\item<+-> An \alert{emission limit} constrains the government 
		\vspace{2mm}
		\item<+-> I consider 2 cases:
				\begin{enumerate}
			\item no research subsidies are available
			\item carbon tax revenues are used to finance research subsidies
				\end{enumerate}
		
	\end{itemize}
	\pause
	\begin{center}
		\begin{figure}
			\centering
			\caption{US net CO$_2$ emission limit in Gt}
			%			\vspace{2mm}
			\includegraphics[width=0.38\textwidth]{../codding_model/own_basedOnFried/optimalPol_010922_revision/figures/all_13Sept22_Tplus30/Emnet.png}
		\end{figure}
	\end{center}
\end{frame}


\begin{frame}{Preview of results}
	\vspace{-3mm}
	\pause
	\begin{itemize}[<+-| alert@+>]
		\setbeamercolor{alerted text}{} %change the font color
		\setbeamerfont{alerted text}{}
		%begin{minipage}[]{1\textwidth}
		%\begin{itemize}
		\item Before the net-zero emission limit: 
		\begin{itemize}
			\item[-] lower carbon tax to maintain some fossil research %\tr{give an example of knowledge spillovers here}
			\item[-] a tax on labor reduces emissions
		\end{itemize}
		\vspace{3mm}
		\item Under the net-zero emission limit: 
		\begin{itemize}
			\item[-]  higher carbon tax to boost green research
			\item[-]  a subsidy on labor stabilizes output
		\end{itemize}
		\vspace{3mm}
		\item A significant gap between first-best and optimal allocation calls for additional policy measures
		%; \tr{e.g. a higher marginal value of leisure}	%	\end{minipage}
\end{itemize}
\end{frame}

\begin{frame}{Contribution to the literature}
\begin{itemize}[<+->]
	\item \alert{Standard to have inelastic labor supply in environmental policy discussion}\\  \footnotesize{ \citep{Acemoglu2012TheChange, Golosov2014OptimalEquilibrium, Acemoglu2016TransitionTechnology, Fried2018ClimateAnalysis, Hart2019TheEconomists}}
	\\  \normalsize{\alert{\ar misses potential distortions on the labor market}}
	\vspace{2mm}
	\item \alert{This paper}: labor market distortions arise from (i) directed technical change and (ii) lack of research subsidies \footnotesize{\citep{Acemoglu2012TheChange}}
	\\ \normalsize{\alert{\ar role for labor income taxation}}
	\vspace{2mm}
	\item \alert{Labor income taxes as environmental policy instrument}
	\begin{itemize}
		\item[-]  fiscal policy generally passive \footnotesize{ \citep{ LansBovenberg1994EnvironmentalTaxation, Goulder1995EnvironmentalGuide, Barrage2019OptimalPolicy}}
		\item[-] arise as environmental policy tool due to income inequality \footnotesize{\citep{Jacobs2019RedistributionCurves, Dobkowitz2022, Douenne2022OptimalHouseholds}}
		\item[\ar] I add a novel motive for the use of labor income taxes within the optimal environmental policy
	\end{itemize}
	% weak double dividend literature (advantage to use env tac revenues to lower existing distortions; deviation of optimal env tax from pigou)
	%			\begin{itemize}
		%			%	\item[-] Research subsidies essential to implement first best.
		%			%	Otherwise, carbon tax higher to bolster green research; costly in terms of output
		%			%	\item[-]  In this case, labor income taxes may help get closer to first best
		%				\item[-] I add a richer, quantitative framework \ar new qualitative insights % increasing carbon tax, potentially fossil subsidy optimal
		%				\item[-]  insights on importance of additional measures
		%			\end{itemize}
	%	With knowledge spillovers \ar (i) increasing policy intervention, (ii) potentially advantageous to maintain some fossil research
	%	\vspace{2mm}
	%	% theirs is an analytical model; qualitative results; I look at a quantitative framework
	%	\item Quantitative framework builds on \cite{Fried2018ClimateAnalysis} %\ar new qualitative insights
	%	\begin{itemize}
		%		\item[-] I add a \alert{dynamic optimal policy analysis} under an exogenous \alert{dynamic emission target} %eventually declining to \alert{net-zero emissions}
		%		\item[-] \alert{elastic labor supply}
		%	\end{itemize}
\end{itemize}
\end{frame}
%\begin{frame}{Mechanism}
%\begin{itemize}
%	\item in theory: optimal carbon tax set so that emitters internalize social costs of emissions
%	\vspace{2mm}
%	\item but, without research subsidy, carbon tax also used to target the direction of research
%	\vspace{2mm}
%	\item \normalsize{causes distortions on the labor market:}\\ households do not correctly internalize the social costs of labor
%\end{itemize}
%%\item<+->  \alert{sizable effect of skill heterogeneity}: \\ with only one skill, the optimal income tax increases social welfare by 0.85\%
%
%\end{frame}
\begin{frame}{Outline}
\tableofcontents
\end{frame}

\section{Model}

\begin{frame}{Model}
\begin{figure}[h]
	%	\vspace{-4mm}
	\centering
	\begin{tikzpicture}[auto,scale=.7, transform shape]
		
		\node[circll] (A) at (-7,4)  {\textbf{{\hyperlink{prodmod}{Production}}}\\ \textbf{{and Research}}};
		\node[circll] (B) at (7,4) {\textbf{\hyperlink{backhh}{{Representative}}}\\ \textbf{\hyperlink{backhh}{{Household}}}};
		\node[circll] (D) at (0,9) {\textbf{Government}};
	\end{tikzpicture}
\end{figure}
\end{frame}

\addtocounter{framenumber}{-1}
\begin{frame}{Model}
\begin{figure}[h]
	%	\vspace{-4mm}
	\centering
	\begin{tikzpicture}[auto,scale=.7, transform shape]
		
		\node[circll] (A) at (-7,4)  {\textbf{{\hyperlink{prodmod}{Production}}}\\ \textbf{{and Research}}};
		\node[circll] (B) at (7,4) {\textbf{\alert{{Representative}}}\\ \textbf{\alert{{Household}}}};
		\node[circll] (D) at (0,9) {\textbf{Government}};
	\end{tikzpicture}
\end{figure}
\end{frame}


\addtocounter{framenumber}{-1}
\begin{frame}{Model}
\begin{figure}[h]
	\vspace{-4mm}
	\centering
	\begin{tikzpicture}[auto,scale=.7, transform shape]
		\node[circll] (A) at (-7,4) {\textbf{{\hyperlink{prodmod}{Production}}}\\ \textbf{{and Research}}};
		\node[circll] (B) at (7,4) {\textbf{\alert{{Representative}}}\\ \textbf{\alert{{Household}}}};
		\node[circll] (D) at (0,9) {\textbf{Government}}; 
		
		\node[draw=none] (B1) at (5,4.25) {};
		\node[draw=none] (B2) at (5,3.5) {};
		\node[draw=none] (BA1) at (-5,4.25) {};
		\node[draw=none] (BA2) at (-5,3.5) {};
		\node[draw=none] (D1) at (1.8,7.8) {};
		
		\node[draw=none] (B22) at (6.4,5.6) {};
		\node[draw=none] (D2) at (2.3,8.5) {};
		\node[draw=none] (B3) at (5.3,5.3) {};
		\node[draw=none] (D3) at (-1.5,7) {};
		\node[draw=none] (A1) at (-4.2,4.6) {};
		
		
		
		\draw [->] (B1) to node[pos=0.75, swap]{Workers and scientists} (BA1);
		\draw [->] (BA2) to node[pos=0.75, swap]{Final good} (B2);
	\end{tikzpicture}
	
\end{figure}
\end{frame}
\addtocounter{framenumber}{-1}
\begin{frame}{Model}
\begin{figure}[h]
	\vspace{-4mm}
	\centering
	\begin{tikzpicture}[auto,scale=.7, transform shape]
		\node[circll] (A) at (-7,4) {\textbf{{\hyperlink{prodmod}{Production}}}\\ \textbf{{and Research}}};
		\node[circll] (B) at (7,4) {\textbf{\alert{{Representative}}}\\ \textbf{\alert{{Household}}}};
		\node[circll] (D) at (0,9) {\textbf{{Government}} }; 
		\node[draw=none] (B1) at (5,4.25) {};
		\node[draw=none] (B2) at (5,3.5) {};
		\node[draw=none] (BA1) at (-5,4.25) {};
		\node[draw=none] (BA2) at (-5,3.5) {};
		\node[draw=none] (D1) at (1.8,7.8) {};
		
		\node[draw=none] (B22) at (6.4,5.6) {};
		\node[draw=none] (D2) at (2.3,8.5) {};
		\node[draw=none] (B3) at (5.3,5.3) {};
		\node[draw=none] (D3) at (-1.5,7) {};
		\node[draw=none] (A1) at (-4.2,4.6) {};
		\node[draw=none] (D4) at (-2,8.3) {};
		\node[draw=none] (A4) at (-6,5.4) {};
		\draw [->] (B22) to node[pos=0.65, swap]{{Tax on labor, $\pmb{\tau_{\iota}}$}} (D2);
		%	\draw [->] (B22) to node[pos=0.45, swap]{\alert{$\pmb{\tau_{\iota}}>0$:  labor supply $\downarrow$ \ar emissions $\downarrow$}} (D2);
		%	\draw [->] (B22) to node[pos=0.3, swap]{\alert{$\pmb{\tau_{\iota}}<0$:    labor supply $\uparrow$ \ar output $\uparrow$}} (D2);
		\draw [->] (D1) to node[pos=0.5, swap]{Transfers} (B3);
		
		\draw [->] (B1) to node[pos=0.75, swap]{Workers and scientists} (BA1);
		\draw [->] (BA2) to node[pos=0.75, swap]{Final good} (B2);
		%	\draw [->] (A4) to node[pos=0.5, swap]{\alert{Tax on carbon, $\pmb{\tau_F}$}}   (D4);
	\end{tikzpicture}
	
\end{figure}
\end{frame}

\addtocounter{framenumber}{-1}
\begin{frame}{Model}
\begin{figure}[h]
	\vspace{-4mm}
	\centering
	\begin{tikzpicture}[auto,scale=.7, transform shape]
		\node[circll] (A) at (-7,4) {\textbf{{\hyperlink{prodmod}{Production}}}\\ \textbf{{and Research}}};
		\node[circll] (B) at (7,4) {\textbf{\hyperlink{backhh}{{Representative}}}\\ \textbf{\hyperlink{backhh}{{Household}}}};
		\node[circll] (D) at (0,9) {\textbf{{Government}} }; 
		\node[draw=none] (B1) at (5,4.25) {};
		\node[draw=none] (B2) at (5,3.5) {};
		\node[draw=none] (BA1) at (-5,4.25) {};
		\node[draw=none] (BA2) at (-5,3.5) {};
		\node[draw=none] (D1) at (1.8,7.8) {};
		
		\node[draw=none] (B22) at (6.4,5.6) {};
		\node[draw=none] (D2) at (2.3,8.5) {};
		\node[draw=none] (B3) at (5.3,5.3) {};
		\node[draw=none] (D3) at (-1.5,7) {};
		\node[draw=none] (A1) at (-4.2,4.6) {};
		\node[draw=none] (D4) at (-2,8.3) {};
		\node[draw=none] (A4) at (-6,5.4) {};
		\draw [->] (B22) to node[pos=0.65, swap]{\alert{Tax on labor, $\pmb{\tau_{\iota}}$:}} (D2);
		\draw [->] (B22) to node[pos=0.45, swap]{\alert{$\pmb{\tau_{\iota}}>0$:  labor supply $\downarrow$ \ar emissions $\downarrow$}} (D2);
		%	\draw [->] (B22) to node[pos=0.3, swap]{\alert{$\pmb{\tau_{\iota}}<0$:    labor supply $\uparrow$ \ar output $\uparrow$}} (D2);
		\draw [->] (D1) to node[pos=0.5, swap]{Transfers} (B3);
		
		\draw [->] (B1) to node[pos=0.75, swap]{Workers and scientists} (BA1);
		\draw [->] (BA2) to node[pos=0.75, swap]{Final good} (B2);
		%	\draw [->] (A4) to node[pos=0.5, swap]{\alert{Tax on carbon, $\pmb{\tau_F}$}}   (D4);
	\end{tikzpicture}
	
\end{figure}
\end{frame}

\addtocounter{framenumber}{-1}
\begin{frame}{Model}
\begin{figure}[h]
	\vspace{-4mm}
	\centering
	\begin{tikzpicture}[auto,scale=.7, transform shape]
		\node[circll] (A) at (-7,4) {\textbf{{\hyperlink{prodmod}{Production}}}\\ \textbf{{and Research}}};
		\node[circll] (B) at (7,4) {\textbf{{{Representative}}}\\ \textbf{{{Household}}}};
		\node[circll] (D) at (0,9) {\textbf{{Government}} }; 
		\node[draw=none] (B1) at (5,4.25) {};
		\node[draw=none] (B2) at (5,3.5) {};
		\node[draw=none] (BA1) at (-5,4.25) {};
		\node[draw=none] (BA2) at (-5,3.5) {};
		\node[draw=none] (D1) at (1.8,7.8) {};
		
		\node[draw=none] (B22) at (6.4,5.6) {};
		\node[draw=none] (D2) at (2.3,8.5) {};
		\node[draw=none] (B3) at (5.3,5.3) {};
		\node[draw=none] (D3) at (-1.5,7) {};
		\node[draw=none] (A1) at (-4.2,4.6) {};
		\node[draw=none] (D4) at (-2,8.3) {};
		\node[draw=none] (A4) at (-6,5.4) {};
		\draw [->] (B22) to node[pos=0.65, swap]{\alert{Tax on labor, $\pmb{\tau_{\iota}}$:}} (D2);
		\draw [->] (B22) to node[pos=0.45, swap]{\alert{$\pmb{\tau_{\iota}}>0$:  labor supply $\downarrow$ \ar emissions $\downarrow$}} (D2);
		\draw [->] (B22) to node[pos=0.3, swap]{\alert{$\pmb{\tau_{\iota}}<0$:    labor supply $\uparrow$ \ar output $\uparrow$}} (D2);
		\draw [->] (D1) to node[pos=0.5, swap]{Transfers} (B3);
		
		\draw [->] (B1) to node[pos=0.75, swap]{Workers and scientists} (BA1);
		\draw [->] (BA2) to node[pos=0.75, swap]{Final good} (B2);
		%	\draw [->] (A4) to node[pos=0.5, swap]{\alert{Tax on carbon, $\pmb{\tau_F}$}}   (D4);
	\end{tikzpicture}
\end{figure}
\end{frame}


\begin{frame}{Representative household}
\hypertarget{backhh}{}
%\text{\textbf{Householdrt}}
\vspace{2mm}
\begin{minipage}[t!]{1\textwidth}
	\begin{align*}
		%	\tikzmarkin{first0}(1.5,2.7)(-1.2,-2.5)
		%	\underset{c_{s,i},c_{n,i}, l_i}{\max} \ \hspace{2mm} U(c_{s,i}, c_{n,i}, l_i; h_n)= 
		\max_{C_t, H_{t}, S_{t}} \log(C_t)-\chi\frac{H_{t}^{1+\sigma}}{1+\sigma}-\chi_s\frac{S_{t}^{1+\sigma}}{1+\sigma}
		\\
		\vspace{4mm}
		\\
		\text{s.t.}\ C_t=(\alert{\pmb{1-\tau_{\iota t}}})\left(w_{t}H_{t}+w_{st}S_t\right)+T_t%+Gov_t
		%\\
		%\hspace{2mm}\ H_{t}\leq \bar{H}; \hspace{4mm} S_{t}\leq \bar{H}
		%	\tikzmarkend{first0}
	\end{align*}
\end{minipage}

\small
\vspace{4mm}
\hspace{-8mm}
\begin{minipage}[t!]{0.26\textwidth}
	\vspace{7mm}
	\begin{itemize}
		\item[] $C_{t}$: consumption\vspace{-2mm}
		\item[] $H_{t}$: hours workers\vspace{-2mm}
		\item[] $S_{t}$: hours scientists\vspace{-2mm}
	\end{itemize}
\end{minipage}
\begin{minipage}[t!]{0.37\textwidth}
	\vspace{8mm}
	\begin{itemize}
		\item[] $w_{t}, w_{st}$: wages  \vspace{-2mm}
		\item[] $\tau_{\iota t}$: marginal income tax rate 
		\vspace{-2mm}	
		\item[] $T_{t}$: government transfers
		%		\vspace{-2mm}	
		%		\item[]%	$Gov_{t}$: government transfers
	\end{itemize}
\end{minipage}
\begin{minipage}[t!]{0.39\textwidth}
	\vspace{8mm}
	\begin{itemize}
		\item[] $\sigma$: curvature disutility of labor  \vspace{-2mm}
		\item[] $\chi$: disutility of work
		\vspace{-2mm}	
		\item[] $\chi_s$: disutility of research
		%		\vspace{-2mm}	
		%		\item[]%	$Gov_{t}$: government transfers
	\end{itemize}
\end{minipage}

\vspace{12mm}
\hfill	\hyperlink{labsup}{\tiny{$\rightarrow$ labor supply}}
\hypertarget{hhopt}{}
\end{frame}

%
%\begin{frame}{Model}
%	\begin{figure}[h]
%		\vspace{-4mm}
%		\centering
%		\begin{tikzpicture}[auto,scale=.7, transform shape]
	%			\node[circll] (A) at (-7,4) {\textbf{{\alert{Production}}}\\ \alert{\textbf{{and Research}}}};
	%			\node[circll] (B) at (7,4) {\textbf{\hyperlink{backhh}{{Representative}}}\\ \textbf{\hyperlink{backhh}{{Household}}}};
	%			\node[circll] (D) at (0,9) {\textbf{{Government}} }; 
	%			\node[draw=none] (B1) at (5,4.25) {};
	%			\node[draw=none] (B2) at (5,3.5) {};
	%			\node[draw=none] (BA1) at (-5,4.25) {};
	%			\node[draw=none] (BA2) at (-5,3.5) {};
	%			\node[draw=none] (D1) at (1.8,7.8) {};
	%			
	%			\node[draw=none] (B22) at (6.4,5.6) {};
	%			\node[draw=none] (D2) at (2.3,8.5) {};
	%			\node[draw=none] (B3) at (5.3,5.3) {};
	%			\node[draw=none] (D3) at (-1.5,7) {};
	%			\node[draw=none] (A1) at (-4.2,4.6) {};
	%			\node[draw=none] (D4) at (-2,8.3) {};
	%			\node[draw=none] (A4) at (-6,5.4) {};
	%			%			\draw [->] (B22) to node[pos=0.5, swap]{{Tax on labor, $\pmb{\tau_{\iota}}$}} (D2);
	%			%			\draw [->] (D1) to node[pos=0.5, swap]{Transfers} (B3);
	%			
	%%			\draw [->] (B1) to node[pos=0.75, swap]{Workers and scientists} (BA1);
	%%			\draw [->] (BA2) to node[pos=0.75, swap]{Final good} (B2);
	%%			\draw [->] (A4) to node[pos=0.5, swap]{\alert{Tax on carbon, $\pmb{\tau_F}$}}   (D4);
	%		\end{tikzpicture}
%		
%	\end{figure}
%\end{frame}
%\addtocounter{framenumber}{-1}
\begin{frame}{Model}
\begin{figure}[h]
	\vspace{-4mm}
	\centering
	\begin{tikzpicture}[auto,scale=.7, transform shape]
		\node[circll] (A) at (-7,4) {\textbf{{\alert{Production}}}\\ \alert{\textbf{{and Research}}}};
		\node[circll] (B) at (7,4) {\textbf{\hyperlink{backhh}{{Representative}}}\\ \textbf{\hyperlink{backhh}{{Household}}}};
		\node[circll] (D) at (0,9) {\textbf{{Government}} }; 
		\node[draw=none] (B1) at (5,4.25) {};
		\node[draw=none] (B2) at (5,3.5) {};
		\node[draw=none] (BA1) at (-5,4.25) {};
		\node[draw=none] (BA2) at (-5,3.5) {};
		\node[draw=none] (D1) at (1.8,7.8) {};
		
		\node[draw=none] (B22) at (6.4,5.6) {};
		\node[draw=none] (D2) at (2.3,8.5) {};
		\node[draw=none] (B3) at (5.3,5.3) {};
		\node[draw=none] (D3) at (-1.5,7) {};
		\node[draw=none] (A1) at (-4.2,4.6) {};
		\node[draw=none] (D4) at (-2,8.3) {};
		\node[draw=none] (A4) at (-6,5.4) {};
		%			\draw [->] (B22) to node[pos=0.5, swap]{{Tax on labor, $\pmb{\tau_{\iota}}$}} (D2);
		%			\draw [->] (D1) to node[pos=0.5, swap]{Transfers} (B3);
		
		\draw [->] (B1) to node[pos=0.75, swap]{Workers and scientists} (BA1);
		\draw [->] (BA2) to node[pos=0.75, swap]{Final good} (B2);
		%\draw [->] (A4) to node[pos=0.5, swap]{\alert{Tax on carbon, $\pmb{\tau_F}$}}   (D4);
	\end{tikzpicture}
	
\end{figure}
\end{frame}


\addtocounter{framenumber}{-1}
\begin{frame}{Model}
\begin{figure}[h]
	\vspace{-4mm}
	\centering
	\begin{tikzpicture}[auto,scale=.7, transform shape]
		\node[circll] (A) at (-7,4) {\textbf{{\hyperlink{prodmod}{Production}}}\\ \textbf{{and Research}}};
		\node[circll] (B) at (7,4) {\textbf{\hyperlink{backhh}{{Representative}}}\\ \textbf{\hyperlink{backhh}{{Household}}}};
		\node[circll] (D) at (0,9) {\textbf{{Government}} }; 
		\node[draw=none] (B1) at (5,4.25) {};
		\node[draw=none] (B2) at (5,3.5) {};
		\node[draw=none] (BA1) at (-5,4.25) {};
		\node[draw=none] (BA2) at (-5,3.5) {};
		\node[draw=none] (D1) at (1.8,7.8) {};
		
		\node[draw=none] (B22) at (6.4,5.6) {};
		\node[draw=none] (D2) at (2.3,8.5) {};
		\node[draw=none] (B3) at (5.3,5.3) {};
		\node[draw=none] (D3) at (-1.5,7) {};
		\node[draw=none] (A1) at (-4.2,4.6) {};
		\node[draw=none] (D4) at (-2,8.3) {};
		\node[draw=none] (A4) at (-6,5.4) {};
		%			\draw [->] (B22) to node[pos=0.5, swap]{{Tax on labor, $\pmb{\tau_{\iota}}$}} (D2);
		%			\draw [->] (D1) to node[pos=0.5, swap]{Transfers} (B3);
		
		\draw [->] (B1) to node[pos=0.75, swap]{Workers and scientists} (BA1);
		\draw [->] (BA2) to node[pos=0.75, swap]{Final good} (B2);
		\draw [->] (A4) to node[pos=0.5, swap]{\alert{Tax on carbon, $\pmb{\tau_F}$}}   (D4);
	\end{tikzpicture}
	
\end{figure}
\end{frame}

\begin{comment}
\begin{frame}{Production}
	\begin{figure}[h]
		\vspace{-10mm}
		\centering
		\begin{tikzpicture}[auto,scale=.7, transform shape]
			\node[circll] (A) at (0,17) {\textbf{Final}\textbf{ Good}}; 
			\node[circll] (B) at (-6,14) {\textbf{Energy}};
			\node[circll] (C) at (5,12) {\textbf{{Non-energy}}};
			\node[circll] (D) at (-10,12) {\textbf{{Fossil}}};
			\node[circll] (E) at (-2,12) {\textbf{{Green}}};
			\node[sphere] (Ems) at (-11,16) {\textbf{Emissions}};
			
			%		\node[circllsmall] (CM) at (4.6,8) {\textbf{{Machines}}};
			%		\node[circllsmall] (CL) at (7.4,8) {\textbf{{Labor}}};			\node[circllsmall] (EM) at (-3.4,8) {\textbf{{Machines}}};
			%	    \node[circllsmall] (EL) at (-0.6,8) {\textbf{{Labor}}};
			%	    \node[circllsmall] (DM) at (-11.4,8) {\textbf{{Machines}}};
			%	    \node[circllsmall] (DL) at (-8.6,8) {\textbf{{Labor}}};
			
			
			\draw [->] (B) to node[pos=0.5, swap]{} (A);
			\draw [->] (C) to node[pos=0.5, swap]{} (A);
			
			\draw [->] (E) to node[pos=0.5, swap]{} (B);
			\draw [->] (D) to node[pos=0.5, swap]{} (B);
			\draw [->] (D) to node[pos=0.5, swap]{} (Ems);
		\end{tikzpicture}
		
	\end{figure}
\end{frame}
content...
\end{comment}

\begin{frame}{Production: final and energy good}
\vspace{-10mm}
\hypertarget{prodmod}{}
\begin{align*}
	%		\tikzmarkin{first}(1.3,1.2)(-1,-0.8)
	\text{Final good}\hspace{4mm}&Y_t =\left(\delta_y^{\frac{1}{\varepsilon_y}}E_t^\frac{\varepsilon_y-1}{\varepsilon_y}+(1-\delta_y)^{\frac{1}{\varepsilon_y}}N_t^\frac{\varepsilon_y-1}{\varepsilon_y}\right)^\frac{\varepsilon_y}{\varepsilon_y-1} \\
	\ \\
	\text{Energy}\hspace{4mm}&E_t =\left({F}_t^\frac{\varepsilon_e-1}{\varepsilon_e}+G_t^\frac{\varepsilon_e-1}{\varepsilon_e}\right)^\frac{\varepsilon_e}{\varepsilon_e-1}\\
	\ \\
	\text{Demand energy producers}\hspace{4mm}&\frac{F_t}{G_t} = \left(\frac{p_{Gt}}{p_{Ft}+\alert{\pmb{\tau_{Ft}}}}\right)^{\varepsilon_e}
	%	\tikzmarkend{third}
\end{align*}

\small
\vspace{4mm}
\hspace{-4mm}
\begin{minipage}[t!]{0.24\textwidth}
	\vspace{0mm}
	\begin{itemize}	
		\item[]$F_t$: fossil energy
		\vspace{-2mm}	
		\item[]$G_t$: green energy
		\vspace{-2mm}	
		\item[]$N_t$: non-energy
	\end{itemize}
\end{minipage}
\begin{minipage}[t!]{0.24\textwidth}
	\vspace{0mm}
	\begin{itemize}
		\item[] $p_{Gt}$: price green  \vspace{-2mm}
		\item[] $p_{Ft}$: price fossil
		\vspace{-2mm}	
		\item[] $\tau_{Ft}$: carbon tax
	\end{itemize}
\end{minipage}
\begin{minipage}[t!]{0.47\textwidth}
	\vspace{0mm}
	\begin{itemize}
		\item[] $\delta_{y}$: weight on energy\vspace{-2mm}
		\item[] $\varepsilon_y$: elasticity of substitution $E_t$ and $N_t$ \vspace{-2mm}
		\item[] $\varepsilon_e$: elasticity of substitution $F_t$ and $G_t$
	\end{itemize}
\end{minipage}
\end{frame}

%\addtocounter{framenumber}{-1}
\begin{frame}{Production: intermediate goods $J\in \{N,F,G\}$ }
\vspace{0mm}
%Competitive producers
\begin{align*}
	\underset{\{x_{Jit}\}_{i=0}^1, L_{Jt}}{\max}\ & p_{Jt}J_t-w_{t}L_{Jt}-\int_{0}^{1}p_{xJit}x_{Jit}di \\ \ \\
	\text{s.t.}\ & J_{t}=L_{Jt}^{1-\alpha_J}\int_{0}^{1}A^{1-\alpha_J}_{Jit}x_{Jit}^{\alpha_J}di
\end{align*}

\small
\vspace{10mm}
\hspace{-4mm}
\begin{minipage}[t!]{0.3\textwidth}
	\vspace{0mm}
	\begin{itemize}	
		\item[]$L_{Jt}\ $: labor 
		\vspace{-2mm}	
		\item[]$x_{Jit}\ $: machines 
		\vspace{-2mm}	
		\item[]$p_{xJit}$: price machine 
	\end{itemize}
\end{minipage}
\begin{minipage}[t!]{0.5\textwidth}
	\vspace{0mm}
	\begin{itemize}
		\item[] $A_{Jit}$: productivity machine $i$ sector $J$ \vspace{-2mm}
		\item[] $J$\ \  : sector N(on-energy),F(ossil),G(reen)
		\vspace{-2mm}	
		\item[] $\alpha_J$\ : capital share 
	\end{itemize}
\end{minipage}
\end{frame}

\begin{frame}{Production: machines and innovation}
\vspace{-8mm}
\begin{align*}
	%	\tikzmarkin{sixth}(6.3,4)(-2.7,-3.8)
	\underset{p_{xJit}, s_{Jit}}{\max}\ & p_{xJit}(1+\zeta_{Jt})x_{Jit}-x_{Jit}-w_{st}s_{Jit}
	\\ 
	\text{s.t.}\ &(1)\ x_{Jit}=\left(\frac{\alpha_Jp_{Jt}}{p_{xJit}}\right)^{\frac{1}{1-\alpha_J}}L_{Jt}A_{Jit}\\ \ \\ %x_{ijt}= \left(\frac{p_{ft}(1-\tau_{jt})\alpha_j}{p_{xijt}}\right)^\frac{1}{1-\alpha_j}A_{ijt}L_{jt}\\
	& (2)\ A_{Jit}=f_{Jt}(s_{Jit})%A_{Jt-1}\left(1+\gamma\left(\frac{s_{Jit}}{\rho_J}\right)^\eta\left(\frac{A_{t-1}}{A_{Jt-1}}\right)^\phi\right)
	%	\tikzmarkend{sixth}
\end{align*}

\small
\vspace{4mm}
\hspace{-4mm}	\begin{minipage}[t!]{0.32\textwidth}
	\vspace{0mm}
	\begin{itemize}
		\item[-] monopolistic competition 
		\vspace{-4mm}
		\item[-] one-period patents
	\end{itemize}	
\end{minipage}
\begin{minipage}[t!]{0.5\textwidth}
	\vspace{0mm}
	\begin{itemize}	
		\item[]$\zeta_{Jt}$: subsidy
		\vspace{-2mm}	
		\item[]$s_{Jit}$: scientists
		\vspace{-2mm}	
		\item[]$A_{Jit}$: productivity machine $i$ sector $J$
	\end{itemize}
\end{minipage}
%\begin{minipage}[t!]{0.32\textwidth}
%	\vspace{0mm}
%	\begin{itemize}
	%		\item[] $\eta$: returns to research  \vspace{-2mm}
	%		\item[] $\rho_j$: research processes
	%		\vspace{-2mm}	
	%		\item[] $\gamma$: productivity scientists
	%	\end{itemize}
%\end{minipage}
\end{frame}

\begin{comment}
\addtocounter{framenumber}{-1}
\begin{frame}{Production: machines and innovation}
	\vspace{-2mm}
	Monopolistically competitive machine producer $i$ in sector $J$
	\begin{align*}
		%	\tikzmarkin{sixth}(6.3,4)(-2.7,-3.8)
		\underset{p_{xJit}, s_{Jit}}{\max}\ & p_{xJit}(1+\zeta_{Jt})x_{Jit}-x_{Jit}-w_{st}s_{Jit}
		\\ 
		s.t.\ &(1)\ x_{Jit}=\left(\alpha_Jp_{Jt}\right)^{\frac{1}{1-\alpha_J}}L_{Jt}A_{Jit}\\ \ \\ %x_{ijt}= \left(\frac{p_{ft}(1-\tau_{jt})\alpha_j}{p_{xijt}}\right)^\frac{1}{1-\alpha_j}A_{ijt}L_{jt}\\
		& (2)\ A_{Jit}=A_{Jt-1}\left(1+\gamma\left(\frac{s_{Jit}}{\rho_J}\right)^\eta\left(\frac{A_{t-1}}{A_{Jt-1}}\right)^\phi\right)
		%	\tikzmarkend{sixth}
	\end{align*}
	
	\small
	\vspace{4mm}
	\hspace{-4mm}
	\begin{minipage}[t!]{0.4\textwidth}
		\vspace{0mm}
		\begin{itemize}	
			\item[]$\zeta_{Jt}$: subsidy
			\vspace{-2mm}	
			\item[]$s_Ji$: scientists
			\vspace{-2mm}	
			\item[]$\phi$: knowledge spillovers
		\end{itemize}
	\end{minipage}
	\begin{minipage}[t!]{0.5\textwidth}
		\vspace{0mm}
		\begin{itemize}
			\item[] $\eta$: returns to research  \vspace{-2mm}
			\item[] $\rho_j$: research processes
			\vspace{-2mm}	
			\item[] $\gamma$: productivity scientists
		\end{itemize}
	\end{minipage}
\end{frame}

content...
\end{comment}

%\begin{frame}{In more detail}
%\begin{enumerate}
%	\item<1-> effect of carbon tax on the allocation of scientists
%	\item<2-> innovation
%\end{enumerate}
%\end{frame}

\begin{frame}{Effect of carbon tax on the allocation of scientists}
\vspace{2mm}
In equilibrium: \large
\begin{align*}
	\overbrace{{\psi_F} p_F{F}\frac{\partial A_{F}}{\partial s_{F}}}^{\text{wage fossil scientists}}=\overbrace{{\psi_G} p_G{G}\frac{\partial A_{G}}{\partial s_{G}}}^{\text{wage green scientists}}
\end{align*}
\normalsize
\begin{itemize}
	\item[] % carbon tax lowers returns to fossil research and raises returns to green research
	\item[] % in equilibrium: scientists transition from fossil to green sector \small{(decreasing returns to research)}
\end{itemize}
\small
\vspace{4mm}
\hspace{-2mm}
\begin{minipage}[t!]{0.4\textwidth}
	\vspace{0mm}
	\begin{itemize}
		\item[] $p_JJ$: revenues sector $J$
		\vspace{-2mm}
		\item[] $\psi_J$ : sector-specific constant
	\end{itemize}
\end{minipage}
\vspace{-5mm}
\begin{minipage}[t!]{0.5\textwidth}
	\vspace{0mm}
	\begin{itemize}	
		\item[] $A_J$: productivity sector $J$
		\vspace{-2mm}			
		\item[] $s_J$ : scientists sector $J$
	\end{itemize}
\end{minipage}
\end{frame}

\addtocounter{framenumber}{-1}

\begin{frame}{Effect of carbon tax on the allocation of scientists}
\vspace{2mm}
%In equilibrium: \large
\begin{align*}
	\overbrace{{\psi_F} \underbrace{p_F{F}}_{\alert{\pmb{\tau_F\uparrow\Rightarrow\downarrow}}}\frac{\partial A_{F}}{\partial s_{F}}}^{\text{wage fossil scientists}}\alert{\pmb{<}}\overbrace{{\psi_G} \underbrace{p_G{G}}_{\alert{\pmb{\tau_F\uparrow\Rightarrow\uparrow}}}\frac{\partial A_{G}}{\partial s_{G}}}^{\text{wage green scientists}}
\end{align*}
\normalsize
\begin{itemize}
	\item carbon tax lowers wages to fossil research and raises wages to green research
	\vspace{2mm}
	\item[]% in equilibrium: scientists transition from fossil to green sector \small{(decreasing returns to research)}
\end{itemize}
\small
\vspace{4mm}
\hspace{-2mm}
\begin{minipage}[t!]{0.4\textwidth}
	\vspace{0mm}
	\begin{itemize}
		\item[] $p_JJ$: revenues sector $J$
		\vspace{-2mm}
		\item[] $\psi_J$ : sector-specific constant
	\end{itemize}
\end{minipage}
\vspace{-5mm}
\begin{minipage}[t!]{0.5\textwidth}
	\vspace{0mm}
	\begin{itemize}	
		\item[] $A_J$: productivity sector $J$
		\vspace{-2mm}			
		\item[] $s_J$ : scientists sector $J$
	\end{itemize}
\end{minipage}
\end{frame}

\addtocounter{framenumber}{-1}

\begin{frame}{Effect of carbon tax on the allocation of scientists}
\vspace{2mm}
In equilibrium: \large
\begin{align*}
	\overbrace{{\psi_F} \underbrace{p_F{F}}_{\tau_F\uparrow\Rightarrow\downarrow}\underbrace{\frac{\partial A_{F}}{\partial s_{F}}}_{\alert{\pmb{s_F\downarrow\Rightarrow\uparrow}}}}^{\text{wage fossil scientists}}\alert{\pmb{=}}\overbrace{{\psi_G} \underbrace{p_G{G}}_{\tau_F\uparrow\Rightarrow\uparrow}\underbrace{\frac{\partial A_{G}}{\partial s_{G}}}_{\alert{\pmb{s_G\uparrow\Rightarrow\downarrow}}}}^{\text{wage green scientists}}
\end{align*}
\normalsize
\begin{itemize}
	\item carbon tax lowers wages to fossil research and raises wages to green research
	\vspace{2mm}
	\item in equilibrium: scientists transition from fossil to green sector \small{(decreasing returns to research)}
\end{itemize}
\small
\vspace{4mm}
\hspace{-2mm}
\begin{minipage}[t!]{0.4\textwidth}
	\vspace{0mm}
	\begin{itemize}
		\item[] $p_JJ$: revenues sector $J$
		\vspace{-2mm}
		\item[] $\psi_J$ : sector-specific constant
	\end{itemize}
\end{minipage}
\vspace{-5mm}
\begin{minipage}[t!]{0.5\textwidth}
	\vspace{0mm}
	\begin{itemize}	
		\item[] $A_J$: productivity sector $J$
		\vspace{-2mm}			
		\item[] $s_J$ : scientists sector $J$
	\end{itemize}
\end{minipage}
\end{frame}


%\begin{frame}{2nd effect of carbon tax: direction of research}
%\vspace{-2mm}
%\begin{align*}
%	w_
%	w_{st}&=\underbrace{\left(\frac{\alert{p_{Jt}}\alpha_J}{p_{xJit}}\right)^{\frac{1}{1-\alpha_J}}\alert{L_{Jt}}}_{\frac{\partial x_{Jit}}{\partial A_{Jit}}}\gamma \eta A_{Jt-1} \left(\frac{A_{t-1}}{A_{Jt-1}}\right)^{\phi}\left(\frac{s_{Jit}}{\rho_J}\right)^{\eta-1}
%	%&	\frac{\eta \gamma \left(\frac{A_{t-1}}{A_{Jt-1}}\right)^\phi(1-\alpha_J)\alpha_Js_{Jt}^{\eta-1}p_{Jt}J_t}{\rho_J^\eta}	
%\end{align*}
%\begin{itemize}
%	\item \alert{carbon tax lowers demand for fossil machines and returns to fossil research}
%\end{itemize}
%\small
%\vspace{7mm}
%\hspace{-4mm}
%\begin{minipage}[t!]{0.3\textwidth}
%	\vspace{0mm}
%	\begin{itemize}
%		\item[] $\eta$: returns to research  \vspace{-2mm}
%		\item[] $\rho_j$: research processes
%	\end{itemize}
%\end{minipage}
%\vspace{-5mm}
%\begin{minipage}[t!]{0.5\textwidth}
%	\vspace{0mm}
%	\begin{itemize}	
%		\item[] $\gamma$: productivity scientists
%		\vspace{-2mm}	
%		\item[] $\phi$: knowledge spillovers, $\phi\geq0$
%	\end{itemize}
%\end{minipage}
%\end{frame}

\begin{frame}{Innovation}
\pause
\vspace{-10mm}
\vspace{4mm}
% talk about productivity of research bcs it determines optimal allocation of research
\large
\begin{align*}
	A_{Jit}=\alert{A_{Jt-1}}\left(1+\gamma\left(\frac{s_{Jit}}{\rho_J}\right)^\eta\left(\frac{A_{t-1}}{A_{Jt-1}}\right)^\phi\right)
\end{align*}
\normalsize
\begin{enumerate}
	\item \alert{path dependency: sector-specific knowledge renders scientists more productive} % \footnotesize{one-period patents} % due to patent structure not taken into account by machine producers when demanding research. 
	%		\begin{itemize}
		%	\item<+-> scientists in most advanced, fossil sector are more productive
		%			\item<+-> shift to green research early on to make green research more productive tomorrow
		%		\end{itemize}
	\item[] \  % knowledge from other sectors increases productivity of scientists
	\item[] \ % decreasing returns to research, $\eta<1$
\end{enumerate}
\small
\vspace{4mm}
\hspace{-2mm}
\begin{minipage}[t!]{0.43\textwidth}
	\vspace{0mm}
	\begin{itemize}
		\item[] $A_{Jt}$: sector-specific technology
		\vspace{-2mm}		
		\item[] $A_t$: aggregate technology
		\vspace{-2mm}
		\item[] $\gamma$ : productivity of scientists
	\end{itemize}
\end{minipage}
\vspace{-5mm}
\begin{minipage}[t!]{0.55\textwidth}
	\vspace{0mm}
	\begin{itemize}	
		\item[] $\rho_J$: number of research processes in sector $J$
		\vspace{-2mm}			
		\item[] $\eta$ : returns to research
		\vspace{-2mm}			
		\item[] $\phi$ : relative importance knowledge spillovers
	\end{itemize}
\end{minipage}
\end{frame}

\addtocounter{framenumber}{-1}
\begin{frame}{Innovation}
\large
\begin{align*}
	A_{Jit}={A_{Jt-1}}\left(1+\gamma\alert{\left(\frac{s_{Jit}}{\rho_J}\right)^{\eta}}{\left(\frac{A_{t-1}}{A_{Jt-1}}\right)^\phi}\right)
\end{align*}
\normalsize
\begin{enumerate}
	\item path dependency: sector-specific knowledge renders scientists more productive
	\item \alert{decreasing returns to research, $\eta<1$}
	\item[] \ % \alert{knowledge spillovers} 
\end{enumerate}
\small
\vspace{4mm}
\hspace{-2mm}
\begin{minipage}[t!]{0.43\textwidth}
	\vspace{0mm}
	\begin{itemize}
		\item[] $A_{Jt}$: sector-specific technology
		\vspace{-2mm}		
		\item[] $A_t$: aggregate technology
		\vspace{-2mm}
		\item[] $\gamma$ : productivity of scientists
	\end{itemize}
\end{minipage}
\vspace{-5mm}
\begin{minipage}[t!]{0.55\textwidth}
	\vspace{0mm}
	\begin{itemize}	
		\item[] \alert{$\rho_J$: number of research processes in sector $J$}
		\vspace{-2mm}			
		\item[] $\eta$ : returns to research
		\vspace{-2mm}			
		\item[] $\phi$ : relative importance knowledge spillovers
	\end{itemize}
\end{minipage}
\end{frame}



\addtocounter{framenumber}{-1}
\begin{frame}{Innovation}
\large
\begin{align*}
	A_{Jit}={A_{Jt-1}}\left(1+\gamma\left(\frac{s_{Jit}}{\rho_J}\right)^\eta\alert{\left(\frac{A_{t-1}}{A_{Jt-1}}\right)^\phi}\right)
\end{align*}
\normalsize
\begin{enumerate}
	\item path dependency: sector-specific knowledge renders scientists more productive
	%				\begin{itemize}
		%			\item scientists in most advanced, fossil sector are more productive
		%			\item shift to green early on to internalize green productivity increase tomorrow
		%		\end{itemize}
	\item decreasing returns to research, $\eta<1$
	\item \alert{knowledge spillovers} from other sectors increases productivity of scientists in sector $J$ \footnotesize{\citep{Barbieri2021GreenPolicy}}
\end{enumerate}
\small
\vspace{4mm}
\hspace{-2mm}
\begin{minipage}[t!]{0.43\textwidth}
	\vspace{0mm}
	\begin{itemize}
		\item[] $A_{Jt}$: sector-specific technology
		\vspace{-2mm}		
		\item[] $A_t$: aggregate technology 
		\vspace{-2mm}
		\item[] $\gamma$ : productivity of scientists
	\end{itemize}
\end{minipage}
\vspace{-5mm}
\begin{minipage}[t!]{0.55\textwidth}
	\vspace{0mm}
	\begin{itemize}	
		\item[] $\rho_J$: number of research processes in sector $J$
		\vspace{-2mm}			
		\item[] $\eta$ : returns to research
		\vspace{-2mm}			
		\item[] $\phi$ : relative importance knowledge spillovers
	\end{itemize}
\end{minipage}
\end{frame}

\begin{frame}{Markets}
\begin{minipage}[t!]{1\textwidth}
	\begin{align*}
		%	\tikzmarkin{8th}(3.6,2.4)(-4.5,-2.2)
		\text{Hours workers}&\hspace{6mm}		H_{t}=L_{Ft}+L_{Gt}+L_{Nt}\\
		\text{Hours scientists}&\hspace{6mm}	S_{t} = \int_{0}^{1}\left(s_{Fit}+s_{Git}+s_{Nit}\right)di\\
		\text{Final good}&\hspace{6mm}	Y_t =C_t+\int_{0}^{1}\left(x_{Fit}+x_{Git}+x_{Nit}\right)di
		%	\tikzmarkend{8th}
	\end{align*}
\end{minipage}
\end{frame}

\begin{comment}
content...
\begin{frame}{Model}
	\begin{figure}[h]
		%	\vspace{-4mm}
		\centering
		\begin{tikzpicture}[auto,scale=.7, transform shape]
			
			\node[circll] (A) at (-7,4)  {\textbf{{\hyperlink{prodmod}{Production}}}\\ \textbf{{and Research}}};
			\node[circll] (B) at (7,4) {\textbf{\hyperlink{backhh}{{Representative}}}\\ \textbf{\hyperlink{backhh}{{Household}}}};
			\node[circll] (D) at (0,9) {\textbf{\alert{Government}}\\ \textbf{max welfare} \\ \textbf{s.t. emission limit} }; 
		\end{tikzpicture}
	\end{figure}
\end{frame}
\addtocounter{framenumber}{-1}

\end{comment}
\begin{frame}{Model}
\begin{figure}[h]
	\vspace{-4mm}
	\centering
	\begin{tikzpicture}[auto,scale=.7, transform shape]
		\node[circll] (A) at (-7,4) {\textbf{{\hyperlink{prodmod}{Production}}}\\ \textbf{{and Research}}};
		\node[circll] (B) at (7,4) {\textbf{\hyperlink{backhh}{{Representative}}}\\ \textbf{\hyperlink{backhh}{{Household}}}};
		\node[circll] (D) at (0,9) {\textbf{\alert{Government}}\\ \textbf{max welfare} \\ \textbf{s.t. emission limit} }; 
		\node[draw=none] (B1) at (5,4.25) {};
		\node[draw=none] (B2) at (5,3.5) {};
		\node[draw=none] (BA1) at (-5,4.25) {};
		\node[draw=none] (BA2) at (-5,3.5) {};
		\node[draw=none] (D1) at (1.8,7.8) {};
		
		\node[draw=none] (B22) at (6.4,5.6) {};
		\node[draw=none] (D2) at (2.3,8.5) {};
		\node[draw=none] (B3) at (5.3,5.3) {};
		\node[draw=none] (D3) at (-1.5,7) {};
		\node[draw=none] (A1) at (-4.2,4.6) {};
		\node[draw=none] (D4) at (-2,8.3) {};
		\node[draw=none] (A4) at (-6,5.4) {};
		\draw [->] (B22) to node[pos=0.5, swap]{{Tax on labor, $\pmb{\tau_{\iota}}$}} (D2);
		\draw [->] (D1) to node[pos=0.5, swap]{Transfers} (B3);
		
		%			\draw [->] (B1) to node[pos=0.75, swap]{Workers and scientists} (BA1);
		%			\draw [->] (BA2) to node[pos=0.75, swap]{Final good} (B2);
		\draw [->] (A4) to node[pos=0.5]{{Tax on carbon, $\pmb{\tau_F}$}}   (D4);
	\end{tikzpicture}
	
\end{figure}
\end{frame}


\begin{frame}{ Government}
\hypertarget{gov}{}
\vspace{-4mm}
\centering
\begin{minipage}[t!]{1\textwidth}
	\begin{align*}
		\max_{\{\tau_{Ft}\}_{t=0}^{\infty}, \{\tau_{\iota t}\}_{t=0}^{\infty}}&\hspace{3mm} \sum_{t=0}^{\infty}\beta^t U\left(C_t,H_t, S_t\right)\\ \ \\
		\text{s.t.} \hspace{4mm}
		&{ (1)\ \ T_t=\tau_{\iota t}w_{t}H_{t}+{{\tau_{Ft}}}F_{t}}+T_{\pi t}\\
		& {(2)\ \  \text{behavior of firms and households}}\\
		& {(3)\ \ \text{resource constraints} }\\
		& \ % {(4)\ \  \omega F_t-\delta \leq\Omega_t }
	\end{align*}
\end{minipage}

\small
\vspace{0mm}
\hspace{-10mm}
\begin{minipage}[t!]{0.5\textwidth}
	\vspace{7mm}
	\begin{itemize}
		\item[] $\beta$\ \ : household discount factor\vspace{-2mm}
		\item[] $T_\pi$: profits minus subsidies \\ \hspace{5.5mm} from machine producers \vspace{0mm}
	\end{itemize}
\end{minipage}
\begin{minipage}[t!]{0.4\textwidth}
	\vspace{8mm}
	\begin{itemize}
		\item[] % $\Omega_{t}$: net emission limit
		\vspace{-2mm}	
		\item[] %$\omega$\ : emissions per unit of fossil \vspace{-2mm}
		\item[] %$\delta$\ \ : carbon sinks
	\end{itemize}
\end{minipage}
\end{frame}

\addtocounter{framenumber}{-1}

\begin{frame}{ Government}
\hypertarget{gov}{}
\vspace{-4mm}
\centering
\begin{minipage}[t!]{1\textwidth}
	\begin{align*}
		\max_{\{\tau_{Ft}\}_{t=0}^{\infty}, \{\tau_{\iota t}\}_{t=0}^{\infty}}&\hspace{3mm} \sum_{t=0}^{\infty}\beta^t U\left(C_t,H_t, S_t\right)\\ \ \\
		\text{s.t.} \hspace{4mm}
		&{ (1)\ \ T_t=\tau_{\iota t}\left(w_{t}H_{t}+w_{st}S_{t}\right)+{{\tau_{Ft}}}F_{t}}+T_{\pi t}\\
		& {(2)\ \  \text{behavior of firms and households}}\\
		& {(3)\ \ \text{resource constraints} }\\
		&{(4)\ \  \alert{\omega F_t-\delta \leq\Omega_t }} \hspace{3mm} \alert{\text{(dynamic emission target)}}
	\end{align*}
\end{minipage}

\small
\vspace{0mm}
\hspace{-10mm}
\begin{minipage}[t!]{0.5\textwidth}
	\vspace{7mm}
	\begin{itemize}
		\item[] $\beta$\ \ : household discount factor\vspace{-2mm}
		\item[] $T_\pi$: profits minus subsidies \\ \hspace{5.5mm} from machine producers \vspace{0mm}
	\end{itemize}
\end{minipage}
\begin{minipage}[t!]{0.45\textwidth}
	\vspace{8mm}
	\begin{itemize}
		\item[] $\Omega_{t}$: net emission limit
		\vspace{-2mm}	
		\item[] $\omega$\ : emissions per unit of fossil \vspace{-0.8mm}
		\item[] $\delta$\ \ : carbon sinks \tiny{\citep{VanVuuren2018AlternativeTechnologies}}
	\end{itemize}
\end{minipage}

%\vspace{0mm}
%\hfill	\hyperlink{govProb}{\tiny{$\rightarrow$ in a nutshell}}
\end{frame}

%%%%%%%%%%%%%%%%%%%%%%%%%%%%%%%
%% Calibration
%%%%%%%%%%%%%%%%%%%%%%%%%%%%%%%
\hypertarget{calback}{}
\section{Calibration}
\begin{frame}{Calibration overview}
\begin{itemize}
	\item
	calibration to the US in 2015-2019
	\item emission limit
	\item model parameters
\end{itemize}
\end{frame}

\begin{frame}{Emission limit}
\vspace{-1mm}
\begin{itemize}
	\item<+->  \textbf{global} CO$_2$ emissions consistent with $1.5^\circ$C climate targets \footnotesize{\citep{IPCC2022}}:
	\vspace{1mm}
	\begin{itemize}
		\item[-] before 2050: remaining carbon budget 510Gt
		\item[-] from 2050 onward: net-zero  emissions
	\end{itemize}
	\vspace{0mm}
	\item<+-> \textbf{equal-per-capita} distribution of  CO$_2$ emissions among countries %\footnotesize{\citep{RobiouDuPont2017EquitableGoals}}
\end{itemize}
\vspace{-2mm}
\pause
\begin{center}
	\begin{minipage}{0.6\textwidth}
		\begin{figure}
			\caption{Net CO$_2$ emission limit in Gt}
			\includegraphics[width=0.7\textwidth]{../codding_model/own_basedOnFried/optimalPol_010922_revision/figures/all_13Sept22_Tplus30/Emnet_goals_o0_lgd0.png}
		\end{figure}
	\end{minipage}
	\hspace{-10mm}
	\begin{minipage}{0.3\textwidth}
		\begin{figure}
			\includegraphics[width=1.4\textwidth]{../codding_model/own_basedOnFried/optimalPol_010922_revision/figures/all_13Sept22_Tplus30/Emnet_goals_Legend_cropped_o0.png}
		\end{figure}
	\end{minipage}
\end{center}
\end{frame}

\addtocounter{framenumber}{-1}
\begin{frame}{Emission limit}
\vspace{-1mm}
\begin{itemize}
	\item  \textbf{global} CO$_2$ emissions consistent with $1.5^\circ$C climate targets \footnotesize{\citep{IPCC2022}}:
	\vspace{1mm}
	\begin{itemize}
		\item[-] before 2050: remaining carbon budget 510Gt
		\item[-] from 2050 onward: net-zero  emissions
	\end{itemize}
	\vspace{0mm}
	\item \textbf{equal-per-capita} distribution of  CO$_2$ emissions among countries 
\end{itemize}
\vspace{-2mm}

\begin{center}
	\begin{minipage}{0.6\textwidth}
		\begin{figure}
			\caption{Net CO$_2$ emission limit in Gt}
			\includegraphics[width=0.7\textwidth]{../codding_model/own_basedOnFried/optimalPol_010922_revision/figures/all_13Sept22_Tplus30/Emnet_goals_o1_lgd0.png}
		\end{figure}
	\end{minipage}
	\hspace{-10mm}
	\begin{minipage}{0.3\textwidth}
		\begin{figure}
			\includegraphics[width=1.4\textwidth]{../codding_model/own_basedOnFried/optimalPol_010922_revision/figures/all_13Sept22_Tplus30/Emnet_goals_Legend_cropped.png}
		\end{figure}
	\end{minipage}
\end{center}
% equal per capita: -85\% reduction relative to 2019 levels (high because also corrects for emissions in US higher than population share!)
% political goal -38\% relative to 2019
% constant ratios/ carbon budget: -62\%
\end{frame}

\begin{frame}{Important parameters}
\vspace{20mm}
\begin{table}[h!]
	\begin{center}
		%		\captionsetup{width=0.3\textwidth}
		%		\caption{ Calibration}
		%		\label{tab:calib}
		\resizebox{5.6in}{!}{
			\begin{tabular}{l|lll}
				%			\hline \hline
				%			\multicolumn{7}{c}{Calibration based on basic needs}\\
				\hline \hline
				\textbf{Parameter}& \textbf{Value}&\textbf{Meaining}& \makecell[l]{\textbf{Source}}\\ 
				\hline 
				%($\sigma$, 	$\sigma_s$) & ($1.33$, $1.33$)&  \makecell[l]{\cite{Chetty2011AreMargins}}  \\
				${{\eta}}$  &0.79 & \small{returns to  research}&\rdelim\}{3}{5cm}[\normalfont\makecell{\cite{Fried2018ClimateAnalysis}}] \\
				${{\phi}}$  &0.75& \small{cross-sector knowledge spillovers}&  \\
				$({\varepsilon_y, \varepsilon_e})$&(0.05, 1.50)&price elasticity of substitution&\\	
				%	($\alpha_F$, $\alpha_G$, $\alpha_N$) &(0.72, 0.91, 0.36)&\\
				({${A_{F0}^{1-alpha_f}}$, ${A_{G0}^{1-alpha_g}}$, ${A_{N0}^{1-alpha_n}}$})&(6.68, 1.50, 2.85)&TFP &energy and fossil shares \citep{EIAEnergy} \\
				%					$\delta$&3.19&sinks in Gt&\rdelim\}{2}{5cm}[\normalfont\makecell{\cite{EPAems}}] \\
				%					$\omega$&217.3&CO$_2$ emissions per unit fossil in Gt&\\
				\hline \hline
			\end{tabular}
		}
	\end{center}
\end{table}

\hypertarget{backca}{}
\vspace{20mm}
\hfill
\hyperlink{calib}{\tiny{$\rightarrow$ all parameters}}
\end{frame}

%%%%%%%%%%%%%%%%%%%%%%%%%%%%%%%
%% Results
%%%%%%%%%%%%%%%%%%%%%%%%%%%%%%%
\hypertarget{resback}{}
\section{Results}

\begin{frame}{Optimal Policy}
\vspace{-3mm}
\begin{figure}[h!!]
	
	\begin{subfigure}{0.45\textwidth}		
		\caption{Tax per ton of carbon in US\$, $\tau_{Ft}$}
		%	\captionsetup{width=.45\linewidth}
		\includegraphics[width=1\textwidth]{../codding_model/own_basedOnFried/optimalPol_010922_revision/figures/all_13Sept22/Single_NC_T_Tauf_emnet1_Sun2_regime4_spillover0_knspil3_noskill0_sep0_xgrowth0_extern0_PV1_sizeequ0_GOV0_etaa0.79.png}
	\end{subfigure}	
	\begin{minipage}[]{0.05\textwidth}
		\ 
	\end{minipage}
	\begin{subfigure}{0.45\textwidth}		
		\caption{Marginal income tax rate in \%, $\tau_{\iota t}$}
		%	\captionsetup{width=.45\linewidth}
		\includegraphics[width=1\textwidth]{../codding_model/own_basedOnFried/optimalPol_010922_revision/figures/all_13Sept22/Single_NC_T_dTaulAvS_emnet1_Sun2_regime4_spillover0_knspil3_noskill0_sep0_xgrowth0_extern0_PV1_sizeequ0_GOV0_etaa0.79.png}
	\end{subfigure}
\end{figure}

\vspace{3mm}
\begin{block}{}
	\begin{itemize}
		\item Labor income tax used to subsidize research
		\item Is there a role for labor income taxation within the optimal environmental policy?
	\end{itemize}
\end{block}	
\hypertarget{backOPT}{}
\vspace{-4mm}
\hfill

\hyperlink{Redis}{\tiny{$\rightarrow$ redistribution,}}
\hyperlink{altems}{\tiny{$\rightarrow$ alternative emission limit,}}\hyperlink{sensphi}{\tiny{$\rightarrow$ sensitivity,}}\hyperlink{compfb}{\tiny{$\rightarrow$ comparison first best}}
\end{frame}


\section*{Mechanism}

\begin{frame}{Investigating the role of labor income taxes as environmental policy instrument}


\begin{itemize}
	\item 	\alert{Problem: Even absent emission target, the optimal policy uses the labor income tax to subsidize research}
	\vspace{3mm}
	\item[\ar] difference of optimal allocation under a carbon-tax-only policy to the benchmark policy not solely explained by motive to lower emissions %(bcs labor income tax driven by subsidizing research \ar reduces emissions as a side effect)
	\vspace{3mm}
	\item Solution: compare benchmark optimal allocation to a counterfactual scenario where the labor income tax is fixed at its optimal level without emission target but emission limit has to be met
	\vspace{3mm}
	\item[\ar] Isolates effect of integrating labor income tax as an environmental policy instrument 
\end{itemize}
%	\alert{Effect of integrated policy}
%	\\
%
%	\ar use of labor income tax not solely due to lower emissions, rather: labor income tax targeted at research, as a side effect, emissions reduce, smaller carbon tax required.
%	\\ 
%	\ar But: want to isolate changes in policy driven by emission target
%	\\
%	\ar Compare optimal allocation in benchmark policy to world where labor income tax is fixed at its optimal level without emission target
%	\\
%	\ar Isolates effect of integrating labor income tax as an environmental policy instrument 
\end{frame}

\begin{frame}{Use of labor income tax in the environmental policy}
\begin{figure}[h!!]
	\centering
	
	\begin{subfigure}{0.45\textwidth}		
		\caption{{Average marginal income tax rate in \%}}
		%	\captionsetup{width=.45\linewidth}
		\includegraphics[width=1\textwidth]{../codding_model/own_basedOnFried/optimalPol_010922_revision/figures/all_13Sept22/NewCalib_polTaulFixed_T_dTaulAvS_Sun2_emnet1_spillover0_knspil3_xgr0_nsk0_sep0_extern0_PV1_etaa0.79_lgd1.png}
	\end{subfigure}
	\begin{minipage}[]{0.05\textwidth}
		\
	\end{minipage}
	\begin{subfigure}{0.45\textwidth}		
		\caption{{Deviation carbon tax in \%}}
		%	\captionsetup{width=.45\linewidth}
		\includegraphics[width=1\textwidth]{../codding_model/own_basedOnFried/optimalPol_010922_revision/figures/all_13Sept22/NewCalib_polTaulFixedPer_T_Tauf_Sun2_emnet1_spillover0_knspil3_xgr0_nsk0_sep0_extern0_PV1_etaa0.79.png}
	\end{subfigure}
\end{figure}
\vspace{3mm}
\begin{block}{}
	\begin{itemize}
		\item In run-up to net-zero limit: labor income tax reduces emissions
		\item Under net-zero limit: reduction of labor income tax boosts production
	\end{itemize}
\end{block}	
\end{frame}
\begin{comment}
\begin{frame}{Deviation from fixed labor income tax policy}
	\hypertarget{mec0}{}
	\vspace{-3mm}
	\centering
	\begin{figure}
		\begin{subfigure}{0.45\textwidth}
			\caption{{\% deviation fossil-to-green scientists }}
			%	\captionsetup{width=.45\linewidth}
			\includegraphics[width=1\textwidth]{../codding_model/own_basedOnFried/optimalPol_010922_revision/figures/all_13Sept22/NewCalib_polTaulFixedPer_T_sffsg_Sun2_emnet1_spillover0_knspil3_xgr0_nsk0_sep0_extern0_PV1_etaa0.79.png}
		\end{subfigure}
		\begin{minipage}[]{0.05\textwidth}
			\
		\end{minipage}
		\begin{subfigure}{0.45\textwidth}
			\caption{{\% deviation green-to-fossil energy }}
			%	\captionsetup{width=.45\linewidth}
			\includegraphics[width=1\textwidth]{../codding_model/own_basedOnFried/optimalPol_010922_revision/figures/all_13Sept22/NewCalib_polTaulFixedPer_T_GFF_Sun2_emnet1_spillover0_knspil3_xgr0_nsk0_sep0_extern0_PV1_etaa0.79.png}
		\end{subfigure}
	\end{figure}
	\vspace{3mm}
	\begin{block}{}
		\begin{itemize}
			\item In run-up to net-zero limit: lower carbon tax to maintain some fossil research
			\item Under net-zero limit: higher carbon tax to foster green research
		\end{itemize}
	\end{block}	
\end{frame}


content...
\end{comment}
\begin{frame}{Decomposing effect of integrated policy }
\hypertarget{mec0}{}
\vspace{-3mm}
\centering
\begin{figure}
	\begin{subfigure}{0.45\textwidth}
		\caption{{\% deviation fossil-to-green scientists }}
		%	\captionsetup{width=.45\linewidth}
		\includegraphics[width=1\textwidth]{../codding_model/own_basedOnFried/optimalPol_010922_revision/figures/all_13Sept22/NewCalib_polTaulFixedTaufJointPer_sffsg_Sun2_emnet1_spillover0_knspil3_xgr0_nsk0_sep0_extern0_PV1_etaa0.79_lgd1.png}
	\end{subfigure}
	\begin{minipage}[]{0.05\textwidth}
		\
	\end{minipage}
	\begin{subfigure}{0.45\textwidth}
		\caption{{\% deviation hours worked }}
		%	\captionsetup{width=.45\linewidth}
		\includegraphics[width=1\textwidth]{../codding_model/own_basedOnFried/optimalPol_010922_revision/figures/all_13Sept22/NewCalib_polTaulFixedTaufJointPer_Hagg_Sun2_emnet1_spillover0_knspil3_xgr0_nsk0_sep0_extern0_PV1_etaa0.79_lgd0.png}
	\end{subfigure}
\end{figure}
\vspace{3mm}
\begin{block}{}
	\begin{itemize}
		\item adjustment in carbon tax directs research activity 
		\item adjustment in labor income tax mitigate side effects on labor market
	\end{itemize}
\end{block}	
\end{frame}

\begin{frame}{Taking stock}
	\begin{itemize}
		\item small effect of integrating income tax
		\item green research subsidies are used in the real world
		\item[\ar]  What is the optimal policy when carbon tax revenues finance green research subsidies?
		\item In particular: Do labor income taxes remain part of the optimal environmental policy?
		\item Yes! Again, labor income taxes correct for distortions on the labor market:
		\begin{itemize}
			\item when carbon tax revenues are not redistributed lump-sum to households, households feel poorer
			\item[\ar] they supply inefficiently high hours!
			\item The labor income tax helps to lower labor supply closer to the efficient level.
		\end{itemize}
	\end{itemize}
\end{frame}



\begin{frame}{Optimal Policy: With research subsidies}
\vspace{-3mm}
\begin{figure}[h!!]
	
	\begin{subfigure}{0.45\textwidth}		
		\caption{Tax per ton of carbon in US\$, $\tau_{Ft}$}
		%	\captionsetup{width=.45\linewidth}
		\includegraphics[width=1\textwidth]{../codding_model/own_basedOnFried/optimalPol_010922_revision/figures/all_13Sept22/Single_NC_T_Tauf_emnet1_Sun2_regimeRS_spillover0_knspil3_noskill0_sep0_xgrowth0_extern0_PV1_sizeequ0_GOV0_etaa0.79.png}
	\end{subfigure}	
	\begin{minipage}[]{0.05\textwidth}
		\ 
	\end{minipage}
	\begin{subfigure}{0.45\textwidth}		
		\caption{Marginal tax rate in \%, $\tau_{\iota t}$}
		%	\captionsetup{width=.45\linewidth}
		\includegraphics[width=1\textwidth]{../codding_model/own_basedOnFried/optimalPol_010922_revision/figures/all_13Sept22/Single_NC_T_dTaulAvS_emnet1_Sun2_regimeRS_spillover0_knspil3_noskill0_sep0_xgrowth0_extern0_PV1_sizeequ0_GOV0_etaa0.79.png}
	\end{subfigure}
\end{figure}


\hypertarget{backOPT}{}
\vspace{-4mm}
\hfill

\hyperlink{Redis}{\tiny{$\rightarrow$ redistribution,}}
\hyperlink{altems}{\tiny{$\rightarrow$ alternative emission limit,}}\hyperlink{sensphi}{\tiny{$\rightarrow$ sensitivity,}}\hyperlink{compfb}{\tiny{$\rightarrow$ comparison first best}}
\end{frame}



\begin{frame}{Taking stock}
	\begin{itemize}
		\item small effect of integrating income tax
		\item green research subsidies are used in the real world
		\item[\ar]  What is the optimal policy when carbon tax revenues finance green research subsidies?
		\item In particular: Do labor income taxes remain part of the optimal environmental policy?
		\item Yes! Again, labor income taxes correct for distortions on the labor market:
		\begin{itemize}
			\item when carbon tax revenues are not redistributed lump-sum to households, households feel poorer
			\item[\ar] they supply inefficiently high hours!
			\item The labor income tax helps to lower labor supply closer to the efficient level.
		\end{itemize}
	\end{itemize}
\end{frame}
\begin{frame}{Optimal Policy: With research subsidies}
\vspace{-3mm}
\begin{figure}[h!!]
	
	\begin{subfigure}{0.45\textwidth}		
		\caption{Tax per ton of carbon in US\$, $\tau_{Ft}$}
		%	\captionsetup{width=.45\linewidth}
		\includegraphics[width=1\textwidth]{../codding_model/own_basedOnFried/optimalPol_010922_revision/figures/all_13Sept22/NewCalib_pol_TvsNoT_taus_RS_emnet1_Sun2_spillover0_knspil3_xgr0_nsk0_sep0_extern0_PV1_etaa0.79_lgd1.png}
	\end{subfigure}	
	\begin{minipage}[]{0.05\textwidth}
		\ 
	\end{minipage}
	\begin{subfigure}{0.45\textwidth}		
		\caption{Marginal tax rate in \%, $\tau_{\iota t}$}
		%	\captionsetup{width=.45\linewidth}
		\includegraphics[width=1\textwidth]{../codding_model/own_basedOnFried/optimalPol_010922_revision/figures/all_13Sept22/NewCalib_pol_TvsNoT_dTaulAvS_RS_emnet1_Sun2_spillover0_knspil3_xgr0_nsk0_sep0_extern0_PV1_etaa0.79_lgd1.png}
	\end{subfigure}
\end{figure}


\hypertarget{backOPT}{}
\vspace{-4mm}
\hfill

\hyperlink{Redis}{\tiny{$\rightarrow$ redistribution,}}
\hyperlink{altems}{\tiny{$\rightarrow$ alternative emission limit,}}\hyperlink{sensphi}{\tiny{$\rightarrow$ sensitivity,}}\hyperlink{compfb}{\tiny{$\rightarrow$ comparison first best}}
\end{frame}

\hypertarget{conc}{}
\section{Conclusion}
\begin{frame}{Conclusion}
\begin{itemize}[<+-| alert@+>]
	\setbeamercolor{alerted text}{} %change the font color
	\setbeamerfont{alerted text}{}
	\item I study the optimal mix of taxes on carbon and income to meet emission targets
	\vspace{3mm}
	
	\item Absent research subsidies: labor income tax combined with carbon tax to substitute for research subsidies
	\item With research subsidies: labor income tax corrects inefficiently high labor supply resulting from missing lump-sum transfers
\end{itemize}
\end{frame}
\begin{frame}[shrink]{References}

\bibliography{../../bib_2_0}
\bibliographystyle{apa}
\end{frame}



%----------------------------------------
%-- appendix
%----------------------------------------
\appendix

%-----------------------------------------
%- Model
%-----------------------------------------
\begin{frame}{Labor supply decision}
\hypertarget{labsup}{}
\begin{align*}
	H_t=\left(\frac{1-\tau_{\iota t}}{\chi}\right)^{\frac{1}{1+\sigma}}
\end{align*}

\vfill
\vspace{0mm}
\hfill 
\hyperlink{hhopt}{\tiny{$\rightarrow$ back}}
\end{frame}




\begin{frame}{In a nutshell: Government trade-off and instruments}
\hypertarget{govProb}{}
\begin{itemize}
	\item 	Goal of government intervention
	\begin{enumerate}
		\item[a)] lower emissions
		\item[b)] keep productivity high
	\end{enumerate}
	\vspace{3mm}
	\item Carbon tax
	\begin{enumerate}
		\item[a)] reduces emissions by lowering fossil demand
		\item[b)] directs research across sectors
		\begin{itemize}
			\item[-] if want to foster \textbf{green} research
			\ar higher carbon tax \ar % but reduces returns to labor %\ar
			costly in terms of output % reduces share of fossil energy 
			\item[-] if want to foster \textbf{fossil} research \ar smaller carbon tax \ar but too high emissions
		\end{itemize}
	\end{enumerate}
	\item Labor income tax can be used to counter side effects of carbon tax 
\end{itemize}

\vspace{5mm}
\hfill	\hyperlink{calback}{\tiny{$\rightarrow$ back}}
\end{frame}

%----------------------------------------
%- Calibration
%----------------------------------------


\begin{frame}{Parameters}
\hypertarget{calib}{}
\vspace{-10mm}
\begin{table}[h!]
	\begin{center}
		%		\captionsetup{width=0.3\textwidth}
		%		\caption{ Calibration}
		%		\label{tab:calib}
		\resizebox{4in}{!}{
			\begin{tabular}{l|ll}
				%			\hline \hline
				%			\multicolumn{7}{c}{Calibration based on basic needs}\\
				\hline \hline
				\textbf{Parameter}& \textbf{Value}& \makecell[l]{\textbf{Target}}\\ 
				\hline
				Household&\multicolumn{2}{c}{}\\
				\hline 
				($\sigma$, 	$\sigma_s$) & ($1.33$, $1.33$)&  \makecell[l]{\cite{Chetty2011AreMargins}}  \\
				%$z_h$& \makecell[l]{skill premium 2005-2016:\\ $w_h/w_l=1.9$\\ \citep{Slavik2020WagePremium}}\\	
				($\chi$, $\chi_s$)& (10.02, 0.48) &  \makecell[l]{average hours worked per\\ economic time endowment\\ by worker: 0.34 \citep{OECDHoursworked}} \\
				$\beta$ & 0.93 &  \makecell[l]{\cite{Barrage2019OptimalPolicy}}\\
				$\bar{H}$&1.00& \makecell[l]{14.5 hours per day \citep{Jones1993OptimalGrowth}} \\
				\hline
				Research&\multicolumn{2}{c}{}
				\\
				\hline 
				${{\eta}}$ &0.79 &  \\
				($\rho_F$, $\rho_G$, $\rho_N$)& (0.01, 0.01, 1.00) &\makecell[l]{\cite{Fried2018ClimateAnalysis}}   \\
				${{\phi}}$ &0.50&  \\
				$\gamma$ & 0.06 &\makecell[l]{maximum aggregate growth:\\4\% per annum \citep{OECDGDP}}\\
				\hline
				Production&\multicolumn{2}{c}{}\\
				\hline
				$({\varepsilon_y, \varepsilon_e})$&(0.05, 1.50)&\cite{Fried2018ClimateAnalysis}\\	
				($\alpha_F$, $\alpha_G$, $\alpha_N$) &(0.72, 0.91, 0.36)&\\
				$\delta_y$&0.38&\makecell[l]{energy expenditure share  \citep{EIAEnergy}}\\
				\hline
				Initial TFP&\multicolumn{2}{c}{}\\
				\hline
				({${A_{F0}^{1-alpha_f}}$, ${A_{G0}^{1-alpha_g}}$, ${A_{N0}^{1-alpha_n}}$})&(8.21, 1.27, 2.80) &fossil to green ratio, energy share to GDP  \\
				\hline 
				Emissions&\multicolumn{2}{c}{}\\
				\hline
				$\delta$&3.19& \makecell[l]{in GtCO$_2$ \citep{EPAems}}\\
				$\omega$&217.3& \cite{EPAems}\\
				\hline \hline
			\end{tabular}
		}
	\end{center}
\end{table}

\vspace{-6mm}
\hfill
\hyperlink{backca}{\tiny{$\rightarrow$ back}}
\end{frame}

%%%%%%%%%%%%%%%%%%%%%%%%%%%%%%%%%%%%%%%5
%%%%%% first best
%%%%%%%%%%%%%%%%%%%%%%%%%%%%%%%%%%%



\begin{frame}{Comparison to first best}

\begin{figure}[h!!]
	\centering
	\begin{subfigure}{0.45\textwidth}		
		\caption{{Green-to-fossil energy ratio}}
		%	\captionsetup{width=.45\linewidth}
		\includegraphics[width=1\textwidth]{../codding_model/own_basedOnFried/optimalPol_010922_revision/figures/all_13Sept22/NewCalib_effBAU_T_GFF_Sun2_emnet1_spillover0_knspil3_xgr0_nsk0_sep0_extern0_PV1_etaa0.79_lgd1.png}
	\end{subfigure}
	\begin{minipage}[]{0.05\textwidth}
		\
	\end{minipage}
	\begin{subfigure}{0.45\textwidth}		
		\caption{{Fossil-to-green scientists}}
		%	\captionsetup{width=.45\linewidth}
		\includegraphics[width=1\textwidth]{../codding_model/own_basedOnFried/optimalPol_010922_revision/figures/all_13Sept22/NewCalib_effBAU_T_sffsg_Sun2_emnet1_spillover0_knspil3_xgr0_nsk0_sep0_extern0_PV1_etaa0.79_lgd0.png}
	\end{subfigure}
\end{figure}
%	\vspace{1mm}
%\begin{block}{}
%	\begin{itemize}
	%		\item First-best: more fossil research and higher green-to-fossil energy ratio
	%		\item[] % \alert{Government dilemma:  more fossil research comes with higher fossil energy use}
	%	\end{itemize}
%\end{block}	
\end{frame}

\addtocounter{framenumber}{-1}
\begin{frame}{Comparison to first best}
\begin{figure}[h!!]
	\centering
	\begin{subfigure}{0.45\textwidth}		
		\caption{Green-to-fossil energy ratio}
		%	\captionsetup{width=.45\linewidth}
		\includegraphics[width=1\textwidth]{../codding_model/own_basedOnFried/optimalPol_010922_revision/figures/all_13Sept22/NewCalib_effBauOpt_T_GFF_Sun2_emnet1_spillover0_knspil3_xgr0_nsk0_sep0_extern0_PV1_etaa0.79_lgd1.png}
	\end{subfigure}
	\begin{minipage}[]{0.05\textwidth}
		\
	\end{minipage}
	\begin{subfigure}{0.45\textwidth}		
		\caption{{Fossil-to-green scientists}}
		%	\captionsetup{width=.45\linewidth}
		\includegraphics[width=1\textwidth]{../codding_model/own_basedOnFried/optimalPol_010922_revision/figures/all_13Sept22/NewCalib_effBauOpt_T_sffsg_Sun2_emnet1_spillover0_knspil3_xgr0_nsk0_sep0_extern0_PV1_etaa0.79_lgd0.png}
	\end{subfigure}
\end{figure}
\vspace{1mm}
\begin{block}{}
	\begin{itemize}
		%\item First-best: more fossil research and higher green-to-fossil energy ratio
		\item {Government dilemma:  more fossil research comes with higher fossil energy use}
	\end{itemize}
\end{block}	
\hypertarget{compfb}{}
\vspace{-4mm}

\hfill
\hyperlink{backOPT}{\tiny{$\rightarrow$ back}}
\end{frame}

\begin{frame}{Optimal Policy: higher emission limit}
\hypertarget{altems}{}
\vspace{-3mm}
\begin{figure}[h!!]
	
	
	\begin{subfigure}{0.4\textwidth}		
		\caption{Tax per ton of carbon,  US\$, $\tau_{Ft}$}
		%	\captionsetup{width=.45\linewidth}
		\includegraphics[width=1\textwidth]{../codding_model/own_basedOnFried/optimalPol_010922_revision/figures/all_13Sept22/Ems_Sens_Tauf_spillover0_knspil0_xgr0_nsk0_sep0_extern0_PV1_etaa0.79_lgd0.png}
	\end{subfigure}	
	\begin{minipage}[]{0.1\textwidth}
		\ 
	\end{minipage}
	\begin{subfigure}{0.4\textwidth}		
		\caption{Marginal tax rate in \%, $\tau_{\iota t}$}
		%	\captionsetup{width=.45\linewidth}
		\includegraphics[width=1\textwidth]{../codding_model/own_basedOnFried/optimalPol_010922_revision/figures/all_13Sept22/Ems_Sens_dTaulAv_spillover0_knspil0_xgr0_nsk0_sep0_extern0_PV1_etaa0.79_lgd1.png}
	\end{subfigure}
	
\end{figure}
\vspace{5mm}
\begin{block}{}
	\begin{itemize}
		\item less strict emission limit (ca. -62\% relative to 2019) allows to exploit availability of labor income tax more
	\end{itemize}
\end{block}	

\vspace{-2mm}
\hfill
\hyperlink{backOPT}{\tiny{$\rightarrow$ back}}\textbf{}
\end{frame}



\begin{frame}{Deviation from carbon-tax-only policy: higher emission limit}
\hypertarget{altemsdecomp}{}
\vspace{-3mm}
\centering
\begin{figure}
	\begin{subfigure}{0.42\textwidth}
		\caption{{\% deviation carbon tax}}
		%	\captionsetup{width=.45\linewidth}
		\includegraphics[width=1\textwidth]{../codding_model/own_basedOnFried/optimalPol_010922_revision/figures/all_13Sept22/EmsDecomp_CTO_Sens_Tauf_spillover0_knspil0_xgr0_nsk0_sep0_extern0_PV1_etaa0.79_lgd1.png}
	\end{subfigure}
	\begin{minipage}[]{0.1\textwidth}
		\
	\end{minipage}
	\begin{subfigure}{0.4\textwidth}
		\caption{{\% deviation fossil-to-green scientists }}
		%	\captionsetup{width=.45\linewidth}
		\includegraphics[width=1\textwidth]{../codding_model/own_basedOnFried/optimalPol_010922_revision/figures/all_13Sept22/EmsDecomp_CTO_Sens_sffsg_spillover0_knspil0_xgr0_nsk0_sep0_extern0_PV1_etaa0.79_lgd0.png}
	\end{subfigure}
\end{figure}
\vspace{3mm}
\begin{block}{}
	\begin{itemize}
		\item less strict emission limit (ca. -62\% relative to 2019) allows to exploit availability of labor income tax more
	\end{itemize}
\end{block}	

\vspace{-3mm}
\hfill	\hyperlink{mec}{\tiny{$\rightarrow$ back}}
\end{frame}


%-- DECOMPOSITION baseline model
\begin{frame}{Decomposition}
\hypertarget{decomp}{}
\centering

\begin{figure}[h!!]
	\centering
	\begin{subfigure}{0.4\textwidth}		
		\caption{{\% deviation  green-to-fossil energy }}
		%	\captionsetup{width=.45\linewidth}
		\includegraphics[width=1\textwidth]{../codding_model/own_basedOnFried/optimalPol_010922_revision/figures/all_13Sept22_Tplus30/CountTAUF_Both_Opt_target_GFF_nsk0_xgr0_knspil0_regime4_spillover0_sep0_extern0_PV1_etaa0.79_lgd1.png}
	\end{subfigure}
	\begin{minipage}[]{0.1\textwidth}
		\
	\end{minipage}
	\begin{subfigure}{0.4\textwidth}		
		\caption{{\% deviation hours worked}}
		%	\captionsetup{width=.45\linewidth}
		\includegraphics[width=1\textwidth]{../codding_model/own_basedOnFried/optimalPol_010922_revision/figures/all_13Sept22_Tplus30/CountTAUF_Both_Opt_target_S_nsk0_xgr0_knspil0_regime4_spillover0_sep0_extern0_PV1_etaa0.79_lgd0.png}
	\end{subfigure}
\end{figure}
\vspace{3mm}
\begin{block}{}
	\begin{itemize}
		\item the change in the carbon tax affects green-to-fossil energy ratio; hours unchanged
		\item the labor tax adjusts hours to (i) reduce emissions or to (ii) raise output
	\end{itemize}
\end{block}	

\vfill
\vspace{-3mm}
\hfill 
\hyperlink{mec}{\tiny{$\rightarrow$ back}}
\end{frame}

%%%%%%%%%%%%%%%%%%%%%%%%%%%%%%%%%%%%%%%%%%%
%%% New Calibration with scientists taxed
%%%%%%%%%%%%%%%%%%%%%%%%%%%%%%%%%%%%%%%%%%%
\begin{frame}{Average Tax rates by labor scientists}
\hypertarget{Redis}{}
\vspace{-3mm}
\begin{figure}[h!!]
	
	\begin{subfigure}{0.4\textwidth}		
		\caption{Average marginal tax scientists in \%}
		%	\captionsetup{width=.45\linewidth}
		\includegraphics[width=1\textwidth]{../codding_model/own_basedOnFried/optimalPol_010922_revision/figures/all_13Sept22/Single_NC_T_dTaulS_emnet1_Sun2_regime4_spillover0_knspil3_noskill0_sep0_xgrowth0_extern0_PV1_sizeequ0_GOV0_etaa0.79.png}
	\end{subfigure}	
	\begin{minipage}[]{0.1\textwidth}
		\ 
	\end{minipage}
	\begin{subfigure}{0.4\textwidth}		
		\caption{Marginal tax rate workers in \%}
		%	\captionsetup{width=.45\linewidth}
		\includegraphics[width=1\textwidth]{../codding_model/own_basedOnFried/optimalPol_010922_revision/figures/all_13Sept22/Single_NC_T_dTaulAv_emnet1_Sun2_regime4_spillover0_knspil3_noskill0_sep0_xgrowth0_extern0_PV1_sizeequ0_GOV0_etaa0.79.png}
	\end{subfigure}
	
\end{figure}
\vspace{5mm}
\begin{block}{}
	The government mainly uses the labor income tax to subsidize research.
\end{block}	
\vspace{-4mm}
\hfill
\hyperlink{backOPT}{\tiny{$\rightarrow$ back}}

%	\hypertarget{backOPT}{}
%	\vspace{-4mm}
%	\hfill
%	\hyperlink{altems}{\tiny{$\rightarrow$ alternative emission limit,}}\hyperlink{sensphi}{\tiny{$\rightarrow$ sensitivity,}}\hyperlink{compfb}{\tiny{$\rightarrow$ comparison first best}}
\end{frame}











\section*{Sensitivity}

\begin{frame}{Sensitivity of optimal policy to knowledge spillovers}
\hypertarget{sensphi}{}
\vspace{-3mm}
\begin{figure}[h!!]
	
	\begin{subfigure}{0.4\textwidth}		
		\caption{Tax per ton of carbon,  US\$, $\tau_{Ft}$}
		%	\captionsetup{width=.45\linewidth}
		\includegraphics[width=1\textwidth]{../codding_model/own_basedOnFried/optimalPol_010922_revision/figures/all_13Sept22/Phi_Sens_Tauf_spillover0_knspil0_xgr0_nsk0_sep0_extern0_PV1_etaa0.79_lgd1.png}
	\end{subfigure}	
	\begin{minipage}[]{0.1\textwidth}
		\ 
	\end{minipage}
	\begin{subfigure}{0.4\textwidth}		
		\caption{Marginal tax rate in \%, $\tau_{\iota t}$}
		%	\captionsetup{width=.45\linewidth}
		\includegraphics[width=1\textwidth]{../codding_model/own_basedOnFried/optimalPol_010922_revision/figures/all_13Sept22/Phi_Sens_dTaulAv_spillover0_knspil0_xgr0_nsk0_sep0_extern0_PV1_etaa0.79_lgd1.png}
	\end{subfigure}
\end{figure}
\vspace{2mm}
\begin{block}{}
	\begin{itemize}
		\item the stronger knowledge spillovers, the higher the optimal carbon tax
		\item at the same time, can profit more from fossil growth in early periods
		\item with knowledge spillovers qualitatively similar results
	\end{itemize}
\end{block}	

\vspace{-5mm}
\hfill
\hyperlink{backOPT}{\tiny{$\rightarrow$ back,}}	\hyperlink{conc}{\tiny{$\rightarrow$ conclusion}}
\end{frame}

\begin{frame}{Sensitivity of optimal policy to other parameters}
\hypertarget{sensother}{}
\vspace{-3mm}
\begin{figure}[h!!]
	
	\begin{subfigure}{0.4\textwidth}		
		\caption{Tax per ton of carbon,  US\$, $\tau_{Ft}$}
		%	\captionsetup{width=.45\linewidth}
		\includegraphics[width=1\textwidth]{../codding_model/own_basedOnFried/optimalPol_010922_revision/figures/all_13Sept22/SensOther_Tauf_spillover0_knspil0_xgr0_nsk0_sep0_extern0_PV1_etaa0.79_lgd1.png}
	\end{subfigure}	
	\begin{minipage}[]{0.1\textwidth}
		\ 
	\end{minipage}
	\begin{subfigure}{0.4\textwidth}		
		\caption{Marginal tax rate in \%, $\tau_{\iota t}$}
		%	\captionsetup{width=.45\linewidth}
		\includegraphics[width=1\textwidth]{../codding_model/own_basedOnFried/optimalPol_010922_revision/figures/all_13Sept22/SensOther_dTaulAv_spillover0_knspil0_xgr0_nsk0_sep0_extern0_PV1_etaa0.79_lgd0.png}
	\end{subfigure}
\end{figure}
\vspace{5mm}
\begin{block}{}
	\begin{itemize}
		\item more responsive labor supply/ higher income effect make labor tax more prominent
		\item green and fossil energy closer substitutes \ar smaller necessary intervention 
	\end{itemize}
\end{block}	

\vspace{-4mm}
\hfill
\hyperlink{backOPT}{\tiny{$\rightarrow$ back,}}\hyperlink{conc}{\tiny{$\rightarrow$ conclusion}}
\end{frame}
\end{document}